
%%%%%%%%%%%%%%%%%%%%%%%%%%%%%%%%%%%%%%%%%%%%%%%%%%%%
%%%                                              %%%
%%%                 Metadata                     %%%
%%%          fill in as appropriate              %%%
%%%                                              %%%
%%%%%%%%%%%%%%%%%%%%%%%%%%%%%%%%%%%%%%%%%%%%%%%%%%%%

\renewcommand{\title}{A grammar of Pite Saami}  %look no further, you can change those things right here.
%\subtitle{}%add a subtitle between the braces if you have one
\newcommand{\BackTitle}{A grammar of Pite Saami}
\newcommand{\BackBody}{Pite Saami is a highly endangered Western Saami language in the Uralic language family currently spoken by a few individuals in Swedish Lapland. This grammar is the first extensive book-length treatment of a Saami language written in English. While focussing on the morphophonology of the main word classes nouns, adjectives and verbs, it also deals with other linguistic structures such as prosody, phonology, phrase types and clauses. Furthermore, it provides an introduction to the language and its speakers, and an outline of a preliminary Pite Saami orthography. An extensive annotated spoken-language corpus collected over the course of five years forms the empirical foundation for this description, and each example includes a specific reference to the corpus in order to facilitate verification of claims made on the data. Descriptions are presented for a general linguistics audience and without attempting to support a specific theoretical approach. This book should be equally useful for scholars of Uralic linguistics, typologists, and even learners of Pite Saami.}
%\dedication{Change dedication in localmetadata.tex}
%\typesetter{Change typesetter in localmetadata.tex}
%\proofreader{Change proofreaders in localmetadata.tex}
\renewcommand{\author}{Joshua Wilbur}
\newcommand{\lsISBN}{978-3-944675-47-3}                     
\newcommand{\lsSeries}{sidl} % use lowercase acronym, e.g. sidl, eotms, tgdi
\newcommand{\lsSeriesNumber}{5} %will be assigned when the book enters the proofreading stage
\newcommand{\lsURL}{http://langsci-press.org/catalog/book/0} % contact the coordinator for the right number
