<<<<<<< HEAD
\chapter{Change title for chapter 1 in chapters/01.tex}
Add content in chapters/01.tex.

\section{About literature}
\citet{Comrie1981} is a useful introduction to typology \is{typology}. %\is = index of subjects
It deals with languages of the whole world, not restricting itself to \ili{Indo-European languages}. %\il = index of languages
\iai{Dionysios Thrax} was also an important figure. %\ia = index of authors. Not necessary for authors whose work you cite.

Lorem ipsum dolor sit amet, consectetur adipiscing elit. In luctus pharetra dui, imperdiet vulputate risus pretium non. Aliquam eget efficitur eros, vel efficitur lectus. Sed a enim libero. Ut posuere velit lectus, vel porta quam posuere vel. Cras in dolor tincidunt erat malesuada malesuada. Quisque elementum nibh id nisl pretium, et rutrum est pellentesque. Curabitur efficitur condimentum tempor. Vivamus venenatis, libero quis blandit volutpat, risus massa ullamcorper neque, et condimentum quam orci sed dui. Ut iaculis tempor enim quis rhoncus. Maecenas imperdiet ultricies nunc, eu sagittis nulla maximus a. Etiam imperdiet eleifend ante. Sed felis sem, sollicitudin vitae ultrices eget, sagittis nec nulla. Donec eget libero maximus, fermentum nibh et, maximus libero. Sed auctor est vel lacus lobortis sollicitudin. Maecenas dapibus nisi leo, sed elementum nisl aliquet ac. Aliquam lobortis, ante in accumsan dignissim, enim odio mattis ex, quis molestie risus lectus et risus. 

\ea\label{ex:1:descartes}
\langinfo{Latin}{}{personal knowledge}\\
\gll cogit-o ergo sum \\
     think-1{\sg}.{\prs}.{\ind} hence exist.1{\sg}.{\prs}.{\ind}\\
\glt `I think therefore I am'
\z

Duis eu interdum urna. In et enim in nibh tincidunt pellentesque. In vestibulum nibh at convallis auctor. Proin porta nisi auctor turpis tristique, eu gravida dolor ultricies. Vivamus pellentesque, erat vel dignissim ornare, neque metus gravida libero, at posuere est ipsum tincidunt metus. Cras a ornare mi, a venenatis justo. Quisque arcu lacus, consectetur in sollicitudin congue, tincidunt at quam. Aliquam tempus tortor nec diam pulvinar porttitor. Nulla scelerisque leo vel orci venenatis, congue consequat urna mattis. Donec dui velit, luctus vitae placerat vitae, tincidunt vitae quam. Vivamus est est, fringilla et est sed, rutrum ornare lectus. Aenean non urna eu urna pharetra porta. Praesent mollis justo ipsum. 

\begin{table}
\caption{Frequencies of word classes}
\label{tab:1:frequencies}
 \begin{tabular}{lllll} % add l for every additional column or remove as necessary
  \lsptoprule
            & nouns & verbs & adjectives & adverbs\\ %table header
  \midrule
  absolute  &   12 &    34  &    23     & 13\\
  relative  &   3.1 &   8.9 &    5.7    & 3.2\\
  \lspbottomrule
 \end{tabular}
\end{table}


Sed vitae lorem lectus. Nunc sit amet venenatis risus. Nullam a metus vitae ligula porttitor mollis. Aenean sit amet faucibus ligula, vitae molestie leo. Suspendisse at ligula ante. Sed posuere mauris et iaculis imperdiet. Suspendisse potenti. Donec in nisi id elit ultricies sollicitudin ac eget augue. Nulla pellentesque, libero ac faucibus suscipit, odio neque tempus elit, nec dictum dolor enim sit amet leo. 

\begin{figure}
\caption{Some XML}
\begin{verbatim}
<LogEvent Action = "1" Value = "107" Cursor = "477">
<LogEvent Action = "1" Value = "107" Cursor = "477">
<LogEvent Action = "1" Value = "107" Cursor = "477">
<LogEvent Action = "1" Value = "107" Cursor = "477">
\end{verbatim}
\end{figure}

\lipsum[1-130]
=======
%% converted from data/Ebarb.xml
\chapter{Introduction}\label{sec:cIntroduction}

Bantu languages commonly signal tense, aspect, mood,
        polarity, and clause-type distinctions with tonal as well
        as segmental cues. Compare, for example, data from the Near
        Future and the Indefinite Future in the Idakho variety of
        the Bantu macrolanguage Luhya. The Near Future (left) is
        marked with the tense prefix \vernacular{la-}, while the
        Indefinite Future (right) takes the tense prefix \vernacular{li-}. Tense
        prefixes are expressed to the left of the verbal stem in
        the examples below, and stem boundaries are indicated with
        square brackets. In addition to these segmental
        differences, a verb like \gloss{‘cut’}surfaces with
        a high tone (henceforth, ‘H’ or ‘H tone’) on the initial
        stem syllable in the Near Future, but a H tone on the final
        syllable in the Indefinite Future.

 
\ea\label{ex:xWhatIsAToneMelody} 
Contrastive Tonal Melodies in Idakho \gloss{‘s/he will
          cut’}


\begin{tabular}{lll}  
  Near Future  &   Indefinite Future  &  \\

                 \vernacular{a-
                }  &   
                 \vernacular{a-
                }  &  \\
\end{tabular}
%\caption{\nocaption}
    
\z

 In Idakho, there are two tonal classes of verbs:
        underlyingly H-toned verbs (henceforth, ‘/H/ verbs’) and
        underlyingly toneless verbs (henceforth, ‘/Ø/ verbs’). /H/
        verbs behave like \gloss{‘cut’}. Below,
        observe that the /H/ verbs all have a H on the initial
        syllable in the Near Future and a H on the final in the
        Indefinite Future. /Ø/ verbs behave differently. In the
        Near Future, there is no H on the verb, and in the
        Indefinite Future, there is a H on the second mora of the
        stem.

 
\ea\label{ex:xWhatIsAToneMelodyPart2} 
/H/ vs. /Ø/ verbs \gloss{‘s/he
          will...’}


\begin{tabular}{lllll}  
    &   Near Future  &   Indefinite
              Future  &   Gloss  &  \\

                 \textbf{/H/}  &   
                 \vernacular{
                a-la[kálaaŋɡa]}  &   
                 \vernacular{
                a-li[kalaaŋɡá]}  &   
                 \gloss{‘fry’}  &  \\

                 \vernacular{
                a-la[βóolitsa]}  &   
                 \vernacular{
                a-li[βoolitsá]}  &   
                 \gloss{‘seduce’}  &  \\
  &     &     &     &  \\

                 \textbf{/Ø/}  &   
                 \vernacular{
                a-la[laxuula]}  &   
                 \vernacular{
                a-li[laxúula]}  &   
                 \gloss{‘release’}  &  \\

                 \vernacular{
                a-la[seeβula]}  &   
                 \vernacular{
                a-li[seéβula]}  &   
                 \gloss{‘say bye’}  &  \\
\end{tabular}
%\caption{\nocaption}
    
\z

 The focus of this dissertation is on verbal ‘tonal
        melodies’, which will refer, in the case of Idakho, to the
        pairing of the tonal properties of /H/ and /Ø/ verbs within
        a particular set of morpho-syntactic constructions. For
        example, the tonal properties of Infinitives and the
        Perfect are identical to those of the Near Future, and, as
        such, may be described as exhibiting the same tonal melody.
        That is, for each of these constructions, /H/ verbs surface
        with a H on the initial syllable, while /Ø/ verbs surface
        with no H. 

 Though /H/ and /Ø/ verbs frequently have strikingly
        different surface patterns, as seen in the examples in \REF{ex:xWhatIsAToneMelodyPart2} above, both
        classes of verbs are subject to the same set of principles
        of tone assignment within a given tonal melody. Differences
        between /H/ and /Ø/ verbs arise from differences in lexical
        tone: /H/ verbs are analyzed as bearing a H tone on the
        initial mora of the stem underlyingly, while /Ø/ verbs have
        no such lexical H.

 Along with a small set of construction-specific tonal
        adjustment rules, the number and position of inflectional
        (or ‘melodic’) Hs distinguish one tonal melody from
        another. Inflectional Hs are analyzed as being contributed
        by the morpho-syntax and enter the derivation as floating
        tones awaiting assignment via Melodic H Assignment (MHA)
        rules. I characterize the Near Future as being \textit{tonally uninflected}, as the
        only H tones which surface in the Near Future are lexical.
        On the other hand, the tonal melody exhibited by the
        Indefinite Future is characterized by one inflectional H
        and a tonal adjustment rule whereby the lexical H in /H/
        verbs is lowered (see § \sectref{sec:sPattern5b} for a detailed
        account of the tonal properties of the Indefinite
        Future).

 In this dissertation, I make two contributions to the
        study of the special role that tone plays in Bantu verbal
        morpho-syntax. First, the dissertation contributes
        extensive novel documentation of the verbal tone system of
        Idakho, known by the autoglossonym \vernacular{[lwiítakho]}, a
        variety of the Luhya cluster of Bantu languages spoken near
        Lake Victoria in western Kenya and eastern Uganda. Second,
        I show how aspects of the Idakho system and that of other
        Luhya varieties like it have contributed to the development
        of rich diversity within the verbal tone systems of
        Luhya.

 Chapter \sectref{sec:cVerbalTone} comprises the
        descriptive component of the dissertation and emphasizes
        the influence of several factors known to influence verb
        tone in Bantu. Because many language consultants
        contributed to the project, the dissertation makes note of
        variation within and across speakers of Idakho. In Chapter \sectref{sec:cPathToPredictability} , I
        demonstrate the role that a preference for prosodically
        well-cued morphological boundaries has played in two
        striking tonal developments within the Luhya macrolanguage:
        the loss of the lexical tonal contrast between /H/ and /Ø/
        verbs, reconstructed even to Proto-Bantu ( \citealt{rStevick1969} ), \footnote{\label{fn:nHvsL}  \citealt{rGreenberg1948} and \citealt{rGuthrie1967-71} , \textit{inter alia}, analyze the
          lexical contrast in Proto-Bantu as one between /H/ and
          /L/.


}%
\sectref{sec:cConclusions} presents
        conclusions.

 The present introductory chapter begins by specifying
        the transcription conventions adopted in this work. In § \sectref{sec:sContextoftheStudy} , I
        contextualize this thesis within the growing body of
        literature on Luhya and Bantu verbal tone systems. § \sectref{sec:sVerbMorpho} provides a basic
        description of those aspects of Idakho verbal morphology
        which are most relevant to the description of Idakho’s
        verbal tonology. The methodology which guided data
        elicitation for the study is described in § \sectref{sec:sMethods} , and § \sectref{sec:sOverviewOfIdakhoVerbalTone} presents an overview of the basic properties
        of Idakho verbal tone patterns.


\section{Transcription and Phonetic Inventory}\label{sec:sTranscriptionConventions}

The present section describes the phonetic inventory
          of Idakho and the hybrid transcription system used in
          this dissertation to represent its members. To enhance
          the usefulness of this document to the Idakho community,
          the local orthography is used as much as possible.
          Transcriptions of the data below deviate from the local
          orthography in two ways. First, orthographic,
          intervocalic <b/v> is transcribed as IPA \vernacular{[β]}, except
          after nasals, in which case orthographic <b>
          corresponds with IPA [b]. Additionally, orthographic
          <i> and <e> are represented as IPA \vernacular{[ɪ]}and \vernacular{[ɛ]},
          respectively, in particular morpho-syntactic environments
          so as to capture the otherwise unrepresented phonemic
          contrast which emerges only word-finally in a limited set
          of verb forms.

 The local orthography corresponds with IPA symbols
          except as indicated in the table below. 

 
\ea\label{ex:xTransConven} 
Transcription
            Conventions 


\begin{tabular}{llllllllllll}  
  
                   \vernacular{
                  <kh>}  &   
                   \vernacular{[x]}  &     &   
                   \vernacular{
                  <sh>}  &   
                   \vernacular{[ʃ]}  &     &   
                   \vernacular{<j>}  &   
                   \vernacular{[d͡ʒ]}  &     &   
                   \vernacular{
                  <ch>}  &   
                   \vernacular{[t͡ʃ]}  &  \\

                   \vernacular{
                  <ng’>}  &   
                   \vernacular{[ŋ]}  &     &   
                   \vernacular{
                  <ng>}  &   
                   \vernacular{[ŋɡ]}  &     &   
                   \vernacular{
                  <nj>}  &   
                   \vernacular{[ɲd͡ʒ]}  &     &   
                   \vernacular{
                  <ny>}  &   
                   \vernacular{[ɲ]}  &  \\
\end{tabular}
%\caption{\nocaption}
    
\z

 The following tables summarize the phonetic inventory
          of Idakho and indicate the symbol I use to represent
          Idakho phones according to their featural description.
          Idakho has a pair of voiced and voiceless affricates,
          which I represent as <j> and <ch>,
          respectively. The series of voiced stops (plosives) and
          the voiced affricate appear only after nasals. The
          segment <f> is represented within angular brackets
          to indicate its marginal use; it appears only in
          borrowings. 

 
\ea\label{ex:xConsonantInvent} 
Consonant Inventory 


\begin{tabular}{llllllllllllll}  
    &   \multicolumn{2}{l}{Bilabial } &   \multicolumn{2}{l}{Labiodental } &   \multicolumn{2}{l}{Alveolar } &   \multicolumn{2}{l}{Palatal } &   \multicolumn{2}{l}{Velar } &   \multicolumn{2}{l}{Glottal } &  \\
Plosive  &   
                   \vernacular{p}  &   
                   \vernacular{b}  &     &     &   
                   \vernacular{t}  &   
                   \vernacular{d}  &     &     &   
                   \vernacular{k}  &   
                   \vernacular{g}  &     &     &  \\
Nasal  &     &   
                   \vernacular{m}  &     &     &     &   
                   \vernacular{n}  &     &   
                   \vernacular{ny}  &     &   
                   \vernacular{ng’}  &     &     &  \\
Trill  &     &     &     &     &     &   
                   \vernacular{r}  &     &     &     &     &     &     &  \\
Fricative  &     &   
                   \vernacular{β}  &   
                   \vernacular{<f>}  &     &     &     &   
                   \vernacular{sh}  &     &   
                   \vernacular{kh}  &     &   
                   \vernacular{h}  &     &  \\
Affricate  &     &     &     &     &     &     &     &     &     &     &     &     &  \\
Approximant  &     &     &     &     &     &     &     &   
                   \vernacular{y}  &     &     &     &     &  \\
Lat. Approx.  &     &     &     &     &     &   
                   \vernacular{l}  &     &     &     &     &     &     &  \\
\end{tabular}
%\caption{\nocaption}
    
\z

 The Idakho vowel system has an ATR contrast, but that
          contrast emerges only among front vowels. 

 
\ea\label{ex:xVowelInvent} 
Vowel Inventory 


\begin{tabular}{lllll}  
    &   Front  &   Central  &   Back  &  \\
High  &   
                   \vernacular{i}  &     &   
                   \vernacular{u}  &  \\

                   \vernacular{ɪ}  &     &     &  \\
Mid  &   
                   \vernacular{e}  &     &   
                   \vernacular{o}  &  \\

                   \vernacular{ɛ}  &     &     &  \\
Low  &     &     &   
                   \vernacular{a}  &  \\
\end{tabular}
%\caption{\nocaption}
    
\z

 Transcriptions in this thesis do not represent the
          allophonic contrast between a lateral approximate \vernacular{[l]}and a
          lateral flap \vernacular{[ɺ]}. Both
          appear as \vernacular{
          <l>}throughout. It appears that the flap
          allophone is observed following front vowels, though I
          did not attend to this contrast while verifying my tonal
          transcriptions.

 Word-final syllables may be long only in a narrow set
          of phrase-final contexts. Idakho has a rule of \regel{Non-Final
          Shortening}whereby long word-final syllables
          are shortened when non-final within the phrase. This
          makes possible alternations like the following. Many
          constructions involving monosyllabic stems, such as \vernacular{a-la\ob [tsia]\cb } \gloss{‘s/he will go’},
          have a long final syllable when the verb form is
          phrase-final, but a short final syllable when an object
          follows the verb form in the phrase, e.g., \vernacular{a-la\ob [tsya]\cb 
          muundu} \gloss{‘s/he will go for
          someone’}. Similarly, passive verb forms, which
          are formed with the addition of the suffix \vernacular{-u}, include
          a long final syllable when the verb form is phrase-final,
          as in \vernacular{
          a\ob [khalaachúi]\cb } \gloss{‘s/he was cut’},
          but a short final syllable when the verb form is
          phrase-medial, as in \vernacular{a\ob [khalaachwí
          {\downstep}tá]\cb } \gloss{‘s/he was not
          cut’}.

 The <i/y> and <u/w> alternations are
          intended to represent the length contrast in these cases,
          rather than to make a claim about the segments’ status as
          a glide or vowel. The acoustics of such sequences do not
          fall within the scope of this dissertation, though
          interested parties may consult the audio archive
          described in Appendix \appref{sec:aIdakhoVerbalToneQuestionnaire} .

 One final point of difference between IPA conventions
          and the transcription system I adopt in this thesis
          relates to the representation of vowel length when the
          first and second half of a long syllable share the same
          vowel quality. Such cases are represented as sequences of
          two vowels, e.g. [ee], rather than with the IPA length
          mark [eː]. 

 I adopt the convention for representing vowel length
          described in the preceding paragraph to facilitate
          marking tone. The tone bearing unit in Idakho is the
          mora. At surface representation, the mora may bear either
          a high tone (henceforth, H) or a low tone (henceforth L);
          at underlying and intermediate representations, moras may
          also be unspecified for tone. Representing long vowels as
          sequences of two vowels makes it possible to mark a mora
          associated with a H with an acute accent uniformly,
          whether that mora is followed by a second mora within the
          same syllable as not, e.g. [ée] rather than [êː]. This
          is desirable because "falling tone" is not a phonological
          category in Idakho, as the circumflex would imply. In
          addition, is simplifies representations of intrasyllabic
          downstep, e.g. [é{\downstep}é]. Toneless moras and moras
          associated with L are unmarked. 



\section{Context of the Study}\label{sec:sContextoftheStudy}

The Luhyas comprise the second largest ethnic group in
          Kenya with a population of over 7 million. \citealt{rLewisEtAl2013} designates
          Luhya as a ‘macrolanguage’, defined as “multiple, closely
          related individual languages that are deemed in some
          usage contexts to be a single language" (ibid). Luhya may
          be divided into more than 20 sub-groups, each with its
          own linguistic variety. Population estimates for known
          sub-ethnicities, compiled from the 2009 Kenya census and
          the 2002 Uganda census, are provided below.

 
\ea\label{ex:xPopulationEstimates} 
Population estimates for Luhya
            sub-groups \footnote{\label{fn:nPopulationEstimatesDetails} Figures presented here for Bukusu and Saamia sum
              the populations reported for Kenyan and Ugandan
              populations: 1,432,810 Bukusu in Kenya and 15,044 in
              Uganda; 124,952 Saamia in Kenya and 279,972 in
              Uganda. 578,583 additional respondents to the 2009
              Kenya census identified their ethnic affiliation as
              “Luhya”. Additionally, there are two markedly
              different Luhya varieties known as Luhya; it is not
              clear what population the provided figure of 273,198
              represents. 


}%



\begin{tabular}{lllllll}  
  Bukusu  &   1,446,854  &   Gisu-Masaaba  &   1,117,661  &   Logoori  &   618,340  &  \\
Saamia  &   404,924  &   Nyole  &   340,507  &   Nyore  &   310,894  &  \\
Wanga  &   309,407  &   Nyala  &   273,198  &   Kabaras  &   252,761  &  \\
Isukha  &   217,327  &   Tiriki  &   209,814  &   
                   \textbf{Idakho}  &   
                   \textbf{170,720}  &  \\
Marachi  &   155,341  &   Marama  &   152,427  &   Kisa  &   137,268  &  \\
Khayo  &   124,555  &   Tsotso  &   121,518  &   Tachoni  &   118,363  &  \\
Gwe  &   75,257  &   Tura  &   30,388  &     &     &  \\
\end{tabular}
%\caption{\nocaption}
    
\z

 Luhya varieties make up a ‘dialect continuum’ in
          which geographically close varieties enjoy higher rates
          of mutual intelligibility, e.g., 89-93\% between Logoori
          and Idakho, while geographically distant varieties have
          considerably lower rates of mutual intelligibility, e.g.,
          21-54\% between Logoori and Bukusu ( \citealt{rKanyoro1983} ). \footnote{\label{fn:nMutualInteliigibiliyDetails} Kanyoro’s (1983: 150) mutual intelligibility measure
            is based on how well 10 speakers per test language were
            able to translate 15 sentences from one variety into
            their own. 


}%
\citealt{rHeineMöhlig1980} , \citealt{rKisembe2005} , \citealt{rKanyoro1983} , and \citealt{rMould1981} identify several
          phonological, morphological, and lexical indexes of
          linguistic diversity within Luhya: the number of
          contrastive vowel qualities (5 vs. 7), the particular
          suite of phonological alternations triggered by nasals,
          and degree of shared vocabulary, \textit{inter alia}. In addition,
          Luhya varieties diverge with respect to the particular
          set of reflexes they inherited through two sporadic
          diachronic processes: Bantu spirantization, an areal
          change whereby stops in Proto-Bantu were inherited as
          fricatives or sibilants into modern Bantu languages, and
          the Luhya Law, a process which, much like Grimm’s Law,
          spirantized voiceless stops and devoiced voiced
          stops.

 Descriptive work on verb tone in Luhya reveals rich
          diversity within the tonal systems in addition to the
          segmental and lexical diversity noted above. With
          increasing access to detailed descriptions of Luhya
          verbal tone systems, Marlo ( \citealt{rMarlo2008a} , \citealt{rMarlo2009a} , \citealt{rMarlo2013} ) develops a typology of
          Luhya verbal tone systems based on reflexes of the */H/
          vs. */Ø/ tonal classes reconstructed for Proto-Bantu ( \citealt{rStevick1969} )
          and associated properties.

 Bantu verbs often fall into one of two tonal classes:
          verbs that are underlyingly high-toned vs. verbs that are
          underlyingly low-toned or toneless. Most modern Bantu
          languages may be characterized as ‘conservative’,
          maintaining the historical /H/ vs. /Ø/ contrast and a set
          of morpho-syntactic contexts in which that contrast is
          transparently revealed on the initial stem syllable.
          There are other ‘predictable’ languages that have lost
          the lexical contrast, treating all verbs equally and
          inflecting all verbs with inflectional tonal melodies ( \citealt{rOdden1989} ), and
          others still, ‘reversive’ systems, in which all verbs are
          inflected with a tonal melody and which now contrast
          between a /L/ class of verbs and a /Ø/ class.

 It is unusual to find multiple examples of two of
          these tone system types within a single macrolanguage.
          The Luhya situation is special in having multiple
          examples of all three system types, with conservative
          varieties situated to the east and predictable varieties
          to the west. Reversive varieties, linguistically
          intermediate between ‘conservative’ and ‘predictable’,
          are geographically intermediate as well. 

 
\ea\label{ex:xDialectMap} 
Luhya dialect map (adapted from \citealt{rMarlo2009a} ,
            in turn from \citealt{rHeineMöhlig1980} )

%\includegraphics[width=\textwidth]{InkScape%20Images/Luyia%20Language%20Map_ToneTypes_Aug_8_2014,%20Labeled,%20reduced.pdf}

\z

 The table below lists by system type references with
          descriptions of Luhya tone. 

 
\ea\label{ex:xLuhyaVerbToneReferences} 
References on Luhya Verbal Tone
            Systems 


\begin{tabular}{llll}  
  System Type  &   Language  &   Reference  &  \\

                   \textit{Predictable:}  &   Khayo  &   
                   \citealt{rMarlo2009b}   &  \\
  &   Tura  &   
                   \citealt{rMarlo2008b}   &  \\
  &   Saamia  &   
                   \citealt{rChagas1976}   &  \\
  &     &   
                   \citealt{rPoletto1998a}   &  \\
  &   Nyala-West  &   
                   \citealt{rOnyango2006}   &  \\
  &     &   
                   \citealt{rMarlo2007}   &  \\
  &     &   
                   \citealt{rEbarbMarlo2009}   &  \\
  &     &     &  \\

                   \textit{Reversive:}  &   Nyala-East  &   
                   \citealt{rOchwaya-Oluoch2003}   &  \\
  &   Marachi  &   
                   \citealt{rMarlo2007}   &  \\
  &     &   
                   \citealt{rBruckhaus2012}   &  \\
  &   Bukusu  &   
                 \citealt{rAusten1974a} , \citealt{rAusten1974b}  &  \\
  &     &   
                   \citealt{rDeBlois1975}   &  \\
  &     &   
                 \citealt{rMutonyi1996} , \citealt{rMutonyi2000}  &  \\
  &     &     &  \\

                   \textit{Conservative:}  &   Logoori  &   
                   \citealt{rLeung1991}   &  \\
  &   Tachoni  &   
                   \citealt{rOdden2009}   &  \\
  &   Tiriki  &   
                   \citealt{rPasterKim2011}   &  \\
\end{tabular}
%\caption{\nocaption}
    
\z

 In Chapter \sectref{sec:cVerbalTone} , I describe
          and analyze the tonal properties of 12 tonal melodies,
          organized into 8 primary ‘Patterns’, in the tonally
          conservative Idakho variety of Luhya. Chapter \sectref{sec:cPathToPredictability} identifies ways in which aspects of tonally
          conservative varieties like Idakho help explain the
          development of predictable systems within Luhya.

 The analysis of Idakho verbal tone presented in Ch. \sectref{sec:cVerbalTone} is cast within
          a derivational, rule-based autosegmental theoretical
          framework, like much work on Bantu tone within the
          autosegmental period (e.g., \citealt{rBickmore1997} , \citealt{rBickmore2000} , \citealt{rBickmore2007} ; \citealt{rCassimjee1986} ; \citealt{rChengKisseberth1979} , \citealt{rChengKisseberth1980} , \citealt{rChengKisseberth1981} ; \citealt{rClements1984} ; \citealt{rDowning1990} ; \citealt{rHymanNgunga1994} ; \citealt{rKidima1991} ; \citealt{rLiphola2001} ; \citealt{rMarlo2008b} , \citealt{rMarlo2009b} ; \citealt{rMwita2008} ; \citealt{rOdden1981} , \citealt{rOdden1987a} , \citealt{rOdden1996} , \citealt{rOdden1998} , \citealt{rOdden2009} ; \citealt{rRobertsKohno2000} ).

 The choice to follow this tradition is a reflection of
          the dissertation’s emphasis on systematic description in
          the service of informing historical developments. Because
          studies in Luhya tone nearly all take a derivational
          approach with autosegmental representations, the goals of
          the thesis are best served by adopting the same
          framework. Doing so facilitates identifying differences
          in the formal statement of tonal rules from one variety
          to another, which naturally reflect systematic
          differences among the varieties themselves. 

 While I hope that the data made available through this
          dissertation are useful in the pursuit of advancing
          theories of phonology and tonal representations, I make
          no attempt to compare alternative analytical frameworks.
          Consult \citealt{rBickmore1999} , \citealt{rdeLacy2002} , and \citealt{rDowning2009} on
          applying an optimality theoretic framework to
          autosegmental representations, or \citealt{rCassimjee1998} and \citealt{rCassimjeeKisseberth1998b} (Optimal Domains Theory) and \citealt{rMcCarthy2004} (Headed Span Theory) on applying an optimality
          theoretic framework to non-autosegmental
          representations.



\section{Verbal Morphology}\label{sec:sVerbMorpho}

The Idakho verb is comprised of a stem and a number of
          ordered inflectional prefixes. The prefixes mark
          subjects, objects, tense-aspect (T/A), and negation. The
          stem itself is comprised of a verbal root, derivational
          suffixes (including the passive suffix, which is a point
          of focus throughout the following description), and a
          marker of mood known as the ‘final vowel’ (hereafter,
          ‘FV’). In addition to the boundaries of the stem, tonal
          rules in Idakho are sensitive to those of a unit of
          structure known as the macrostem. The macrostem is
          comprised of the verb stem and any object prefix. There
          are two slots for object prefixes, though two object
          prefixes may appear in a single verb form in certain
          combinations. See § \sectref{sec:sObjectPrefixes} for some
          discussion on what possible object prefix
          combinations.

 Stem boundaries are marked with square brackets, and
          macrostem boundaries are marked with curly brackets. A
          simplified schema is provided below in \REF{ex:xVerbStructureTemplate} . The possible
          underlying tonal specifications for each morpheme type
          are also indicated.

 
\ea\label{ex:xVerbStructureTemplate} 
Simplified Verb
              Morphology 
\z

 
\begin{tabular}{llllllllllllllllll}  
  Ø  &     &   L/Ø  &     &   H/Ø  &     &     &   H  &     &   H  &     &   H/Ø  &     &   H/Ø  &     &   Ø  &     &  \\
│  &     &   │  &     &   │  &     &     &   │  &     &   │  &     &   │  &     &   │  &     &   │  &     &  \\
Neg  &   -  &   Subj  &   -  &   T/A  &   -  &   \ob   &   Obj
               \textsubscript{1} &   -  &   Obj
               \textsubscript{2} &   [  &   Root  &   -  &   Deriv  &   -  &   FV  &   ]\cb   &  \\
\end{tabular}
%\caption{\nocaption}
     In Bantu languages, nouns are grouped into classes ( \citealt{rKatamba2003} ).
          The classes are in turn grouped into singular/plural
          pairs, or ‘genders’. Noun class membership is typically
          marked on nouns with a unique prefix. \footnote{\label{fn:nUniquePrefixes} The Class 1 and Class 3 noun class prefixes are
            segmentally identical, though differences between the
            two classes emerge in the agreement morphology
            expressed on certain modifiers. Cf. Class 1 \vernacular{mu-sá{\downstep}átsá
            w-áanje} \gloss{‘my man'}vs. \vernacular{mu-rwí} \gloss{‘my head’}.


}%
\vernacular{mu-}, in the
          singular, but the Class 2 prefix \vernacular{βa-}in the
          plural.

 
\ea\label{ex:xNounClassMarkingOnNouns} 
Noun Class Marking on
            Nouns 


\begin{tabular}{llll}  
  Class 1  &   Class 2  &   Gloss  &  \\

                   \vernacular{
                  mu-sáatsa}  &   
                   \vernacular{
                  βa-sáatsa}  &   
                   \gloss{‘man’}  &  \\

                   \vernacular{
                  mú-{\downstep}yáyi}  &   
                   \vernacular{
                  βá-{\downstep}yáyi}  &   
                   \gloss{‘boy’}  &  \\
\end{tabular}
%\caption{\nocaption}
    
\z

 In addition, Idakho encodes the class to which verbal
          arguments belong through the selection of differentiated
          subject and object prefixes. The following sections
          identify subject and object prefixes according to the
          person, number, and noun class of the argument. In
          addition, tense, negation, and other prefixal markers are
          listed and their tonal impact broadly discussed. 


\subsection{Subject Prefixes}\label{sec:sSubjectPrefixes}

The form that each subject prefix takes is indicated
            in \REF{ex:xSubjectPrefixes} below. Singular
            prefixes appear left of their plural counterparts. In
            Idakho, there are unique subject markers for 1 \textsuperscript{st}and 2 \textsuperscript{nd}person subjects,
            while 3 \textsuperscript{rd}person human
            subjects are marked with the same markers that are used
            for subjects belonging to Classes 1 and 2.

 
\ea\label{ex:xSubjectPrefixes} 
Subject Prefixes 


\begin{tabular}{llllll}  
  Singular  &     &     &   Plural  &     &  \\
1
                   \textsuperscript{
                  st}person &   
                     \vernacular{Ǹ-}  &     &   1
                   \textsuperscript{
                  st}person &   
                     \vernacular{khù-}  &  \\
2
                   \textsuperscript{
                  nd}person &   
                     \vernacular{ù-}  &   2
                   \textsuperscript{
                  nd}person &   
                     \vernacular{mù-}  &  \\
Cl. 1  &   
                     \vernacular{a- / y-}  &   Cl. 2  &   
                     \vernacular{βa-}  &  \\
Cl. 3  &   
                     \vernacular{ku-}  &     &   Cl. 4  &   
                     \vernacular{chi-}  &  \\
Cl. 5  &   
                     \vernacular{li-}  &     &   Cl. 6  &   
                     \vernacular{ka-}  &  \\
Cl. 7  &   
                     \vernacular{shi-}  &     &   Cl. 8  &   
                     \vernacular{βi-}  &  \\
Cl. 9  &   
                     \vernacular{i-}  &     &   Cl. 10  &   
                     \vernacular{tsi-}  &  \\
Cl. 11  &   
                     \vernacular{lu-}  &     &     &     &  \\
Cl. 12 (Dim)  &   
                     \vernacular{kha-}  &     &   Cl. 13 (Dim)  &   
                     \vernacular{ru-}  &  \\
Cl. 14  &   
                     \vernacular{βu-}  &     &     &     &  \\
Cl. 20 (Aug)  &   
                     \vernacular{ku-}  &     &   Cl. 4 (Aug)  &   
                     \vernacular{chi-}  &  \\
\end{tabular}
%\caption{\nocaption}
    
\z

 As reflected in \REF{ex:xSubjectPrefixes} above, all subject
            prefixes are underlyingly toneless, except in the case
            of 1 \textsuperscript{st}and 2 \textsuperscript{nd}person subjects,
            which I analyze as being underlyingly /L/. The
            importance of this distinction will be made clear in § \sectref{sec:sP1aSubjects} .

 As indicated above, 3 \textsuperscript{rd}sg human
            subjects may take either \vernacular{a-}or \vernacular{y-}. The
            selection between the two is phonologically
            conditioned, with \vernacular{a-}being
            selected when the following morpheme is
            consonant-initial and \vernacular{y-}being
            selected when the following morpheme is
            vowel-initial.



\subsection{Object Prefixes}\label{sec:sObjectPrefixes}

Object prefixes, which appear immediately left of
            the stem, are largely homophonous with their subject
            prefix counterparts with a few exceptions. Notably,
            object marking in Class 1 shifts from \vernacular{u-}and \vernacular{a- / y-}to \vernacular{khú-}and \vernacular{mú-}for 2 \textsuperscript{nd}and 3 \textsuperscript{rd}person objects,
            respectively. For the Class 9 object prefix, some of my
            consultants preferred \vernacular{chí-},
            while others used \vernacular{
            í-.}Tonally the two sets differ with respect
            to their underlying specification. In particular,
            object prefixes are underlyingly /H/ in contrast to
            their subject prefix counterparts.

 
\ea\label{ex:ObjectPrefixes} 
Object Prefixes 


\begin{tabular}{lllll}  
  Reflexive  &   
                     \vernacular{í} \footnote{\label{fn:nReflexiveFloatingH} The present work does not include a
                      description of the tonal properties of
                      reflexive object prefixes; while the
                      questionnaire includes many paradigms with
                      reflexive object prefixes, time constraints
                      demand that these data be reported only in
                      later work. In other Luhya varieties, such as
                      Nyala-West ( \citealt{rEbarbEtAlInPrep} ), the
                      tonal properties of reflexive object prefixes
                      differ from those of other object prefixes in
                      ways that motivate analyzing the H reflexives
                      contribute as underlyingly floating rather
                      than underlyingly associated to the
                      corresponding segmental material. It appears
                      that this differential treatment is not
                      required in the case of Idakho, though this
                      claim merits further verification.


}%
 &     &   
                     \vernacular{}  &  \\
1sg  &   
                     \vernacular{Ń-}  &   1pl  &   
                     \vernacular{khú-}  &  \\
2sg  &   
                     \vernacular{khú-}  &   2pl  &   
                     \vernacular{mú-}  &  \\
3sg  &   
                     \vernacular{mú-}  &   3pl  &   
                     \vernacular{βá-}  &  \\
Cl. 3  &   
                     \vernacular{kú-}  &   Cl. 4  &   
                     \vernacular{chí-}  &  \\
Cl. 5  &   
                     \vernacular{lí-}  &   Cl. 6  &   
                     \vernacular{ká-}  &  \\
Cl. 7  &   
                     \vernacular{shí-}  &   Cl. 8  &   
                     \vernacular{βí-}  &  \\
Cl. 9  &   
                     \vernacular{í- /
                    chí-}  &   Cl. 10  &   
                     \vernacular{tsí-}  &  \\
Cl. 11  &   
                     \vernacular{lú-}  &     &     &  \\
Cl. 12 (Dim)  &   
                     \vernacular{khá-}  &   Cl. 13 (Dim)  &   
                     \vernacular{rú-}  &  \\
Cl. 14  &   
                     \vernacular{βú-}  &     &     &  \\
Cl. 20 (Aug)  &   
                     \vernacular{kú-}  &   Cl. 4 (Aug)  &   
                     \vernacular{chí-}  &  \\
\end{tabular}
%\caption{\nocaption}
    
\z

 The 1 \textsuperscript{st}sg object prefix
            is a homorganic nasal which is involved in a number of
            segmental alternations: (i) lengthening of the
            preceding syllable, (ii) obstruent voicing, and (iii)
            hardening (of liquids). Additionally, the nasal is
            deleted before voiceless fricatives, though deletion in
            this manner does not bleed lengthening of the preceding
            syllable, nor does it result in the failure of the H
            which it contributes to be deleted. As will be
            demonstrated in § \sectref{sec:sP1aObjects} , the 1 \textsuperscript{st}sg H will be
            realized as a rising tone on the lengthened preceding
            syllable in the basic case, as nasals are not tone
            bearing units at surface representation. \footnote{\label{fn:nNasalsAsTBUs} One likely exception to the claim that nasals are
              not tone bearing units at surface representation
              relates to geminate nasal sequences which result from
              the deletion of vowels between identical sonorants.
              In such cases, nasals appear to preserve the mora
              contributed by the deleted vowel. 


}%


 Verbal forms which combine two object prefixes are
            uncommonly used but two object prefixes may co-occur
            just in case one is the 1 \textsuperscript{st}sg, in which
            case the 1 \textsuperscript{st}sg is ordered
            nearer to the stem. It is not clear whether the
            reflexive object prefix is able to combine with other
            object prefixes, including the 1 \textsuperscript{st}sg. One speaker
            produced two verbal forms which included the reflexive
            object prefix and a CV- object prefix, but expressed
            uncertainty regarding the grammaticality of such forms.
            Another speaker [SB] produced a verbal form which
            contained a reflexive object prefix in combination with
            a 1 \textsuperscript{st}sg and asserted
            that it was well formed ( \vernacular{
            a-laá-nz-i[bechela]} \gloss{‘he will shave himself for
            me’}). The form that was produced ordered the
            reflexive object prefix nearer to the stem than the 1 \textsuperscript{st}sg. The tonal
            properties of this combination were not investigated
            after it was determined to be possible. Verbal forms
            with two CV- object prefixes are not possible in
            Idakho.



\subsection{Tense, Aspect, and Mood}\label{sec:sTenseAspectMorphemes}

Tense, aspect, and mood contrasts are marked
            tonally, as described in Ch. \sectref{sec:cVerbalTone} , as well as
            segmentally through prefixation and suffixation.

 Prefixes situated between subject and object markers
            typically express tense-aspect information, but may
            also signal clause-type and polarity distinctions.
            Prefixes appearing in this position, or set of
            positions, will be referred to simply as ‘tense
            prefixes’ (‘Tns’, in data displays), in spite of the
            fact that doing so surely oversimplifies the
            morpho-syntactic functions fulfilled by this set of
            prefixes. 

 Known ‘tense’ prefixes and their associated
            constructions are listed below, where “Subj. Rel.” and
            “Neg." abbreviate “Subject Relative" and “Negative,"
            respectively. 

 
\ea\label{ex:xTensePrefixes} 
Idakho Tense
              Prefixes 


\begin{tabular}{lllll}  
  a.  &   
                     \vernacular{la-}  &   Near Fut.  &   
                     \vernacular{
                    SP-la[ROOT-a]}  &  \\
  &     &   Near Future Neg.  &   
                     \vernacular{SP-la[ROOT-a]
                    tá(awe)}  &  \\
b.  &   
                     \vernacular{li-}  &   Indefinite Future  &   
                     \vernacular{
                    SP-li[ROOT-a]}  &  \\
  &   
                     \vernacular{}  &   Indefinite Future Neg.  &   
                     \vernacular{SP-li[ROOT-a]
                    tá(awe)}  &  \\
c.  &   
                     \vernacular{lí-}  &   Indefinite Future: Subj.
                  Rel.  &   
                     \vernacular{
                    SP-lí[ROOT-a]}  &  \\
d.  &   
                     \vernacular{akha-}  &   Remote Future  &   
                     \vernacular{
                    SP-akha[ROOT-ɛ]}  &  \\
  &   
                     \vernacular{}  &   Remote Future Neg.  &   
                     \vernacular{SP-akha[ROOT-ɛ]
                    tá(awe)}  &  \\
  &   
                     \vernacular{}  &   Future Perfective  &   
                     \vernacular{
                    SP-akha[ROOT-ile]}  &  \\
  &     &   Future Perfective Neg.  &   
                     \vernacular{SP-akha[ROOT-ile]
                    tá(awe)}  &  \\
e.  &   
                     \vernacular{ákha-}  &   Immediate Past  &   
                     \vernacular{
                    SP-ákha[ROOT-a]}  &  \\
  &     &   Immediate Past Neg.  &   
                     \vernacular{SP-ákha[ROOT-a]
                    tá(awe)}  &  \\
f.  &   
                     \vernacular{kha-}  &   Imperative
                   \textsubscript{sg}Neg. &   
                     \vernacular{SP-kha[ROOT-a]
                    tá(awe)}  &  \\
  &   
                     \vernacular{}  &   Subjunctive Neg.  &   
                     \vernacular{SP-kha[ROOT-a]
                    tá(awe)}  &  \\
  &   
                     \vernacular{}  &   Imperative
                   \textsubscript{pl}Neg. &   
                     \vernacular{SP-kha[ROOT-i]
                    tá(awe)}  &  \\
  &   
                     \vernacular{}  &   Conditional Neg.  &   
                     \vernacular{ni-SP-kha[ROOT-a]
                    tá(awe)}  &  \\
  &   
                     \vernacular{}  &   Hesternal Perfective Neg.: Subj.
                  Rel.  &   
                     \vernacular{
                    SP-a-kha[ROOT-ile]}  &  \\
g.  &   
                     \vernacular{khá-}  &   Future Neg.: Subj. Rel.  &   
                     \vernacular{SP-khá[ROOT-a]
                    tá(awe)}  &  \\
  &   
                     \vernacular{}  &   Present Neg.: Subj. Rel.  &   
                     \vernacular{
                    SP-khá[ROOT-aa(nga)] tá(awe)}  &  \\
  &   
                     \vernacular{}  &   Subjunctive Neg.: Subj. Rel.  &   
                     \vernacular{SP-khá[ROOT-a]
                    tá(awe)}  &  \\
  &   
                     \vernacular{}  &   Perfective Neg.: Subj. Rel.  &   
                     \vernacular{SP-khá[ROOT-ile]
                    tá(awe)}  &  \\
h.  &   
                     \vernacular{a-}  &   Hesternal Perfective  &   
                     \vernacular{
                    SP-a[ROOT-ile]}  &  \\
  &   
                     \vernacular{}  &   Hesternal Perfective Neg.  &   
                     \vernacular{SP-a[ROOT-ile]
                    tá(awe)}  &  \\
i.  &   
                     \vernacular{á-}  &   Hesternal Perfective: Subj.
                  Rel.  &   
                     \vernacular{
                    SP-á[ROOT-ile]}  &  \\
j.  &   
                     \vernacular{aa-}  &   Remote Past  &   
                     \vernacular{
                    SP-aa[ROOT-a]}  &  \\
  &     &   Remote Past Neg.  &   
                     \vernacular{SP-aa[ROOT-a]
                    tá}  &  \\
k.  &   
                     \vernacular{aá-}  &   Habitual  &   
                     \vernacular{
                    SP-aá[ROOT-a]}  &  \\
  &   
                     \vernacular{}  &   Habitual Neg.  &   
                     \vernacular{SP-aá[ROOT-a]
                    tá(awe)}  &  \\
l.  &   
                     \vernacular{shi-}  &   Persistive  &   
                     \vernacular{
                    SP-shi[ROOT-aa(nga)]}  &  \\
  &   
                     \vernacular{}  &   Persistive Neg.  &   
                     \vernacular{
                    SP-shi[ROOT-aa(nga)] tá(awe)}  &  \\
m.  &   
                     \vernacular{shí-}  &   Persistive: Subj. Rel.  &   
                     \vernacular{
                    SP-shí[ROOT-aa(nga)]}  &  \\
  &   
                     \vernacular{}  &   Persistive Neg.: Subj. Rel.  &   
                     \vernacular{
                    SP-shí[ROOT-aa(nga)] tá(awe)}  &  \\
\end{tabular}
%\caption{\nocaption}
    
\z

 In the table above, there are a number of H vs. L/Ø
            pairings among the tense prefixes that appear not to be
            accidental, in particular: \vernacular{li-}vs. \vernacular{lí-}, \vernacular{kha-}vs. \vernacular{khá-}, \vernacular{a-}vs. \vernacular{á-}, and \vernacular{shi-}vs. \vernacular{shí-.}The
            H-toned members of these pairs only appear in subject
            relative constructions, whereas the toneless member is
            observed in matrix clauses. This seems to be part of a
            process whereby an inflectional H is linked to a mora
            preceding the macrostem boundary in only specific
            constructions. The subject relatives noted above fall
            within this set of particular constructions.

 The cases above may be related to another
            construction in which underlying toneless prefixes
            exceptionally surface H: the Conditional Negative \vernacular{ni-SP-kha[ROOT-a]
            tá}(Pattern 2b, § \sectref{sec:sPattern2b} ). Uniquely
            in this context, subject prefixes are realized with a
            H.

 Two additional prefixes appear in a position
            preceding the subject prefix: (i) the \vernacular{ni-}prefix
            or particle, which marks the Crastinal Future and
            conditional tenses, and (ii) the negative marker \vernacular{shi-}.

 The negative marker \vernacular{shi-}was
            optional for most of the speakers I consulted. The
            tonal implications for the decision to include this
            marker (or not) were not systematically investigated
            for this study, though it is at least clear in the case
            of the Near Future Negative that when the marker is not
            included one tone pattern obtains, whereas another
            pattern obtains when the prefixal negative marker is
            present, as shown in \REF{ex:xNegMarkedVsUnmarked} . Notice that,
            in addition to the tonal differences, the tense prefix
            is eliminated in the form marked with the prefixal
            negative marker.

 
\ea\label{ex:xNegMarkedVsUnmarked} 
Tonal Alternations Induced by
              Prefixal Negative Marker: \gloss{‘s/he will
              not...’}


\begin{tabular}{lllll}  
    &   No Prefixal Negative Marker  &   Prefixal Negative Marker  &   Gloss  &  \\
/H/  &   
                     \vernacular{
                    a-la\ob [βó{\downstep}ólítsá]\cb  tá}  &   
                     \vernacular{sh-a\ob [βoolitsa]\cb 
                    tá}  &   
                     \gloss{‘seduce’}  &  \\
/Ø/  &   
                     \vernacular{
                    a-lá\ob [lákhúúlá]\cb  tá}  &   
                     \vernacular{sh-a\ob [lakhúula]\cb 
                    tá}  &   
                     \gloss{‘release’}  &  \\
\end{tabular}
%\caption{\nocaption}
    
\z

 The verb stem terminates with one of four final
            vowels (FV) or two aspectual suffixes. The FV is a
            marker of mood, and Idakho has four: \vernacular{-a}, \vernacular{-ɛ}, \vernacular{-i}, and \vernacular{-ɪ}. The
            most commonly appearing FV is \vernacular{-a}, with
            the other three appearing in subjunctive and imperative
            constructions.

 The aspectual \vernacular{
            -aanga}suffix is obligatory in all Present
            and present-based Persistive contexts, but is variably
            truncated to \vernacular{-aa}. It
            admittedly remains mysterious to the author what
            licenses the truncated form and under what conditions
            the suffix adds an imperfective meaning, as it is
            reported to contribute in other Bantu languages ( \citealt{rNurse2003} , \citealt{rNurse2008} ).

 The form that the perfective suffix
            takes—represented in the morphological schema in \REF{ex:xTensePrefixes} simply as \vernacular{-ile}—is
            complex and dependent upon prosodic properties of the
            verb root. Only monosyllabic roots \REF{ex:xPerfectiveMorphology} a-e) take the full suffix as either \vernacular{-ile}or \vernacular{-ele}.
            Disyllabic roots \REF{ex:xPerfectiveMorphology} f-i) invariably take \vernacular{-i}. Trisyllabic
            and longer roots with an underlyingly long penultimate
            syllable (that is, penultimate within the stem, final
            within the root) also invariably take \vernacular{-i} \REF{ex:xPerfectiveMorphology} l-m, o), while trisyllabic and longer
            roots with an underlyingly short penultimate syllable
            will take \vernacular{-ɪ}or \vernacular{-ɛ},
            depending on the height of the preceding vowel \REF{ex:xPerfectiveMorphology} j-k, n, p). In this last case, the
            perfective suffix also triggers lengthening of the
            penult.

 
\ea\label{ex:xPerfectiveMorphology} 
Idakho Perfective
              Morphology 


\begin{tabular}{lllll}  
    &   Infinitives (‘to...’  &   Perfect (‘he has...+Pst Part’)  &   Gloss  &  \\
a.  &   
                     \vernacular{
                    khu\ob [ráa]\cb }  &   
                     \vernacular{
                    a-a\ob [réele]\cb }  &   
                     \gloss{‘bury’}  &  \\
b.  &   
                     \vernacular{
                    khu\ob [khúa]\cb }  &   
                     \vernacular{
                    a-a\ob [khwéele]\cb }  &   
                     \gloss{‘pay dowry’}  &  \\
c.  &   
                     \vernacular{
                    khu\ob [lía]\cb }  &   
                     \vernacular{
                    a-a\ob [líili]\cb }  &   
                     \gloss{‘eat’}  &  \\
d.  &   
                     \vernacular{
                    khu\ob [kua]\cb }  &   
                     \vernacular{
                    a-a\ob [kwiili]\cb }  &   
                     \gloss{‘fall’}  &  \\
e.  &   
                     \vernacular{
                    khu\ob [sia]\cb }  &   
                     \vernacular{
                    a-a\ob [syeele]\cb }  &   
                     \gloss{‘grind’}  &  \\
f.  &   
                     \vernacular{
                    khu\ob [lúma]\cb }  &   
                     \vernacular{
                    a-a\ob [lúmi]\cb }  &   
                     \gloss{‘bite’}  &  \\
g.  &   
                     \vernacular{
                    khu\ob [βéka]\cb }  &   
                     \vernacular{
                    a-a\ob [βéchi]\cb }  &   
                     \gloss{‘shave’}  &  \\
h.  &   
                     \vernacular{
                    khu\ob [téekha]\cb }  &   
                     \vernacular{
                    a-a\ob [téeshi]\cb }  &   
                     \gloss{‘cook’}  &  \\
i.  &   
                     \vernacular{
                    khu\ob [loonda]\cb }  &   
                     \vernacular{
                    a-a\ob [loondi]\cb }  &   
                     \gloss{‘follow’}  &  \\
j.  &   
                     \vernacular{
                    khu\ob [khálaka]\cb }  &   
                     \vernacular{
                    a-a\ob [khálaachɛ]\cb }  &   
                     \gloss{‘cut’}  &  \\
k.  &   
                     \vernacular{
                    khu\ob [kumila]\cb }  &   
                     \vernacular{
                    a-a\ob [kumiilɪ]\cb }  &   
                     \gloss{‘hold’}  &  \\
l.  &   
                     \vernacular{
                    khu\ob [kálaanga]\cb }  &   
                     \vernacular{
                    a-a\ob [kálaanji]\cb }  &   
                     \gloss{‘fry’}  &  \\
m.  &   
                     \vernacular{
                    khu\ob [homoola]\cb }  &   
                     \vernacular{
                    a-a\ob [homooli]\cb }  &   
                     \gloss{‘massage’}  &  \\
n.  &   
                     \vernacular{
                    khu\ob [βóolitsa]\cb }  &   
                     \vernacular{
                    a-a\ob [βóoliitsɪ]\cb }  &   
                     \gloss{‘seduce’}  &  \\
o.  &   
                     \vernacular{
                    khu\ob [khóng’oonda]\cb }  &   
                     \vernacular{
                    a-a\ob [khóng’oondi]\cb }  &   
                     \gloss{‘knock’}  &  \\
p.  &   
                     \vernacular{
                    khu\ob [βóholola]\cb }  &   
                     \vernacular{
                    a-a\ob [βóholoolɛ]\cb }  &   
                     \gloss{‘untie’}  &  \\
\end{tabular}
%\caption{\nocaption}
    
\z

 The data in \REF{ex:xPerfectiveMorphology} reveal as well
            that the perfective suffix triggers spirantization in
            root final velar consonants, e.g. \REF{ex:xPerfectiveMorphology} g-h, j, l). All the front FVs mentioned
            above condition the same alternations.



\section{Methods}\label{sec:sMethods}

The description of Idakho’s verbal tone system
          presented in Ch. \sectref{sec:cVerbalTone} is based on
          over 250 combined hours of interviews with 22 native
          speakers of Idakho living in the U.S., Nairobi, and rural
          areas in western Kenya to which Idakho is indigenous.
          Data collection for the study began with (now Dr.) Phoebe
          Khasiala Wakhungu during a year-long field methods course
          at Indiana University in 2009-10. In consultation with
          Phoebe, I prepared a questionnaire to survey the
          properties of constructions that I did not investigate
          during the field methods course. Michael Marlo
          administered this questionnaire on my behalf to two
          Idakho speakers during the summer of 2011, when he was in
          Kenya for his own research. Data collection for the
          thesis culminated with 9 months of fieldwork in western
          Kenya from September 2012 through June 2013.

 The majority of my time in Kenya was spent in Kaimosi,
          where I interviewed the study’s primary consultants. Near
          the end of my trip, I traveled to three villages within
          Idakholand for interviews of approximately 3-5 hours with
          nine additional speakers of Idakho. During these
          interviews, I administered a questionnaire surveying the
          core properties of the Idakho verbal tone system. I did
          this to assess the type and extent of variation within a
          community of speakers of a single variety. Moses Egesa
          administered the same questionnaire on my behalf to yet
          another nine speakers who current live in Nairobi. The
          two sets of recordings may soon serve as the empirical
          basis for a study on the impact of language contact and
          attrition in an urban setting on Bantu verbal tone
          systems. 

 Though Idakho is the clear focus of this dissertation,
          during my fieldwork I carried out several parallel
          studies of comparable breadth on the verbal tone systems
          of other undocumented or little-documented Luhya
          varieties: Isukha, Nyore, Marama, Kabras, and Nyala-East.
          These will be the focus of separate manuscripts in time,
          though a limited amount of data from most of these
          varieties may be found in Chapter \sectref{sec:cPathToPredictability} ―a
          discussion of historical developments within Luhya
          tone.

 This section is organized as follows: § \sectref{sec:sHardware} lists the
          hardware used to present prompts, record audio, and store
          recordings. § \sectref{sec:sSoftware} lists computer
          programs used in the presentation of prompts, analysis of
          audio recordings, and preparation of the dissertation. § \sectref{sec:sRecruitment} describes
          language consultant recruitment, and § \sectref{sec:sSocioProfiles} touches on
          the sociolinguistic background of this study’s primary
          language consultants. § \sectref{sec:sQuestionnairePrep} briefly
          summarizes issues informing the preparation of the
          study’s elicitation prompts and describes the structure
          of the questionnaire upon which the following description
          is based. This questionnaire, reformatted to conserve
          space, may be viewed in full in Appendix \appref{sec:aIdakhoVerbalToneQuestionnaire} .


\subsection{Hardware}\label{sec:sHardware}

The audio archive upon which the dissertation is
            based was recorded using a Marantz PMD 660 recorder
            with Super MOD upgrades by Oade Brothers Audio. The
            recorder is powered by four Sanyo Eneloop 2000 MAH LOW
            Discharge AA rechargable batteries and writes to Lexar
            Professional 400x 8 GB SDHC cards. 

 The recorder was paired with a Sanken COS-11D
            omni-directional lavalier microphone using phantom
            power. An improvised head mount was used to position
            the microphone near the participants’ mouths. 

 Interviews were held in a building near the main
            tarmacked road passing through Kaimosi. Along with
            myself, the building was occupied by a seamstress, a
            barber, and other small businesses. While my neighbors
            were courteous, ambient noise was occasionally
            unavoidable, not least of all during the rainy
            season. 

 Questionnaires were presented to study participants
            using one of two laptops. The laptops I brought to use
            for this purpose were chosen for their long battery
            life. Power resources permitting, participants viewed
            the questionnaire on an external monitor. 

 Copies of the original recordings are currently
            stored on separately housed external hard drives: a 2
            TB Western Digital Passport and a 3 TB Seagate
            Expansion. An additional copy of the original
            recordings is also stored in DVD format (Taiyo
            Yuden/JVC 8X DVD-R 4.7GB Silver Thermal Lacquer) in a
            third location. Arrangements are still being made to
            host a copy of the recordings in a more permanent,
            publicly-accessible venue. 



\subsection{Software}\label{sec:sSoftware}

The questionnaires developed for this study were
            prepared in and presented to the participants using
            Microsoft Word (2010/2013), as shown in \REF{ex:xQuestionnaireScreenshot} . Using
            this format to present the elicitation prompts, I paced
            participants’ progress through the questionnaire and
            solicited repetitions of some verb forms by manually
            highlighting the first letter of the desired verb form
            among the many other forms displayed on the screen
            concurrently. I added impressionistic tonal
            transcriptions as we progressed through the
            questionnaire. I also took note of tonal variants,
            where they were possible, and corrected errors in the
            form and meaning of elicitation prompts.

 
\ea\label{ex:xQuestionnaireScreenshot} 
Questionnaire
              Screenshot 

%\includegraphics[width=\textwidth]{QuestionnaireScreenshot3.pdf}

\z

 Verifying and correcting initial transcriptions of
            tone and vowel length relied heavily on Praat ( \citealt{rPraat} ).
            Audacity ( \citealt{rAudacity} )
            helped repair several recordings that failed to close
            properly after accidental power loss. Audacity was also
            used to remove silence from the original recordings in
            the preparation of the audio archive; this was done to
            reduce the total file size of the archive.

 The dissertation itself has been authored with
            XLingPaper \footnote{\label{fn:nXLingPaper} XLingPaper is freely downloadable at \href{http://www.xlingpaper.org}{
              http://www.xlingpaper.org}.


}%
\citealt{rBlack2009} )
            using XMLmind XML Editor. These programs help to
            minimize formatting inconsistencies. Additionally,
            readers of the ‘Portable Document Format (PDF)’ may
            enjoy the extensive use of hyperlinks: all citations
            link directly to the relevant reference in the works
            cited section. Left-clicking on all section and
            numbered display references will similarly bring the
            reader to the referenced section or numbered display.
            Return to the referring page by right-clicking within
            the document and selecting ‘Previous View’ (or pressing
            ‘Alt + Left Arrow’).

 Derivations and formal rules were drawn using
            Inkscape ( \citealt{rInkscape} ). \footnote{\label{fn:nInkscape} Inkscape is freely downloadable at \href{http://www.inkscape.org}{
              http://www.inkscape.org}.


}%




\subsection{Informant Recruitment}\label{sec:sRecruitment}

Recruitment of study participants proceeded by word
            of mouth, beginning with an introduction to Billystrom
            Jivetti (Wiley College) facilitated by Michael Marlo
            (University of Missouri). Billy connected me with his
            sister-in-law, Lydia Songole, who then identified the
            majority of the study’s primary language consultants at
            Friends College Kaimosi, where she worked as an
            instructor, and surrounding educational institutions.
            For my studies of Nyore, Isukha, Marama, Kabras, and
            Nyala-East verbal tone, I consulted only one speaker
            each. I worked with two speakers of Idakho as primary
            consultants on the Idakho study: Julius Ingosi
            (henceforth, ‘JI’) and Sylvester Burula (‘SB’). 

 Sylvester Burula later coordinated the recruitment
            of 18 additional Idakho speakers to respond to the
            questionnaire developed to elicit additional
            information on just core properties of Idakho verbal
            tone. Burula identified three friends of his living in
            different villages within Idakholand; each of these
            three then recruited an additional five participants:
            two currently living within the same village, and three
            currently living in Nairobi. The three originally
            contacted by Burula were asked to recruit only siblings
            or close friends who grew up in the same village. \REF{ex:xTotalIdakhoSpeakers} lists by place
            of primary residence at the time of the interview the
            total number of Idakho speakers consulted for this
            study.

 
\ea\label{ex:xTotalIdakhoSpeakers} 
Number of Speakers by
              Location 


\begin{tabular}{llll}  
    &   Residence  &   Speakers  &  \\
a.  &   Kaimosi Sub-Location, Shaviringa
                  Location  &   2  &  \\
b.  &   Malinya Sub-Location, Shirumba
                  Location  &   3  &  \\
c.  &   Makhokho Sub-Location, Iguhu
                  Location  &   3  &  \\
d.  &   Shivagala Sub-Location, Shirumba
                  Location  &   3  &  \\
e.  &   Nairobi  &   9  &  \\
\end{tabular}
%\caption{\nocaption}
    
\z

 The interviews were conducted in English, and all
            study participants are proficient speakers of English.
            Their ages range from 22, at the very youngest, through
            50 years old. While there is an even ratio between male
            and female participants among the primary language
            consultants, only 2 of the additional 18 Idakho
            speakers are female. While I specified to those helping
            me identify study participants that “middle aged”
            speakers would be desirable, I gave no guidance
            regarding the sex and gender of study recruits. 



\subsection{Sociolinguistic Profiles}\label{sec:sSocioProfiles}

The description below is based on the speech of two
            primary consultants: Sylvester Burula and Julius
            Ingosi. Both currently live in Jeptulu, an area where
            the closely related and mutually intelligible Tiriki
            variety of Luhya predominates, but they spent their
            childhoods in the Malinya Sub-Location within the
            Shirumba Location in Kakamega District. 

 Burula is now in his mid 30s and has spent the
            majority of his life within or very near his home
            village. He was born and stayed in Shikulienyi Village
            within the Malinya Sub-Location through the completion
            of his secondary education. He spent 4 years in
            Nairobi, where he earned a Bachelors in Business and
            Economics. After completing his Bachelors degree, he
            returned to work near his home village until moving to
            Jeptulu (Tirikiland) approximately two years before our
            interviews began, where he now teaches business,
            economics, and project management. 

 Burula speaks Idakho, English, Swahili, and Isukha
            (another Luhya variety which is perhaps even more
            similar to Idakho than Tiriki). While he can converse
            comfortably in each of these languages, he indicated
            that Idakho was the primary language of communication
            when interacting with his parents, his wife, his
            siblings, and his children. 

 Ingosi, currently in his early 40s, spent the first
            10 years of his life in Malinya Village and the next 11
            years of his life in Kabras (where the Kabras Luhya
            variety is spoken). From his early 20s on, he has lived
            in Jeptulu (Tirikiland). Ingosi holds a Bachelors of
            Divinity and serves as a pastor at his church. He also
            operates a motorcycle taxi service. 

 Though he considers Idakho to be his mother tongue
            and feels most comfortable using Idakho, Ingosi is
            highly multilingual, and his language selection
            decisions are complex. He speaks the Idakho, Tiriki,
            and Kabras Luhya varieties, as well as Swahili and
            English. He and I primarily communicated in English. He
            speaks Idakho with his mother, who also speaks Kabras.
            He speaks a “mixture [of] Idakho and Tiriki” with his
            Tiriki wife and primarily English and Swahili with his
            three children. With his siblings, he alternates
            between Idakho, Kabras, English, and Swahili. 

 While his linguistic background is quite rich,
            Ingosi’s tonal system is largely consistent with that
            of Burula (differences, where present, will be noted at
            the relevant section in the description to follow). It
            is particularly noteworthy that Ingosi does not have as
            part of his tonal phonology several rules which
            characterize the other Luhya varieties with which he
            has had extensive contact. For instance, /H/ verbs in
            the Near Future ordinarily surface with a H on the
            initial syllable when the verb is long in both Tiriki ( \citealt{rMarloInPrepB} )
            and Kabras \REF{ex:xFlopInKabras} c-d). Both varieties have
            a rule of \regel{Flop}, which shifts a
            singly linked H from the phrase-final syllable to the
            penult \REF{ex:xFlopInKabras} c-d). In addition, Kabras
            has an additional rule, \regel{Penultimate Doubling},
            which spreads a singly linked H from the penult onto
            the phrase-final syllable

 
\ea\label{ex:xFlopInKabras} 
 \regel{Flop}and \regel{Penultimate
              Doubling}in Kabras and Tiriki, Near Future
              /H/: \gloss{‘s/he
              will...’}


\begin{tabular}{llllll}  
  \multicolumn{1}{l}{ } &   \multicolumn{2}{l}{
                     \textbf{Kabras} } &   \multicolumn{2}{l}{
                     \textbf{Tiriki} } &  \\
a.  &   
                     \vernacular{
                    a-lá[khua]}  &   
                     \gloss{‘pay dowry’}  &   
                     \vernacular{
                    a-lá[ng’wa]}  &   
                     \gloss{‘pay dowry’}  &  \\
b.  &   
                     \vernacular{
                    a-la[βéká]]}  &   
                     \gloss{‘shave’}  &   
                     \vernacular{
                    a-la[véka]}  &   
                     \gloss{‘shave’}  &  \\
c.  &   
                     \vernacular{
                    a-la[khálaka]}  &   
                     \gloss{‘cut’}  &   
                     \vernacular{
                    a-la[khálaka]}  &   
                     \gloss{‘cut’}  &  \\
d.  &   
                     \vernacular{
                    a-la[káraanga]}  &   
                     \gloss{‘fry’}  &   
                     \vernacular{
                    a-la[kálaanga]}  &   
                     \gloss{‘fry’}  &  \\
\end{tabular}
%\caption{\nocaption}
    
\z

 Ingosi, like other Idakho speakers consulted for
            this study, exhibits neither \regel{Flop}nor \regel{Penultimate
            Doubling}.



\subsection{Questionnaire Preparation}\label{sec:sQuestionnairePrep}

Preparing the questionnaires which inform the
            description of Idakho verbal tone presented in Chapter \sectref{sec:cVerbalTone} was guided
            by the methodology articulated at length in \citealt{rMarlo2013} . The methodology
            capitalizes on the insight that Bantu verbal tone
            melodies appear to be unitary exponents of a
            culmination of morpho-syntactic features rather than a
            composite of morphological tonal features, with verbal
            morphemes consistently exerting a predictable influence
            on the tonal properties of the verb forms in which they
            appear. Melody selection is conditioned by the verb
            form’s particular combination of tense, aspect, mood,
            polarity, and clause-type features.

 For a given construction, a particular tonal melody
            is idiosyncratically selected. This melody then
            interacts with additional phonological, morphological,
            and syntactic factors. These include (i) the presence
            of underlyingly high-toned affixes, (ii) verbal stem
            length, (iii) stem syllable weight, and (iv) the
            position of the verb within its clause, \textit{inter alia}. These
            factors must therefore be controlled for and
            systematically varied to arrive at a complete
            description of a Bantu verbal tone system. In the
            following, I illustrate the importance of these factors
            among the tonal systems of several Luhya varieties. The
            structure of the questionnaire used in this study is
            informed by examples such as those presented below.

 High-toned affixes in Luhya may trigger alternations
            in stem tone properties. In Tachoni, for instance, the
            Hortative is tonally marked with a melodic H on the
            second syllable of the stem: e.g., \vernacular{
            xa-ba[karáange]} \gloss{‘let them fry’}.
            Subject prefixes in Tachoni are high-toned only in
            certain morpho-syntactic contexts. The negative
            subjunctive-based future is one such context. This
            construction also realizes a melodic H on the second
            stem syllable, but it spreads leftward in this context
            following a high-toned subject prefix: \vernacular{
            si-na-xú[káráánganile]} \gloss{‘we won’t fry for
            e.o.’}( \citealt{rOdden2009} ).
            In this case, then, the presence of a high-toned
            subject prefix causes the stem initial syllables to be
            realized with a H where it would otherwise not be
            expected to do so (by a process commonly referred to as \regel{Plateau}).

 Nyala-West shows that the number of high-toned
            prefixes must also be varied to gain a full picture of
            a tonal system. One tonal melody realizes a melodic H
            on the first mora of the second syllable in Nyala-West,
            e.g.: \vernacular{
            si-i[mbaangúlul-a]} \gloss{‘I am not
            disarranging’}. The presence of one
            high-toned object prefix does not alter the stem tone
            pattern ( \vernacular{
            si-βa-chí[saambúlul-a]} \gloss{‘they are not de-roofing
            them
            }), but
            the addition of a second high-toned object prefix
            triggers a striking stem pattern alternation. In this
            alternation, the H of the second object prefix is
            realized on the stem initial mora and the melodic H is
            realized, not on the second syllable as expected, but
            on the final mora of the stem: \vernacular{
            s-aa-mú-u[fúunix-ir-á]} \gloss{‘he is not covering him
            for me’}( \citealt{rEbarbEtAlInPrep} ). This example is representative of a
            larger pattern in Luhya, and makes clear that the
            number of high-toned prefixes should be varied
            systematically in a study of Luhya verbal tonology. \footnote{\label{fn:nNoOPx2inBukusuTura} Not all varieties permit two object prefixes, e.g.
              Bukusu ( \citealt{rMutonyi1996}  \citealt{rMutonyi2000} ) and Tura ( \citealt{rMarlo2008b} ), but Idakho does.


}%


 In order to develop a truly comprehensive
            description, the length of the verbal stem must also be
            varied as some melodies may respond differently to the
            constraints of a smaller domain of tone assignment.
            Consider two melodies from Nyala-West: one realizes a
            melodic H which spans from the second stem syllable
            through the final in long stems, as in \vernacular{
            a-ri[paangúlúlá]} \gloss{‘he will
            disarrange’}, while the other realizes a H on
            the first mora of the second stem syllable, as in \vernacular{
            sii[ngaráangaanga]} \gloss{‘I do not fry’}.
            In disyllabic stems, the first melody has the same
            characterization as in longer stems, i.e., the melodic
            H is realized on the second/final syllable: \vernacular{
            a-ri[xwees-á]} \gloss{‘he will fall’}.
            Contrastively, disyllabic forms expressing the other
            melody realize a melodic H on the first stem syllable
            rather than the second/final: \vernacular{
            si-i[ndeéx-a]} \gloss{‘I am not
            cooking’}. These two melodies differ further
            in monosyllabic stems. The first fails to realize a
            melodic H, but the second realizes a H on the sole stem
            syllable: \vernacular{a-ri[fu-a]} \gloss{‘he will
            die’}vs. \vernacular{
            sa-a[rí-a]} \gloss{‘he is not
            eating’}( \citealt{rEbarbEtAlInPrep} ).

 Syllable weight also must be varied, especially
            within the first two stem syllables. A comparison of
            two similar, but distinct, melodies in the Nyala-West
            and Idakho varieties of Luhya demonstrate the
            importance of varying the length of the stem initial
            syllable. In Nyala-West, one melody realizes a H on the
            first mora of the second stem syllable regardless of
            the weight of the stem initial syllable: cf. \vernacular{
            si-i[mbaangúlul-a]} \gloss{‘I am not
            disarranging’}and \vernacular{
            si-i[ngaráang-aang-a]} \gloss{‘I do not
            fry’}(ibid). Contrastively, one tonal melody
            in Idakho realizes a melodic H on the first mora of the
            second stem syllable only when the initial syllable is
            monomoraic. When it is bimoraic, the melodic H is
            realized as a rising tone on the initial syllable: cf. \vernacular{
            βa[βalítsaanga]} \gloss{‘they are
            counting’}vs. \vernacular{
            βa[syeéβaanga]} \gloss{‘they are
            dancing’}. The above contrast between the two
            similar, but distinct Nyala-West and Idakho tonal
            melodies would have gone unnoticed if the weight of the
            stem initial syllable was not systematically varied in
            both studies.

 A similar distinction manifests in the Crastinal
            Future in Idakho, demonstrating the importance of
            varying the weight of the second stem syllable. The
            Crastinal Future is tonally marked with a melodic H on
            the second mora after the initial syllable. That is,
            the melodic H is realized on the second stem syllable
            when that syllable is bimoraic, as in \vernacular{
            na-a[kalaánjɛ]} \gloss{‘he will fry’},
            but on the third when the second syllable is
            monomoraic, as in \vernacular{
            na-a[khalachɛ́]} \gloss{‘he will
            cut’}.

 Finally, the position of a verb within its clause
            and the tonal properties of any post-verbal elements
            can have a dramatic impact on the realization of verbal
            tone. In Tiriki, some infinitives are fully toneless
            phrase finally, e.g.: \vernacular{
            xu[molom-el-a]} \gloss{‘to speak for’}.
            When an object follows, the same verb will be realized
            (i) the same, if that noun is toneless, or (ii)
            entirely high-toned, if that noun bears a H: cf., \vernacular{xu[molomel-a]
            mulimi} \gloss{‘to speak for a
            farmer’}vs. \vernacular{xú[mólóm-él-á]
            mú-lína} \gloss{‘to speak for a
            friend’}( \citealt{rPasterKim2011} ). This is known as \regel{H Tone
            Anticipation}and plays a role in Idakho
            verbal tone as well.

 The primary questionnaires used in this study were
            prepared with the above factors in mind. After an
            initial period of vocabulary elicitation using a
            compilation of approximately 700 Proto-Bantu
            reconstructions as prompts. The full compilation used
            appears appended in \citealt{rMarlo2013} .
            After a sufficiently large sample of prosodically
            diverse verbs was secured, I briefly surveyed the
            tense-aspect system, probing in particular for
            constructions described in previous studies on Luhya
            tone. I conjugated some 10,000+ verb forms for each
            language and organized these into over a thousand
            paradigms containing between 4 and 20 individual verb
            forms. The verbs in each paradigm were carefully
            controlled for their prosodic properties, so that a
            variety of stem shapes were represented in each
            paradigm.

 For each of the approximately 22 constructions
            surveyed in this study, dozens of paradigms were
            compiled for an initial investigation. Each paradigm
            was composed of verb forms that held constant a number
            of factors: the presence, number, and kind (CV- vs. 1 \textsuperscript{st}sg vs.
            reflexive) of object prefixes, the choice of subject,
            the position of the verb within the phrase, the tonal
            properties of post-verbal elements, and whether the
            initial segment of the verbal root is a consonant or a
            vowel.

 For instance, each of the verb forms contained
            within the first full paradigm of Near Future data held
            the following constant: the verb root is underlyingly
            H-toned, the verb root begins with a consonant, the
            verb form has no object prefixes of any kind, and the
            verb appears finally within the phrase. 

 
\ea\label{ex:xNearFutHSampleParadigm} 
Sample Paradigm: Near Future /H/
              C-Initial, OPx0 Phrase-Final 


\begin{tabular}{lll}  
  Idakho  &   Gloss  &  \\

                     \vernacular{
                    a-la[ráa]}  &   
                     \gloss{‘s/he will
                    bury’}  &  \\

                     \vernacular{
                    a-la[ng’úa]}  &   
                     \gloss{‘s/he will
                    drink’}  &  \\

                     \vernacular{
                    a-la[lía]}  &   
                     \gloss{‘s/he will
                    eat’}  &  \\

                     \vernacular{
                    a-la[khúa]}  &   
                     \gloss{‘s/he will pay
                    dowry’}  &  \\

                     \vernacular{
                    a-la[lúma]}  &   
                     \gloss{‘s/he will
                    bite’}  &  \\

                     \vernacular{
                    a-la[βéka]}  &   
                     \gloss{‘s/he will
                    shave’}  &  \\

                     \vernacular{
                    a-la[téekha]}  &   
                     \gloss{‘s/he will
                    cook’}  &  \\

                     \vernacular{
                    a-la[léera]}  &   
                     \gloss{‘s/he will
                    bring’}  &  \\

                     \vernacular{
                    a-la[khálaka]}  &   
                     \gloss{‘s/he will
                    cut’}  &  \\

                     \vernacular{
                    a-la[kálaanga]}  &   
                     \gloss{‘s/he will
                    fry’}  &  \\

                     \vernacular{
                    a-la[sítaaka]}  &   
                     \gloss{‘s/he will
                    accuse’}  &  \\

                     \vernacular{
                    a-la[βóolitsa]}  &   
                     \gloss{‘s/he will
                    seduce’}  &  \\

                     \vernacular{
                    a-la[sáanditsa]}  &   
                     \gloss{‘s/he will
                    thank’}  &  \\

                     \vernacular{
                    a-la[khóng’oonda]}  &   
                     \gloss{‘s/he will
                    knock’}  &  \\

                     \vernacular{
                    a-la[βóholola]}  &   
                     \gloss{‘s/he will
                    untie’}  &  \\

                     \vernacular{
                    a-la[βóyong’ana]}  &   
                     \gloss{‘s/he will go
                    around’}  &  \\

                     \vernacular{
                    a-la[ng’óng’oolitsa]}  &   
                     \gloss{‘s/he will make a
                    (particular) funny face’}  &  \\

                     \vernacular{
                    a-la[língakanyinya]}  &   
                     \gloss{‘s/he will
                    crumple’}  &  \\
\end{tabular}
%\caption{\nocaption}
    
\z

 Subsequent paradigms systematically vary a number
            of properties which can influence verb tone. For
            instance, the second full paradigm contains only phrase
            final verb forms with underlyingly H-toned verb roots
            and no object prefixes, but verb roots are
            vowel-initial, rather than consonant-initial as in the
            sample paradigm in \REF{ex:xNearFutHSampleParadigm} above.

 Later paradigms include object prefixes of various
            kinds, all of which contribute a H. Prompts for verb
            forms with canonical object prefixes with a prosodic
            shape consonant-vowel (CV-) are included. Some
            paradigms include object prefixes with non-canonical
            prosidic shapes: 1 \textsuperscript{st}person singular
            objects, marked with a homorganic nasal (N-) and
            reflexive objects, marked with just a high front vowel
            ( \vernacular{i-}). Two
            object prefixes are possible in certain combinations.
            The questionnaire I developed includes verb forms with
            a CV- object prefix in combination with a 1 \textsuperscript{st}person singular
            object prefix, though other combinations may be
            possibile. \footnote{\label{fn:nNoCVCV} Verb forms with two CV- object prefixes are not
              possible in Idakho. 


}%


 Still later paradigms investigated how the inclusion
            of full noun objects influence the tonal properties of
            the verbs they follow. Both H-toned and toneless
            objects are used, though only a subset of the preceding
            verb forms are tested. In particular, the questionnaire
            includes phrase-medial verb forms with no object
            prefixes, one CV object prefix, and a CV object prefix
            in combination with a 1 \textsuperscript{sg}person singular
            object prefix.

 The following display summarizes the properties of
            the 32 paradigms elicited for each construction as part
            of the primary questionnaire. 

 
\ea\label{ex:xInitialParadigmSet} 
Initial Paradigm Set: Near
              Future 


\begin{tabular}{ll}  
  /H/ C-Initial  &  \\
/H/ V-Initial  &  \\
/Ø/ C-Initial  &  \\
/Ø/ V-Initial  &  \\
/H/ C-Initial + OP  &  \\
/H/ V-Initial + OP  &  \\
/Ø/ C-Initial + OP  &  \\
/Ø/ V-Initial + OP  &  \\
/H/ C-Initial + OP
                   \textsubscript{1sg} &  \\
/H/ V-Initial + OP
                   \textsubscript{1sg} &  \\
/Ø/ C-Initial + OP
                   \textsubscript{1sg} &  \\
/Ø/ V-Initial + OP
                   \textsubscript{1sg} &  \\
/H/ C-Initial + OP
                   \textsubscript{refl} &  \\
/H/ V-Initial + OP
                   \textsubscript{refl} &  \\
/Ø/ C-Initial + OP
                   \textsubscript{refl} &  \\
/Ø/ V-Initial + OP
                   \textsubscript{refl} &  \\
/H/ C-Initial + OP + OP
                   \textsubscript{1sg} &  \\
/H/ V-Initial + OP + OP
                   \textsubscript{1sg} &  \\
/Ø/ C-Initial + OP + OP
                   \textsubscript{1sg} &  \\
/Ø/ V-Initial + OP + OP
                   \textsubscript{1sg} &  \\
/H/ C-Initial Phrase-medial, H-toned
                  object  &  \\
/H/ C-Initial Phrase-medial, Toneless
                  object  &  \\
/Ø/ C-Initial Phrase-medial, H-toned
                  object  &  \\
/Ø/ C-Initial Phrase-medial, Toneless
                  object  &  \\
/H/ C-Initial + OP Phrase-medial, H-toned
                  object  &  \\
/H/ C-Initial + OP Phrase-medial, Toneless
                  object  &  \\
/Ø/ C-Initial + OP Phrase-medial, H-toned
                  object  &  \\
/Ø/ C-Initial + OP Phrase-medial, Toneless
                  object  &  \\
/H/ C-Initial + OP + OP
                   \textsubscript{
                  1sg}Phrase-medial, H-toned object &  \\
/H/ C-Initial + OP + OP
                   \textsubscript{
                  1sg}Phrase-medial, Toneless object &  \\
/Ø/ C-Initial + OP + OP
                   \textsubscript{
                  1sg}Phrase-medial, H-toned object &  \\
/Ø/ C-Initial + OP + OP
                   \textsubscript{
                  1sg}Phrase-medial, Toneless object &  \\
\end{tabular}
%\caption{\nocaption}
    
\z

 Time permitting, some subset of each construction’s
            original paradigms was re-elicited with slight
            modification to briefly survey how melodies interact
            with other factors, such as clause-type (two types of
            questions and relative clauses), and the presence of
            H-toned suffixes and enclitics. The full questionnaire
            used in my study of Idakho verbal tone may be found in
            Appendix \appref{sec:aIdakhoVerbalToneQuestionnaire} . For reasons of time, most of the
            paradigms eliciting information on these topics are not
            described in this thesis.



\section{Overview of Idakho Verbal Tone}\label{sec:sOverviewOfIdakhoVerbalTone}

Idakho is a conservative variety of Luhya, contrasting
          between underlyingly /H/ and /Ø/ classes of verbal roots.
          The present work identifies 8 distinct patterns, some of
          which may be further divided into sub-patterns. These
          patterns are distinguished by the number and position of
          melodic Hs within the macrostem, the tonal properties of
          verbal affixes, and how melodic Hs interact with the
          verb’s position in the phrase. In this section, I
          describe the basic properties of each pattern, preview
          the general analytical approach adopted in the thesis,
          and highlight some of the most pervasive tonal rules that
          influence Idakho’s tonal melodies. 

 Pattern 1 comprises constructions, including
          Infinitives, which are not inflected with a melodic H. In
          the simplest Pattern 1 forms, the root H surfaces on the
          initial mora of the stem. In Pattern 1a, this is the only
          H in the verb form. In Pattern 1b, the tense prefix also
          surfaces H. The construction representing Pattern 1b, the
          Immediate Past, is somewhat unusual in being disyllabic \vernacular{ákha-}. \footnote{\label{fn:nPossiblyDecomposable} It may be that \vernacular{
            ákha-}decomposes into two morphemes, \vernacular{á-}and \vernacular{kha-}.
            Morphemes with the same forms are recruited to mark
            other constructions, though it is not clear that the
            same forms contribute equivalent meanings across
            constructions.


}%


 
\ea\label{ex:xPattern1Summ} 

\begin{tabular}{llll}  
  
                   \textbf{Pattern 1a}  &   \multicolumn{2}{l}{
                   \ul{Near Future: 
                  } } &  \\
/H/  &   
                   \vernacular{
                  a-la\ob [khálaka]\cb }  &   
                   \gloss{‘s/he will
                  cut’}  &  \\
  &   
                   \vernacular{
                  a-la\ob [βóolitsa]\cb }  &   
                   \gloss{‘s/he will
                  seduce’}  &  \\
/Ø/  &   
                   \vernacular{
                  a-la\ob [kulikha]\cb }  &   
                   \gloss{‘s/he will
                  name’}  &  \\
  &     &     &  \\
  &   
                   \vernacular{
                  a-la\ob [lakhuula]\cb }  &   
                   \gloss{‘s/he will
                  release’}  &  \\

                   \textbf{Pattern 1b}  &   \multicolumn{2}{l}{
                   \ul{Immediate Past (TP =
                  H): 
                  } } &  \\
/H/  &   
                   \vernacular{
                  y-á{\downstep}khá\ob [khálaka]\cb }  &   
                   \gloss{‘s/he just
                  cut’}  &  \\
  &   
                   \vernacular{
                  y-á{\downstep}khá\ob [βóolitsa]\cb }  &   
                   \gloss{‘s/he just
                  seduced’}  &  \\
/Ø/  &   
                   \vernacular{
                  y-ákha\ob [kulikha]\cb }  &   
                   \gloss{‘s/he just
                  named’}  &  \\
  &   
                   \vernacular{
                  y-ákha\ob [lakhuula]\cb }  &   
                   \gloss{‘s/he just
                  released’}  &  \\
\end{tabular}
%\caption{\nocaption}
    
\z

 The Pattern 1b forms involving /H/ verbs illustrate a
          tonal process which is common both in Idakho and in Bantu
          languages generally: \regel{Plateau}( \citealt{rKisseberthOdden2003} ). The downstepped H on the second mora of the
          tense prefix arrives there via this leftward spreading
          process. In this case, the root H spreads leftward onto
          the mora intervening between the root H and the H of the
          tense prefix.

 Verbal forms exhibiting the properties of Pattern 2
          are characterized by an all L surface pattern in /H/
          verbs and a melodic H which surfaces on the second stem
          mora in /Ø/ verbs. As in Pattern 1, the ‘b’ subtype of
          Pattern 2 includes a construction with a H-toned prefix
          where the ‘a’ subtype does not: subject prefixes surface
          H in the Conditional Negative. 

 
\ea\label{ex:xPattern2Summ} 

\begin{tabular}{llll}  
  
                   \textbf{Pattern 2a}  &   \multicolumn{2}{l}{
                   \ul{Subjunctive Neg.: 
                  } } &  \\
/H/  &   
                   \vernacular{u-kha\ob [khalaka]\cb 
                  tá}  &   
                   \gloss{‘do not cut!’}  &  \\
  &   
                   \vernacular{u-kha\ob [βoolitsa]\cb 
                  tá}  &   
                   \gloss{‘do not
                  seduce!’}  &  \\
/Ø/  &   
                   \vernacular{u-kha\ob [kulíkha]\cb 
                  tá}  &   
                   \gloss{‘do not name!’}  &  \\
  &   
                   \vernacular{u-kha\ob [seéβula]\cb 
                  tá}  &   
                   \gloss{‘do not say
                  goodbye!’}  &  \\
  &     &     &  \\

                   \textbf{Pattern 2b}  &   \multicolumn{2}{l}{
                   \ul{Conditional Neg. (SP =
                  H): 
                  } } &  \\
/H/  &   
                   \vernacular{na-á-kha\ob [khalaka]\cb 
                  tá}  &   
                   \gloss{‘if s/he does not
                  cut’}  &  \\
  &   
                   \vernacular{
                  na-á-kha\ob [βoolitsa]\cb  tá}  &   
                   \gloss{‘if s/he does not
                  seduce’}  &  \\
/Ø/  &   
                   \vernacular{
                  na-á-kha\ob [kulíkha]\cb  tá}  &   
                   \gloss{‘if s/he does not
                  name’}  &  \\
  &   
                   \vernacular{
                  na-á-kha\ob [seéβula]\cb  tá}  &   
                   \gloss{‘if s/he does not say
                  goodbye’}  &  \\
\end{tabular}
%\caption{\nocaption}
    
\z

 Pattern 3 may be observed in the affirmative
          Subjunctive and the subjunctive-based Crastinal Future.
          These constructions take a unique melody in which /H/ and
          /Ø/ both express a melodic H on the second mora after the
          initial syllable of the macrostem. In other words:
          regardless of the weight of the initial syllable of the
          macrostem, if the second syllable is short the melodic H
          will surface on the third syllable, otherwise it will be
          realized on the second half of a long second syllable
          (i.e., as a rising tone on the second stem syllable).
          While the lexical contrast is neutralized in the basic
          forms presented below, § \sectref{sec:sPattern3} will show that
          the lexical contrast re-emerges in other contexts.

 
\ea\label{ex:xPattern3Summ} 

\begin{tabular}{llll}  
  
                   \textbf{Pattern 3}  &   \multicolumn{2}{l}{
                   \ul{Subjunctive: 
                  } } &  \\
/H/  &   
                   \vernacular{
                  a\ob [βoolitsɪ́]\cb }  &   
                   \gloss{‘let him/her
                  seduce’}  &  \\
  &   
                   \vernacular{
                  a\ob [kalaánjɛ]\cb }  &   
                   \gloss{‘let him/her
                  fry’}  &  \\
  &   
                   \vernacular{
                  a\ob [tsuunzuúnɪ]\cb }  &   
                   \gloss{‘let him/her
                  suck’}  &  \\
  &   
                   \vernacular{
                  a\ob [βoyong’ánɛ]\cb }  &   
                   \gloss{‘let him/her go
                  around’}  &  \\
/Ø/  &   
                   \vernacular{
                  a\ob [seebulɪ́]\cb }  &   
                   \gloss{‘let him/her say
                  goodbye’}  &  \\
  &   
                   \vernacular{
                  a\ob [lakhuúlɪ]\cb }  &   
                   \gloss{‘let him/her
                  release’}  &  \\
  &   
                   \vernacular{
                  a\ob [siinjilítsɪ]\cb }  &   
                   \gloss{‘let him/her make
                  stand’}  &  \\
  &   
                   \vernacular{
                  a\ob [seβulúkhaɲːɪ]\cb }  &   
                   \gloss{‘let him/her
                  scatter’}  &  \\
\end{tabular}
%\caption{\nocaption}
    
\z

 In Pattern 4, the melodic H surfaces on the initial
          mora of the macrostem in both /H/ and /Ø/ verbs. As with
          Pattern 3, the lexical contrast between /H/ and /Ø/ verbs
          is neutralized in the small sample of data given below,
          but the contrast re-emerges in other contexts. 

 
\ea\label{ex:xPattern4Summ} 

\begin{tabular}{llll}  
  
                   \textbf{Pattern 4}  &   \multicolumn{2}{l}{
                   \ul{Remote Past: 
                  } } &  \\
/H/  &   
                   \vernacular{
                  y-aa\ob [khálaka]\cb }  &   
                   \gloss{‘s/he cut’}  &  \\
  &   
                   \vernacular{
                  y-aa\ob [βóolitsa]\cb }  &   
                   \gloss{‘s/he seduced’}  &  \\
/Ø/  &   
                   \vernacular{
                  y-aa\ob [kúlikha]\cb }  &   
                   \gloss{‘s/he named’}  &  \\
  &   
                   \vernacular{
                  y-aa\ob [lákhuula]\cb }  &   
                   \gloss{‘s/he
                  released’}  &  \\
\end{tabular}
%\caption{\nocaption}
    
\z

 The Pattern 5 subtypes are grouped together due to
          similarities in the formal analysis I offer in § \sectref{sec:sPattern5} , though there
          are many notable differences in their basic tonal
          properties. In Pattern 5a, /H/ verbs surface with a H on
          all moras of the third stem syllable, and /Ø/ verbs
          surface with a H on the second stem mora. In Pattern 5b,
          /Ø/ verbs surface with a H on the second stem mora as
          well, but /H/ verbs surface with a H on the final
          syllable rather than the third. Pattern 5c sets itself
          apart from the other Pattern 5 constructions in having a
          H-toned subject prefix. In addition, /H/ verbs surface
          with a downstepped H which spans the full length of the
          stem, while /Ø/ verbs surface with a downstepped H
          spanning the first two moras of the stem.

 
\ea\label{ex:xPattern5Summ} 

\begin{tabular}{llll}  
  
                   \textbf{Pattern 5a}  &   \multicolumn{2}{l}{
                   \ul{Present: 
                  } } &  \\
/H/  &   
                   \vernacular{
                  a\ob [khalakáánga]\cb }  &   
                   \gloss{‘s/he is
                  cutting’}  &  \\
  &   
                   \vernacular{
                  a\ob [βoolitsáánga]\cb }  &   
                   \gloss{‘s/he is
                  seducing’}  &  \\
/Ø/  &   
                   \vernacular{
                  a\ob [kulíkhaanga]\cb }  &   
                   \gloss{‘s/he is
                  naming’}  &  \\
  &   
                   \vernacular{
                  a\ob [lakhúulaanga]\cb }  &   
                   \gloss{‘s/he is
                  releasing’}  &  \\
  &     &     &  \\

                   \textbf{Pattern 5b}  &   \multicolumn{2}{l}{
                   \ul{Indefinite Future: 
                  } } &  \\
/H/  &   
                   \vernacular{
                  a-li\ob [khalaká]\cb }  &   
                   \gloss{‘s/he will
                  cut’}  &  \\
  &   
                   \vernacular{
                  a-li\ob [βoyong’aná]\cb }  &   
                   \gloss{‘s/he will go
                  around’}  &  \\
/Ø/  &   
                   \vernacular{
                  a-li\ob [kulíkha]\cb }  &   
                   \gloss{‘s/he will
                  name’}  &  \\
  &   
                   \vernacular{
                  a-li\ob [lakhúula]\cb }  &   
                   \gloss{‘s/he will
                  release’}  &  \\
  &     &     &  \\

                   \textbf{Pattern 5c}  &   \multicolumn{2}{l}{
                   \ul{Conditional (SP = H): 
                  } } &  \\
/H/  &   
                   \vernacular{
                  na-á\ob [{\downstep}kháláká]\cb }  &   
                   \gloss{‘if s/he cuts’}  &  \\
  &   
                   \vernacular{
                  na-á\ob [{\downstep}βóólítsá]\cb }  &   
                   \gloss{‘if s/he
                  seduces’}  &  \\
/Ø/  &   
                   \vernacular{
                  na-á\ob [{\downstep}kúlíkha]\cb }  &   
                   \gloss{‘if s/he
                  names’}  &  \\
  &   
                   \vernacular{
                  na-á\ob [{\downstep}lákhúula]\cb }  &   
                   \gloss{‘if s/he
                  releases’}  &  \\
\end{tabular}
%\caption{\nocaption}
    
\z

 Pattern 6 includes affirmative Imperative
          constructions. /H/ verbs are characterized by a H on the
          FV. /Ø/ verbs also surface with a H on the FV, but this H
          is downstepped in /Ø/ verbs following a H which spans all
          preceding moras of the stem. 

 
\ea\label{ex:xPattern6Summ} 

\begin{tabular}{llll}  
  
                   \textbf{Pattern 6}  &   \multicolumn{2}{l}{
                   \ul{Imp.
                  } } &  \\
/H/  &   
                   \vernacular{
                  \ob [khalaká]\cb }  &   
                   \gloss{‘cut!’}  &  \\
  &   
                   \vernacular{
                  \ob [βoolitsá]\cb }  &   
                   \gloss{‘seduce!’}  &  \\
/Ø/  &   
                   \vernacular{
                  \ob [kúlí{\downstep}khá]\cb }  &   
                   \gloss{‘name!’}  &  \\
  &   
                   \vernacular{
                  \ob [lákhúú{\downstep}lá]\cb }  &   
                   \gloss{‘release!’}  &  \\
\end{tabular}
%\caption{\nocaption}
    
\z

 Pattern 7 is reserved for the Hesternal Perfective. In
          this construction, /H/ verbs surface with a H spanning
          the full length of the stem. The stem in /Ø/ verbs is
          similarly H throughout the full stem, but downstep is
          observed between the second and third moras of the
          stem. 

 
\ea\label{ex:xPattern7Summ} 

\begin{tabular}{llll}  
  
                   \textbf{Pattern 7}  &   \multicolumn{2}{l}{
                   \ul{Hesternal Perf.: 
                  } } &  \\
/H/  &   
                   \vernacular{
                  y-a\ob [khálááchɛ́]\cb }  &   
                   \gloss{‘s/he cut’}  &  \\
  &   
                   \vernacular{
                  y-a\ob [βóólíítsɪ́]\cb }  &   
                   \gloss{‘s/he seduced’}  &  \\
/Ø/  &   
                   \vernacular{
                  y-a\ob [lákhú{\downstep}úlí]\cb }  &   
                   \gloss{‘s/he
                  released’}  &  \\
  &   
                   \vernacular{
                  y-a\ob [séé{\downstep}βúúlɪ́]\cb }  &   
                   \gloss{‘s/he said
                  goodbye’}  &  \\
\end{tabular}
%\caption{\nocaption}
    
\z

 Finally Pattern 8 is found only in the Habitual and is
          characterized by both a H-toned tense prefix and a
          melodic H which surfaces on all moras of the stem in both
          /H/ and /Ø/ verbs. 

 
\ea\label{ex:xPattern8Summ} 

\begin{tabular}{llll}  
  
                   \textbf{Pattern 8}  &   \multicolumn{2}{l}{
                   \ul{Habitual (TP = H): 
                  } } &  \\
/H/  &   
                   \vernacular{
                  y-aá\ob [{\downstep}kháláká]\cb }  &   
                   \gloss{‘s/he will
                  cut’}  &  \\
  &   
                   \vernacular{
                  y-aá\ob [{\downstep}βóólítsá]\cb }  &   
                   \gloss{‘s/he will
                  seduce’}  &  \\
/Ø/  &   
                   \vernacular{
                  y-aá\ob [{\downstep}kúlíkhá]\cb }  &   
                   \gloss{‘s/he will
                  name’}  &  \\
  &   
                   \vernacular{
                  y-aá\ob [{\downstep}lákhúúlá]\cb }  &   
                   \gloss{‘s/he will
                  release’}  &  \\
\end{tabular}
%\caption{\nocaption}
    
\z

 I analyze differences among tonal melodies as arising
          from differences in (i) the construction-specific number
          of inflectional Hs, (ii) the positions targeted by
          melodic H assignment (MHA) rules, which are often
          construction-specific, and (iii) a number of tonal
          adjustment rules, which may spread, delete, or lower
          inflectional and lexical Hs. 

 Constructions may contribute up to two melodic Hs, or
          none at all. Pattern 1 constructions are tonally
          uninflected, receiving no inflectional Hs. Constructions
          exhibiting the properties of Patterns 2, 3, and 5 all
          receive one inflectional H, while constructions
          exhibiting the properties of Patterns 4, 6, 7, and 8 all
          receive two inflectional Hs. Not all inflectional Hs
          ultimate surface in a particular verb form. It is
          especially common for /H/ verbs not to realize all
          inflectional Hs. 

 The positions targeted by the various melodic H
          assignment rules are summarized in \REF{ex:xTargetsOfMHA} below. The second mora of the stem is a
          commonly targeted position for melodic H assignment in
          /Ø/ verbs, but melodic Hs may target any of the positions
          listed below, as licensed by morpho-syntactic
          context.

 
\ea\label{ex:xTargetsOfMHA} 
Targets of Melodic H
            Assignment 


\begin{tabular}{llll}  
  a.  &   Initial mora of the stem  &   Pattern 6  &  \\
b.  &   Second mora of the stem  &   Patterns 2, 4, 5, 6, 7  &  \\
c.  &   All moras of the third syllable of the
                stem  &   Pattern 5a  &  \\
d.  &   Final mora of the stem  &   Patterns 5b, 5c, 6, 7, 8  &  \\
e.  &   Initial mora of the macrostem  &   Pattern 4  &  \\
f.  &   2
                 \textsuperscript{nd}mora after
                the initial syllable of the macrostem &   Pattern 3  &  \\
\end{tabular}
%\caption{\nocaption}
    
\z

 The tonal properties of /H/ verbs can be and very
          often are considerably different than those of /Ø/ verbs.
          Under my analysis, these differences are frequently
          attributed to the order in which melodic H assignment
          rules apply and whether the tonal class of a verb form
          permits the earliest melodic H assignment rules to apply
          to it. For instance, the rule targeting the second stem
          mora applies relatively early in the derivation, and
          requires that the mora preceding the targeted second mora
          is toneless. While this rule applies freely in many /Ø/
          verbs, it is blocked from applying in /H/ verbs because
          the root H occupies the mora preceding its target.
          Inflectional Hs are therefore assigned in /H/ verbs by
          later rules which target different positions. 

 Once assigned to the verb, inflectional Hs are then
          subject to a number of tonal adjustment rules. One such
          rule was mentioned previously in this section with
          reference to Pattern 1b, namely: \regel{Plateau}, \footnote{\label{fn:nPlateauOnlyMostlyGeneral} While \regel{Plateau}applies
            generally in Idakho, it also only applies optionally.
            My corpus of recordings includes particularly few
            instances of \regel{Plateau}applying in
            the Conditional Negative.


}%
\regel{Meeussen’s Rule}, which
          deletes the second of two adjacent Hs. This common rule
          may also be illustrated using data from Pattern 1 verb
          forms. Consider, for example, the data below. As was
          noted in § \sectref{sec:sObjectPrefixes} , object
          prefixes in Idakho are analyzed as underlyingly H. In /Ø/
          Pattern 1 verbs, object prefixes contribute a H to the
          verb form which surfaces \textit{in situ} \REF{ex:xIllustrateMR} a). In \REF{ex:xIllustrateMR} b), one observes that the
          introduction of a H-toned object prefix causes the root H
          in /H/ verbs not to surface.

 
\ea\label{ex:xIllustrateMR} 
Meeussen’s Rule, Near Future: \gloss{‘s/he
            will...(him/her)’}


\begin{tabular}{lllll}  
    &   No Object Prefix  &   One Object Prefix  &   Gloss  &  \\
a.  &   
                   \vernacular{
                  a-la\ob [lakhuula]\cb }  &   
                   \vernacular{a-la\ob 
                  }  &   
                   \gloss{‘release’}  &  \\
b.  &   
                   \vernacular{
                  a-la\ob [khálaka]\cb }  &   
                   \vernacular{a-la\ob 
                  }  &   
                   \gloss{‘cut’}  &  \\
\end{tabular}
%\caption{\nocaption}
    
\z

  \regel{Meeussen’s Rule}is
          analyzed as preceding \regel{Plateau}in the
          derivation, such that the \regel{Plateau}does not feed \regel{Meeussen’s Rule}.

 Finally, all tonally inflected constructions (Patterns
          2-8) are affected by a general rule of \regel{Initial Lowering},
          whereby Hs underlyingly initial within the macrostem are
          rendered L. The pervasiveness of this rule within the
          verbal tonology of Idakho is one reason that the root H
          fails to surface in the majority of the data presented in
          this section. Within Luhya, there is evidence that it has
          also directly contributed to the introduction of tonal
          melodies in historically uninflected constructions: see
          how in Chapter \sectref{sec:cPathToPredictability} .



\section{Organization}\label{sec:sOrganization}

The present work makes contributions in two areas of
          Bantu tone. First, I will describe and analyze the verbal
          tone patterns of Idakho. Following this, I will describe
          and offer an account for a series of tonal developments
          within Luhya. 

 Chapter \sectref{sec:cVerbalTone} presents and
          analyzes in more detail the basic properties of Idakho’s
          12 tonal melodies. It also describes how these tonal
          melodies interact with a verb’s stem length and phrase
          position, as well as the presence of H-toned object
          prefixes, L-toned subject prefixes, and spuriously
          H-toned passive suffixes.

 Chapter \sectref{sec:cPathToPredictability} is
          comprised of two components, both relating to the
          emergence of ‘predictability’ in western Luhya varieties.
          The first section offers an account of how tonal melodies
          came to be used in constructions in which there is no
          historical precedence for tonal inflection. The second
          section describes patterns of neutralization in the
          vowel-initial verb roots of tonally conservative
          varieties like Idakho. Both sections recommend a unified
          account of their respective phenomena in terms of a
          speaker preference for clear prosodic cues of stem
          boundary position.

 Chapter \sectref{sec:cConclusions} briefly
          concludes.



\chapter{Idakho Verbal Tone}\label{sec:cVerbalTone}

This chapter provides a comprehensive description and
        analysis of the tonal properties of the Idakho verb. In the
        following sections, I detail the properties of Idakho’s
        twelve tonal melodies, which I group into eight ‘Patterns’.
        Tonal melodies belonging to the same Pattern share most of
        the same core properties, but differ with respect to a
        peripheral property, such as the underlying tonal
        specification of verbal prefixes or the construction’s
        behavior in a phrase-medial context. The chapter devotes
        one section to each of the eight Patterns I identify and
        concludes with a final summary. 

 The first eight sections have the same general structure
        and begin with a description of the melody as it is
        realized in its most morphologically simple form using
        representative lexical items with varying prosodic
        properties (namely, syllable weight and stem length).
        Following this, the influence of single and multiple object
        prefixes will be demonstrated, and an analysis offered of
        the pattern’s core properties. Next, peripheral properties
        of the tonal melodies will be presented, including the
        effect, if any, of (i) the verb's position within its
        phrase, (ii) the choice of verbal subject, and (iii) the
        presence of the spuriously H-toned passive suffix. Finally,
        each section will conclude with a discussion of other
        morpho-syntactic contexts in which the melody in question
        may be observed. 

 The chapter is organized in this way so that this
        dissertation may be used as a reference in which it is easy
        to look up data of a specific type (e.g., the behavior of
        passives in Pattern 2b). This structure can make it more
        difficult to present clear analytical arguments,
        particularly of the peripheral properties of tonal melodies
        such as passives and phrase-position effects. I have
        endeavored to provide useful summaries at various points
        throughout the chapter to remedy this limitation. 


\section{Pattern 1: The lexical pattern}\label{sec:sPattern1}

The description of Idakho’s verbal tone system begins
          with an examination of the tonal properties of Pattern 1,
          which is characterized by a H on the initial stem mora in
          /H/ stems and /Ø/ stems surface all L in basic contexts.
          There is no melodic H in Pattern 1; only lexical Hs
          surface, revealing the lexical contrast between /H/ and
          /Ø/ verbs. 

 Two sub-patterns comprise Pattern 1: Pattern 1a and
          Pattern 1b. Pattern 1a constructions are marked with
          either a toneless tense prefix or no tense prefix;
          Pattern 1b constructions are marked with a H-toned tense
          prefix. 


\subsection{Pattern 1a: Near Future}\label{sec:sPattern1a}

The Near Future illustrates the properties of the
            lexical pattern. It is marked with the toneless \vernacular{la-}tense
            prefix and is uninflected with a melodic H. As shown in \REF{ex:xNearFutCH} , the root H of /H/ verbs on the initial
            stem mora is the only H that surfaces in
            consonant-initial (henceforth, ‘C-initial’).

 
\ea\label{ex:xNearFutCH} 
Near Future C-Initial /H/ \gloss{‘s/he
              will...’}


\begin{tabular}{lllll}  
  Subj  &   Tns  &   Stem  &   Gloss  &  \\

                     \vernacular{a-}  &   
                     \vernacular{la}  &   
                     \vernacular{
                    \ob [khúa]\cb }  &   
                   \gloss{‘pay
                  dowry’}[SB] \footnote{\label{fn:nSpeakerInitials} I use speaker initials in square brackets to
                    indicate that recordings for the particular
                    example are only available for one speaker.
                    This represents differences in the stimuli
                    presented to each speaker rather than systemic
                    phonological differences unless explicitly
                    acknowledged in the text. 


}%
 &  \\

                     \vernacular{a-}  &   
                     \vernacular{la}  &   
                     \vernacular{
                    \ob [βéka]\cb }  &   
                     \gloss{‘shave’}  &  \\

                     \vernacular{a-}  &   
                     \vernacular{la}  &   
                     \vernacular{
                    \ob [téekha]\cb }  &   
                     \gloss{‘cook’}  &  \\

                     \vernacular{a-}  &   
                     \vernacular{la}  &   
                     \vernacular{
                    \ob [khálaka]\cb }  &   
                     \gloss{‘cut’}  &  \\

                     \vernacular{a-}  &   
                     \vernacular{la}  &   
                     \vernacular{
                    \ob [kálaanga]\cb }  &   
                     \gloss{‘fry’}  &  \\

                     \vernacular{a-}  &   
                     \vernacular{la}  &   
                     \vernacular{
                    \ob [sáanditsa]\cb }  &   
                     \gloss{‘thank’}  &  \\

                     \vernacular{a-}  &   
                     \vernacular{la}  &   
                     \vernacular{
                    \ob [βóyong’ana]\cb }  &   
                     \gloss{‘go around’}  &  \\
\end{tabular}
%\caption{\nocaption}
    
\z

 In forms involving stems beginning with vowels
            (henceforth, ‘V-Initial’), the lexical H surfaces on
            the latter half of the syllable which straddles the
            left stem boundary. The vowel of the tense prefix \vernacular{la-}deletes
            preceding the initial vowel of the verbal root, which
            in turn undergoes compensatory lengthening. The result
            is a surface rising tone on the first syllable that
            includes segmental content contributed by the verb
            root.

 
\ea\label{ex:xNearFutVH} 
Near Future V-Initial /H/ \gloss{‘s/he
              will...’}


\begin{tabular}{lllll}  
  Subj  &   Tns  &   Stem  &   Gloss  &  \\

                     \vernacular{a-}  &   
                     \vernacular{li}  &   
                     \vernacular{
                    \ob [íra]\cb }  &   
                     \gloss{‘kill’}  &  \\

                     \vernacular{a-}  &   
                     \vernacular{lo}  &   
                     \vernacular{
                    \ob [ónonyinya]\cb }  &   
                     \gloss{‘spoil’}  &  \\

                     \vernacular{a-}  &   
                     \vernacular{la}  &   
                     \vernacular{
                    \ob [ábukhanyinya]\cb }  &   
                     \gloss{‘separate’}  &  \\
\end{tabular}
%\caption{\nocaption}
    
\z

 In contrast, /Ø/ verbs are realized all L in both
            C-initial \REF{ex:xNearFutCØ} and V-initial \REF{ex:xNearFutVØ} stems.

 
\ea\label{ex:xNearFutCØ} 
Near Future C-Initial /Ø/ \gloss{‘s/he
              will...’}


\begin{tabular}{lllll}  
  Subj  &   Tns  &   Stem  &   Gloss  &  \\

                     \vernacular{a-}  &   
                     \vernacular{la}  &   
                     \vernacular{
                    \ob [kwaa]\cb }  &   
                     \gloss{‘fall’}  &  \\

                     \vernacular{a-}  &   
                     \vernacular{la}  &   
                     \vernacular{
                    \ob [lekha]\cb }  &   
                     \gloss{‘leave’}  &  \\

                     \vernacular{a-}  &   
                     \vernacular{la}  &   
                     \vernacular{
                    \ob [reeβa]\cb }  &   
                     \gloss{‘ask’}  &  \\

                     \vernacular{a-}  &   
                     \vernacular{la}  &   
                     \vernacular{
                    \ob [sosana]\cb }  &   
                   \gloss{
                  ‘resemble’}[SB] &  \\

                     \vernacular{a-}  &   
                     \vernacular{la}  &   
                     \vernacular{
                    \ob [lakhuula]\cb }  &   
                     \gloss{‘release’}  &  \\

                     \vernacular{a-}  &   
                     \vernacular{la}  &   
                     \vernacular{
                    \ob [seeβula]\cb }  &   
                     \gloss{‘say
                    goodbye’}  &  \\

                     \vernacular{a-}  &   
                     \vernacular{la}  &   
                     \vernacular{
                    \ob [kalushitsa]\cb }  &   
                   \gloss{
                  ‘return’}[SB] &  \\
\end{tabular}
%\caption{\nocaption}
    
\z

 
\ea\label{ex:xNearFutVØ} 
Near Future V-Initial /Ø/ \gloss{‘s/he
              will...’}


\begin{tabular}{lllll}  
  Subj  &   Tns  &   Stem  &   Gloss  &  \\

                     \vernacular{a-}  &   
                     \vernacular{le}  &   
                     \vernacular{
                    \ob [enya]\cb }  &   
                     \gloss{‘want’}  &  \\

                     \vernacular{a-}  &   
                     \vernacular{le}  &   
                     \vernacular{
                    \ob [eyela]\cb }  &   
                     \gloss{‘wipe for’}  &  \\

                     \vernacular{a-}  &   
                     \vernacular{li}  &   
                     \vernacular{
                    \ob [iluula]\cb }  &   
                     \gloss{‘winnow’}  &  \\

                     \vernacular{a-}  &   
                     \vernacular{la}  &   
                     \vernacular{
                    \ob [ambakhana]\cb }  &   
                     \gloss{‘refuse’}  &  \\
\end{tabular}
%\caption{\nocaption}
    
\z


\subsubsection{Near Future with Object Prefixes}\label{sec:sP1aObjects}

The lexical contrast among verb roots is lost when
              an object prefix is introduced. The object prefix
              itself surfaces H, but the stems of both /H/ \REF{ex:xNearFutCHOP} and /Ø/ \REF{ex:xNearFutCØOP} verbs are all L.

 
\ea\label{ex:xNearFutCHOP} 
Near Future C-Initial /H/ + OP \gloss{‘s/he
                will...him/her'}


\begin{tabular}{llllll}  
  Subj  &   Tns  &   Obj  &   Stem  &   Gloss  &  \\

                       \vernacular{a-}  &   
                       \vernacular{la}  &   
                       \vernacular{\ob mú}  &   
                       \vernacular{
                      [khwaa]\cb }  &   
                       \gloss{‘pay
                      dowry’}  &  \\

                       \vernacular{a-}  &   
                       \vernacular{la}  &   
                       \vernacular{\ob mú}  &   
                       \vernacular{[βeka]\cb 
                      }  &   
                       \gloss{‘shave’}  &  \\

                       \vernacular{a-}  &   
                       \vernacular{la}  &   
                       \vernacular{\ob mú}  &   
                       \vernacular{
                      [leera]\cb }  &   
                       \gloss{‘bring’}  &  \\

                       \vernacular{a-}  &   
                       \vernacular{la}  &   
                       \vernacular{\ob mú}  &   
                       \vernacular{
                      [khalaka]\cb }  &   
                       \gloss{‘cut’}  &  \\

                       \vernacular{a-}  &   
                       \vernacular{la}  &   
                       \vernacular{\ob mú}  &   
                       \vernacular{
                      [βoolitsa]\cb }  &   
                       \gloss{‘seduce’}  &  \\

                       \vernacular{a-}  &   
                       \vernacular{la}  &   
                       \vernacular{\ob mú}  &   
                       \vernacular{
                      [βoyong’ana]\cb }  &   
                       \gloss{‘go
                      around’}  &  \\
\end{tabular}
%\caption{\nocaption}
    
\z

 
\ea\label{ex:xNearFutCØOP} 
Near Future C-Initial /Ø/ + OP \gloss{‘s/he
                will...him/her'}


\begin{tabular}{llllll}  
  Subj  &   Tns  &   Obj  &   Stem  &   Gloss  &  \\

                       \vernacular{a-}  &   
                       \vernacular{la}  &   
                       \vernacular{\ob mú}  &   
                       \vernacular{
                      [tsia]\cb }  &   
                       \gloss{‘go (for)’}  &  \\

                       \vernacular{a-}  &   
                       \vernacular{la}  &   
                       \vernacular{\ob mú}  &   
                       \vernacular{
                      [lekha]\cb }  &   
                       \gloss{‘leave’}  &  \\

                       \vernacular{a-}  &   
                       \vernacular{la}  &   
                       \vernacular{\ob mú}  &   
                       \vernacular{
                      [loonda]\cb }  &   
                       \gloss{‘follow’}  &  \\

                       \vernacular{a-}  &   
                       \vernacular{la}  &   
                       \vernacular{\ob mú}  &   
                       \vernacular{
                      [kumila]\cb }  &   
                     \gloss{
                    ‘hold’}[JI] &  \\

                       \vernacular{a-}  &   
                       \vernacular{la}  &   
                       \vernacular{\ob mú}  &   
                       \vernacular{
                      [lakhuula]\cb }  &   
                       \gloss{‘release’}  &  \\

                       \vernacular{a-}  &   
                       \vernacular{la}  &   
                       \vernacular{\ob mú}  &   
                       \vernacular{
                      [kalushitsa]\cb }  &   
                     \gloss{
                    ‘return’}[SB] &  \\
\end{tabular}
%\caption{\nocaption}
    
\z

 The same generalization holds in V-Initial stems.
              In \REF{ex:xNearFutVHOP} , the H of the object prefix surfaces \textit{in situ}, but the root
              H fails to surface. The result is a falling tone in
              the syllable which straddles the stem boundary.

 
\ea\label{ex:xNearFutVHOP} 
Near Future V-Initial /H/ + OP \gloss{‘s/he
                will...him/her'}


\begin{tabular}{llllll}  
  Subj  &   Tns  &   Obj  &   Stem  &   Gloss  &  \\

                       \vernacular{a-}  &   
                       \vernacular{la}  &   
                       \vernacular{\ob mwí}  &   
                       \vernacular{
                      [ira]\cb }  &   
                       \gloss{‘kill’}  &  \\

                       \vernacular{a-}  &   
                       \vernacular{la}  &   
                       \vernacular{\ob mwó}  &   
                       \vernacular{
                      [ononyinya]\cb }  &   
                       \gloss{‘spoil’}  &  \\

                       \vernacular{a-}  &   
                       \vernacular{la}  &   
                       \vernacular{\ob mwá}  &   
                       \vernacular{
                      [abukhanyinya]\cb }  &   
                       \gloss{‘separate’}  &  \\
\end{tabular}
%\caption{\nocaption}
    
\z

 /Ø/ V-initial verbs exhibit the same surface
              tonal properties in \REF{ex:xNearFutVØOP} , in which only the H of the object
              prefix surfaces.

 
\ea\label{ex:xNearFutVØOP} 
Near Future V-Initial /Ø/ + OP \gloss{‘s/he
                will...him/her'}


\begin{tabular}{llllll}  
  Subj  &   Tns  &   Obj  &   Stem  &   Gloss  &  \\

                       \vernacular{a-}  &   
                       \vernacular{la-}  &   
                       \vernacular{\ob mwé}  &   
                       \vernacular{
                      [enya]\cb }  &   
                       \gloss{‘want’}  &  \\

                       \vernacular{a-}  &   
                       \vernacular{la-}  &   
                       \vernacular{\ob mwé}  &   
                       \vernacular{
                      [eyela]\cb }  &   
                       \gloss{‘wipe for’}  &  \\

                       \vernacular{a-}  &   
                       \vernacular{la-}  &   
                       \vernacular{\ob mwá}  &   
                       \vernacular{
                      [ambakhana]\cb }  &   
                       \gloss{‘refuse’}  &  \\
\end{tabular}
%\caption{\nocaption}
    
\z

 The H of the 1 \textsuperscript{st}sg object
              prefix surfaces as a rising tone on the lengthened
              pre-stem syllable. In the displays below, the
              tone-bearing mora contributed by the 1 \textsuperscript{st}sg object
              prefix is represented under the ‘Obj’ column, and the
              segmental contribution of the prefix is represented
              under the ‘Stem’ column. This is done because the
              nasal object prefix frequently triggers phonological
              alternations such as hardening and post-nasal
              obstruent voicing, and representing the nasal nearer
              to the alternating segment renders these alternations
              easier to recognize. \footnote{\label{fn:nNasalDeletion2} Nasals delete before voiceless fricatives. 


}%
\textsuperscript{st}sg object
              prefixes \REF{ex:xNearFutCHOP1sg} , as was the case
              with CV- object prefixes \REF{ex:xNearFutCHOP} .

 
\ea\label{ex:xNearFutCHOP1sg} 
Near Future C-Initial /H/ + OP \textsubscript{1sg} \gloss{‘s/he
                will...me'}


\begin{tabular}{llllll}  
  Subj  &   Tns  &   Obj  &   Stem  &   Gloss  &  \\

                       \vernacular{a-}  &   
                       \vernacular{la}  &   
                       \vernacular{\ob á}  &   
                       \vernacular{
                      [khwaa]\cb }  &   
                       \gloss{‘pay
                      dowry’}  &  \\

                       \vernacular{a-}  &   
                       \vernacular{la}  &   
                       \vernacular{\ob á}  &   
                       \vernacular{
                      [mbeka]\cb }  &   
                       \gloss{‘shave’}  &  \\

                       \vernacular{a-}  &   
                       \vernacular{la}  &   
                       \vernacular{\ob á}  &   
                       \vernacular{
                      [ndeera]\cb }  &   
                       \gloss{‘bring’}  &  \\

                       \vernacular{a-}  &   
                       \vernacular{la}  &   
                       \vernacular{\ob á}  &   
                       \vernacular{
                      [khalaka]\cb }  &   
                       \gloss{‘cut’}  &  \\

                       \vernacular{a-}  &   
                       \vernacular{la}  &   
                       \vernacular{\ob á}  &   
                       \vernacular{
                      [mboolitsa]\cb }  &   
                       \gloss{‘seduce’}  &  \\

                       \vernacular{a-}  &   
                       \vernacular{la}  &   
                       \vernacular{\ob á}  &   
                       \vernacular{
                      [mboyong’ana]\cb }  &   
                       \gloss{‘go
                      around’}  &  \\
\end{tabular}
%\caption{\nocaption}
    
\z

 In \REF{ex:xNearFutCØOP1sg} it is shown that
              the H contributed by the 1 \textsuperscript{sg}object prefix
              is the only H that surfaces in /Ø/ verbs with this
              object prefix in the Near Future.

 
\ea\label{ex:xNearFutCØOP1sg} 
Near Future C-Initial /Ø/ + OP \textsubscript{1sg} \gloss{‘s/he
                will...me'}


\begin{tabular}{llllll}  
  Subj  &   Tns  &   Obj  &   Stem  &   Gloss  &  \\

                       \vernacular{a-}  &   
                       \vernacular{la}  &   
                       \vernacular{\ob á}  &   
                       \vernacular{
                      [ndekha]\cb }  &   
                       \gloss{‘leave’}  &  \\

                       \vernacular{a-}  &   
                       \vernacular{la}  &   
                       \vernacular{\ob á}  &   
                       \vernacular{
                      [noonda]\cb }  &   
                       \gloss{‘follow’}  &  \\

                       \vernacular{a-}  &   
                       \vernacular{la}  &   
                       \vernacular{\ob á}  &   
                       \vernacular{
                      [ndakhuula]\cb }  &   
                       \gloss{‘release’}  &  \\

                       \vernacular{a-}  &   
                       \vernacular{la}  &   
                       \vernacular{\ob á}  &   
                       \vernacular{
                      [seeβula]\cb }  &   
                       \gloss{‘say goodbye
                      (to)’}  &  \\
\end{tabular}
%\caption{\nocaption}
    
\z

 When a 1 \textsuperscript{st}sg object
              prefix appears in conjunction with a CV- object
              prefix, a falling tone on the pre-stem syllable is
              observed. Here again the lexical contrast is
              neutralized: the stem is toneless in both /H/ and /Ø/
              stems. I am not aware of any ditransitive
              monosyllabic stems; a valence-increasing suffix like
              the applicative \vernacular{-il/el}is
              required to license a second object prefix, so that
              even verb forms with monosyllabic roots involve
              disyllabic stems.

 
\ea\label{ex:xNearFutCHOPOP1sg} 
Near Future C-Initial /H/ + OP + OP \textsubscript{1sg} \gloss{‘s/he will...him/her
                for me'}


\begin{tabular}{lllllll}  
  Subj  &   Tns  &   Obj
                     \textsubscript{CV} &   Obj
                     \textsubscript{1sg} &   Stem  &   Gloss  &  \\

                       \vernacular{a-}  &   
                       \vernacular{la}  &   
                       \vernacular{\ob mú-}  &   
                       \vernacular{u}  &   
                       \vernacular{
                      [ndeela]\cb }  &   
                       \gloss{‘bury’}  &  \\

                       \vernacular{a-}  &   
                       \vernacular{la}  &   
                       \vernacular{\ob mú-}  &   
                       \vernacular{u}  &   
                       \vernacular{
                      [mbechela]\cb }  &   
                       \gloss{‘shave’}  &  \\

                       \vernacular{a-}  &   
                       \vernacular{la}  &   
                       \vernacular{\ob mú-}  &   
                       \vernacular{u}  &   
                       \vernacular{
                      [ndeerela]\cb }  &   
                       \gloss{‘bring’}  &  \\

                       \vernacular{a-}  &   
                       \vernacular{la}  &   
                       \vernacular{\ob mú-}  &   
                       \vernacular{u}  &   
                       \vernacular{
                      [khalachila]\cb }  &   
                       \gloss{‘cut’}  &  \\
\end{tabular}
%\caption{\nocaption}
    
\z

 
\ea\label{ex:xNearFutCØOPOP1sg} 
Near Future C-Initial /Ø/ + OP + OP \textsubscript{1sg} \gloss{‘s/he will...him/her
                for me'}


\begin{tabular}{lllllll}  
  Subj  &   Tns  &   Obj
                     \textsubscript{CV} &   Obj
                     \textsubscript{1sg} &   Stem  &   Gloss  &  \\

                       \vernacular{a-}  &   
                       \vernacular{la}  &   
                       \vernacular{\ob mú-}  &   
                       \vernacular{u}  &   
                       \vernacular{
                      [nziila]\cb }  &   
                       \gloss{‘go (for)’}  &  \\

                       \vernacular{a-}  &   
                       \vernacular{la}  &   
                       \vernacular{\ob mú-}  &   
                       \vernacular{u}  &   
                       \vernacular{
                      [ndeshela]\cb }  &   
                       \gloss{‘leave’}  &  \\

                       \vernacular{a-}  &   
                       \vernacular{la}  &   
                       \vernacular{\ob mú-}  &   
                       \vernacular{u}  &   
                       \vernacular{
                      [noondela]\cb }  &   
                       \gloss{‘follow’}  &  \\

                       \vernacular{a-}  &   
                       \vernacular{la}  &   
                       \vernacular{\ob mú-}  &   
                       \vernacular{u}  &   
                       \vernacular{
                      [ngulishila]\cb }  &   
                     \gloss{
                    ‘name’}[SB] &  \\
\end{tabular}
%\caption{\nocaption}
    
\z

 A summary of the primary tonal properties of the
              Near Future is presented schematically in the
              following display. 

 
\ea\label{ex:xNearFutSchematic} 
A Schematic Representation of the
                Near Future’s Tonal Properties 


\begin{tabular}{lllll}  
    &   \multicolumn{3}{l}{
                       \ul{/H/ Verbs} } &  \\
  &   
                       \textit{Subj + Tns}  &   \multicolumn{2}{l}{
                       \textit{Macrostem} } &  \\
OPsx0  &   
                       \vernacular{a-la}  &   
                       \vernacular{\ob }  &   
                       \vernacular{[C
                      }  &  \\
OPsx1  &   
                       \vernacular{a-la}  &   
                       \vernacular{\ob C
                      }  &   
                       \vernacular{[C
                      }  &  \\
OPsx2  &   
                       \vernacular{a-la}  &   
                       \vernacular{\ob C
                      }  &   
                       \vernacular{[C
                      }  &  \\
  &     &   \multicolumn{2}{l}{ } &  \\
  &   \multicolumn{3}{l}{
                       \textbf{
                        } } &  \\
  &   
                       \textit{Subj + Tns}  &   \multicolumn{2}{l}{
                       \textit{Macrostem} } &  \\
OPsx0  &   
                       \vernacular{a-la}  &   
                       \vernacular{\ob }  &   
                     \vernacular{
                    [CVCVCV]}\cb  &  \\
OPsx1  &   
                       \vernacular{a-la}  &   
                       \vernacular{\ob C
                      }  &   
                       \vernacular{
                      [CVCVCV]\cb }  &  \\
OPsx2  &   
                       \vernacular{a-la}  &   
                       \vernacular{\ob C
                      }  &   
                       \vernacular{
                      [CVCVCV]\cb }  &  \\
\end{tabular}
%\caption{\nocaption}
    
\z

 The common cross-Bantu \regel{Meeussen's Rule}may
              be invoked to explain the tonal properties of all of
              the forms above involving object prefixes. In Idakho, \regel{Meeussen's
              Rule}deletes a H after H. Recall that in
              forms involving only one object prefix, the H of the
              object prefix surfaces \textit{in situ}, but the root
              H does not surface at all. In forms with two object
              prefixes, there are up to three adjacent underlying
              Hs: (i) the H of the CV object prefix, (ii) the H of
              the 1 \textsuperscript{st}sg object
              prefix, and (iii) the lexical H (in /H/ stems). Only
              the leftmost underlying H surfaces in such cases. I
              account for this by stipulating that \regel{Meeussen’s
              Rule}applies iteratively from
              right-to-left. I formulate \regel{Meeussen’s Rule}as
              in \REF{ex:xMeeussensRule} below.

 
\ea\label{ex:xMeeussensRule} 
 \regel{Meeussen’s
                  Rule} 

%\includegraphics[width=\textwidth]{InkScape%20Images/Rules/MeeussensRule.pdf}

\z

 In the derivation below, the analysis of all Near
              Future forms with a single object prefix is
              modelled. 

 
\ea\label{ex:xDerivNearFutHOP} 
 Derivation,
                  /H/ Near Future + OP: \vernacular{
                  a-la\ob mú[khalaka]\cb } \gloss{‘s/he will cut
                  him/her’} 

%\includegraphics[width=\textwidth]{InkScape%20Images/Derivations/DerivNearFutHOP.pdf}

\z

 Finally, the derivation below illustrates the
              iterativity and directionality of \regel{Meeussen’s
              Rule}.

 
\ea\label{ex:xDerivNearFutHOPx2} 
 Derivation,
                  /H/ Near Future + OPx2: \vernacular{
                  a-la\ob mú-u[mbechela]\cb } \gloss{‘s/he will shave
                  him/her for me’} 

%\includegraphics[width=\textwidth]{InkScape%20Images/Derivations/DerivNearFutHOPx2.pdf}

\z



\subsubsection{Near Future: Phrase Medially}\label{sec:sP1aPhraseMed}

The basic tonal properties of the Near Future are
              unaffected by the verb’s position within its phrase,
              with one notable exception. It continues to be the
              case that only the leftmost underlying H surfaces on
              the verb, but an additional H spanning the full
              length of the verb up until the leftmost underlying H
              is also realized when the post-verbal word is
              H-toned, as shown below. 

 Four pairs of /H/ and /Ø/ verbs each are provided
              below, half with and half without an object prefix.
              For each pair, the first member involves a H-toned
              complement, while the second involves a toneless
              complement. Notice that the stem tone properties of
              the verb are the same as they are pre-pausally only
              when a toneless object follows. 

 
\ea\label{ex:xNearFutPhraseMedial} 
Near Future Phrase Medially [SB] \gloss{‘s/he will...(for
                him/her)’}


\begin{tabular}{lllll}  
  
                       %\includegraphics[width=\textwidth]{InkScape%20Images/H%20Stems.svg}
 &   
                       %\includegraphics[width=\textwidth]{InkScape%20Images/No%20OP.svg}
 &   
                       \vernacular{a-la\ob [rá]\cb 
                      {\downstep}músáatsa}  &   
                       \gloss{‘bury the man’
                      }  &  \\

                       \vernacular{a-la\ob [rá]\cb 
                      muundu}  &   
                       \gloss{‘bury
                      somebody’}  &  \\
  &     &  \\

                       \vernacular{
                      a-la\ob [khá{\downstep}láká]\cb  músáatsa}  &   
                       \gloss{‘cut the
                      man’}  &  \\

                       \vernacular{a-la\ob [khálaka]\cb 
                      muundu}  &   
                       \gloss{‘cut
                      somebody’}  &  \\
  &     &     &  \\

                       %\includegraphics[width=\textwidth]{InkScape%20Images/One%20OP.svg}
 &   
                       \vernacular{
                      a-la\ob mú[{\downstep}réélá]\cb  músáatsa}  &   
                       \gloss{‘bury the
                      man’}  &  \\

                       \vernacular{a-la\ob mú[reela]\cb 
                      muundu}  &   
                       \gloss{‘bury
                      somebody’}  &  \\
  &     &  \\

                       \vernacular{
                      a-la\ob mú[{\downstep}kháláchílá]\cb 
                      músáatsa}  &   
                       \gloss{‘cut the
                      man’}  &  \\

                       \vernacular{
                      a-la\ob mú[khalachila]\cb  muundu}  &   
                       \gloss{‘cut
                      somebody’}  &  \\
  &     &     &  \\

                       %\includegraphics[width=\textwidth]{InkScape%20Images/0%20Stems.svg}
 &   
                       %\includegraphics[width=\textwidth]{InkScape%20Images/No%20OP.svg}
 &   
                       \vernacular{a-lá\ob [tsyá]\cb 
                      músáatsa}  &   
                       \gloss{‘go for the man’
                      }  &  \\

                       \vernacular{a-la\ob [tsya]\cb 
                      muundu}  &   
                       \gloss{‘go for
                      somebody’}  &  \\
  &     &  \\

                       \vernacular{
                      a-lá\ob [sééβúlá]\cb  músáatsa}  &   
                       \gloss{‘say goodbye to the
                      man’}  &  \\

                       \vernacular{a-la\ob [seeβula]\cb 
                      muundu}  &   
                       \gloss{‘say goodbye to
                      somebody’}  &  \\
  &     &     &  \\

                       %\includegraphics[width=\textwidth]{InkScape%20Images/One%20OP.svg}
 &   
                       \vernacular{
                      a-la\ob mú[{\downstep}tsíílá]\cb  músáatsa}  &   
                       \gloss{‘go for the
                      man’}  &  \\

                       \vernacular{
                      a-la\ob mú[tsiila]\cb  muundu}  &   
                       \gloss{‘go for
                      somebody’}  &  \\
  &     &  \\

                       \vernacular{
                      a-la\ob mú[{\downstep}sééβúlílá]\cb 
                      músáatsa}  &   
                       \gloss{‘say goodbye to the
                      man’}  &  \\

                       \vernacular{
                      a-la\ob mú[seeβulila]\cb  muundu}  &   
                       \gloss{‘say goodbye to
                      somebody’}  &  \\
\end{tabular}
%\caption{\nocaption}
    
\z

 The leftward spreading of post-verbal Hs is
              attested in other tonally conservative Luhya
              varieties, e.g. Tiriki ( \citealt{rMarloInPrepB} , \citealt{rPasterKim2011} ) and Logoori ( \citealt{rLeung1991} ).
              While it is clear that \regel{H Tone
              Anticipation}is a ubiquitous feature of the
              Idakho verbal tone system as well, the leftward
              extent of the resulting H span is sometimes less
              clear.

 H spans in Idakho are characterized by \textit{crescendo}, whereby
              later syllables within a single phonological H span
              are produced with progressively higher pitch. \footnote{\label{fn:nCrescendoInBukusu}  \citealt{rAusten1974b} observes that \textit{crescendo}also
                characterizes sequences of H-toned syllables
                without downstep in the Bukusu variety of
                Luhya.


}%
\vernacular{la-},
              though my recordings include productions in which the
              H span appears to begin on the stem and even word
              initial moras. In light of descriptive challenges
              associated with the left edge of the H span, no
              attempt is made to specify the terminus of leftward
              spreading in the formal statement of \regel{H Tone
              Anticipation}below.

 
\ea\label{ex:xHToneAnticipation} 
 \regel{H Tone
                  Anticipation} 

%\includegraphics[width=\textwidth]{InkScape%20Images/Rules/HToneAnticipation.pdf}

\z

 The formalism offered above has the following
              notable features: the σ \textsubscript{0}in initial
              position within the post-verbal word is intended to
              capture the generalization that, while the underlying
              H undergoing leftward spreading via \regel{H Tone
              Anticipation}may be initial within the
              word, it need not be. The rule is structured to also
              capture the generalization that leftward spreading
              will only take place if doing so will result in a H
              span that reaches into the verb stem. The latter
              feature is important to account for the fact that \gloss{‘man’}is
              produced as \vernacular{
              musáatsa}in isolation, not * \vernacular{
              músáatsa}.

 The data in \REF{ex:xNearFutPhraseMedial} also
              illustrate an effect of a rule of \regel{Non-Final
              Shortening}, whereby underlyingly long word
              final syllables are shortened when non-final within
              the phrase. This is the case in monosyllabic stems:
              the final and only stem syllable is underlyingly
              bimoraic, with one mora originating from the verbal
              root and the second from the FV. Pre-pausally that
              underlying length is expressed. However, when another
              word follows the verb, the final syllable is
              shortened. The Nyala-West Luhya variety also exhibits
              such a rule ( \citealt{rEbarbEtAlInPrep} ), and it may be formulated as below.

 
\ea\label{ex:xNonFinalShortening} 
 \regel{Non-Final
                  Shortening} 

%\includegraphics[width=\textwidth]{InkScape%20Images/Rules/NonFinalShortening.pdf}

\z

 Finally, note that for post-verbal complements
              with an initial H, that H will be downstepped
              relative to the lexical H in monosyllabic, as in,
              e.g., \vernacular{a-la\ob [rá]\cb 
              {\downstep}mú{\downstep}yáyi} \gloss{‘she will bury the
              boy’}.



\subsubsection{Near Future: Impact of Subject
              Choice}\label{sec:sP1aSubjects}

In \REF{ex:xSubjNearFutH} below, observe that
              the root H surfaces in verb forms with 3 \textsuperscript{rd}person
              subjects, but does not in forms with 1 \textsuperscript{st}and 2 \textsuperscript{nd}person
              subjects.

 
\ea\label{ex:xSubjNearFutH} 
Subject Choice in the Near Future
                /H/ \gloss{‘...will
                bring’}[SB]


\begin{tabular}{llll}  
    &   Singular  &   Plural  &  \\
1
                     \textsuperscript{
                    st}Person &   
                       \vernacular{
                      n-a\ob [leera]\cb } \footnote{\label{fn:nLatDeletion} The 1 \textsuperscript{st}sg
                        subject prefix triggers deletion of the
                        tense prefix <l> as a diachronic
                        extension of a general process in Idakho
                        whereby laterals delete following nasals
                        when a second NC cluster appears later in
                        the word, e.g., \vernacular{
                        a-la-mú[loonda]} \gloss{‘she will follow
                        him’}vs. \vernacular{
                        a-la-á[noonda]} \vernacular{‘she will
                        follow me’}(/ \vernacular{
                        a-la-N[loonda]}/). This is known
                        as Meinhof’s Law ( \citealt{rHyman2003b} ).

 Cases in which one observes Meinhof’s
                        Law differ from cases in which /nl/
                        sequences which are not followed by a
                        nasal, e.g., \vernacular{
                        a-la-mú[leera]} \gloss{‘she will bring
                        him’}vs. \vernacular{
                        a-la-á[ndeera]} \gloss{‘she will bring
                        me’}. In this case the lateral
                        undergoes hardening rather than
                        deletion.


}%
 &   
                       \vernacular{
                      khu-la\ob [leera]\cb }  &  \\
2
                     \textsuperscript{
                    nd}Person &   
                       \vernacular{
                      u-la\ob [leera]\cb }  &   
                       \vernacular{
                      mu-la\ob [leera]\cb }  &  \\
3
                     \textsuperscript{
                    rd}Person &   
                       \vernacular{
                      a-la\ob [léera]\cb }  &   
                       \vernacular{
                      βa-la\ob [léera]\cb }  &  \\
\end{tabular}
%\caption{\nocaption}
    
\z

 There is no H in /Ø/ verbs regardless of subject
              selection, as shown in \REF{ex:xSubjNearFutØ} .

 
\ea\label{ex:xSubjNearFutØ} 
Subject Choice in the Near Future
                /Ø/ \gloss{‘...will
                ask’}[SB]


\begin{tabular}{llll}  
    &   Singular  &   Plural  &  \\
1
                     \textsuperscript{
                    st}Person &   
                       \vernacular{
                      n-a\ob [reeβa]\cb }  &   
                       \vernacular{
                      khu-la\ob [reeβa]\cb }  &  \\
2
                     \textsuperscript{
                    nd}Person &   
                       \vernacular{
                      u-la\ob [reeβa]\cb }  &   
                       \vernacular{
                      mu-la\ob [reeβa]\cb }  &  \\
3
                     \textsuperscript{
                    rd}Person &   
                       \vernacular{
                      a-la\ob [reeβa]\cb }  &   
                       \vernacular{
                      βa-la\ob [reeβa]\cb }  &  \\
\end{tabular}
%\caption{\nocaption}
    
\z

 After the 1 \textsuperscript{st}and 2 \textsuperscript{nd}person subject
              prefixes, the object prefix H, as well as the root H,
              fails to surface. This is shown in \REF{ex:xSubjNearFutHOP} below.

 
\ea\label{ex:xSubjNearFutHOP} 
Subject Choice in the Near Future
                /H/ + OP \gloss{‘...will bring
                him/her’}[SB]


\begin{tabular}{llll}  
    &   Singular  &   Plural  &  \\
1
                     \textsuperscript{
                    st}Person &   
                       \vernacular{
                      n-a\ob mu[leera]\cb }  &   
                       \vernacular{
                      khu-la\ob mu[leera]\cb }  &  \\
2
                     \textsuperscript{
                    nd}Person &   
                       \vernacular{
                      u-la\ob mu[leera]\cb }  &   
                       \vernacular{
                      mu-la\ob mu[leera]\cb }  &  \\
3
                     \textsuperscript{
                    rd}Person &   
                       \vernacular{
                      a-la\ob mú[leera]\cb }  &   
                       \vernacular{
                      βa-la\ob mú[leera]\cb }  &  \\
\end{tabular}
%\caption{\nocaption}
    
\z

 In /Ø/ verbs, the object prefix H surfaces in
              verb forms with 3 \textsuperscript{rd}person
              subjects, but not with 1 \textsuperscript{st}or 2 \textsuperscript{nd}person
              subjects.

 
\ea\label{ex:xSubjNearFutØOP} 
Subject Choice in the Near Future
                /Ø/ + OP \gloss{‘...will ask
                him/her’}[SB]


\begin{tabular}{llll}  
    &   Singular  &   Plural  &  \\
1
                     \textsuperscript{
                    st}Person &   
                       \vernacular{
                      n-a\ob mu[reeβa]\cb }  &   
                       \vernacular{
                      khu-la\ob mu[reeβa]\cb }  &  \\
2
                     \textsuperscript{
                    nd}Person &   
                       \vernacular{
                      u-la\ob mu[reeβa]\cb }  &   
                       \vernacular{
                      mu-la\ob mu[reeβa]\cb }  &  \\
3
                     \textsuperscript{
                    rd}Person &   
                       \vernacular{
                      a-la\ob mú[reeβa]\cb }  &   
                       \vernacular{
                      βa-la\ob mú[reeβa]\cb }  &  \\
\end{tabular}
%\caption{\nocaption}
    
\z

 The data in \REF{ex:xSubjNearFutH} , \REF{ex:xSubjNearFutHOP} , and \REF{ex:xSubjNearFutØOP} form the basis for
              positing underlying L for 1 \textsuperscript{st}and 2 \textsuperscript{nd}person subject
              prefixes and underlying tonelessness for 3 \textsuperscript{rd}person subject
              prefixes. I analyze the failure of the root and
              object prefix Hs to surface as resulting from the L
              of 1 \textsuperscript{st}and 2 \textsuperscript{nd}person subject
              prefixes spreading rightward through the initial mora
              of the stem, overwriting any tones present on the
              following syllable. I posit the rule \regel{L Spread I},
              formulated in \REF{ex:xLSpreadI} .

 
\ea\label{ex:xLSpreadI} 
 \regel{L Spread
                  I} 

%\includegraphics[width=\textwidth]{InkScape%20Images/Rules/L%20Spread%20I.pdf}

\z

 In verb forms without an object prefix, the
              subject prefix L spreads first to the toneless tense
              prefix \vernacular{la-},
              then second to the initial mora of the stem,
              delinking the root H.

 
\ea\label{ex:xDerivNearFutHSubj} 
 Derivation,
                  /H/ Near Future 2
                   \vernacular{
                  u-la\ob [leera]\cb } \gloss{‘you will
                  bring’} 

%\includegraphics[width=\textwidth]{InkScape%20Images/Derivations/DerivNearFutHSubj.pdf}

\z

 The tonal properties of verb forms with an object
              prefix proceeds in much the same way. First, the L of
              the subject prefix spreads to the toneless tense
              prefix \vernacular{la-},
              then to the object prefix \vernacular{mú-},
              and finally to the initial syllable of the stem. Both
              the object prefix and root Hs are delinked in the
              process.

 
\ea\label{ex:xDerivNearFutHOPSubj} 
 Derivation,
                  /H/ Near Future 2
                   \vernacular{
                  u-la\ob mu[leera]\cb } \gloss{‘you will bring
                  him/her’} 

%\includegraphics[width=\textwidth]{InkScape%20Images/Derivations/DerivNearFutHOPSubj.pdf}

\z



\subsubsection{Near Future: Passives}\label{sec:sP1aPassives}

In the Near Future, the passive suffix \vernacular{-u
              }does
              not surface H. Consider the data below.

 
\ea\label{ex:xNearFutPassives} 
Near Future: Passives \gloss{‘s/he will
                be...’}[SB]


\begin{tabular}{lllll}  
  \multicolumn{2}{l}{/H/ Stems } &   \multicolumn{2}{l}{/Ø/ Stems } &  \\

                       \vernacular{
                      a-la\ob [khálak-u-a]\cb }  &   
                       \gloss{‘cut’}  &   
                       \vernacular{
                      a-la\ob [lakhuul-u-a]\cb }  &   
                       \gloss{‘released’}  &  \\

                       \vernacular{
                      a-la\ob [tsúunzuun-u-a]\cb }  &   
                       \gloss{‘sucked’}  &   
                       \vernacular{
                      a-la\ob [kalushil-u-a]\cb }  &   
                       \gloss{‘defended’}  &  \\
\end{tabular}
%\caption{\nocaption}
    
\z

 It is important to note the absence of a H on the
              passive suffix in this context, as its presence has
              an impact on stem tone in other contexts. It is
              therefore necessary to account for when the passive
              does and does not realize a H. 

 I account for the absence of a passive H in the
              Near Future in the following manner. I assert that
              the H contributed by the passive suffix is
              underlyingly floating and is only expressed when it
              comes to be linked to its sponsoring vowel via a rule
              of \regel{Passive H
              Assignment}in those constructions in which
              it is operative. In the Near Future, the rule is not
              operative, and so the passive H remains floating
              without being expressed on the verb form.

 The circumstances under which \regel{Passive H
              Assignment}applies are complex, and will be
              described in more detail in § \sectref{sec:sP2aPassives} and § \sectref{sec:sP2aOtherTenses} .



\subsubsection{Pattern 1a: Other Verbal
              Contexts}\label{sec:sP1aOtherTenses}

I have shown that the following properties
              characterize the Near Future: (i) the most
              morphologically simple forms realize a H on the
              initial stem mora in /H/ stems and surface all L in
              /Ø/ stems, (ii) the lexical contrast is neutralized
              in forms with an object prefix, where the object
              prefix H is realized \textit{in situ}and the
              lexical H fails to surface in /H/ stems, (iii) the
              verb’s position within its phrase does not affect
              stem tone, (iv) the lexical H does not surface in
              forms with 1 \textsuperscript{st}and 2 \textsuperscript{nd}person
              subjects, and (v) the passive suffix does not surface
              H. The tenses below exhibit these properties as
              well.

 
\ea\label{ex:xP1aTenses} 
Other Pattern 1a Verbal Contexts \footnote{\label{fn:nInfinitivesP1a} The Infinitive has not been tested for each of
                  the properties enumerated above, but the
                  available data are sufficient to uniquely
                  identify it as patterning with the others. 


}%



\begin{tabular}{llll}  
  a.  &   Near Future Negative
                     \footnote{\label{fn:nNegativeNearFut} The Near Future Negative only patterns
                      with its affirmative counterpart when it is
                      not marked with the negative prefix \vernacular{
                      shi-}, in which case the Near
                      Future Negative selects the Pattern described
                      in \sectref{sec:sPattern2} .


}%
 &   
                       \vernacular{SP-la[ROOT-a]
                      tá(awe)}  &  \\
b.  &   Perfect  &   
                       \vernacular{
                      SP-Ø[ROOT-ile]}  &  \\
c.  &   Perfect Negative  &   
                       \vernacular{SP-Ø[ROOT-ile]
                      tá(awe)}  &  \\
d.  &   Infinitive  &   
                       \vernacular{
                      khu[ROOT-a]}  &  \\
\end{tabular}
%\caption{\nocaption}
    
\z

 The following displays demonstrate that for each
              of the constructions in \REF{ex:xP1aTenses} , /H/ verbs realize a H on the stem
              initial mora as in the Near Future. The negative
              constructions have the added complication \regel{H Tone
              Anticipation}spreads the H of the negative
              particular \vernacular{
              tá(awe)}onto all moras beyond the initial
              mora of the stem.

 
\ea\label{ex:xP1aHStems} 
Morphologically Simple /H/ Stems \footnote{\label{fn:nP1aGlosses} The examples included in the current section
                  use \vernacular{
                  -khálak-} \gloss{‘cut’}and \vernacular{
                  -lakhuul-} \gloss{‘release’}as
                  representative of /H/ and /Ø/ verbal roots,
                  respectively. The basic gloss for the tenses
                  discussed in this section is the following: Near
                  Future Negative - \gloss{‘s/he will
                  not...’}; Perfect - \gloss{‘s/he
                  has...-ed’}; Perfect Negative - \gloss{‘s/he has
                  not...-ed’}; Infinitive - \gloss{‘to...’}.


}%



\begin{tabular}{llllll}  
    &   Subj  &   Tns  &   Stem  &   Neg  &  \\
Near Fut Neg  &   
                       \vernacular{a-}  &   
                       \vernacular{la}  &   
                       \vernacular{
                      \ob [khá{\downstep}láká]\cb }  &   
                       \vernacular{tá(awe)
                      }  &  \\
Perfect  &   
                       \vernacular{aa-} \footnote{\label{fn:nSubjMarkingPerf} 3 \textsuperscript{rd}sg
                        subjects may alternatively select \vernacular{
                        u-}as the subject prefix with no
                        apparent effect on meaning or tone. In
                        addition, the vowel of the subject prefix
                        is lengthened in the Perfect and Perfect
                        Negative (but not other constructions which
                        involve the perfective suffix). I assume
                        that the Perfect contributes a floating
                        mora which lengthens the subject
                        prefix.


}%
 &   Ø  &   
                       \vernacular{
                      \ob [khálaachɛ]\cb }  &     &  \\
Perf Neg  &   
                       \vernacular{aa-}  &   Ø  &   
                       \vernacular{
                      \ob [khá{\downstep}lááchɛ́]\cb }  &   
                       \vernacular{
                      tá(awe)}  &  \\
Infinitive [JI]  &   \multicolumn{2}{l}{
                       \vernacular{khu} } &   
                       \vernacular{
                      \ob [khálaka]\cb }  &     &  \\
\end{tabular}
%\caption{\nocaption}
    
\z

 /Ø/ verbs surface all L in the most
              morphologically simple forms, except in the negative
              constructions. In negative constructions, the H of
              the \vernacular{
              tá(awe)}spreads left via \regel{H Tone
              Anticipation}far into the verb. \footnote{\label{fn:nHTAPurposelyVague} The prose description of how far the negative H
                extends into the verb is purposely vague. The left
                edge of the H span is not clear in my corpus. 


}%


 
\ea\label{ex:xP1aØStems} 
Morphologically Simple /Ø/
                Stems 


\begin{tabular}{llllll}  
    &   Subj  &   Tns  &   Stem  &   Neg  &  \\
Near Fut Neg  &   
                       \vernacular{a-}  &   
                       \vernacular{lá}  &   
                       \vernacular{
                      \ob [lákhúúlá]\cb }  &   
                       \vernacular{
                      tá(awe)}  &  \\
Perfect  &   
                       \vernacular{aa-}  &   Ø  &   
                       \vernacular{
                      \ob [lakhuuli]\cb }  &     &  \\
Perf Neg  &   
                       \vernacular{aa-}  &   Ø  &   
                       \vernacular{
                      \ob [lákhúúlí]\cb }  &   
                       \vernacular{
                      tá(awe)}  &  \\
Infinitive [JI]  &   \multicolumn{2}{l}{
                       \vernacular{khu} } &   
                       \vernacular{
                      \ob [lakhuula]\cb }  &     &  \\
\end{tabular}
%\caption{\nocaption}
    
\z

 As in the Near Future, the tenses in \REF{ex:xP1aTenses} also neutralize the lexical contrast in
              forms with an object prefix. \footnote{\label{fn:nNoInfinitiveData} No recordings of Infinitives with object
                prefixes were collected. However, several instances
                of this construction were encountered during the
                course of my interviews with both primary
                consultants, and the tonal properties are
                consistent with those of Near Future forms with
                object prefixes. 


}%


 
\ea\label{ex:xP1aOPHStems} 
/H/ Stems with an Object
                Prefix 


\begin{tabular}{lllllll}  
    &   Subj  &   Tns  &   Obj  &   Stem  &   Neg  &  \\
Near Fut Neg  &   
                       \vernacular{a-}  &   
                       \vernacular{la}  &   
                       \vernacular{\ob mú}  &   
                       \vernacular{
                      [{\downstep}kháláká]\cb }  &   
                       \vernacular{
                      tá(awe)}  &  \\
Perfect  &   
                       \vernacular{aa-}  &   Ø  &   
                       \vernacular{\ob mú}  &   
                       \vernacular{
                      [khalaachɛ]\cb }  &     &  \\
Perf Neg  &   
                       \vernacular{aa-}  &   Ø  &   
                       \vernacular{\ob mú}  &   
                       \vernacular{
                      [{\downstep}khálááchɛ́]\cb }  &   
                       \vernacular{
                      tá(awe)}  &  \\
\end{tabular}
%\caption{\nocaption}
    
\z

 
\ea\label{ex:xP1aOPØStems} 
/Ø/ Stems with an Object
                Prefix 


\begin{tabular}{lllllll}  
    &   Subj  &   Tns  &   Obj  &   Stem  &   Neg  &  \\
Near Fut Neg  &   
                       \vernacular{a-}  &   
                       \vernacular{la}  &   
                       \vernacular{\ob mú}  &   
                       \vernacular{
                      [{\downstep}lákhúúlá]\cb }  &   
                       \vernacular{
                      tá(awe)}  &  \\
Perfect  &   
                       \vernacular{aa-}  &   Ø  &   
                       \vernacular{\ob mú}  &   
                       \vernacular{
                      [lakhuuli]\cb }  &     &  \\
Perf Neg  &   
                       \vernacular{aa-}  &   Ø  &   
                       \vernacular{\ob mú}  &   
                       \vernacular{
                      [{\downstep}lákhúúlí]\cb }  &   
                       \vernacular{
                      tá(awe)}  &  \\
\end{tabular}
%\caption{\nocaption}
    
\z

 As in the Near Future, the stem tone properties
              of the Near Future Negative, the Perfect, and the
              Perfect Negative are the same phrase-medially as they
              are pre-pausally if one abstracts away from the
              regularly applying process of \regel{H Tone
              Anticipation}. Examples of both /H/ and /Ø/
              stems followed by a H-toned noun, \vernacular{musáatsa} \gloss{‘man’}or \vernacular{mú{\downstep}yáyi} \gloss{‘boy’}, and a
              toneless noun, \vernacular{muundu} \gloss{
              ‘person/somebody’}are provided below. When
              a H-toned object follows the verb form, the H of the
              object spreads leftward far into the verb.

 
\ea\label{ex:xP1aPhraseMed} 
Tenses Like the Near Future Phrase
                Medially 


\begin{tabular}{llll}  
  
                       \textbf{Near Fut Neg}  &   
                    /H/  &   
                       \vernacular{
                      a-la\ob [khá{\downstep}láká]\cb  músá{\downstep}átsá
                      tá(awe)}  &  \\

                       \vernacular{
                      a-la\ob [khá{\downstep}láká]\cb  múúndú
                      tá(awe)}  &  \\
  &     &  \\

                    /Ø/  &   
                       \vernacular{
                      a-lá\ob [lákhúúlá]\cb  músá{\downstep}átsá
                      tá(awe)}  &  \\

                       \vernacular{
                      a-lá\ob [lákhúúlá]\cb  múúndú
                      tá(awe)}  &  \\
  &     &     &  \\

                       \textbf{Perfect [SB]}  &   
                    /H/  &   
                       \vernacular{
                      aa\ob [khá{\downstep}lááchɛ́]\cb  músáatsa}  &  \\

                       \vernacular{aa\ob [khálaachɛ]\cb 
                      muundu}  &  \\
  &     &  \\

                    /Ø/  &   
                       \vernacular{
                      aa\ob [lákhúúlí]\cb  músáatsa}  &  \\

                       \vernacular{aa\ob [lakhuuli]\cb 
                      muundu}  &  \\
  &     &     &  \\

                       \textbf{Perf Neg [SB]}  &   
                    /H/  &   
                       \vernacular{
                      aa\ob [khá{\downstep}lááchɛ́]\cb  mú{\downstep}yá{\downstep}yí
                      tá}  &  \\

                       \vernacular{
                      aa\ob [khá{\downstep}lááchɛ́]\cb  múúndú tá}  &  \\
  &     &  \\

                    /Ø/  &   
                       \vernacular{
                      aa\ob [lákhúúlí]\cb  mú{\downstep}yá{\downstep}yí tá}  &  \\

                       \vernacular{
                      aa\ob [lákhúúlí]\cb  múúndú tá}  &  \\
\end{tabular}
%\caption{\nocaption}
    
\z

 The lexical H fails to surface in /H/ stems in
              the Near Future Negative, the Perfective, and the
              Perfective Negative, when the subject is 1 \textsuperscript{st}or 2 \textsuperscript{nd}person as in
              the Near Future. Forms with the 2 \textsuperscript{nd}sg subject
              provided below for each of these three tenses.

 
\ea\label{ex:xP1aHStems2Sg} 
/H/ Stems with 2 \textsuperscript{nd}sg Subjects \vernacular{
                ‘bring'}


\begin{tabular}{llllll}  
    &   Subj  &   Tns  &   Stem  &   Neg  &  \\
Near Fut Neg  &   
                       \vernacular{u-}  &   
                       \vernacular{la}  &   
                       \vernacular{
                      \ob [leera]\cb }  &   
                       \vernacular{
                      tá(awe)}  &  \\
Perf  &   
                       \vernacular{uu-}  &   Ø  &   
                       \vernacular{
                      \ob [leera]\cb }  &     &  \\
Perf Neg  &   
                       \vernacular{uu-}  &   Ø  &   
                       \vernacular{
                      \ob [leera]\cb }  &   
                       \vernacular{
                      tá(awe)}  &  \\
\end{tabular}
%\caption{\nocaption}
    
\z

 In \REF{ex:xP1aHStems2Sg} , \regel{H Tone
              Anticipation}appears unexpectedly not to
              apply in the negative constructions, though this
              appears to be an artifact of the short stem size and
              the fuzziness of the leftward extent of H spans in
              Idakho. In Perfect Negative forms with longer stem
              sizes, \regel{H Tone
              Anticipation}more clearly spreads
              post-verbal Hs some distance into the verbs affected
              by subject-induced lowering of the root H, as in \vernacular{uu[βohólóólɛ́]
              mú{\downstep}yáyi} \gloss{‘you have
              untied the boy’}.

 The display below demonstrates that, as in the
              Near Future, the passive suffix does not contribute a
              H in the Near Future Negative, the Perfective, and
              the Perfective Negative. Though the passive suffix
              surfaces H in the negative constructions, it does so
              only as part of a H span generated by \regel{H Tone
              Anticipation}.

 
\ea\label{ex:xP1aPassive} 
/H/ \& /Ø/ Stems with the
                Passive Suffix [SB] 


\begin{tabular}{lll}  
  
                       \textbf{Near Fut Neg}  &   
                       \vernacular{
                      a-la\ob [khá{\downstep}lák-w-á]\cb  tá}  &  \\

                       \vernacular{
                      a-lá\ob [lákhúúl-w-á]\cb  tá}  &  \\
  &     &  \\

                       \textbf{Perfect}  &   
                       \vernacular{
                      uu\ob [khálaach-u-i]\cb } \footnote{\label{fn:nSPinPerf} Recall that 3 \textsuperscript{rd}sg
                        subjects can be marked either by \vernacular{
                        u-}or \vernacular{a-}.
                        See also fn. \fnref{fn:nSubjMarkingPerf} .


}%
 &  \\

                       \vernacular{
                      uu\ob [lakhuul-u-i]\cb }  &  \\
  &     &  \\

                       \textbf{Perf Neg}  &   
                       \vernacular{
                      aa\ob [khá{\downstep}láách-w-í]\cb  tá}  &  \\

                       \vernacular{
                      aa\ob [lákhúúl-w-í]\cb  tá}  &  \\
\end{tabular}
%\caption{\nocaption}
    
\z

 Finally, note that the Near Future Negative
              demonstrates that the post-verbal negative element \vernacular{
              tá(awe)}will also be downstepped relative
              to the lexical H in /H/ monosyllabic stems as in,
              e.g., \vernacular{a-la[rá]
              {\downstep}tá} \gloss{‘he will not
              bury’}and \vernacular{a-la[ng’wá]
              {\downstep}tá} \gloss{‘he will not
              drink’}. Both examples derive from
              intermediate representations in which the verb stem
              is comprised of two moras, one from the root, and one
              from the FV: \vernacular{a-la[ráa]
              {\downstep}tá} \gloss{‘he will not
              bury’}and \vernacular{a-la[ng’úa]
              {\downstep}tá} \gloss{‘he will not
              drink’}. The L that intervenes between the
              root H and the negative H, delinked through the
              process of \regel{Non-Final
              Shortening}, triggers downstep.



\subsection{Pattern 1b: Immediate Past}\label{sec:sPattern1b}

Pattern 1b has the same basic properties as Pattern
            1a, with the difference that verbal contexts comprising
            Pattern 1b are all marked with the H-toned \vernacular{
            ákha-}tense prefix. Data from the Immediate
            Past illustrate the properties of Pattern 1b.

 In \REF{ex:xImmPastCH} below, we see that /H/ verbs surface with
            a H on the word-initial syllable, followed by a
            downstepped H which spans from the pre-stem syllable
            through the initial mora of the stem.

 
\ea\label{ex:xImmPastCH} 
Immediate Past C-Initial /H/ \gloss{‘s/he
              just...’}


\begin{tabular}{lllll}  
  Subj  &   Tns  &   Stem  &   Gloss  &  \\

                     \vernacular{y-}  &   
                     \vernacular{á{\downstep}khá}  &   
                     \vernacular{
                    \ob [khwáa]\cb }  &   
                     \gloss{‘paid dowry’}  &  \\

                     \vernacular{y-}  &   
                     \vernacular{á{\downstep}khá}  &   
                     \vernacular{
                    \ob [βéka]\cb }  &   
                     \gloss{‘shaved’}  &  \\

                     \vernacular{y-}  &   
                     \vernacular{á{\downstep}khá}  &   
                     \vernacular{
                    \ob [téekha]\cb }  &   
                     \gloss{‘cooked’}  &  \\

                     \vernacular{y-}  &   
                     \vernacular{á{\downstep}khá}  &   
                     \vernacular{
                    \ob [khálaka]\cb }  &   
                     \gloss{‘cut’}  &  \\

                     \vernacular{y-}  &   
                     \vernacular{á{\downstep}khá}  &   
                     \vernacular{
                    \ob [kálaanga]\cb }  &   
                     \gloss{‘fried’}  &  \\

                     \vernacular{y-}  &   
                     \vernacular{á{\downstep}khá}  &   
                     \vernacular{
                    \ob [sáanditsa]\cb }  &   
                     \gloss{‘thanked’}  &  \\

                     \vernacular{y-}  &   
                     \vernacular{á{\downstep}khá}  &   
                     \vernacular{
                    \ob [βóyong’ana]\cb }  &   
                     \gloss{‘went
                    around’}  &  \\
\end{tabular}
%\caption{\nocaption}
    
\z

 In /Ø/ verbs, only the word-initial syllable is H,
            while the remainder of the verb surfaces L. This is
            shown in \REF{ex:xImmPastCØ} .

 
\ea\label{ex:xImmPastCØ} 
Immediate Past C-Initial /Ø/ \gloss{‘s/he
              just...’}


\begin{tabular}{lllll}  
  Subj  &   Tns  &   Stem  &   Gloss  &  \\

                     \vernacular{y-}  &   
                     \vernacular{ákha}  &   
                     \vernacular{
                    \ob [kwaa]\cb }  &   
                     \gloss{‘fell’}  &  \\

                     \vernacular{y-}  &   
                     \vernacular{ákha}  &   
                     \vernacular{
                    \ob [lekha]\cb }  &   
                     \gloss{‘left’}  &  \\

                     \vernacular{y-}  &   
                     \vernacular{ákha}  &   
                     \vernacular{
                    \ob [reeβa]\cb }  &   
                     \gloss{‘asked’}  &  \\

                     \vernacular{y-}  &   
                     \vernacular{ákha}  &   
                     \vernacular{
                    \ob [kulikha]\cb }  &   
                   \gloss{
                  ‘named’}[SB] &  \\

                     \vernacular{y-}  &   
                     \vernacular{ákha}  &   
                     \vernacular{
                    \ob [lakhuula]\cb }  &   
                     \gloss{‘released’}  &  \\

                     \vernacular{y-}  &   
                     \vernacular{ákha}  &   
                     \vernacular{
                    \ob [seeβula]\cb }  &   
                     \gloss{‘said
                    goodbye’}  &  \\

                     \vernacular{y-}  &   
                     \vernacular{ákha}  &   
                     \vernacular{
                    \ob [kalushitsa]\cb }  &   
                   \gloss{
                  ‘returned’}[SB] &  \\
\end{tabular}
%\caption{\nocaption}
    
\z

 I analyze the leftward spreading of the lexical H
            onto the syllable preceding the tense prefix H as an
            instance of a rule common within Bantu languages, \regel{Plateau}, which
            renders HØH (or HLH) sequences as HHH ( \citealt{rKisseberthOdden2003} ). In Idakho, the latter two syllables in
            such sequences are downstepped relative to the
            underlying H of the tense prefix. I formulate \regel{Plateau}in \REF{ex:xPlateau} . \footnote{\label{fn:nPlateausVariable} This rule applies optionally and is more
              frequently observed in the speech of JI than that of
              SB. Furthermore, Plateau is more likely to apply when
              the toneless mora belongs to the same syllable as the
              second H. It may be the case that there are specific
              constructions in which \regel{Plateau}fails to
              apply, such as the Conditional Negative.


}%


 
\ea\label{ex:xPlateau} 
 \regel{Plateau} 

%\includegraphics[width=\textwidth]{InkScape%20Images/Rules/Plateau.pdf}

\z

 The rule as formulated above has three features
            which are not immediately motivated by the data above.
            In particular, the formulation allows for \regel{Plateau}to apply
            across multiple moras (and syllables) and for leftward
            spreading of the H across moras bearing a L.
            Furthermore, the rule is restricted to applying only
            within the grammatical word (the phonological word less
            enclitics). These features will be justified in Pattern
            5a (§ \sectref{sec:sPattern5a} ), Pattern 2b
            (§ \sectref{sec:sPattern2b} ), and
            Pattern 2a (§ \sectref{sec:sPattern2a} ),
            respectively.

 There are several reasons to analyze \regel{Plateau}as applying
            very late. Among the many tonal rules I posit, none are
            crucially required to apply after \regel{Plateau}. Indeed, the
            opposite is required of several of them. The
            observation that resultant surface H \vernacular{{\downstep}}H sequences
            are not reduced to HØ by \regel{Meeussen’s
            Rule}suggests that \regel{Plateau}minimally
            follows \regel{Meeussen’s Rule}.

 What’s more, the observation that \regel{Plateau}generates
            downstep structures may indicate that \regel{Plateau}not only can
            spread across Ls, but must, if downstep is to be
            analyzed as resulting from a floating L between two Hs.
            A late rule of \regel{Default L Insertion}is
            often regarded as necessary merely as a matter of
            phonetic implementation (e.g., \citealt{rYip2002} ),
            though \citealt{rPasterKim2011} argue that it
            plays an important role in accounting for patterns of
            downstep in the Tiriki variety of Luhya. If \regel{Plateau}applies after
            a process of \regel{Default L Insertion},
            whereby default Ls are assigned to moras remaining
            without tonal specifications near the end of a
            phonological derivation, then the observed downstep in
            the forms in \REF{ex:xImmPastCH} may be attributed to a floating L between
            the H of the tense prefix and the root H.


\subsubsection{Immediate Past with Object
              Prefixes}\label{sec:sP1bObjects}

As in Pattern 1a, the lexical contrast is lost in
              Immediate Past forms with an object prefix. Note that
              the object prefix H, which constitutes the second H
              in a HØH sequence, spreads left by Plateau. 

 
\ea\label{ex:xImmPastCHOP} 
Immediate Past C-Initial /H/ + OP \gloss{‘s/he
                just...him/her’}


\begin{tabular}{llllll}  
  Subj  &   Tns  &   Obj  &   Stem  &   Gloss  &  \\

                       \vernacular{y-}  &   
                       \vernacular{
                      á{\downstep}khá}  &   
                       \vernacular{\ob mú-}  &   
                       \vernacular{[ra]\cb }  &   
                       \gloss{‘buried’}  &  \\

                       \vernacular{y-}  &   
                       \vernacular{
                      á{\downstep}khá}  &   
                       \vernacular{\ob mú-}  &   
                       \vernacular{
                      [βeka]\cb }  &   
                       \gloss{‘shaved’}  &  \\

                       \vernacular{y-}  &   
                       \vernacular{
                      á{\downstep}khá}  &   
                       \vernacular{\ob mú-}  &   
                       \vernacular{
                      [leera]\cb }  &   
                       \gloss{‘brought’}  &  \\

                       \vernacular{y-}  &   
                       \vernacular{
                      á{\downstep}khá}  &   
                       \vernacular{\ob mú-}  &   
                       \vernacular{
                      [khalaka]\cb }  &   
                       \gloss{‘cut’}  &  \\

                       \vernacular{y-}  &   
                       \vernacular{
                      á{\downstep}khá}  &   
                       \vernacular{\ob mú-}  &   
                       \vernacular{
                      [βoolitsa]\cb }  &   
                       \gloss{‘seduced’}  &  \\

                       \vernacular{y-}  &   
                       \vernacular{
                      á{\downstep}khá}  &   
                       \vernacular{\ob mú-}  &   
                       \vernacular{
                      [βoyong’ana]\cb }  &   
                       \gloss{‘went
                      around’}  &  \\
\end{tabular}
%\caption{\nocaption}
    
\z

 
\ea\label{ex:xImmPastCØOP} 
Immediate Past C-Initial /Ø/ + OP \gloss{‘s/he
                just...him/her’}


\begin{tabular}{llllll}  
  Subj  &   Tns  &   Obj  &   Stem  &   Gloss  &  \\

                       \vernacular{y-}  &   
                       \vernacular{
                      á{\downstep}khá}  &   
                       \vernacular{\ob mú-}  &   
                       \vernacular{
                      [tsia]\cb }  &   
                       \gloss{‘went
                      (for)’}  &  \\

                       \vernacular{y-}  &   
                       \vernacular{
                      á{\downstep}khá}  &   
                       \vernacular{\ob mú-}  &   
                       \vernacular{
                      [lekha]\cb }  &   
                       \gloss{‘left’}  &  \\

                       \vernacular{y-}  &   
                       \vernacular{
                      á{\downstep}khá}  &   
                       \vernacular{\ob mú-}  &   
                       \vernacular{
                      [loonda]\cb }  &   
                       \gloss{‘followed’}  &  \\

                       \vernacular{y-}  &   
                       \vernacular{
                      á{\downstep}khá}  &   
                       \vernacular{\ob mú-}  &   
                       \vernacular{
                      [kulikha]\cb }  &   
                     \gloss{
                    ‘named’}[SB] &  \\

                       \vernacular{y-}  &   
                       \vernacular{
                      á{\downstep}khá}  &   
                       \vernacular{\ob mú-}  &   
                       \vernacular{
                      [lakhuula]\cb }  &   
                       \gloss{‘released’}  &  \\

                       \vernacular{y-}  &   
                       \vernacular{
                      á{\downstep}khá}  &   
                       \vernacular{\ob mú-}  &   
                       \vernacular{
                      [kalushitsa]\cb }  &   
                       \gloss{‘returned’
                      [SB]}  &  \\
\end{tabular}
%\caption{\nocaption}
    
\z

 When two object prefixes appear on a single verb
              form, a falling tone surfaces on the pre-stem
              syllable as in Pattern 1a. Here again the lexical
              contrast is neutralized: the stem is toneless in both
              /H/ and /Ø/ stems. 

 
\ea\label{ex:xImmPastCHOPOP1sg} 
Immediate Past C-Initial /H/ + OP +
                OP \textsubscript{1sg} \gloss{‘s/he just...him/her
                for me’}


\begin{tabular}{lllllll}  
  Subj  &   Tns  &   Obj
                     \textsubscript{CV} &   Obj
                     \textsubscript{1sg} &   Stem  &   Gloss  &  \\

                       \vernacular{y-}  &   
                       \vernacular{
                      á{\downstep}khá}  &   
                       \vernacular{\ob mú-}  &   
                       \vernacular{u}  &   
                       \vernacular{
                      [ndeela]\cb }  &   
                       \gloss{‘buried’}  &  \\

                       \vernacular{y-}  &   
                       \vernacular{
                      á{\downstep}khá}  &   
                       \vernacular{\ob mú-}  &   
                       \vernacular{u}  &   
                       \vernacular{
                      [mbechela]\cb }  &   
                       \gloss{‘shaved’}  &  \\

                       \vernacular{y-}  &   
                       \vernacular{
                      á{\downstep}khá}  &   
                       \vernacular{\ob mú-}  &   
                       \vernacular{u}  &   
                       \vernacular{
                      [ndeerela]\cb }  &   
                       \gloss{‘brought’}  &  \\

                       \vernacular{y-}  &   
                       \vernacular{
                      á{\downstep}khá}  &   
                       \vernacular{\ob mú-}  &   
                       \vernacular{u}  &   
                       \vernacular{
                      [khalachila]\cb }  &   
                       \gloss{‘cut’}  &  \\
\end{tabular}
%\caption{\nocaption}
    
\z

 
\ea\label{ex:xImmPastCØOPOP1sg} 
Immediate Past C-Initial /Ø/ + OP +
                OP \textsubscript{1sg} \gloss{‘s/he just...him/her
                for me’}


\begin{tabular}{lllllll}  
  Subj  &   Tns  &   Obj
                     \textsubscript{CV} &   Obj
                     \textsubscript{1sg} &   Stem  &   Gloss  &  \\

                       \vernacular{y-}  &   
                       \vernacular{
                      á{\downstep}khá}  &   
                       \vernacular{\ob mú-}  &   
                       \vernacular{u}  &   
                       \vernacular{
                      [nziila]\cb }  &   
                       \gloss{‘went
                      (for)’}  &  \\

                       \vernacular{y-}  &   
                       \vernacular{
                      á{\downstep}khá}  &   
                       \vernacular{\ob mú-}  &   
                       \vernacular{u}  &   
                       \vernacular{
                      [ndeshela]\cb }  &   
                       \gloss{‘left’}  &  \\

                       \vernacular{y-}  &   
                       \vernacular{
                      á{\downstep}khá}  &   
                       \vernacular{\ob mú-}  &   
                       \vernacular{u}  &   
                       \vernacular{
                      [noondela]\cb }  &   
                       \gloss{‘followed’}  &  \\

                       \vernacular{y-}  &   
                       \vernacular{
                      á{\downstep}khá}  &   
                       \vernacular{\ob mú-}  &   
                       \vernacular{u}  &   
                       \vernacular{
                      [ngulishila]\cb }  &   
                     \gloss{
                    ‘named’}[SB] &  \\
\end{tabular}
%\caption{\nocaption}
    
\z

 The primary properties of Pattern 1b are
              summarized schematically below. 

 
\ea\label{ex:xImmPastSchematic} 
A Schematic Representation of the
                Immediate Past’s Tonal Properties 


\begin{tabular}{lllll}  
    &   \multicolumn{3}{l}{
                       \ul{/H/ Verbs} } &  \\
  &   
                       \textit{Subj + Tns}  &   \multicolumn{2}{l}{
                       \textit{Macrostem} } &  \\
OPsx0  &   
                       \vernacular{
                      y-á{\downstep}khá}  &   
                       \vernacular{\ob }  &   
                       \vernacular{[C
                      }  &  \\
OPsx1  &   
                       \vernacular{
                      y-á{\downstep}khá}  &   
                       \vernacular{\ob C
                      }  &   
                       \vernacular{[C
                      }  &  \\
OPsx2  &   
                       \vernacular{
                      y-á{\downstep}khá}  &   
                       \vernacular{\ob C
                      }  &   
                       \vernacular{[C
                      }  &  \\
  &   \multicolumn{2}{l}{ } &     &  \\
  &   \multicolumn{3}{l}{
                       \textbf{
                        } } &  \\
  &   
                       \textit{Subj + Tns}  &   \multicolumn{2}{l}{
                       \textit{Macrostem} } &  \\
OPsx0  &   
                       \vernacular{
                      y-ákha}  &   
                       \vernacular{\ob }  &   
                     \vernacular{
                    [CVCVCV]}\cb  &  \\
OPsx1  &   
                       \vernacular{
                      y-á{\downstep}khá}  &   
                       \vernacular{\ob C
                      }  &   
                       \vernacular{
                      [CVCVCV]\cb }  &  \\
OPsx2  &   
                       \vernacular{
                      y-á{\downstep}khá}  &   
                       \vernacular{\ob C
                      }  &   
                       \vernacular{
                      [CVCVCV]\cb }  &  \\
\end{tabular}
%\caption{\nocaption}
    
\z



\subsubsection{Immediate Past: Phrase Medially}\label{sec:sP1bPhraseMed}

The data below demonstrate that, as in the Near
              Future, the verb's position within its phrase does
              not impact its tone in the Immediate Past. The verb's
              tonal properties are the same phrase-medially as they
              are pre-pausally, except where the H of H-toned
              objects spreads left onto the verb stem via \regel{H Tone
              Anticipation}.

 
\ea\label{ex:xImmPastPhraseMedial} 
Immediate Past Phrase Medially [SB] \gloss{‘s/he just...(for
                him/her)’}


\begin{tabular}{lllll}  
  
                       %\includegraphics[width=\textwidth]{InkScape%20Images/H%20Stems.svg}
 &   
                       %\includegraphics[width=\textwidth]{InkScape%20Images/No%20OP.svg}
 &   
                       \vernacular{y-á{\downstep}khá\ob [rá]\cb 
                      {\downstep}músáatsa}  &   
                       \gloss{‘buried the
                      man’}  &  \\

                       \vernacular{y-á{\downstep}khá\ob [rá]\cb 
                      muundu}  &   
                       \gloss{‘buried
                      somebody’}  &  \\
  &     &  \\

                       \vernacular{
                      y-á{\downstep}khá\ob [khá{\downstep}láká]\cb 
                      músáatsa}  &   
                       \gloss{‘cut the
                      man’}  &  \\

                       \vernacular{
                      y-á{\downstep}khá\ob [khálaka]\cb  muundu}  &   
                       \gloss{‘cut
                      somebody’}  &  \\
  &     &     &  \\

                       %\includegraphics[width=\textwidth]{InkScape%20Images/One%20OP.svg}
 &   
                       \vernacular{
                      y-á{\downstep}khá\ob mú[{\downstep}réélá]\cb 
                      músáatsa}  &   
                       \gloss{‘buried the
                      man’}  &  \\

                       \vernacular{
                      y-á{\downstep}khá\ob mú[reela]\cb  muundu}  &   
                       \gloss{‘buried
                      somebody’}  &  \\
  &     &  \\

                       \vernacular{
                      y-á{\downstep}khá\ob mú[{\downstep}kháláchílá]\cb 
                      músáatsa}  &   
                       \gloss{‘cut the
                      man’}  &  \\

                       \vernacular{
                      y-á{\downstep}khá\ob mú[khalachila]\cb  muundu}  &   
                       \gloss{‘cut
                      somebody’}  &  \\
  &     &     &  \\

                       %\includegraphics[width=\textwidth]{InkScape%20Images/0%20Stems.svg}
 &   
                       %\includegraphics[width=\textwidth]{InkScape%20Images/No%20OP.svg}
 &   
                       \vernacular{
                      y-á{\downstep}khá\ob [tsyá]\cb  músáatsa}  &   
                       \gloss{‘went for the
                      man’}  &  \\

                       \vernacular{y-ákha\ob [tsya]\cb 
                      muundu}  &   
                       \gloss{‘went for
                      somebody’}  &  \\
  &     &  \\

                       \vernacular{
                      y-á{\downstep}khá\ob [sééβúlá]\cb 
                      músáatsa}  &   
                       \gloss{‘said goodbye to the
                      man’}  &  \\

                       \vernacular{
                      y-ákha\ob [seeβula]\cb  muundu}  &   
                       \gloss{‘said goodbye to
                      somebody’}  &  \\
  &     &     &  \\

                       %\includegraphics[width=\textwidth]{InkScape%20Images/One%20OP.svg}
 &   
                       \vernacular{
                      y-á{\downstep}khá\ob mú[{\downstep}tsíílá]\cb 
                      músáatsa}  &   
                       \gloss{‘went for the
                      man’}  &  \\

                       \vernacular{
                      y-á{\downstep}khá\ob mú[tsiila]\cb  muundu}  &   
                       \gloss{‘went for
                      somebody’}  &  \\
  &     &  \\

                       \vernacular{
                      y-á{\downstep}khá\ob mú[{\downstep}sééβúlílá]\cb 
                      músáatsa}  &   
                       \gloss{‘said goodbye to the
                      man’}  &  \\

                       \vernacular{
                      y-á{\downstep}khá\ob mú[seeβulila]\cb  muundu}  &   
                       \gloss{‘said goodbye to
                      somebody’}  &  \\
\end{tabular}
%\caption{\nocaption}
    
\z



\subsubsection{Immediate Past: Impact of Subject
              Choice}\label{sec:sP1bSubjects}

One difference between the Immediate Past and
              Pattern 1a verbal contexts is that the choice of
              verbal subject does not cause the failure of any
              underlying Hs to surface in the Immediate Past. The
              Hs contributed by the tense prefix and the verbal
              root both surface when the subject of the verb is 1 \textsuperscript{st}or 2 \textsuperscript{nd}person in
              forms lacking an object prefix.

 
\ea\label{ex:xSubjImmPastH} 
Subject Choice in the Immediate
                Past /H/ \gloss{‘...just
                brought’}[SB]


\begin{tabular}{llll}  
    &   Singular  &   Plural  &  \\
1
                     \textsuperscript{
                    st}Person &   
                       \vernacular{
                      n-á{\downstep}khá\ob [léera]\cb }  &   
                       \vernacular{
                      khw-á{\downstep}khá\ob [léera]\cb }  &  \\
2
                     \textsuperscript{
                    nd}Person &   
                       \vernacular{
                      w-á{\downstep}khá\ob [léera]\cb }  &   
                       \vernacular{
                      mw-á{\downstep}khá\ob [léera]\cb }  &  \\
3
                     \textsuperscript{
                    rd}Person &   
                       \vernacular{
                      y-á{\downstep}khá\ob [léera]\cb }  &   
                       \vernacular{
                      β-á{\downstep}khá\ob [léera]\cb }  &  \\
\end{tabular}
%\caption{\nocaption}
    
\z

 
\ea\label{ex:xSubjImmPastØ} 
Subject Choice in the Immediate
                Past /Ø/ \gloss{‘...just
                asked’}[SB]


\begin{tabular}{llll}  
    &   Singular  &   Plural  &  \\
1
                     \textsuperscript{
                    st}Person &   
                       \vernacular{
                      n-ákha\ob [reeβa]\cb }  &   
                       \vernacular{
                      khw-ákha\ob [reeβa]\cb }  &  \\
2
                     \textsuperscript{
                    nd}Person &   
                       \vernacular{
                      w-ákha\ob [reeβa]\cb }  &   
                       \vernacular{
                      mw-ákha\ob [reeβa]\cb }  &  \\
3
                     \textsuperscript{
                    rd}Person &   
                       \vernacular{
                      y-ákha\ob [reeβa]\cb }  &   
                       \vernacular{
                      β-ákha\ob [reeβa]\cb }  &  \\
\end{tabular}
%\caption{\nocaption}
    
\z

 In forms containing an object prefix, the Hs
              contributed by the tense prefix and the object prefix
              are both realized, as in forms with 3 \textsuperscript{rd}person
              subjects. Predictably, the root H is deleted
              following the H of the object prefix via \regel{Meeussen's
              Rule}.

 
\ea\label{ex:xSubjImmPastHOP} 
Subject Choice in the Immediate
                Past /H/ + OP \gloss{‘...just brought
                him/her’}[SB]


\begin{tabular}{llll}  
    &   Singular  &   Plural  &  \\
1
                     \textsuperscript{
                    st}Person &   
                       \vernacular{
                      n-á{\downstep}khá\ob mú[leera]\cb }  &   
                       \vernacular{
                      khw-á{\downstep}khá\ob mú[leera]\cb }  &  \\
2
                     \textsuperscript{
                    nd}Person &   
                       \vernacular{
                      w-á{\downstep}khá\ob mú[leera]\cb }  &   
                       \vernacular{
                      mw-á{\downstep}khá\ob mú[leera]\cb }  &  \\
3
                     \textsuperscript{
                    rd}Person &   
                       \vernacular{
                      y-á{\downstep}khá\ob mú[leera]\cb }  &   
                       \vernacular{
                      β-á{\downstep}khá\ob mú[leera]\cb }  &  \\
\end{tabular}
%\caption{\nocaption}
    
\z

 
\ea\label{ex:xSubjImmPastØOP} 
Subject Choice in the Immediate
                Past /Ø/ + OP \gloss{‘...just asked
                him/her’}[SB]


\begin{tabular}{llll}  
    &   Singular  &   Plural  &  \\
1
                     \textsuperscript{
                    st}Person &   
                       \vernacular{
                      n-á{\downstep}khá\ob mú[reeβa]\cb }  &   
                       \vernacular{
                      khw-á{\downstep}khá\ob mú[reeβa]\cb }  &  \\
2
                     \textsuperscript{
                    nd}Person &   
                       \vernacular{
                      w-á{\downstep}khá\ob mú[reeβa]\cb }  &   
                       \vernacular{
                      mw-á{\downstep}khá\ob mú[reeβa]\cb }  &  \\
3
                     \textsuperscript{
                    rd}Person &   
                       \vernacular{
                      y-á{\downstep}khá\ob mú[reeβa]\cb }  &   
                       \vernacular{
                      β-á{\downstep}khá\ob mú[reeβa]\cb }  &  \\
\end{tabular}
%\caption{\nocaption}
    
\z

 Note that the word initial syllable in the above
              forms is short, despite there seemingly being two
              underlying vowels in the sequence: that of the
              subject prefix and that of the vowel-initial tense
              prefix. Note also that the subject prefix for 3 \textsuperscript{rd}sg subjects in
              the above tenses is \vernacular{y-}rather than \vernacular{a-}as is found
              when the segment immediately following the subject
              marker position is a consonant.

 These two observations support an analysis whereby
              subject prefixes are selected from among one set when
              preceding a consonant and from another when preceding
              a vowel. Prefixes belonging to the latter set are
              composed of non-moraic segments which are, as such,
              incapable of bearing tone. The absence of any subject
              induced tonal alternations in the above tenses, then,
              may be analyzed as resulting from the absence of a L
              associated to the 1 \textsuperscript{st}and 2 \textsuperscript{nd}person
              prefixes which precede vowel-initial morphemes.



\subsubsection{Immediate Past: Passives}\label{sec:sP1bPassives}

Finally, as in the Near Future, the passive suffix
              surfaces L in the Immediate Past. 

 
\ea\label{ex:xImmPastPassives} 
Immediate Past: Passives \gloss{‘s/he was
                just...’}[SB]


\begin{tabular}{lllll}  
  \multicolumn{2}{l}{/H/ Stems } &   \multicolumn{2}{l}{/Ø/ Stems } &  \\

                       \vernacular{
                      y-á{\downstep}khá\ob [khálak-u-a]\cb }  &   
                       \gloss{‘cut’}  &   
                       \vernacular{
                      y-ákha\ob [lakhuul-u-a]\cb }  &   
                       \gloss{‘released’}  &  \\

                       \vernacular{
                      y-á{\downstep}khá\ob [tsúunzuun-u-a]\cb }  &   
                       \gloss{‘sucked’}  &   
                       \vernacular{
                      y-ákha\ob [kalushits-u-a]\cb }  &   
                       \gloss{‘returned’}  &  \\
\end{tabular}
%\caption{\nocaption}
    
\z



\subsubsection{Pattern 1b: Other Verbal
              Contexts}\label{sec:sP1bOtherTenses}

There are several other verbal contexts in which
              the properties of Pattern 1b may be observed; in
              addition to the Immediate Past, the verbal contexts
              listed in \REF{ex:xP1bTenses} exhibit the properties of Pattern 1b. As
              in the Immediate Past, the tense prefix is H-toned \vernacular{
              ákha-}and the selection of subject prefix
              does not affect verb tone.

 
\ea\label{ex:xP1bTenses} 
Other Pattern 1b Verbal
                Contexts 


\begin{tabular}{llll}  
  a.  &   Immediate Past Negative  &   
                       \vernacular{SP-ákha[ROOT-a]
                      tá(awe)}  &  \\
b.  &   Remote Future
                     \footnote{\label{fn:nBurulaIndefFut} There
                      appears to be another tense which is
                      segmentally and semantically similar, but
                      tonally distinct. This similarity led to
                      several time consuming diversions which did
                      not ultimately result in much clarity.
                      According to my best estimation, the Remote
                      Future is marked with the \vernacular{
                      ákha-}tense prefix, the final
                      vowel \vernacular{-ɛ},
                      and the lexical tonal pattern as described in
                      this section. There may be another tense
                      marked with a toneless \vernacular{
                      akha-}tense prefix, the perfective
                      suffix \vernacular{-irɛ},
                      and a Pattern 2 tonal melody (refer to § \sectref{sec:sPattern2a} ).
                      All the data in this section come from JI,
                      who consistently produced these forms with
                      the FV \vernacular{-ɛ}and
                      the lexical tonal pattern.


}%
 &   
                       \vernacular{SP-ákha[ROOT-ɛ]
                      }  &  \\
c.  &   Remote Future Negative  &   
                       \vernacular{SP-ákha[ROOT-ɛ]
                      tá(awe)}  &  \\
\end{tabular}
%\caption{\nocaption}
    
\z

 The following display shows that, in /H/ stems,
              the lexical H is realized on the initial stem mora
              and spreads onto the pre-stem mora in morphologically
              simple forms, just as in the Immediate Future. In
              addition, the H of \vernacular{
              tá(awe)}spreads onto the peninitial
              syllable of the stem via \regel{H Tone
              Anticipation}.

 
\ea\label{ex:xP1bHStems} 
Morphologically Simple /H/ Stems \footnote{\label{fn:nP1bGlosses} The examples included in the current section
                  use \vernacular{
                  -khálak-} \gloss{‘cut’}and \vernacular{
                  -lakhuul-} \gloss{‘release’}as
                  representative of /H/ and /Ø/ verbal roots,
                  respectively. The basic gloss for the tenses
                  discussed in this section is the following:
                  Immediate Past - \gloss{‘s/he
                  just...-ed’}; Immediate Past Negative - \gloss{‘s/he did not
                  just...-ed’}; Indefinite Future - \gloss{‘s/he
                  will...’}; Indefinite Future Negative - \gloss{‘s/he will
                  not...’}.


}%



\begin{tabular}{llllll}  
    &   Subj  &   Tns  &   Stem  &   Neg  &  \\
Imm Pst Neg  &   
                       \vernacular{y-}  &   
                       \vernacular{
                      á{\downstep}khá}  &   
                       \vernacular{
                      \ob [khá{\downstep}láká]\cb }  &   
                       \vernacular{
                      tá(awe)}  &  \\
Rem Fut [JI]  &   
                       \vernacular{y-}  &   
                       \vernacular{
                      á{\downstep}khá}  &   
                       \vernacular{
                      \ob [khálachɛ]\cb }  &     &  \\
Rem Fut Neg [JI]  &   
                       \vernacular{y-}  &   
                       \vernacular{
                      á{\downstep}khá}  &   
                       \vernacular{
                      \ob [khá{\downstep}láchɛ́]\cb }  &   
                       \vernacular{
                      táawe}  &  \\
\end{tabular}
%\caption{\nocaption}
    
\z

 In /Ø/ stems, the underlying H of the tense
              prefix surfaces. In negative constructions, the H of \vernacular{
              tá(awe)}predictably spreads left through
              the pre-stem syllable via \regel{H Tone
              Anticipation}.

 
\ea\label{ex:xP1bØStems} 
Morphologically Simple /Ø/
                Stems 


\begin{tabular}{llllll}  
    &   Subj  &   Tns  &   Stem  &   Neg  &  \\
Imm Pst Neg  &   
                       \vernacular{y-}  &   
                       \vernacular{
                      á{\downstep}khá}  &   
                       \vernacular{
                      \ob [lákhúúlá]\cb }  &   
                       \vernacular{
                      tá(awe)}  &  \\
Rem Fut [JI]  &   
                       \vernacular{y-}  &   
                       \vernacular{ákha}  &   
                       \vernacular{
                      \ob [lakhuulɛ]\cb }  &     &  \\
Rem Fut Neg [JI]  &   
                       \vernacular{y-}  &   
                       \vernacular{
                      á{\downstep}khá}  &   
                       \vernacular{
                      \ob [lákhúúlɛ́]\cb }  &   
                       \vernacular{
                      táawe}  &  \\
\end{tabular}
%\caption{\nocaption}
    
\z

 As in the Immediate Past, the verbal contexts
              listed in \REF{ex:xP1bTenses} lose the lexical contrast in forms with
              an object prefix, and the object prefix H spreads
              left by Plateau.

 
\ea\label{ex:xP1bOPHStems} 
/H/ Stems with an Object
                Prefix 


\begin{tabular}{lllllll}  
    &   Subj  &   Tns  &   Obj  &   Stem  &   Neg  &  \\
Imm Pst Neg  &   
                       \vernacular{y-}  &   
                       \vernacular{
                      á{\downstep}khá}  &   
                       \vernacular{\ob mú}  &   
                       \vernacular{
                      [{\downstep}kháláká]\cb }  &   
                       \vernacular{
                      tá(awe)}  &  \\
Rem Fut [JI]  &   
                       \vernacular{y-}  &   
                       \vernacular{
                      á{\downstep}khá}  &   
                       \vernacular{\ob mú}  &   
                       \vernacular{
                      [khalachɛ]\cb }  &     &  \\
Rem Fut Neg [JI]  &   
                       \vernacular{y-}  &   
                       \vernacular{
                      á{\downstep}khá}  &   
                       \vernacular{\ob mú}  &   
                       \vernacular{
                      [{\downstep}kháláchɛ́]\cb }  &   
                       \vernacular{
                      táawe}  &  \\
\end{tabular}
%\caption{\nocaption}
    
\z

 
\ea\label{ex:xP1bOPØStems} 
/Ø/ Stems with an Object
                Prefix 


\begin{tabular}{lllllll}  
    &   Subj  &   Tns  &   Obj  &   Stem  &   Neg  &  \\
Imm Pst Neg  &   
                       \vernacular{y-}  &   
                       \vernacular{
                      á{\downstep}khá}  &   
                       \vernacular{\ob mú}  &   
                       \vernacular{
                      [{\downstep}lákhúúlá]\cb }  &   
                       \vernacular{
                      tá(awe)}  &  \\
Rem Fut [JI]  &   
                       \vernacular{y-}  &   
                       \vernacular{
                      á{\downstep}khá}  &   
                       \vernacular{\ob mú}  &   
                       \vernacular{
                      [lakhuulɛ]\cb }  &     &  \\
Rem Fut Neg [JI]  &   
                       \vernacular{y-}  &   
                       \vernacular{
                      á{\downstep}khá}  &   
                       \vernacular{\ob mú}  &   
                       \vernacular{
                      [{\downstep}lákhúúlɛ́]\cb }  &   
                       \vernacular{
                      táawe}  &  \\
\end{tabular}
%\caption{\nocaption}
    
\z

 Below, it may be observed that the verb's
              position within its phrase does not impact its tone
              in this set of tenses. 

 
\ea\label{ex:xP1bPhraseMed} 
/H/ Stems Phrase
                Medially 


\begin{tabular}{lll}  
  
                       \textbf{Imm Pst Neg}  &   
                       \vernacular{
                      y-á{\downstep}khá\ob [khá{\downstep}láká]\cb  músá{\downstep}átsá
                      tá(awe)}  &  \\

                       \vernacular{
                      y-á{\downstep}khá\ob [khá{\downstep}láká]\cb  múúndú
                      tá(awe)}  &  \\

                       \textbf{Rem Fut [JI]}  &   
                       \vernacular{
                      y-á{\downstep}khá\ob [khá{\downstep}láchɛ́]\cb 
                      músáatsa}  &  \\

                       \vernacular{
                      y-á{\downstep}khá\ob [khálachɛ]\cb  muundu}  &  \\

                       \textbf{Rem Fut Neg
                      [JI]}  &   
                       \vernacular{
                      y-á{\downstep}khá\ob [khá{\downstep}láchɛ́]\cb  muśá{\downstep}átsá
                      táawe}  &  \\

                       \vernacular{
                      y-á{\downstep}khá\ob [khá{\downstep}láchɛ́]\cb  múúndú
                      táawe}  &  \\
\end{tabular}
%\caption{\nocaption}
    
\z

 As in the Immediate Past, the choice of verbal
              subject does not cause the failure of any underlying
              Hs to surface in the Immediate Past Negative. \footnote{\label{fn:nNoIndefFutData} No data is available for verbs in the Indefinite
                Future and Indefinite Future Negative. 


}%


 
\ea\label{ex:xSubjImmPastNegH} 
Subject Choice in the Immediate
                Past Negative /H/ \gloss{‘...did not just
                bring’}[SB]


\begin{tabular}{llll}  
    &   Singular  &   Plural  &  \\
1
                     \textsuperscript{
                    st}Person &   
                       \vernacular{
                      n-á{\downstep}khá\ob [lé{\downstep}érá]\cb  tá}  &   
                       \vernacular{
                      khw-á{\downstep}khá\ob [lé{\downstep}érá]\cb  tá}  &  \\
2
                     \textsuperscript{
                    nd}Person &   
                       \vernacular{
                      w-á{\downstep}khá\ob [lé{\downstep}érá]\cb  tá}  &   
                       \vernacular{
                      mw-á{\downstep}khá\ob [lé{\downstep}érá]\cb  tá}  &  \\
3
                     \textsuperscript{
                    rd}Person &   
                       \vernacular{
                      y-á{\downstep}khá\ob [lé{\downstep}érá]\cb  tá}  &   
                       \vernacular{
                      β-á{\downstep}khá\ob [lé{\downstep}érá]\cb  tá}  &  \\
\end{tabular}
%\caption{\nocaption}
    
\z

 
\ea\label{ex:xSubjImmPastNegHOP} 
Subject Choice in the Immediate
                Past Negative /H/ + OP \gloss{‘...did not just
                bring him/her’}[SB]


\begin{tabular}{llll}  
    &   Singular  &   Plural  &  \\
1
                     \textsuperscript{
                    st}Person &   
                       \vernacular{
                      n-á{\downstep}khá\ob mú[{\downstep}léérá]\cb  tá}  &   
                       \vernacular{
                      khw-á{\downstep}khá\ob mú[{\downstep}léérá]\cb  tá}  &  \\
2
                     \textsuperscript{
                    nd}Person &   
                       \vernacular{
                      w-á{\downstep}khá\ob mú[{\downstep}léérá]\cb  tá}  &   
                       \vernacular{
                      mw-á{\downstep}khá\ob mú[{\downstep}léérá]\cb  tá}  &  \\
3
                     \textsuperscript{
                    rd}Person &   
                       \vernacular{
                      y-á{\downstep}khá\ob mú[{\downstep}léérá]\cb  tá}  &   
                       \vernacular{
                      β-á{\downstep}khá\ob mú[{\downstep}léérá]\cb  tá}  &  \\
\end{tabular}
%\caption{\nocaption}
    
\z

 Finally, as in the Immediate Past, the passive
              suffix surfaces does not contribute a H in Immediate
              Past Negative. It does, however, come to be H due to
              the leftward spreading of the negative H via \regel{H Tone Anticipation} \footnote{\label{fn:nNoDataP1bPassives} Paradigms with passives in the Remote Future and
                Remote Future Negative were elicited from SB,
                though it is not clear at this time if SB’s passive
                data is related to JI’s non-passive data, upon
                which the characterization of the Remote Future’s
                properties is based. One notable unexpected
                property of SB’s productions of Remote Future
                passive forms is that the tense prefix is L, rather
                than H. While the lexical H surfaces on the initial
                stem mora, as in \vernacular{
                y-akha\ob [khálakua]\cb } \gloss{‘s/he will be
                cut’}, /Ø/ verbs surface with a H on the
                second stem mora, as in \vernacular{
                y-akha\ob [lakhúulua]\cb } \gloss{‘s/he will be
                released’}.


}%


 
\ea\label{ex:xP1bPassive} 
/H/ \& /Ø/ Stems with the
                Passive Suffix [SB] 


\begin{tabular}{lll}  
  
                       \textbf{Imm Pst}  &   
                       \vernacular{
                      y-á{\downstep}khá\ob [khálak-u-a]\cb }  &  \\

                       \vernacular{
                      y-ákha\ob [lakhuul-u-a]\cb }  &  \\

                       \textbf{Imm Pst Neg}  &   
                       \vernacular{
                      y-á{\downstep}khá\ob [khá{\downstep}lák-w-á]\cb  tá}  &  \\

                       \vernacular{
                      y-á{\downstep}khá\ob [lákhúúl-w-á]\cb  tá}  &  \\
\end{tabular}
%\caption{\nocaption}
    
\z



\subsection{Summary of Pattern 1}\label{sec:sP1InterimSumm}

The preceding sections have detailed the tonal
            properties of the ‘lexical pattern’, i.e., the pattern
            which emerges when no melodic H is imposed on the verb
            stem. The key features of the lexical pattern are the
            following: (i) /H/ stems surface with the underlying
            root H on the initial stem mora while /Ø/ stems surface
            toneless in morphologically simply forms, (ii) \regel{Meeussen's
            Rule}applies iteratively from right-to-left,
            leaving only the leftmost of a series of adjacent
            underlying H to surface, and (iii) 1 \textsuperscript{st}and 2 \textsuperscript{nd}person subject
            prefixes prevent root and object prefix Hs from being
            realized as long as the subject prefix appears before a
            consonant-initial morpheme.

 Somewhat less remarkable, but noteworthy, properties
            of the lexical pattern are that the verb's position
            within its phrase does not impact its tone and the
            passive suffix does not surface H. 



\section{Pattern 2: The second mora pattern}\label{sec:sPattern2}

While many verbal contexts exhibit the lexical pattern
          described in § \sectref{sec:sPattern1} , others are
          inflected with a melodic H. Pattern 2 groups several
          verbal contexts in which the melodic H targets the second
          mora of the stem. Additionally, tenses which select this
          melody share in common the property that underlying Hs
          regularly fail to surface when initial within the
          macrostem, a unit of structure which includes the verbal
          stem and any object prefixes.


\subsection{Pattern 2a: Subjunctive Negative}\label{sec:sPattern2a}

The present section details the tonal properties of
            Pattern 2a. The description primarily draws from the
            Subjunctive Negative, which is marked by the toneless \vernacular{
            kha-}prefix, the FV \vernacular{-a}, and a
            melodic H which surfaces on the second stem mora in /Ø/
            stems.

 The following displays illustrate how this pattern
            is realized in morphologically simple forms. /H/ stems
            surface with an all L surface pattern in both C- and
            V-initial stems. 

 
\ea\label{ex:xSubjNegCH} 
Subjunctive Negative C-Initial /H/ \gloss{‘let him/her
              not...’}


\begin{tabular}{llllll}  
  Subj  &   Tns  &   Stem  &   Neg  &   Gloss  &  \\

                     \vernacular{a-}  &   
                     \vernacular{kha}  &   
                     \vernacular{
                    \ob [khwa]\cb }  &   
                     \vernacular{
                    tá(awe)}  &   
                     \gloss{‘pay dowry’}  &  \\

                     \vernacular{a-}  &   
                     \vernacular{kha}  &   
                     \vernacular{
                    \ob [βeka]\cb }  &   
                     \vernacular{
                    tá(awe)}  &   
                     \gloss{‘shave’}  &  \\

                     \vernacular{a-}  &   
                     \vernacular{kha}  &   
                     \vernacular{
                    \ob [teekha]\cb }  &   
                     \vernacular{
                    tá(awe)}  &   
                     \gloss{‘cook’}  &  \\

                     \vernacular{a-}  &   
                     \vernacular{kha}  &   
                     \vernacular{
                    \ob [khalaka]\cb }  &   
                     \vernacular{
                    tá(awe)}  &   
                     \gloss{‘cut’}  &  \\

                     \vernacular{a-}  &   
                     \vernacular{kha}  &   
                     \vernacular{
                    \ob [kalaanga]\cb }  &   
                     \vernacular{
                    tá(awe)}  &   
                     \gloss{‘fry’}  &  \\

                     \vernacular{a-}  &   
                     \vernacular{kha}  &   
                     \vernacular{
                    \ob [saanditsa]\cb }  &   
                     \vernacular{
                    tá(awe)}  &   
                     \gloss{‘thank’}  &  \\

                     \vernacular{a-}  &   
                     \vernacular{kha}  &   
                     \vernacular{
                    \ob [βoyong’ana]\cb }  &   
                     \vernacular{
                    tá(awe)}  &   
                     \gloss{‘go around’}  &  \\
\end{tabular}
%\caption{\nocaption}
    
\z

 
\ea\label{ex:xSubjNegVH} 
Subjunctive Negative V-Initial /H/ \gloss{‘let him/her
              not...’}


\begin{tabular}{llllll}  
  Subj  &   Tns  &   Stem  &   Neg  &   Gloss  &  \\

                     \vernacular{a-}  &   
                     \vernacular{khi}  &   
                     \vernacular{\ob [ira]\cb }  &   
                     \vernacular{
                    tá(awe)}  &   
                     \gloss{‘kill’}  &  \\

                     \vernacular{a-}  &   
                     \vernacular{kho}  &   
                     \vernacular{
                    \ob [ononyinya]\cb }  &   
                     \vernacular{
                    tá(awe)}  &   
                     \gloss{‘spoil’}  &  \\

                     \vernacular{a-}  &   
                     \vernacular{kha}  &   
                     \vernacular{
                    \ob [abukhanyinya]\cb }  &   
                     \vernacular{
                    tá(awe)}  &   
                     \gloss{‘separate’}  &  \\
\end{tabular}
%\caption{\nocaption}
    
\z

 In contrast, /Ø/ stems realize a melodic H on the
            second stem mora, \footnote{\label{fn:nAntepenultspreadingForeshadowing} The melodic H undergoes a rightward spreading
              process in very long stems, the details of which are
              provided in § \sectref{sec:sP2aPrepenultimateDoubling} .


}%
\vernacular{
            tá(awe)}particle. The melodic H is realized
            on the sole stem mora in this case. Notice also that
            the H of the negative \vernacular{tá(awe)}element
            is downstepped relative to the verb final melodic Hs of
            monosyllabic and CVCV stems, and, unlike previously
            discussed negative constructions, does not spread onto
            the verb in stems of any size.

 
\ea\label{ex:xSubjNegCØ} 
Subjunctive Negative C-Initial /Ø/ \gloss{‘let him/her
              not...’}


\begin{tabular}{llllll}  
  Subj  &   Tns  &   Stem  &   Neg  &   Gloss  &  \\

                     \vernacular{a-}  &   
                     \vernacular{kha}  &   
                     \vernacular{
                    \ob [kwá]\cb }  &   
                     \vernacular{
                    {\downstep}tá(awe)}  &   
                     \gloss{‘fall’}  &  \\

                     \vernacular{a-}  &   
                     \vernacular{kha}  &   
                     \vernacular{
                    \ob [tsyá]\cb }  &   
                     \vernacular{
                    {\downstep}tá(awe)}  &   
                     \gloss{‘go’}  &  \\

                     \vernacular{a-}  &   
                     \vernacular{kha}  &   
                     \vernacular{
                    \ob [lekhá]\cb }  &   
                     \vernacular{
                    {\downstep}tá(awe)}  &   
                     \gloss{‘leave’}  &  \\

                     \vernacular{a-}  &   
                     \vernacular{kha}  &   
                     \vernacular{
                    \ob [reéβa]\cb }  &   
                     \vernacular{
                    tá(awe)}  &   
                     \gloss{‘ask’}  &  \\

                     \vernacular{a-}  &   
                     \vernacular{kha}  &   
                     \vernacular{
                    \ob [kulíkha]\cb }  &   
                     \vernacular{
                    tá(awe)}  &   
                   \gloss{
                  ‘name’}[SB] &  \\

                     \vernacular{a-}  &   
                     \vernacular{kha}  &   
                     \vernacular{
                    \ob [lakhúula]\cb }  &   
                     \vernacular{
                    tá(awe)}  &   
                     \gloss{‘release’}  &  \\

                     \vernacular{a-}  &   
                     \vernacular{kha}  &   
                     \vernacular{\ob [seéβúla]\cb 
                    }  &   
                     \vernacular{
                    tá(awe)}  &   
                     \gloss{‘say goodbye
                    (to)’}  &  \\

                     \vernacular{a-}  &   
                     \vernacular{kha}  &   
                     \vernacular{
                    \ob [kalúshítsa]\cb }  &   
                     \vernacular{
                    tá(awe)}  &   
                   \gloss{
                  ‘return’}[SB] &  \\
\end{tabular}
%\caption{\nocaption}
    
\z

 The same generalization holds for V-initial stems.
            Note that in vowel-initial stems, the second stem mora
            is always on the second stem syllable, as the first
            vowel represented within square brackets
            morphologically originates in the tense prefix. 

 
\ea\label{ex:xSubjNegVØ} 
Subjunctive Negative V-Initial /Ø/ \gloss{‘let him/her
              not...’}


\begin{tabular}{llllll}  
  Subj  &   Tns  &   Stem  &   Neg  &   Gloss  &  \\

                     \vernacular{a-}  &   
                     \vernacular{khe}  &   
                     \vernacular{
                    \ob [enyá]\cb }  &   
                     \vernacular{
                    {\downstep}tá(awe)}  &   
                     \gloss{‘want’}  &  \\

                     \vernacular{a-}  &   
                     \vernacular{khe}  &   
                     \vernacular{
                    \ob [eyéla]\cb }  &   
                     \vernacular{
                    tá(awe)}  &   
                     \gloss{‘wipe for’}  &  \\

                     \vernacular{a-}  &   
                     \vernacular{khi}  &   
                     \vernacular{
                    \ob [ilúula]\cb }  &   
                     \vernacular{
                    tá(awe)}  &   
                     \gloss{‘winnow’}  &  \\

                     \vernacular{a-}  &   
                     \vernacular{kha}  &   
                     \vernacular{
                    \ob [ambákhana]\cb }  &   
                     \vernacular{
                    tá(awe)}  &   
                     \gloss{‘refuse’}  &  \\
\end{tabular}
%\caption{\nocaption}
    
\z


\subsubsection{Subjunctive Negative with Object
              Prefixes}\label{sec:sP2aObjects}

The root H is able to surface in /H/ stems when an
              object prefix is present. The H of the object prefix,
              however, does not surface. This generalization holds
              in both C- and V-initial /H/ stems. 

 
\ea\label{ex:xSubjNegCHOP} 
Subjunctive Negative C-Initial /H/
                + OP \gloss{‘let him/her
                not...him/her’}


\begin{tabular}{lllllll}  
  Subj  &   Tns  &   Obj  &   Stem  &   Neg  &   Gloss  &  \\

                       \vernacular{a-}  &   
                       \vernacular{kha}  &   
                       \vernacular{\ob mu}  &   
                       \vernacular{
                      [khwá]\cb }  &   
                       \vernacular{
                      {\downstep}tá(awe)}  &   
                       \gloss{‘pay
                      dowry’}  &  \\

                       \vernacular{a-}  &   
                       \vernacular{kha}  &   
                       \vernacular{\ob mu}  &   
                       \vernacular{
                      [βéka]\cb }  &   
                       \vernacular{
                      tá(awe)}  &   
                       \gloss{‘shave’}  &  \\

                       \vernacular{a-}  &   
                       \vernacular{kha}  &   
                       \vernacular{\ob mu}  &   
                       \vernacular{
                      [léera]\cb }  &   
                       \vernacular{
                      tá(awe)}  &   
                       \gloss{‘bring’}  &  \\

                       \vernacular{a-}  &   
                       \vernacular{kha}  &   
                       \vernacular{\ob mu}  &   
                       \vernacular{
                      [khálaka]\cb }  &   
                       \vernacular{
                      tá(awe)}  &   
                       \gloss{‘cut’}  &  \\

                       \vernacular{a-}  &   
                       \vernacular{kha}  &   
                       \vernacular{\ob mu}  &   
                       \vernacular{
                      [βóolitsa]\cb }  &   
                       \vernacular{
                      tá(awe)}  &   
                       \gloss{‘seduce’}  &  \\

                       \vernacular{a-}  &   
                       \vernacular{kha}  &   
                       \vernacular{\ob mu}  &   
                       \vernacular{
                      [βóyong’ana]\cb }  &   
                       \vernacular{
                      tá(awe)}  &   
                       \gloss{‘go
                      around’}  &  \\
\end{tabular}
%\caption{\nocaption}
    
\z

 
\ea\label{ex:xSubjNegVHOP} 
Subjunctive Negative V-Initial /H/
                + OP \gloss{‘let him/her
                not...him/her’}


\begin{tabular}{lllllll}  
  Subj  &   Tns  &   Obj  &   Stem  &   Neg  &   Gloss  &  \\

                       \vernacular{a-}  &   
                       \vernacular{kha}  &   
                       \vernacular{\ob mwi}  &   
                       \vernacular{
                      \ob [íra]\cb }  &   
                       \vernacular{
                      tá(awe)}  &   
                       \gloss{‘kill’}  &  \\

                       \vernacular{a-}  &   
                       \vernacular{kha}  &   
                       \vernacular{\ob mwo}  &   
                       \vernacular{
                      \ob [ónonyinya]\cb }  &   
                       \vernacular{
                      tá(awe)}  &   
                       \gloss{‘spoil’}  &  \\

                       \vernacular{a-}  &   
                       \vernacular{kha}  &   
                       \vernacular{\ob mwa}  &   
                       \vernacular{
                      \ob [ábukhanyinya]\cb }  &   
                       \vernacular{
                      tá(awe)}  &   
                       \gloss{‘separate’}  &  \\
\end{tabular}
%\caption{\nocaption}
    
\z

 The H of the object prefix similarly does not
              surface in /Ø/ stems. The melodic H is realized on
              the second stem mora, just as in forms with no object
              prefix. 

 
\ea\label{ex:xSubjNegCØOP} 
Subjunctive Negative C-Initial /Ø/
                + OP \gloss{‘let him/her
                not...him/her’}


\begin{tabular}{lllllll}  
  Subj  &   Tns  &   Obj  &   Stem  &   Neg  &   Gloss  &  \\

                       \vernacular{a-}  &   
                       \vernacular{kha}  &   
                       \vernacular{\ob mu}  &   
                       \vernacular{
                      [tsyá]\cb }  &   
                       \vernacular{
                      {\downstep}tá(awe)}  &   
                       \gloss{‘go (for)’}  &  \\

                       \vernacular{a-}  &   
                       \vernacular{kha}  &   
                       \vernacular{\ob mu}  &   
                       \vernacular{
                      [lekhá]\cb }  &   
                       \vernacular{
                      {\downstep}tá(awe)}  &   
                       \gloss{‘leave’}  &  \\

                       \vernacular{a-}  &   
                       \vernacular{kha}  &   
                       \vernacular{\ob mu}  &   
                       \vernacular{
                      [loónda]\cb }  &   
                       \vernacular{
                      tá(awe)}  &   
                       \gloss{‘follow’}  &  \\

                       \vernacular{a-}  &   
                       \vernacular{kha}  &   
                       \vernacular{\ob mu}  &   
                       \vernacular{
                      [kulíkha]\cb }  &   
                       \vernacular{tá}  &   
                     \gloss{
                    ‘name’}[SB] &  \\

                       \vernacular{a-}  &   
                       \vernacular{kha}  &   
                       \vernacular{\ob mu}  &   
                       \vernacular{
                      [seéβula]\cb }  &   
                       \vernacular{
                      tá(awe)}  &   
                       \gloss{‘say
                      goodbye’}  &  \\

                       \vernacular{a-}  &   
                       \vernacular{kha}  &   
                       \vernacular{\ob mu}  &   
                       \vernacular{
                      [kalúshitsa]\cb }  &   
                       \vernacular{tá}  &   
                     \gloss{
                    ‘return’}[SB] &  \\
\end{tabular}
%\caption{\nocaption}
    
\z

 
\ea\label{ex:xSubjNegVØOP} 
Subjunctive Negative V-Initial /Ø/
                + OP \gloss{‘let him
                not...her’}


\begin{tabular}{lllllll}  
  Subj  &   Tns  &   Obj  &   Stem  &   Neg  &   Gloss  &  \\

                       \vernacular{a-}  &   
                       \vernacular{kha}  &   
                       \vernacular{\ob mwe}  &   
                       \vernacular{
                      [enyá]\cb }  &   
                       \vernacular{
                      {\downstep}tá(awe)}  &   
                       \gloss{‘want’}  &  \\

                       \vernacular{a-}  &   
                       \vernacular{kha}  &   
                       \vernacular{\ob mwe}  &   
                       \vernacular{
                      [eyéla]\cb }  &   
                       \vernacular{
                      tá(awe)}  &   
                       \gloss{‘wipe for’}  &  \\

                       \vernacular{a-}  &   
                       \vernacular{kha}  &   
                       \vernacular{\ob mwa}  &   
                       \vernacular{
                      [ambákhana]\cb }  &   
                       \vernacular{
                      tá(awe)}  &   
                       \gloss{‘refuse’}  &  \\
\end{tabular}
%\caption{\nocaption}
    
\z

 Hs contributed by 1 \textsuperscript{st}sg object
              prefixes similarly fail to surface with both /H/ and
              /Ø/ stems. The stem tone properties in such forms are
              also the same.

 
\ea\label{ex:xSubjNegCHOP1sg} 
Subjunctive Negative C-Initial /H/
                + OP \textsubscript{1sg} \gloss{‘let him/her
                not...me’}


\begin{tabular}{lllllll}  
  Subj  &   Tns  &   Obj  &   Stem  &   Neg  &   Gloss  &  \\

                       \vernacular{a-}  &   
                       \vernacular{kha}  &   
                       \vernacular{\ob a}  &   
                       \vernacular{
                      [ryá]\cb }  &   
                       \vernacular{{\downstep}tá}  &   
                     \gloss{‘fear’}[SB] \footnote{\label{fn:nRiaVsNdia} Though /r/ typically hardens following 1 \textsuperscript{st}sg
                      subject and object prefixes \vernacular{
                      N-/Ń-}, SB prefers \vernacular{
                      a-kha-a[ryá] {\downstep}tá}in this case, as
                      it would otherwise be homophonous with \vernacular{
                      a-kha-a[ndyá] tá} \gloss{‘let him not eat
                      me’}. These two meanings are
                      homophonous for JI, with hardening of both
                      /l/ and /r/.


}%
 &  \\

                       \vernacular{a-}  &   
                       \vernacular{kha}  &   
                       \vernacular{\ob a}  &   
                       \vernacular{
                      [mbéka]\cb }  &   
                       \vernacular{
                      tá(awe)}  &   
                       \gloss{‘shave’}  &  \\

                       \vernacular{a-}  &   
                       \vernacular{kha}  &   
                       \vernacular{\ob a}  &   
                       \vernacular{
                      [ndéera]\cb }  &   
                       \vernacular{
                      tá(awe)}  &   
                       \gloss{‘bring’}  &  \\

                       \vernacular{a-}  &   
                       \vernacular{kha}  &   
                       \vernacular{\ob a}  &   
                       \vernacular{
                      [khálaka]\cb }  &   
                       \vernacular{
                      tá(awe)}  &   
                       \gloss{‘cut’}  &  \\

                       \vernacular{a-}  &   
                       \vernacular{kha}  &   
                       \vernacular{\ob a}  &   
                       \vernacular{
                      [mbóolitsa]\cb }  &   
                       \vernacular{
                      tá(awe)}  &   
                       \gloss{‘seduce’}  &  \\

                       \vernacular{a-}  &   
                       \vernacular{kha}  &   
                       \vernacular{\ob a}  &   
                       \vernacular{
                      [mbóyong’ana]\cb }  &   
                       \vernacular{
                      tá(awe)}  &   
                       \gloss{‘go
                      around’}  &  \\
\end{tabular}
%\caption{\nocaption}
    
\z

 
\ea\label{ex:xSubjNegCØOP1sg} 
Subjunctive Negative C-Initial /Ø/
                + OP \textsubscript{1sg} \gloss{‘let him/her
                not...me’}


\begin{tabular}{lllllll}  
  Subj  &   Tns  &   Obj  &   Stem  &   Neg  &   Gloss  &  \\

                       \vernacular{a-}  &   
                       \vernacular{kha}  &   
                       \vernacular{\ob a}  &   
                       \vernacular{
                      [syá]\cb }  &   
                       \vernacular{{\downstep}tá}  &   
                     \gloss{
                    ‘grind’}[SB] &  \\

                       \vernacular{a-}  &   
                       \vernacular{kha}  &   
                       \vernacular{\ob a}  &   
                       \vernacular{
                      [ndekhá]\cb }  &   
                       \vernacular{
                      {\downstep}tá(awe)}  &   
                       \gloss{‘leave’}  &  \\

                       \vernacular{a-}  &   
                       \vernacular{kha}  &   
                       \vernacular{\ob a}  &   
                       \vernacular{
                      [noónda]\cb }  &   
                       \vernacular{
                      tá(awe)}  &   
                       \gloss{‘follow’}  &  \\

                       \vernacular{a-}  &   
                       \vernacular{kha}  &   
                       \vernacular{\ob a}  &   
                       \vernacular{
                      [ngulíkha]\cb }  &   
                       \vernacular{tá}  &   
                     \gloss{
                    ‘name’}[SB] &  \\

                       \vernacular{a-}  &   
                       \vernacular{kha}  &   
                       \vernacular{\ob a}  &   
                       \vernacular{
                      [seéβula]\cb }  &   
                       \vernacular{
                      tá(awe)}  &   
                       \gloss{‘say goodbye
                      (to)’}  &  \\

                       \vernacular{a-}  &   
                       \vernacular{kha}  &   
                       \vernacular{\ob a}  &   
                       \vernacular{
                      [ngalúshitsa]\cb }  &   
                       \vernacular{tá}  &   
                     \gloss{
                    ‘return’}[SB] &  \\
\end{tabular}
%\caption{\nocaption}
    
\z

 My corpus of Subjunctive Negative data involving
              both a CV- and a 1 \textsuperscript{st}sg object
              prefix reveal a discrepancy between my two
              consultants which I suspect is rooted in the
              infrequency of constructions with both object
              prefixes, though I did not become aware of the
              discrepancy in time to investigate the matter
              further. To demonstrate the basic properties of this
              pattern, examples are taken from one of the other
              Pattern 2a constructions discussed in § \sectref{sec:sP2aOtherTenses} : the
              Imperative \textsubscript{sg}Negative. In
              this context, the available recordings show agreement
              between both speakers. In particular, the long
              pre-stem syllable surfaces with a rising tone. In /H/
              verbs, the root H fails to surface, and in /Ø/ verbs
              the melodic H surfaces on the first two moras of the
              stem. \footnote{\label{fn:nSubjNegOPOPDiscrepancy} The available data indicate that the study’s
                primary consultants differ in the following ways
                with respect to Subjunctive Negative with CV- and 1 \textsubscript{sg}object
                prefixes. JI realizes a rise on the pre-stem
                syllable. The stem is all L in both /H/ verbs ( \vernacular{
                a-kha\ob mu-ú[mbechela]\cb } \gloss{‘let him/her not shave
                him/her for me’}) and /Ø/ verbs, ( \vernacular{
                a-kha\ob mu-ú[ndeshela]\cb } \gloss{‘let him/her not leave
                him/her for me’}). In contrast, SB
                realizes the pre-stem syllable as L, and produces
                /H/ verbs with a H on the stem initial mora ( \vernacular{
                a-kha\ob mu-u[mbéchela]\cb } \gloss{‘let him/her not shave
                him/her for me’}), and /Ø/ verbs with a
                melodic H on the second stem mora ( \vernacular{
                a-kha\ob mu-u[ndeshéla]\cb } \gloss{‘let him/her not leave
                him/her for me’}).


}%


 
\ea\label{ex:xImpSgNegCHOPOP1sg} 
Imperative \textsubscript{sg}Negative
                C-Initial /H/ + OP + OP \textsubscript{1sg} \gloss{‘do not...him/her for
                me!’}


\begin{tabular}{llllllll}  
  Subj  &   Tns  &   Obj
                     \textsubscript{CV} &   Obj
                     \textsubscript{1sg} &   Stem  &   Neg  &   Gloss  &  \\

                       \vernacular{u-}  &   
                       \vernacular{kha}  &   
                       \vernacular{\ob mu-}  &   
                       \vernacular{ú}  &   
                       \vernacular{
                      [ndeela]\cb }  &   
                       \vernacular{
                      tá(awe)}  &   
                       \gloss{‘bury’}  &  \\

                       \vernacular{u-}  &   
                       \vernacular{kha}  &   
                       \vernacular{\ob mu-}  &   
                       \vernacular{ú}  &   
                       \vernacular{
                      [mbechela]\cb }  &   
                       \vernacular{
                      tá(awe)}  &   
                       \gloss{‘shave’}  &  \\

                       \vernacular{u-}  &   
                       \vernacular{kha}  &   
                       \vernacular{\ob mu-}  &   
                       \vernacular{ú}  &   
                       \vernacular{
                      [ndeerela]\cb }  &   
                       \vernacular{
                      tá(awe)}  &   
                       \gloss{‘bring’}  &  \\

                       \vernacular{u-}  &   
                       \vernacular{kha}  &   
                       \vernacular{\ob mu-}  &   
                       \vernacular{ú}  &   
                       \vernacular{
                      [khalachila]\cb }  &   
                       \vernacular{
                      tá(awe)}  &   
                       \gloss{‘cut’}  &  \\
\end{tabular}
%\caption{\nocaption}
    
\z

 
\ea\label{ex:xImpSgNegCØOPOP1sg} 
Imperative \textsubscript{sg}Negative
                C-Initial /Ø/ + OP + OP \textsubscript{1sg} \gloss{‘do not...him/her for
                me!’}


\begin{tabular}{llllllll}  
  Subj  &   Tns  &   Obj
                     \textsubscript{CV} &   Obj
                     \textsubscript{1sg} &   Stem  &   Neg  &   Gloss  &  \\

                       \vernacular{u-}  &   
                       \vernacular{kha}  &   
                       \vernacular{\ob mu-}  &   
                       \vernacular{ú}  &   
                       \vernacular{
                      [{\downstep}nzííla]\cb }  &   
                       \vernacular{
                      tá(awe)}  &   
                       \gloss{‘go (for)’}  &  \\

                       \vernacular{u-}  &   
                       \vernacular{kha}  &   
                       \vernacular{\ob mu-}  &   
                       \vernacular{ú}  &   
                       \vernacular{
                      [{\downstep}ndéshéla]\cb }  &   
                       \vernacular{
                      tá(awe)}  &   
                     \gloss{
                    ‘leave’}[SB] &  \\

                       \vernacular{u-}  &   
                       \vernacular{kha}  &   
                       \vernacular{\ob mu-}  &   
                       \vernacular{ú}  &   
                       \vernacular{
                      [{\downstep}nóóndela]\cb }  &   
                       \vernacular{
                      tá(awe)}  &   
                       \gloss{‘follow’}  &  \\

                       \vernacular{u-}  &   
                       \vernacular{kha}  &   
                       \vernacular{\ob mu-}  &   
                       \vernacular{ú}  &   
                       \vernacular{
                      [{\downstep}ndákhúulila]\cb }  &   
                       \vernacular{
                      tá(awe)}  &   
                       \gloss{‘release’}  &  \\
\end{tabular}
%\caption{\nocaption}
    
\z

 In \REF{ex:xImpSgNegCØOPOP1sg} above, the /Ø/
              verbs with long initial syllables realize the melodic
              H as a level, rather than rising, tone which is
              downstepped relative to the H of the object prefix.
              This is another instance of \regel{Plateau}. \regel{Plateau}was first
              noted in the discussion of Pattern 1b (§ \sectref{sec:sPattern1b} ). As in
              that context, \regel{Plateau}appears to
              apply variably.

 The available data for Subjunctive Negative forms
              involving both a CV- and a 1 \textsuperscript{st}sg object
              prefix show that, in that context, SB does not have a
              rise on the pre-stem syllable and JI does not have
              the melodic H in /Ø/ stems (refer to fn. \fnref{fn:nSubjNegOPOPDiscrepancy} for exemplary data). For SB, this
              deviation from the basic pattern has the result of
              also allowing the root H to surface in /H/ stems (as
              will be argued momentarily, the root H fails to
              surface in the basic pattern as a result of being
              deleted by \regel{Meeussen's
              Rule}following the object prefix H; with
              the H of the object prefix missing, the root H is
              free to surface).

 The forms presented in \REF{ex:xImpSgNegCHOPOP1sg} and \REF{ex:xImpSgNegCØOPOP1sg} are taken as
              representative of the basic pattern because these
              forms reflect what appears in all of the other
              parallel contexts to be discussed in § \sectref{sec:sP2aOtherTenses} . In
              addition to the infrequency with which forms
              involving two object prefixes are used in daily
              conversation, speaker fatigue may have played a role
              in the divergences noted here. There were
              considerably more paradigms elicited for the
              Subjunctive Negative (which is why it was selected as
              the tense to illustrate this pattern), and these
              double object constructions were elicited last among
              those many paradigms. The task is quite taxing,
              particularly when the elicitation prompts involve
              complicated and infrequently used constructions.

 There are four key observations regarding the
              Subjunctive Negative’s tonal properties of which an
              analysis is offered below: (i) underlying macrostem
              initial Hs fail to surface, (ii) the melodic H, when
              realized, surfaces on the second stem mora, (iii) the
              melodic H is not realized in /H/ stems, and (iv) the
              root H surfaces only in forms with precisely one
              object prefix. These properties are summarized
              schematically in the following display. The position
              of underlying Hs is indicated with a single
              underline, and the melodic H, when it appears, is
              indicated with double underlining. Curly brackets
              mark boundaries of the macrostem. 

 
\ea\label{ex:xSubjNegSchematic} 
A Schematic Representation of the
                Subjunctive Negative’s Tonal
                Properties 


\begin{tabular}{lllll}  
    &   \multicolumn{3}{l}{/H/ Verbs } &  \\
  &   
                       \textit{Subj + Tns}  &   \multicolumn{2}{l}{
                       \textit{Macrostem} } &  \\
OPsx0  &   
                       \vernacular{a-kha}  &   
                       \vernacular{\ob }  &   
                       \vernacular{[C
                      }  &  \\
OPsx1  &   
                       \vernacular{
                      a-kha-}  &   
                       \vernacular{\ob C
                      }  &   
                       \vernacular{[C
                      }  &  \\
OPsx2  &   
                       \vernacular{
                      a-kha-}  &   
                       \vernacular{\ob C
                      }  &   
                       \vernacular{[C
                      }  &  \\
  &   \multicolumn{2}{l}{ } &     &  \\
  &   \multicolumn{3}{l}{
                       \textbf{
                        } } &  \\
  &   
                       \textit{Subj + Tns}  &   \multicolumn{2}{l}{
                       \textit{Macrostem} } &  \\
OPsx0  &   
                       \vernacular{a-kha}  &   
                       \vernacular{\ob }  &   
                     \vernacular{[CV(C)
                    }\cb  &  \\
OPsx1  &   
                       \vernacular{
                      a-kha-}  &   
                       \vernacular{\ob C
                      }  &   
                       \vernacular{[CV(C)
                      }  &  \\
OPsx2  &   
                       \vernacular{
                      a-kha-}  &   
                       \vernacular{\ob C
                      }  &   
                       \vernacular{[{\downstep}CV́(C)
                      }  &  \\
\end{tabular}
%\caption{\nocaption}
    
\z

 In analyzing the properties of the Subjunctive
              Negative, I posit a rule \regel{Default Melodic H
              Assignment}(henceforth \regel{Default MHA}), which
              assigns a floating inflectional H to the second mora
              of the stem. It is formulated as in \REF{ex:xDefaultMHA} .

 
\ea\label{ex:xDefaultMHA} 
 \regel{Default Melodic H
                  Assignment} 

%\includegraphics[width=\textwidth]{InkScape%20Images/Rules/DefaultMHA.pdf}

\z

 While \regel{Default MHA}accounts
              for the tonal properties of simple /Ø/ verb forms
              rather directly, two observations relating to the
              tonal properties of /H/ verbs require further
              elaboration: (i) the root H is not realized in forms
              lacking an object prefix, and (ii) \regel{Default MHA}does not
              assign a melodic H to the second mora of the stem in
              /H/ verbs.

 The failure of the root H to surface when there
              are no object prefixes is analyzed as one instance of
              the broader generalization that /H/s initial within
              the macrostem routinely fail to surface in tonally
              inflected constructions. I posit a rule \regel{Initial Lowering},
              which renders Hs associated to the initial macrostem
              mora as L to account for this observation.

 
\ea\label{ex:xInitialLowering} 
 \regel{Initial
                  Lowering} 

%\includegraphics[width=\textwidth]{InkScape%20Images/Rules/InitialLowering.pdf}

\z

 The fact that \regel{Default MHA}fails to
              assign the melodic H to the second stem mora in /H/
              verbs is accounted for by a condition built into the
              formal statement of the rule. The condition requires
              that the mora preceding the target be toneless. This
              feature of the rule is frequently exploited in
              analyzing the surface tone patterns of the many
              melodies that \regel{Default
              MHA}interacts with.

 That the root H surfaces in forms with one object
              prefix, but not two, is achieved through \regel{Initial Lowering}, \regel{Meeussen's Rule},
              and the crucial ordering relationship which holds
              between them. In forms with one object prefix, \regel{Initial
              Lowering}first lowers the H of the object
              prefix, bleeding \regel{Meeussen’s
              Rule}.

 
\ea\label{ex:xDerivSubjNegHOP} 
 Derivation,
                  /H/ Subj. Neg. + OP: \vernacular{
                  a-kha\ob mu[khálaka]\cb  tá} \gloss{‘let him/her not
                  cut him/her’} 

%\includegraphics[width=\textwidth]{InkScape%20Images/Derivations/DerivSubjNegHOP.pdf}

\z

 When a second object prefix is present, the H of
              the leftmost object prefix is lowered by \regel{Initial Lowering},
              and the H of the rightmost object prefix remains.
              Following the H of the object prefix, the root H is
              now subject to deletion by \regel{Meeussen’s
              Rule}.

 Verb forms with two object prefixes illustrate the
              need to order \regel{Default MHA}before \regel{Meeussen’s Rule}in
              the derivation. In forms with less than two object
              prefixes, the initial mora of the stem is occupied by
              some tone: when no object prefix is present, the
              initial mora is L, and when one object prefix is
              present, the initial mora is H. In both cases, \regel{Default MHA}would
              not apply, regardless of its ordering relationship
              with \regel{Meeussen’s Rule}.
              This is because the rule requires that the mora
              preceding its target be toneless. However, in forms
              with two object prefixes, the initial mora of the
              stem would be toneless at the time \regel{Default MHA}applies,
              were it to be preceded in the derivation by \regel{Meeussen’s Rule}.
              Because \regel{Meeussen’s Rule}does
              not apply first, \regel{Default MHA}is
              blocked from applying by the root H. The tonal
              properties of forms with two object prefixes are
              derived in \REF{ex:xDerivImpSgNegHOPx2} below.

 
\ea\label{ex:xDerivImpSgNegHOPx2} 
 Derivation,
                  /H/ Subj. Neg. + OPx2: \vernacular{
                  u-kha\ob mu-ú[ndeela]\cb  tá} \gloss{‘do not bury
                  him/her for me’} 

%\includegraphics[width=\textwidth]{InkScape%20Images/Derivations/DerivImpSgNegHOPx2.pdf}

\z

 I adopt the analysis described above for its
              simplicity, though there are numerous analytical
              alternatives capable of deriving the same surface
              patterns. One notable alternative involves analyzing
              the failure of macrostem-initial Hs as resulting from
              a rule of \regel{Initial Deletion} \REF{ex:xInitialDeletion} , which would
              delete, rather than lower, macrostem-initial Hs. I
              sketch an analysis that invokes \regel{Initial
              Deletion}such a rule in the paragraphs
              below \REF{ex:xInitialDeletion} .

 
\ea\label{ex:xInitialDeletion} 
 \regel{Initial
                  Deletion} 

%\includegraphics[width=\textwidth]{InkScape%20Images/Rules/InitialDeletion.pdf}

\z

 Recall that in /H/ Subjunctive Negative forms
              lacking an object prefix, e.g., \vernacular{a-kha\ob [khalaka]\cb 
              tá} \gloss{‘let him/her not
              cut’}, the melodic H fails to surface
              despite the fact that the root H does not surface
              either. This may be analyzed under a deletion
              approach by ordering \regel{Initial
              Deletion}after \regel{Default MHA}. The
              root H is therefore present and blocks \regel{Default MHA}at the
              relevant point in the derivation.

 When an object prefix is added to the
              construction, as in \vernacular{a-kha\ob mu[khálaka]\cb 
              tá} \gloss{‘let him/her not
              cut’}, we see that \regel{Initial
              Deletion}must precede \regel{Meeussen’s Rule}. If
              the order were reversed, both the root H and the
              object prefix H would be deleted. The required rules
              and crucial rankings are therefore the following: 1) \regel{Default MHA}, 2) \regel{Initial Deletion},
              and 3) \regel{Meeussen’s
              Rule}.

 In verb forms with two object prefixes, as in \vernacular{
              a-kha\ob mu-ú[khalachila]\cb  tá} \gloss{‘let him/her not cut
              him/her for me’}, \regel{Default MHA}is first
              blocked from applying by the root H. Then, \regel{Initial
              Deletion}deletes the H of the leftmost
              object prefix. Finally, \regel{Meeussen’s
              Rule}deletes the root H immediately
              following the H of the rightmost object prefix.

 While the \regel{Initial
              Deletion}analysis satisfactorially acounts
              for the tonal properties of /H/ Subjunctive Negative
              verb forms, I ultimately reject this analysis in
              favor of the \regel{Initial
              Lowering}approached advocated previously.
              The justification for this choice is presented in § \sectref{sec:sPattern5} , where it
              is shown that a lowering analysis offers a simple
              account for limiting the leftward extent of H spans
              resulting from \regel{H Tone
              Anticipation}in Pattern 5a and 5b
              phrase-medial constructions and the properties of
              short /H/ verbs in Pattern 5b and 5c phrase-final
              constructions.

 We turn now to /Ø/ monosyllabic verbs, which
              realize the melodic H despite consisting of just one
              mora (and not two, as required by \regel{Default MHA}). In
              monosyllabic /Ø/ verbs lacking an object prefix, the
              melodic H appears on the initial, and only, mora of
              the stem, as in \vernacular{a-kha\ob [kwá]\cb 
              {\downstep}tá} \gloss{‘let him not
              fall’}. I argue that this form derives from
              an intermediate representation in which such verb
              forms consist of two moras, one contributed by the
              verb root and the other by the FV. The melodic H is
              assigned to the second stem mora, in this case also
              the final, in the usual way by \regel{Default MHA}. The
              melodic H is subsequently shifted to the left by a
              rule \regel{Final Rise
              Elimination}, formulated in \REF{ex:xFinalRiseElimination} below.
              Finally, the long stem syllable is shortened via \regel{Non-Final
              Shortening}(first introduced in § \sectref{sec:sP1aPhraseMed} in \REF{ex:xNonFinalShortening} ).

 
\ea\label{ex:xFinalRiseElimination} 
 \regel{Final Rise
                  Elimination} 

%\includegraphics[width=\textwidth]{InkScape%20Images/Rules/FinalRiseElimination.pdf}

\z

 The analysis of a /Ø/ monosyllabic stem is
              illustrated in the derivation below. 

 
\ea\label{ex:xDerivSubjNegØShort} 
 Derivation,
                  /Ø/ Subj. Neg.: \vernacular{a-kha\ob [kwá]\cb 
                  {\downstep}tá} \gloss{‘let him/her not
                  fall’} 

%\includegraphics[width=\textwidth]{InkScape%20Images/Derivations/DerivSubjNeg0Short.pdf}

\z

  \regel{Meeussen’s Rule}does
              not delete the H of the negative element \vernacular{
              tá}because the two Hs are not within the
              domain of its application, namely, a single word.

 The ordering relationship which holds between \regel{Meeussen’s Rule}and \regel{Final Rise
              Elimination}is indeterminate in this case
              due to the morphology, despite their potential for
              interaction. In particular, the melodic H could come
              to be adjacent to a pre-stem H via \regel{Final Rise
              Elimination}. Within the context of the
              Subjunctive Negative, Hs are realized on the pre-stem
              mora only in constructions with two object prefixes.
              Such constructions in the Subjunctive Negative are
              only possible with the applicative suffix \vernacular{-il},
              which removes the environment for \regel{Final Rise
              Elimination}.

 While data from the Subjunctive Negative cannot
              inform the ordering relationship that holds between \regel{Final Rise
              Elimination}and \regel{Meeussen’s Rule},
              data from the Conditional (Pattern 5c, § \sectref{sec:sPattern5c} ) can. The
              Conditional also realizes a H on the second stem mora
              in /Ø/ verbs, and subject prefixes, which immediately
              precede the stem, exceptionally surface with a H. In
              monosyllabic stems, the melodic H surfaces as a
              downstepped fall on the final syllable following the
              H of the subject prefix: \vernacular{
              na-á\ob [{\downstep}kúa]\cb } \gloss{‘if s/he
              falls’}.

 Finally, if the melodic H is in place on the
              initial mora of the stem at the point in the
              derivation at which \regel{Initial
              Lowering}applies, one might expect the
              melodic H to be lowered just as root Hs are. Because
              the melodic H is not lowered in this context, I argue
              that \regel{Initial
              Lowering}applies prior to \regel{Final Rise
              Elimination}in the derivation.



\subsubsection{Subjunctive Negative: Phrase
              Medially}\label{sec:sP2aPhraseMed}

The tonal properties of forms in the Subjunctive
              Negative are unaffected by the verb’s position within
              its phrase. Four pairs of /H/ and /Ø/ stems are
              provided below, half with and half without an object
              prefix. For each pair, the first member involves a
              H-toned complement, while the second involves a
              toneless complement. In each case, the stem tonal
              properties are the same as the pre-pausal
              counterparts. \footnote{\label{fn:nVariabilityOfPlateau}  \regel{Plateau}may
                variably apply to spread the H of the negative
                element \vernacular{tá(awe)}and
                the H of H-toned post-verbal words leftward up
                until a preceding H, even when the preceding H is
                associated with the verb stem. Due to the variable
                nature of \regel{Plateau}and the
                challenges introduced by the phonetic effect of
                ‘crescendo’, whereby H spans are produced with
                progressively higher pitch ( \citealt{rAusten1974b} ), such potential H spans are not
                represented in my transcriptions. For instance,
                productions of the phrase \gloss{‘let him/her not cut
                the boy’}may in some cases be best
                transcribed as \vernacular{
                a-kha\ob mu[khálaka]\cb  mú{\downstep}yáyi tá}and in
                other cases as \vernacular{
                a-kha\ob mu[khá{\downstep}láká]\cb  mu{\downstep}yá{\downstep}yí
                tá}.


}%


 
\ea\label{ex:xSubjNegPhraseMedial} 
Subjunctive Negative Phrase
                Medially \gloss{‘let him/her
                not...’} \footnote{\label{fn:nNoApplicative} The object prefix and the full object
                  complement are co-referential in the examples
                  below. This is a marked, but grammatical,
                  construction which emphasizes the identity of the
                  object. 


}%



\begin{tabular}{lllll}  
  
                       %\includegraphics[width=\textwidth]{InkScape%20Images/H%20Stems.svg}
 &   
                       %\includegraphics[width=\textwidth]{InkScape%20Images/No%20OP.svg}
 &   
                       \vernacular{a-kha\ob [ra]\cb 
                      mú{\downstep}yáyi tá(awe)}  &   
                       \gloss{‘bury the
                      boy’}  &  \\

                       \vernacular{a-kha\ob [ra]\cb 
                      muundu tá(awe)}  &   
                       \gloss{‘bury
                      somebody’}  &  \\
  &     &  \\

                       \vernacular{a-kha\ob [khalaka]\cb 
                      mú{\downstep}yáyi tá(awe)}  &   
                       \gloss{‘cut the
                      boy’}  &  \\

                       \vernacular{a-kha\ob [khalaka]\cb 
                      muundu tá(awe)}  &   
                       \gloss{‘cut
                      somebody’}  &  \\
  &     &     &  \\

                       %\includegraphics[width=\textwidth]{InkScape%20Images/One%20OP.svg}
 &   
                       \vernacular{a-kha\ob mu[rá]\cb 
                      {\downstep}mú{\downstep}yáyi tá(awe)}  &   
                       \gloss{‘bury the
                      boy’}  &  \\

                       \vernacular{a-kha\ob mu[rá]\cb 
                      muundu tá(awe)}  &   
                       \gloss{‘bury
                      somebody’}  &  \\
  &     &  \\

                       \vernacular{
                      a-kha\ob mu[khálaka]\cb  mú{\downstep}yáyi
                      tá(awe)}  &   
                       \gloss{‘cut the
                      boy’}  &  \\

                       \vernacular{
                      a-kha\ob mu[khálaka]\cb  muundu
                      tá(awe)}  &   
                       \gloss{‘cut
                      somebody’}  &  \\
  &     &     &  \\

                       %\includegraphics[width=\textwidth]{InkScape%20Images/0%20Stems.svg}
 &   
                       %\includegraphics[width=\textwidth]{InkScape%20Images/No%20OP.svg}
 &   
                       \vernacular{a-kha\ob [tsyá]\cb 
                      {\downstep}mú{\downstep}yáyi tá(awe)}  &   
                       \gloss{‘go for the
                      boy’}  &  \\

                       \vernacular{a-kha\ob [tsyá]\cb 
                      muundu tá(awe)}  &   
                       \gloss{‘go for
                      somebody’}  &  \\
  &     &  \\

                       \vernacular{
                      a-kha\ob [seébula]\cb  mú{\downstep}yáyi
                      tá(awe)}  &   
                       \gloss{‘say goodbye to the
                      boy’}  &  \\

                       \vernacular{
                      a-kha\ob [seébula]\cb  muundu tá(awe)}  &   
                       \gloss{‘say goodbye to
                      somebody’}  &  \\
  &     &     &  \\

                       %\includegraphics[width=\textwidth]{InkScape%20Images/One%20OP.svg}
 &   
                       \vernacular{a-kha\ob mu[tsyá]\cb 
                      {\downstep}mú{\downstep}yáyi tá(awe)}  &   
                       \gloss{‘go for the
                      boy’}  &  \\

                       \vernacular{a-kha\ob mu[tsyá]\cb 
                      muundu tá(awe)}  &   
                       \gloss{‘go for
                      somebody’}  &  \\
  &     &  \\

                       \vernacular{
                      a-kha\ob mu[seéβula]\cb  mú{\downstep}yáyi
                      tá(awe)}  &   
                       \gloss{‘say goodbye to the
                      boy’}  &  \\

                       \vernacular{
                      a-kha\ob mu[seéβula]\cb  muundu
                      tá(awe)}  &   
                       \gloss{‘say goodbye to
                      somebody’}  &  \\
\end{tabular}
%\caption{\nocaption}
    
\z



\subsubsection{Subjunctive Negative: Pre-penultimate
              Doubling}\label{sec:sP2aPrepenultimateDoubling}

Uncommonly long stems reveal that the melodic H in
              /Ø/ stems participate in a binary spreading process
              whereby the melodic H, originating on the second stem
              mora, spreads one mora to the right just in case the
              target mora belongs to a pre-penultimate syllable.
              Consider the additional /Ø/ stems below. 

 
\ea\label{ex:xSubjNegØPrePenultDoubling} 
Subjunctive Negative C-Initial /Ø/ \gloss{‘let him/her
                not...’}


\begin{tabular}{llllll}  
  Subj  &   Tns  &   Stem  &   Neg  &   Gloss  &  \\

                       \vernacular{a-}  &   
                       \vernacular{kha}  &   
                       \vernacular{
                      \ob [reéβáreeba]\cb }  &   
                       \vernacular{
                      {\downstep}tá(awe)}  &   
                       \gloss{‘repeatedly
                      ask’}  &  \\

                       \vernacular{a-}  &   
                       \vernacular{kha}  &   
                       \vernacular{
                      \ob [seβúlúkhanyinya]\cb }  &   
                       \vernacular{
                      tá(awe)}  &   
                       \gloss{‘scatter’}  &  \\
\end{tabular}
%\caption{\nocaption}
    
\z

  \regel{Pre-Penultimate
              Doubling}applies even when the verb is not
              phrase-final, as shown in \REF{ex:xSubjNegØPrePenultDoublingPhraseMedial} .

 
\ea\label{ex:xSubjNegØPrePenultDoublingPhraseMedial} 
Subjunctive Negative Phrase
                Medially C-Initial /Ø/ \gloss{‘let him/her
                not...the boy’ / ‘don't repeatedly ask the boy for
                him!’}


\begin{tabular}{lllllll}  
  Subj  &   Tns + Obj  &   Stem  &   Obj  &   Neg  &   Gloss  &  \\

                       \vernacular{a-}  &   
                       \vernacular{kha}  &   
                       \vernacular{
                      \ob [reéβáreeba]\cb }  &   
                       \vernacular{
                      mú{\downstep}yáyi}  &   
                       \vernacular{
                      tá(awe)}  &   
                       \gloss{‘repeatedly
                      ask’}  &  \\

                       \vernacular{a-}  &   
                       \vernacular{kha}  &   
                       \vernacular{
                      \ob [siínjílitsa]\cb }  &   
                       \vernacular{
                      mú{\downstep}yáyi}  &   
                       \vernacular{
                      tá(awe)}  &   
                       \gloss{‘make
                      stand’}  &  \\

                       \vernacular{u-}  &   
                       \vernacular{
                      kha-mu}  &   
                       \vernacular{
                      \ob [reéβáreeβela]\cb }  &   
                       \vernacular{
                      mú{\downstep}yáyi}  &   
                       \vernacular{tá}  &   
                       \gloss{‘repeatedly
                      ask’} \footnote{\label{fn:nThisIsImpSg} This example is from the Imperative \textsubscript{
                        sg}Negative, a construction marked
                        with the same tonal melody.


}%
 &  \\
\end{tabular}
%\caption{\nocaption}
    
\z

 In addition, phrase-medial forms involving
              shorter stems reveal that \regel{Pre-Penultimate
              Doubling}is sensitive to the melodic H’s
              position within the phrase rather than within the
              verb. That is, it will double the melodic H to the
              penultimate and even final syllables of the verb so
              long as the target precedes the penultimate syllable
              of the phrase.

 
\ea\label{ex:xImpSgNegØPrePenultDoublingPhraseMedial} 
Imperative \textsubscript{sg}Negative
                Phrase Medially C-Initial /Ø/ \gloss{‘do
                not...somebody!’}[SB]


\begin{tabular}{lllllll}  
  Subj  &   Tns + Obj  &   Stem  &   Obj  &   Neg  &   Gloss  &  \\

                       \vernacular{u-}  &   
                       \vernacular{kha}  &   
                       \vernacular{
                      \ob [kulíkhá]\cb }  &   
                       \vernacular{
                      muundu}  &   
                       \vernacular{tá}  &   
                       \gloss{‘repeatedly
                      ask’}  &  \\

                       \vernacular{u-}  &   
                       \vernacular{kha}  &   
                       \vernacular{
                      \ob [seéβúla]\cb }  &   
                       \vernacular{
                      muundu}  &   
                       \vernacular{tá}  &   
                       \gloss{‘make
                      stand’}  &  \\
\end{tabular}
%\caption{\nocaption}
    
\z

 It is noteworthy that \regel{Pre-Penultimate
              Doubling}does not impact the position of
              root or object prefix Hs when they surface, as in
              e.g., \vernacular{
              a-kha-mu[βóyong’ana] tá} \gloss{‘let him/her not cut
              him/her’}. This again motivates reference
              to the morphological origins of tone in tonal
              rules.

 Finally, a melodic H will not double onto the
              second mora of a long antepenultimate syllable, as in
              e.g., \vernacular{
              u-kha-mu-ú[{\downstep}ndákhúulila] tá} \gloss{‘let him/her not release
              him/her for me’}.

  \regel{Pre-Penultimate
              Doubling}is stated formally in \REF{ex:xPrePenultimateDoubling} .

 
\ea\label{ex:xPrePenultimateDoubling} 
 \regel{Pre-Penultimate
                  Doubling} 

%\includegraphics[width=\textwidth]{InkScape%20Images/Rules/PrePenultimateDoubling.pdf}

\z

 There may be further restrictions on the
              application \regel{Pre-Penultimate
              Doubling}whereby the rule is blocked if its
              application would result in a sequence of adjacent
              Hs, e.g., \vernacular{a-kha\ob mu[kulíkha]\cb 
              mú{\downstep}yáyi tá} \gloss{‘let him/her not name
              the boy’}. That is, the process may be
              constrainted by the OCP, though the data in my corpus
              are unclear.



\subsubsection{Subjunctive Negative: Impact of Subject
              Choice}\label{sec:sP2aSubjects}

As the data below demonstrate, subject choice has
              no impact on the stem tonal properties of verbs in
              the Subjunctive Negative. /H/ and /Ø/ stems lacking
              an object prefix are realized the same whether the
              subject is 1 \textsuperscript{st}, 2 \textsuperscript{nd}, or 3 \textsuperscript{rd}person. /H/
              stems surface without the root H, and /Ø/ stems
              realize the melodic H on the second stem mora.

 
\ea\label{ex:xSubjSubjNegH} 
Subject Choice in the Subjunctive
                Negative /H/ \gloss{‘let...not
                bring’}[SB]


\begin{tabular}{llll}  
    &   Singular  &   Plural  &  \\
1
                     \textsuperscript{
                    st}Person &   
                       \vernacular{kha\ob [leera]\cb  tá
                      }  &   
                       \vernacular{khu-kha\ob [leera]\cb 
                      tá}  &  \\
2
                     \textsuperscript{
                    nd}Person &   
                       \vernacular{u-kha\ob [leera]\cb 
                      tá}  &   
                       \vernacular{mu-kha\ob [leera]\cb 
                      tá}  &  \\
3
                     \textsuperscript{
                    rd}Person &   
                       \vernacular{a-kha\ob [leera]\cb 
                      tá}  &   
                       \vernacular{βa-kha\ob [leera]\cb 
                      tá}  &  \\
\end{tabular}
%\caption{\nocaption}
    
\z

 
\ea\label{ex:xSubjSubjNegØ} 
Subject Choice in the Subjunctive
                Negative /Ø/ \gloss{‘let...not
                ask’}[SB]


\begin{tabular}{llll}  
    &   Singular  &   Plural  &  \\
1
                     \textsuperscript{
                    st}Person &   
                       \vernacular{kha\ob [reéβa]\cb 
                      tá}  &   
                       \vernacular{
                      khu-kha\ob [reéβa]\cb  tá}  &  \\
2
                     \textsuperscript{
                    nd}Person &   
                       \vernacular{u-kha\ob [reéβa]\cb 
                      tá}  &   
                       \vernacular{mu-kha\ob [reéβa]\cb 
                      tá}  &  \\
3
                     \textsuperscript{
                    rd}Person &   
                       \vernacular{a-kha\ob [reéβa]\cb 
                      tá}  &   
                       \vernacular{βa-kha\ob [reéβa]\cb 
                      tá}  &  \\
\end{tabular}
%\caption{\nocaption}
    
\z

 That no tonal alternations are triggered by the
              choice of subject in /H/ stems is unsurprising; the
              root H fails to surface even in forms with 3 \textsuperscript{rd}person
              subjects.

 The melodic H surfaces in all /Ø/ verbs regardless
              of the choice of subject. This may be accounted for
              in several ways, but the interpretation afforded by
              the formulation of \regel{L Spread I}provided
              in \REF{ex:xLSpreadI} is that the L of 1 \textsuperscript{st}and 2 \textsuperscript{nd}person
              subjects spreads only through the initial mora of the
              stem, and does so only after \regel{Default MHA}has
              associated the melodic H to the second stem mora; if \regel{Default MHA}were
              ordered after \regel{L Spread I}, it
              would fail to apply because the mora preceding the
              target of \regel{Default MHA}would
              not be toneless, as required by the statement of the
              rule given in \REF{ex:xDefaultMHA} .

 Unfortunately, my corpus does not include
              Subjunctive Negative forms with both an object prefix
              and a 1 \textsuperscript{st}or 2 \textsuperscript{nd}person
              subject. Notewithstanding this gap, I would not
              expect subject-induced tonal alternations in such
              forms either based on data from the Indefinite Future
              (Pattern 5b, § \sectref{sec:sP5bSubjects} ). The
              Indefinite Future is a construction with formally
              similar tonal properties, and Indefinite Future verb
              forms with object prefixes all pattern the same,
              regardless of the verbal subject.



\subsubsection{Subjunctive Negative: Passives}\label{sec:sP2aPassives}

The passive data available for the Subjunctive
              Negative are inappropriate for the current
              section. 

 As shown in \REF{ex:xImpSgNegPassives} , /Ø/ stems are
              realized with one more H than their passive-less
              counterparts, which surfaces on the final syllable
              (cf. \vernacular{u-kha[lakhúula]
              tá} \gloss{‘don't
              release!’}). /H/ verbs surface all L both
              with and without a passive suffix. \footnote{\label{fn:nDifferentKindofSubjunctiveNeg} Passive data was collected (accidentally) for a
                version of the Subjunctive Negative which does not
                include the \vernacular{
                kha-}prefix that appears in the
                Subjunctive Negative forms described in preceding
                sections related to Pattern 2a. Such forms exhibit
                the tonal properties of Pattern 3 (§ \sectref{sec:sPattern3} ) and the
                FV selected by Subjunctives in the affirmative. Cf. \vernacular{a\ob [khalakwí]\cb 
                {\downstep}tá} \gloss{‘let him/her not be
                cut’}.

 This situation parallels that of the two
                versions of the Near Future Negative. When marked
                with the negative \vernacular{
                shi-}prefix, the Near Future Negative
                takes the all L / second mora pattern: /H/, \vernacular{sh-a\ob [khalaka]\cb 
                tá} \gloss{‘s/he will not
                cut’}; /Ø/: \vernacular{sh-a\ob [lakhúula]\cb 
                tá} \gloss{‘s/he will not
                release’}. Contrastively, when the
                negative prefix is absent, the Near Future Negative
                exhibits the same tonal properties of its
                affirmative counterpart, Pattern 1a: /H/, \vernacular{
                a-la\ob [khá{\downstep}láká]\cb  tá} \gloss{‘s/he will not
                cut’}; /Ø/: \vernacular{
                a-lá\ob [lákhúúlá]\cb  tá} \gloss{‘s/he will not
                release’}.


}%


 
\ea\label{ex:xImpSgNegPassives} 
Imperative \textsubscript{sg}Negative:
                Passives \gloss{‘don't
                be...!’}[SB]


\begin{tabular}{lllll}  
  \multicolumn{2}{l}{/H/ Stems } &   \multicolumn{2}{l}{/Ø/ Stems } &  \\

                       \vernacular{
                      u-kha\ob [khalak-w-a] tá}  &   
                       \gloss{‘cut’}  &   
                       \vernacular{
                      u-kha\ob [lakhúul-w-á]\cb  {\downstep}tá}  &   
                       \gloss{‘released’}  &  \\

                       \vernacular{
                      u-kha\ob [tsuunzuun-w-a] tá}  &   
                       \gloss{‘sucked’}  &   
                       \vernacular{
                      u-kha\ob [kalúshits-w-á]\cb  {\downstep}tá}  &   
                       \gloss{‘returned’}  &  \\
\end{tabular}
%\caption{\nocaption}
    
\z

 Recall that the passive suffix does not
              contribute a H in the Near Future (§ \sectref{sec:sP1aPassives} ) and
              other tenses exhibiting the lexical pattern (§ \sectref{sec:sPattern1} ),
              regardless of the tonal class to which the verbal
              root belongs. An important property that Near Future
              passives and Imperative \textsubscript{sg}share in common
              is that /H/ verbs do not realize a melodic H on the
              stem.

 I attribute the H that surfaces on the passive
              suffix in /Ø/ verbs as resulting from the successful
              application of a rule, \regel{Passive H
              Assignment}. As shown in \REF{ex:xPassiveHAssignmentPrelim} ,
              passive Hs, which enter the derivation as floating
              tones are assigned to the penultimate mora of the
              stem under certain conditions.

 
\ea\label{ex:xPassiveHAssignmentPrelim} 
 \regel{Passive H Assignment
                  (Preliminary)} 

%\includegraphics[width=\textwidth]{InkScape%20Images/Rules/PassiveHAssignmentPrelim.pdf}

\z

 The formulation of \regel{Passive H
              Assignment}is preliminary and will be
              modified in § \sectref{sec:sP2aOtherTenses} on the
              basis of data from other Pattern 2a constructions.
              The current formulation is, however, sufficient to
              account for the distribution of passive Hs in the
              Idakho data seen so far. /Ø/ verb forms with passives
              in the Subjunctive Negative meet each of the
              conditions listed in \REF{ex:xPassiveHAssignmentPrelim} , unlike
              /H/ verbs in the Subjunctive Negative and verbs of
              both tonal classes in all Pattern 1 (§ \sectref{sec:sPattern1} )
              constructions.

 Pattern 1 constructions do not realize the passive
              H because they are not tonally inflected and, by
              extension, they do not realize a melodic H on the
              verb form, thus violating the rule’s latter two
              conditions. Although the Subjunctive Negative is a
              context inflected with a melodic H, the passive H is
              not realized in /H/ verbs because such verbs violate
              the condition that the melodic H be associated to the
              stem. 

  \citealt{rHymanKatamba1990a} take a
              similar approach to passive and causative Hs in
              (Lu)Ganda. According to them, these suffixes will
              realize the Hs they contribute only under the
              following three conditions: (i) the verb must contain
              the causative or passive suffix, (ii) the verb must
              appear in a tense with a melodic H, and (iii) the
              verb must appear in a tense with the historically
              perfective suffix \vernacular{
              *-i̧de}.

 The Idakho data require a stronger version of \citealt{rHymanKatamba1990a} ’s
              second criterion for Ganda: not only must the verb
              form appear in a tense with a melodic H, but that H
              must be realized. It is also not crucial in Idakho
              that the verb appear in a tense with the perfective
              suffix.

 In § \sectref{sec:sP2aOtherTenses} , the
              role of the perfective suffix in the distribution of
              passive Hs in Idakho will be revisited in light of
              data from other Pattern 2a constructions, and the
              final formulation of \regel{Passive H
              Assignment}will be given.



\subsubsection{Pattern 2a: Other Verbal
              Contexts}\label{sec:sP2aOtherTenses}

The preceding section gave a comprehensive
              description of the tonal properties of the
              Subjunctive Negative, supplemented with a few
              examples taken from the form-identical Imperative \textsubscript{sg}Negative. In
              particular, it was shown that: (i) a rule of \regel{Initial
              Lowering}lowers all macrostem initial Hs
              (root Hs in morphologically simple forms or the
              leftmost object prefix), (ii) a melodic H is realized
              on the second stem syllable in all /Ø/ stems, (iii)
              the verb's position within its phrase and the choice
              of verbal subject do not impact stem tone, and (iv)
              the passive H is realized on the final syllable in
              /Ø/ stems when the passive suffix is present. The
              tenses below exhibit all of these same
              properties.

 
\ea\label{ex:xP2aTenses} 
Other Pattern 2a Verbal
                Contexts 


\begin{tabular}{llll}  
  a.  &   Imperative
                     \textsubscript{
                    sg}Negative &   
                       \vernacular{u-kha[ROOT-a]
                      tá(awe)}  &  \\
b.  &   Imperative
                     \textsubscript{
                    pl}Negative &   
                       \vernacular{mu-kha[ROOT-i]
                      tá(awe)}  &  \\
c.  &   Near Future Negative  &   
                       \vernacular{shi-SP[ROOT-a]
                      tá(awe)}  &  \\
d.  &   Hodiernal Perfective  &   
                       \vernacular{
                      SP[ROOT-ile]}  &  \\
e.  &   Hodiernal Perfective Negative  &   
                       \vernacular{SP[ROOT-ile]
                      tá(awe)}  &  \\
\end{tabular}
%\caption{\nocaption}
    
\z

 For each of the tenses listed in \REF{ex:xP2aTenses} , /H/ stems fail to realize the root H \REF{ex:xP2aHStems} , and /Ø/ stems take a melodic H on the
              second stem mora \REF{ex:xP2aØStems} in the most morphologically simple
              forms.

 
\ea\label{ex:xP2aHStems} 
Morphologically Simple /H/ Stems \footnote{\label{fn:nP2aGlosses} The examples included in the current section
                  use \vernacular{
                  -khálak-} \gloss{‘cut’}and \vernacular{
                  -lakhuul-} \gloss{‘release’}as
                  representative of /H/ and /Ø/ verbal roots,
                  respectively. The basic gloss for the tenses
                  discussed in this section is the following:
                  Imperative \textsubscript{sg/pl}Negative
                  - \gloss{‘don't...!’};
                  Near Future Negative - \gloss{‘s/he will
                  not...’}; Hodiernal Perfective - \gloss{
                  ‘s/he...\ob PAST\cb ’}; Hodiernal Perfective
                  Negative - \gloss{‘s/he did
                  not...’}.


}%



\begin{tabular}{lllllll}  
    &   Neg  &   Subj  &   Tns  &   Stem  &   Neg  &  \\
Imp
                     \textsubscript{sg}Neg &   
                       \vernacular{Ø}  &   
                       \vernacular{u-}  &   
                       \vernacular{kha}  &   
                       \vernacular{
                      \ob [khalaka]\cb }  &   
                       \vernacular{
                      tá(awe)}  &  \\
Imp
                     \textsubscript{pl}Neg &   
                       \vernacular{Ø}  &   
                       \vernacular{mu-}  &   
                       \vernacular{kha}  &   
                       \vernacular{
                      \ob [khalachi]\cb }  &   
                       \vernacular{
                      tá(awe)}  &  \\
Near Fut Neg [SB]  &   
                       \vernacular{sh-}  &   
                       \vernacular{a}  &   Ø  &   
                       \vernacular{
                      \ob [khalaka]\cb }  &   
                       \vernacular{tá}  &  \\
Hod Perf  &   Ø  &   
                       \vernacular{a}  &   Ø  &   
                       \vernacular{
                      \ob [khalaache]\cb }  &   
                       \vernacular{}  &  \\
Hod Perf Neg  &   Ø  &   
                       \vernacular{a}  &   Ø  &   
                       \vernacular{
                      \ob [khalaache]\cb }  &   
                       \vernacular{
                      tá(awe)}  &  \\
\end{tabular}
%\caption{\nocaption}
    
\z

 
\ea\label{ex:xP2aØStems} 
Morphologically Simple /Ø/
                Stems 


\begin{tabular}{lllllll}  
    &   Neg  &   Subj  &   Tns  &   Stem  &   Neg  &  \\
Imp
                     \textsubscript{sg}Neg &   
                       \vernacular{Ø}  &   
                       \vernacular{u-}  &   
                       \vernacular{kha}  &   
                       \vernacular{
                      \ob [lakhúula]\cb }  &   
                       \vernacular{
                      tá(awe)}  &  \\
Imp
                     \textsubscript{pl}Neg &   
                       \vernacular{Ø}  &   
                       \vernacular{mu-}  &   
                       \vernacular{kha}  &   
                       \vernacular{
                      \ob [lakhúuli]\cb }  &   
                       \vernacular{
                      tá(awe)}  &  \\
Near Fut Neg [SB]  &   
                       \vernacular{sh-}  &   
                       \vernacular{a}  &   Ø  &   
                       \vernacular{
                      \ob [lakhúula]\cb }  &   
                       \vernacular{tá}  &  \\
Hod Perf  &   Ø  &   
                       \vernacular{a}  &   Ø  &   
                       \vernacular{
                      \ob [lakhúuli]\cb }  &   
                       \vernacular{}  &  \\
Hod Perf Neg  &   Ø  &   
                       \vernacular{a}  &   Ø  &   
                       \vernacular{
                      \ob [lakhúuli]\cb }  &   
                       \vernacular{
                      tá(awe)}  &  \\
\end{tabular}
%\caption{\nocaption}
    
\z

 As in the Subjunctive Negative, the H contributed
              by a single object prefix fails to surface, while the
              root H re-emerges in /H/ stems. The melodic H
              continues to be realized on the second stem mora in
              /Ø/ stems. 

 
\ea\label{ex:xP2aOPHStems} 
/H/ Stems with an Object
                Prefix 


\begin{tabular}{llllllll}  
    &   Neg  &   Subj  &   Tns  &   Obj  &   Stem  &   Neg  &  \\
Imp
                     \textsubscript{sg}Neg &   
                       \vernacular{Ø}  &   
                       \vernacular{u-}  &   
                       \vernacular{kha}  &   
                       \vernacular{\ob mu}  &   
                       \vernacular{
                      [khálaka]\cb }  &   
                       \vernacular{
                      tá(awe)}  &  \\
Imp
                     \textsubscript{pl}Neg &   
                       \vernacular{Ø}  &   
                       \vernacular{mu-}  &   
                       \vernacular{kha}  &   
                       \vernacular{\ob mu}  &   
                       \vernacular{
                      [khálachi]\cb }  &   
                       \vernacular{
                      tá(awe)}  &  \\
Near Fut Neg [SB]  &   
                       \vernacular{sh-}  &   
                       \vernacular{a}  &   Ø  &   
                       \vernacular{\ob mu}  &   
                       \vernacular{
                      [khálaka]\cb }  &   
                       \vernacular{tá}  &  \\
Hod Perf  &   Ø  &   
                       \vernacular{a}  &   Ø  &   
                       \vernacular{\ob mu}  &   
                       \vernacular{
                      [khálaache]\cb }  &   
                       \vernacular{}  &  \\
Hod Perf Neg  &   Ø  &   
                       \vernacular{a}  &   Ø  &   
                       \vernacular{\ob mu}  &   
                       \vernacular{
                      [khálaache]\cb }  &   
                       \vernacular{
                      tá(awe)}  &  \\
\end{tabular}
%\caption{\nocaption}
    
\z

 
\ea\label{ex:xP2aOPØStems} 
/Ø/ Stems with an Object
                Prefix 


\begin{tabular}{llllllll}  
    &   Neg  &   Subj  &   Tns  &   Obj  &   Stem  &   Neg  &  \\
Imp
                     \textsubscript{sg}Neg &   
                       \vernacular{Ø}  &   
                       \vernacular{u-}  &   
                       \vernacular{kha}  &   
                       \vernacular{\ob mu}  &   
                       \vernacular{
                      [lakhúula]\cb }  &   
                       \vernacular{
                      tá(awe)}  &  \\
Imp
                     \textsubscript{pl}Neg &   
                       \vernacular{Ø}  &   
                       \vernacular{mu-}  &   
                       \vernacular{kha}  &   
                       \vernacular{\ob mu}  &   
                       \vernacular{
                      [lakhúuli]\cb }  &   
                       \vernacular{
                      tá(awe)}  &  \\
Near Fut Neg [SB]  &   
                       \vernacular{sh-}  &   
                       \vernacular{a}  &   Ø  &   
                       \vernacular{\ob mu}  &   
                       \vernacular{
                      [lakhúula]\cb }  &   
                       \vernacular{tá}  &  \\
Hod Perf  &   Ø  &   
                       \vernacular{a}  &   Ø  &   
                       \vernacular{\ob mu}  &   
                       \vernacular{
                      [lakhúuli]\cb }  &   
                       \vernacular{}  &  \\
Hod Perf Neg  &   Ø  &   
                       \vernacular{a}  &   Ø  &   
                       \vernacular{\ob mu}  &   
                       \vernacular{
                      [lakhúuli]\cb }  &   
                       \vernacular{
                      tá(awe)}  &  \\
\end{tabular}
%\caption{\nocaption}
    
\z

 As in the Subjunctive Negative, the Hodiernal
              Perfective, the Hodiernal Perfective Negative, and
              the negative Imperatives surface with the same tonal
              properties whether they appear pre-pausally or
              phrase-medially. Examples are provided below of both
              /H/ and /Ø/ stems before a H-toned noun, \vernacular{
              mú{\downstep}yáyi} \gloss{‘boy’}, and a
              toneless noun \vernacular{muundu} \gloss{
              ‘person/somebody’}.

 
\ea\label{ex:xP2aPhraseMed} 
Tenses Like the Subjunctive
                Negative Phrase Medially 


\begin{tabular}{llll}  
  
                       \textbf{Imp
                      }  &   
                    /H/  &   
                       \vernacular{u-kha\ob [khalaka]\cb 
                      mú{\downstep}yáyi tá(awe)}  &  \\

                       \vernacular{u-kha\ob [khalaka]\cb 
                      múúndú tá(awe)
                      }  &  \\
  &     &  \\

                    /Ø/  &   
                       \vernacular{
                      u-kha\ob [lakhúula]\cb  mú{\downstep}yáyi
                      tá(awe)}  &  \\

                       \vernacular{
                      u-kha\ob [lakhúula]\cb  múúndú
                      tá(awe)}  &  \\
  &     &     &  \\

                       \textbf{Imp
                      }  &   
                    /H/  &   
                       \vernacular{
                      mu-kha\ob [khalachi]\cb  mú{\downstep}yáyi
                      tá(awe)}  &  \\

                       \vernacular{
                      mu-kha\ob [khalachi]\cb  muundu tá(awe)}  &  \\
  &     &  \\

                    /Ø/  &   
                       \vernacular{
                      mu-kha\ob [lakhúuli]\cb  mú{\downstep}yáyi
                      tá(awe)}  &  \\

                       \vernacular{
                      mu-kha\ob [lakhúuli]\cb  muundu
                      tá(awe)}  &  \\
  &   \multicolumn{1}{l}{ } &     &  \\

                       \textbf{Hod Perf}  &   
                    /H/  &   
                       \vernacular{a\ob [khalaache]\cb 
                      mú{\downstep}yáyi}  &  \\

                       \vernacular{a\ob [khalaache]\cb 
                      muundu}  &  \\
  &     &  \\

                    /Ø/  &   
                       \vernacular{a\ob [lakhúuli]\cb 
                      mú{\downstep}yáyi}  &  \\

                       \vernacular{a\ob [lakhúuli]\cb 
                      muundu}  &  \\
  &     &     &  \\

                       \textbf{Hod Perf Neg}  &   
                    /H/  &   
                       \vernacular{a\ob [khalaache]\cb 
                      mú{\downstep}yáyi tá(awe)}  &  \\

                       \vernacular{a\ob [khalaache]\cb 
                      muundu tá(awe)}  &  \\
  &     &  \\

                    /Ø/  &   
                       \vernacular{a\ob [lakhúuli]\cb 
                      mú{\downstep}yáyi tá(awe)}  &  \\

                       \vernacular{a\ob [lakhúuli]\cb 
                      muundu tá(awe)}  &  \\
\end{tabular}
%\caption{\nocaption}
    
\z

 Both of the negative Imperatives obligatorily
              take 2 \textsuperscript{nd}person
              subjects, and the subject was not varied during the
              collection of data for these other Pattern 2a tenses.
              While there is little reason to expect
              subject-induced tonal alternations in these contexts,
              the unavailability of data preclude further
              comment.

 Finally, observe that the passive suffix realizes
              its H in the Imperative \textsubscript{pl}Negative only in
              /Ø/ stems, just as in the Imperative \textsubscript{sg}Negative forms
              presented in § \sectref{sec:sP2aPassives} above.

 
\ea\label{ex:xP2aPassive} 
/H/ \& /Ø/ Stems with the
                Passive Suffix 


\begin{tabular}{lll}  
  
                       \textbf{Imp
                      }  &   
                       \vernacular{
                      mu-kha\ob [khalak-w-i]\cb  tá}  &  \\

                       \vernacular{
                      mu-kha\ob [lakhúul-w-í]\cb  {\downstep}tá}  &  \\
\end{tabular}
%\caption{\nocaption}
    
\z

 A surprising difference emerges when one compares
              the behavior of passive Hs in Hodiernal Perfective
              and Hodiernal Perfective Negative constructions
              against passive H behavior in other P2a
              constructions: the stem tone properties are the same
              as in non-passive contexts except that a H is
              realized within the final stem syllable in both /H/
              and /Ø/ stems, rather than only in the latter.
              Consider the data below. 

 
\ea\label{ex:xHodPerfPassives} 
Hodiernal Perfective \&
                Hodiernal Perfective Negative: Passives \gloss{‘s/he was
                (not)...ed’}[SB] \footnote{\label{fn:nAffvsNegFVDiff} The final syllable is long in the affirmative,
                  but short in the negative because the negative
                  element \vernacular{
                  tá}triggers \regel{Non-Final
                  Shortening} \REF{ex:xNonFinalShortening} .


}%



\begin{tabular}{lllll}  
  \multicolumn{2}{l}{/H/ Stems } &   \multicolumn{2}{l}{/Ø/ Stems } &  \\

                       \vernacular{
                      a[khalaachúi]}  &   
                       \gloss{‘cut’}  &   
                       \vernacular{
                      a\ob [lakhúulúi]\cb }  &   
                       \gloss{‘released’}  &  \\

                       \vernacular{
                      a[tsuunzuunúi]}  &   
                       \gloss{‘sucked’}  &   
                       \vernacular{
                      a\ob [kalúshiitsúi]\cb }  &   
                       \gloss{‘returned’}  &  \\

                       \vernacular{a[khalaachwí]
                      {\downstep}tá}  &   
                       \gloss{‘cut’}  &   
                       \vernacular{a\ob [lakhúulwí]\cb 
                      {\downstep}tá}  &   
                       \gloss{‘released’}  &  \\

                       \vernacular{a[tsuunzuunwí]
                      {\downstep}tá}  &   
                       \gloss{‘sucked’}  &   
                       \vernacular{
                      a\ob [kalúshiitswí]\cb  {\downstep}tá}  &   
                       \gloss{‘returned’}  &  \\
\end{tabular}
%\caption{\nocaption}
    
\z

 The data in \REF{ex:xHodPerfPassives} differ from the
              Imperative \textsubscript{sg}Negative (and
              other tenses like it) in that both /H/ and /Ø/ stems
              realize the passive H, rather than just /Ø/ stems.
              These facts motivate a revision to the preliminary
              formulation of \regel{Passive H
              Assignment}given in \REF{ex:xPassiveHAssignmentPrelim} .

 The preliminary formulation required that the verb
              appear in a context inflected with a melodic H and
              that the melodic H be realized on the stem. To
              account for the Hodiernal Perfective and Hodiernal
              Perfective Negative data, I revise the second
              criterion such that it is disjunctive, satisfied by
              fulfilling \textit{either}of the two
              following conditions: (i) the melodic H is realized
              on the stem \textit{or}(ii) the verb
              appears in a tense with the perfective suffix.

 The final version of \regel{Passive H
              Assignment}is formulated below.

 
\ea\label{ex:xPassiveHAssignmentFinal} 
 \regel{Passive H Assignment
                  (Final)} 

%\includegraphics[width=\textwidth]{InkScape%20Images/Rules/PassiveHAssignmentFinal.pdf}

\z

 The Hodiernal Perfective data show that the
              perfective suffix plays an important role in
              determining whether the passive H will appear in a
              given verbal form, but that role is notably different
              from the role it plays in Ganda. In Ganda, the
              perfective suffix is necessary, and not just
              sufficient, to license the passive H ( \citealt{rHymanKatamba1990a} ).



\subsection{Pattern 2b: Conditional Negative}\label{sec:sPattern2b}

Pattern 2b has the same core properties as Pattern
            2a, except that subject prefixes are H-toned in Pattern
            2b. The Conditional Negative is the only construction
            that exhibits the properties of this melody. The
            morphology of the Conditional Negative is very much
            like the Subjunctive Negative, except that it is marked
            with the \vernacular{ni-} \footnote{\label{fn:nNiParticleInCondNeg} The particle is hypothesized to be underlyingly \vernacular{/ni-/}on
              parallel with the Crastinal Future. In the Crastinal
              Future, the vowel of the \vernacular{ni-}particle
              elides and triggers compensatory lengthening of the
              subject prefix vowel when combined with vowel-initial
              subject prefixes (e.g., \vernacular{na-a}for 3 \textsubscript{rd}singular and \vernacular{nu-u}for 2 \textsubscript{nd}singular
              subjects, but \vernacular{ni-βa}for 3 \textsubscript{rd}plural and \vernacular{ni-mu}for 2 \textsubscript{nd}plural
              subjects). The only tonal data available for the
              Conditional Negative involves 3 \textsubscript{rd}singular
              subjects.


}%
\sectref{sec:sP2bObjects} , we will
            see that the Conditional Negative also differs from
            Pattern 2a constructions in exhibiting a striking tonal
            alternation, \regel{Pinball Shift}, in
            forms with two object prefixes.

 In the Conditional Negative, /H/ verbs surface all L \REF{ex:xCondNegCH} , and /Ø/ verbs realize a melodic H on the
            second stem mora \REF{ex:xCondNegCØ} . In addition, the subject prefix is
            H.

 
\ea\label{ex:xCondNegCH} 
Conditional Negative C-Initial /H/ \gloss{‘if s/he does
              not...’}


\begin{tabular}{lllllll}  
  Part  &   Subj  &   Tns  &   Stem  &   Neg  &   Gloss  &  \\

                     \vernacular{na-}  &   
                     \vernacular{á-}  &   
                     \vernacular{kha}  &   
                     \vernacular{
                    \ob [khwa]\cb }  &   
                     \vernacular{
                    tá(awe)}  &   
                     \gloss{‘pay dowry’}  &  \\

                     \vernacular{na-}  &   
                     \vernacular{á-}  &   
                     \vernacular{kha}  &   
                     \vernacular{
                    \ob [βeka]\cb }  &   
                     \vernacular{
                    tá(awe)}  &   
                     \gloss{‘shave’}  &  \\

                     \vernacular{na-}  &   
                     \vernacular{á-}  &   
                     \vernacular{kha}  &   
                     \vernacular{
                    \ob [teekha]\cb }  &   
                     \vernacular{
                    tá(awe)}  &   
                     \gloss{‘cook’}  &  \\

                     \vernacular{na-}  &   
                     \vernacular{á-}  &   
                     \vernacular{kha}  &   
                     \vernacular{
                    \ob [khalaka]\cb }  &   
                     \vernacular{
                    tá(awe)}  &   
                     \gloss{‘cut’}  &  \\

                     \vernacular{na-}  &   
                     \vernacular{á-}  &   
                     \vernacular{kha}  &   
                     \vernacular{
                    \ob [kalaanga]\cb }  &   
                     \vernacular{
                    tá(awe)}  &   
                     \gloss{‘fry’}  &  \\

                     \vernacular{na-}  &   
                     \vernacular{á-}  &   
                     \vernacular{kha}  &   
                     \vernacular{
                    \ob [βoolitsa]\cb }  &   
                     \vernacular{
                    tá(awe)}  &   
                     \gloss{‘seduce’}  &  \\

                     \vernacular{na-}  &   
                     \vernacular{á-}  &   
                     \vernacular{kha}  &   
                     \vernacular{
                    \ob [βoyong’ana]\cb }  &   
                     \vernacular{
                    tá(awe)}  &   
                     \gloss{‘go around’}  &  \\
\end{tabular}
%\caption{\nocaption}
    
\z

 
\ea\label{ex:xCondNegCØ} 
Conditional Negative C-Initial /Ø/ \gloss{‘if s/he does
              not...’} \footnote{\label{fn:nNoPlateauInP2b} The formulation of \regel{Plateau}in \REF{ex:xPlateau} predicts that the melodic H should
                undergo leftward spreading through the tense prefix
                in this context, though my recordings include few
                productions in which \regel{Plateau}was
                unambiguously operative. While many Conditional
                Negative forms meet the structural description of \regel{Plateau}, only
                those tokens in which \regel{Plateau}clearly
                played a roll in determining the surface tone
                pattern are transcribed as having undergone the
                rule.


}%



\begin{tabular}{lllllll}  
  Part  &   Subj  &   Tns  &   Stem  &   Neg  &   Gloss  &  \\

                     \vernacular{na-}  &   
                     \vernacular{á-}  &   
                     \vernacular{kha}  &   
                     \vernacular{
                    \ob [kwá]\cb }  &   
                     \vernacular{
                    {\downstep}tá(awe)}  &   
                     \gloss{‘fall’}  &  \\

                     \vernacular{na-}  &   
                     \vernacular{á-}  &   
                     \vernacular{kha}  &   
                     \vernacular{
                    \ob [lekhá]\cb }  &   
                     \vernacular{
                    {\downstep}tá(awe)}  &   
                     \gloss{‘leave’}  &  \\

                     \vernacular{na-}  &   
                     \vernacular{á-}  &   
                     \vernacular{kha}  &   
                     \vernacular{
                    \ob [reéβa]\cb }  &   
                     \vernacular{
                    tá(awe)}  &   
                     \gloss{‘ask’}  &  \\

                     \vernacular{na-}  &   
                     \vernacular{á-}  &   
                     \vernacular{kha}  &   
                     \vernacular{
                    \ob [kulíkha]\cb }  &   
                     \vernacular{
                    tá(awe)}  &   
                     \gloss{‘name’}  &  \\

                     \vernacular{na-}  &   
                     \vernacular{á-}  &   
                     \vernacular{kha}  &   
                     \vernacular{
                    \ob [lakhúula]\cb }  &   
                     \vernacular{
                    tá(awe)}  &   
                     \gloss{‘release’}  &  \\

                     \vernacular{na-}  &   
                     \vernacular{á-}  &   
                     \vernacular{kha}  &   
                     \vernacular{
                    \ob [seéβula]\cb }  &   
                     \vernacular{
                    tá(awe)}  &   
                     \gloss{‘say
                    goodbye’}  &  \\

                     \vernacular{na-}  &   
                     \vernacular{á-}  &   
                     \vernacular{kha}  &   
                     \vernacular{
                    \ob [kalúshitsa]\cb }  &   
                     \vernacular{
                    tá(awe)}  &   
                     \gloss{‘return’}  &  \\
\end{tabular}
%\caption{\nocaption}
    
\z


\subsubsection{Conditional Negative with Object
              Prefixes}\label{sec:sP2bObjects}

As in the Subjunctive Negative, Hs contributed by
              a single object prefix do not surface, though the
              root H does. /Ø/ stems realize the melodic H on the
              second stem mora. 

 
\ea\label{ex:xCondNegCHOP} 
Conditional Negative C-Initial /H/
                + OP \gloss{‘if s/he does
                not...him/her’}


\begin{tabular}{llllllll}  
  Part  &   Subj  &   Tns  &   Obj  &   Stem  &   Neg  &   Gloss  &  \\

                       \vernacular{na-}  &   
                       \vernacular{á-}  &   
                       \vernacular{kha}  &   
                       \vernacular{\ob mu}  &   
                       \vernacular{
                      [khwá]\cb }  &   
                       \vernacular{
                      tá(awe)}  &   
                       \gloss{‘pay
                      dowry’}  &  \\

                       \vernacular{na-}  &   
                       \vernacular{á-}  &   
                       \vernacular{kha}  &   
                       \vernacular{\ob mu}  &   
                       \vernacular{
                      [βéka]\cb }  &   
                       \vernacular{
                      tá(awe)}  &   
                       \gloss{‘shave’}  &  \\

                       \vernacular{na-}  &   
                       \vernacular{á-}  &   
                       \vernacular{kha}  &   
                       \vernacular{\ob mu}  &   
                       \vernacular{
                      [téekha]\cb }  &   
                       \vernacular{
                      tá(awe)}  &   
                       \gloss{‘cook’}  &  \\

                       \vernacular{na-}  &   
                       \vernacular{á-}  &   
                       \vernacular{kha}  &   
                       \vernacular{\ob mu}  &   
                       \vernacular{
                      [khálaka]\cb }  &   
                       \vernacular{
                      tá(awe)}  &   
                       \gloss{‘cut’}  &  \\

                       \vernacular{na-}  &   
                       \vernacular{á-}  &   
                       \vernacular{kha}  &   
                       \vernacular{\ob mu}  &   
                       \vernacular{
                      [βóolitsa]\cb }  &   
                       \vernacular{
                      tá(awe)}  &   
                       \gloss{‘seduce’}  &  \\

                       \vernacular{na-}  &   
                       \vernacular{á-}  &   
                       \vernacular{kha}  &   
                       \vernacular{\ob mu}  &   
                       \vernacular{
                      [βóyong’ana]\cb }  &   
                       \vernacular{
                      tá(awe)}  &   
                       \gloss{‘go
                      around’}  &  \\
\end{tabular}
%\caption{\nocaption}
    
\z

 
\ea\label{ex:xCondNegCØOP} 
Conditional Negative C-Initial /Ø/
                + OP \gloss{‘if s/he does
                not...(to) him/her’}


\begin{tabular}{llllllll}  
  Part  &   Subj  &   Tns  &   Obj  &   Stem  &   Neg  &   Gloss  &  \\

                       \vernacular{na-}  &   
                       \vernacular{á-}  &   
                       \vernacular{kha}  &   
                       \vernacular{\ob mu}  &   
                       \vernacular{
                      [kwá]\cb }  &   
                       \vernacular{
                      {\downstep}tá(awe)}  &   
                       \gloss{‘fall’}  &  \\

                       \vernacular{na-}  &   
                       \vernacular{á-}  &   
                       \vernacular{kha}  &   
                       \vernacular{\ob mu}  &   
                       \vernacular{
                      [lekhá]\cb }  &   
                       \vernacular{
                      {\downstep}tá(awe)}  &   
                       \gloss{‘leave’}  &  \\

                       \vernacular{na-}  &   
                       \vernacular{á-}  &   
                       \vernacular{kha}  &   
                       \vernacular{\ob mu}  &   
                       \vernacular{
                      [reéβa]\cb }  &   
                       \vernacular{
                      tá(awe)}  &   
                       \gloss{‘ask’}  &  \\

                       \vernacular{na-}  &   
                       \vernacular{á-}  &   
                       \vernacular{kha}  &   
                       \vernacular{\ob mu}  &   
                       \vernacular{
                      [kulíkha]\cb }  &   
                       \vernacular{
                      tá(awe)}  &   
                       \gloss{‘name’}  &  \\

                       \vernacular{na-}  &   
                       \vernacular{á-}  &   
                       \vernacular{kha}  &   
                       \vernacular{\ob mu}  &   
                       \vernacular{
                      [seéβula]\cb }  &   
                       \vernacular{
                      tá(awe)}  &   
                       \gloss{‘say
                      goodbye’}  &  \\

                       \vernacular{na-}  &   
                       \vernacular{á-}  &   
                       \vernacular{kha}  &   
                       \vernacular{\ob mu}  &   
                       \vernacular{
                      [kalúshitsa]\cb }  &   
                       \vernacular{
                      tá(awe)}  &   
                       \gloss{‘return’}  &  \\
\end{tabular}
%\caption{\nocaption}
    
\z

 Conditional Negative forms with two object
              prefixes are tonally identical to parallel Pattern 2a
              except that, in addition to the H on the subject
              prefix, there is also an unexpected H on the FV. In
              /Ø/ stems, the melodic H surfaces as usual on the
              second stem mora and spreads left to the initial mora
              via \regel{Plateau}. \footnote{\label{fn:nCondNegOPOPDiscrepancies} JI does not produce the melodic H in Conditional
                Negative /Ø/ verbs with two object prefixes, e.g., \vernacular{
                na-á-kha\ob mu-ú[ndeshela]\cb  táawe}for \gloss{‘if s/he does not
                leave him/her for me’}. This point of
                divergence between my primary consultants was noted
                previously in \fnref{fn:nSubjNegOPOPDiscrepancy} . Additionally, his recordings for /H/
                stems with two object prefixes in phrase final
                position indicate that he does not realize a H on
                the FV either. This may be a case of free
                variation, as his productions of the same verb
                forms in a phrase-medial context exhibit the
                effects of Pinball Shift.


}%


 
\ea\label{ex:xCondNegCHOPOP1sg} 
Conditional Negative C-Initial /H/
                + OP + OP \textsubscript{1sg} \gloss{‘if s/he does
                not...him/her for
                me’}[SB]


\begin{tabular}{lllllllll}  
  Part  &   Subj  &   Tns  &   Obj
                     \textsubscript{CV} &   Obj
                     \textsubscript{1sg} &   Stem  &   Neg  &   Gloss  &  \\

                       \vernacular{na-}  &   
                       \vernacular{á-}  &   
                       \vernacular{kha}  &   
                       \vernacular{\ob mu-}  &   
                       \vernacular{ú}  &   
                       \vernacular{
                      [ndeelá]\cb }  &   
                       \vernacular{
                      {\downstep}tá(awe)}  &   
                       \gloss{‘bury’}  &  \\

                       \vernacular{na-}  &   
                       \vernacular{á-}  &   
                       \vernacular{kha}  &   
                       \vernacular{\ob mu-}  &   
                       \vernacular{ú}  &   
                       \vernacular{
                      [mbechelá]\cb }  &   
                       \vernacular{
                      {\downstep}tá(awe)}  &   
                       \gloss{‘shave’}  &  \\

                       \vernacular{na-}  &   
                       \vernacular{á-}  &   
                       \vernacular{kha}  &   
                       \vernacular{\ob mu-}  &   
                       \vernacular{ú}  &   
                       \vernacular{
                      [ndeerelá]\cb }  &   
                       \vernacular{
                      {\downstep}tá(awe)}  &   
                       \gloss{‘bring’}  &  \\

                       \vernacular{na-}  &   
                       \vernacular{á-}  &   
                       \vernacular{kha}  &   
                       \vernacular{\ob mu-}  &   
                       \vernacular{ú}  &   
                       \vernacular{
                      [khalachilá]\cb }  &   
                       \vernacular{
                      {\downstep}tá(awe)}  &   
                       \gloss{‘cut’}  &  \\
\end{tabular}
%\caption{\nocaption}
    
\z

 
\ea\label{ex:xCondNegCØOPOP1sg} 
Conditional Negative C-Initial /Ø/
                + OP + OP \textsubscript{1sg} \gloss{‘if s/he does
                not...(to) him/her for
                me’}[SB]


\begin{tabular}{lllllllll}  
  Part  &   Subj  &   Tns  &   Obj
                     \textsubscript{CV} &   Obj
                     \textsubscript{1sg} &   Stem  &   Neg  &   Gloss  &  \\

                       \vernacular{na-}  &   
                       \vernacular{á-}  &   
                       \vernacular{kha}  &   
                       \vernacular{\ob mu-}  &   
                       \vernacular{ú}  &   
                       \vernacular{
                      [{\downstep}nzííla]\cb }  &   
                       \vernacular{
                      tá(awe)}  &   
                       \gloss{‘go’}  &  \\

                       \vernacular{na-}  &   
                       \vernacular{á-}  &   
                       \vernacular{kha}  &   
                       \vernacular{\ob mu-}  &   
                       \vernacular{ú}  &   
                       \vernacular{
                      [{\downstep}ndéshéla]\cb }  &   
                       \vernacular{
                      tá(awe)}  &   
                       \gloss{‘leave’}  &  \\

                       \vernacular{na-}  &   
                       \vernacular{á-}  &   
                       \vernacular{kha}  &   
                       \vernacular{\ob mu-}  &   
                       \vernacular{ú}  &   
                       \vernacular{
                      [{\downstep}nóóndela]\cb }  &   
                       \vernacular{
                      tá(awe)}  &   
                       \gloss{‘follow’}  &  \\

                       \vernacular{na-}  &   
                       \vernacular{á-}  &   
                       \vernacular{kha}  &   
                       \vernacular{\ob mu-}  &   
                       \vernacular{ú}  &   
                       \vernacular{
                      [{\downstep}ngúlíshila]\cb }  &   
                       \vernacular{
                      tá(awe)}  &   
                       \gloss{‘name’}  &  \\
\end{tabular}
%\caption{\nocaption}
    
\z

 The unexpected H on the FV is noteworthy as it
              illustrates yet another way in which sequences of
              adjacent underlying Hs are treated in Idakho’s tonal
              system. Ordinarily, multiple potential Hs at the left
              edge of the stem are reduced to one H by \regel{Meeussen’s
              Rule}and/or \regel{Initial Lowering}.
              In Pattern 1 constructions, HH becomes HL and HHH
              becomes HLL. In Pattern 2a constructions HH becomes
              LH and HHH becomes LHL. What is true of Pattern 2a is
              true of Pattern 2b as well, except that in Pattern
              2b, one of the lost Hs is preserved on the FV.

 I analyze the H appearing on the FV in /H/ stems
              as the root H, originating on the stem initial
              syllable and undergoing a long distance shifting
              process \regel{Pinball Shift}. The
              Nyala-West variety of Luhya ( \citealt{rEbarbEtAlInPrep} ) and the non-Luhya language Ciruri ( \citealt{rMassamba1982} , \citealt{rMassamba1984} ) also attest versions
              of \regel{Pinball Shift}. The
              H immediately preceding the stem is the conditioning
              structure. Building from the analysis of Pattern 2a,
              I derive the tonal properties of Pattern 2b with a
              rule \regel{Pinball Shift}. This
              rule delinks and subsequently re-associates the root
              H to the final mora of the stem, as formalized in \REF{ex:xPinballShift} .

 
\ea\label{ex:xPinballShift} 
 \regel{Pinball
                  Shift} 

%\includegraphics[width=\textwidth]{InkScape%20Images/Rules/PinballShift.pdf}

\z

 Note that this analysis requires that \regel{Pinball
              Shift}precede \regel{Meeussen's Rule}so
              that \regel{Meeussen’s
              Rule}should not bleed \regel{Pinball Shift}, and \regel{Pinball Shift}must
              follow \regel{Initial Lowering}in
              the derivation to account for the fact that Pinball
              Shift is not observed in /H/ stems containing a
              single object prefix, as shown in the derivation
              below.

 
\ea\label{ex:xDerivCondNegHOPx2} 
 Derivation,
                  /H/ Cond. Neg. + OPx2: \vernacular{
                  na-á-kha\ob mu-ú[ndeelá]\cb  {\downstep}tá} \gloss{‘if s/he does not
                  bury him/her for me’} 

%\includegraphics[width=\textwidth]{InkScape%20Images/Derivations/DerivCondNegHOPx2.pdf}

\z

 An alternative would be to re-interpret \regel{Meeussen's Rule}as a
              delinking, rather than deletion, rule and \regel{Pinball Shift}as a
              later applying rule which associates the root H to
              the final mora. While this approach could be made to
              work, it complicates the explanation of stranded
              melodic Hs in /H/ verbs with no or only a single
              object prefix. Recall that in this case, the melodic
              H, which surfaces on the second stem mora in /Ø/
              verbs, fails to be assigned to /H/ stems due to the
              proximity of the root H. If there were a rule which
              associates floating Hs to the FV in the Conditional
              Negative, one would expect the melodic H to surface
              on the FV in these contexts. Under this approach, the
              assignment rule would have to be restricted to
              interact only with stranded root Hs, while ignoring
              melodic Hs. While there is precedent for rules
              referring to the morphological origins of tones, I
              advocate for the more direct approach described
              above.

 The properties of the Conditional Negative are
              summarized schematically in \REF{ex:xCondNegSchematic} .

 
\ea\label{ex:xCondNegSchematic} 
A Schematic Representation of the
                Conditional Negative’s Tonal
                Properties 


\begin{tabular}{lllll}  
    &   \multicolumn{3}{l}{
                       \ul{/H/ Verbs} } &  \\
  &   
                       \textit{Part + Subj +
                      Tns}  &   \multicolumn{2}{l}{
                       \textit{Macrostem} } &  \\
OPsx0  &   
                       \vernacular{
                      na-á-kha}  &   
                       \vernacular{\ob }  &   
                       \vernacular{[C
                      }  &  \\
OPsx1  &   
                       \vernacular{
                      na-á-kha-}  &   
                       \vernacular{\ob C
                      }  &   
                       \vernacular{[C
                      }  &  \\
OPsx2  &   
                       \vernacular{
                      na-á-kha-}  &   
                       \vernacular{\ob C
                      }  &   
                       \vernacular{[C
                      }  &  \\
  &   \multicolumn{2}{l}{ } &     &  \\
  &   \multicolumn{3}{l}{
                       \textbf{
                        } } &  \\
  &   
                       \textit{Part + Subj +
                      Tns}  &   \multicolumn{2}{l}{
                       \textit{Macrostem} } &  \\
OPsx0  &   
                       \vernacular{
                      na-á-kha}  &   
                       \vernacular{\ob }  &   
                     \vernacular{[CVC
                    }\cb  &  \\
OPsx1  &   
                       \vernacular{
                      na-á-kha-}  &   
                       \vernacular{\ob C
                      }  &   
                       \vernacular{[CVC
                      }  &  \\
OPsx2  &   
                       \vernacular{
                      na-á-kha-}  &   
                       \vernacular{\ob C
                      }  &   
                       \vernacular{[CVC
                      }  &  \\
\end{tabular}
%\caption{\nocaption}
    
\z



\subsubsection{Conditional Negative: Phrase
              Medially}\label{sec:sP2bPhraseMed}

As in Pattern 2a, the position of the verb within
              its phrase has no impact on the stem tone properties
              of the Conditional Negative. The verb tone
              phrase-medially is the same as in phrase final
              position. 

 
\ea\label{ex:xCondNegPhraseMedial} 
Conditional Negative Phrase
                Medially [SB] \gloss{‘if s/he does
                not...(for him/her)’}


\begin{tabular}{lllll}  
  
                       %\includegraphics[width=\textwidth]{InkScape%20Images/H%20Stems.svg}
 &   
                       %\includegraphics[width=\textwidth]{InkScape%20Images/No%20OP.svg}
 &   
                       \vernacular{na-á-kha\ob [ra]\cb 
                      mú{\downstep}yáyi tá}  &   
                       \gloss{‘bury the
                      boy’}  &  \\

                       \vernacular{na-á-kha\ob [ra]\cb 
                      muundu tá}  &   
                       \gloss{‘bury
                      somebody’}  &  \\
  &     &  \\

                       \vernacular{
                      na-á-kha\ob [khalaka]\cb  mú{\downstep}yáyi tá}  &   
                       \gloss{‘cut the
                      boy’}  &  \\

                       \vernacular{
                      na-á-kha\ob [khalaka]\cb  muundu tá}  &   
                       \gloss{‘cut
                      somebody’}  &  \\
  &     &     &  \\

                       %\includegraphics[width=\textwidth]{InkScape%20Images/One%20OP.svg}
 &   
                       \vernacular{
                      na-á-kha\ob mu[réela]\cb  mú{\downstep}yáyi
                      tá}  &   
                       \gloss{‘bury the
                      boy’}  &  \\

                       \vernacular{
                      na-á-kha\ob mu[réela]\cb  muundu tá}  &   
                       \gloss{‘bury
                      somebody’}  &  \\
  &     &  \\

                       \vernacular{
                      na-á-kha\ob mu[khálachila]\cb  mú{\downstep}yáyi
                      tá}  &   
                       \gloss{‘cut the
                      boy’}  &  \\

                       \vernacular{
                      na-á-kha\ob mu[khálachila]\cb  muundu
                      tá}  &   
                       \gloss{‘cut
                      somebody’}  &  \\
  &     &     &  \\

                       %\includegraphics[width=\textwidth]{InkScape%20Images/0%20Stems.svg}
 &   
                       %\includegraphics[width=\textwidth]{InkScape%20Images/No%20OP.svg}
 &   
                       \vernacular{
                      na-á-{\downstep}khá\ob [tsyá]\cb  mú{\downstep}yáyi tá}  &   
                       \gloss{‘go for the
                      boy’}  &  \\

                       \vernacular{
                      na-á-{\downstep}khá\ob [tsyá]\cb  muundu tá}  &   
                       \gloss{‘go for
                      somebody’}  &  \\
  &     &  \\

                       \vernacular{
                      na-á-kha\ob [seéβula]\cb  mú{\downstep}yáyi
                      tá}  &   
                       \gloss{‘say goodbye to the
                      boy’}  &  \\

                       \vernacular{
                      na-á-kha\ob [seéβula]\cb  muundu tá}  &   
                       \gloss{‘say goodbye to
                      somebody’}  &  \\
  &     &     &  \\

                       %\includegraphics[width=\textwidth]{InkScape%20Images/One%20OP.svg}
 &   
                       \vernacular{
                      na-á-kha\ob mu[tsiíla]\cb  mú{\downstep}yáyi
                      tá}  &   
                       \gloss{‘go for the
                      boy’}  &  \\

                       \vernacular{
                      na-á-kha\ob mu[tsiíla]\cb  muundu tá}  &   
                       \gloss{‘go for
                      somebody’}  &  \\
  &     &  \\

                       \vernacular{
                      na-á-kha\ob mu[seéβulila]\cb  mú{\downstep}yáyi
                      tá}  &   
                       \gloss{‘say goodbye to the
                      boy’}  &  \\

                       \vernacular{
                      na-á-kha\ob mu[seéβulila]\cb  muundu
                      tá}  &   
                       \gloss{‘say goodbye to
                      somebody’}  &  \\
\end{tabular}
%\caption{\nocaption}
    
\z

 In addition, /H/ verbs with two object prefixes
              surface with a H on the FV phrase-medially, as in \vernacular{
              na-á-kha-mu-ú[ndeerelá] muundu tá} \gloss{‘if he does not bring
              somebody for me for him/her’}.



\subsubsection{Conditional Negative: Impact of Subject
              Choice}\label{sec:sP2bSubjects}

The H involved in the rise on the word initial
              syllable in the Conditional Negative is associated to
              the mora contributed by the subject prefix. The
              subject prefix appears to be H in this context for no
              other reason than that it appears in the Conditional
              Negative construction. A similar phenomenon is
              observed in the Tachoni variety of Luhya, in which
              subject prefixes are ordinarily toneless, but surface
              with a H tone in tenses involving the \vernacular{
              ni-}particle ( \citealt{rOdden2009} :
              317-319).

 My corpus does not include any examples with 1 \textsuperscript{st}and 2 \textsuperscript{nd}person
              subjects for this construction. This tense was not
              included in the survey which tested the impact of
              subject choice on stem tonal properties, and so it
              remains to be seen how the /L/ vs. /Ø/ contrast in
              subject prefixes plays out in this context.



\subsubsection{Conditional Negative: Passives}\label{sec:sP2bPassives}

As in the Subjunctive Negative, the passive suffix
              realizes a H in the final syllable only in
              Conditional Negative /Ø/ verbs, which surface also
              with a melodic H on the second stem syllable. /H/
              stems do not realize the root H, the melodic H, or
              the passive H. This is consistent with the analysis
              of passives developed in § \sectref{sec:sP2aOtherTenses} .

 
\ea\label{ex:xCondNegPassives} 
Conditional Negative: Passives \gloss{‘if s/he is
                not...’}[SB]


\begin{tabular}{lllll}  
  \multicolumn{2}{l}{/H/ Stems } &   \multicolumn{2}{l}{/Ø/ Stems } &  \\

                       \vernacular{
                      na-á-kha\ob [khalak-u-a]\cb }  &   
                       \gloss{‘cut’}  &   
                       \vernacular{
                      na-á-kha\ob [lakhúul-ú-a]\cb }  &   
                       \gloss{‘released’}  &  \\

                       \vernacular{
                      na-á-kha\ob [tsuunzuun-u-a]\cb }  &   
                       \gloss{‘sucked’}  &   
                       \vernacular{
                      na-á-kha\ob [kalúshits-ú-a]\cb }  &   
                       \gloss{‘returned’}  &  \\
\end{tabular}
%\caption{\nocaption}
    
\z



\subsection{Summary of Pattern 2}\label{sec:sP2InterimSumm}

Key features of constructions which select Pattern 2
            melodies are (i) a rule of \regel{Initial Lowering}and
            (ii) a melodic H which surfaces on the second stem mora
            in /Ø/ stems. \regel{Initial Lowering}has
            the effect of lowering the root H in forms which do not
            involve object prefixes, and the H associated with the
            leftmost object prefix in forms with one or more object
            prefixes. The verbs position within its phrase and the
            choice of subject appear not to impact stem tonal
            properties in tenses which exhibit this pattern.

 Pattern 2 constructions reveal a complex set of
            conditions which determine the distribution of passive
            Hs. In particular, verbs realizing a passive H must
            appear in a context inflected with a melodic H. In
            addition, the verbal form must satisfy either of the
            following two conditions: (i) the melodic H surfaces on
            the stem or (ii) the verb appears in a tense with the
            perfective suffix. 



\section{Pattern 3: The subjunctive pattern}\label{sec:sPattern3}

The affirmative Subjunctive, as well as the
          affirmative and negative of the subjunctive-based future,
          exhibit the properties of Pattern 3. The basic tonal
          melody associated with this pattern is expressed as a H
          which surfaces on the second mora after the initial
          syllable of the macrostem. \regel{Initial Lowering}also
          characterizes Pattern 3.


\subsection{Pattern 3: Subjunctive}\label{sec:sPattern3x}

This section details the tonal properties of Pattern
            3, as illustrated by the Subjunctive. The Subjunctive
            is marked by the FV \vernacular{-ɛ}, which
            raises to \vernacular{
            -ɪ}following high vowels, and a melodic H
            which surfaces on the second mora after the initial
            syllable of the macrostem. The Subjunctive takes no
            tense prefix.

 The data below demonstrate that, in stems comprising
            three or more syllables, the melodic H surfaces on the
            second stem syllable when the same is long, but on the
            third stem syllable otherwise. In other words, the
            melodic H is realized on the second mora beyond the
            first stem syllable. In stems of fewer than three
            syllables, the melodic H surfaces on the FV. Note also
            that the root H is not realized. 

 
\ea\label{ex:xSubjCH} 
Subjunctive C-Initial /H/ \gloss{‘let
              him/her...’}


\begin{tabular}{llll}  
  Subj  &   Stem  &   Gloss  &  \\

                     \vernacular{a}  &   
                     \vernacular{
                    \ob [khúi]\cb }  &   
                     \gloss{‘pay dowry’}  &  \\

                     \vernacular{a}  &   
                     \vernacular{
                    \ob [βechɛ́]\cb }  &   
                     \gloss{‘shave’}  &  \\

                     \vernacular{a}  &   
                     \vernacular{
                    \ob [teeshɛ́]\cb }  &   
                     \gloss{‘cook’}  &  \\

                     \vernacular{a}  &   
                     \vernacular{
                    \ob [khalachɛ́]\cb }  &   
                     \gloss{‘cut’}  &  \\

                     \vernacular{a}  &   
                     \vernacular{
                    \ob [kalaánjɛ]\cb }  &   
                     \gloss{‘fry’}  &  \\

                     \vernacular{a}  &   
                     \vernacular{
                    \ob [sitaáchɛ]\cb }  &   
                     \gloss{‘accuse’}  &  \\

                     \vernacular{a}  &   
                     \vernacular{
                    \ob [βoolitsɪ́]\cb }  &   
                     \gloss{‘seduce’}  &  \\

                     \vernacular{a}  &   
                     \vernacular{
                    \ob [saanditsɪ́]\cb }  &   
                     \gloss{‘thank’}  &  \\

                     \vernacular{a}  &   
                     \vernacular{
                    \ob [tsuunzuúnɪ]\cb }  &   
                     \gloss{‘suck’}  &  \\

                     \vernacular{a}  &   
                     \vernacular{
                    \ob [βoyong’ánɛ]\cb }  &   
                     \gloss{‘go around’}  &  \\

                     \vernacular{a}  &   
                     \vernacular{
                    \ob [ng’ong’oólitsɪ]\cb }  &   
                     \gloss{‘tease’}  &  \\
\end{tabular}
%\caption{\nocaption}
    
\z

 
\ea\label{ex:xSubjVH} 
Subjunctive V-Initial /H/ \gloss{‘let
              him/her...’}


\begin{tabular}{llll}  
  Subj  &   Stem  &   Gloss  &  \\

                     \vernacular{y}  &   
                     \vernacular{
                    \ob [iirɪ́]\cb }  &   
                     \gloss{‘kill’}  &  \\

                     \vernacular{y}  &   
                     \vernacular{\ob [oonoɲ́ɲɪ]\cb 
                    }  &   
                     \gloss{‘spoil’}  &  \\

                     \vernacular{y}  &   
                     \vernacular{
                    \ob [aabukhányːɪ]\cb }  &   
                     \gloss{‘separate’}  &  \\
\end{tabular}
%\caption{\nocaption}
    
\z

 The melodic H targets the same position in /Ø/
            verbs―the melodic H surfaces on the second mora after
            the initial syllable of the macrostem. Because \regel{Initial
            Lowering}lowers the root H in /H/ verbs, the
            lexical contrast is neutralized in Subjunctive forms
            lacking an object prefix.

 
\ea\label{ex:xSubjCØ} 
Subjunctive C-Initial /Ø/ \gloss{‘let
              him/her...’}


\begin{tabular}{llll}  
  Subj  &   Stem  &   Gloss  &  \\

                     \vernacular{a}  &   
                     \vernacular{
                    \ob [kúi]\cb }  &   
                     \gloss{‘fall’}  &  \\

                     \vernacular{a}  &   
                     \vernacular{
                    \ob [leshɛ́]\cb }  &   
                   \gloss{
                  ‘leave’}[SB] &  \\

                     \vernacular{a}  &   
                     \vernacular{
                    \ob [reeβɛ́]\cb }  &   
                     \gloss{‘ask’}  &  \\

                     \vernacular{a}  &   
                     \vernacular{
                    \ob [kulishɪ́]\cb }  &   
                   \gloss{
                  ‘name’}[SB] &  \\

                     \vernacular{a}  &   
                     \vernacular{
                    \ob [lakhuúlɪ]\cb }  &   
                     \gloss{‘release’}  &  \\

                     \vernacular{a}  &   
                     \vernacular{
                    \ob [seeβulɪ́]\cb }  &   
                     \gloss{‘say goodbye
                    (to)’}  &  \\

                     \vernacular{a}  &   
                     \vernacular{
                    \ob [kalushítsɪ]\cb }  &   
                   \gloss{
                  ‘return’}[SB] &  \\

                     \vernacular{a}  &   
                     \vernacular{
                    \ob [siinjilítsɪ]\cb }  &   
                     \gloss{‘make stand’}  &  \\

                     \vernacular{a}  &   
                     \vernacular{
                    \ob [seβulúkhaɲːi]\cb }  &   
                     \gloss{‘scatter’}  &  \\
\end{tabular}
%\caption{\nocaption}
    
\z

 V-initial /Ø/ stems pattern with C-initial
            stems. 

 
\ea\label{ex:xSubjVØ} 
Subjunctive V-Initial /Ø/ \gloss{‘let
              him/her...’}


\begin{tabular}{llll}  
  Subj  &   Stem  &   Gloss  &  \\

                     \vernacular{a}  &   
                     \vernacular{
                    [eenyɛ́]}  &   
                     \gloss{‘want’}  &  \\

                     \vernacular{a}  &   
                     \vernacular{
                    [eeyelɛ́]}  &   
                     \gloss{‘wipe for’}  &  \\

                     \vernacular{a}  &   
                     \vernacular{
                    [iiluúlɪ]}  &   
                     \gloss{‘winnow’}  &  \\

                     \vernacular{a}  &   
                     \vernacular{
                    [aambakhánɛ]}  &   
                     \gloss{‘refuse’}  &  \\
\end{tabular}
%\caption{\nocaption}
    
\z


\subsubsection{Subjunctive with Object Prefixes}\label{sec:sP3xObjects}

In /H/ verbs with an object prefix, a single H
              appears on the initial mora of the stem. I analyze
              this as the root H. 

 
\ea\label{ex:xSubjCHOP} 
Subjunctive C-Initial /H/ + OP \gloss{‘let
                him/her...him/her’}


\begin{tabular}{lllll}  
  Subj  &   Obj  &   Stem  &   Gloss  &  \\

                       \vernacular{a}  &   
                       \vernacular{\ob mu}  &   
                       \vernacular{
                      [rɛ́ɛ]\cb }  &   
                       \gloss{‘bury’}  &  \\

                       \vernacular{a}  &   
                       \vernacular{\ob mu}  &   
                       \vernacular{
                      [βéchɛ]\cb }  &   
                       \gloss{‘shave’}  &  \\

                       \vernacular{a}  &   
                       \vernacular{\ob mu}  &   
                       \vernacular{
                      [léerɛ]\cb }  &   
                       \gloss{‘bring’}  &  \\

                       \vernacular{a}  &   
                       \vernacular{\ob mu}  &   
                       \vernacular{
                      [khálachɛ]\cb }  &   
                       \gloss{‘cut’}  &  \\

                       \vernacular{a}  &   
                       \vernacular{\ob mu}  &   
                       \vernacular{
                      [βóolitsɪ]\cb }  &   
                       \gloss{‘seduce’}  &  \\

                       \vernacular{a}  &   
                       \vernacular{\ob mu}  &   
                       \vernacular{
                      [βóyong’anɛ]\cb }  &   
                       \gloss{‘go
                      around’}  &  \\
\end{tabular}
%\caption{\nocaption}
    
\z

 
\ea\label{ex:xSubjVHOP} 
Subjunctive V-Initial /H/ + OP \gloss{‘let
                him/her...him/her’}


\begin{tabular}{lllll}  
  Subj  &   Obj  &   Stem  &   Gloss  &  \\

                       \vernacular{a}  &   
                       \vernacular{\ob mw}  &   
                       \vernacular{
                      [iírɪ]\cb }  &   
                       \gloss{‘kill’}  &  \\

                       \vernacular{a}  &   
                       \vernacular{\ob mw}  &   
                       \vernacular{
                      [oónonyːɪ]\cb }  &   
                       \gloss{‘spoil’}  &  \\

                       \vernacular{a}  &   
                       \vernacular{\ob mw}  &   
                       \vernacular{
                      [aábukhanyːɪ]\cb }  &   
                       \gloss{‘separate’}  &  \\
\end{tabular}
%\caption{\nocaption}
    
\z

 The H of the object prefix similarly does not
              surface in /Ø/ stems, while the melodic H is realized
              on the second stem mora. 

 
\ea\label{ex:xSubjCØOP} 
Subjunctive C-Initial /Ø/ + OP \gloss{‘let
                him/her...him/her’}


\begin{tabular}{lllll}  
  Subj  &   Obj  &   Stem  &   Gloss  &  \\

                       \vernacular{a}  &   
                       \vernacular{\ob mu}  &   
                       \vernacular{
                      [tsíi]\cb }  &   
                       \gloss{‘go (for)’}  &  \\

                       \vernacular{a}  &   
                       \vernacular{\ob mu}  &   
                       \vernacular{
                      [leshɛ́]\cb }  &   
                       \gloss{‘leave’}  &  \\

                       \vernacular{a}  &   
                       \vernacular{\ob mu}  &   
                       \vernacular{
                      [loóndɛ]\cb }  &   
                       \gloss{‘follow’}  &  \\

                       \vernacular{a}  &   
                       \vernacular{\ob mu}  &   
                       \vernacular{
                      [kulíshɪ]\cb }  &   
                     \gloss{
                    ‘name’}[SB] &  \\

                       \vernacular{a}  &   
                       \vernacular{\ob mu}  &   
                       \vernacular{
                      [seéβulɪ]\cb }  &   
                       \gloss{‘say
                      goodbye’}  &  \\

                       \vernacular{a}  &   
                       \vernacular{\ob mu}  &   
                       \vernacular{
                      [kalúshitsɪ]\cb }  &   
                     \gloss{
                    ‘return’}[SB] &  \\
\end{tabular}
%\caption{\nocaption}
    
\z

 
\ea\label{ex:xSubjVØOP} 
Subjunctive V-Initial /Ø/ + OP \gloss{‘let
                him...him/her’}


\begin{tabular}{lllll}  
  Subj  &   Obj  &   Stem  &   Gloss  &  \\

                       \vernacular{a}  &   
                       \vernacular{\ob mw}  &   
                       \vernacular{
                      [eenyɛ́]\cb }  &   
                       \gloss{‘want’}  &  \\

                       \vernacular{a}  &   
                       \vernacular{\ob mw}  &   
                       \vernacular{
                      [eeyélɛ]\cb }  &   
                       \gloss{‘wipe for’}  &  \\

                       \vernacular{a}  &   
                       \vernacular{\ob mw}  &   
                       \vernacular{
                      [aambákhanɛ]\cb }  &   
                       \gloss{‘refuse’}  &  \\
\end{tabular}
%\caption{\nocaption}
    
\z

 As shown in \REF{ex:xSubjCHOP1sg} - \REF{ex:xSubjCØOP1sg} , the tonal properties of Subjunctive
              verb forms with 1 \textsuperscript{st}sg object
              prefixes are identical to those involving CV- object
              prefixes. A H surfaces on the initial stem mora in
              /H/ verbs, while a H surfaces on the second stem mora
              in /Ø/ verbs.

 
\ea\label{ex:xSubjCHOP1sg} 
Subjunctive C-Initial /H/ + OP \textsubscript{1sg} \gloss{‘let
                him/her...me’}


\begin{tabular}{lllll}  
  Subj  &   Obj  &   Stem  &   Gloss  &  \\

                       \vernacular{a}  &   
                       \vernacular{\ob a}  &   
                       \vernacular{
                      [ríi]\cb }  &   
                     \gloss{
                    ‘fear’}[SB] &  \\

                       \vernacular{a}  &   
                       \vernacular{\ob a}  &   
                       \vernacular{
                      [mbéchɛ]\cb }  &   
                       \gloss{‘shave’}  &  \\

                       \vernacular{a}  &   
                       \vernacular{\ob a}  &   
                       \vernacular{
                      [ndéerɛ]\cb }  &   
                       \gloss{‘bring’}  &  \\

                       \vernacular{a}  &   
                       \vernacular{\ob a}  &   
                       \vernacular{
                      [khálachɛ]\cb }  &   
                       \gloss{‘cut’}  &  \\

                       \vernacular{a}  &   
                       \vernacular{\ob a}  &   
                       \vernacular{
                      [mbóolitsɪ]\cb }  &   
                       \gloss{‘seduce’}  &  \\

                       \vernacular{a}  &   
                       \vernacular{\ob a}  &   
                       \vernacular{
                      [mbóyong’anɛ]\cb }  &   
                       \gloss{‘go
                      around’}  &  \\
\end{tabular}
%\caption{\nocaption}
    
\z

 
\ea\label{ex:xSubjCØOP1sg} 
Subjunctive C-Initial /Ø/ + OP \textsubscript{1sg} \gloss{‘let
                him/her...me’}


\begin{tabular}{lllll}  
  Subj  &   Obj  &   Stem  &   Gloss  &  \\

                       \vernacular{a}  &   
                       \vernacular{\ob a}  &   
                       \vernacular{
                      [síi]}  &   
                     \gloss{
                    ‘grind’}[SB] &  \\

                       \vernacular{a}  &   
                       \vernacular{\ob a}  &   
                       \vernacular{
                      [ndeshɛ́]}  &   
                       \gloss{‘leave’}  &  \\

                       \vernacular{a}  &   
                       \vernacular{\ob a}  &   
                       \vernacular{
                      [noóndɛ]}  &   
                       \gloss{‘follow’}  &  \\

                       \vernacular{a}  &   
                       \vernacular{\ob a}  &   
                       \vernacular{
                      [ngulíshɛ]}  &   
                     \gloss{
                    ‘name’}[SB] &  \\

                       \vernacular{a}  &   
                       \vernacular{\ob a}  &   
                       \vernacular{
                      [seéβulɪ]}  &   
                       \gloss{‘say goodbye
                      (to)’}  &  \\

                       \vernacular{a}  &   
                       \vernacular{\ob a}  &   
                       \vernacular{
                      [ngalúshitsɪ]}  &   
                     \gloss{
                    ‘return’}[SB] &  \\
\end{tabular}
%\caption{\nocaption}
    
\z

 When CV- and 1 \textsuperscript{st}sg object
              prefixes appear together in the Subjunctive, a rising
              tone surfaces on the pre-stem syllable. In /H/ verbs,
              the stem is all L.

 
\ea\label{ex:xSubjCHOPOP1sg} 
Subjunctive C-Initial /H/ + OP + OP \textsubscript{1sg} \gloss{‘let
                him/her...him/her for
                me’}


\begin{tabular}{llllll}  
  Subj  &   Obj
                     \textsubscript{CV} &   Obj
                     \textsubscript{1sg} &   Stem  &   Gloss  &  \\

                       \vernacular{a}  &   
                       \vernacular{\ob mu-}  &   
                       \vernacular{ú}  &   
                       \vernacular{
                      [ndeelɛ]\cb }  &   
                       \gloss{‘bury’}  &  \\

                       \vernacular{a}  &   
                       \vernacular{\ob mu-}  &   
                       \vernacular{ú}  &   
                       \vernacular{
                      [mbechelɛ]\cb }  &   
                       \gloss{‘shave’}  &  \\

                       \vernacular{a}  &   
                       \vernacular{\ob mu-}  &   
                       \vernacular{ú}  &   
                       \vernacular{
                      [ndeerelɛ]\cb }  &   
                       \gloss{‘bring’}  &  \\

                       \vernacular{a}  &   
                       \vernacular{\ob mu-}  &   
                       \vernacular{ú}  &   
                       \vernacular{
                      [khalachilɪ]\cb }  &   
                       \gloss{‘cut’}  &  \\
\end{tabular}
%\caption{\nocaption}
    
\z

 /Ø/ verbs with two object prefixes also realize a
              rise on the pre-stem syllable, but a H appears on the
              first two moras of the stem. 

 
\ea\label{ex:xSubjCØOPOP1sg} 
Subjunctive C-Initial /Ø/ + OP + OP \textsubscript{1sg} \gloss{‘let
                him/her...him/her for
                me’}[SB]


\begin{tabular}{llllll}  
  Subj  &   Obj
                     \textsubscript{CV} &   Obj
                     \textsubscript{1sg} &   Stem  &   Gloss  &  \\

                       \vernacular{u-}  &   
                       \vernacular{\ob mu-}  &   
                       \vernacular{ú}  &   
                       \vernacular{
                      [{\downstep}nzíílɪ]\cb }  &   
                       \gloss{‘go (for)’}  &  \\

                       \vernacular{u-}  &   
                       \vernacular{\ob mu-}  &   
                       \vernacular{ú}  &   
                       \vernacular{
                      [{\downstep}ndéshélɛ]\cb }  &   
                       \gloss{‘leave’}  &  \\

                       \vernacular{u-}  &   
                       \vernacular{\ob mu-}  &   
                       \vernacular{ú}  &   
                       \vernacular{
                      [{\downstep}nóóndelɛ]\cb }  &   
                       \gloss{‘follow’}  &  \\

                       \vernacular{u-}  &   
                       \vernacular{\ob mu-}  &   
                       \vernacular{ú}  &   
                       \vernacular{
                      [{\downstep}ndákhúulilɪ]\cb }  &   
                       \gloss{‘release’}  &  \\
\end{tabular}
%\caption{\nocaption}
    
\z

 The Subjunctive has the following tonal
              properties: (i) underlying macrostem initial Hs fail
              to surface, (ii) the melodic H, when realized,
              surfaces on the second mora after the initial
              syllable of the macrostem in forms with and without
              object prefixes, regardless of the verb’s tonal
              class, and (iii) the melodic H does not surface in
              /H/ verbs with one or more object prefixes. These
              properties are summarized schematically in the
              following display. As before, the position of
              underlying Hs is indicated with a single underline,
              and the melodic H, when it appears, is indicated with
              double underlining. 

 
\ea\label{ex:xSubjSchematic} 
A Schematic Representation of the
                Subjunctive’s Tonal Properties 


\begin{tabular}{lllll}  
    &   \multicolumn{3}{l}{
                       \ul{/H/ Verbs} } &  \\
  &   
                       \textit{Subj}  &   \multicolumn{2}{l}{
                       \textit{Macrostem} } &  \\
OPsx0  &   
                       \vernacular{a}  &   
                       \vernacular{\ob }  &   
                       \vernacular{[C
                      }  &  \\
OPsx1  &   
                       \vernacular{a}  &   
                       \vernacular{\ob C
                      }  &   
                       \vernacular{[C
                      }  &  \\
OPsx2  &   
                       \vernacular{a}  &   
                       \vernacular{\ob C
                      }  &   
                       \vernacular{[C
                      }  &  \\
  &   \multicolumn{3}{l}{ } &  \\
  &   \multicolumn{3}{l}{
                       \textbf{
                        } } &  \\
  &   
                       \textit{Subj}  &   \multicolumn{2}{l}{
                       \textit{Macrostem} } &  \\
OPsx0  &   
                       \vernacular{a}  &   
                       \vernacular{\ob }  &   
                     \vernacular{[CV(V)CV(C)
                    }\cb  &  \\
OPsx1  &   
                       \vernacular{a}  &   
                       \vernacular{\ob C
                      }  &   
                       \vernacular{[CV(C)
                      }  &  \\
OPsx2  &   
                       \vernacular{a}  &   
                       \vernacular{\ob C
                      }  &   
                       \vernacular{[CV(C)
                      }  &  \\
\end{tabular}
%\caption{\nocaption}
    
\z

 The observation that macrostem initial Hs are not
              realized receives the same analysis as it does in
              Pattern 2. Namely, \regel{Initial Lowering} \REF{ex:xInitialLowering} lowers Hs initial
              within the macrostem, causing the leftmost object
              prefix H in forms with object prefixes and root Hs in
              forms without object prefixes to fail to
              surface.

 We turn next to the observation that the melodic
              H, when it surfaces, consistently surfaces on the
              second mora beyond the initial syllable of the
              macrostem in all forms, regardless of the verb's
              tonal class and the presence of object prefixes (this
              claim will be qualified shortly to accommodate forms
              with very short macrostems). As a starting point, we
              may formulate a rule of \regel{Subjunctive Melodic H
              Assignment}as in \REF{ex:xSubjunctiveMHAPrelim} below.

 
\ea\label{ex:xSubjunctiveMHAPrelim} 
 \regel{Subjunctive Melodic
                  H Assignment (Preliminary)} 

%\includegraphics[width=\textwidth]{InkScape%20Images/Rules/SubjunctiveMHAPrelim.pdf}

\z

 The rule above, in combination with \regel{Initial Lowering},
              accounts for the position of the melodic H in both
              /H/ and /Ø/ verbs, with and without object prefixes.
              Below, I derive one /H/ verb \REF{ex:xDerivSubjH} and one /Ø/ verb \REF{ex:xDerivSubjØ} , illustrating how these two rules work
              together to produce identical surface patterns in
              verbs of both tonal classes.

 
\ea\label{ex:xDerivSubjH} 
 
                  Derivation: \vernacular{
                  a\ob [khalachɛ́]\cb } \gloss{‘let him/her
                  cut’} 

%\includegraphics[width=\textwidth]{InkScape%20Images/Derivations/DerivSubjH.pdf}

\z

 
\ea\label{ex:xDerivSubjØ} 
 
                  Derivation: \vernacular{
                  a\ob [kulishɪ́]\cb } \gloss{‘let him/her
                  name’} 

%\includegraphics[width=\textwidth]{InkScape%20Images/Derivations/DerivSubj0.pdf}

\z

 When an object prefix is present, the melodic H
              is again assigned to the second mora after the
              initial syllable of the macrostem by \regel{Subjunctive MHA}.
              The derivation in \REF{ex:xDerivSubjØOP} shows how \regel{Initial Lowering}and \regel{Subjunctive
              MHA}successfully apply to generate the
              surface tonal pattern of a verb form with three short
              stem syllables.

 
\ea\label{ex:xDerivSubjØOP} 
 
                  Derivation: \vernacular{
                  a\ob mu[kulíshɪ]\cb } \gloss{‘let him/her name
                  him/her’} 

%\includegraphics[width=\textwidth]{InkScape%20Images/Derivations/DerivSubj0OP.pdf}

\z

 All of the same rules apply in /H/ verbs, even \regel{Subjunctive MHA}.
              The failure of the melodic H to surface in /H/ verbs
              when an object prefix is present is attributed to \regel{Meeussen’s Rule}:
              the melodic H is first assigned to the mora
              immediately following the root H by \regel{Subjunctive MHA},
              only to later be deleted by \regel{Meeussen’s
              Rule}.

 
\ea\label{ex:xDerivSubjHOP} 
 
                  Derivation: \vernacular{
                  a\ob mu[kálachɛ]\cb } \gloss{‘let him/her cut
                  him/her’} 

%\includegraphics[width=\textwidth]{InkScape%20Images/Derivations/DerivSubjHOP.pdf}

\z

 The failure of the melodic H to be assigned to
              the above forms may also be analyzed as resulting
              from a requirement that the mora preceding the target
              of \regel{Subjunctive MHA}be
              toneless—the same requirement imposed upon \regel{Default MHA}. The
              present work acknowledges the descriptive adequacy of
              both approaches, though deletion by \regel{Meeussen’s Rule}is
              advocated on the assumption that the requirement of
              tonelessness unnecessarily adds complexity to the
              formal statement of \regel{Subjunctive
              MHA}.

 The rule of \regel{Subjunctive MHA}as
              stated formally in \REF{ex:xSubjunctiveMHAPrelim} requires two
              syllables in the macrostem—three if the second
              syllable is short. This rule needs to be revised in
              order to account for the verb forms with mono- and
              disyllabic macrostems repeated in \REF{ex:xSubjCHShort} .

 
\ea\label{ex:xSubjCHShort} 
Short Subjunctive C-Initial /Ø/ \gloss{‘let
                him/her...’}


\begin{tabular}{llll}  
  Subj  &   Stem  &   Gloss  &  \\

                       \vernacular{a}  &   
                       \vernacular{
                      \ob [kúi]\cb }  &   
                       \gloss{‘fall’}  &  \\

                       \vernacular{a}  &   
                       \vernacular{
                      \ob [leshɛ́]\cb }  &   
                     \gloss{
                    ‘leave’}[SB] &  \\

                       \vernacular{a}  &   
                       \vernacular{
                      \ob [reeβɛ́]\cb }  &   
                       \gloss{‘ask’}  &  \\
\end{tabular}
%\caption{\nocaption}
    
\z

 In disyllabic stems, there is no second mora
              beyond the initial syllable. Instead the initial mora
              of the macrostem is followed by a single mora. In
              monosyllabic stems, there is only the initial
              syllable of the macrostem. The original formulation
              of \regel{Subjunctive MHA} \REF{ex:xSubjunctiveMHAPrelim} would not
              apply in such forms, as the structural description of
              the rule is not met. To accommodate these shorter
              stems, I invoke parenthesis notation as shown in the
              final formulation of \regel{Subjunctive MHA}in \REF{ex:xSubjunctiveMHAFinal} .

 
\ea\label{ex:xSubjunctiveMHAFinal} 
 \regel{Subjunctive Melodic
                  H Assignment (Final)} 

%\includegraphics[width=\textwidth]{InkScape%20Images/Rules/SubjunctiveMHAFinal.pdf}

\z

 The second largest extension of the rule \REF{ex:xSubjunctiveMHAFinal2nd} accounts
              for disyllabic stems with long initial syllables, as
              derived in \REF{ex:xDerivSubjØCVVCV} .

 
\ea\label{ex:xSubjunctiveMHAFinal2nd} 
 \regel{Subjunctive Melodic
                  H Assignment (2nd Largest Extension)} 

%\includegraphics[width=\textwidth]{InkScape%20Images/Rules/SubjunctiveMHAFinal2nd.pdf}

\z

 
\ea\label{ex:xDerivSubjØCVVCV} 
 
                  Derivation: \vernacular{
                  a\ob [reeβɛ́]\cb } \gloss{‘let him/her
                  ask’} 

%\includegraphics[width=\textwidth]{InkScape%20Images/Derivations/DerivSubj0CVVCV.pdf}

\z

 The smallest extension of the rule \REF{ex:xSubjunctiveMHAFinalSmallest} ,
              accounts for disyllabic stems with short initial
              syllables as well as monosyllabic stems.

 
\ea\label{ex:xSubjunctiveMHAFinalSmallest} 
 \regel{Subjunctive Melodic
                  H Assignment (Smallest Extension)} 

%\includegraphics[width=\textwidth]{InkScape%20Images/Rules/SubjunctiveMHAFinalSmallest.pdf}

\z

 As shown in \REF{ex:xDerivSubjØCVCV} , the melodic H is
              assigned directly to its surface position on the FV
              in disyllabic stems with short initial
              syllables.

 
\ea\label{ex:xDerivSubjØCVCV} 
 
                  Derivation: \vernacular{
                  a\ob [leshɛ́]\cb } \gloss{‘let him/her leave
                  (behind)’} 

%\includegraphics[width=\textwidth]{InkScape%20Images/Derivations/DerivSubj0CVCV.pdf}

\z

 The melodic H is first assigned to the FV in
              monosyllabic stems by the smallest extension of \regel{Subjunctive MHA} \REF{ex:xSubjunctiveMHAFinalSmallest} . The
              melodic H then shifts left to the initial mora of the
              long final syllable via \regel{Final Rise
              Elimination}.

 
\ea\label{ex:xDerivSubjØCVV} 
 
                  Derivation: \vernacular{
                  a\ob [kúi]\cb } \gloss{‘let him/her
                  fall’} 

%\includegraphics[width=\textwidth]{InkScape%20Images/Derivations/DerivSubj0CVV.pdf}

\z

 In monosyllabic /H/ verbs, the melodic H surfaces
              on the initial mora of the stem―the same position as
              the underlying root H. In disyllabic and longer
              stems, the root H is lowered by \regel{Initial Lowering}.
              Next, the melodic H is assigned to the FV by the
              smallest extension of \regel{Subjunctive MHA} \REF{ex:xSubjunctiveMHAFinalSmallest} .
              Finally, the melodic H shifts to the initial mora of
              the stem by \regel{Final Rise
              Elimination}.

  \regel{Final Rise
              Elimination}does not require that the mora
              to which it shifts a H be toneless. I assume that a
              tone assigned to a mora already bearing another tone
              will replace the previously linked tone—this in
              contrast to other conceivable scenarios in which
              either \regel{Final Rise
              Elimination}fails to apply, or the targeted
              mora bears both tones.

 
\ea\label{ex:xDerivSubjHCVV} 
 
                  Derivation: \vernacular{
                  a\ob [khúi]\cb } \gloss{‘let him pay
                  dowry’} 

%\includegraphics[width=\textwidth]{InkScape%20Images/Derivations/DerivSubjHCVV.pdf}

\z



\subsubsection{Subjunctive: Phrase Medially}\label{sec:sP3xPhraseMed}

The tonal properties of the Subjunctive are
              unaffected by the verb's position within its phrase.
              Four pairs of /H/ and /Ø/ stems are provided below,
              half with and half without an object prefix. For each
              pair, the first member involves a H-toned complement,
              while the second involves a toneless complement. In
              each case, the stem tonal properties are the same as
              the pre-pausal counterparts. 

 
\ea\label{ex:xSubjPhraseMedial} 
Subjunctive Phrase Medially \gloss{‘let him/her...(for
                him/her)’}


\begin{tabular}{lllll}  
  
                       %\includegraphics[width=\textwidth]{InkScape%20Images/H%20Stems.svg}
 &   
                       %\includegraphics[width=\textwidth]{InkScape%20Images/No%20OP.svg}
 &   
                       \vernacular{a[ré]
                      {\downstep}mú{\downstep}yáyi}  &   
                       \gloss{‘bury the
                      boy’}  &  \\

                       \vernacular{a[ré]
                      muundu}  &   
                       \gloss{‘bury
                      somebody’}  &  \\
  &     &  \\

                       \vernacular{a[khalachɛ́]
                      {\downstep}mú{\downstep}yáyi}  &   
                       \gloss{‘cut the
                      boy’}  &  \\

                       \vernacular{a[khalachɛ́]
                      muundu}  &   
                       \gloss{‘cut
                      somebody’}  &  \\
  &     &     &  \\

                       %\includegraphics[width=\textwidth]{InkScape%20Images/One%20OP.svg}
 &   
                       \vernacular{a-mu[ré]
                      {\downstep}mú{\downstep}yáyi}  &   
                       \gloss{‘bury the
                      boy’}  &  \\

                       \vernacular{a-mu[ré]
                      muundu}  &   
                       \gloss{‘bury
                      somebody’}  &  \\
  &     &  \\

                       \vernacular{a-mu[khálachɛ]
                      mú{\downstep}yáyi}  &   
                       \gloss{‘cut the
                      boy’}  &  \\

                       \vernacular{a-mu[khálachɛ]
                      muundu}  &   
                       \gloss{‘cut
                      somebody’}  &  \\
  &     &     &  \\

                       %\includegraphics[width=\textwidth]{InkScape%20Images/0%20Stems.svg}
 &   
                       %\includegraphics[width=\textwidth]{InkScape%20Images/No%20OP.svg}
 &   
                       \vernacular{a[tsí]
                      {\downstep}mú{\downstep}yáyi}  &   
                       \gloss{‘go for the
                      boy’}  &  \\

                       \vernacular{a[tsí]
                      muundu}  &   
                       \gloss{‘go for
                      somebody’}  &  \\
  &     &  \\

                       \vernacular{a[seebulɪ́]
                      {\downstep}mú{\downstep}yáyi}  &   
                       \gloss{‘say goodbye to the
                      boy’}  &  \\

                       \vernacular{a[seebulɪ́]
                      muundu}  &   
                       \gloss{‘say goodbye to
                      somebody’}  &  \\
  &     &     &  \\

                       %\includegraphics[width=\textwidth]{InkScape%20Images/One%20OP.svg}
 &   
                       \vernacular{a-mu[tsí]
                      {\downstep}mú{\downstep}yáyi}  &   
                       \gloss{‘go for the
                      boy’}  &  \\

                       \vernacular{a-mu[tsí]
                      muundu}  &   
                       \gloss{‘go for
                      somebody’}  &  \\
  &     &  \\

                       \vernacular{a-mu[seéβulɪ]
                      mú{\downstep}yáyi}  &   
                       \gloss{‘say goodbye to the
                      boy’}  &  \\

                       \vernacular{a-mu[seéβulɪ]
                      muundu}  &   
                       \gloss{‘say goodbye to
                      somebody’}  &  \\
\end{tabular}
%\caption{\nocaption}
    
\z



\subsubsection{Subjunctive: Impact of Subject
              Choice}\label{sec:sP3xSubjects}

Data were not elicited to test for the influence
              of subject choice in the tonal properties of
              subjunctives. However, it is reported in § \sectref{sec:sP3xOtherTenses} that
              the Crastinal Future, another context in which the
              pattern described above manifests, does not show
              evidence of subject induced tonal effects.



\subsubsection{Subjunctive: Passives}\label{sec:sP3xPassives}

In verbs long enough to accommodate both the
              melodic H and the passive H, both are realized; this
              includes any verb in which the melodic H is realized
              prior to the final syllable. In all shorter stems,
              the expected position of the melodic and passive Hs
              are the same, and only one H surfaces. 

 
\ea\label{ex:xSubjPassives} 
Subjunctive: Passives \gloss{‘let him/her
                be...’}[SB]


\begin{tabular}{lllll}  
  \multicolumn{2}{l}{/H/ Stems } &   \multicolumn{2}{l}{/Ø/ Stems } &  \\

                       \vernacular{
                      a[khalak-ú-i]}  &   
                       \gloss{‘cut’}  &   
                       \vernacular{
                      a[lakhuú{\downstep}l-ú-i]}  &   
                       \gloss{‘released’}  &  \\

                       \vernacular{
                      a[tsuunzuú{\downstep}n-ú-i]}  &   
                       \gloss{‘sucked’}  &   
                       \vernacular{
                      a[kalushí{\downstep}tsíl-ú-i]}  &   
                       \gloss{‘returned
                      for’}  &  \\
\end{tabular}
%\caption{\nocaption}
    
\z

 The available passive data for the Subjunctive is
              consistent with the analysis of passive H assignment
              developed thus far. The melodic H is realized in each
              of the above forms, and, provided that the verb stem
              is capable of accommodating both Hs, both
              surface. 

 The tonal properties of Subjunctives with both an
              object prefix and the passive suffix is unknown. The
              analysis predicts that /Ø/ verbs, because they
              realize a melodic H on the second stem mora, should
              also realize the passive H. On the other hand, /H/
              verbs do not realize the melodic H when an object
              prefix is present. Given this, the analysis developed
              in § \sectref{sec:sP2aOtherTenses} predicts that the passive H would not be
              realized in such forms. I hope to test this
              prediction in future work.



\subsubsection{Pattern 3: Other Verbal Contexts}\label{sec:sP3xOtherTenses}

The tonal pattern of the Subjunctive characterizes
              the subjunctive-based Crastinal Future as well, in
              both the affirmative and negative. In these contexts,
              as in the Subjunctive, the melodic H is realized on
              the second mora beyond the initial syllable of the
              macrostem in the absence of any object prefixes. When
              an object prefix is present, /H/ verbs do not realize
              the melodic H, and /Ø/ verbs realize the melodic H on
              the second stem mora. 

 The Crastinal Future and the Crastinal Future
              Negative similarly interact with phrasal position and
              passivization in the same way as the Subjunctive.
              That is, the tonal properties of the verb are the
              same whether in phrase final or phrase-medial
              position, and the passive suffix realizes a H
              provided the verbal stem is of sufficient length to
              express both the melodic and passive H. 

 
\ea\label{ex:xP3xTenses} 
Other Pattern 3 Verbal
                Contexts 


\begin{tabular}{llll}  
  a.  &   Crastinal Future  &   
                       \vernacular{
                      ni-SP[ROOT-ɛ]}  &  \\
b.  &   Crastinal Future Negative  &   
                       \vernacular{ni-SP[ROOT-ɛ]
                      tá(awe)}  &  \\
\end{tabular}
%\caption{\nocaption}
    
\z

 For each of the tenses listed in \REF{ex:xP3xTenses} , both /H/ and /Ø/ verbs realize the
              melodic H on the second mora beyond the initial
              syllable of the macrostem when no object prefix is
              present, as shown in \REF{ex:xP3xHStems} and \REF{ex:xP3xØStems} , respectively. Additionally, the root H
              is not realized due to \regel{Initial
              Lowering}.

 
\ea\label{ex:xP3xHStems} 
Morphologically Simple /H/ Stems
                [SB] \footnote{\label{fn:nP3xGlosses} The examples included in the current section
                  use \vernacular{
                  -khálak-} \gloss{‘cut’}and \vernacular{
                  -khóng’oond-} \gloss{‘knock’}to
                  illustrate the properties of /H/ verbs, and \vernacular{
                  -kulix-} \gloss{‘name’}and \vernacular{
                  -lakhuul-} \gloss{‘release’}as
                  representative of /H/ and /Ø/ verbal roots,
                  respectively. The basic gloss for the tenses
                  discussed in this section is the following:
                  Crastinal Future - \gloss{‘s/he
                  will...’}; Crastinal Future Negative - \gloss{‘s/he will
                  not...’}.


}%



\begin{tabular}{llllll}  
    &   Part  &   Subj  &   Stem  &   Neg  &  \\
Crast Fut  &   
                       \vernacular{na-}  &   
                       \vernacular{a}  &   
                       \vernacular{
                      \ob [khalachɛ́]\cb }  &     &  \\

                       \vernacular{na-}  &   
                       \vernacular{a}  &   
                       \vernacular{
                      \ob [khong’oóndɛ]\cb }  &     &  \\
Crast Fut Neg  &   
                       \vernacular{na-}  &   
                       \vernacular{a}  &   
                       \vernacular{
                      \ob [khalachɛ́]\cb }  &   
                       \vernacular{{\downstep}tá}  &  \\

                       \vernacular{na-}  &   
                       \vernacular{a}  &   
                       \vernacular{
                      \ob [khong’oóndɛ]\cb }  &   
                       \vernacular{tá}  &  \\
\end{tabular}
%\caption{\nocaption}
    
\z

 
\ea\label{ex:xP3xØStems} 
Morphologically Simple /Ø/ Stems
                [SB] 


\begin{tabular}{llllll}  
    &   Part  &   Subj  &   Stem  &   Neg  &  \\
Crast Fut  &   
                       \vernacular{na-}  &   
                       \vernacular{a}  &   
                       \vernacular{
                      \ob [kulishɪ́]\cb }  &     &  \\

                       \vernacular{na-}  &   
                       \vernacular{a}  &   
                       \vernacular{
                      \ob [lakhuúlɪ]\cb }  &     &  \\
Crast Fut Neg  &   
                       \vernacular{na-}  &   
                       \vernacular{a}  &   
                       \vernacular{
                      \ob [kulishɪ́]\cb }  &   
                       \vernacular{{\downstep}tá}  &  \\

                       \vernacular{na-}  &   
                       \vernacular{a}  &   
                       \vernacular{
                      \ob [lakhuúlɪ]\cb }  &   
                       \vernacular{tá}  &  \\
\end{tabular}
%\caption{\nocaption}
    
\z

 As in the Subjunctive, the H contributed by a
              single object prefix fails to surface, while the root
              H re-emerges in /H/ stems. The melodic H continues to
              be realized on the second mora beyond the initial
              syllable of the macrostem, which is incidentally also
              the second mora of the stem, in /Ø/ verbs. In /H/
              verbs, the melodic H does not surface. 

 
\ea\label{ex:xP3xOPHStems} 
/H/ Stems with an Object Prefix
                [SB] 


\begin{tabular}{lllllll}  
    &   Part  &   Subj  &   Obj  &   Stem  &   Neg  &  \\
Crast Fut  &   
                       \vernacular{na-}  &   
                       \vernacular{a}  &   
                       \vernacular{\ob mu}  &   
                       \vernacular{
                      [khálachɛ]\cb }  &     &  \\

                       \vernacular{na-}  &   
                       \vernacular{a}  &   
                       \vernacular{\ob mu}  &   
                       \vernacular{
                      [khóng’oondɛ]\cb }  &     &  \\
Crast Fut Neg  &   
                       \vernacular{na-}  &   
                       \vernacular{a}  &   
                       \vernacular{\ob mu}  &   
                       \vernacular{
                      [khálachɛ]\cb }  &   
                       \vernacular{tá}  &  \\

                       \vernacular{na-}  &   
                       \vernacular{a}  &   
                       \vernacular{\ob mu}  &   
                       \vernacular{
                      [khóng’oondɛ]\cb }  &   
                       \vernacular{tá}  &  \\
\end{tabular}
%\caption{\nocaption}
    
\z

 
\ea\label{ex:xP3xOPØStems} 
/Ø/ Stems with an Object Prefix
                [SB] 


\begin{tabular}{lllllll}  
    &   Part  &   Subj  &   Obj  &   Stem  &   Neg  &  \\
Crast Fut  &   
                       \vernacular{na-}  &   
                       \vernacular{a}  &   
                       \vernacular{\ob mu}  &   
                       \vernacular{
                      [kulíshɪ]\cb }  &     &  \\

                       \vernacular{na-}  &   
                       \vernacular{a}  &   
                       \vernacular{\ob mu}  &   
                       \vernacular{
                      [lakhúulɪ]\cb }  &     &  \\
Crast Fut Neg  &   
                       \vernacular{na-}  &   
                       \vernacular{a}  &   
                       \vernacular{\ob mu}  &   
                       \vernacular{
                      [kulíshɪ]\cb }  &   
                       \vernacular{tá}  &  \\

                       \vernacular{na-}  &   
                       \vernacular{a}  &   
                       \vernacular{\ob mu}  &   
                       \vernacular{
                      [lakhúulɪ]\cb }  &   
                       \vernacular{tá}  &  \\
\end{tabular}
%\caption{\nocaption}
    
\z

 As in the Subjunctive, the Crastinal Future and
              the Crastinal Future Negative have identical tonal
              properties phrase-medially as they do phrase finally.
              Examples are provided below of both /H/ and /Ø/ stems
              before a H-toned noun, \vernacular{
              mú{\downstep}yáyi} \gloss{‘boy’}, and a
              toneless noun \vernacular{muundu} \gloss{
              ‘person/somebody’}.

 
\ea\label{ex:xP3xPhraseMed} 
Tenses Like the Subjunctive Phrase
                Medially [SB] 


\begin{tabular}{llll}  
  
                       \textbf{Crast Fut}  &   
                    /H/  &   
                       \vernacular{na-a[khalachɛ́]
                      {\downstep}mú{\downstep}yáyi}  &  \\

                       \vernacular{na-a[khalachɛ́]
                      muundu}  &  \\
  &     &  \\

                    /Ø/  &   
                       \vernacular{na-a[lakhuúlɪ]
                      mú{\downstep}yáyi}  &  \\

                       \vernacular{na-a[lakhuúlɪ]
                      muundu}  &  \\
  &     &     &  \\

                       \textbf{Crast Fut Neg}  &   
                    /H/  &   
                       \vernacular{na-a[khalachɛ́]
                      {\downstep}mú{\downstep}yáyi tá}  &  \\

                       \vernacular{na-a[khalachɛ́]
                      muundu tá}  &  \\
  &     &  \\

                    /Ø/  &   
                       \vernacular{na-a[lakhuúlɪ]
                      mú{\downstep}yáyi tá}  &  \\

                       \vernacular{na-a[lakhuúlɪ]
                      muundu tá}  &  \\
\end{tabular}
%\caption{\nocaption}
    
\z

 The somewhat limited dataset in \REF{ex:xSubjSubjFutH} show no evidence of
              subject-induced tonal alternations in Pattern 3.
              These data are from in the Crastinal Future.

 
\ea\label{ex:xSubjSubjFutH} 
Subject Choice in the Crastinal
                Future /H/ \gloss{‘...will
                bring’}[SB]


\begin{tabular}{llll}  
    &   Singular  &   Plural  &  \\
1
                     \textsuperscript{
                    st}Person &   
                       \vernacular{
                      ne-e[ndeerɛ́]}  &   
                       \vernacular{
                      nɪ-khu[leerɛ́]}  &  \\
2
                     \textsuperscript{
                    nd}Person &   
                       \vernacular{
                      nu-u[leerɛ́]}  &   
                       \vernacular{
                      ni-mu[leerɛ́]}  &  \\
3
                     \textsuperscript{
                    rd}Person &   
                       \vernacular{
                      na-a[leerɛ́]}  &   
                       \vernacular{
                      ni-βa[leerɛ́]}  &  \\
\end{tabular}
%\caption{\nocaption}
    
\z

 Finally, the passive suffix realizes its H in the
              Crastinal Future and Crastinal Future Negative so
              long as the stem is long enough to accommodate the
              melodic H in the penultimate syllable of the stem or
              earlier, just as in the Subjunctive. 

 
\ea\label{ex:xP3xPassive} 
/H/ \& /Ø/ Stems with the
                Passive Suffix \gloss{‘s/he will (not)
                be...’}[SB]


\begin{tabular}{lllll}  
  
                       \textbf{Crast Fut}  &   /H/  &   
                       \vernacular{
                      na-a[khalak-ú-i]}  &   
                       \gloss{‘cut’}  &  \\

                       \vernacular{
                      na-a[tsuunzuú{\downstep}n-ú-i]}  &   
                       \gloss{‘sucked’}  &  \\
/Ø/  &   
                       \vernacular{
                      na-a[lakhuú{\downstep}l-ú-i]}  &   
                       \gloss{‘released’}  &  \\

                       \vernacular{
                      na-a[kalushítsil-ú-i]}  &   
                       \gloss{‘returned
                      for’}  &  \\

                       \textbf{Crast Fut Neg}  &   /H/  &   
                       \vernacular{
                      na-a[khalak-w-í] {\downstep}tá}  &   
                       \gloss{‘cut’}  &  \\

                       \vernacular{
                      na-a[tsuunzuú{\downstep}n-w-í] {\downstep}tá}  &   
                       \gloss{‘sucked’}  &  \\
/Ø/  &   
                       \vernacular{
                      na-a[lakhuú{\downstep}l-w-í] {\downstep}tá}  &   
                       \gloss{‘released’}  &  \\

                       \vernacular{
                      na-a[kalushí{\downstep}tsíl-w-í] {\downstep}tá}  &   
                       \gloss{‘returned
                      for’}  &  \\
\end{tabular}
%\caption{\nocaption}
    
\z



\section{Pattern 4: The initial mora pattern}\label{sec:sPattern4}

The Remote Past and the Remote Past Negative take a
          fourth pattern. Pattern 4 is characterized by a melodic H
          which targets the initial mora of the macrostem. \footnote{\label{fn:nMarloAcknowledgementRemPast} I thank Michael Marlo for improving the analysis of
            melodic H assignment in Pattern 4. 


}%



\subsection{Pattern 4: Remote Past}\label{sec:sPattern4x}

The tonal properties of the Remote Past are detailed
            in this section. The Remote Past is marked by the tense
            prefix \vernacular{aa-}, and
            the FV \vernacular{-a}. In
            addition, in all forms lacking object prefixes, a
            melodic H surfaces on the initial mora of the stem
            regardless of the verb’s tonal class.

 As shown in \REF{ex:xRemPastCH} , /H/ verbs have a H on the initial stem
            mora.

 
\ea\label{ex:xRemPastCH} 
Remote Past C-Initial /H/ \gloss{
              ‘s/he...’}


\begin{tabular}{lllll}  
  Subj  &   Tns  &   Stem  &   Gloss  &  \\

                     \vernacular{y-}  &   
                     \vernacular{aa}  &   
                     \vernacular{
                    \ob [khúa]\cb }  &   
                     \gloss{‘paid dowry’}  &  \\

                     \vernacular{y-}  &   
                     \vernacular{aa}  &   
                     \vernacular{
                    \ob [βéka]\cb }  &   
                     \gloss{‘shaved’}  &  \\

                     \vernacular{y-}  &   
                     \vernacular{aa}  &   
                     \vernacular{
                    \ob [téekha]\cb }  &   
                     \gloss{‘cooked’}  &  \\

                     \vernacular{y-}  &   
                     \vernacular{aa}  &   
                     \vernacular{
                    \ob [khálaka]\cb }  &   
                     \gloss{‘cut’}  &  \\

                     \vernacular{y-}  &   
                     \vernacular{aa}  &   
                     \vernacular{
                    \ob [kálaanga]\cb }  &   
                     \gloss{‘fried’}  &  \\

                     \vernacular{y-}  &   
                     \vernacular{aa}  &   
                     \vernacular{
                    \ob [sáanditsa]\cb }  &   
                     \gloss{‘thanked’}  &  \\

                     \vernacular{y-}  &   
                     \vernacular{aa}  &   
                     \vernacular{
                    \ob [tsúunzuuna]\cb }  &   
                     \gloss{‘sucked’}  &  \\

                     \vernacular{y-}  &   
                     \vernacular{aa}  &   
                     \vernacular{
                    \ob [βóyong’ana]\cb }  &   
                     \gloss{‘went
                    around’}  &  \\
\end{tabular}
%\caption{\nocaption}
    
\z

 Before vowel-initial stems, one mora of the tense
            prefix is deleted. The remaining mora assimilates to
            the quality of the initial vowel of the root. 

 
\ea\label{ex:xRemPastVH} 
Remote Past V-Initial /H/ \gloss{
              ‘s/he...’}


\begin{tabular}{lllll}  
  Subj  &   Tns  &   Stem  &   Gloss  &  \\

                     \vernacular{y-}  &   
                     \vernacular{i}  &   
                     \vernacular{
                    \ob [íra]\cb }  &   
                     \gloss{‘killed’}  &  \\

                     \vernacular{y-}  &   
                     \vernacular{o}  &   
                     \vernacular{
                    \ob [ónoɲɲa]\cb }  &   
                     \gloss{‘spoiled’}  &  \\

                     \vernacular{y-}  &   
                     \vernacular{a}  &   
                     \vernacular{
                    \ob [ábukhanyːa]\cb }  &   
                     \gloss{‘separated’}  &  \\
\end{tabular}
%\caption{\nocaption}
    
\z

 In this context, the contrast between /H/ and /Ø/
            verbs is neutralized. As \REF{ex:xRemPastCØ} and \REF{ex:xRemPastVØ} show, /Ø/ verbs also have a H on the
            initial mora of the stem.

 
\ea\label{ex:xRemPastCØ} 
Remote Past C-Initial /Ø/ \gloss{
              ‘s/he...’}


\begin{tabular}{lllll}  
  Subj  &   Tns  &   Stem  &   Gloss  &  \\

                     \vernacular{y-}  &   
                     \vernacular{aa}  &   
                     \vernacular{
                    \ob [kúa]\cb }  &   
                     \gloss{‘fell’}  &  \\

                     \vernacular{y-}  &   
                     \vernacular{aa}  &   
                     \vernacular{
                    \ob [lékha]\cb }  &   
                     \gloss{‘left’}  &  \\

                     \vernacular{y-}  &   
                     \vernacular{aa}  &   
                     \vernacular{
                    \ob [réeβa]\cb }  &   
                     \gloss{‘asked’}  &  \\

                     \vernacular{y-}  &   
                     \vernacular{aa}  &   
                     \vernacular{
                    \ob [sósana]\cb }  &   
                     \gloss{‘resembled’}  &  \\

                     \vernacular{y-}  &   
                     \vernacular{aa}  &   
                     \vernacular{
                    \ob [lákhuula]\cb }  &   
                     \gloss{‘released’}  &  \\

                     \vernacular{y-}  &   
                     \vernacular{aa}  &   
                     \vernacular{
                    \ob [séeβula]\cb }  &   
                     \gloss{‘said goodbye
                    (to)’}  &  \\

                     \vernacular{y-}  &   
                     \vernacular{aa}  &   
                     \vernacular{
                    \ob [kálushitsa]\cb }  &   
                   \gloss{
                  ‘returned’}[SB] &  \\

                     \vernacular{y-}  &   
                     \vernacular{aa}  &   
                     \vernacular{
                    \ob [síinjilitsa]\cb }  &   
                     \gloss{‘made stand’}  &  \\

                     \vernacular{y-}  &   
                     \vernacular{aa}  &   
                     \vernacular{
                    \ob [séβulukhaɲːa]\cb }  &   
                     \gloss{‘scattered’}  &  \\
\end{tabular}
%\caption{\nocaption}
    
\z

 
\ea\label{ex:xRemPastVØ} 
Remote Past V-Initial /Ø/ \gloss{
              ‘s/he...’}


\begin{tabular}{lllll}  
  Subj  &   Tns  &   Stem  &   Gloss  &  \\

                     \vernacular{y-}  &   
                     \vernacular{e}  &   
                     \vernacular{[énya]}  &   
                     \gloss{‘wanted’}  &  \\

                     \vernacular{y-}  &   
                     \vernacular{e}  &   
                     \vernacular{
                    [éyela]}  &   
                     \gloss{‘wiped for’}  &  \\

                     \vernacular{y-}  &   
                     \vernacular{i}  &   
                     \vernacular{
                    [íluula]}  &   
                     \gloss{‘winnowed’}  &  \\

                     \vernacular{y-}  &   
                     \vernacular{a}  &   
                     \vernacular{
                    [ámbakhana]}  &   
                     \gloss{‘refused’}  &  \\
\end{tabular}
%\caption{\nocaption}
    
\z


\subsubsection{Remote Past with Object Prefixes}\label{sec:sP4xObjects}

A single object prefix surfaces H, when present.
              In /H/ verbs, there is no H on the stem. 

 
\ea\label{ex:xRemPastCHOP} 
Remote Past C-Initial /H/ + OP \gloss{
                ‘s/he...him/her’}[JI] \footnote{\label{fn:nBurulasHabitualPronunciations} SB’s productions of these data all pattern
                  like the segmentally identical Habitual (§ \sectref{sec:sPattern8} ).


}%



\begin{tabular}{llllll}  
  Subj  &   Tns  &   Obj  &   Stem  &   Gloss  &  \\

                       \vernacular{y-}  &   
                       \vernacular{aa}  &   
                       \vernacular{\ob mú}  &   
                       \vernacular{
                      [raa]\cb }  &   
                       \gloss{‘buried’}  &  \\

                       \vernacular{y-}  &   
                       \vernacular{aa}  &   
                       \vernacular{\ob mú}  &   
                       \vernacular{
                      [βeka]\cb }  &   
                       \gloss{‘shaved’}  &  \\

                       \vernacular{y-}  &   
                       \vernacular{aa}  &   
                       \vernacular{\ob mú}  &   
                       \vernacular{
                      [leera]\cb }  &   
                       \gloss{‘brought’}  &  \\

                       \vernacular{y-}  &   
                       \vernacular{aa}  &   
                       \vernacular{\ob mú}  &   
                       \vernacular{
                      [khalaka]\cb }  &   
                       \gloss{‘cut’}  &  \\

                       \vernacular{y-}  &   
                       \vernacular{aa}  &   
                       \vernacular{\ob mú}  &   
                       \vernacular{
                      [βoolitsa]\cb }  &   
                       \gloss{‘seduced’}  &  \\

                       \vernacular{y-}  &   
                       \vernacular{aa}  &   
                       \vernacular{\ob mú}  &   
                       \vernacular{
                      [βoyong’ana]\cb }  &   
                       \gloss{‘went
                      around’}  &  \\
\end{tabular}
%\caption{\nocaption}
    
\z

 
\ea\label{ex:xRemPastVHOP} 
Remote Past V-Initial /H/ + OP \gloss{
                ‘s/he...him/her’}


\begin{tabular}{llllll}  
  Subj  &   Tns  &   Obj  &   Stem  &   Gloss  &  \\

                       \vernacular{y-}  &   
                       \vernacular{aa}  &   
                       \vernacular{\ob mwí}  &   
                       \vernacular{
                      [ira]\cb }  &   
                       \gloss{‘killed’}  &  \\

                       \vernacular{y-}  &   
                       \vernacular{aa}  &   
                       \vernacular{\ob mwó}  &   
                       \vernacular{
                      [ononyːa]\cb }  &   
                       \gloss{‘spoiled’}  &  \\

                       \vernacular{y-}  &   
                       \vernacular{aa}  &   
                       \vernacular{\ob mwá}  &   
                       \vernacular{
                      [abukhanyːa]\cb }  &   
                       \gloss{
                      ‘separated’}  &  \\
\end{tabular}
%\caption{\nocaption}
    
\z

 In /Ø/ verbs, the melodic H surfaces on the first
              and second moras of the stem, rather than just the
              initial as in /Ø/ verbs without an object
              prefix. 

 
\ea\label{ex:xRemPastCØOP} 
Remote Past C-Initial /Ø/ + OP \gloss{
                ‘s/he...him/her’} \footnote{\label{fn:nRemPastTonelessOPproperties} There are some inconsistencies in the primary
                  consultants’ productions of these forms, varying
                  between many tonal variants (e.g., JI produced in
                  one instance the form \vernacular{
                  y-aa\ob mú[{\downstep}lékha]\cb } \gloss{‘s/he left
                  him/her’}rather than the expected \vernacular{
                  y-aa\ob mú[{\downstep}lékhá]\cb }. These
                  transcriptions better reflect the productions of
                  the speakers who participated in my study of
                  micro-variation, who consistently produced these
                  forms with a melodic H on the second stem
                  mora.


}%



\begin{tabular}{llllll}  
  Subj  &   Tns  &   Obj  &   Stem  &   Gloss  &  \\

                       \vernacular{y-}  &   
                       \vernacular{aa}  &   
                       \vernacular{\ob mú}  &   
                       \vernacular{
                      [{\downstep}tsía]\cb }  &   
                       \gloss{‘went
                      (for)’}  &  \\

                       \vernacular{y-}  &   
                       \vernacular{aa}  &   
                       \vernacular{\ob mú}  &   
                       \vernacular{
                      [{\downstep}lékhá]\cb }  &   
                       \gloss{‘left’}  &  \\

                       \vernacular{y-}  &   
                       \vernacular{aa}  &   
                       \vernacular{\ob mú}  &   
                       \vernacular{
                      [{\downstep}lóónda]\cb }  &   
                       \gloss{‘followed’}  &  \\

                       \vernacular{y-}  &   
                       \vernacular{aa}  &   
                       \vernacular{\ob mú}  &   
                       \vernacular{
                      [{\downstep}kúlíkha]\cb }  &   
                       \gloss{‘named’}  &  \\

                       \vernacular{y-}  &   
                       \vernacular{aa}  &   
                       \vernacular{\ob mú}  &   
                       \vernacular{
                      [{\downstep}sééβula]\cb }  &   
                       \gloss{‘said
                      goodbye’}  &  \\

                       \vernacular{y-}  &   
                       \vernacular{aa}  &   
                       \vernacular{\ob mú}  &   
                       \vernacular{
                      [{\downstep}kálúshila]\cb }  &   
                       \gloss{‘defended’}  &  \\
\end{tabular}
%\caption{\nocaption}
    
\z

 
\ea\label{ex:xRemPastVØOP} 
Remote Past V-Initial /Ø/ + OP \gloss{
                ‘s/he...him/her’}


\begin{tabular}{llllll}  
  Subj  &   Tns  &   Obj  &   Stem  &   Gloss  &  \\

                       \vernacular{y-}  &   
                       \vernacular{aa}  &   
                       \vernacular{\ob mwé}  &   
                       \vernacular{
                      [{\downstep}ényá]\cb }  &   
                       \gloss{‘wanted’}  &  \\

                       \vernacular{y-}  &   
                       \vernacular{aa}  &   
                       \vernacular{\ob mwé}  &   
                       \vernacular{
                      [{\downstep}éyéla]\cb }  &   
                       \gloss{‘wiped
                      for’}  &  \\

                       \vernacular{y-}  &   
                       \vernacular{aa}  &   
                       \vernacular{\ob mwá}  &   
                       \vernacular{
                      [{\downstep}ámbákhana]\cb }  &   
                       \gloss{‘refused’}  &  \\
\end{tabular}
%\caption{\nocaption}
    
\z

 The tonal properties of Remote Future forms with
              a 1 \textsuperscript{st}sg object
              prefix are the same as forms with a CV- object
              prefix: the object prefix itself is H, with the
              result that the pre-stem syllable has a rising tone.
              In /H/ verbs, the stem is all L \REF{ex:xRemPastCHOP1sg} , while /Ø/ verbs
              are H through the first two stem moras \REF{ex:xRemPastCØOP1sg} . One mora
              contributed by the tense prefix is deleted.

 
\ea\label{ex:xRemPastCHOP1sg} 
Remote Past C-Initial /H/ + OP \textsubscript{1sg} \gloss{‘s/he...(for)
                me’}


\begin{tabular}{llllll}  
  Subj  &   Tns  &   Obj  &   Stem  &   Gloss  &  \\

                       \vernacular{y-}  &   
                       \vernacular{a}  &   
                       \vernacular{\ob á}  &   
                       \vernacular{
                      [khua]\cb }  &   
                       \gloss{‘paid
                      dowry’}  &  \\

                       \vernacular{y-}  &   
                       \vernacular{a}  &   
                       \vernacular{\ob á}  &   
                       \vernacular{
                      [mbeka]\cb }  &   
                       \gloss{‘shaved’}  &  \\

                       \vernacular{y-}  &   
                       \vernacular{a}  &   
                       \vernacular{\ob á}  &   
                       \vernacular{
                      [ndeera]\cb }  &   
                       \gloss{‘brought’}  &  \\

                       \vernacular{y-}  &   
                       \vernacular{a}  &   
                       \vernacular{\ob á}  &   
                       \vernacular{
                      [khalaka]\cb }  &   
                       \gloss{‘cut’}  &  \\

                       \vernacular{y-}  &   
                       \vernacular{a}  &   
                       \vernacular{\ob á}  &   
                       \vernacular{
                      [mboolitsa]\cb }  &   
                       \gloss{‘seduced’}  &  \\

                       \vernacular{y-}  &   
                       \vernacular{a}  &   
                       \vernacular{\ob á}  &   
                       \vernacular{
                      [mboyong’ana]\cb }  &   
                       \gloss{‘went
                      around’}  &  \\
\end{tabular}
%\caption{\nocaption}
    
\z

 
\ea\label{ex:xRemPastCØOP1sg} 
Remote Past C-Initial /Ø/ + OP \textsubscript{1sg} \gloss{
                ‘s/he...me’}


\begin{tabular}{llllll}  
  Subj  &   Tns  &   Obj  &   Stem  &   Gloss  &  \\

                       \vernacular{y-}  &   
                       \vernacular{a}  &   
                       \vernacular{\ob á}  &   
                       \vernacular{
                      [{\downstep}ndékhá]}  &   
                       \gloss{‘left’}  &  \\

                       \vernacular{y-}  &   
                       \vernacular{a}  &   
                       \vernacular{\ob á}  &   
                       \vernacular{
                      [{\downstep}nóónda]}  &   
                       \gloss{‘followed’}  &  \\

                       \vernacular{y-}  &   
                       \vernacular{a}  &   
                       \vernacular{\ob á}  &   
                       \vernacular{
                      [{\downstep}ngúlíkha]}  &   
                     \gloss{
                    ‘named’}[SB] &  \\

                       \vernacular{y-}  &   
                       \vernacular{a}  &   
                       \vernacular{\ob á}  &   
                       \vernacular{
                      [{\downstep}sééβula]}  &   
                       \gloss{‘said goodbye
                      (to)’}  &  \\

                       \vernacular{y-}  &   
                       \vernacular{a}  &   
                       \vernacular{\ob á}  &   
                       \vernacular{
                      [{\downstep}lákhúula]}  &   
                       \gloss{‘released’}  &  \\

                       \vernacular{y-}  &   
                       \vernacular{a}  &   
                       \vernacular{\ob á}  &   
                       \vernacular{
                      [{\downstep}ngálúhkaɲːa]}  &   
                       \gloss{‘turned
                      over’}  &  \\
\end{tabular}
%\caption{\nocaption}
    
\z

 When CV- and 1 \textsuperscript{st}sg object
              prefixes appear together in the Remote Past, a
              falling tone surfaces on the pre-stem syllable. In
              /H/ verbs, the stem surfaces all L, while /Ø/ verbs
              in the same context surface with a H on the first two
              moras of the stem and the pre-stem mora.

 
\ea\label{ex:xRemPastCHOPOP1sg} 
Remote Past C-Initial /H/ + OP + OP \textsubscript{1sg} \gloss{‘s/he...him/her for
                me’}


\begin{tabular}{lllllll}  
  Subj  &   Tns  &   Obj
                     \textsubscript{CV} &   Obj
                     \textsubscript{1sg} &   Stem  &   Gloss  &  \\

                       \vernacular{y-}  &   
                       \vernacular{aa}  &   
                       \vernacular{\ob mú-}  &   
                       \vernacular{u}  &   
                       \vernacular{
                      [ndeela]\cb }  &   
                       \gloss{‘buried’}  &  \\

                       \vernacular{y-}  &   
                       \vernacular{aa}  &   
                       \vernacular{\ob mú-}  &   
                       \vernacular{u}  &   
                       \vernacular{
                      [mbechela]\cb }  &   
                       \gloss{‘shaved’}  &  \\

                       \vernacular{y-}  &   
                       \vernacular{aa}  &   
                       \vernacular{\ob mú-}  &   
                       \vernacular{u}  &   
                       \vernacular{
                      [ndeerela]\cb }  &   
                       \gloss{‘brought’}  &  \\

                       \vernacular{y-}  &   
                       \vernacular{aa}  &   
                       \vernacular{\ob mú-}  &   
                       \vernacular{u}  &   
                       \vernacular{
                      [khalachila]\cb }  &   
                       \gloss{‘cut’}  &  \\
\end{tabular}
%\caption{\nocaption}
    
\z

 
\ea\label{ex:xRemPastCØOPOP1sg} 
Remote Past C-Initial /Ø/ + OP + OP \textsubscript{1sg} \gloss{‘s/he...him/her for
                me’}[SB]


\begin{tabular}{lllllll}  
  Subj  &   Tns  &   Obj
                     \textsubscript{CV} &   Obj
                     \textsubscript{1sg} &   Stem  &   Gloss  &  \\

                       \vernacular{y-}  &   
                       \vernacular{aa}  &   
                       \vernacular{\ob mú-}  &   
                       \vernacular{{\downstep}ú}  &   
                       \vernacular{
                      [nzííla]\cb }  &   
                       \gloss{‘went
                      (for)’}  &  \\

                       \vernacular{y-}  &   
                       \vernacular{aa}  &   
                       \vernacular{\ob mú-}  &   
                       \vernacular{{\downstep}ú}  &   
                       \vernacular{
                      [ndéshéla]\cb }  &   
                       \gloss{‘left’}  &  \\

                       \vernacular{y-}  &   
                       \vernacular{aa}  &   
                       \vernacular{\ob mú-}  &   
                       \vernacular{{\downstep}ú}  &   
                       \vernacular{
                      [nóóndela]\cb }  &   
                       \gloss{‘followed’}  &  \\

                       \vernacular{y-}  &   
                       \vernacular{aa}  &   
                       \vernacular{\ob mú-}  &   
                       \vernacular{{\downstep}ú}  &   
                       \vernacular{
                      [ndákhúulila]\cb }  &   
                       \gloss{‘released’}  &  \\
\end{tabular}
%\caption{\nocaption}
    
\z

 The Remote Past has the following core tonal
              properties: (i) both /H/ and /Ø/ verbs surface with a
              H on the initial mora of the stem in forms lacking
              object prefixes, (ii) only /Ø/ verbs realize a
              melodic H in forms involving object prefixes, and
              (iii) that melodic H is realized on the first two
              moras of the stem, rather than just the first. 

 In addition, the tonal properties of object
              prefixes in the Remote Past differ considerably from
              the tonal properties of object prefixes in other
              contexts inflected with a melodic H. In particular,
              rather than surfacing L, a single object prefix will
              surface H, and the pre-stem syllable surfaces with a
              falling tone, rather than rising tone, when the verb
              form has two object prefixes. 

 The properties of the Remote Past are summarized
              schematical in \REF{ex:xRemPastSchematic} . As before, the
              position of underlying Hs is underlined, and the
              melodic H, when it appears, is indicated with double
              underlining.

 
\ea\label{ex:xRemPastSchematic} 
A Schematic Representation of the
                Remote Past's Tonal Properties 


\begin{tabular}{lllll}  
    &   \multicolumn{3}{l}{
                       \ul{/H/ Verbs} } &  \\
  &   
                       \textit{Subj + Tns}  &   \multicolumn{2}{l}{
                       \textit{Macrostem} } &  \\
OPsx0  &   
                       \vernacular{y-aa}  &   
                       \vernacular{\ob }  &   
                       \vernacular{[C
                      }  &  \\
OPsx1  &   
                       \vernacular{y-aa}  &   
                       \vernacular{\ob C
                      }  &   
                       \vernacular{[C
                      }  &  \\
OPsx2  &   
                       \vernacular{y-aa}  &   
                       \vernacular{\ob C
                      }  &   
                       \vernacular{[C
                      }  &  \\
  &   \multicolumn{2}{l}{ } &     &  \\
  &   \multicolumn{3}{l}{
                       \textbf{
                        } } &  \\
  &   
                       \textit{Subj + Tns}  &   \multicolumn{2}{l}{
                       \textit{Macrostem} } &  \\
OPsx0  &   
                       \vernacular{y-aa}  &   
                       \vernacular{\ob }  &   
                     \vernacular{[C
                    }\cb  &  \\
OPsx1  &   
                       \vernacular{y-aa}  &   
                       \vernacular{\ob C
                      }  &   
                       \vernacular{[{\downstep}C
                      }  &  \\
OPsx2  &   
                       \vernacular{y-aa}  &   
                       \vernacular{\ob C
                      }  &   
                       \vernacular{[
                      }  &  \\
\end{tabular}
%\caption{\nocaption}
    
\z

 While the tonal patterns of the Remote Past (and
              the Remote Past Negative, § \sectref{sec:sP4xOtherTenses} ) are
              a surface exception to the otherwise general
              observation that Hs are not realized initially within
              the macrostem in constructions inflected with a
              melodic H, I argue that \regel{Initial
              Lowering}applies in all contexts inflected
              with a melodic H, including the Remote Past. The
              general approach is to posit a rule \regel{Macrostem-Initial Melodic H
              Assignment}(henceforth \regel{M-Initial MHA}),
              which follows \regel{Initial Lowering}in
              the derivation and targets the macrostem initial mora
              for melodic H assignment. In this approach, I propose
              that the Remote Past contributes two melodic Hs. \footnote{\label{fn:nOther2MHPatterns} I analyze Patterns 6 (§ \sectref{sec:sPattern6} ) and 7 (§ \sectref{sec:sPattern7} ) as
                having two melodic Hs as well.


}%


 
\ea\label{ex:xMInitialMHA} 
 \regel{Macrostem Initial
                  Melodic H Assignment} 

%\includegraphics[width=\textwidth]{InkScape%20Images/Rules/MInitialMHA.pdf}

\z

 The tonal properties of the Remote Past in forms
              without object prefixes are simply derived with \regel{Initial
              Lowering}followed by \regel{M-Initial MHA}. In
              /H/ verbs (e.g., \vernacular{
              y-aa\ob [khálaka]\cb } \gloss{‘s/he cut’}),
              though the root H is lowered first by \regel{Initial Lowering},
              the melodic H is assigned to the same position by \regel{M-Initial MHA}.

 I analyze the Hs that surface on object prefixes
              as melodic Hs. Though object prefixes are /H/, the
              underlying Hs are lowered by \regel{Initial Lowering}.
              Subsequently, \regel{M-Initial
              MHA}applies to repopulate the object prefix
              with a melodic H. In /H/ verbs, \regel{Meeussen’s Rule}then
              deletes the root H.

 
\ea\label{ex:xDerivRemPastHOP} 
 Derivation,
                  /H/ Rem. Past + OP: \vernacular{
                  y-aa\ob mú[khalaka]\cb } \gloss{‘s/he cut him/her’
                  } 

%\includegraphics[width=\textwidth]{InkScape%20Images/Derivations/DerivRemPastHOP.pdf}

\z

 A second melodic H is assigned to the second stem
              mora in /Ø/ verbs by \regel{Default MHA}. The
              second melodic H then spreads left onto the
              stem-initial mora via \regel{Plateau}.

 
\ea\label{ex:xDerivRemPastØOP} 
 Derivation,
                  /Ø/ Rem. Past + OP: \vernacular{
                  y-aa\ob mú[{\downstep}kúlíkha]\cb } \gloss{‘s/he named
                  him/her’} 

%\includegraphics[width=\textwidth]{InkScape%20Images/Derivations/DerivRemPast0OP.pdf}

\z

 The introduction of a second melodic H in /Ø/
              verbs with an object prefix raises the following
              issue: why does only one melodic H surface in verbs
              without an object prefix? 

 The observation that both /H/ and /Ø/ verbs
              realize the melodic H on the initial mora of the
              stem, rather than the second, can be taken as an
              indication that \regel{M-Initial
              MHA}precedes \regel{Default MHA}in the
              derivation. If we assume this, it is easy to explain
              the failure of the second melodic H to be assigned to
              the second stem mora as resulting from the condition
              on \regel{Default MHA}that the
              target be preceded by a toneless mora.

 In /Ø/ verbs with two object prefixes, \regel{Plateau}extends the
              melodic H span even onto the pre-stem mora.

 
\ea\label{ex:xDerivRemPastØOPx2} 
 Derivation,
                  /Ø/ Rem. Past + OPx2: \vernacular{
                  y-aa\ob mú-{\downstep}ú[kúlíshila]\cb } \gloss{‘s/he named him/her
                  for me’} 

%\includegraphics[width=\textwidth]{InkScape%20Images/Derivations/DerivRemPast0OPx2.pdf}

\z

 In /H/ verbs, \regel{Initial
              Lowering}lowers the /H/ of the leftmost
              object prefix. \regel{M-Initial MHA}then
              replaces the resultant L with a melodic H. The H of
              the second object prefix and the root H are deleted
              by \regel{Meeussen’s Rule},
              which applies iteratively from right to left (as in
              Pattern 1, § \sectref{sec:sP1aObjects} ). \regel{Default MHA}does not
              assign the second melodic H to the second stem mora
              in /H/ verbs because of the condition which prevents
              the rule from applying if its target is preceded by a
              tone bearing mora; at the relevant time, the mora
              preceding the target of \regel{Default MHA}is
              occupied by the root H.

 
\ea\label{ex:xDerivRemPastHOPx2} 
 Derivation,
                  /H/ Rem. Past + OPx2: \vernacular{
                  y-aa\ob mú-u[ndeela]\cb } \gloss{‘s/he buried
                  him/her for me’} 

%\includegraphics[width=\textwidth]{InkScape%20Images/Derivations/DerivRemPastHOPx2.pdf}

\z

  \regel{Plateau}does not
              apply following the iterative application of \regel{Meeussen’s
              Rule}.



\subsubsection{Remote Past: Phrase Medially}\label{sec:sP4xPhraseMed}

The tonal properties of the Remote Past are
              unaffected by the verb's position within its phrase. \regel{H Tone
              Anticipation}does not spread the H of
              H-toned complements onto the stem, as it does in
              parallel tonally uninflected contexts such as the
              Near Future (§ \sectref{sec:sP1aPhraseMed} ). Four
              pairs of /H/ and /Ø/ stems are provided below, half
              with and half without an object prefix. For each
              pair, the first member involves a H-toned complement,
              while the second involves a toneless complement. In
              each case, the stem tonal properties are the same as
              the pre-pausal counterparts.

 
\ea\label{ex:xRemPastPhraseMedial} 
Remote Past Phrase Medially \gloss{‘s/he...(for
                him/her)’}


\begin{tabular}{lllll}  
  
                       %\includegraphics[width=\textwidth]{InkScape%20Images/H%20Stems.svg}
 &   
                       %\includegraphics[width=\textwidth]{InkScape%20Images/No%20OP.svg}
 &   
                       \vernacular{y-aa[rá]
                      {\downstep}mú{\downstep}yáyi}  &   
                       \gloss{‘buried the
                      boy’}  &  \\

                       \vernacular{y-aa[rá]
                      muundu}  &   
                       \gloss{‘buried
                      somebody’}  &  \\
  &     &  \\

                       \vernacular{y-aa[khálaka]
                      mú{\downstep}yáyi}  &   
                       \gloss{‘cut the
                      boy’}  &  \\

                       \vernacular{y-aa[khálaka]
                      muundu}  &   
                       \gloss{‘cut
                      somebody’}  &  \\
  &     &     &  \\

                       %\includegraphics[width=\textwidth]{InkScape%20Images/One%20OP.svg}
 &   
                       \vernacular{y-aa-mú[reela]
                      mú{\downstep}yáyi}  &   
                       \gloss{‘buried the
                      boy’}  &  \\

                       \vernacular{y-aa-mú[reela]
                      muundu}  &   
                       \gloss{‘buried
                      somebody’}  &  \\
  &     &  \\

                       \vernacular{
                      y-aa-mú[khalaka] mú{\downstep}yáyi}  &   
                       \gloss{‘cut the
                      boy’}  &  \\

                       \vernacular{
                      y-aa-mú[khalaka] muundu}  &   
                       \gloss{‘cut
                      somebody’}  &  \\
  &     &     &  \\

                       %\includegraphics[width=\textwidth]{InkScape%20Images/0%20Stems.svg}
 &   
                       %\includegraphics[width=\textwidth]{InkScape%20Images/No%20OP.svg}
 &   
                       \vernacular{y-aa[tsyá]
                      {\downstep}mú{\downstep}yáyi}  &   
                       \gloss{‘went for the
                      boy’}  &  \\

                       \vernacular{y-aa[tsyá]
                      muundu}  &   
                       \gloss{‘went for
                      somebody’}  &  \\
  &     &  \\

                       \vernacular{y-aa[séebula]
                      mú{\downstep}yáyi}  &   
                       \gloss{‘said goodbye to the
                      boy’}  &  \\

                       \vernacular{y-aa[séebula]
                      muundu}  &   
                       \gloss{‘said goodbye to
                      somebody’}  &  \\
  &     &     &  \\

                       %\includegraphics[width=\textwidth]{InkScape%20Images/One%20OP.svg}
 &   
                       \vernacular{
                      y-aa-mú[{\downstep}tsííla] mú{\downstep}yáyi}  &   
                       \gloss{‘went for the
                      boy’}  &  \\

                       \vernacular{
                      y-aa-mú[{\downstep}tsííla] muundu}  &   
                       \gloss{‘went for
                      somebody’}  &  \\
  &     &  \\

                       \vernacular{
                      y-aa-mú[{\downstep}sééβula] mú{\downstep}yáyi}  &   
                       \gloss{‘said goodbye to the
                      boy’}  &  \\

                       \vernacular{
                      y-aa-mú[{\downstep}sééβula] muundu}  &   
                       \gloss{‘said goodbye to
                      somebody’}  &  \\
\end{tabular}
%\caption{\nocaption}
    
\z



\subsubsection{Remote Past: Impact of Subject
              Choice}\label{sec:sP4xSubjects}

Neither tonal class exhibits tonal alternations
              conditioned by the choice of subject. 

 
\ea\label{ex:xSubjRemPastH} 
Subject Choice in the Remote Past
                /H/ \gloss{
                ‘...brought’}[SB]


\begin{tabular}{llll}  
    &   Singular  &   Plural  &  \\
1
                     \textsuperscript{
                    st}Person &   
                       \vernacular{
                      n-aa[léera]}  &   
                       \vernacular{
                      khw-aa[léera]}  &  \\
2
                     \textsuperscript{
                    nd}Person &   
                       \vernacular{
                      w-aa[léera]}  &   
                       \vernacular{
                      mw-aa[léera]}  &  \\
3
                     \textsuperscript{
                    rd}Person &   
                       \vernacular{
                      y-aa[léera]}  &   
                       \vernacular{
                      β-aa[léera]}  &  \\
\end{tabular}
%\caption{\nocaption}
    
\z

 
\ea\label{ex:xSubjRemPastØ} 
Subject Choice in the Remote Past
                /Ø/ \gloss{
                ‘...asked’}[SB]


\begin{tabular}{llll}  
    &   Singular  &   Plural  &  \\
1
                     \textsuperscript{
                    st}Person &   
                       \vernacular{
                      n-aa[réeβa]}  &   
                       \vernacular{
                      khw-aa[réeβa]}  &  \\
2
                     \textsuperscript{
                    nd}Person &   
                       \vernacular{
                      w-aa[réeβa]}  &   
                       \vernacular{
                      mw-aa[réeβa]}  &  \\
3
                     \textsuperscript{
                    rd}Person &   
                       \vernacular{
                      y-aa[réeβa]}  &   
                       \vernacular{
                      β-aa[réeβa]}  &  \\
\end{tabular}
%\caption{\nocaption}
    
\z



\subsubsection{Remote Past: Passives}\label{sec:sP4xPassives}

Passive Hs surface in the Remote Past, as
              illustrated in \REF{ex:xRemPastPassives} . The passive H
              spans from the second stem mora through the penult in
              both /H/ and /Ø/ verbs.

 
\ea\label{ex:xRemPastPassives} 
Remote Past: Passives \gloss{‘s/he
                was...’}[SB]


\begin{tabular}{lllll}  
  \multicolumn{2}{l}{/H/ Stems } &   \multicolumn{2}{l}{/Ø/ Stems } &  \\

                       \vernacular{
                      y-aa[khá{\downstep}lák-ú-a]}  &   
                       \gloss{‘cut’}  &   
                       \vernacular{
                      y-aa[lá{\downstep}khúúl-ú-a]}  &   
                       \gloss{‘released’}  &  \\

                       \vernacular{
                      y-aa[tsú{\downstep}únzúún-ú-a]}  &   
                       \gloss{‘sucked’}  &   
                       \vernacular{
                      y-aa[ká{\downstep}lúshíts-ú-a]}  &   
                       \gloss{‘returned’}  &  \\
\end{tabular}
%\caption{\nocaption}
    
\z

 The tonal properties of the data in \REF{ex:xRemPastPassives} is expect given
              the analysis of passives developed to this point. The
              H in stem initial position in /H/ verbs without an
              object prefix (cf. \vernacular{y-aa[kh
              } \gloss{‘he cut’}) is
              indeed a melodic H, rather than the root H. Recall
              from \sectref{sec:sP2aOtherTenses} that
              the passive H is licensed if the following two
              conditions are met: (i) the verb appears in a context
              inflected with a melodic H and (ii) the melodic H is
              realized on the stem \textit{or}the verb appears in
              a tense with the perfective suffix. The Remote Past
              forms in \REF{ex:xRemPastPassives} all satisfy
              condition (i) and the first clause of condition
              (ii)

 The formulation of \regel{M-Initial MHA},
              reproduced in \REF{ex:xMInitialMHA} , ensures that the melodic H is assigned
              to the initial mora while simultaneously delinking
              the root H. If \regel{M-Initial MHA}were
              prevented from applying by the root H in /H/ verbs
              without an object prefix, the conditions of \regel{Passive H
              Assignment}would not be satisfied. My
              analysis would therefore incorrectly predict that the
              passive H is not realized in this context.



\subsubsection{Pattern 4: Other Verbal Contexts}\label{sec:sP4xOtherTenses}

The Remote Past’s negative counterpart is the only
              other context characterized by the particular tonal
              properties of Pattern 4. 

 
\ea\label{ex:xP4xTenses} 
Other Pattern 4 Verbal
                Contexts 


\begin{tabular}{llll}  
  a.  &   Remote Past Negative  &   
                       \vernacular{SP-aa[ROOT-a]
                      tá(awe)}  &  \\
\end{tabular}
%\caption{\nocaption}
    
\z

 As in the Remote Past, the Remote Past Negative
              realizes a melodic H on the (macro)stem-initial mora
              in both /H/ and /Ø/ verbs in the absence of any
              object prefixes. The melodic H continues to be
              assigned to the macrostem-initial mora in
              constructions with object prefixes, and /Ø/ verbs
              with object prefixes realize an additional melodic H
              on the second stem mora which spreads left via \regel{Plateau}.
              Furthermore, the verb’s position within its phrase
              does not influence stem tone in the Remote Past
              Negative, and passive Hs are licensed in verbs from
              both tonal classes.

 Observe below that both /H/ and /Ø/ verbs realize
              the melodic H on the initial mora of the (macro)stem
              when no object prefix is present. 

 
\ea\label{ex:xP4xHStems} 
Morphologically Simple /H/ Stems
                [SB] \footnote{\label{fn:nP4xGlosses} The examples included in the current section
                  use \vernacular{
                  -khálak-} \gloss{‘cut’}and \vernacular{
                  -khóng’oond-} \gloss{‘knock’}to
                  illustrate the properties of /H/ verbs, and \vernacular{
                  -kulix-} \gloss{‘name’}and \vernacular{
                  -lakhuul-} \gloss{‘release’}as
                  representative of /H/ and /Ø/ verbal roots,
                  respectively. The basic gloss for the Remote Past
                  Negative is \gloss{‘s/he did
                  not...’}.


}%



\begin{tabular}{llllll}  
    &   Subj  &   Tns  &   Stem  &   Neg  &  \\
Rem Past Neg  &   
                       \vernacular{y-}  &   
                       \vernacular{aa}  &   
                       \vernacular{
                      \ob [khálaka]\cb }  &   
                       \vernacular{tá}  &  \\

                       \vernacular{y-}  &   
                       \vernacular{aa}  &   
                       \vernacular{
                      \ob [khóng’oonda]\cb }  &   
                       \vernacular{tá}  &  \\
\end{tabular}
%\caption{\nocaption}
    
\z

 
\ea\label{ex:xP4xØStems} 
Morphologically Simple /Ø/ Stems
                [SB] 


\begin{tabular}{llllll}  
    &   Subj  &   Tns  &   Stem  &   Neg  &  \\
Rem Past Neg  &   
                       \vernacular{y-}  &   
                       \vernacular{aa}  &   
                       \vernacular{
                      \ob [kúlikha]\cb }  &   
                       \vernacular{tá}  &  \\

                       \vernacular{y-}  &   
                       \vernacular{aa}  &   
                       \vernacular{
                      \ob [lákhuula]\cb }  &   
                       \vernacular{tá}  &  \\
\end{tabular}
%\caption{\nocaption}
    
\z

 As in the affirmative, the melodic H is assigned
              to the macrostem-initial mora by \regel{M-Initial MHA}in
              both tonal classes. The macrostem-initial H causes
              the root H to delete via \regel{Meeussen’s Rule},
              but a second melodic H is assigned to the second stem
              mora in /Ø/ verbs by \regel{Default MHA}. The
              second melodic H then spreads left in /Ø/ verbs to
              the stem-initial mora via \regel{Plateau}.

 
\ea\label{ex:xP4xOPHStems} 
/H/ Stems with an Object Prefix
                [SB] 


\begin{tabular}{lllllll}  
    &   Subj  &   Tns  &   Obj  &   Stem  &   Neg  &  \\
Rem Past Neg  &   
                       \vernacular{y-}  &   
                       \vernacular{aa}  &   
                       \vernacular{\ob mú}  &   
                       \vernacular{
                      [khalaka]\cb }  &   
                       \vernacular{tá}  &  \\

                       \vernacular{y-}  &   
                       \vernacular{aa}  &   
                       \vernacular{\ob mú}  &   
                       \vernacular{
                      [khong’oonda]\cb }  &   
                       \vernacular{tá}  &  \\
\end{tabular}
%\caption{\nocaption}
    
\z

 
\ea\label{ex:xP4xOPØStems} 
/Ø/ Stems with an Object Prefix
                [SB] 


\begin{tabular}{lllllll}  
    &   Subj  &   Tns  &   Obj  &   Stem  &   Neg  &  \\
Rem Past Neg  &   
                       \vernacular{y-}  &   
                       \vernacular{aa}  &   
                       \vernacular{\ob mú}  &   
                       \vernacular{
                      [{\downstep}kúlíkha]\cb }  &   
                       \vernacular{tá}  &  \\

                       \vernacular{y-}  &   
                       \vernacular{aa}  &   
                       \vernacular{\ob mú}  &   
                       \vernacular{
                      [{\downstep}lákhúula]\cb }  &   
                       \vernacular{tá}  &  \\
\end{tabular}
%\caption{\nocaption}
    
\z

 As in the affirmative, the Remote Past Negative
              is unaffected by the verb’s position within the
              phrase. The tonal melody persists, and complement Hs
              are not anticipated into the verb stem. Examples are
              provided below of both /H/ and /Ø/ stems before a
              H-toned noun, \vernacular{
              mú{\downstep}yáyi} \gloss{‘boy’}, and a
              toneless noun \vernacular{muundu} \gloss{
              ‘person/somebody’}.

 
\ea\label{ex:xP4xPhraseMed} 
Tenses Like the Remote Past Phrase
                Medially [SB] 


\begin{tabular}{llll}  
  
                       \textbf{Rem Past Neg}  &   
                    /H/  &   
                       \vernacular{y-aa[khálaka]
                      mú{\downstep}yáyi}  &  \\

                       \vernacular{y-a[khálaka]
                      muundu}  &  \\
  &     &  \\

                    /Ø/  &   
                       \vernacular{y-aa[lákhuula]
                      mú{\downstep}yáyi}  &  \\

                       \vernacular{y-aa[lákhuula]
                      muundu}  &  \\
\end{tabular}
%\caption{\nocaption}
    
\z

 No Remote Past Negative data is available that
              directly bears on whether the choice of subject
              impacts stem tone in this context, though data from
              the affirmative Remote Past (§ \sectref{sec:sP4xSubjects} ) suggest
              that no subject-induced tonal alternations should be
              expected here.

 Finally, observe that the passive suffix realizes
              its H in the Remote Past Negative. 

 
\ea\label{ex:xP4xPassive} 
/H/ \& /Ø/ Stems with the
                Passive Suffix \gloss{‘s/he was
                not...’}[SB]


\begin{tabular}{lllll}  
  
                       \textbf{Rem Past Neg}  &   /H/  &   
                       \vernacular{
                      y-aa[khá{\downstep}lák-w-á] {\downstep}tá}  &   
                       \gloss{‘cut’}  &  \\

                       \vernacular{
                      y-aa[tsú{\downstep}únzúún-w-á] {\downstep}tá}  &   
                       \gloss{‘sucked’}  &  \\
/Ø/  &   
                       \vernacular{
                      y-aa[lá{\downstep}khúúl-w-á] {\downstep}tá}  &   
                       \gloss{‘released’}  &  \\

                       \vernacular{
                      y-aa[ká{\downstep}lúshítsíl-w-á] {\downstep}tá}  &   
                       \gloss{‘returned
                      for’}  &  \\
\end{tabular}
%\caption{\nocaption}
    
\z



\section{Pattern 5: The third / final vowel
          patterns}\label{sec:sPattern5}

Pattern 5 groups three similar but distinct tonal
          melodies. The Present and related constructions select
          Pattern 5a. In Pattern 5a, all moras of the third stem
          syllable are H in /H/ verbs and the second stem mora is H
          in /Ø/ verbs. The Indefinite Future takes Pattern 5b. /Ø/
          verbs have a H on the second stem mora this sub-pattern
          as well, but /H/ verbs are H on the FV instead of the
          third syllable. Finally, the Conditional selects Pattern
          5c. Pattern 5c is similar to Pattern 5b, but different in
          several important ways. One notable difference is that
          subject prefixes are H in Pattern 5c, but not in Pattern
          5b. 

 The choice to subsume these three melodies under a
          single primary pattern is motivated primarily by the
          following considerations: (i) the surface tonal
          properties of these melodies are identical in many
          paradigms, (ii) all the melodies exhibit \regel{Initial Lowering}and
          interact with the resulting L in ways that other melodies
          do not, (iii) the Present (5a) and the Indefinite Future
          (5b) both lose the melodic H in a phrase-medial context
          (though the Conditional (5c) does not) and (iv) there is
          speaker-internal variation with respect to which melody
          is selected for the Present and Indefinite Future, with
          the Indefinite Future forms occasionally taking on the
          properties of the Present. \footnote{\label{fn:nPresIndefFutDIachrony} Many other Luhya varieties have the same melody for
            both the Present and Indefinite Future (Khayo, \citealt{rMarlo2009b} ;
            Tura, \citealt{rMarlo2008b} ;
            Nyala-West, \citealt{rMarlo2007} ;
            Bukusu, \citealt{rMutonyi2000} ;
            Marachi, \citealt{rMarlo2007} ;
            Wanga, \citealt{rEbarbGreenMarlo2014} ; Tachoni, \citealt{rOdden2009} ;
            Logoori, \citealt{rLeung1991} ).
            These constructions also share the same core tonal
            properties in Tiriki, but they differ with respect to
            their behavior phrase-medially ( \citealt{rMarloInPrepB} ).


}%



\subsection{Pattern 5a: Present}\label{sec:sPattern5a}

I this section, I use data from the Present to
            illustrate the tonal properties of Pattern 5a. The
            Present is not marked by any tense prefix, but takes
            the imperfective \vernacular{
            -aang}suffix and the FV \vernacular{-a}.
            Tonally, the Present is marked by a melodic H which
            surfaces on the second stem mora in /Ø/ stems and the
            third stem syllable in /H/ stems.

 As shown in \REF{ex:xPresCH} , /H/ verbs have a H on the third stem
            syllable. In verbs with just three syllables, \footnote{\label{fn:nMinimumPres} Aspects of the morphology preclude the appearance
              of mono- and disyllabic stem shapes. The imperfective
              suffix \vernacular{
              -aang}adds a syllable, and monosyllabic
              roots additionally take a semantically null stem
              extender \vernacular{
              -ets/-its}( \citealt{rMarlo2006} ).


}%
\footnote{\label{fn:nSpreadvsShiftinPresent} There is some variation, both within and between
              speakers, with respect to whether the melodic H is
              realized on both the final and penultimate syllable
              in trisyllabic stems or just the penult. Speakers
              appear to favor productions with the melodic H on
              both syllables, but it is not uncommon for the final
              syllable a sharp falling contour that is atypical of
              short final syllables with a H. 


}%


 
\ea\label{ex:xPresCH} 
Present C-Initial /H/ \gloss{
              ‘s/he...’}


\begin{tabular}{llll}  
  Subj  &   Stem  &   Gloss  &  \\

                     \vernacular{a-}  &   
                     \vernacular{
                    \ob [βekaángá]\cb }  &   
                     \gloss{‘shaves’}  &  \\

                     \vernacular{a-}  &   
                     \vernacular{
                    \ob [reetsáángá]\cb }  &   
                     \gloss{‘buries’}  &  \\

                     \vernacular{a-}  &   
                     \vernacular{
                    \ob [teekháángá]\cb }  &   
                     \gloss{‘cooks’}  &  \\

                     \vernacular{a-}  &   
                     \vernacular{
                    \ob [khalakáánga]\cb }  &   
                     \gloss{‘cuts’}  &  \\

                     \vernacular{a-}  &   
                     \vernacular{
                    \ob [kalaangáánga]\cb }  &   
                     \gloss{‘fries’}  &  \\

                     \vernacular{a-}  &   
                     \vernacular{
                    \ob [βoolitsáánga]\cb }  &   
                     \gloss{‘seduces’}  &  \\

                     \vernacular{a-}  &   
                     \vernacular{
                    \ob [βoyong’ánáanga]\cb }  &   
                     \gloss{‘goes
                    around’}  &  \\
\end{tabular}
%\caption{\nocaption}
    
\z

 In most longer stems, the melodic H surfaces just
            on the third syllable of the stem, but exceptionally
            long verbs like \vernacular{
            a\ob [βoyong’ánáanga]\cb } \gloss{‘s/he goes
            around’}show that the following mora can be H
            as well.

 Vowel-initial stems also have a H on the third
            syllable. When considering the longer two data, bear in
            mind that the geminate nasals derive from an
            intermediate representation in which a high front vowel
            intervenes between two nasals, e.g. \vernacular{
            y\ob [onoɲíɲáanga]\cb }for \gloss{‘s/he
            spoils’}.

 
\ea\label{ex:xPresVH} 
Present V-Initial /H/ \gloss{
              ‘s/he...’}


\begin{tabular}{llll}  
  Subj  &   Stem  &   Gloss  &  \\

                     \vernacular{y-}  &   
                     \vernacular{\ob [iraángá]\cb 
                    }  &   
                     \gloss{‘kills’}  &  \\

                     \vernacular{y-}  &   
                     \vernacular{
                    \ob [onoɲ́ːáanga]\cb }  &   
                     \gloss{‘spoils’}  &  \\

                     \vernacular{y-}  &   
                     \vernacular{
                    \ob [abukháɲ́ːaanga]\cb }  &   
                     \gloss{‘separates’}  &  \\
\end{tabular}
%\caption{\nocaption}
    
\z

 /Ø/ stems realize a melodic H on the second stem
            mora regardless of the size of the verb stem.
            Additionally, the melodic H spreads one mora to the
            right via \regel{Pre-Penultimate
            Doubling} \REF{ex:xPrePenultimateDoubling} when at
            least one mora intervenes between the second and
            penultimate moras of the stem.

 
\ea\label{ex:xPresCØ} 
Present C-Initial /Ø/ \gloss{
              ‘s/he...’}


\begin{tabular}{llll}  
  Subj  &   Stem  &   Gloss  &  \\

                     \vernacular{a-}  &   
                     \vernacular{
                    \ob [lekháanga]\cb }  &   
                     \gloss{‘leaves’}  &  \\

                     \vernacular{a-}  &   
                     \vernacular{
                    \ob [kwiítsáanga]\cb }  &   
                     \gloss{‘falls’}  &  \\

                     \vernacular{a-}  &   
                     \vernacular{
                    \ob [reéβáanga]\cb }  &   
                     \gloss{‘asks’}  &  \\

                     \vernacular{a-}  &   
                     \vernacular{
                    \ob [kulíkháanga]\cb }  &   
                   \gloss{
                  ‘names’}[SB] &  \\

                     \vernacular{a-}  &   
                     \vernacular{
                    \ob [lakhúúlaanga]\cb }  &   
                     \gloss{‘releases’}  &  \\

                     \vernacular{a-}  &   
                     \vernacular{
                    \ob [seéβúlaanga]\cb }  &   
                     \gloss{‘says goodbye
                    (to)’}  &  \\

                     \vernacular{a-}  &   
                     \vernacular{
                    \ob [kalúshítsaanga]\cb }  &   
                   \gloss{
                  ‘returns’}[SB] &  \\
\end{tabular}
%\caption{\nocaption}
    
\z

 
\ea\label{ex:xPresVØ} 
Present V-Initial /Ø/ \gloss{
              ‘s/he...’}


\begin{tabular}{llll}  
  Subj  &   Stem  &   Gloss  &  \\

                     \vernacular{y-}  &   
                     \vernacular{
                    \ob [enyáanga]\cb }  &   
                     \gloss{‘wants’}  &  \\

                     \vernacular{y-}  &   
                     \vernacular{
                    \ob [eyélaanga]\cb }  &   
                     \gloss{‘wipes for’}  &  \\

                     \vernacular{y-}  &   
                     \vernacular{
                    \ob [ambákhánaanga]\cb }  &   
                     \gloss{‘refuses’}  &  \\

                     \vernacular{y-}  &   
                     \vernacular{
                    \ob [eléélitsaanga]\cb }  &   
                     \gloss{‘hangs up
                    (s.t.)’}  &  \\
\end{tabular}
%\caption{\nocaption}
    
\z


\subsubsection{Present with Object Prefixes}\label{sec:sP5aObjects}

When an object prefix is present in /H/ verbs, the
              object prefix is L, the root H is on the initial
              mora, and the melodic H spans the second mora through
              the third syllable. 

 
\ea\label{ex:xPresCHOP} 
Present C-Initial /H/ + OP \gloss{
                ‘s/he...him/her’}


\begin{tabular}{lllll}  
  Subj  &   Obj  &   Stem  &   Gloss  &  \\

                       \vernacular{a-}  &   
                       \vernacular{\ob mu}  &   
                       \vernacular{
                      [ré{\downstep}étsáángá]\cb }  &   
                       \gloss{‘buries’}  &  \\

                       \vernacular{a-}  &   
                       \vernacular{\ob mu}  &   
                       \vernacular{
                      [βé{\downstep}káángá]\cb }  &   
                       \gloss{‘shaves’}  &  \\

                       \vernacular{a-}  &   
                       \vernacular{\ob mu}  &   
                       \vernacular{
                      [lé{\downstep}éráángá]\cb }  &   
                       \gloss{‘brings’}  &  \\

                       \vernacular{a-}  &   
                       \vernacular{\ob mu}  &   
                       \vernacular{
                      [khá{\downstep}lákáánga]\cb }  &   
                       \gloss{‘cuts’}  &  \\

                       \vernacular{a-}  &   
                       \vernacular{\ob mu}  &   
                       \vernacular{
                      [βó{\downstep}ólítsáánga]\cb }  &   
                       \gloss{‘seduces’}  &  \\

                       \vernacular{a-}  &   
                       \vernacular{\ob mu}  &   
                       \vernacular{
                      [βó{\downstep}yóng’áánga]\cb }  &   
                       \gloss{‘goes
                      around’}  &  \\
\end{tabular}
%\caption{\nocaption}
    
\z

 
\ea\label{ex:xPresVHOP} 
Present V-Initial /H/ + OP \gloss{
                ‘s/he...him/her’}


\begin{tabular}{lllll}  
  Subj  &   Obj  &   Stem  &   Gloss  &  \\

                       \vernacular{a-}  &   
                       \vernacular{\ob mw}  &   
                       \vernacular{
                      [ií{\downstep}ráángá]\cb }  &   
                       \gloss{‘kills’}  &  \\

                       \vernacular{a-}  &   
                       \vernacular{\ob mw}  &   
                       \vernacular{
                      [oó{\downstep}nóɲːáánga]\cb }  &   
                       \gloss{‘spoils’}  &  \\

                       \vernacular{a-}  &   
                       \vernacular{\ob mw}  &   
                       \vernacular{
                      [aá{\downstep}βúkháɲːaanga]\cb }  &   
                       \gloss{
                      ‘separates’}  &  \\
\end{tabular}
%\caption{\nocaption}
    
\z

 In /Ø/ verbs, the object prefix is L and the
              melodic H surfaces on the second stem mora. 

 
\ea\label{ex:xPresCØOP} 
Present C-Initial /Ø/ + OP \gloss{
                ‘s/he...him/her’}


\begin{tabular}{lllll}  
  Subj  &   Obj  &   Stem  &   Gloss  &  \\

                       \vernacular{a-}  &   
                       \vernacular{\ob mu}  &   
                       \vernacular{
                      [tsiítsaanga]\cb }  &   
                       \gloss{‘goes
                      (for)’}  &  \\

                       \vernacular{a-}  &   
                       \vernacular{\ob mu}  &   
                       \vernacular{
                      [lekháanga]\cb }  &   
                       \gloss{‘leaves’}  &  \\

                       \vernacular{a-}  &   
                       \vernacular{\ob mu}  &   
                       \vernacular{
                      [loóndaanga]\cb }  &   
                       \gloss{‘follows’}  &  \\

                       \vernacular{a-}  &   
                       \vernacular{\ob mu}  &   
                       \vernacular{
                      [kulíkhaanga]\cb }  &   
                       \gloss{‘names’}  &  \\

                       \vernacular{a-}  &   
                       \vernacular{\ob mu}  &   
                       \vernacular{
                      [lakhúulaanga]\cb }  &   
                       \gloss{‘releases’}  &  \\

                       \vernacular{a-}  &   
                       \vernacular{\ob mu}  &   
                       \vernacular{
                      [seéβúlaanga]\cb }  &   
                       \gloss{‘says
                      goodbye’}  &  \\

                       \vernacular{a-}  &   
                       \vernacular{\ob mu}  &   
                       \vernacular{
                      [kalúshítsaanga]\cb }  &   
                     \gloss{
                    ‘returns’}[SB] &  \\
\end{tabular}
%\caption{\nocaption}
    
\z

 
\ea\label{ex:xPresVØOP} 
Present V-Initial /Ø/ + OP \gloss{
                ‘s/he...him/her’}


\begin{tabular}{lllll}  
  Subj  &   Obj  &   Stem  &   Gloss  &  \\

                       \vernacular{a-}  &   
                       \vernacular{\ob mw}  &   
                       \vernacular{
                      [eenyáanga]\cb }  &   
                       \gloss{‘kills’}  &  \\

                       \vernacular{a-}  &   
                       \vernacular{\ob mw}  &   
                       \vernacular{
                      [eeyélaanga]\cb }  &   
                       \gloss{‘spoils’}  &  \\

                       \vernacular{a-}  &   
                       \vernacular{\ob mw}  &   
                       \vernacular{
                      [aambákhánaanga]\cb }  &   
                       \gloss{
                      ‘separates’}  &  \\

                       \vernacular{a-}  &   
                       \vernacular{\ob mw}  &   
                       \vernacular{
                      [eeléélitsaanga]\cb }  &   
                       \gloss{‘holds
                      hangingly’}  &  \\
\end{tabular}
%\caption{\nocaption}
    
\z

 When a second object prefix is added to /H/
              verbs, the long pre-stem syllable bears a rising tone
              and a downstepped level H spans the first through the
              third stem syllables, except in stems with more than
              four syllables, in which the H spans into the fourth
              syllable. 

 
\ea\label{ex:xPresCHOPOP1sg} 
Present C-Initial /H/ + OP + OP \textsubscript{1sg} \gloss{‘s/he...him/her for
                me’}


\begin{tabular}{llllll}  
  Subj  &   Obj
                     \textsubscript{CV} &   Obj
                     \textsubscript{1sg} &   Stem  &   Gloss  &  \\

                       \vernacular{a-}  &   
                       \vernacular{\ob mu-}  &   
                       \vernacular{ú}  &   
                       \vernacular{
                      [{\downstep}ndééláángá]\cb }  &   
                       \gloss{‘buries’}  &  \\

                       \vernacular{a-}  &   
                       \vernacular{\ob mu-}  &   
                       \vernacular{ú}  &   
                       \vernacular{
                      [{\downstep}mbéchéláánga]\cb }  &   
                       \gloss{‘shaves’}  &  \\

                       \vernacular{a-}  &   
                       \vernacular{\ob mu-}  &   
                       \vernacular{ú}  &   
                       \vernacular{
                      [{\downstep}ndééréláánga]\cb }  &   
                       \gloss{‘brings’}  &  \\

                       \vernacular{a-}  &   
                       \vernacular{\ob mu-}  &   
                       \vernacular{ú}  &   
                       \vernacular{
                      [{\downstep}mbóólítsíláanga]\cb }  &   
                       \gloss{‘seduces’}  &  \\

                       \vernacular{a-}  &   
                       \vernacular{\ob mu-}  &   
                       \vernacular{ú}  &   
                       \vernacular{
                      [{\downstep}mbóhólólélaanga]\cb }  &   
                       \gloss{‘unties’}  &  \\
\end{tabular}
%\caption{\nocaption}
    
\z

 In /Ø/ verbs, the downstepped H on the stem spans
              from the first mora through the third mora, except
              when the third mora is the second half of a long
              second syllable. In this case, the melodic H spans
              only through the second stem mora. 

 
\ea\label{ex:xPresCØOPOP1sg} 
Present C-Initial /Ø/ + OP + OP \textsubscript{1sg} \gloss{‘s/he...him/her for
                me’}[SB]


\begin{tabular}{llllll}  
  Subj  &   Obj
                     \textsubscript{CV} &   Obj
                     \textsubscript{1sg} &   Stem  &   Gloss  &  \\

                       \vernacular{a-}  &   
                       \vernacular{\ob mu-}  &   
                       \vernacular{ú}  &   
                       \vernacular{
                      [{\downstep}nzííláanga]\cb }  &   
                       \gloss{‘goes
                      (for)’}  &  \\

                       \vernacular{a-}  &   
                       \vernacular{\ob mu-}  &   
                       \vernacular{ú}  &   
                       \vernacular{
                      [{\downstep}ndéshéláanga]\cb }  &   
                       \gloss{‘leaves’}  &  \\

                       \vernacular{a-}  &   
                       \vernacular{\ob mu-}  &   
                       \vernacular{ú}  &   
                       \vernacular{
                      [{\downstep}nóóndélaanga]\cb }  &   
                       \gloss{‘follows’}  &  \\

                       \vernacular{a-}  &   
                       \vernacular{\ob mu-}  &   
                       \vernacular{ú}  &   
                       \vernacular{
                      [{\downstep}ndákhúulilaanga]\cb }  &   
                       \gloss{‘releases’}  &  \\
\end{tabular}
%\caption{\nocaption}
    
\z

 The core properties of the melody characterizing
              the Present may be summarized as follows: (i)
              underlying macrostem-initial Hs fail to surface, (ii)
              the melodic H surfaces on the second stem mora in /Ø/
              verbs, (iii) the melodic H surfaces on all moras of
              the third syllable in /H/ verbs, (iv) the melodic H
              spreads leftward through the second stem syllable in
              /H/ verbs with a single object prefix, and (v) the
              melodic H spreads leftward through the initial
              syllable of the stem in /H/ verbs with two object
              prefixes. These properties are summarized
              schematically in the following display. In addition
              to these properties, I will offer an analysis of the
              fact that the melodic H spreads left into the second
              stem syllable in trisyllabic /H/ stems, even when no
              object prefix is present. 

 
\ea\label{ex:xPresSchematic} 
A Schematic Representation of the
                Present’s Tonal Properties 


\begin{tabular}{lllll}  
    &   \multicolumn{3}{l}{
                       \ul{/H/ Verbs} } &  \\
  &   
                       \textit{Subj + Tns}  &   \multicolumn{2}{l}{
                       \textit{Macrostem} } &  \\
OPsx0  &   
                       \vernacular{a-li}  &   
                       \vernacular{\ob }  &   
                       \vernacular{[C
                      }  &  \\
OPsx1  &   
                       \vernacular{a-li-}  &   
                       \vernacular{\ob C
                      }  &   
                       \vernacular{[C
                      }  &  \\
OPsx2  &   
                       \vernacular{a-li-}  &   
                       \vernacular{\ob C
                      }  &   
                       \vernacular{[CVCVC
                      }  &  \\
  &     &     &     &  \\
  &   \multicolumn{3}{l}{
                       \textbf{
                        } } &  \\
  &   
                       \textit{Subj + Tns}  &   \multicolumn{2}{l}{
                       \textit{Macrostem} } &  \\
OPsx0  &   
                       \vernacular{a-li}  &   
                       \vernacular{\ob }  &   
                     \vernacular{[CV(C)
                    }\cb  &  \\
OPsx1  &   
                       \vernacular{a-li-}  &   
                       \vernacular{\ob C
                      }  &   
                       \vernacular{[CV(C)
                      }  &  \\
OPsx2  &   
                       \vernacular{a-li-}  &   
                       \vernacular{\ob C
                      }  &   
                       \vernacular{[{\downstep}C
                      }  &  \\
\end{tabular}
%\caption{\nocaption}
    
\z

 That macrostem-initial Hs fail to surface is
              accounted for as in previously discussed melodies via \regel{Initial Lowering} \REF{ex:xInitialLowering} .

 Two rules of melodic H assignment are invoked in
              the analysis of the tonal properties of verbs in the
              Present: \regel{Third Syllable
              MHA}and \regel{Default MHA} \REF{ex:xDefaultMHA} . \regel{Third Syllable
              MHA}follows \regel{Default MHA}in the
              derivation. This ordering captures the fact that,
              while the melodic H is assigned directly to the
              second mora in forms involving /Ø/ verbs, assignment
              by \regel{Default MHA}is
              blocked in /H/ verbs owing to the condition on \regel{Default MHA}that the
              target of melodic H assignment be preceded by a
              toneless mora. The melodic H is then assigned in /H/
              verbs by the later applying \regel{Third Syllable MHA},
              formalized below.

 
\ea\label{ex:xThirdSyllableMHA} 
 \regel{Third Syllable
                  Melodic H Assignment} 

%\includegraphics[width=\textwidth]{InkScape%20Images/Rules/ThirdSyllableMHA.pdf}

\z

 The tonal properties of all /Ø/ verbs and /H/
              verbs with more than three syllables are accounted
              for by \regel{Third Syllable MHA, 
              }, and \regel{Pre-Penultimate
              Doubling} \REF{ex:xPrePenultimateDoubling} .

 In /H/ verbs, \regel{Third Syllable
              MHA}assigns the melodic H to the third
              syllable of the stem. The melodic H then doubles onto
              the next mora so long as it precedes the penultimate
              mora by \regel{Pre-Penultimate
              Doubling}.

 
\ea\label{ex:xDerivPresH} 
 Derivation,
                  /H/ Present: \vernacular{
                  a\ob [βoyong’ánáanga]\cb } \gloss{‘s/he goes
                  around’} 

%\includegraphics[width=\textwidth]{InkScape%20Images/Derivations/DerivPresH.pdf}

\z

 In /Ø/ verbs, the melodic H is assigned by \regel{Default MHA},
              bleeding \regel{Third Syllable MHA}.
              The melodic H then spreads to the right by \regel{Pre-Penultimate
              Doubling}.

 
\ea\label{ex:xDerivPresØ} 
 Derivation,
                  /Ø/ Present: \vernacular{
                  a\ob [reéβáanga]\cb } \gloss{‘s/he
                  asks’} 

%\includegraphics[width=\textwidth]{InkScape%20Images/Derivations/DerivPres0.pdf}

\z

 Trisyllabic /H/ stems, in which the melodic H
              spreads left to the penultimate syllable, motivate an
              additional rule of leftward spreading, \regel{Final Spread}.

 
\ea\label{ex:xFinalSpread} 
 \regel{Final
                  Spread} 

%\includegraphics[width=\textwidth]{InkScape%20Images/Rules/FinalSpread.pdf}

\z

 The melodic H spreads through both moras of the
              penultimate syllable in /H/ stems with long initial
              syllables, e.g. \vernacular{
              a\ob [teekháángá]\cb } \gloss{‘s/he cooks’}.
              The melodic H spreads only through the rightmost mora
              of the penult in /H/ stems with short initial
              syllables, e.g. \vernacular{
              a\ob [bekaángá]\cb } \gloss{‘s/he
              shaves’}. The rule posited above does not
              predict this difference by itself.

 The present work analyzes this difference as
              arising from a rule of \regel{L Spread II}, which
              spreads the Ls that result from \regel{Initial
              Lowering}onto the peninitial (second) mora.
              The result of \regel{L Spread II}is that
              the first two moras of trisyllabic stems are L before
              the melodic H. \footnote{\label{fn:nSecondLSpreadRule} Note that this is the second rule posited which
                spreads a L; the first was introduced in the
                analysis of subject-induced tonal effects in the
                lexical pattern, whereby L-toned 1 \textsuperscript{st}and 2 \textsuperscript{nd}person
                subjects conditioned the lowering of root and
                object prefix Hs. While \regel{L Spread
                I}iteratively spreads prefixal Ls up to
                and including the stem-initial mora, \regel{L Spread
                II}spreads just macrostem-initial Ls onto
                a toneless peninitial mora


}%


 
\ea\label{ex:xLSpreadII} 
 \regel{L Spread
                  II} 

%\includegraphics[width=\textwidth]{InkScape%20Images/Rules/LSpreadII.pdf}

\z

 The formulation of \regel{Final Spread}in \REF{ex:xFinalSpread} calls for a toneless target. The full
              extension of \regel{Final Spread}applies
              when the initial span of two L moras generated by \regel{L Spread II}is fully
              contained within the first syllable, as it is when
              the initial syllable of the stem is long. One form
              like this is derived in \REF{ex:xDerivPresHTrisyllabicLong} ; \regel{Default MHA}and \regel{Pre-Penultimate
              Doubling}, which do not apply in this case,
              are excluded for space.

 
\ea\label{ex:xDerivPresHTrisyllabicLong} 
 Derivation,
                  /H/ Trisyllabic Present: \vernacular{
                  a\ob [reetsáángá]\cb } \gloss{‘s/he
                  asks’} 

%\includegraphics[width=\textwidth]{InkScape%20Images/Derivations/DerivPresHTrisyllabicLong.pdf}

\z

 When the initial syllable is short, \regel{L Spread II}results
              in an initial L span which extends into the
              penultimate syllable. This prevents the full
              extension of \regel{Final Spread}from
              applying, so instead the smaller extension of \regel{Final Spread}spreads
              the melodic H onto just the second mora of the long
              penultimate syllable, creating a rise. This is shown
              in \REF{ex:xDerivPresHTrisyllabicShort} .

 
\ea\label{ex:xDerivPresHTrisyllabicShort} 
 Derivation,
                  /H/ Trisyllabic Present: \vernacular{
                  a\ob [βekaángá]\cb } \gloss{‘s/he
                  shaves’} 

%\includegraphics[width=\textwidth]{InkScape%20Images/Derivations/DerivPresHTrisyllabicShort.pdf}

\z

 Finally, recall that the melodic H surfaces not
              only on the third stem syllable in /H/ verbs with
              object prefixes, but also on the second syllable, as
              in \vernacular{
              a\ob mu[βé{\downstep}káángá]\cb } \gloss{‘s/he shaves
              him/her’}, and even the initial syllable,
              as in \vernacular{
              a\ob mu-ú[{\downstep}mbéchéláánga]\cb } \gloss{‘s/he shaves him/her for
              me’}. Both instances may be analyzed as
              outcomes of a rule of \regel{Plateau}, as
              originally formalized in \REF{ex:xPlateau} . The non-applying rule \regel{Final Spread}is
              omitted from this derivation.

 
\ea\label{ex:xDerivPresHOPx2} 
 Derivation,
                  /H/ Present OPx2: \vernacular{
                  a\ob mu-ú[{\downstep}mbéchéláánga]\cb } \gloss{‘s/he shaves
                  him/her for me’} 

%\includegraphics[width=\textwidth]{InkScape%20Images/Derivations/DerivPresHOPx2.pdf}

\z



\subsubsection{Present: Phrase Medially}\label{sec:sP5aPhraseMed}

The melodic H is lost in the Present when the verb
              is non-final within the phrase. Compare, for example
              phrase-final \vernacular{
              a\ob [kulíkháanga]\cb } \gloss{‘s/he
              names’}and \vernacular{a\ob [kulikhaanga]\cb 
              muundu} \gloss{‘s/he names someone’.
              Interestingly}, the effect of \regel{Initial
              Lowering}persists in a phrase-medial
              context, cf. /H/-toned \vernacular{
              a\ob [khalakáánga]\cb } \gloss{‘s/he cuts’ and 
              } \gloss{‘s/he cuts
              somebody’}.

 Additionally, \regel{H Tone
              Anticipation}spreads the H from H-toned
              complements onto the verbal stem. This H appears to
              spread a greater distance in /Ø/ verbs than it does
              in /H/ verbs, though the left edge of the span is
              unclear in both cases, so the transcriptions in \REF{ex:xPresPhraseMedial} should be
              considered approximations only and verified for
              accuracy once \regel{H Tone
              Anticipation}in Idakho is better
              understood. \footnote{\label{fn:nLEdgeUncertainty} Though the left edge of the H span is fuzzy, it
                is very clear that the melodic H is lost
                phrase-medially and that post-verbal Hs spread into
                the verb stem by \regel{H Tone
                Anticipation}. The disclaimers here
                regarding the left edge of H spans resulting from \regel{H Tone
                Anticipation}do not also apply to the
                left edge of H spans resulting from \regel{Final Spread}. The
                left edge of spreading is very clear in cases of \regel{Final Spread}.


}%


 In /H/ verbs, the span of the complement H extends
              to and includes the peninitial or post-peninitial
              mora in forms without an object prefix. In verbs with
              especially long stems, the left edge of the H-span
              more clearly ends at the third stem mora. When an
              object prefix is present, the complement H spreads
              leftwards up to and including the second stem
              syllable. 

 In /Ø/ verbs, the span extends onto the
              stem-initial syllable or beyond, perhaps even
              reaching onto the subject prefix. The span may even
              reach the subject prefix in forms with an object
              prefix, though in the transcriptions below, the left
              edge of the span is given a conservative estimate as
              terminating at the stem-initial syllable. 

 Four pairs of /H/ and /Ø/ stems are provided
              below, half with and half without an object prefix.
              For each pair, the first member involves a H-toned
              complement, while the second involves a toneless
              complement. 

 
\ea\label{ex:xPresPhraseMedial} 
Present Phrase Medially \gloss{‘s/he...(for
                him/her)’}


\begin{tabular}{lllll}  
  
                       %\includegraphics[width=\textwidth]{InkScape%20Images/H%20Stems.svg}
 &   
                       %\includegraphics[width=\textwidth]{InkScape%20Images/No%20OP.svg}
 &   
                       \vernacular{a\ob [reetsaangá]\cb 
                      mú{\downstep}yáyi}  &   
                       \gloss{‘buries the
                      boy’}  &  \\

                       \vernacular{a\ob [reetsaanga]\cb 
                      muundu}  &   
                       \gloss{‘buries
                      somebody’}  &  \\
  &     &  \\

                       \vernacular{
                      a\ob [boyong’ánáángá]\cb  mú{\downstep}yáyi}  &   
                       \gloss{‘goes around the
                      boy’}  &  \\

                       \vernacular{
                      a\ob [boyong’anaanga]\cb  muundu}  &   
                       \gloss{‘goes around
                      somebody’}  &  \\
  &     &     &  \\

                       %\includegraphics[width=\textwidth]{InkScape%20Images/One%20OP.svg}
 &   
                       \vernacular{
                      a\ob mu[réetsáángá]\cb  mú{\downstep}yáyi}  &   
                       \gloss{‘buries the
                      boy’}  &  \\

                       \vernacular{
                      a\ob mu[réetsaanga]\cb  muundu}  &   
                       \gloss{‘buries
                      somebody’}  &  \\
  &     &  \\

                       \vernacular{
                      a\ob mu[βó{\downstep}yóng’ánáángá]\cb 
                      mú{\downstep}yáyi}  &   
                       \gloss{‘goes around the
                      boy’}  &  \\

                       \vernacular{
                      a\ob mu[βóyong’anaanga]\cb  muundu}  &   
                       \gloss{‘goes around
                      somebody’}  &  \\
  &     &     &  \\

                       %\includegraphics[width=\textwidth]{InkScape%20Images/0%20Stems.svg}
 &   
                       %\includegraphics[width=\textwidth]{InkScape%20Images/No%20OP.svg}
 &   
                       \vernacular{
                      a\ob [tsíítsáángá]\cb  mú{\downstep}yáyi}  &   
                       \gloss{‘goes for the
                      boy’}  &  \\

                       \vernacular{a\ob [tsiitsaanga]\cb 
                      muundu}  &   
                       \gloss{‘goes for
                      somebody’}  &  \\
  &     &  \\

                       \vernacular{
                      a\ob [sééβúláángá]\cb  mú{\downstep}yáyi}  &   
                       \gloss{‘says goodbye to the
                      boy’}  &  \\

                       \vernacular{a\ob [seeβulaanga]\cb 
                      muundu}  &   
                       \gloss{‘says goodbye to
                      somebody’}  &  \\
  &     &     &  \\

                       %\includegraphics[width=\textwidth]{InkScape%20Images/One%20OP.svg}
 &   
                       \vernacular{
                      a\ob mu[tsíítsáángá]\cb  mú{\downstep}yáyi}  &   
                       \gloss{‘goes for the
                      boy’}  &  \\

                       \vernacular{
                      a\ob mu[tsiitsaanga]\cb  muundu}  &   
                       \gloss{‘goes for
                      somebody’}  &  \\
  &     &  \\

                       \vernacular{
                      a\ob mu[sééβúláángá]\cb  mú{\downstep}yáyi}  &   
                       \gloss{‘says goodbye to the
                      boy’}  &  \\

                       \vernacular{
                      a\ob mu[seeβulaanga]\cb  muundu}  &   
                       \gloss{‘says goodbye to
                      somebody’}  &  \\
\end{tabular}
%\caption{\nocaption}
    
\z



\subsubsection{Present: Impact of Subject
              Choice}\label{sec:sP5aSubjects}

The choice of subject has no impact on the verbal
              tone properties of verbs in the Present. /H/ and /Ø/
              stems lacking an object prefix are realized the same
              whether the subject is 1 \textsuperscript{st}, 2 \textsuperscript{nd}, or 3 \textsuperscript{rd}person. /H/
              stems surface with a melodic H on the FV, while /Ø/
              stems realize the melodic H on the second stem
              mora.

 
\ea\label{ex:xPresSubjectsH} 
Subject Choice in the Present /H/ \gloss{
                ‘...bring(s)’}[SB]


\begin{tabular}{llll}  
    &   Singular  &   Plural  &  \\
1
                     \textsuperscript{
                    st}Person &   
                       \vernacular{
                      \ob [ndeeráángá]\cb }  &   
                       \vernacular{
                      khu\ob [leeráángá]\cb }  &  \\
2
                     \textsuperscript{
                    nd}Person &   
                       \vernacular{
                      u\ob [leeráángá]\cb }  &   
                       \vernacular{
                      mu\ob [leeráángá]\cb }  &  \\
3
                     \textsuperscript{
                    rd}Person &   
                       \vernacular{
                      a\ob [leeráángá]\cb }  &   
                       \vernacular{
                      βa\ob [leeráángá]\cb }  &  \\
\end{tabular}
%\caption{\nocaption}
    
\z

 
\ea\label{ex:xPresSubjectsØ} 
Subject Choice in the Present /Ø/ \gloss{
                ‘...ask(s)’}[SB]


\begin{tabular}{llll}  
    &   Singular  &   Plural  &  \\
1
                     \textsuperscript{
                    st}Person &   
                       \vernacular{
                      \ob [ndeéβáanga]\cb }  &   
                       \vernacular{
                      khu\ob [reéβáanga]\cb }  &  \\
2
                     \textsuperscript{
                    nd}Person &   
                       \vernacular{
                      u\ob [reéβáanga]\cb }  &   
                       \vernacular{
                      mu\ob [reéβáanga]\cb }  &  \\
3
                     \textsuperscript{
                    rd}Person &   
                       \vernacular{
                      a\ob [reéβáanga]\cb }  &   
                       \vernacular{
                      βa\ob [reéβáanga]\cb }  &  \\
\end{tabular}
%\caption{\nocaption}
    
\z

 The effect of subject choice in Present forms
              with an object prefix was only tested in /Ø/ verbs,
              which do not exhibit any subject-induced tonal
              alternations. 

 
\ea\label{ex:xPresSubjectsØOP} 
Subject Choice in the Present /Ø/ +
                OP \gloss{‘...ask(s)
                him/her’}[SB]


\begin{tabular}{llll}  
    &   Singular  &   Plural  &  \\
1
                     \textsuperscript{
                    st}Person &   
                       \vernacular{
                      \ob mu[reéβáanga]\cb }  &   
                       \vernacular{
                      khu\ob mu[reéβáanga]\cb }  &  \\
2
                     \textsuperscript{
                    nd}Person &   
                       \vernacular{
                      u\ob mu[reéβáanga]\cb }  &   
                       \vernacular{
                      mu\ob mu[reéβáanga]\cb }  &  \\
3
                     \textsuperscript{
                    rd}Person &   
                       \vernacular{
                      a\ob mu[reéβáanga]\cb }  &   
                       \vernacular{
                      βa\ob mu[reéβáanga]\cb }  &  \\
\end{tabular}
%\caption{\nocaption}
    
\z



\subsubsection{Present: Passives}\label{sec:sP5aPassives}

My corpus of passive data show that the passive
              suffix variably realizes a H in the Present. The
              melodic H surfaces in its expected position in both
              verb types (third syllable in /H/ verbs, second mora
              in /Ø/ verbs), and the passive H, when present, is
              realized on the penultimate mora of the long final
              syllable and spreads left via \regel{Plateau}. In § \sectref{sec:sP5aOtherTenses} , I
              suggest that phonological factors do not determine
              whether the passive H surfaces in the Present and
              other Pattern 5a constructions, this instead being an
              area of free variation.

 
\ea\label{ex:xPresPassives} 
Present: Passives \gloss{‘s/he is
                being...’}[SB]


\begin{tabular}{lllll}  
  \multicolumn{2}{l}{/H/ Stems } &   \multicolumn{2}{l}{/Ø/ Stems } &  \\

                       \vernacular{
                      a[khalakwáá{\downstep}ng-ú-a]}  &   
                       \gloss{‘cut’}  &   
                       \vernacular{
                      a[lakhúú{\downstep}lwááng-ú-a]}  &   
                       \gloss{‘released’}  &  \\

                       \vernacular{
                      a[tsuunzuunwáá{\downstep}ng-ú-a]}  &   
                       \gloss{‘sucked’}  &   
                       \vernacular{
                      a[kalúshítswaang-u-a]}  &   
                       \gloss{‘returned’}  &  \\
\end{tabular}
%\caption{\nocaption}
    
\z

 In the Present, the passive is doubly marked
              segmentally: the passive suffix \vernacular{-u/w}is
              realized both before and after the imperfective
              suffix \vernacular{-aang}.
              While the passive H surfaces near both instances of
              the segmental morpheme, the passive H as being
              assigned to the penultimate mora of the long final
              syllable by \regel{Passive H
              Assignment}as formalized in \REF{ex:xPassiveHAssignmentFinal} . The
              passive H may then approximate the leftmost instance
              of the segmental morpheme via \regel{Plateau}.

 There is an unfortunate gap in the data that would
              have provided an informative test of the analysis of
              passive H assignment developed in this thesis
              (particularly as discussed in reference to Pattern 2a
              data in § \sectref{sec:sP2aOtherTenses} ). My
              corpus does not include phrase-medial data for
              Present forms involving the passive suffix. Given
              that the melodic H fails to surface in phrase-medial
              Present forms, my analysis of passives predicts that
              the passive H would also not surface. I hope to test
              this prediction in later work.



\subsubsection{Pattern 5a: Other Verbal
              Contexts}\label{sec:sP5aOtherTenses}

Several other morphologically related verbal
              contexts share the tonal melody that characterizes
              the Present. Each of the contexts listed in \REF{ex:xP5aTenses} below exhibits \regel{Initial Lowering},
              and takes a melodic H which targets the second stem
              mora (/Ø/ verbs) or all moras of the third syllable
              (/H/ verbs) in the basic case. Additionally, the
              choice of verbal subject does not impact stem tone
              and the melodic H is lost when in phrase-medial
              forms. In this section, I also discuss complications
              in Pattern 5a constructions with respect to whether
              the passive H is realized.

 
\ea\label{ex:xP5aTenses} 
Other Pattern 5a Verbal
                Contexts 


\begin{tabular}{llll}  
  a.  &   Present Negative  &   
                       \vernacular{
                      SP[ROOT(-its)-aang-a] tá(awe)}  &  \\
b.  &   Persistive  &   
                       \vernacular{
                      SP-shi[ROOT(-its)-aang-a]}  &  \\
c.  &   Persistive Negative  &   
                       \vernacular{
                      SP-shi[ROOT(-its)-aang-a] tá(awe)}  &  \\
\end{tabular}
%\caption{\nocaption}
    
\z

 The melodic H is realized on all moras of the
              third syllable in /H/ verbs and on the second stem
              mora in /Ø/ Pattern 5a constructions. The H of the
              negative particle \vernacular{
              tá(awe)}spreads left onto the FV via \regel{Plateau}when the
              melodic H surfaces on the penultimate syllable, as in
              the Present Negative and Persistive Negative /H/
              examples in \REF{ex:xP5aHStems} below. It is also noteworthy that the
              negative \vernacular{
              tá(awe)}does not create the environment
              for \regel{Phrase-Medial H
              Deletion}.

 
\ea\label{ex:xP5aHStems} 
Morphologically Simple /H/ Stems \footnote{\label{fn:nP5aGlosses} The examples included in the current section
                  use \vernacular{
                  -khálak-} \gloss{‘cut’}and \vernacular{
                  -lakhuul-} \gloss{‘release’}as
                  representative of /H/ and /Ø/ verbal roots,
                  respectively. The basic gloss for the verbal
                  contexts discussed in this section is the
                  following: Present Negative - \gloss{‘s/he is
                  not...’}; Persistive - \gloss{‘s/he is
                  still...’}, and Persistive Negative - \gloss{‘s/he is not
                  still...’}.


}%



\begin{tabular}{llllll}  
    &   Subj  &   Tns  &   Stem  &   Neg  &  \\
Pres Neg  &   
                       \vernacular{a-}  &   
                       \vernacular{Ø}  &   
                       \vernacular{
                      \ob [khalakáá{\downstep}ngá]\cb }  &   
                       \vernacular{
                      tá(awe)}  &  \\
Pers  &   
                       \vernacular{a-}  &   
                       \vernacular{shi}  &   
                       \vernacular{
                      \ob [khalakáánga]\cb }  &     &  \\
Pers Neg  &   
                       \vernacular{a-}  &   
                       \vernacular{shi}  &   
                       \vernacular{
                      \ob [khalakáá{\downstep}ngá]\cb }  &   
                       \vernacular{
                      tá(awe)}  &  \\
\end{tabular}
%\caption{\nocaption}
    
\z

 
\ea\label{ex:xP5aØStems} 
Morphologically Simple /Ø/
                Stems 


\begin{tabular}{llllll}  
    &   Subj  &   Tns  &   Stem  &   Neg  &  \\
Pres Neg  &   
                       \vernacular{a-}  &   
                       \vernacular{Ø}  &   
                       \vernacular{
                      \ob [lakhúulaanga]\cb }  &   
                       \vernacular{
                      tá(awe)}  &  \\
Pers  &   
                       \vernacular{a-}  &   
                       \vernacular{shi}  &   
                       \vernacular{
                      \ob [lakhúulaanga]\cb }  &     &  \\
Pers Neg  &   
                       \vernacular{a-}  &   
                       \vernacular{shi}  &   
                       \vernacular{
                      \ob [lakhúulaanga]\cb }  &   
                       \vernacular{
                      tá(awe)}  &  \\
\end{tabular}
%\caption{\nocaption}
    
\z

 A single object prefix H will not surface, but
              the presence of the object prefix pushes the root H
              in /H/ verbs out of macrostem-initial position. The
              root H therefore surfaces on the initial mora of the
              stem in /H/ verbs. The emergence of the root H in /H/
              verbs also creates the environment for the melodic H
              to spread left via \regel{Plateau}onto the
              peninitial mora of the stem.

 
\ea\label{ex:xP5aOPHStems} 
/H/ Stems with an Object
                Prefix 


\begin{tabular}{lllllll}  
    &   Subj  &   Tns  &   Obj  &   Stem  &   Neg  &  \\
Pres Neg  &   
                       \vernacular{a-}  &   
                       \vernacular{Ø}  &   
                       \vernacular{\ob mu}  &   
                       \vernacular{
                      [khá{\downstep}lákáá{\downstep}ngá]\cb }  &   
                       \vernacular{
                      tá(awe)}  &  \\
Pers  &   
                       \vernacular{a-}  &   
                       \vernacular{shi}  &   
                       \vernacular{\ob mu}  &   
                       \vernacular{
                      [khá{\downstep}lákáánga]\cb }  &   
                       \vernacular{}  &  \\
Pers Neg  &   
                       \vernacular{a-}  &   
                       \vernacular{shi}  &   
                       \vernacular{\ob mu}  &   
                       \vernacular{
                      [khá{\downstep}lákáá{\downstep}ngá]\cb }  &   
                       \vernacular{
                      tá(awe)}  &  \\
\end{tabular}
%\caption{\nocaption}
    
\z

 As in forms without an object prefix, the melodic
              H surfaces on the second stem mora in /Ø/ verbs 

 
\ea\label{ex:xP5aOPØStems} 
/Ø/ Stems with an Object
                Prefix 


\begin{tabular}{lllllll}  
    &   Subj  &   Tns  &   Obj  &   Stem  &   Neg  &  \\
Pres Neg  &   
                       \vernacular{a-}  &   
                       \vernacular{Ø}  &   
                       \vernacular{\ob mu}  &   
                       \vernacular{
                      [lakhúulaanga]\cb }  &   
                       \vernacular{
                      tá(awe)}  &  \\
Pers  &   
                       \vernacular{a-}  &   
                       \vernacular{shi}  &   
                       \vernacular{\ob mu}  &   
                       \vernacular{
                      [lakhúulaanga]\cb }  &     &  \\
Pers Neg  &   
                       \vernacular{a-}  &   
                       \vernacular{shi}  &   
                       \vernacular{\ob mu}  &   
                       \vernacular{
                      [lakhúulaanga]\cb }  &   
                       \vernacular{
                      tá(awe)}  &  \\
\end{tabular}
%\caption{\nocaption}
    
\z

 As in the Present, the Present Negative and
              Persistive contexts lose the melodic H
              phrase-medially. Examples are provided below of a /H/
              verb \vernacular{
              -khálak-} \gloss{‘cut’}and /Ø/
              verb \vernacular{
              -lakhuul-} \gloss{
              ‘release’}before a H-toned noun, \vernacular{
              mú{\downstep}yáyi} \gloss{‘boy’}, and a
              toneless noun \vernacular{muundu} \gloss{‘person /
              somebody’}. The H of the H-toned complement \vernacular{
              mú{\downstep}yáyi}spreads left via \regel{H Tone
              Anticipation}onto the post-peninitial
              syllable of /H/ stems \footnote{\label{fn:nIngosiPropensitytoHTA} This characterization of the leftward extent of \regel{H Tone
                Anticipation}is much more apparent in the
                productions of JI than in those of SB.


}%


 
\ea\label{ex:xP5aPhraseMed} 
Verbal Contexts Like the Present
                Phrase Medially 


\begin{tabular}{llll}  
  
                       \textbf{Pres Neg}  &   
                    /H/  &   
                       \vernacular{
                      a[khalakáángá] mú{\downstep}yáyi
                      tá(awe)}  &  \\

                       \vernacular{a[khalakaanga]
                      muundu tá(awe)}  &  \\
  &     &  \\

                    /Ø/  &   
                       \vernacular{
                      a[lákhúúláángá] mú{\downstep}yáyi
                      tá(awe)}  &  \\

                       \vernacular{
                      a[lákhúúláángá] múúndú
                      tá(awe)}  &  \\
  &     &     &  \\

                       \textbf{Pers}  &   
                    /H/  &   
                       \vernacular{
                      a-shi[khalakáángá] mú{\downstep}yáyi}  &  \\

                       \vernacular{
                      a-shi[khalakaanga] muundu}  &  \\
  &     &  \\

                    /Ø/  &   
                       \vernacular{
                      a-shi[lákhúúláángá]
                      mú{\downstep}yáyi}  &  \\

                       \vernacular{
                      a-shi[lakhuulaanga] muundu}  &  \\
  &     &     &  \\

                       \textbf{Pers Neg}  &   
                    /H/  &   
                       \vernacular{
                      a-shi[khalakáángá] mú{\downstep}yáyí
                      {\downstep}tá(awe)}  &  \\

                       \vernacular{
                      a-shi[khalakáángá] múúndú
                      tá(awe)}  &  \\
  &     &  \\

                    /Ø/  &   
                       \vernacular{
                      a-shi[lákhúúláángá] mú{\downstep}yáyi
                      tá(awe)}  &  \\

                       \vernacular{
                      a-shi[lákhúúláángá] múúndú
                      tá(awe)}  &  \\
\end{tabular}
%\caption{\nocaption}
    
\z

 No data concerning the impact of subject choice
              on the stem tone of the verbal contexts under
              consideration in this section are currently
              available. 

 Passives are a locus of much variation in Idakho.
              My analysis of passive Hs, as developed in reference
              to the tonal melodies described earlier in the thesis
              (see § \sectref{sec:sP2aOtherTenses} ),
              predicts that each of the Present and Present-related
              forms in \REF{ex:xPresPassivesRepeat} - \REF{ex:xPersNegPassives} will realize the
              passive H on the penult and all moras to the right of
              the melodic H. However, the data show that several
              verb forms do not have H on the passive, despite
              meeting the criteria for passive H assignment
              (appearing in a context that takes a melodic H and
              realizing that melodic H on the verb).

 
\ea\label{ex:xPresPassivesRepeat} 
Present: Passives \gloss{‘s/he is
                being...’}[SB; repeated from \REF{ex:xPresPassives} ]


\begin{tabular}{lllll}  
  \multicolumn{2}{l}{/H/ Stems } &   \multicolumn{2}{l}{/Ø/ Stems } &  \\

                       \vernacular{
                      a[khalakwáá{\downstep}ng-ú-a]}  &   
                       \gloss{‘cut’}  &   
                       \vernacular{
                      a[lakhú{\downstep}úlwááng-ú-a]}  &   
                       \gloss{‘released’}  &  \\

                       \vernacular{
                      a[tsuunzuunwáá{\downstep}ng-ú-a]}  &   
                       \gloss{‘sucked’}  &   
                       \vernacular{
                      a[kalúshítswaang-u-a]}  &   
                       \gloss{‘returned’}  &  \\
\end{tabular}
%\caption{\nocaption}
    
\z

 
\ea\label{ex:xPresNegPassives} 
Present Negative: Passives \gloss{‘s/he is not
                being...’}[SB]


\begin{tabular}{lll}  
  \multicolumn{2}{l}{/H/ Stems } &  \\

                       \vernacular{
                      a[khalakwáá{\downstep}ng-w-á] tá}  &   
                       \gloss{‘cut’}  &  \\

                       \vernacular{
                      a[tsuunzuunwáá{\downstep}ng-w-á] tá}  &   
                       \gloss{‘sucked’}  &  \\
  &  \\

                       \textbf{/Ø/ Stems}  &  \\

                       \vernacular{
                      a[lakhúulwaang-w-a] tá}  &   
                       \gloss{‘released’}  &  \\

                       \vernacular{
                      a[kalúshítswaang-w-a] tá}  &   
                       \gloss{‘returned’}  &  \\
\end{tabular}
%\caption{\nocaption}
    
\z

 
\ea\label{ex:xPersPassives} 
Persistive: Passives \gloss{‘s/he is still
                being...’}[SB]


\begin{tabular}{lll}  
  \multicolumn{2}{l}{/H/ Stems } &  \\

                       \vernacular{
                      a-shi[khalakwááng-u-a]}  &   
                       \gloss{‘cut’}  &  \\

                       \vernacular{
                      a-shi[tsuunzuunwááng-u-a]}  &   
                       \gloss{‘sucked’}  &  \\
  &  \\

                       \textbf{/Ø/ Stems}  &  \\

                       \vernacular{
                      a-shi[lakhú{\downstep}úlwááng-ú-a]}  &   
                       \gloss{‘released’}  &  \\

                       \vernacular{
                      a-shi[kalúshí{\downstep}tswááng-ú-a]}  &   
                       \gloss{‘returned’}  &  \\
\end{tabular}
%\caption{\nocaption}
    
\z

 
\ea\label{ex:xPersNegPassives} 
Persistive Negative: Passives \gloss{‘s/he is not still
                being...’}[SB]


\begin{tabular}{lll}  
  \multicolumn{2}{l}{/H/ Stems } &  \\

                       \vernacular{
                      a-shi[khalakwáá{\downstep}ng-w-á] {\downstep}tá}  &   
                       \gloss{‘cut’}  &  \\

                       \vernacular{
                      a-shi[tsuunzuunwáá{\downstep}ng-w-á] {\downstep}tá}  &   
                       \gloss{‘sucked’}  &  \\
  &  \\

                       \textbf{/Ø/ Stems}  &  \\

                       \vernacular{
                      a-shi[lakhúulwaang-w-a] tá}  &   
                       \gloss{‘released’}  &  \\

                       \vernacular{
                      a-shi[kalúshítswaang-w-a] tá}  &   
                       \gloss{‘returned’}  &  \\
\end{tabular}
%\caption{\nocaption}
    
\z

 My corpus of Pattern 5a passive data is limited
              to 1-2 productions of each of the forms appearing in \REF{ex:xPresPassivesRepeat} - \REF{ex:xPersNegPassives} , spoken by a
              single speaker (SB). What is striking about the
              distribution of passive Hs in these data is the lack
              of correspondence between constructions. This
              suggests that forms with and without the passive H
              are in free variation with one another and that the
              presence of the passive H is not predicted by
              phonological features of the verb form.

 In SB’s productions of the Present data, the
              passive H surfaced in both /H/ verbs, but only one
              /Ø/ verb. None of his Present Negative productions
              realized the passive H. In the Persistive, tokens of
              the /Ø/ verbs express the passive H, but not /H/
              verbs. The inverse was true of his Persistive
              Negative productions, with /H/ verbs expressing the
              passive H, but not /Ø/ verbs. \footnote{\label{fn:nSuperficiallyHPassives} The final syllable is H in Present Negative and
                Persistive Negative /H/ verbs, though this is
                because of \regel{Plateau}, which
                spreads the H of negative \vernacular{
                tá}left. The lack of downstep between
                the final syllable and \vernacular{
                tá}supports this interpretation.


}%


 The variation observed above in the Present and
              related verbal contexts is unusual among the passive
              data elicited for this study, and it is not clear at
              this time what the reason for this variation is. In
              future work, I will explore the possibility that the
              occasional failure of the passive H to surface in the
              above forms is related to a larger pattern of
              variation in the form of the Present and related
              contexts. In particular, \gloss{‘s/he cuts, s/he is
              cutting’}may be expressed either in full as
              in \vernacular{
              a[khalakáánga]}or with the final syllable
              truncated as in \vernacular{
              a[khalakáá]}. \footnote{\label{fn:nPresentTruncation} It is not clear what conditions truncation in
                the Present and related contexts. Speakers offered
                contradictory judgements regarding a potential
                aspectual contrast (habitual vs. progressive). In
                Tiriki, the truncation may be conditioned by phrase
                position, where the truncated form is selected
                phrase-medially (Marlo, p.c.). The prompts that I
                presented to the speakers involving the present
                tense verbs in phrase-medial position included the
                final syllable. No speaker explicitly rejected the
                prompts, though one speaker spontaneously switched
                to truncated variants mid-paradigm in one instance.
                I asked if he preferred the truncated form. His
                response indicated that both forms are
                acceptable. 


}%




\subsection{Pattern 5b: Indefinite Future}\label{sec:sPattern5b}

The present section details the tonal properties of
            Pattern 5b, exhibited by the Indefinite Future and the
            Indefinite Future Negative. The Indefinite Future is
            marked by the toneless \vernacular{li-}prefix,
            the FV \vernacular{-a}, and a
            melodic H which surfaces on the second stem mora in /Ø/
            stems and the FV in /H/ stems. Because of the formal
            similarity between Pattern 5a and 5b, the analysis of
            the Indefinite Future will be given in tandem with the
            description.

 The Pattern 5b melodic H in /H/ verbs interacts with
            prosodic properties of the verb stem. In stems
            comprised of three or more moras, the melodic H is
            realized on the final mora of the stem. However, in
            bimoraic stems, which include both CVV and CVCV stems,
            the melodic H does not surface. 

 
\ea\label{ex:xIndefFutCH} 
Indefinite Future C-Initial /H/ \gloss{‘s/he
              will...’}


\begin{tabular}{lllll}  
  Subj  &   Tns  &   Stem  &   Gloss  &  \\

                     \vernacular{a-}  &   
                     \vernacular{li}  &   
                     \vernacular{
                    \ob [khua]\cb }  &   
                     \gloss{‘pay dowry’}  &  \\

                     \vernacular{a-}  &   
                     \vernacular{li}  &   
                     \vernacular{
                    \ob [βeka]\cb }  &   
                     \gloss{‘shave’}  &  \\

                     \vernacular{a-}  &   
                     \vernacular{li}  &   
                     \vernacular{
                    \ob [teekhá]\cb }  &   
                     \gloss{‘cook’}  &  \\

                     \vernacular{a-}  &   
                     \vernacular{li}  &   
                     \vernacular{
                    \ob [khalaká]\cb }  &   
                     \gloss{‘cut’}  &  \\

                     \vernacular{a-}  &   
                     \vernacular{li}  &   
                     \vernacular{
                    \ob [kalaangá]\cb }  &   
                     \gloss{‘fry’}  &  \\

                     \vernacular{a-}  &   
                     \vernacular{li}  &   
                     \vernacular{
                    \ob [βoolitsá]\cb }  &   
                     \gloss{‘seduce’}  &  \\

                     \vernacular{a-}  &   
                     \vernacular{li}  &   
                     \vernacular{
                    \ob [βoyong’aná]\cb }  &   
                     \gloss{‘go around’}  &  \\
\end{tabular}
%\caption{\nocaption}
    
\z

 The transcriptions of vowel initial /H/ verbs in
            the Indefinite Future are based on JI’s productions. In
            all forms, the FV bears the melodic H, and it surfaces
            verb finally even in short verbs like \vernacular{a-li[irá]} \gloss{‘s/he will
            kill’}, despite the morphological stem being
            bimoraic.

 
\ea\label{ex:xIndefFutVH} 
Indefinite Future V-Initial /H/ \gloss{‘s/he
              will...’}[JI]


\begin{tabular}{lllll}  
  Subj  &   Tns  &   Stem  &   Gloss  &  \\

                     \vernacular{a-}  &   
                     \vernacular{li}  &   
                     \vernacular{
                    \ob [irá]\cb }  &   
                     \gloss{‘kill’}  &  \\

                     \vernacular{a-}  &   
                     \vernacular{li}  &   
                     \vernacular{
                    \ob [onoɲːá]\cb }  &   
                     \gloss{‘spoil’}  &  \\

                     \vernacular{a-}  &   
                     \vernacular{li}  &   
                     \vernacular{
                    \ob [aβukhaɲːá]\cb }  &   
                     \gloss{‘separates’}  &  \\
\end{tabular}
%\caption{\nocaption}
    
\z

 SB’s productions of vowel-initial Indefinite Future
            /Ø/ verbs differ from JI’s. He produced very long verbs
            such as \vernacular{
            a-li[aβukháɲːa]} \gloss{‘s/he will
            scatter’}with a melodic H on the third stem
            syllable, rather than on the final. The productions
            with long stems seem to have been produced with Pattern
            5a, the melody described in the preceding section (§ \sectref{sec:sPattern5a} ) exemplified
            by the Present.

 The unexpected behavior of long vowel-initial /H/
            verbs does not appear to be a tonal alternation
            conditioned by the initial segment of the morphological
            stem; rather, it is a case of free variation whereby
            the Indefinite Future may take either Pattern 5a or
            Pattern 5b. This is consistent with comparative
            evidence from other Luhya varieties, in which the
            Present and the Indefinite Future both take a single
            melody closer in character to Idakho’s Pattern 5b (as
            noted in the introduction of § \sectref{sec:sPattern5} ), and the
            observation that some diachronic changes in Luhya
            verbal tone systems advance first in vowel initial
            verbs (see Ch. \sectref{sec:cPathToPredictability} ).

 The H on the FV of the bimoraic stem is also a
            surprise, given the data presented above for
            consonant-initial verb stems. Here, one might expect
            for the root H to lower, with the resultant L spreading
            right via \regel{L Spread II}thereby
            blocking melodic H assignment.

 I analyze the unexpected melodic H in \gloss{‘s/he will
            kill’}by stipulating that Hs lowered through \regel{Initial Lowering}only
            spread left via \regel{L Spread II}when the L
            is aligned with the left edge of its syllable. \regel{L Spread II}is
            reproduced in \REF{ex:xLSpreadIIReproduced} .

 
\ea\label{ex:xLSpreadIIReproduced} 
 \regel{L Spread II} 

%\includegraphics[width=\textwidth]{InkScape%20Images/Rules/LSpreadII.pdf}

\z

 The melodic H surfaces on the second stem mora of
            /Ø/ verbs with bimoraic stems or longer. Monosyllabic
            stems have a fall. 

 
\ea\label{ex:xIndefFutCØ} 
Indefinite Future C-Initial /Ø/ \gloss{‘s/he
              will...’}


\begin{tabular}{lllll}  
  Subj  &   Tns  &   Stem  &   Gloss  &  \\

                     \vernacular{a-}  &   
                     \vernacular{li}  &   
                     \vernacular{
                    \ob [kúa]\cb }  &   
                     \gloss{‘fall’}  &  \\

                     \vernacular{a-}  &   
                     \vernacular{li}  &   
                     \vernacular{
                    \ob [lekhá]\cb }  &   
                     \gloss{‘leave’}  &  \\

                     \vernacular{a-}  &   
                     \vernacular{li}  &   
                     \vernacular{
                    \ob [reéβa]\cb }  &   
                     \gloss{‘ask’}  &  \\

                     \vernacular{a-}  &   
                     \vernacular{li}  &   
                     \vernacular{
                    \ob [kulíkha]\cb }  &   
                   \gloss{
                  ‘name’}[SB] &  \\

                     \vernacular{a-}  &   
                     \vernacular{li}  &   
                     \vernacular{
                    \ob [lakhúula]\cb }  &   
                     \gloss{‘release’}  &  \\

                     \vernacular{a-}  &   
                     \vernacular{li}  &   
                     \vernacular{
                    \ob [seéβula]\cb }  &   
                     \gloss{‘say goodbye
                    (to)’}  &  \\

                     \vernacular{a-}  &   
                     \vernacular{li}  &   
                     \vernacular{
                    \ob [kalúshitsa]\cb }  &   
                   \gloss{
                  ‘return’}[SB] &  \\
\end{tabular}
%\caption{\nocaption}
    
\z

 Vowel-initial /Ø/ verbs also have a H on the second
            mora. 

 
\ea\label{ex:xIndefFutVØ} 
Indefinite Future V-Initial /Ø/ \gloss{‘s/he
              will...’}


\begin{tabular}{lllll}  
  Subj  &   Tns  &   Stem  &   Gloss  &  \\

                     \vernacular{a-}  &   
                     \vernacular{li}  &   
                     \vernacular{
                    \ob [enyá]\cb }  &   
                     \gloss{‘want’}  &  \\

                     \vernacular{a-}  &   
                     \vernacular{li}  &   
                     \vernacular{
                    \ob [eyéla]\cb }  &   
                     \gloss{‘wipe for’}  &  \\

                     \vernacular{a-}  &   
                     \vernacular{li}  &   
                     \vernacular{
                    \ob [ambákhana]\cb }  &   
                     \gloss{‘refuse’}  &  \\

                     \vernacular{a-}  &   
                     \vernacular{li}  &   
                     \vernacular{
                    \ob [eléelitsa]\cb }  &   
                     \gloss{‘hang up
                    (s.t.)’}  &  \\
\end{tabular}
%\caption{\nocaption}
    
\z


\subsubsection{Indefinite Future with Object
              Prefixes}\label{sec:sP5bObjects}

The root H surfaces \textit{in situ}in Indefinite
              Future forms which include one object prefix.
              Additionally, the melodic H surfaces on the final
              vowel and all preceding syllables through the
              peninitial as a level downstepped H span. The object
              prefix surfaces L.

 The above characterization applies to the
              productions of SB and the overwhelming majority of
              Idakholand residents I interviewed. JI’s productions
              appear to have the properties of Present and
              Indefinite Future forms in the Tiriki variety of
              Idakho, i.e.: with a H, the root H, on the initial
              stem mora and a melodic H which spans the remainder
              of the stem through the penult ( \citealt{rMarloInPrepB} ).

 
\ea\label{ex:xIndefFutCHOP} 
Indefinite Future C-Initial /H/ +
                OP \gloss{‘s/he
                will...him/her’}


\begin{tabular}{llllll}  
  Subj  &   Tns  &   Obj  &   Stem  &   Gloss  &  \\

                       \vernacular{a-}  &   
                       \vernacular{li}  &   
                       \vernacular{\ob mu}  &   
                       \vernacular{
                      [ráa]\cb }  &   
                       \gloss{‘pay dowry
                      (for)’}  &  \\

                       \vernacular{a-}  &   
                       \vernacular{li}  &   
                       \vernacular{\ob mu}  &   
                       \vernacular{
                      [βé{\downstep}ká]\cb }  &   
                       \gloss{‘bite’}  &  \\

                       \vernacular{a-}  &   
                       \vernacular{li}  &   
                       \vernacular{\ob mu}  &   
                       \vernacular{
                      [lé{\downstep}érá]\cb }  &   
                       \gloss{‘bring’}  &  \\

                       \vernacular{a-}  &   
                       \vernacular{li}  &   
                       \vernacular{\ob mu}  &   
                       \vernacular{
                      [khá{\downstep}láká]\cb }  &   
                       \gloss{‘cut’}  &  \\

                       \vernacular{a-}  &   
                       \vernacular{li}  &   
                       \vernacular{\ob mu}  &   
                       \vernacular{
                      [βó{\downstep}ólítsá]\cb }  &   
                       \gloss{‘seduce’}  &  \\

                       \vernacular{a-}  &   
                       \vernacular{li}  &   
                       \vernacular{\ob mu}  &   
                       \vernacular{
                      [βó{\downstep}yóng’áná]\cb }  &   
                       \gloss{‘descend
                      for’}  &  \\
\end{tabular}
%\caption{\nocaption}
    
\z

 In /Ø/ stems, the melodic H is realized on the
              second stem mora except in monosyllabic stem, which
              realize a fall. As usual, the H of the object prefix
              does not surface. 

 
\ea\label{ex:xIndefFutCØOP} 
Indefinite Future C-Initial /Ø/ +
                OP \gloss{‘s/he
                will...him/her/them
                }


\begin{tabular}{llllll}  
  Subj  &   Tns  &   Obj  &   Stem  &   Gloss  &  \\

                       \vernacular{a-}  &   
                       \vernacular{li}  &   
                       \vernacular{\ob mu}  &   
                       \vernacular{
                      [tsía]\cb }  &   
                       \gloss{‘go (for)’}  &  \\

                       \vernacular{a-}  &   
                       \vernacular{li}  &   
                       \vernacular{\ob mu}  &   
                       \vernacular{
                      [lekhá]\cb }  &   
                       \gloss{‘leave’}  &  \\

                       \vernacular{a-}  &   
                       \vernacular{li}  &   
                       \vernacular{\ob mu}  &   
                       \vernacular{
                      [loónda]\cb }  &   
                       \gloss{‘follow’}  &  \\

                       \vernacular{a-}  &   
                       \vernacular{li}  &   
                       \vernacular{\ob mu}  &   
                       \vernacular{
                      [kulíkha]\cb }  &   
                       \gloss{‘name’}  &  \\

                       \vernacular{a-}  &   
                       \vernacular{li}  &   
                       \vernacular{\ob mu}  &   
                       \vernacular{
                      [seéβula]\cb }  &   
                       \gloss{‘say
                      goodbye’}  &  \\

                       \vernacular{a-}  &   
                       \vernacular{li}  &   
                       \vernacular{\ob mu}  &   
                       \vernacular{
                      [kalúshitsa]\cb }  &   
                       \gloss{‘return’}  &  \\

                       \vernacular{a-}  &   
                       \vernacular{li}  &   
                       \vernacular{\ob βi}  &   
                       \vernacular{
                      [seβúlúkhaɲːa]\cb }  &   
                       \gloss{‘scatter’}  &  \\
\end{tabular}
%\caption{\nocaption}
    
\z

 The productions of both primary consultants
              diverge from the characterization of Pattern 5b
              developed thus far in forms with two object prefixes.
              The expected pattern would be a rise on the pre-stem
              syllable in combination with a downstepped melodic H
              which spans the full length of the stem in /H/ verbs
              or through the second stem mora in /Ø/ verbs. JI has
              the expected rise, but has different patterns of stem
              tone. SB’s productions are missing the pre-stem rise
              but have stem tone patterns that are consistent with
              the preceding description of Pattern 5b. 

 JI’s productions include a rise on the pre-stem
              syllable, as expected, but surface with a downstepped
              H span across all but the final stem syllables in /H/
              verbs, e.g., \vernacular{
              a-li\ob mu-ú[{\downstep}mbéchéla]\cb } \gloss{‘s/he will shave him/her
              for me’}. In /Ø/ verbs, the melodic H is
              absent, e.g. \vernacular{
              a-li\ob mu-ú[ndeshela]\cb } \gloss{‘s/he will leave him/her
              for me’}. \footnote{\label{fn:nNoMHinOPOP} JI’s never produces the second mora melodic H in
                /Ø/ verb forms with two object prefixes. 


}%


 The stem tone properties of SB’s productions are
              identical to Indefinite Future forms with a single
              object prefix. That is, the pre-stem syllable is L,
              the root H is realized \textit{in situ}, and the
              melodic H surfaces as a downstepped H which spans
              from the second stem mora through the FV, e.g., \vernacular{
              a-li\ob mu-u[mbé{\downstep}chélá]\cb } \gloss{‘s/he will shave him/her
              for me’}. In /Ø/ verbs, the pre-stem
              syllable is again L and the melodic H surfaces on the
              second stem mora.

 The unexpected tonal properties of the
              consultants' productions of these verb forms were not
              recognized early enough to solicit judgments about
              the possibility of the tonal variant that my analysis
              predicts. Appendix \appref{sec:sIndefFut} includes additional verified
              transcriptions of SB producing Indefinite Future
              forms with two object prefixes.

 The core properties of the tonal melody that
              characterizes the Indefinite Future are as follows:
              (i) underlying macrostem-initial Hs fail to surface,
              (ii) the melodic H surfaces on the second stem mora
              in /Ø/ verbs, (iii) the melodic H surfaces on the
              final syllable in /H/ stems, and (iv) the melodic H
              spreads leftward through the second stem syllable in
              /H/ verbs with an object. These properties are
              summarized schematically in the following
              display. 

 
\ea\label{ex:xIndefFutSchematic} 
A Schematic Representation of the
                Indefinite Future's Tonal
                Properties 


\begin{tabular}{lllll}  
    &   \multicolumn{3}{l}{
                       \ul{/H/ Verbs} } &  \\
  &   
                       \textit{Subj + Tns}  &   \multicolumn{2}{l}{
                       \textit{Macrostem} } &  \\
OPsx0  &   
                       \vernacular{a-li}  &   
                       \vernacular{\ob }  &   
                       \vernacular{[C
                      }  &  \\
OPsx1  &   
                       \vernacular{a-li-}  &   
                       \vernacular{\ob C
                      }  &   
                       \vernacular{[C
                      }  &  \\
  &   \multicolumn{2}{l}{ } &     &  \\
  &   \multicolumn{3}{l}{
                       \textbf{
                        } } &  \\
  &   
                       \textit{Subj + Tns}  &   \multicolumn{2}{l}{
                       \textit{Macrostem} } &  \\
OPsx0  &   
                       \vernacular{a-li}  &   
                       \vernacular{\ob }  &   
                     \vernacular{[CV(C)
                    }\cb  &  \\
OPsx1  &   
                       \vernacular{a-li-}  &   
                       \vernacular{\ob C
                      }  &   
                       \vernacular{[CV(C)
                      }  &  \\
\end{tabular}
%\caption{\nocaption}
    
\z

 The first observation—namely, that
              macrostem-initial Hs fail to surface—is accounted for
              by the now familiar rule of \regel{Initial
              Lowering}.

 Two rules of melodic H assignment are invoked in
              the analysis of the tonal properties of verbs in the
              Indefinite Future: \regel{Final MHA}and \regel{Default MHA}. \regel{Final MHA}follows \regel{Default MHA}in the
              derivation. This means that the melodic H is assigned
              directly to the second mora in forms involving /Ø/
              verbs. The melodic H is not assigned to the same
              position in /H/ verbs owing to the condition on \regel{Default MHA}that the
              target of melodic H assignment be preceded by a
              toneless mora.

 The melodic H in /H/ verbs is instead assigned to
              the final mora of the stem via \regel{Final MHA}, as
              formalized in \REF{ex:xFinalMHA} . \regel{Final MHA}require
              that the target of melodic H assignment is toneless.
              This feature of the rule will be justified
              shortly.

 
\ea\label{ex:xFinalMHA} 
 \regel{Final Melodic H
                  Assignment} 

%\includegraphics[width=\textwidth]{InkScape%20Images/Rules/FinalMHA.pdf}

\z

 The two smallest stem shapes, CVV and CVCV, fail
              to realize the melodic H in /H/ stems in the
              Indefinite Future. I argue that this failure is the
              result of the rule \regel{L Spread II}―first
              introduced in the analysis of Pattern 5a, § \sectref{sec:sPattern5a} . \regel{L Spread II}spreads
              the Ls that result from \regel{Initial
              Lowering}onto the peninitial mora. \regel{Final MHA}fails to
              apply in bimoraic stems because the target of \regel{Final MHA}bears a
              L.

 . 

 
\ea\label{ex:xDerivIndefFutHShort} 
 Derivation,
                  /H/ CVCV Indefinite Future: \vernacular{
                  a-li\ob [βeka]\cb } \gloss{‘s/he will
                  shave’} 

%\includegraphics[width=\textwidth]{InkScape%20Images/Derivations/DerivIndefFutHShort.pdf}

\z

 An alternative analysis in which \regel{Final MHA}requires
              that the mora preceding its target be toneless can
              account for the facts of /H/ Indefinite Future verbs
              without an object prefix. However, that approach
              fails because it makes incorrect predictions about
              /H/ bimoraic stems in forms with an object prefix. In
              particular, it predicts * \vernacular{
              a-li-mu[lúma]} \gloss{‘s/he will bite
              him/her’}, rather than the attested \vernacular{
              a-li-mu[lú{\downstep}má]}.

 The melodic H surfaces not only on the FV in /H/
              verbs with an object prefix, but also on all
              preceding syllables through the second stem syllable.
              This is the outcome of \regel{Plateau} \REF{ex:xPlateau} .

 
\ea\label{ex:xDerivIndefFutHOP} 
 Derivation,
                  /H/ Indefinite Future + OP: \vernacular{
                  a-li\ob mu[khá{\downstep}láká]\cb } \gloss{‘s/he will cut
                  him/her’} 

%\includegraphics[width=\textwidth]{InkScape%20Images/Derivations/DerivIndefFutHOP.pdf}

\z

  \regel{L Spread II}is
              omitted from the derivation above because it does not
              apply. It does not apply because the potential target
              of \regel{L Spread II}―the
              second mora of the macrostem―bears the root H.



\subsubsection{Indefinite Future: Phrase
              Medially}\label{sec:sP5bPhraseMed}

As in the Present (Pattern 5a, § \sectref{sec:sP5aPhraseMed} ), the
              melodic H is lost in the Indefinite Future when the
              verb is not final within the phrase. The effects of
              Initial Lowering persist in a phrase-medial context.
              Additionally, \regel{H Tone
              Anticipation}spreads the H from H-toned
              complements onto the verbal stem. The complement H
              appears to spread a greater distance in /Ø/ verbs
              than it does in /H/ verbs, though the left edge of
              the span is unclear in both cases. In /Ø/ verbs, the
              span extends onto the stem-initial syllable or
              beyond, while in /H/ verbs, the span ends at the
              peninitial or post-peninitial syllable. The
              complement H will not spread onto short /H/
              stems.

 Four pairs of /H/ and /Ø/ stems are provided
              below, half with and half without an object prefix.
              For each pair, the first member involves a H-toned
              complement, while the second involves a toneless
              complement. 

 
\ea\label{ex:xIndefFutPhraseMedial} 
Indefinite Future Phrase Medially \gloss{‘s/he will...(for
                him/her)’}


\begin{tabular}{lllll}  
  
                       %\includegraphics[width=\textwidth]{InkScape%20Images/H%20Stems.svg}
 &   
                       %\includegraphics[width=\textwidth]{InkScape%20Images/No%20OP.svg}
 &   
                       \vernacular{a-li\ob [ra]\cb 
                      mú{\downstep}yáyi}  &   
                       \gloss{‘bury the
                      boy’}  &  \\

                       \vernacular{a-li\ob [ra]\cb 
                      muundu}  &   
                       \gloss{‘bury
                      somebody’}  &  \\
  &     &  \\

                       \vernacular{a-li\ob [khalaká]\cb 
                      mú{\downstep}yáyi}  &   
                       \gloss{‘cut the
                      boy’}  &  \\

                       \vernacular{a-li\ob [khalaka]\cb 
                      muundu}  &   
                       \gloss{‘cut
                      somebody’}  &  \\
  &     &     &  \\

                       %\includegraphics[width=\textwidth]{InkScape%20Images/One%20OP.svg}
 &   
                       \vernacular{
                      a-li\ob mu[réelá]\cb  mú{\downstep}yáyi}  &   
                       \gloss{‘bury the
                      boy’}  &  \\

                       \vernacular{a-li\ob mu[réela]\cb 
                      muundu}  &   
                       \gloss{‘bury
                      somebody’}  &  \\
  &     &  \\

                       \vernacular{
                      a-li\ob mu[khá{\downstep}láchilá]\cb  mú{\downstep}yáyi}  &   
                       \gloss{‘cut the
                      boy’}  &  \\

                       \vernacular{
                      a-li\ob mu[khá{\downstep}láchilá]\cb  muundu}  &   
                       \gloss{‘cut
                      somebody’}  &  \\
  &     &     &  \\

                       %\includegraphics[width=\textwidth]{InkScape%20Images/0%20Stems.svg}
 &   
                       %\includegraphics[width=\textwidth]{InkScape%20Images/No%20OP.svg}
 &   
                       \vernacular{
                      a-li\ob [tsíílá]\cb  mú{\downstep}yáyi}  &   
                       \gloss{‘go for the
                      boy’}  &  \\

                       \vernacular{a-li\ob [tsiila]\cb 
                      muundu}  &   
                       \gloss{‘go for
                      somebody’}  &  \\
  &     &  \\

                       \vernacular{
                      a-li\ob [séébúlílá]\cb  mú{\downstep}yáyi}  &   
                       \gloss{‘say goodbye to the
                      boy’}  &  \\

                       \vernacular{
                      a-li\ob [seebulila]\cb  muundu}  &   
                       \gloss{‘say goodbye to
                      somebody’}  &  \\
  &     &     &  \\

                       %\includegraphics[width=\textwidth]{InkScape%20Images/One%20OP.svg}
 &   
                       \vernacular{a-li\ob mu[tsyá]\cb 
                      mú{\downstep}yáyi}  &   
                       \gloss{‘go for the
                      boy’}  &  \\

                       \vernacular{a-li\ob mu[tsya]\cb 
                      muundu}  &   
                       \gloss{‘go for
                      somebody’}  &  \\
  &     &  \\

                       \vernacular{
                      a-li\ob mu[sééβúlá]\cb  mú{\downstep}yáyi}  &   
                       \gloss{‘say goodbye to the
                      boy’}  &  \\

                       \vernacular{
                      a-li\ob mu[seeβula]\cb  muundu}  &   
                       \gloss{‘say goodbye to
                      somebody’}  &  \\
\end{tabular}
%\caption{\nocaption}
    
\z



\subsubsection{Indefinite Future: Impact of Subject
              Choice}\label{sec:sP5bSubjects}

The choice of subject has no impact on the verbal
              tone properties of verbs in the Indefinite Future.
              /H/ and /Ø/ stems lacking an object prefix are
              realized the same whether the subject is 1 \textsuperscript{st}, 2 \textsuperscript{nd}, or 3 \textsuperscript{rd}person. /H/
              stems surface with a H on the FV, while /Ø/ stems
              realize the melodic H on the second stem mora.

 
\ea\label{ex:xIndefFutSubjectsH} 
Subject Choice in the Indefinite
                Future /H/ \gloss{‘...will
                bring’}[SB]


\begin{tabular}{llll}  
    &   Singular  &   Plural  &  \\
1
                     \textsuperscript{
                    st}Person &   
                       \vernacular{
                      nɪ-lɪ\ob [leerá]\cb }  &   
                       \vernacular{
                      khu-li\ob [leerá]\cb }  &  \\
2
                     \textsuperscript{
                    nd}Person &   
                       \vernacular{
                      u-li\ob [leerá]\cb }  &   
                       \vernacular{
                      mu-li\ob [leerá]\cb }  &  \\
3
                     \textsuperscript{
                    rd}Person &   
                       \vernacular{
                      a-li\ob [leerá]\cb }  &   
                       \vernacular{
                      βa-li\ob [leerá]\cb }  &  \\
\end{tabular}
%\caption{\nocaption}
    
\z

 
\ea\label{ex:xIndefFutSubjectsØ} 
Subject Choice in the Indefinite
                Future /Ø/ \gloss{‘...will
                ask’}[SB]


\begin{tabular}{llll}  
    &   Singular  &   Plural  &  \\
1
                     \textsuperscript{
                    st}Person &   
                       \vernacular{
                      nɪ-lɪ\ob [reéβa]\cb }  &   
                       \vernacular{
                      khu-li\ob [reéβa]\cb }  &  \\
2
                     \textsuperscript{
                    nd}Person &   
                       \vernacular{
                      u-li\ob [reéβa]\cb }  &   
                       \vernacular{
                      mu-li\ob [reéβa]\cb }  &  \\
3
                     \textsuperscript{
                    rd}Person &   
                       \vernacular{
                      a-li\ob [reéβa]\cb }  &   
                       \vernacular{
                      βa-li\ob [reéβa]\cb }  &  \\
\end{tabular}
%\caption{\nocaption}
    
\z

 Indefinite Future forms with an object prefix are
              similarly unaffected by varying the subject of the
              verb. /H/ verbs realize the root H and a melodic H
              which spans from the second syllable through the FV,
              and /Ø/ verbs surface with just a melodic H on the
              second stem mora. 

 
\ea\label{ex:xIndefFutSubjectsHOP} 
Subject Choice in the Indefinite
                Future /H/ + OP \gloss{‘...will bring
                him/her’}[SB]


\begin{tabular}{llll}  
    &   Singular  &   Plural  &  \\
1
                     \textsuperscript{
                    st}Person &   
                       \vernacular{
                      nɪ-lɪ\ob mu[léerá]\cb }  &   
                       \vernacular{
                      khu-li\ob mu[léerá]\cb }  &  \\
2
                     \textsuperscript{
                    nd}Person &   
                       \vernacular{
                      u-li\ob mu[léerá]\cb }  &   
                       \vernacular{
                      mu-li\ob mu[léerá]\cb }  &  \\
3
                     \textsuperscript{
                    rd}Person &   
                       \vernacular{
                      a-li\ob mu[léerá]\cb }  &   
                       \vernacular{
                      βa-li\ob mu[léerá]\cb }  &  \\
\end{tabular}
%\caption{\nocaption}
    
\z

 
\ea\label{ex:xIndefFutSubjectsØOP} 
Subject Choice in the Indefinite
                Future /Ø/ + OP \gloss{‘...will ask
                him/her’}[SB]


\begin{tabular}{llll}  
    &   Singular  &   Plural  &  \\
1
                     \textsuperscript{
                    st}Person &   
                       \vernacular{
                      nɪ-lɪ\ob mu[reéβa]\cb }  &   
                       \vernacular{
                      khu-li\ob mu[reéβa]\cb }  &  \\
2
                     \textsuperscript{
                    nd}Person &   
                       \vernacular{
                      u-li\ob mu[reéβa]\cb }  &   
                       \vernacular{
                      mu-li\ob mu[reéβa]\cb }  &  \\
3
                     \textsuperscript{
                    rd}Person &   
                       \vernacular{
                      a-li\ob mu[reéβa]\cb }  &   
                       \vernacular{
                      βa-li\ob mu[reéβa]\cb }  &  \\
\end{tabular}
%\caption{\nocaption}
    
\z



\subsubsection{Indefinite Future: Passives}\label{sec:sP5bPassives}

The passive suffix contributes a H in Indefinite
              Future. This can be seen clearly in /Ø/ verbs with
              stems exceeding three moras. 

 
\ea\label{ex:xIndefFutPassives} 
Indefinite Future: Passives \gloss{‘s/he will
                be...’}[SB]


\begin{tabular}{lllll}  
  \multicolumn{2}{l}{/H/ Stems } &   \multicolumn{2}{l}{/Ø/ Stems } &  \\

                       \vernacular{
                      a-li\ob [khalak-ú-a]\cb }  &   
                       \gloss{‘cut’}  &   
                       \vernacular{
                      a-li\ob [lakhúul-ú-a]\cb }  &   
                       \gloss{‘released’}  &  \\

                       \vernacular{
                      a-li\ob [tsuunzuun-ú-a]\cb }  &   
                       \gloss{‘sucked’}  &   
                       \vernacular{
                      a-li\ob [kalú{\downstep}shíts-ú-a]\cb }  &   
                       \gloss{‘returned’}  &  \\
\end{tabular}
%\caption{\nocaption}
    
\z

 The positions where the passive H and the melodic
              H are expected in /H/ verbs is the same: the final
              syllable of the stem. The passive H would be assigned
              directly to the penultimat mora of the long final
              syllable, and the melodic H would be assigned to the
              final mora by \regel{Final MHA}, later
              shifted to the penultimate mora by \regel{Final Rise
              Elimination}. Only one H appears in the
              forms in \REF{ex:xIndefFutPassives} . I analyze that
              H as the melodic H.



\subsubsection{Pattern 5b: Other Verbal
              Contexts}\label{sec:sP5bOtherTenses}

The Indefinite Future Negative is the only other
              construction that selects the melody that
              characterizes the Indefinite Future affirmative. The
              Indefinite Future Negative exhibits \regel{Initial Lowering},
              which lowers macrostem-initial Hs, and is marked by a
              melodic H which targets the second stem mora (/Ø/
              verbs) or the FV (/H/ verbs). As well, the choice of
              verbal subject does not impact stem tone and the
              melodic H is lost when Indefinite Future Negative
              forms are not final within the phrase. Finally, the
              passive H is realized on the final syllable in /Ø/
              stems when the passive suffix is present.

 
\ea\label{ex:xP5bTenses} 
Other Pattern 5b Verbal
                Contexts 


\begin{tabular}{llll}  
  a.  &   Indefinite Future Negative  &   
                       \vernacular{
                      SP-li[ROOT-a]}  &  \\
\end{tabular}
%\caption{\nocaption}
    
\z

 Observe that the melodic H is realized on the FV
              in /H/ verbs and on the second stem mora in /Ø/ verbs
              in the most morphologically simple forms. 

 
\ea\label{ex:xP5bHStems} 
Morphologically Simple /H/ Stems \gloss{‘s/he will
                not...’}[SB] \footnote{\label{fn:nP5bGlosses} The examples included in the current section
                  use \vernacular{
                  -khálak-} \gloss{‘cut’}and \vernacular{
                  -lakhuul-} \gloss{‘release’}as
                  representative of /H/ and /Ø/ verbal roots,
                  respectively. The basic gloss for the Indefinite
                  Future is \gloss{‘s/he
                  will...’}.


}%



\begin{tabular}{llllll}  
    &   Subj  &   Tns  &   Stem  &   Neg  &  \\
Indef Fut Neg  &   
                       \vernacular{a-}  &   
                       \vernacular{li}  &   
                       \vernacular{
                      \ob [khalaká]\cb }  &   
                       \vernacular{tá}  &  \\
\end{tabular}
%\caption{\nocaption}
    
\z

 
\ea\label{ex:xP5bØStems} 
Morphologically Simple /Ø/ Stems \gloss{‘s/he will
                not...’}


\begin{tabular}{llllll}  
    &   Subj  &   Tns  &   Stem  &   Neg  &  \\
Indef Fut Neg  &   
                       \vernacular{a-}  &   
                       \vernacular{li}  &   
                       \vernacular{
                      \ob [lakhúula]\cb }  &   
                       \vernacular{
                      tá(awe)}  &  \\
\end{tabular}
%\caption{\nocaption}
    
\z

 The absence of downstep between the melodic H and
              the H of the negative marker \vernacular{tá}is
              unexpected considering that in all the data presented
              to this point, \vernacular{
              tá(awe)}has been downstepped related to
              stem-final melodic Hs. See § \sectref{sec:sPattern8} on the
              Habitual, another constrution that has mysterious
              patterns of downstep. While it is more common for
              productions to include the melodic H, JI and others
              who participated in the survey within Idakholand
              appear to have a free tonal variant without a melodic
              H on the stem at all, instead surfacing with the same
              stem tone patterns as phrase-medial forms followed by
              a H-toned word H span. This may indicate that the
              status of \vernacular{
              tá(awe)}as a trigger (or not) for \regel{Phrase-Medial Melodic H
              Deletion}is in flux.

 As in the Indefinite Future, the H contributed by
              a single object prefix fails to surface, but the
              presence of the object prefix pushes the root H out
              of macrostem-initial position, allowing it to
              re-emerge in /H/ stems. The melodic H continues to be
              realized on the second stem mora in /Ø/ stems. 

 
\ea\label{ex:xP5bOPHStems} 
/H/ Stems with an Object Prefix \gloss{‘s/he will
                not...him/her’}


\begin{tabular}{lllllll}  
    &   Subj  &   Tns  &   Obj  &   Stem  &   Neg  &  \\
Indef Fut Neg  &   
                       \vernacular{a-}  &   
                       \vernacular{li}  &   
                       \vernacular{\ob mu}  &   
                       \vernacular{
                      [khá{\downstep}láká]\cb }  &   
                       \vernacular{
                      tá(awe)}  &  \\
\end{tabular}
%\caption{\nocaption}
    
\z

 
\ea\label{ex:xP5bOPØStems} 
/Ø/ Stems with an Object Prefix \gloss{‘s/he will
                not...him/her’}


\begin{tabular}{lllllll}  
    &   Subj  &   Tns  &   Obj  &   Stem  &   Neg  &  \\
Indef Fut Neg  &   
                       \vernacular{a-}  &   
                       \vernacular{li-}  &   
                       \vernacular{\ob mu}  &   
                       \vernacular{
                      [lakhúula]\cb }  &   
                       \vernacular{
                      tá(awe)}  &  \\
\end{tabular}
%\caption{\nocaption}
    
\z

 The study’s consultants tended not to downstep
              the H of the negative particle \vernacular{tá(awe)}in
              /H/ verbs, instead more commonly, though not
              exclusively, producing the negative H at or slightly
              above the pitch of the FV. Note that the same is not
              true of cases in which the melodic H surfaces on the
              FV in /Ø/ verbs. That is, in CV(V) and CVCV /Ø/
              stems, the melodic H is realized on the FV, and the H
              of the negative particle \vernacular{tá(awe)}is
              downstepped relative to the melodic H. Again, it
              seems as though the consultants are in some cases
              treating \vernacular{
              tá(awe)}as a bonafide word capable of
              triggering \regel{Phrase-Medial Melodic H
              Deletion}.

 As in the Indefinite Future, the Indefinite Future
              Negative loses the melodic H phrase-medially.
              Examples are provided below of a /H/ verb \vernacular{-khálak-} \gloss{‘cut’}and /Ø/
              verb \vernacular{-lakhuul-} \gloss{
              ‘release’}before a H-toned noun, \vernacular{
              mú{\downstep}yáyi} \gloss{‘boy’}, and a
              toneless noun \vernacular{muundu} \gloss{
              ‘person/somebody’}. The H of the H-toned
              complement \vernacular{
              mú{\downstep}yáyi}spreads left via \regel{H Tone
              Anticipation}onto the post-peninitial
              syllable of /H/ stems and the initial syllable (or
              beyond) of /Ø/ stems.

 
\ea\label{ex:xP5bPhraseMed} 
Verbal Contexts Like the Indefinite
                Future Phrase Medially 


\begin{tabular}{llll}  
  
                       \textbf{Indef Fut Neg}  &   
                    /H/  &   
                       \vernacular{a-li\ob [khalaká]\cb 
                      mú{\downstep}yáyi tá(awe)}  &  \\

                       \vernacular{a-li\ob [khalaká]\cb 
                      múúndú tá(awe)}  &  \\
  &     &  \\

                    /Ø/  &   
                       \vernacular{
                      a-li\ob [lákhúúlá]\cb  mú{\downstep}yáyi
                      tá(awe)}  &  \\

                       \vernacular{
                      a-li\ob [lákhúúlá]\cb  múúndú
                      tá(awe)}  &  \\
\end{tabular}
%\caption{\nocaption}
    
\z

 No data concerning the impact of subject choice
              on the stem tone of Indefinite Future Negative forms
              is available. 

 Finally, the passive suffix realizes its H in the
              /Ø/, but not /H/, Indefinite Future Negative verbs as
              well. 

 
\ea\label{ex:xP5bPassive} 
/H/ \& /Ø/ Stems with the
                Passive Suffix 


\begin{tabular}{lll}  
  
                       \textbf{Indef Fut Neg}  &   
                       \vernacular{
                      a-li\ob [khalak-w-á]\cb  {\downstep}tá}  &  \\

                       \vernacular{
                      a-li\ob [lakhúul-w-á]\cb  {\downstep}tá}  &  \\
\end{tabular}
%\caption{\nocaption}
    
\z

 Note that there is downstep in the passives in \REF{ex:xP5bPassive} . These forms are crucially different
              from the cases discussed above in which no downstep
              is observed in that the final syllable is
              underlyingly bimoraic here, with one mora contributed
              by the passive suffix, the other from the FV. The
              passive H in /Ø/ verbs is assigned directly to the
              penultimate mora of the final syllable, while the
              melodic H is assigned ot the final mora first, then
              shifted to the penultimate mora by \regel{Final Rise
              Elimination}. I analyze downstep in these
              cases as resulting from a rule, \regel{Default L
              Insertion}, assigning a L tone to the
              toneless final mora prior to the final syllable
              shortening via \regel{Non-Final
              Shortening}. Downstep occurs in the above
              case, then, because of the floating L that intervenes
              between the H on the final syllable of the verb and
              the negative H.



\subsection{Pattern 5c: Conditional}\label{sec:sPattern5c}

The tonal properties of the Conditional pattern
            closely with those of the Indefinite Future, with the
            notable difference that the Conditional does not lose
            its melodic H in a phrase-medial context. The
            Conditional is marked by the \vernacular{
            ni-}particle, the FV \vernacular{-a}, and a
            melodic H which surfaces on the second stem mora in /Ø/
            verbs and the FV in /H/ verbs. The \vernacular{
            ni-}particle does not occupy the typical
            position of a tense prefix. Instead, it surfaces to the
            left of subject prefixes, which exceptionally bear a H
            in the Conditional. Because of the many formal
            similarities with Pattern 5b, the description and
            analysis of this construction will be given in
            tandem.

 As in the Indefinite Future, the melodic H does not
            surface in /H/ verbs in bimoraic stems. In all longer
            stems, a downstepped level H spans the full length of
            the stem. 

 
\ea\label{ex:xCondCH} 
Conditional C-Initial /H/ \gloss{‘if
              s/he...’}


\begin{tabular}{lllll}  
  Part  &   Subj  &   Stem  &   Gloss  &  \\

                     \vernacular{na-}  &   
                     \vernacular{á}  &   
                     \vernacular{
                    \ob [khua]\cb }  &   
                     \gloss{‘pays dowry’}  &  \\

                     \vernacular{na-}  &   
                     \vernacular{á}  &   
                     \vernacular{
                    \ob [βeka]\cb }  &   
                     \gloss{‘shaves’}  &  \\

                     \vernacular{na-}  &   
                     \vernacular{á}  &   
                     \vernacular{
                    \ob [{\downstep}téékhá]\cb }  &   
                     \gloss{‘cooks’}  &  \\

                     \vernacular{na-}  &   
                     \vernacular{á}  &   
                     \vernacular{
                    \ob [{\downstep}kháláká]\cb }  &   
                     \gloss{‘cuts’}  &  \\

                     \vernacular{na-}  &   
                     \vernacular{á}  &   
                     \vernacular{
                    \ob [{\downstep}káláángá]\cb }  &   
                     \gloss{‘fries’}  &  \\

                     \vernacular{na-}  &   
                     \vernacular{á}  &   
                     \vernacular{
                    \ob [{\downstep}βóólítsá]\cb }  &   
                     \gloss{‘seduces’}  &  \\

                     \vernacular{na-}  &   
                     \vernacular{á}  &   
                     \vernacular{
                    \ob [{\downstep}βóyóng’áná]\cb }  &   
                     \gloss{‘goes
                    around’}  &  \\
\end{tabular}
%\caption{\nocaption}
    
\z

 The root H is lowered by \regel{Initial Lowering}. In
            forms with bimoraic stems, the resultant L spreads to
            the final mora by \regel{L Spread II}, thereby
            blocking \regel{Final MHA}. \regel{Final MHA}applies in
            longer stems, assigning the melodic H to the final mora
            of the stem. Because of the subject prefix H on the
            pre-stem syllable, the melodic H then spreads left
            through the initial syllable of the stem via \regel{Plateau}.

 The vowel-initial data largely conform with the
            generalizations noted above concerning
            consonant-initial stems. In particular, the melodic H
            is assigned to and spreads left from the FV. 

 
\ea\label{ex:xCondVH} 
Conditional V-Initial /H/ \gloss{‘if
              s/he...’}


\begin{tabular}{lllll}  
  Part  &   Subj  &   Stem  &   Gloss  &  \\

                     \vernacular{ni-}  &   
                     \vernacular{yí}  &   
                     \vernacular{
                    \ob [{\downstep}írá]\cb }  &   
                     \gloss{‘kills’}  &  \\

                     \vernacular{ni-}  &   
                     \vernacular{yó}  &   
                     \vernacular{
                    \ob [{\downstep}ónóɲːá]\cb }  &   
                     \gloss{‘spoils’}  &  \\

                     \vernacular{ni-}  &   
                     \vernacular{yá}  &   
                     \vernacular{
                    \ob [{\downstep}áβúkháɲːá]\cb }  &   
                     \gloss{‘separates’}  &  \\
\end{tabular}
%\caption{\nocaption}
    
\z

 The melodic H on the FV in the form \vernacular{
            ni-yí\ob [{\downstep}írá]} \gloss{‘if s/he
            kills’}is somewhat unexpected given that
            bimoraic stems in C-initial stems do not realize the
            melodic H. However, the Conditional data is consistent
            with the parallel Indefinite Future form: \vernacular{a-li[irá]} \gloss{‘s/he will
            kill’}. As in the Indefinite Future, I argue
            that \regel{L Spread II}does not
            apply in vowel-initial stems. \regel{Final MHA}is therefore
            not blocked from applying in \gloss{‘if s/he
            kills’}.

 The melodic H appears on the first two stem moras in
            /Ø/ verbs. 

 
\ea\label{ex:xCondCØ} 
Conditional C-Initial /Ø/ \gloss{‘if s/he...’
              }


\begin{tabular}{lllll}  
  Part  &   Subj  &   Stem  &   Gloss  &  \\

                     \vernacular{na-}  &   
                     \vernacular{á}  &   
                     \vernacular{
                    \ob [{\downstep}kúa]\cb }  &   
                     \gloss{‘falls’}  &  \\

                     \vernacular{na-}  &   
                     \vernacular{á}  &   
                     \vernacular{
                    \ob [{\downstep}lékhá]\cb }  &   
                     \gloss{‘leaves’}  &  \\

                     \vernacular{na-}  &   
                     \vernacular{á}  &   
                     \vernacular{
                    \ob [{\downstep}rééβa]\cb }  &   
                     \gloss{‘asks’}  &  \\

                     \vernacular{na-}  &   
                     \vernacular{á}  &   
                     \vernacular{
                    \ob [{\downstep}kúlíkha]\cb }  &   
                   \gloss{
                  ‘names’}[SB] &  \\

                     \vernacular{na-}  &   
                     \vernacular{á}  &   
                     \vernacular{
                    \ob [{\downstep}lákhúula]\cb }  &   
                     \gloss{‘releases’}  &  \\

                     \vernacular{na-}  &   
                     \vernacular{á}  &   
                     \vernacular{
                    \ob [{\downstep}sééβula]\cb }  &   
                     \gloss{‘says goodbye
                    (to)’}  &  \\

                     \vernacular{na-}  &   
                     \vernacular{á}  &   
                     \vernacular{
                    \ob [{\downstep}kálúshitsa]\cb }  &   
                   \gloss{
                  ‘returns’}[SB] &  \\

                     \vernacular{na-}  &   
                     \vernacular{á}  &   
                     \vernacular{
                    \ob [{\downstep}síínjílitsa]\cb }  &   
                   \gloss{‘makes
                  stand’}[SB] &  \\

                     \vernacular{na-}  &   
                     \vernacular{á}  &   
                     \vernacular{
                    \ob [{\downstep}séβúlúkhaɲːa]\cb }  &   
                   \gloss{
                  ‘scatters’}[SB] &  \\
\end{tabular}
%\caption{\nocaption}
    
\z

 The melodic H is assigned to the second stem mora
            by \regel{Default MHA}, then
            spreads left by \regel{Plateau}.

 Vowel-initial stems also realize the melodic H as a
            downstepped H span on the first two stem moras, as
            shown in \REF{ex:xCondVØ} .

 
\ea\label{ex:xCondVØ} 
Conditional V-Initial /Ø/ \gloss{‘if
              s/he...’}


\begin{tabular}{lllll}  
  Part  &   Subj  &   Stem  &   Gloss  &  \\

                     \vernacular{ni-}  &   
                     \vernacular{yé}  &   
                     \vernacular{
                    \ob [{\downstep}ényá]\cb }  &   
                     \gloss{‘wants’}  &  \\

                     \vernacular{ni-}  &   
                     \vernacular{yé}  &   
                     \vernacular{
                    \ob [{\downstep}éyéla]\cb }  &   
                     \gloss{‘wipes for’}  &  \\

                     \vernacular{ni-}  &   
                     \vernacular{yá}  &   
                     \vernacular{
                    \ob [{\downstep}ámbákhana]\cb }  &   
                     \gloss{‘refuses’}  &  \\
\end{tabular}
%\caption{\nocaption}
    
\z


\subsubsection{Conditional with Object Prefixes}\label{sec:sP5cObjects}

As in the Indefinite Future, the root H surfaces \textit{in situ}in Conditional
              forms with one object prefix, and the melodic H
              surfaces on the peninitial through the final moras as
              a level downstepped H span. Because the subject
              prefix is H, the root H spreads onto the object
              prefix via \regel{Plateau}.

 
\ea\label{ex:xCondCHOP} 
Conditional C-Initial /H/ + OP \gloss{‘if s/he...him/her’
                }


\begin{tabular}{llllll}  
  Part  &   Subj  &   Obj  &   Stem  &   Gloss  &  \\

                       \vernacular{na-}  &   
                       \vernacular{á}  &   
                       \vernacular{\ob {\downstep}mú}  &   
                       \vernacular{
                      [khúa]\cb }  &   
                       \gloss{‘pays dowry
                      (for)’}  &  \\

                       \vernacular{na-}  &   
                       \vernacular{á}  &   
                       \vernacular{\ob {\downstep}mú}  &   
                       \vernacular{
                      [βé{\downstep}ká]\cb }  &   
                       \gloss{‘bites’}  &  \\

                       \vernacular{na-}  &   
                       \vernacular{á}  &   
                       \vernacular{\ob {\downstep}mú}  &   
                       \vernacular{
                      [lé{\downstep}érá]\cb }  &   
                       \gloss{‘brings’}  &  \\

                       \vernacular{na-}  &   
                       \vernacular{á}  &   
                       \vernacular{\ob {\downstep}mú}  &   
                       \vernacular{
                      [khá{\downstep}láká]\cb }  &   
                       \gloss{‘cuts’}  &  \\

                       \vernacular{na-}  &   
                       \vernacular{á}  &   
                       \vernacular{\ob {\downstep}mú}  &   
                       \vernacular{
                      [βó{\downstep}ólítsá]\cb }  &   
                       \gloss{‘seduces’}  &  \\

                       \vernacular{na-}  &   
                       \vernacular{á}  &   
                       \vernacular{\ob {\downstep}mú}  &   
                       \vernacular{
                      [khó{\downstep}ng’óóndá]\cb }  &   
                       \gloss{‘knocks’}  &  \\

                       \vernacular{na-}  &   
                       \vernacular{á}  &   
                       \vernacular{\ob {\downstep}mú}  &   
                       \vernacular{
                      [βó{\downstep}yóng’áná]\cb }  &   
                       \gloss{‘goes
                      around’}  &  \\
\end{tabular}
%\caption{\nocaption}
    
\z

 In /Ø/ stems, the melodic H is realized on the
              second stem mora, just as in forms with no object
              prefix. As usual, the H of the object prefix does not
              surface, though the melodic H spreads left onto the
              stem-initial mora and the object prefix via \regel{Plateau}.

 
\ea\label{ex:xCondCØOP} 
Conditional C-Initial /Ø/ + OP \gloss{‘if
                s/he...him/her/them
                }


\begin{tabular}{llllll}  
  Part  &   Subj  &   Obj  &   Stem  &   Gloss  &  \\

                       \vernacular{na-}  &   
                       \vernacular{á}  &   
                       \vernacular{\ob {\downstep}mú}  &   
                       \vernacular{
                      [tsía]\cb }  &   
                       \gloss{‘goes
                      (for)’}  &  \\

                       \vernacular{na-}  &   
                       \vernacular{á}  &   
                       \vernacular{\ob {\downstep}mú}  &   
                       \vernacular{
                      [lékhá]\cb }  &   
                       \gloss{‘leaves’}  &  \\

                       \vernacular{na-}  &   
                       \vernacular{á}  &   
                       \vernacular{\ob {\downstep}mú}  &   
                       \vernacular{
                      [lóónda]\cb }  &   
                       \gloss{‘follows’}  &  \\

                       \vernacular{na-}  &   
                       \vernacular{á}  &   
                       \vernacular{\ob {\downstep}mú}  &   
                       \vernacular{
                      [kúlíkha]\cb }  &   
                       \gloss{‘names’}  &  \\

                       \vernacular{na-}  &   
                       \vernacular{á}  &   
                       \vernacular{\ob {\downstep}mú}  &   
                       \vernacular{
                      [sééβula]\cb }  &   
                       \gloss{‘says
                      goodbye’}  &  \\

                       \vernacular{na-}  &   
                       \vernacular{á}  &   
                       \vernacular{\ob {\downstep}mú}  &   
                       \vernacular{
                      [kálúshitsa]\cb }  &   
                       \gloss{‘returns’}  &  \\

                       \vernacular{na-}  &   
                       \vernacular{á}  &   
                       \vernacular{\ob {\downstep}βí}  &   
                       \vernacular{
                      [séβúlúkhaɲːa]\cb }  &   
                       \gloss{‘scatters’}  &  \\
\end{tabular}
%\caption{\nocaption}
    
\z

 The expected suite of changes are observed when a
              second object prefix is added. In both /H/ and /Ø/
              verbs, the long pre-stem syllable realizes a level \vernacular{{\downstep}}H following
              the H of the subject prefix: the macrostem-initial
              object prefix H is lowered by \regel{Initial Lowering},
              and the H of the second object prefix surfaces \textit{in situ}and spreads
              left via \regel{Plateau}.
              Immediately following the H of the second object
              prefix, the root H is deleted by \regel{Meeussen’s Rule}.
              The melodic H surfaces in its expected position in
              both verb types: /H/ verbs take a H on the FV; /Ø/
              verbs take a H on the second stem mora. In both
              cases, the melodic H spreads left through the initial
              stem syllable via \regel{Plateau}.

 
\ea\label{ex:xCondCHOPOP1sg} 
Conditional C-Initial /H/ + OP + OP \textsubscript{1sg} \gloss{‘if s/he...him/her
                for me’}


\begin{tabular}{lllllll}  
  Part  &   Subj  &   Obj
                     \textsubscript{CV} &   Obj
                     \textsubscript{1sg} &   Stem  &   Gloss  &  \\

                       \vernacular{na-}  &   
                       \vernacular{á-}  &   
                       \vernacular{
                      \ob {\downstep}mú-}  &   
                       \vernacular{ú}  &   
                       \vernacular{
                      [{\downstep}ndéélá]\cb }  &   
                       \gloss{‘buries’}  &  \\

                       \vernacular{na-}  &   
                       \vernacular{á-}  &   
                       \vernacular{
                      \ob {\downstep}mú-}  &   
                       \vernacular{ú}  &   
                       \vernacular{
                      [{\downstep}mbéchélá]\cb }  &   
                       \gloss{‘shaves’}  &  \\

                       \vernacular{na-}  &   
                       \vernacular{á-}  &   
                       \vernacular{
                      \ob {\downstep}mú-}  &   
                       \vernacular{ú}  &   
                       \vernacular{
                      [{\downstep}ndéérélá]\cb }  &   
                       \gloss{‘brings’}  &  \\

                       \vernacular{na-}  &   
                       \vernacular{á-}  &   
                       \vernacular{
                      \ob {\downstep}mú-}  &   
                       \vernacular{ú}  &   
                       \vernacular{
                      [{\downstep}mbóólítsílá]\cb }  &   
                       \gloss{‘seduces’}  &  \\

                       \vernacular{na-}  &   
                       \vernacular{á-}  &   
                       \vernacular{
                      \ob {\downstep}mú-}  &   
                       \vernacular{ú}  &   
                       \vernacular{
                      [{\downstep}mbóhólólélá]\cb }  &   
                       \gloss{‘unties’}  &  \\
\end{tabular}
%\caption{\nocaption}
    
\z

 
\ea\label{ex:xCondCØOPOP1sg} 
Conditional C-Initial /Ø/ + OP + OP \textsubscript{1sg} \gloss{‘if s/he...him/her
                for me’}[SB]


\begin{tabular}{lllllll}  
  Part  &   Subj  &   Obj
                     \textsubscript{CV} &   Obj
                     \textsubscript{1sg} &   Stem  &   Gloss  &  \\

                       \vernacular{na-}  &   
                       \vernacular{á-}  &   
                       \vernacular{
                      \ob {\downstep}mú-}  &   
                       \vernacular{ú}  &   
                       \vernacular{
                      [{\downstep}nzííla]\cb }  &   
                       \gloss{‘goes
                      (for)’}  &  \\

                       \vernacular{na-}  &   
                       \vernacular{á-}  &   
                       \vernacular{
                      \ob {\downstep}mú-}  &   
                       \vernacular{ú}  &   
                       \vernacular{
                      [{\downstep}ndéshéla]\cb }  &   
                       \gloss{‘leaves’}  &  \\

                       \vernacular{na-}  &   
                       \vernacular{á-}  &   
                       \vernacular{
                      \ob {\downstep}mú-}  &   
                       \vernacular{ú}  &   
                       \vernacular{
                      [{\downstep}nóóndela]\cb }  &   
                       \gloss{‘follows’}  &  \\

                       \vernacular{na-}  &   
                       \vernacular{á-}  &   
                       \vernacular{
                      \ob {\downstep}mú-}  &   
                       \vernacular{ú}  &   
                       \vernacular{
                      [{\downstep}ndákhúulila]\cb }  &   
                       \gloss{‘releases’}  &  \\

                       \vernacular{na-}  &   
                       \vernacular{á-}  &   
                       \vernacular{
                      \ob {\downstep}mú-}  &   
                       \vernacular{ú}  &   
                       \vernacular{
                      [{\downstep}síínjílitsila]\cb }  &   
                       \gloss{‘makes
                      stand’}  &  \\
\end{tabular}
%\caption{\nocaption}
    
\z

 In summary, the core tonal properties of the
              Conditional are the following: (i) underlying
              macrostem-initial Hs fail to surface, (ii) the
              melodic H surfaces on the second stem mora in /Ø/
              verbs, (iii) the melodic H surfaces on the final
              syllable in /H/ verbs, (iv) underlying (or
              intermediate representations which include) HØ(Ø \textsubscript{0})H and HL(LØ \textsubscript{0})H sequences are
              realized as H \vernacular{{\downstep}}H(H \textsubscript{0})H and (v) root
              Hs in verb forms with an object prefix shift to the
              second mora of long stem-initial syllables. These
              properties are summarized schematically in the
              following display.

 
\ea\label{ex:xCondSchematic} 
A Schematic Representation of the
                Conditional’s Tonal Properties 


\begin{tabular}{lllll}  
    &   \multicolumn{3}{l}{
                       \ul{/H/ Verbs} } &  \\
  &   
                       \textit{Part + Subj}  &   \multicolumn{2}{l}{
                       \textit{Macrostem} } &  \\
OPsx0  &   
                       \vernacular{na-á}  &   
                       \vernacular{\ob }  &   
                       \vernacular{[C
                      }  &  \\
OPsx1  &   
                       \vernacular{na-á}  &   
                       \vernacular{\ob C
                      }  &   
                       \vernacular{[C
                      }  &  \\
OPsx2  &   
                       \vernacular{na-á}  &   
                       \vernacular{\ob {\downstep}C
                      }  &   
                       \vernacular{[{\downstep}C
                      }  &  \\
  &     &  \\
  &   \multicolumn{3}{l}{
                       \textbf{
                        } } &  \\
  &   
                       \textit{Part + Subj}  &   \multicolumn{2}{l}{
                       \textit{Macrostem} } &  \\
OPsx0  &   
                       \vernacular{na-á}  &   
                       \vernacular{\ob }  &   
                     \vernacular{[{\downstep}C
                    }\cb  &  \\
OPsx1  &   
                       \vernacular{na-á}  &   
                       \vernacular{\ob C
                      }  &   
                       \vernacular{
                      }  &  \\
OPsx2  &   
                       \vernacular{na-á}  &   
                       \vernacular{\ob {\downstep}C
                      }  &   
                       \vernacular{[{\downstep}C
                      }  &  \\
\end{tabular}
%\caption{\nocaption}
    
\z

 Because of the high degree of formal similarity
              between the Conditional and other Pattern 5 melody
              sub-types, the analysis of the generalizations noted
              in \REF{ex:xCondSchematic} and the preceding
              paragraph has been given in tandem with the relevant
              description.



\subsubsection{Conditional: Phrase Medially}\label{sec:sP5cPhraseMed}

The Conditional is distinct from other Pattern 5
              contexts in that nonfinality within the phrase does
              not condition the loss of the melodic H and \regel{H Tone
              Anticipation}does not spread post-verbal Hs
              onto the verb stem. Instead, /H/ verbs realize (i)
              the melodic H on the FV in all forms and (ii) the
              root H in forms containing an object prefix. /Ø/
              verbs take a melodic H on the second stem mora, which
              spreads left via \regel{Plateau}and doubles
              onto the following mora via \regel{Pre-Penultimate
              Doubling}. Note that, because \regel{Pre-Penultimate
              Doubling}is sensitive to the position of
              the H targeted for spreading within the phrase,
              rather than position within the verb, the rule
              applies even when it does not in parallel
              phrase-final forms, cf. \vernacular{
              na-á\ob [{\downstep}sééβula]\cb } \gloss{‘if s/he says
              goodbye’}and \vernacular{
              na-á\ob {\downstep}mú[tsííla]\cb } \gloss{‘if s/he goes for
              him/her’}.

 Four pairs of /H/ and /Ø/ verbs are provided
              below, half with and half without an object prefix.
              For each pair, the first member involves a H-toned
              complement \vernacular{mú{\downstep}yáyi} \vernacular{‘boy’}, while
              the second involves a toneless complement \vernacular{muundu} \vernacular{
              ‘somebody’}.

 
\ea\label{ex:xCondPhraseMedial} 
Conditional Phrase Medially \gloss{‘if s/he...(for
                him/her)’
                }


\begin{tabular}{lllll}  
  
                       %\includegraphics[width=\textwidth]{InkScape%20Images/H%20Stems.svg}
 &   
                       %\includegraphics[width=\textwidth]{InkScape%20Images/No%20OP.svg}
 &   
                       \vernacular{na-á\ob [ra]\cb 
                      mú{\downstep}yáyi}  &   
                       \gloss{‘buries the
                      boy’}  &  \\

                       \vernacular{na-á\ob [ra]\cb 
                      muundu}  &   
                       \gloss{‘buries
                      somebody’}  &  \\
  &     &  \\

                       \vernacular{
                      na-á\ob [khalaká]\cb  {\downstep}mú{\downstep}yáyi}  &   
                       \gloss{‘cuts the
                      boy’}  &  \\

                       \vernacular{
                      na-á\ob [khalaká]\cb  muundu}  &   
                       \gloss{‘cuts
                      somebody’}  &  \\
  &     &     &  \\

                       %\includegraphics[width=\textwidth]{InkScape%20Images/One%20OP.svg}
 &   
                       \vernacular{
                      na-á\ob {\downstep}mú[réé{\downstep}lá]\cb  {\downstep}mú{\downstep}yáyi}  &   
                       \gloss{‘buries the
                      boy’}  &  \\

                       \vernacular{
                      na-á\ob {\downstep}mú[réé{\downstep}lá]\cb  muundu}  &   
                       \gloss{‘buries
                      somebody’}  &  \\
  &     &  \\

                       \vernacular{
                      na-á\ob {\downstep}mú[khá{\downstep}láchílá]\cb 
                      {\downstep}mú{\downstep}yáyi}  &   
                       \gloss{‘cuts the
                      boy’}  &  \\

                       \vernacular{
                      na-á\ob {\downstep}mú[khá{\downstep}láchílá]\cb 
                      muundu}  &   
                       \gloss{‘cuts
                      somebody’}  &  \\
  &     &   \multicolumn{1}{l}{ } &   \multicolumn{1}{l}{ } &  \\

                       %\includegraphics[width=\textwidth]{InkScape%20Images/0%20Stems.svg}
 &   
                       %\includegraphics[width=\textwidth]{InkScape%20Images/No%20OP.svg}
 &   
                       \vernacular{na-á\ob [{\downstep}tsyá]\cb 
                      {\downstep}mú{\downstep}yáyi}  &   
                       \gloss{‘goes for the
                      boy’}  &  \\

                       \vernacular{na-á\ob [{\downstep}tsyá]\cb 
                      muundu}  &   
                       \gloss{‘goes for
                      somebody’}  &  \\
  &     &  \\

                       \vernacular{
                      na-á\ob [{\downstep}séébúla]\cb  mú{\downstep}yáyi}  &   
                       \gloss{‘says goodbye to the
                      boy’}  &  \\

                       \vernacular{
                      na-á\ob [{\downstep}séébúla]\cb  muundu}  &   
                       \gloss{‘says goodbye to
                      somebody’}  &  \\
  &     &     &  \\

                       %\includegraphics[width=\textwidth]{InkScape%20Images/One%20OP.svg}
 &   
                       \vernacular{
                      na-á\ob {\downstep}mú[tsíílá]\cb  {\downstep}mú{\downstep}yáyi}  &   
                       \gloss{‘goes for the
                      boy’}  &  \\

                       \vernacular{
                      na-á\ob {\downstep}mú[tsíílá]\cb  muundu}  &   
                       \gloss{‘goes for
                      somebody’}  &  \\
  &     &  \\

                       \vernacular{
                      na-á\ob {\downstep}mú[sééβúlila]\cb 
                      mú{\downstep}yáyi}  &   
                       \gloss{‘says goodbye to the
                      boy’}  &  \\

                       \vernacular{
                      na-á\ob {\downstep}mú[sééβúlila]\cb  muundu}  &   
                       \gloss{‘says goodbye to
                      somebody’}  &  \\
\end{tabular}
%\caption{\nocaption}
    
\z



\subsubsection{Conditional: Impact of Subject
              Choice}\label{sec:sP5cSubjects}

Data was unfortunately not elicited to test
              whether the Conditional exhibits subject-induced
              tonal alternations. 



\subsubsection{Conditional: Passives}\label{sec:sP5cPassives}

The passive suffix predictably contributes a H in
              the Conditional, though this is evident only in /Ø/
              stems of sufficient length. In /H/ verbs, the melodic
              H surface on the penultimate mora of the long final
              syllable. It is first assigned to the final mora by \regel{Final MHA}and later
              shifted to the penultimate mora by \regel{Final Rise
              Elimination}. The passive suffix
              contributes a H in the /Ø/ Conditional forms below
              because they satisfy the criteria whereby passive Hs
              are licensed: (i) the Conditional is a context
              inflected with a melodic H and (ii) the melodic H
              surfaces on the verb stem (§ \sectref{sec:sP2aOtherTenses} ).

 
\ea\label{ex:xCondPassives} 
Conditional: Passives \gloss{‘if s/he
                is...’}[SB]


\begin{tabular}{lllll}  
  \multicolumn{2}{l}{/H/ Stems } &   \multicolumn{2}{l}{/Ø/ Stems } &  \\

                       \vernacular{
                      a-li[khalak-ú-a]}  &   
                       \gloss{‘cut’}  &   
                       \vernacular{
                      a-li[lakhú{\downstep}úl-ú-a]}  &   
                       \gloss{‘released’}  &  \\

                       \vernacular{
                      a-li[tsuunzuun-ú-a]}  &   
                       \gloss{‘sucked’}  &   
                       \vernacular{
                      a-li[kalú{\downstep}shíts-ú-a]}  &   
                       \gloss{‘returned’}  &  \\
\end{tabular}
%\caption{\nocaption}
    
\z



\subsection{Summary of Pattern 5}\label{sec:sP5InterimSumm}

Though the tonal melodies comprising Pattern 5 are
            distinct, the basic surface tone patterns of most verbs
            exhibiting these melodies are shared among all three
            Pattern 5 sub-types: /H/ verbs have a melodic H near or
            at the right edge of the verb stem, /Ø/ verbs take a
            melodic H on the second stem mora, and \regel{Initial
            Lowering}applies throughout. Finally, these
            three melodies together motivate a rule \regel{L Spread II}, which
            influences the realization of melodic Hs in shorter
            stems.

 The Present and the Indefinite Future further share
            in common the loss of the melodic H in a phrase-medial
            context. There is a high degree of overlap in the
            surface tonal properties of these two contexts, such
            that it is only evident that the Present targets the
            third stem syllable, while the Indefinite Future
            targets the FV for melodic H assignment in uncommonly
            long stems. It seems likely that this similarity
            between the melodies of the Present and Indefinite
            Future tenses is responsible for the synchronic
            intra-speaker variation noted in § \sectref{sec:sPattern5b} , whereby
            forms inflected for the Indefinite Future optionally
            take the melody associated with the Present.

 The tonal properties of the Indefinite Future are
            nearly identical to those of the Conditional, with two
            exceptions. The following are true of the Conditional,
            but not the Indefinite Future: (i) subject prefixes are
            H and (ii) the melodic H is not lost
            phrase-medially. 



\section{Pattern 6: The Imperative patterns}\label{sec:sPattern6}

Imperatives exhibit a unique tonal melody with two
          melodic Hs: one on the final vowel, and a second in /Ø/
          verbs on the first or second stem mora. In addition,
          Imperatives motivate a rightward spreading rule, which is
          unique to the imperative contexts. 


\subsection{Pattern 6: Imperative
            }\label{sec:sP6xImpSg}

The Imperative \textsubscript{sg}illustrates the
            properties of Pattern 6. The Imperative \textsubscript{sg}has no overt
            inflection preceding the macrostem, and in forms
            without an object prefix, the FV \vernacular{-a}is
            selected.

 In this pattern, /H/ verbs have a H just on the FV.
            As in Patterns 5b and 5c (§ \sectref{sec:sPattern5b} - \sectref{sec:sPattern5c} )―other
            contexts with melodic H on the FV―bimoraic /H/ verbs do
            not accommodate the melodic H on the FV.

 
\ea\label{ex:xImpSgCH} 
Imperative \textsubscript{sg}C-Initial /H/ \gloss{
              ‘...!’}


\begin{tabular}{lll}  
  Stem  &   Gloss  &  \\

                     \vernacular{
                    \ob [khua]\cb }  &   
                     \gloss{‘pay dowry’}  &  \\

                     \vernacular{
                    \ob [βeka]\cb }  &   
                     \gloss{‘shave’}  &  \\

                     \vernacular{
                    \ob [teekhá]\cb }  &   
                     \gloss{‘cook’}  &  \\

                     \vernacular{
                    \ob [khalaká]\cb }  &   
                     \gloss{‘cut’}  &  \\

                     \vernacular{
                    \ob [kalaangá]\cb }  &   
                     \gloss{‘fry’}  &  \\

                     \vernacular{
                    \ob [saanditsá]\cb }  &   
                     \gloss{‘thank’}  &  \\

                     \vernacular{
                    \ob [βoyong’aná]\cb }  &   
                     \gloss{‘go around’}  &  \\
\end{tabular}
%\caption{\nocaption}
    
\z

 Note that the same pattern obtains in V-Initial /H/
            stems, which are realized with an epenthetic glide word
            initially. 

 
\ea\label{ex:xImpSgVH} 
Imperative \textsubscript{sg}V-Initial /H/ \gloss{
              ‘...!’}


\begin{tabular}{lll}  
  Stem  &   Gloss  &  \\

                     \vernacular{
                    \ob [yira]\cb }  &   
                     \gloss{‘kill’}  &  \\

                     \vernacular{
                    \ob [yononyinyá]\cb }  &   
                     \gloss{‘spoil’}  &  \\

                     \vernacular{
                    \ob [yabukhanyinyá]\cb }  &   
                     \gloss{‘separate’}  &  \\
\end{tabular}
%\caption{\nocaption}
    
\z

 In \REF{ex:xImpSgCØ} and \REF{ex:xImpSgVØ} , we see that /Ø/ verbs one level H
            spanning from beginning of the stem through the
            penultimate mora and a second \vernacular{{\downstep}}H on the
            FV.

 
\ea\label{ex:xImpSgCØ} 
Imperative \textsubscript{sg}C-Initial /Ø/ \gloss{
              ‘...!’}


\begin{tabular}{lll}  
  Stem  &   Gloss  &  \\

                     \vernacular{
                    \ob [kúa]\cb }  &   
                     \gloss{‘fall’}  &  \\

                     \vernacular{
                    \ob [lé{\downstep}khá]\cb }  &   
                     \gloss{‘leave’}  &  \\

                     \vernacular{
                    \ob [réé{\downstep}bá]\cb }  &   
                     \gloss{‘ask’}  &  \\

                     \vernacular{
                    \ob [lóó{\downstep}ndá]\cb }  &   
                     \gloss{‘follow’}  &  \\

                     \vernacular{
                    \ob [kúlí{\downstep}khá]\cb }  &   
                     \gloss{‘name’}  &  \\

                     \vernacular{
                    \ob [lákhúú{\downstep}lá]\cb }  &   
                     \gloss{‘release’}  &  \\

                     \vernacular{
                    \ob [séébú{\downstep}lá]\cb }  &   
                     \gloss{‘say
                    goodbye’}  &  \\

                     \vernacular{
                    \ob [kálúshí{\downstep}tsá]\cb }  &   
                   \gloss{
                  ‘return’}[SB] &  \\
\end{tabular}
%\caption{\nocaption}
    
\z

 
\ea\label{ex:xImpSgVØ} 
Imperative \textsubscript{sg}V-Initial /Ø/ \gloss{
              ‘...!’}


\begin{tabular}{lll}  
  Stem  &   Gloss  &  \\

                     \vernacular{
                    \ob [yé{\downstep}nyá]\cb }  &   
                     \gloss{‘want’}  &  \\

                     \vernacular{
                    \ob [yéyé{\downstep}lá]\cb }  &   
                     \gloss{‘wipe for’}  &  \\

                     \vernacular{
                    \ob [yílúú{\downstep}lá]\cb }  &   
                     \gloss{‘winnow’}  &  \\

                     \vernacular{
                    \ob [yámbákhá{\downstep}ná]\cb }  &   
                     \gloss{‘refuse’}  &  \\
\end{tabular}
%\caption{\nocaption}
    
\z


\subsubsection{Imperative
              }\label{sec:sP6xObjects}

One salient difference between Imperative \textsubscript{sg}constructions
              with and without object prefixes is that, in the
              forms with an object prefix, a harmonizing front
              vowel \vernacular{
              -i/-ɪ/-ɛ}is selected for the FV―a
              characteristic of the Subjunctive―rather than \vernacular{-a}. One
              further difference is that rightward spreading from
              the initial mora of the stem is not observed in forms
              with an object prefix; instead, the second melodic H
              appears on the second stem mora in /Ø/ verbs.

 When a CV- object prefix is present, /H/ verbs
              realize the root H in addition to a melodic H
              spanning from the peninitial mora through the FV. A
              novel feature of /H/ verbs in the Imperative is that
              root Hs are realized as a rising tone if the initial
              syllable is long; this in contrast to the observation
              that the root H, when it surfaces, does so on the
              stem-initial mora in all non-imperative contexts. The
              melodic H assigned to the FV spreads left up to the
              stem-initial H, after which it is downstepped. 

 
\ea\label{ex:xImpSgCHOP} 
Imperative \textsubscript{sg}C-Initial /H/
                + OP \gloss{
                ‘...him/her!’}


\begin{tabular}{llll}  
  Obj  &   Stem  &   Gloss  &  \\

                       \vernacular{\ob mu}  &   
                       \vernacular{
                      [rɛ́ɛ]\cb }  &   
                       \gloss{‘bury’}  &  \\

                       \vernacular{\ob mu}  &   
                       \vernacular{
                      [βé{\downstep}chɛ́]\cb }  &   
                       \gloss{‘shave’}  &  \\

                       \vernacular{\ob mu}  &   
                       \vernacular{
                      [leé{\downstep}rɛ́]\cb }  &   
                       \gloss{‘bring’}  &  \\

                       \vernacular{\ob mu}  &   
                       \vernacular{
                      [khá{\downstep}láchɛ́]\cb }  &   
                       \gloss{‘cut’ [JI]}  &  \\

                       \vernacular{\ob mu}  &   
                       \vernacular{
                      [βoó{\downstep}lítsɪ́]\cb }  &   
                       \gloss{‘seduce’
                      [JI]}  &  \\

                       \vernacular{\ob mu}  &   
                       \vernacular{
                      [βó{\downstep}yóng’ánɛ́]\cb }  &   
                       \gloss{‘go
                      around’}  &  \\
\end{tabular}
%\caption{\nocaption}
    
\z

 
\ea\label{ex:xImpSgVHOP} 
Imperative \textsubscript{sg}V-Initial /H/
                + OP \gloss{
                ‘...him/her!’}


\begin{tabular}{llll}  
  Obj  &   Stem  &   Gloss  &  \\

                       \vernacular{\ob mwi}  &   
                       \vernacular{
                      [í{\downstep}rɪ́]\cb }  &   
                       \gloss{‘kill’}  &  \\

                       \vernacular{\ob mwo}  &   
                       \vernacular{
                      [ó{\downstep}nóɲːɪ́]\cb }  &   
                       \gloss{‘spoil’}  &  \\

                       \vernacular{\ob mwa}  &   
                       \vernacular{
                      [á{\downstep}βúkháɲːɪ́]\cb }  &   
                       \gloss{‘separate’}  &  \\
\end{tabular}
%\caption{\nocaption}
    
\z

 A considerable amount of variation is observed in
              /Ø/ stems with object prefixes. The variation centers
              on whether the H near the left edge surfaces on the
              final mora of the initial syllable or on the second
              stem mora. In forms containing long initial
              syllables, the generalizations overlap. They differ
              only with respect to forms with short initial
              syllables: in one case, /Ø/ verbs with an object
              prefix will realize the H on the initial syllable,
              and in the other, the H will surface on the second
              syllable. 

 For the purposes of the analysis developed below,
              it is assumed that forms which realize the left edge
              H on the second stem mora are basic, though it
              appears as though there is a change in progress
              whereby the /Ø/ stems are taking on the pattern of
              the /H/ stems in which the left edge H shifts only in
              heavy initial syllables. Two factors which may be
              contributing to variation in this context are: (i)
              the unusual parallelism between root H and melodic H
              position in long initial syllables and (ii) the rule
              of melodic H assignment that targets the initial mora
              of the stem in /Ø/ without object prefixes. 

 In stems of sufficient length, the melodic H
              assigned to the FV will spread left onto the
              post-peninitial mora via \regel{Plateau}.

 
\ea\label{ex:xImpSgCØOP} 
Imperative \textsubscript{sg}C-Initial /Ø/
                + OP \gloss{
                ‘...him/her!’} \footnote{\label{fn:nIngosiImpSgØOPData} JI favors realizing the left-edge melodic H on
                  the initial stem syllable when it is short,
                  rather than on the second stem syllable. 


}%



\begin{tabular}{llll}  
  Obj  &   Stem  &   Gloss  &  \\

                       \vernacular{\ob mu}  &   
                       \vernacular{
                      [tsíi]\cb }  &   
                       \gloss{‘go (for)’}  &  \\

                       \vernacular{\ob mu}  &   
                       \vernacular{
                      [leshɛ́]\cb }  &   
                       \gloss{‘leave’
                      [SB]}  &  \\

                       \vernacular{\ob mu}  &   
                       \vernacular{
                      [loó{\downstep}dɛ́]\cb }  &   
                       \gloss{‘follow’}  &  \\

                       \vernacular{\ob mu}  &   
                       \vernacular{
                      [kulí{\downstep}shɪ́]\cb }  &   
                       \gloss{‘cut’ [SB]}  &  \\

                       \vernacular{\ob mu}  &   
                       \vernacular{
                      [seé{\downstep}βúlɪ́]\cb }  &   
                       \gloss{‘say goodbye
                      (to)’}  &  \\

                       \vernacular{\ob mu}  &   
                       \vernacular{
                      [kalú{\downstep}kháɲːɪ́]\cb }  &   
                       \gloss{‘turn
                      over’}  &  \\
\end{tabular}
%\caption{\nocaption}
    
\z

 In /Ø/ V-Initial stems with an object prefix,
              both SB and JI prefer the pattern which characterizes
              /H/ V-Initial stems: the melodic H on the left edge
              surfaces on the first mora of the stem, rather than
              the second. See § \sectref{sec:cPathToPredictability} for discussion of additional instances
              of diachronic change in which vowel-initial stems
              neutralize the lexical contrast in advance of
              C-Initial stems.

 
\ea\label{ex:xImpSgVØOP} 
Imperative \textsubscript{sg}V-Initial /Ø/
                + OP \gloss{
                ‘...him/her!’}


\begin{tabular}{llll}  
  Obj  &   Stem  &   Gloss  &  \\

                       \vernacular{\ob mwe}  &   
                       \vernacular{
                      [é{\downstep}nyɛ́]\cb }  &   
                       \gloss{‘kill’}  &  \\

                       \vernacular{\ob mwo}  &   
                       \vernacular{
                      [ó{\downstep}nóɲːɪ́]\cb }  &   
                       \gloss{‘spoil’}  &  \\

                       \vernacular{\ob mwa}  &   
                       \vernacular{
                      [á{\downstep}βúkháɲːɪ́]\cb }  &   
                       \gloss{‘separate’}  &  \\
\end{tabular}
%\caption{\nocaption}
    
\z

 Imperative \textsubscript{sg}forms with a 1 \textsuperscript{st}sg object
              prefix exhibit the same tonal properties as forms
              with a CV- object prefix, as shown below. Curiously,
              the verb selects the FV \vernacular{-a},
              rather than \vernacular{
              -i/-ɪ/-ɛ}.

 
\ea\label{ex:xImpSgCHOP1sg} 
Imperative \textsubscript{sg}C-Initial /H/
                + OP \textsubscript{1sg} \gloss{‘...me!’
                }


\begin{tabular}{llll}  
  Obj  &   Stem  &   Gloss  &  \\

                       \vernacular{\ob m}  &   
                       \vernacular{
                      [bé{\downstep}ká]\cb }  &   
                       \gloss{‘shave’}  &  \\

                       \vernacular{\ob n}  &   
                       \vernacular{
                      [deé{\downstep}rá]\cb }  &   
                       \gloss{‘bring’
                      }  &  \\

                       \vernacular{\ob Ø}  &   
                       \vernacular{
                      [khá{\downstep}láká]\cb }  &   
                       \gloss{‘cut’ [JI]}  &  \\

                       \vernacular{\ob m}  &   
                       \vernacular{
                      [boó{\downstep}lítsá]\cb }  &   
                       \gloss{‘seduce’
                      [JI]}  &  \\

                       \vernacular{\ob m}  &   
                       \vernacular{
                      [bó{\downstep}yóng’áná]\cb }  &   
                       \gloss{‘go
                      around’}  &  \\
\end{tabular}
%\caption{\nocaption}
    
\z

 
\ea\label{ex:xImpSgCØOP1sg} 
Imperative \textsubscript{sg}C-Initial /Ø/
                + OP \textsubscript{1sg} \gloss{‘...me!’
                }


\begin{tabular}{llll}  
  Obj  &   Stem  &   Gloss  &  \\

                       \vernacular{\ob n}  &   
                       \vernacular{
                      [dé{\downstep}khá]\cb }  &   
                       \gloss{‘leave’}  &  \\

                       \vernacular{\ob n}  &   
                       \vernacular{
                      [doó{\downstep}dá]\cb }  &   
                       \gloss{‘follow’}  &  \\

                       \vernacular{\ob n}  &   
                       \vernacular{
                      [gulí{\downstep}khá]\cb }  &   
                       \gloss{‘cut’}  &  \\

                       \vernacular{\ob Ø}  &   
                       \vernacular{
                      [seé{\downstep}βúlá]\cb }  &   
                       \gloss{‘say goodbye
                      (to)’}  &  \\

                       \vernacular{\ob n}  &   
                       \vernacular{
                      [galú{\downstep}kháɲːá]\cb }  &   
                       \gloss{‘turn
                      over’}  &  \\
\end{tabular}
%\caption{\nocaption}
    
\z

 In forms with two object prefixes, the pre-stem
              syllable surfaces with a rising tone. The root H
              fails to surface in /H/ stems, but /Ø/ verbs realize
              the melodic H on the first two stem moras. Both verb
              classes realize the melodic H on the FV, which
              spreads leftward via \regel{Plateau}where
              possible. \footnote{\label{fn:nImpSgOPx2} Note that JI, who produces /Ø/ stems with the
                /H/ stem pattern, does not realize the left-edge H
                in /Ø/ stems with two object prefixes. 


}%


 
\ea\label{ex:xImpSgCHOPx2} 
Imperative \textsubscript{sg}C-Initial /H/
                + OP + OP \textsubscript{1sg} \gloss{‘...him/her for
                me!’}


\begin{tabular}{lllll}  
  Obj
                     \textsubscript{CV} &   Obj
                     \textsubscript{1sg} &   Stem  &   Gloss  &  \\

                       \vernacular{\ob mu-}  &   
                       \vernacular{ú}  &   
                       \vernacular{
                      [{\downstep}ndéélɛ́]\cb }  &   
                       \gloss{‘bury’}  &  \\

                       \vernacular{\ob mu-}  &   
                       \vernacular{ú}  &   
                       \vernacular{
                      [{\downstep}mbéchélɛ́]\cb }  &   
                       \gloss{‘shave’}  &  \\

                       \vernacular{\ob mu-}  &   
                       \vernacular{ú}  &   
                       \vernacular{
                      [{\downstep}ndéérélɛ́]\cb }  &   
                       \gloss{‘bring’}  &  \\

                       \vernacular{\ob mu-}  &   
                       \vernacular{ú}  &   
                       \vernacular{
                      [{\downstep}kháláchílɪ́]\cb }  &   
                       \gloss{‘cut’}  &  \\
\end{tabular}
%\caption{\nocaption}
    
\z

 
\ea\label{ex:xImpSgCØOPx2} 
Imperative \textsubscript{sg}C-Initial /Ø/
                + OP + OP \textsubscript{1sg} \gloss{‘...him/her for
                me!’}[SB]


\begin{tabular}{lllll}  
  Obj
                     \textsubscript{CV} &   Obj
                     \textsubscript{1sg} &   Stem  &   Gloss  &  \\

                       \vernacular{\ob mu-}  &   
                       \vernacular{ú}  &   
                       \vernacular{
                      [{\downstep}nzíí{\downstep}lɪ́]\cb }  &   
                       \gloss{‘go (for)’}  &  \\

                       \vernacular{\ob mu-}  &   
                       \vernacular{ú}  &   
                       \vernacular{
                      [{\downstep}ndéshé{\downstep}lɛ́]\cb }  &   
                       \gloss{‘leave’}  &  \\

                       \vernacular{\ob mu-}  &   
                       \vernacular{ú}  &   
                       \vernacular{
                      [{\downstep}nóó{\downstep}ndélɛ́]\cb }  &   
                       \gloss{‘follow’}  &  \\

                       \vernacular{\ob mu-}  &   
                       \vernacular{ú}  &   
                       \vernacular{
                      [{\downstep}ngúlí{\downstep}shílɪ́]\cb }  &   
                       \gloss{‘name’}  &  \\
\end{tabular}
%\caption{\nocaption}
    
\z

 The principal properties of the Imperative \textsubscript{sg}are summarized
              schematically in \REF{ex:xImpSgSchematic} below. Forms
              without object prefixes fail to realize the lexical H
              but do realize a melodic H on the FV in both /H/ and
              /Ø/ verbs. In addition, /Ø/ verbs realize an
              additional melodic H which spans from the initial
              mora through the penultimate mora. In forms with
              object prefixes, macrostem-initial Hs fail to
              surface. When just one object prefix is present, /H/
              verbs realize the root H and a melodic H spanning
              across the remaining stem syllables, and /Ø/ verbs
              realize a H on the second stem mora and a second H
              spanning across all moras to the right of the first
              melodic H. In forms with two object prefixes the H of
              the innermost object prefix surfaces, resulting in a
              rising tone on the pre-stem syllable. /H/ verbs
              realize a single span across the full length of the
              verb stem, while /Ø/ verbs realize two H spans: one
              comprising the first two moras of the stem and
              another comprising the rest of the stem.

 
\ea\label{ex:xImpSgSchematic} 
A Schematic Representation of the
                Tonal Properties of the Imperative \textsubscript{
                sg}


\begin{tabular}{llll}  
    &   \multicolumn{2}{l}{
                       \ul{/H/ Verbs} } &  \\
  &   
                       \textit{Obj}  &   \multicolumn{1}{l}{
                       \textit{Stem} } &  \\
OPsx0  &   
                       \vernacular{\ob }  &   
                       \vernacular{[C
                      }  &  \\
OPsx1  &   
                       \vernacular{\ob C
                      }  &   
                       \vernacular{[C(
                      }  &  \\
OPsx2  &   
                       \vernacular{\ob C
                      }  &   
                       \vernacular{[{\downstep}C
                      }  &  \\
  &     &     &  \\
  &   \multicolumn{2}{l}{
                       \textbf{
                        } } &  \\
  &   
                       \textit{Obj}  &   \multicolumn{1}{l}{
                       \textit{Stem} } &  \\
OPsx0  &   
                       \vernacular{\ob }  &   
                     \vernacular{[C
                    }\cb  &  \\
OPsx1  &   
                       \vernacular{\ob C
                      }  &   
                       \vernacular{[CV(C)
                      }  &  \\
OPsx2  &   
                       \vernacular{\ob C
                      }  &   
                       \vernacular{[{\downstep}C
                      }  &  \\
\end{tabular}
%\caption{\nocaption}
    
\z

 Below, I offer an analysis of the Imperative \textsubscript{sg}that unifies
              constructions with and without object prefixes. The
              broad approach of the analysis is to describe the
              tonal patterns of Imperative \textsubscript{sg}constructions in
              terms of three distinct melodic H assignment rules
              targeting the first, second, and final moras of the
              stem, along with two spreading rules which generate H
              spans emanating from the melodic Hs. The analysis
              additionally exploits \regel{Initial
              Lowering}.

 Among the rules alluded to above, two are unique
              to affirmative imperative constructions: \regel{Initial MHA}and \regel{Rightward Spread}. \regel{Initial MHA}assigns
              a melodic H to the initial mora of the stem, and \regel{Rightward
              Spread}subsequently spreads melodic Hs in
              stem-initial position iteratively rightward.

 
\ea\label{ex:xInitialMHA} 
 \regel{Initial Melodic H
                  Assignment} 

%\includegraphics[width=\textwidth]{InkScape%20Images/Rules/InitialMHA.pdf}

\z

 
\ea\label{ex:xRightwardSpread} 
 \regel{Rightward
                  Spread} 

%\includegraphics[width=\textwidth]{InkScape%20Images/Rules/RightwardSpread.pdf}

\z

 Because the analysis of the imperatives is
              complex, all but one of the rules relevant to
              imperative constructions are included in each of the
              derivations to follow. \regel{L Spread II}is
              omitted because the role it serves in accounting for
              the failure of the melodic H to surface in /H/ CVV
              and CVCV stems in the imperative is the same as its
              role in the Indefinite Future, as detailed in § \sectref{sec:sP5bObjects} , and
              illustrated in \REF{ex:xDerivIndefFutHShort} . \footnote{\label{fn:nExtraSupport} In the Indefinite Future, \regel{L Spread
                II}spreads the output L of \regel{Initial
                Lowering}onto the second mora. This L
                blocks \regel{Final MHA}from
                assigning the melodic H to the FV in bimoraic,
                C-initial stems. In bimoraic \textit{V-initial}stems, \regel{L Spread II}does
                not apply because the output L of \regel{Initial
                Lowering}is not aligned with the left
                edge of a syllable boundary, as in \vernacular{
                a-li\ob [irá]\cb } \gloss{‘s/he will
                kill’}.

 The fact that the same vowel-initial root takes
                on the C-initial pattern (i.e. no H on bimoraic
                stems) in the Imperative \textsubscript{sg}context
                confirms this analysis. Though /H/ verbs in both
                the Indefinite Future and the Imperative \textsubscript{sg}are assigned a
                melodic H through the same mechanisms ( \regel{Initial Lowering}, \regel{L Spread II}, and \regel{Final MHA}), these
                two constructions differ with respect to whether
                the verb \gloss{
                ‘kill’}realizes a H on the FV. In the
                Indefinite Future is does, as shown in the
                preceding paragraph, and in the Imperative \textsubscript{sg}, it does not,
                as in \vernacular{
                \ob [yira]\cb } \gloss{‘kill!’}.
                The epenthetic glide in the Imperative \textsubscript{sg}creates the
                environment for \regel{L Spread II}to
                apply by aligning the output L of \regel{Initial
                Lowering}with a syllable boundary. The
                application of \regel{L Spread II}bleeds \regel{Final MHA}.


}%


 In /Ø/ verbs, the derivation proceeds in the
              following steps: (i) one melodic H is assigned to the
              initial mora of the stem via \regel{Initial MHA} \REF{ex:xInitialMHA} , (ii) a second melodic H is assigned to
              the FV via \regel{Final MHA} \REF{ex:xFinalMHA} , and the leftmost melodic H spreads
              iteratively right through the penult via an
              Imperative-specific rule of \regel{Rightward Spread} \REF{ex:xRightwardSpread} . Note that \regel{Rightward
              Spread}crucially applies before \regel{Plateau}to capture
              the fact that the leftmost melodic H should spread
              rightward rather than the rightmost melodic H
              leftward.

 
\ea\label{ex:xDerivImpSgØ} 
 Derivation,
                  /Ø/ Imperative
                   \vernacular{
                  \ob [kúlí{\downstep}khá]\cb } \gloss{‘name!’} 

%\includegraphics[width=\textwidth]{InkScape%20Images/Derivations/DerivImpSg0.pdf}

\z

 In trimoraic and longer stems (cf. \vernacular{
              \ob [khalaká]\cb } \gloss{‘cut!’}), /H/
              verbs realize a melodic H on the FV only. The absence
              of a second melodic H spanning throughout the
              remainder of the verb stem is built into the formal
              statement of \regel{Initial MHA}as given
              in \REF{ex:xInitialMHA} : this rule of melodic H assignment
              requires that the target mora be toneless. Regardless
              of this rule’s ordering relationship with \regel{Initial Lowering},
              the initial mora of the stem will be toned and so
              block the application of \regel{Initial MHA}.

 
\ea\label{ex:xDerivImpSgH} 
 Derivation,
                  /H/ Imperative
                   \vernacular{
                  \ob [khalaká]\cb } \gloss{‘cut!’} 

%\includegraphics[width=\textwidth]{InkScape%20Images/Derivations/DerivImpSgH.pdf}

\z

 Imperative \textsubscript{sg}forms with
              object prefixes require further elaboration of the
              analysis developed thus far. All verbs realize one
              melodic H on the FV, which arrives at this position
              via \regel{Final MHA}. An
              additional shifting rule is recruited to account for
              tonal properties near the left stem boundary.

 Note that the root H surfaces in forms with object
              prefixes, but does not spread by \regel{Rightward Spread}.
              The statement of the rule given in \REF{ex:xRightwardSpread} captures this
              observation by specifying that only melodic Hs are
              targeted for \regel{Rightward
              Spread}.

 Instead, the root H is affected by a rule, \regel{Heavy Shift}, which
              shifts the root H to the second stem mora only within
              a long initial syllable. When the initial syllable is
              short, the root H surfaces \textit{in situ}. I formulate
              the rule in \REF{ex:xHeavyShift} .

 
\ea\label{ex:xHeavyShift} 
 \regel{Heavy
                  Shift} 

%\includegraphics[width=\textwidth]{InkScape%20Images/Rules/HeavyShift.pdf}

\z

 We next turn to the question of why the leftmost
              melodic H of /Ø/ verbs is not assigned to the
              stem-initial and spread right in forms with an object
              prefix, as it is in forms without. Instead, the
              leftmost melodic H surfaces on the second stem mora.
              In this context, the H of the object prefix is
              lowered via \regel{Initial Lowering}and
              one melodic H is assigned to the FV by \regel{Final MHA}; because \regel{Initial MHA}requires
              that the mora preceding the target be toneless, \regel{Initial MHA}fails to
              apply. This leaves the second melodic H free to be
              assigned to the stem by the later applying \regel{Default MHA}, which
              targets the second mora of the stem. \regel{Plateau}then applies
              to spread the melodic H on the FV leftward until the
              resulting span abuts with the leftmost melodic H on
              the second stem mora.

 
\ea\label{ex:xDerivImpSgØOP} 
 Derivation,
                  /Ø/ Imperative
                   \vernacular{
                  \ob mu[seé{\downstep}βúlɪ́]\cb } \gloss{‘say goodbye to
                  him/her!’} 

%\includegraphics[width=\textwidth]{InkScape%20Images/Derivations/DerivImpSg0OP.pdf}

\z

 Most of the properties of /H/ stems involving two
              object prefixes are straightforwardly derived from
              the analysis presented thus far. The right edge
              melodic H surfaces on the FV as a result of \regel{Final MHA}. That the
              pre-stem syllable realizes a rise, rather than a
              fall, follows from the ordering \regel{of Initial
              Lowering}before \regel{Meeussen's Rule}(as
              discussed with reference to Pattern 2a, § \sectref{sec:sP2aObjects} ). I
              attribute the failure of the root H to be realized in
              forms with short stem-initial syllables to the
              application of \regel{Meeussen's Rule},
              which deletes the root H immediately following the H
              of the innermost object prefix.

 What remains to be explained, then, is the failure
              of the root H to be realized in forms with long
              stem-initial syllables, in which the root H could
              plausibly escape deletion by \regel{Meeussen's Rule}by
              shifting to the second stem mora via \regel{Heavy Shift}. That
              the root H does not surface in this case indicates
              that \regel{Meeussen's
              Rule}precedes \regel{Heavy Shift}in the
              derivation.

 
\ea\label{ex:xDerivImpSgHOPx2} 
 Derivation,
                  /H/ Imperative
                   \vernacular{
                  \ob mu[seé{\downstep}βúlɪ́]\cb } \gloss{‘say goodbye to
                  him/her!’} 

%\includegraphics[width=\textwidth]{InkScape%20Images/Derivations/DerivImpSgHOPx2.pdf}

\z



\subsubsection{Imperative
              }\label{sec:sP6xPhraseMed}

Imperatives exhibit an unusual pattern in which
              constructions which include an object prefix retain
              the melodic H, while it is lost in constructions
              without an object prefix. \regel{Initial
              Lowering}remains in force in both cases.
              Recall that the melodic H targeting the second mora
              of the stem will double onto a pre-penultimate
              syllable within the phrase.

 Four pairs of /H/ and /Ø/ stems are provided
              below, half with and half without an object prefix.
              For each pair, the first member involves a H-toned
              complement \vernacular{
              mú{\downstep}yáyi} \gloss{‘boy’}, while
              the second involves a toneless complement \vernacular{muundu} \gloss{
              ‘somebody’}.

 
\ea\label{ex:xImpSgPhraseMedial} 
Imperative \textsubscript{sg}Phrase
                Medially \gloss{‘...(for
                him/her)!’}


\begin{tabular}{lllll}  
  
                       %\includegraphics[width=\textwidth]{InkScape%20Images/H%20Stems.svg}
 &   
                       %\includegraphics[width=\textwidth]{InkScape%20Images/No%20OP.svg}
 &   
                       \vernacular{\ob [ra]\cb 
                      mú{\downstep}yáyi}  &   
                       \gloss{‘bury the
                      boy’}  &  \\

                       \vernacular{\ob [ra]\cb 
                      muundu}  &   
                       \gloss{‘bury
                      somebody’}  &  \\
  &     &  \\

                       \vernacular{\ob [khalaká]\cb 
                      mú{\downstep}yáyi}  &   
                       \gloss{‘cut the
                      boy’}  &  \\

                       \vernacular{\ob [khalaka]\cb 
                      muundu}  &   
                       \gloss{‘cut
                      somebody’}  &  \\
  &     &     &  \\

                       %\includegraphics[width=\textwidth]{InkScape%20Images/One%20OP.svg}
 &   
                       \vernacular{
                      \ob mu[βé{\downstep}chélɛ́]\cb  {\downstep}mú{\downstep}saatsa}  &   
                       \gloss{‘bury the
                      man’}  &  \\

                       \vernacular{
                      \ob mu[βé{\downstep}chélɛ́]\cb  muundu}  &   
                       \gloss{‘bury
                      somebody’}  &  \\
  &     &  \\

                       \vernacular{
                      \ob mu[khá{\downstep}láchílɪ́]\cb  {\downstep}mú{\downstep}yáyi}  &   
                       \gloss{‘cut the
                      boy’}  &  \\

                       \vernacular{
                      \ob mu[khá{\downstep}láchílɪ́]\cb  muundu}  &   
                       \gloss{‘cut
                      somebody’}  &  \\
  &     &     &  \\

                       %\includegraphics[width=\textwidth]{InkScape%20Images/0%20Stems.svg}
 &   
                       %\includegraphics[width=\textwidth]{InkScape%20Images/No%20OP.svg}
 &   
                       \vernacular{\ob [tsyá]\cb 
                      mú{\downstep}yáyi}  &   
                       \gloss{‘go for the
                      boy’}  &  \\

                       \vernacular{\ob [tsya]\cb 
                      muundu}  &   
                       \gloss{‘go for
                      somebody’}  &  \\
  &     &  \\

                       \vernacular{\ob [sééβúlá]\cb 
                      mú{\downstep}yáyi}  &   
                       \gloss{‘say goodbye to the
                      boy’}  &  \\

                       \vernacular{\ob [seeβula]\cb 
                      muundu}  &   
                       \gloss{‘say goodbye to
                      somebody’}  &  \\
  &     &     &  \\

                       %\includegraphics[width=\textwidth]{InkScape%20Images/One%20OP.svg}
 &   
                       \vernacular{\ob mu[tsiílí]\cb 
                      {\downstep}músáatsa}  &   
                     \gloss{‘go for the
                    man’}[SB] &  \\

                       \vernacular{\ob mu[tsiílí]\cb 
                      muundu}  &   
                     \gloss{‘go for
                    somebody’}[SB] &  \\
  &     &  \\

                       \vernacular{
                      \ob mu[seéβú{\downstep}lílɪ́]\cb  {\downstep}mú{\downstep}yáyi}  &   
                     \gloss{‘say goodbye to the
                    boy’}[SB] &  \\

                       \vernacular{
                      mu[seéβú{\downstep}lílɪ́]\cb  muundu}  &   
                     \gloss{‘say goodbye to
                    somebody’}[SB] &  \\
\end{tabular}
%\caption{\nocaption}
    
\z



\subsubsection{Imperative
              }\label{sec:sP6xSubjects}

The Imperative \textsubscript{sg}does not take a
              subject prefix, and the Imperative \textsubscript{pl}takes only the 2 \textsuperscript{nd}plural subject
              prefix. Thus, we cannot test the choice of subject
              prefix in the Imperative \textsubscript{sg}.



\subsubsection{Imperative
              }\label{sec:sP6xPassives}

The tonal properties of Imperative \textsubscript{sg}forms are the
              nearly identical with and without the passive suffix.
              Both tonal classes have a H on the final syllable,
              and /Ø/ verbs have an additional H spanning all
              preceding syllables of the stem. One difference is
              that, because the passive suffix adds a mora to the
              final syllable and \regel{Final Rise
              Elimination}renders final rises as falling
              tones, Imperative \textsubscript{sg}forms have a
              falling contour rather than a level H on the final
              syllable.

 
\ea\label{ex:xImpSgPassives} 
Imperative \textsubscript{sg}: Passives \gloss{
                ‘be...!’}[SB]


\begin{tabular}{lllll}  
  \multicolumn{2}{l}{/H/ Stems } &   \multicolumn{2}{l}{/Ø/ Stems } &  \\

                       \vernacular{
                      \ob [khalak-ú-a]\cb }  &   
                       \gloss{‘cut’}  &   
                       \vernacular{
                      y-a\ob [lákhúú{\downstep}l-ú-a]\cb }  &   
                       \gloss{‘released’}  &  \\

                       \vernacular{
                      \ob [tsúúnzúún-ú-a]\cb 
                      }  &   
                       \gloss{‘sucked’}  &   
                       \vernacular{
                      y-a\ob [kálúshí{\downstep}ts-ú-a]\cb }  &   
                       \gloss{‘returned’}  &  \\
\end{tabular}
%\caption{\nocaption}
    
\z



\subsubsection{Pattern 6: Other Verbal Contexts}\label{sec:sP6xOtherTenses}

The tonal properties of the Imperative \textsubscript{pl}are identical to
              those of its singular counterpart. In particular, the
              Imperative \textsubscript{pl}takes two
              melodic Hs: one targeting the FV in all verbs, and an
              additional melodic H in /Ø/ verbs that surfaces
              either as a span on all stem moras preceding the FV
              or on the second mora of the stem, depending on
              whether an object prefix is present. Additionally,
              while phrase-medial Imperative \textsubscript{pl}forms without
              any object prefixes lose all melodic Hs, melodic Hs
              are retained phrase-medially if an object prefix is
              present.

 The Imperative \textsubscript{pl}selects the FV \vernacular{-i}and
              takes no subject prefix.

 
\ea\label{ex:xP6xTenses} 
Other Pattern 6 Verbal
                Contexts 


\begin{tabular}{llll}  
  a.  &   Imperative
                     \textsubscript{pl} &   
                       \vernacular{
                      [ROOT-i]}  &  \\
\end{tabular}
%\caption{\nocaption}
    
\z

 In /H/ verbs without an object prefix, a melodic
              H is assigned to the FV, and the root H is not
              realized due to \regel{Initial Lowering}.
              /Ø/ verbs also take a melodic H on the FV, but also
              assign a rightward spreading melodic H to the
              stem-initial mora.

 
\ea\label{ex:xP6xHStems} 
Morphologically Simple /H/ Stems
                [SB] \footnote{\label{fn:nP6xGlosses} The examples included in the current section
                  use \vernacular{
                  -khálak-} \gloss{‘cut’}and \vernacular{
                  -βóyong’an-} \gloss{‘go around’}to
                  illustrate the properties of /H/ verbs, and \vernacular{
                  -kulix-} \gloss{‘name’}and \vernacular{
                  -kalushits-} \gloss{‘return’}as
                  representative of /H/ and /Ø/ verbal roots,
                  respectively. The basic gloss for the Imperative \textsubscript{pl}is \gloss{‘...!’}.


}%



\begin{tabular}{lll}  
    &   Stem  &  \\
Imperative
                     \textsubscript{pl} &   
                       \vernacular{
                      \ob [khalachí]\cb }  &  \\

                       \vernacular{
                      \ob [βoyong’aní]\cb }  &  \\
\end{tabular}
%\caption{\nocaption}
    
\z

 
\ea\label{ex:xP6xØStems} 
Morphologically Simple /Ø/ Stems
                [SB] 


\begin{tabular}{lll}  
    &   Stem  &  \\
Imperative
                     \textsubscript{pl} &   
                       \vernacular{
                      \ob [kúlí{\downstep}shí]\cb }  &  \\

                       \vernacular{
                      \ob [kálúshí{\downstep}tsí]\cb }  &  \\
\end{tabular}
%\caption{\nocaption}
    
\z

 When an object prefix is present, the H it
              contributes fails to be realized. The root H surfaces
              on the initial mora, and the melodic H on the FV
              spreads left onto the peninitial mora via \regel{Plateau}. /Ø/ verbs
              take melodic Hs on the second and final moras of the
              stem, the latter of which spreads left again by \regel{Plateau}.

 
\ea\label{ex:xP6xOPHStems} 
/H/ Stems with an Object Prefix
                [SB] 


\begin{tabular}{llll}  
    &   Obj  &   Stem  &  \\
Imperative
                     \textsubscript{pl} &   
                       \vernacular{\ob mu}  &   
                       \vernacular{
                      [khá{\downstep}láchí]\cb }  &  \\

                       \vernacular{\ob mu}  &   
                       \vernacular{
                      [βó{\downstep}yóng’ání]\cb }  &  \\
\end{tabular}
%\caption{\nocaption}
    
\z

 
\ea\label{ex:xP6xOPØStems} 
/Ø/ Stems with an Object Prefix
                [SB] 


\begin{tabular}{llll}  
    &   Obj  &   Stem  &  \\
Imperative
                     \textsubscript{pl} &   
                       \vernacular{\ob mu}  &   
                       \vernacular{
                      [kúlí{\downstep}shí]\cb }  &  \\

                       \vernacular{\ob mu}  &   
                       \vernacular{
                      [kálú{\downstep}shítsí]\cb }  &  \\
\end{tabular}
%\caption{\nocaption}
    
\z

 Melodic Hs are deleted phrase-medially in the
              Imperative \textsubscript{pl}only in verb
              forms lacking an object prefix. \regel{Initial Lowering}is
              in effect, and \regel{H Tone
              Anticipation}spreads post-verbal Hs onto
              the verb stem. Examples are provided below of both
              /H/ and /Ø/ stems before a H-toned noun, \vernacular{
              mú{\downstep}yáyi} \gloss{‘boy’}, and a
              toneless noun \vernacular{muundu} \gloss{
              ‘person/somebody’}.

 
\ea\label{ex:xImpPPhraseMedial} 
Imperative \textsubscript{pl}Phrase
                Medially \gloss{‘...(for
                him/her)!’}


\begin{tabular}{llll}  
  
                       %\includegraphics[width=\textwidth]{InkScape%20Images/H%20Stems.svg}
 &   
                       \vernacular{\ob [βoolitsílí]\cb 
                      mú{\downstep}yáyi}  &   
                       \gloss{‘seduce the
                      boy’}  &  \\

                       \vernacular{\ob [βoolitsili]\cb 
                      muundu}  &   
                       \gloss{‘seduce
                      somebody’}  &  \\
  &     &  \\

                       \vernacular{
                      \ob mu[βoó{\downstep}lítsílí]\cb  {\downstep}mú{\downstep}yáyi}  &   
                       \gloss{‘seduce the
                      boy’}  &  \\

                       \vernacular{
                      \ob mu[βoó{\downstep}lítsílí]\cb  muundu}  &   
                       \gloss{‘seduce
                      somebody’}  &  \\
  &     &     &  \\

                       %\includegraphics[width=\textwidth]{InkScape%20Images/0%20Stems.svg}
 &   
                       \vernacular{
                      \ob [kálúshítsí]\cb  mú{\downstep}saatsa}  &   
                       \gloss{‘return the
                      man’}  &  \\

                       \vernacular{\ob [kalushitsi]\cb 
                      muundu}  &   
                       \gloss{‘return
                      somebody’}  &  \\
  &     &  \\

                       \vernacular{
                      \ob mu[kalúshí{\downstep}tsílí]\cb  {\downstep}mú{\downstep}yáyi}  &   
                       \gloss{‘return the
                      boy’}  &  \\

                       \vernacular{
                      \ob mu[kalúshí{\downstep}tsílí]\cb  muundu}  &   
                       \gloss{‘return
                      somebody’}  &  \\
\end{tabular}
%\caption{\nocaption}
    
\z

 Because the choice of subject is restricted to 2 \textsuperscript{nd}pl, there is
              no data available relating to the effect of subject
              choice on the tonal properties of the Imperative \textsubscript{pl}.

 The Imperatives pattern together with respect to
              the passive H. In /H/ verbs, the final syllable
              realizes a fall, while /Ø/ verbs have a final fall as
              well as a melodic H spanning from the initial to the
              penultimate syllable. 

 
\ea\label{ex:xP6xPassive} 
/H/ \& /Ø/ Stems with the
                Passive Suffix \gloss{‘do not
                be...!’}[SB]


\begin{tabular}{lllll}  
  
                       \textbf{Imperative
                      }  &   /H/  &   
                       \vernacular{
                      \ob [khalak-ú-i]\cb }  &   
                       \gloss{‘cut’}  &  \\

                       \vernacular{
                      \ob [tsuunzuun-ú-i]\cb }  &   
                       \gloss{‘sucked’}  &  \\
/Ø/  &   
                       \vernacular{
                      \ob [lákhúú{\downstep}l-ú-i]\cb }  &   
                       \gloss{‘released’}  &  \\

                       \vernacular{
                      \ob [kálúshí{\downstep}ts-ú-i]\cb }  &   
                       \gloss{‘returned’}  &  \\
\end{tabular}
%\caption{\nocaption}
    
\z



\section{Pattern 7: The second mora and final vowel
          pattern}\label{sec:sPattern7}

In the Hesternal Perfective, both /H/ and /Ø/ verbs
          surface with a melodic H on the FV. In addition, /Ø/
          verbs realize a second melodic H on the second stem mora. \regel{Initial Lowering}, as
          usual, lowers macrostem-initial Hs, and a rule of
          leftward spreading which creates long H spans across the
          length of the verb.


\subsection{Pattern 7: Hesternal Perfective}\label{sec:sPattern7x}

The Hesternal Perfective is presented as
            representative of the tonal properties of Pattern 7.
            The Hesternal Perfective is marked by the toneless
            tense prefix \vernacular{a-}, the
            perfective suffix \vernacular{-ile}—the
            complicated allomorphy of which is described in § \sectref{sec:sTenseAspectMorphemes} —and two melodic Hs.

 In /H/ verbs, the root H does not surface. Instead,
            /H/ verbs realize a melodic H which spans the full
            length of the verb stem. As is typical with H spans in
            Idakho, JI produces his with a strong crescendo effect
            throughout the verb stem, with the pitch increasing
            with each subsequent syllable, while SB’s spans
            characteristically exhibit a sharp increase in pitch on
            the final syllable. 

 
\ea\label{ex:xHestPerfCH} 
Hesternal Perfective C-Initial /H/ \gloss{‘s/he...’
              }


\begin{tabular}{lllll}  
  Subj  &   Tns  &   Stem  &   Gloss  &  \\

                     \vernacular{y}  &   
                     \vernacular{a}  &   
                     \vernacular{
                    \ob [khwéélé]\cb }  &   
                     \gloss{‘paid dowry’}  &  \\

                     \vernacular{y}  &   
                     \vernacular{a}  &   
                     \vernacular{
                    \ob [βéchí]\cb }  &   
                     \gloss{‘shaved’}  &  \\

                     \vernacular{y}  &   
                     \vernacular{a}  &   
                     \vernacular{
                    \ob [tééshí]\cb }  &   
                     \gloss{‘cooked’}  &  \\

                     \vernacular{y}  &   
                     \vernacular{a}  &   
                     \vernacular{
                    \ob [khálááchɛ́]\cb }  &   
                     \gloss{‘cut’}  &  \\

                     \vernacular{y}  &   
                     \vernacular{a}  &   
                     \vernacular{
                    \ob [káláánjí]\cb }  &   
                     \gloss{‘fried’}  &  \\

                     \vernacular{y}  &   
                     \vernacular{a}  &   
                     \vernacular{
                    \ob [βóólíítsɪ́]\cb }  &   
                     \gloss{‘thanked’}  &  \\

                     \vernacular{y}  &   
                     \vernacular{a}  &   
                     \vernacular{
                    \ob [tsúúnzúúní]\cb }  &   
                     \gloss{‘sucked’}  &  \\

                     \vernacular{y}  &   
                     \vernacular{a}  &   
                     \vernacular{
                    \ob [βóyóng’áánɛ́]\cb }  &   
                     \gloss{‘went
                    around’}  &  \\
\end{tabular}
%\caption{\nocaption}
    
\z

 An epenthetic glide appears at the beginning of /H/
            verbs with V-initial stems. These forms also surface
            all H on the stem. 

 
\ea\label{ex:xHestPerfVH} 
Hesternal Perfective V-Initial /H/ \gloss{
              ‘s/he...’}


\begin{tabular}{lllll}  
  Subj  &   Tns  &   Stem  &   Gloss  &  \\

                     \vernacular{y-}  &   
                     \vernacular{a}  &   
                     \vernacular{
                    \ob [yírɪ́]\cb }  &   
                     \gloss{‘killed’}  &  \\

                     \vernacular{y-}  &   
                     \vernacular{a}  &   
                     \vernacular{
                    \ob [yónóɲːɪ́]\cb }  &   
                     \gloss{‘spoiled’}  &  \\

                     \vernacular{y-}  &   
                     \vernacular{a}  &   
                     \vernacular{
                    \ob [yábúkháɲːɪ́]\cb }  &   
                     \gloss{‘separated’}  &  \\
\end{tabular}
%\caption{\nocaption}
    
\z

 /Ø/ verbs are characterized by two H spans on the
            verb stem. The first spans across the first two or
            three moras of the stem, while the second surfaces on
            the remainder of the stem. The initial span extends
            into the third stem mora in two cases: (i) when the
            second stem mora is the first mora in a long syllable
            or (ii) when the third mora belongs to a syllable
            preceding the penult. Bimoraic stems (second example in \REF{ex:xHestPerfCØ} ) surface with a single H spanning both
            moras.

 
\ea\label{ex:xHestPerfCØ} 
Hesternal Perfective C-Initial /Ø/ \gloss{
              ‘s/he...’}


\begin{tabular}{lllll}  
  Subj  &   Tns  &   Stem  &   Gloss  &  \\

                     \vernacular{y-}  &   
                     \vernacular{a}  &   
                     \vernacular{
                    \ob [kwíí{\downstep}lí]\cb }  &   
                     \gloss{‘fell’}  &  \\

                     \vernacular{y-}  &   
                     \vernacular{a}  &   
                     \vernacular{
                    \ob [lékhí]\cb }  &   
                     \gloss{‘left’}  &  \\

                     \vernacular{y-}  &   
                     \vernacular{a}  &   
                     \vernacular{
                    \ob [réé{\downstep}βí]\cb }  &   
                     \gloss{‘asked’}  &  \\

                     \vernacular{y-}  &   
                     \vernacular{a}  &   
                     \vernacular{\ob [kúlíí{\downstep}shɪ́]\cb 
                    }  &   
                   \gloss{
                  ‘named’}[SB] &  \\

                     \vernacular{y-}  &   
                     \vernacular{a}  &   
                     \vernacular{
                    \ob [séé{\downstep}βúúlɪ́]\cb }  &   
                     \gloss{‘said goodbye
                    (to)’}  &  \\

                     \vernacular{y-}  &   
                     \vernacular{a}  &   
                     \vernacular{
                    \ob [kálú{\downstep}shíítsɪ́]\cb }  &   
                   \gloss{
                  ‘returned’}[SB] &  \\

                     \vernacular{y-}  &   
                     \vernacular{a}  &   
                     \vernacular{
                    \ob [síínjí{\downstep}líítsɪ́]\cb }  &   
                     \gloss{‘made stand’}  &  \\

                     \vernacular{y-}  &   
                     \vernacular{a}  &   
                     \vernacular{
                    \ob [séβúlú{\downstep}kháɲííɲɪ́]\cb }  &   
                     \gloss{‘scattered’}  &  \\
\end{tabular}
%\caption{\nocaption}
    
\z

 The same generalizations that apply to C-initial
            stems apply to V-initial stems. 

 
\ea\label{ex:xHestPerfVØ} 
Hesternal Perfective V-Initial /Ø/ \gloss{
              ‘s/he...’}


\begin{tabular}{lllll}  
  Subj  &   Tns  &   Stem  &   Gloss  &  \\

                     \vernacular{y-}  &   
                     \vernacular{a}  &   
                     \vernacular{
                    [yényí]}  &   
                     \gloss{‘wanted’}  &  \\

                     \vernacular{y-}  &   
                     \vernacular{a}  &   
                     \vernacular{
                    [yéyéé{\downstep}lɛ́]}  &   
                     \gloss{‘wiped for’}  &  \\

                     \vernacular{y-}  &   
                     \vernacular{a}  &   
                     \vernacular{
                    [yílúú{\downstep}lɪ́]}  &   
                     \gloss{‘winnowed’}  &  \\

                     \vernacular{y-}  &   
                     \vernacular{a}  &   
                     \vernacular{
                    [yámbá{\downstep}kháánɛ́]}  &   
                     \gloss{‘refused’}  &  \\
\end{tabular}
%\caption{\nocaption}
    
\z

 The transcriptions in \REF{ex:xHestPerfCØ} only characterize SB’s productions of /Ø/
            verbs in the Hesternal Perfective. JI has only one H,
            which spans the entire stem, just like /H/ verbs, e.g. \vernacular{
            y-a\ob [káluśhíítsɪ́]\cb } \gloss{‘s/he
            returned’}.


\subsubsection{Hesternal Perfective with Object
              Prefixes}\label{sec:sP7xObjects}

The root H surfaces in /H/ verbs with an object
              prefix, while the H of the object prefix itself does
              not. In addition, a melodic H spans the remainder of
              the verb stem. 

 
\ea\label{ex:xHestPerfCHOP} 
Hesternal Perfective C-Initial /H/
                + OP \gloss{
                ‘s/he...him/her’}


\begin{tabular}{llllll}  
  Subj  &   Tns  &   Obj  &   Stem  &   Gloss  &  \\

                       \vernacular{y-}  &   
                       \vernacular{
                      }  &   
                       \vernacular{\ob mu}  &   
                       \vernacular{
                      [ré{\downstep}élé]\cb }  &   
                       \gloss{‘buried’}  &  \\

                       \vernacular{y-}  &   
                       \vernacular{
                      }  &   
                       \vernacular{\ob mu}  &   
                       \vernacular{
                      [βé{\downstep}chí]\cb }  &   
                       \gloss{‘shaved’}  &  \\

                       \vernacular{y-}  &   
                       \vernacular{a}  &   
                       \vernacular{\ob mu}  &   
                       \vernacular{
                      [lé{\downstep}érí]\cb }  &   
                       \gloss{‘brought’}  &  \\

                       \vernacular{y-}  &   
                       \vernacular{a}  &   
                       \vernacular{\ob mu}  &   
                       \vernacular{
                      [khá{\downstep}lááchɛ́]\cb }  &   
                       \gloss{‘cut’}  &  \\

                       \vernacular{y-}  &   
                       \vernacular{a}  &   
                       \vernacular{\ob mu}  &   
                       \vernacular{
                      [βó{\downstep}ólíítsɪ́]\cb }  &   
                       \gloss{‘seduced’}  &  \\

                       \vernacular{y-}  &   
                       \vernacular{a}  &   
                       \vernacular{\ob mu}  &   
                       \vernacular{
                      [βó{\downstep}yóng’áánɛ́]\cb }  &   
                       \gloss{‘went
                      around’}  &  \\
\end{tabular}
%\caption{\nocaption}
    
\z

 
\ea\label{ex:xHestPerfVHOP} 
Hesternal Perfective V-Initial /H/
                + OP \gloss{
                ‘s/he...him/her’}


\begin{tabular}{llllll}  
  Subj  &   Tns  &   Obj  &   Stem  &   Gloss  &  \\

                       \vernacular{y-}  &   
                       \vernacular{a}  &   
                       \vernacular{\ob mwi}  &   
                       \vernacular{
                      [í{\downstep}rí]\cb }  &   
                       \gloss{‘killed’}  &  \\

                       \vernacular{y-}  &   
                       \vernacular{a}  &   
                       \vernacular{\ob mwo}  &   
                       \vernacular{
                      [ó{\downstep}nónyíínyɪ́]\cb }  &   
                       \gloss{‘spoiled’}  &  \\

                       \vernacular{y-}  &   
                       \vernacular{a}  &   
                       \vernacular{\ob mwa}  &   
                       \vernacular{
                      [á{\downstep}βúkhányíínyɪ́]\cb }  &   
                       \gloss{
                      ‘separated’}  &  \\
\end{tabular}
%\caption{\nocaption}
    
\z

 Again the H of the object prefix fails to surface
              in /Ø/ verbs; as in forms without an object prefix,
              two melodic H spans surface on the verb stem: one
              across the first two or three moras, the other across
              the remainder of the stem. 

 
\ea\label{ex:xHestPerfCØOP} 
Hesternal Perfective C-Initial /Ø/
                + OP \gloss{‘s/he...him/her’
                }


\begin{tabular}{llllll}  
  Subj  &   Tns  &   Obj  &   Stem  &   Gloss  &  \\

                       \vernacular{y-}  &   
                       \vernacular{a}  &   
                       \vernacular{\ob mu}  &   
                       \vernacular{
                      [tsíí{\downstep}lí]\cb }  &   
                       \gloss{‘went
                      (for)’}  &  \\

                       \vernacular{y-}  &   
                       \vernacular{a}  &   
                       \vernacular{\ob mu}  &   
                       \vernacular{
                      [léshí]\cb }  &   
                       \gloss{‘left’}  &  \\

                       \vernacular{y-}  &   
                       \vernacular{a}  &   
                       \vernacular{\ob mu}  &   
                       \vernacular{
                      [lóó{\downstep}ndí]\cb }  &   
                       \gloss{‘followed’}  &  \\

                       \vernacular{y-}  &   
                       \vernacular{a}  &   
                       \vernacular{\ob mu}  &   
                       \vernacular{
                      [kúlíí{\downstep}shɪ́]\cb }  &   
                     \gloss{
                    ‘named’}[SB] &  \\

                       \vernacular{y-}  &   
                       \vernacular{a}  &   
                       \vernacular{\ob mu}  &   
                       \vernacular{
                      [séé{\downstep}βúúlɪ́]\cb }  &   
                       \gloss{‘said
                      goodbye’}  &  \\

                       \vernacular{y-}  &   
                       \vernacular{a}  &   
                       \vernacular{\ob mu}  &   
                       \vernacular{
                      [kálú{\downstep}shíítsɪ́]\cb }  &   
                     \gloss{
                    ‘returned’}[SB] &  \\

                       \vernacular{y-}  &   
                       \vernacular{a}  &   
                       \vernacular{\ob mu}  &   
                       \vernacular{
                      [séβúlú{\downstep}khányíínyɪ́]\cb }  &   
                       \gloss{
                      ‘scattered’}  &  \\
\end{tabular}
%\caption{\nocaption}
    
\z

 The tokens in my corpus of vowel-initial /Ø/
              Hesternal Perfective verbs with an object prefix do
              not follow the generalizations which hold true of
              consonant-initial /Ø/ verbs. Both speakers produce an
              internally consistent pattern in this context, but
              each produces a different pattern. 

 SB applies the pattern described above for /H/
              vowel-initial verbs, surfacing with a H on the stem
              initial mora and a downstepped H span on the
              remainder of the stem. 

 
\ea\label{ex:xHestPerfVØOPSB} 
Hesternal Perfective V-Initial /Ø/
                + OP \gloss{‘s/he...him/her/it
                }[SB]


\begin{tabular}{llllll}  
  Subj  &   Tns  &   Obj  &   Stem  &   Gloss  &  \\

                       \vernacular{y-}  &   
                       \vernacular{a}  &   
                       \vernacular{\ob mwe}  &   
                       \vernacular{
                      [é{\downstep}nyí]\cb }  &   
                       \gloss{‘wanted’}  &  \\

                       \vernacular{y-}  &   
                       \vernacular{a}  &   
                       \vernacular{\ob mwe}  &   
                       \vernacular{
                      [é{\downstep}yéélɛ́]\cb }  &   
                       \gloss{‘wiped
                      for’}  &  \\

                       \vernacular{y-}  &   
                       \vernacular{a}  &   
                       \vernacular{\ob βwi}  &   
                       \vernacular{
                      [í{\downstep}lúúlí]\cb }  &   
                       \gloss{‘winnowed’}  &  \\

                       \vernacular{y-}  &   
                       \vernacular{a}  &   
                       \vernacular{\ob mwa}  &   
                       \vernacular{
                      [á{\downstep}mbákháánɛ́]\cb }  &   
                       \gloss{‘refused’}  &  \\
\end{tabular}
%\caption{\nocaption}
    
\z

 I propose that the SB’s data constitute one of
              several instantiations within Luhya of a diachronic
              pressure to replace tonal patterns which weakly
              signal the left morphological stem boundary with
              tonal patterns that more clearly demarcate the
              boundary. This hypothesis is the topic of Ch. \sectref{sec:cPathToPredictability} .

 JI produces a pattern with only a single melodic H
              span throughout the full length of the stem. JI’s
              behavior is simply a reflection of JI’s preference
              for inflecting Hesternal Perfective /Ø/ verbs with a
              melodic H on the FV only—a preference which prevails
              even in consonant-initial stems. 

 
\ea\label{ex:xHestPerfVØOPJI} 
Hesternal Perfective V-Initial /Ø/
                + OP \gloss{‘s/he...him/her/it
                }[JI]


\begin{tabular}{llllll}  
  Subj  &   Tns  &   Obj  &   Stem  &   Gloss  &  \\

                       \vernacular{y-}  &   
                       \vernacular{a}  &   
                       \vernacular{\ob mwe}  &   
                       \vernacular{
                      [ényí]\cb }  &   
                       \gloss{‘wanted’}  &  \\

                       \vernacular{y-}  &   
                       \vernacular{a}  &   
                       \vernacular{\ob mwe}  &   
                       \vernacular{
                      [éyéélɛ́]\cb }  &   
                       \gloss{‘wiped
                      for’}  &  \\

                       \vernacular{y-}  &   
                       \vernacular{a}  &   
                       \vernacular{\ob βwi}  &   
                       \vernacular{
                      [ílúúlí]\cb }  &   
                       \gloss{‘winnowed’}  &  \\

                       \vernacular{y-}  &   
                       \vernacular{a}  &   
                       \vernacular{\ob mwa}  &   
                       \vernacular{
                      [ámbákháánɛ́]\cb }  &   
                       \gloss{‘refused’}  &  \\
\end{tabular}
%\caption{\nocaption}
    
\z

 The tonal properties of Hesternal Perfective verb
              forms involving 1 \textsuperscript{st}sg object
              prefixes are identical to those involving CV- object
              prefixes.

 
\ea\label{ex:xHestPerfCHOP1sg} 
Hesternal Perfective C-Initial /H/
                + OP \textsubscript{1sg} \gloss{
                ‘s/he...me’}


\begin{tabular}{llllll}  
  Subj  &   Tns  &   Obj  &   Stem  &   Gloss  &  \\

                       \vernacular{y-}  &   
                       \vernacular{a}  &   
                       \vernacular{\ob a}  &   
                       \vernacular{
                      [rí{\downstep}ílí]\cb }  &   
                     \gloss{
                    ‘feared’}[SB] &  \\

                       \vernacular{y-}  &   
                       \vernacular{a}  &   
                       \vernacular{\ob a}  &   
                       \vernacular{
                      [mbé{\downstep}chí]\cb }  &   
                       \gloss{‘shaved’}  &  \\

                       \vernacular{y-}  &   
                       \vernacular{a}  &   
                       \vernacular{\ob a}  &   
                       \vernacular{
                      [ndé{\downstep}érí]\cb }  &   
                       \gloss{‘brought’}  &  \\

                       \vernacular{y-}  &   
                       \vernacular{a}  &   
                       \vernacular{\ob a}  &   
                       \vernacular{
                      [khá{\downstep}lááchɛ́]\cb }  &   
                       \gloss{‘cut’}  &  \\

                       \vernacular{y-}  &   
                       \vernacular{a}  &   
                       \vernacular{\ob a}  &   
                       \vernacular{
                      [mbó{\downstep}ólíítsɪ́]\cb }  &   
                       \gloss{‘seduced’}  &  \\

                       \vernacular{y-}  &   
                       \vernacular{a}  &   
                       \vernacular{\ob a}  &   
                       \vernacular{
                      [mbó{\downstep}yóng’áánɛ́]\cb }  &   
                       \gloss{‘went
                      around’}  &  \\
\end{tabular}
%\caption{\nocaption}
    
\z

 
\ea\label{ex:xHestPerfCØOP1sg} 
Hesternal Perfective C-Initial /Ø/
                + OP \textsubscript{1sg} \gloss{
                ‘s/he...me’}[SB]


\begin{tabular}{llllll}  
  Subj  &   Tns  &   Obj  &   Stem  &   Gloss  &  \\

                       \vernacular{y-}  &   
                       \vernacular{a}  &   
                       \vernacular{\ob a}  &   
                       \vernacular{
                      [síé{\downstep}lɛ́]}  &   
                       \gloss{‘ground’}  &  \\

                       \vernacular{y-}  &   
                       \vernacular{a}  &   
                       \vernacular{\ob a}  &   
                       \vernacular{
                      [ndéshí]}  &   
                       \gloss{‘left’}  &  \\

                       \vernacular{y-}  &   
                       \vernacular{a}  &   
                       \vernacular{\ob a}  &   
                       \vernacular{
                      [nóó{\downstep}ndɛ]}  &   
                       \gloss{‘followed’}  &  \\

                       \vernacular{y-}  &   
                       \vernacular{a}  &   
                       \vernacular{\ob a}  &   
                       \vernacular{
                      [ngúlíí{\downstep}shɪ́]}  &   
                       \gloss{‘named’}  &  \\

                       \vernacular{y-}  &   
                       \vernacular{a}  &   
                       \vernacular{\ob a}  &   
                       \vernacular{
                      [séé{\downstep}βúúlɪ́]}  &   
                       \gloss{‘said goodbye
                      (to)’}  &  \\

                       \vernacular{y-}  &   
                       \vernacular{a}  &   
                       \vernacular{\ob a}  &   
                       \vernacular{
                      [ngálú{\downstep}kháɲːɪ́]}  &   
                       \gloss{‘returned’}  &  \\

                       \vernacular{y-}  &   
                       \vernacular{a}  &   
                       \vernacular{\ob a}  &   
                       \vernacular{
                      [síínjí{\downstep}líítsɪ́]}  &   
                       \gloss{‘make
                      stand’}  &  \\
\end{tabular}
%\caption{\nocaption}
    
\z

 In Hesternal Perfective forms involving both a
              CV- and a 1 \textsuperscript{st}sg object
              prefix, a rising tone surfaces on the pre-stem
              syllable. In /H/ verbs, a single level H span
              surfaces across the full length of the verb stem,
              while /Ø/ verbs again realize two melodic H spans
              with the same positional characteristics as Hesternal
              Perfective /Ø/ verb forms with fewer object
              prefixes.

 
\ea\label{ex:xHestPerfCHOPOP1sg} 
Hesternal Perfective C-Initial /H/
                + OP + OP \textsubscript{1sg} \gloss{‘s/he...him/her for
                me’}[JI] \footnote{\label{fn:nSBHestPerfOPx2} SB mysteriously produced these forms with a H
                  on the initial mora of the stem and a downstepped
                  level H span from the second stem mora through
                  the FV, cf. \vernacular{
                  y-a\ob mu-u[khá{\downstep}láchíílɪ́]\cb } \gloss{‘s/he cut him/her
                  for me’}. Despite this curiosity, he
                  produced vowel-initial /H/ verbs with two object
                  prefixes with the expected pattern, i.e. a rise
                  on the pre-stem syllable followed by a
                  downstepped melodic H spanning the full length of
                  the verb.


}%



\begin{tabular}{lllllll}  
  Subj  &   Tns  &   Obj
                     \textsubscript{CV} &   Obj
                     \textsubscript{1sg} &   Stem  &   Gloss  &  \\

                       \vernacular{y-}  &   
                       \vernacular{a}  &   
                       \vernacular{\ob mu-}  &   
                       \vernacular{ú}  &   
                       \vernacular{
                      [{\downstep}mbéchéélɛ́]\cb }  &   
                       \gloss{‘shaved’}  &  \\

                       \vernacular{y-}  &   
                       \vernacular{a}  &   
                       \vernacular{\ob mu-}  &   
                       \vernacular{ú}  &   
                       \vernacular{
                      [{\downstep}ndééléélɛ́]\cb }  &   
                       \gloss{
                      ‘buried/brought’}  &  \\

                       \vernacular{y-}  &   
                       \vernacular{
                      }  &   
                       \vernacular{\ob mu-}  &   
                       \vernacular{ú}  &   
                       \vernacular{
                      [{\downstep}kháláchíílɪ́]\cb }  &   
                       \gloss{‘cut’}  &  \\
\end{tabular}
%\caption{\nocaption}
    
\z

 
\ea\label{ex:xHestPerfCØOPOP1sg} 
Hesternal Perfective C-Initial /Ø/
                + OP + OP \textsubscript{1sg} \gloss{‘s/he...him/her for
                me’}[SB]


\begin{tabular}{lllllll}  
  Subj  &   Tns  &   Obj
                     \textsubscript{CV} &   Obj
                     \textsubscript{1sg} &   Stem  &   Gloss  &  \\

                       \vernacular{y-}  &   
                       \vernacular{
                      }  &   
                       \vernacular{\ob mu-}  &   
                       \vernacular{ú}  &   
                       \vernacular{
                      [{\downstep}ngálú{\downstep}shíílɪ́\cb }  &   
                       \gloss{‘named’}  &  \\

                       \vernacular{y-}  &   
                       \vernacular{a}  &   
                       \vernacular{\ob mu-}  &   
                       \vernacular{ú}  &   
                       \vernacular{
                      [{\downstep}ndákhú{\downstep}úlíílɪ́]\cb }  &   
                       \gloss{‘released’}  &  \\

                       \vernacular{y-}  &   
                       \vernacular{a}  &   
                       \vernacular{\ob mu-}  &   
                       \vernacular{ú}  &   
                       \vernacular{
                      [{\downstep}mbóómbé{\downstep}lítsíílɪ́]\cb }  &   
                       \gloss{‘released’}  &  \\
\end{tabular}
%\caption{\nocaption}
    
\z

 The notable features of the tonal melody
              characterizing the Hesternal Perfective are the
              following: (i) underlying macrostem-initial Hs fail
              to surface, (ii) /H/ and /Ø/ verbs realize a melodic
              H which spans from the FV throughout much of the
              stem, and (iii) /Ø/ verbs realize an additional
              melodic H span across the first 2-3 moras of the
              stem. These properties are summarized schematically
              in the following display. As elsewhere, the position
              of underlying Hs is indicated with a single
              underline, and the melodic H, when it appears, is
              indicated with double underlining. 

 
\ea\label{ex:xHestPerfSchematic} 
A Schematic Representation of the
                Hesternal Perfective’s Tonal
                Properties 


\begin{tabular}{lllll}  
    &   \multicolumn{3}{l}{
                       \ul{/H/ Verbs} } &  \\
  &   
                       \textit{Subj + Tns}  &   \multicolumn{2}{l}{
                       \textit{Macrostem} } &  \\
OPsx0  &   
                       \vernacular{y-a}  &   
                       \vernacular{\ob }  &   
                       \vernacular{[C
                      }  &  \\
OPsx1  &   
                       \vernacular{y-a}  &   
                       \vernacular{\ob C
                      }  &   
                       \vernacular{[C
                      }  &  \\
OPsx2  &   
                       \vernacular{y-a}  &   
                       \vernacular{\ob C
                      }  &   
                       \vernacular{[{\downstep}C
                      }  &  \\
  &   \multicolumn{2}{l}{ } &     &  \\
  &   \multicolumn{2}{l}{
                       \textbf{
                        } } &  \\
  &   
                       \textit{Subj + Tns}  &   \multicolumn{2}{l}{
                       \textit{Macrostem} } &  \\
OPsx0  &   
                       \vernacular{y-a}  &   
                       \vernacular{\ob }  &   
                     \vernacular{[C
                    }\cb  &  \\
OPsx1  &   
                       \vernacular{y-a}  &   
                       \vernacular{\ob C
                      }  &   
                     \vernacular{[C
                    }\cb  &  \\
OPsx2  &   
                       \vernacular{y-a}  &   
                       \vernacular{\ob C
                      }  &   
                     \vernacular{[C
                    }\cb  &  \\
\end{tabular}
%\caption{\nocaption}
    
\z

 The melodic H spanning the entire length of /H/
              verb stems may be derived in three steps: (i) \regel{Initial Lowering} \REF{ex:xInitialLowering} lowers the root
              H, (ii) \regel{Final MHA} \REF{ex:xFinalMHA} assigns a melodic H to the FV, and (iii)
              a rule of \regel{Leftward
              Spread}iteratively spreads the melodic H
              left up until the left stem boundary.

 
\ea\label{ex:xLeftwardSpread} 
 \regel{Leftward
                  Spread} 

%\includegraphics[width=\textwidth]{InkScape%20Images/Rules/LeftwardSpread.pdf}

\z

 This analysis, which accounts for the surface
              tone patterns of /H/ verbs of any stem shape, is
              modeled in the following derivation. 

 
\ea\label{ex:xDerivHestPerfH} 
 Derivation,
                  /H/ Hest. Perf.: \vernacular{
                  y-a\ob [khálááchɛ́]\cb } \gloss{‘s/he
                  cut’} 

%\includegraphics[width=\textwidth]{InkScape%20Images/Derivations/DerivHestPerfH.pdf}

\z

 The basic tone pattern in /Ø/ may be derived in a
              similar manner, except that in the case of /Ø/ verbs,
              a second melodic H assignment rule is operative. \regel{Final MHA}and \regel{Default MHA} \REF{ex:xDefaultMHA} apply to assign melodic Hs to the FV and
              the second stem mora, respectively. Subsequently,
              both melodic Hs undergo leftward spreading via \regel{Plateau}and \regel{Leftward Spread},
              again respectively.

 
\ea\label{ex:xDerivHestPerfØTrisyllabic} 
 Derivation,
                  /Ø/ Hest. Perf.: \vernacular{
                  y-a\ob [séé{\downstep}βúúlɪ́]\cb } \gloss{‘s/he said
                  goodbye’} 

%\includegraphics[width=\textwidth]{InkScape%20Images/Derivations/DerivHestPerf0Trisyllabic.pdf}

\z

 There are two cases in which the initial melodic
              H span extends through the third stem mora: in
              uncommonly long stems and when the second mora of the
              stem is the first mora of a long second
              syllable. 

 The leftmost melodic H in long /Ø/ verbs may shift
              to the right by one mora when eligible for \regel{Pre-Penultimate
              Doubling}, as in \vernacular{
              y-a\ob [séβúlú{\downstep}kháɲíínɪ́]\cb } \vernacular{‘s/he
              scattered’}. That downstep occurs between
              the third and fourth moras, rather than the second
              and third, indicates that \regel{Pre-Penultimate
              Doubling}precedes \regel{Plateau}in the
              derivation.

 
\ea\label{ex:xDerivHestPerfØLonger} 
 Derivation,
                  /Ø/ Hest. Perf.: \vernacular{
                  y-a\ob [síínjí{\downstep}líítsɪ́]\cb } \gloss{‘s/he made
                  stand’} 

%\includegraphics[width=\textwidth]{InkScape%20Images/Derivations/DerivHestPerf0Longer.pdf}

\z

 When the second mora of the stem is also the
              first mora of a long second stem syllable, an
              optional rule of \regel{Fall Elimination} \footnote{\label{fn:nFallElimInLuhya} Within Luhya, \regel{Fall
                Elimination}is also attested in Marachi ( \citealt{rMarlo2007} ) and Wanga ( \citealt{rEbarbGreenMarlo2014} ).


}%


 
\ea\label{ex:xFallElimination} 
 \regel{Fall Elimination
                  (Optional)} 

%\includegraphics[width=\textwidth]{InkScape%20Images/Rules/FallElimination.pdf}

\z

 Again, this rightward spreading rule takes
              precedence, applying before the melodic H on the FV
              spreads left via \regel{Plateau}. The
              present analysis assumes that \regel{Fall
              Elimination}applies to an intermediate
              representation which will have undergone \regel{Default MHA}and \regel{Leftward
              Spread}.

 
\ea\label{ex:xDerivHestPerfØFallElim} 
 Derivation,
                  /Ø/ Hest. Perf.: \vernacular{
                  y-a\ob [kúlíí{\downstep}shɪ́]\cb } \gloss{‘s/he
                  named’} 

%\includegraphics[width=\textwidth]{InkScape%20Images/Derivations/DerivHestPerf0FallElim.pdf}

\z

 The root H predictably is realized in Hesternal
              Perfective forms with an object prefix. While the H
              of the object prefix itself is lowered, the root H is
              shielded from \regel{Initial Lowering}.
              The melodic H on the FV spreads left onto the second
              mora of the stem.

 
\ea\label{ex:xDerivHestPerfHOP} 
 Derivation,
                  /H/ Hest. Perf. + OP: \vernacular{
                  y-a\ob mu[khá{\downstep}lááchɛ́]\cb } \gloss{‘s/he cut
                  him/her’} 

%\includegraphics[width=\textwidth]{InkScape%20Images/Derivations/DerivHestPerfHOP.pdf}

\z

 When two object prefixes are present, the root H
              is again lost; in this case, the root H is deleted
              via \regel{Meeussen’s Rule},
              clearing the way for the melodic H to spread through
              the initial mora of the stem.

 
\ea\label{ex:xDerivHestPerfHOPx2} 
 Derivation,
                  /H/ Hest. Perf. + OPx2: \vernacular{
                  y-a\ob mu-ú[{\downstep}kháláchíílɪ́]\cb } \gloss{‘s/he cut him/her
                  for me’} 

%\includegraphics[width=\textwidth]{InkScape%20Images/Derivations/DerivHestPerfHOPx2.pdf}

\z

 Object prefixes do not influence stem tone in /Ø/
              Hesternal Perfective verbs. The same analysis
              illustrated in \REF{ex:xDerivHestPerfØTrisyllabic} and \REF{ex:xDerivHestPerfØLonger} derive the
              stem tone properties of all /Ø/ verbs, and the
              regular application of \regel{Initial
              Lowering}predicts the observed tonal
              properties of the pre-stem syllable.



\subsubsection{Hesternal Perfective: Phrase
              Medially}\label{sec:sP7xPhraseMed}

In the Hesternal Perfective, all melodic Hs are
              lost in phrase-medial position. \regel{Initial
              Lowering}remains in force, but the verb
              stem only bears a H if it spreads left onto the verb
              from the post-verbal word via \regel{H Tone
              Anticipation}. Here as elsewhere, the
              leftward extent of the H span is difficult to
              identify, though the root H, lowered or not, does
              appear to limit spreading in /H/ verbs.

 Four pairs of /H/ and /Ø/ stems are provided
              below, half with and half without an object prefix.
              For each pair, the first member involves a H-toned
              complement \vernacular{
              mú{\downstep}yáyi} \gloss{‘boy’}, while
              the second involves a toneless complement \vernacular{muundu} \gloss{
              ‘somebody’}.

 
\ea\label{ex:xHestPerfPhraseMedial} 
Hesternal Perfective Phrase
                Medially \gloss{‘s/he...(for
                him/her)’}


\begin{tabular}{lllll}  
  
                       %\includegraphics[width=\textwidth]{InkScape%20Images/H%20Stems.svg}
 &   
                       %\includegraphics[width=\textwidth]{InkScape%20Images/No%20OP.svg}
 &   
                       \vernacular{y-a\ob [reelɛ]\cb 
                      mú{\downstep}yáyi}  &   
                       \gloss{‘buried the
                      boy’}  &  \\

                       \vernacular{y-a\ob [reelɛ]\cb 
                      muundu}  &   
                       \gloss{‘buried
                      somebody’}  &  \\
  &     &  \\

                       \vernacular{
                      y-a\ob [khalaachɛ́]\cb  mú{\downstep}yáyi}  &   
                       \gloss{‘cut the
                      boy’}  &  \\

                       \vernacular{y-a\ob [khalaachɛ]\cb 
                      muundu}  &   
                       \gloss{‘cut
                      somebody’}  &  \\
  &     &     &  \\

                       %\includegraphics[width=\textwidth]{InkScape%20Images/One%20OP.svg}
 &   
                       \vernacular{
                      y-a\ob mu[ré{\downstep}éléélɛ́]\cb  mú{\downstep}yáyi}  &   
                       \gloss{‘buried/brought the
                      boy’}  &  \\

                       \vernacular{
                      y-a\ob mu[réeleelɛ]\cb  muundu}  &   
                       \gloss{‘buried/brought
                      somebody’}  &  \\
  &     &  \\

                       \vernacular{
                      y-a\ob mu[khá{\downstep}láchíílɪ́]\cb 
                      mú{\downstep}yáyi}  &   
                       \gloss{‘cut the
                      boy’}  &  \\

                       \vernacular{
                      y-a\ob mu[khálachiilɪ]\cb  muundu}  &   
                       \gloss{‘cut
                      somebody’}  &  \\
  &     &     &  \\

                       %\includegraphics[width=\textwidth]{InkScape%20Images/0%20Stems.svg}
 &   
                       %\includegraphics[width=\textwidth]{InkScape%20Images/No%20OP.svg}
 &   
                       \vernacular{y-a\ob [tsíílí]\cb 
                      mú{\downstep}yáyi}  &   
                       \gloss{‘went for the
                      boy’}  &  \\

                       \vernacular{y-a\ob [tsiili]\cb 
                      muundu}  &   
                       \gloss{‘went for
                      somebody’}  &  \\
  &     &  \\

                       \vernacular{
                      y-a\ob [sééβúúlɪ́]\cb  mú{\downstep}yáyi}  &   
                       \gloss{‘said goodbye to the
                      boy’}  &  \\

                       \vernacular{y-a\ob [seeβuulɪ]\cb 
                      muundu}  &   
                       \gloss{‘said goodbye to
                      somebody’}  &  \\
  &     &     &  \\

                       %\includegraphics[width=\textwidth]{InkScape%20Images/One%20OP.svg}
 &   
                       \vernacular{
                      y-a\ob mu[tsíílíílɪ́]\cb  mú{\downstep}yáyi}  &   
                       \gloss{‘went for the
                      boy’}  &  \\

                       \vernacular{
                      y-a\ob mu[tsiiliilɪ]\cb  muundu}  &   
                       \gloss{‘went for
                      somebody’}  &  \\
  &     &  \\

                       \vernacular{
                      y-a\ob mu[sééβúlíílɪ́]\cb 
                      mú{\downstep}yáyi}  &   
                       \gloss{‘said goodbye to the
                      boy’}  &  \\

                       \vernacular{
                      y-a\ob mu[seeβuliilɪ]\cb  muundu}  &   
                       \gloss{‘said goodbye to
                      somebody’}  &  \\
\end{tabular}
%\caption{\nocaption}
    
\z



\subsubsection{Hesternal Perfective: Impact of Subject
              Choice}\label{sec:sP7xSubjects}

Data was not elicited to test for the influence of
              subject choice in Hesternal Perfective stem tone. 



\subsubsection{Hesternal Perfective: Passives}\label{sec:sP7xPassives}

The stem tone properties of Hesternal Perfective
              are the same both with and without the passive suffix
              with one exception. The final syllable is long with a
              fall in passive forms, but short with a level H in
              non-passive forms. In /H/ verbs, the root H is
              lowered by \regel{Initial Lowering}.
              The melodic H is assigned to the FV by \regel{Final MHA}, shifts
              to the penult by \regel{Final Rise
              Elimination}, then spreads left by \regel{Leftward Spread}.
              /Ø/ verbs realize take melodic Hs on the second and
              final stem moras. After \regel{Final Rise
              Elimination}shifts the rightmost H, both Hs
              spread left. The melodic H assigned to the second
              stem mora in \vernacular{
              y-a\ob [lákhúú{\downstep}l-ú-i]\cb } \gloss{‘s/he was
              released’}spreads to the second mora of the
              long syllable via \regel{Fall Elimination} \REF{ex:xFallElimination} .

 
\ea\label{ex:xHestPerfPassives} 
Hesternal Perfective: Passives \gloss{‘s/he
                was...’}[SB]


\begin{tabular}{lllll}  
  \multicolumn{2}{l}{/H/ Stems } &   \multicolumn{2}{l}{/Ø/ Stems } &  \\

                       \vernacular{
                      y-a\ob [kháláách-ú-i]\cb }  &   
                       \gloss{‘cut’}  &   
                       \vernacular{
                      y-a\ob [lákhúú{\downstep}l-ú-i]\cb }  &   
                       \gloss{‘released’}  &  \\

                       \vernacular{
                      y-a\ob [tsúúnzúún-ú-i]\cb }  &   
                       \gloss{‘sucked’}  &   
                       \vernacular{
                      y-a\ob [kálú{\downstep}shííts-ú-i]\cb }  &   
                       \gloss{‘returned
                      for’}  &  \\
\end{tabular}
%\caption{\nocaption}
    
\z



\subsubsection{Pattern 7: Other Verbal Contexts}\label{sec:sP7xOtherTenses}

The tonal properties of the Hesternal Perfective
              Negative are identical to those of its affirmative
              counterpart. In particular, the Hesternal Perfective
              Negative takes two melodic Hs: both tonal classes of
              verbs realize one on the FV and /Ø/ verbs realize a
              second melodic H on the second mora of the stem.
              Additionally, the Hesternal Perfective Negative loses
              both melodic Hs phrase-medially. 

 The Hesternal Perfective and its negative
              counterpart differ only in that the latter includes
              the H-toned negative element \vernacular{tá},
              which is downstepped relative to the melodic H on the
              FV.

 
\ea\label{ex:xP7xTenses} 
Other Pattern 7 Verbal
                Contexts 


\begin{tabular}{llll}  
  a.  &   Hesternal Perfective
                    Negative  &   
                       \vernacular{SP-a[ROOT-ile]
                      tá(awe)}  &  \\
\end{tabular}
%\caption{\nocaption}
    
\z

 Melodic Hs are assigned to the FV in /H/ verbs and
              to the FV and the second stem mora in /Ø/ verbs; all
              melodic Hs spread left via \regel{Leftward Spread} \REF{ex:xLeftwardSpread} up until the left
              stem boundary or a preceding H. The root H is not
              realized due to \regel{Initial
              Lowering}.

 
\ea\label{ex:xP7xHStems} 
Morphologically Simple /H/ Stems
                [SB] \footnote{\label{fn:nP7xGlosses} The examples included in the current section
                  use \vernacular{
                  -khálak-} \gloss{‘cut’}and \vernacular{
                  -βóyong’an-} \gloss{‘go around’}to
                  illustrate the properties of /H/ verbs, and \vernacular{
                  -kulix-} \gloss{‘name’}and \vernacular{
                  -kalushits-} \gloss{‘return’}as
                  representative of /H/ and /Ø/ verbal roots,
                  respectively. The basic gloss for the Hesternal
                  Perfective Negative is \gloss{‘s/he did
                  not...’}.


}%



\begin{tabular}{llllll}  
    &   Subj  &   Tns  &   Stem  &   Neg  &  \\
Hest Perf Neg  &   
                       \vernacular{y-}  &   
                       \vernacular{a}  &   
                       \vernacular{
                      \ob [khálááchɛ́]\cb }  &   
                       \vernacular{{\downstep}tá}  &  \\

                       \vernacular{y-}  &   
                       \vernacular{a}  &   
                       \vernacular{
                      \ob [βóyóng’áánɛ́]\cb }  &   
                       \vernacular{{\downstep}tá}  &  \\
\end{tabular}
%\caption{\nocaption}
    
\z

 
\ea\label{ex:xP7xØStems} 
Morphologically Simple /Ø/ Stems
                [SB] \footnote{\label{fn:nDownstepInHestPerfNeg} Most of SB’s productions of /Ø/ verbs in this
                  context do exhibit downstep between the FV and
                  the H of the negative element \vernacular{tá}.
                  It is unclear at this time why downstep is so
                  infrequently osberved in these forms. The same
                  unexpected behavior is not also observed in /H/
                  verbs.


}%



\begin{tabular}{llllll}  
    &   Subj  &   Tns  &   Stem  &   Neg  &  \\
Hest Perf Neg  &   
                       \vernacular{y-}  &   
                       \vernacular{a}  &   
                       \vernacular{
                      \ob [kúlíí{\downstep}shɪ́]\cb }  &   
                       \vernacular{{\downstep}tá}  &  \\

                       \vernacular{y-}  &   
                       \vernacular{a}  &   
                       \vernacular{
                      \ob [kálúshí{\downstep}ítsɪ́]\cb }  &   
                       \vernacular{{\downstep}tá}  &  \\
\end{tabular}
%\caption{\nocaption}
    
\z

 When an object prefix is present, the root H
              surfaces on the initial mora, and the melodic H on
              the FV spreads left onto the peninitial mora. Stem
              tone in /Ø/ verbs both with and without an object
              prefix is identical. 

 
\ea\label{ex:xP7xOPHStems} 
/H/ Stems with an Object Prefix
                [SB] 


\begin{tabular}{lllllll}  
    &   Subj  &   Tns  &   Obj  &   Stem  &   Neg  &  \\
Hest Perf Neg  &   
                       \vernacular{y-}  &   
                       \vernacular{a}  &   
                       \vernacular{\ob mu}  &   
                       \vernacular{
                      [khá{\downstep}lááchɛ́]\cb }  &   
                       \vernacular{{\downstep}tá}  &  \\

                       \vernacular{y-}  &   
                       \vernacular{a}  &   
                       \vernacular{\ob mu}  &   
                       \vernacular{
                      [βó{\downstep}yóng’áánɛ́]\cb }  &   
                       \vernacular{{\downstep}tá}  &  \\
\end{tabular}
%\caption{\nocaption}
    
\z

 
\ea\label{ex:xP7xOPØStems} 
/Ø/ Stems with an Object Prefix
                [SB] 


\begin{tabular}{lllllll}  
    &   Subj  &   Tns  &   Obj  &   Stem  &   Neg  &  \\
Hest Perf Neg  &   
                       \vernacular{y-}  &   
                       \vernacular{a}  &   
                       \vernacular{\ob mu}  &   
                       \vernacular{
                      [kúlíí{\downstep}shɪ́]\cb }  &   
                       \vernacular{{\downstep}tá}  &  \\

                       \vernacular{y-}  &   
                       \vernacular{a}  &   
                       \vernacular{\ob mu}  &   
                       \vernacular{
                      [kálúshí{\downstep}ítsɪ́]\cb }  &   
                       \vernacular{{\downstep}tá}  &  \\
\end{tabular}
%\caption{\nocaption}
    
\z

 Melodic Hs are deleted phrase-medially in the
              Hesternal Perfective Negative. \regel{Initial Lowering}is
              in effect, and \regel{H Tone
              Anticipation}spreads post-verbal Hs onto
              the verb stem. It is unclear how far \regel{H Tone
              Anticipation}spreads the H of the negative
              element \vernacular{tá}into the
              verb stem. Examples are provided below of both /H/
              and /Ø/ stems before a H-toned noun, \vernacular{
              mú{\downstep}yáyi} \gloss{‘boy’}, and a
              toneless noun \vernacular{muundu} \gloss{
              ‘person/somebody’}.

 
\ea\label{ex:xP7xPhraseMed} 
Constructions Like the Hesternal
                Perfective Phrase Medially [SB] 


\begin{tabular}{llll}  
  
                       \textbf{Hest Perf Neg}  &   
                    /H/  &   
                       \vernacular{y-a[khalaachɛ́]
                      {\downstep}mú{\downstep}yáyi tá}  &  \\

                       \vernacular{y-a[khalachɛ́]
                      múúndú tá}  &  \\
  &     &  \\

                    /Ø/  &   
                       \vernacular{
                      y-a[kúlííshɪ́] mú{\downstep}yáyi tá}  &  \\

                       \vernacular{
                      y-a[kúlííshɪ́] múúndú tá}  &  \\
\end{tabular}
%\caption{\nocaption}
    
\z

 It is noteworthy that the negative element \vernacular{
              tá(awe)}does not trigger deletion of the
              melodic Hs, though post-verbal complements do.

 As with the affirmative Hesternal Perfective,
              there is no data available relating to the effect of
              subject choice on the tonal properties of the
              Hesternal Perfective Negative. 

 As in the affirmative, the Hesternal Perfective
              Negative has the same stem tone properties with and
              without a passive suffix, except that there is no
              final fall in the Hesternal Perfective Negative,
              because the finally syllable shortens by \regel{Non-Final
              Shortening}.

 
\ea\label{ex:xP7xPassive} 
/H/ \& /Ø/ Stems with the
                Passive Suffix \gloss{‘s/he was
                not...’}[SB]


\begin{tabular}{lllll}  
  
                       \textbf{Hest Perf Neg}  &   /H/  &   
                       \vernacular{
                      y-a[kháláách-w-í] {\downstep}tá}  &   
                       \gloss{‘cut’}  &  \\

                       \vernacular{
                      y-a[tsúúnzúún-w-í] {\downstep}tá}  &   
                       \gloss{‘sucked’}  &  \\
/Ø/  &   
                       \vernacular{
                      y-a[lákhúú{\downstep}l-w-í] {\downstep}tá}  &   
                       \gloss{‘released’}  &  \\

                       \vernacular{
                      y-a[kálúshí{\downstep}íts-w-í] {\downstep}tá}  &   
                       \gloss{‘returned’}  &  \\
\end{tabular}
%\caption{\nocaption}
    
\z



\section{Pattern 8: The first through final
          pattern}\label{sec:sPattern8}

The Habitual and the Habitual Negative exhibit the
          properties of Pattern 8. The basic tonal melody
          associated with this pattern is expressed as a H across
          the full length of the verb stem in both /H/ and /Ø/
          verbs. 


\subsection{Pattern 8: Habitual
            }\label{sec:sPattern8x}

The tonal properties of Pattern 8 will be described
            with reference to the Habitual in this section. The
            Habitual is marked by the FV \vernacular{-a}, the
            tense prefix \vernacular{aá-}, and
            a melodic H which surfaces on all moras of the stem;
            the lexical contrast is neutralized in forms lacking
            any object prefixes.

 The following display illustrates that /H/ verbs
            realize the H of the tense prefix \textit{in situ}and a
            downstepped level H across the full length of the
            stem.

 
\ea\label{ex:xHabCH} 
Habitual C-Initial /H/ \gloss{‘s/he is
              ever/always...’
              }


\begin{tabular}{lllll}  
  Subj  &   Tns  &   Stem  &   Gloss  &  \\

                     \vernacular{y-}  &   
                     \vernacular{aá}  &   
                     \vernacular{
                    \ob [{\downstep}ng’úa]\cb }  &   
                     \gloss{‘drinking’}  &  \\

                     \vernacular{y-}  &   
                     \vernacular{aá}  &   
                     \vernacular{
                    \ob [{\downstep}βéká]\cb }  &   
                     \gloss{‘shaving’}  &  \\

                     \vernacular{y-}  &   
                     \vernacular{aá}  &   
                     \vernacular{
                    \ob [{\downstep}téékhɛ́]\cb }  &   
                     \gloss{‘cooking’}  &  \\

                     \vernacular{y-}  &   
                     \vernacular{aá}  &   
                     \vernacular{
                    \ob [{\downstep}kháláká]\cb }  &   
                     \gloss{‘cutting’}  &  \\

                     \vernacular{y-}  &   
                     \vernacular{aá}  &   
                     \vernacular{
                    \ob [{\downstep}káláángá]\cb }  &   
                     \gloss{‘frying’}  &  \\

                     \vernacular{y-}  &   
                     \vernacular{aá}  &   
                     \vernacular{
                    \ob [{\downstep}βóólítsá]\cb }  &   
                     \gloss{‘seducing’}  &  \\

                     \vernacular{y-}  &   
                     \vernacular{aá}  &   
                     \vernacular{
                    \ob [{\downstep}tsúúnzúúná]\cb }  &   
                     \gloss{‘sucking’}  &  \\

                     \vernacular{y-}  &   
                     \vernacular{aá}  &   
                     \vernacular{
                    \ob [{\downstep}βóyóng’áná]\cb }  &   
                     \gloss{‘going
                    around’}  &  \\
\end{tabular}
%\caption{\nocaption}
    
\z

 
\ea\label{ex:xHabVH} 
Habitual V-Initial /H/ \gloss{‘s/he is
              ever/always...’}


\begin{tabular}{lllll}  
  Subj  &   Tns  &   Stem  &   Gloss  &  \\

                     \vernacular{y-}  &   
                     \vernacular{aá}  &   
                     \vernacular{
                    \ob [{\downstep}yírá]\cb }  &   
                     \gloss{‘killing’}  &  \\

                     \vernacular{y-}  &   
                     \vernacular{aá}  &   
                     \vernacular{
                    \ob [{\downstep}yónóɲːá]\cb }  &   
                     \gloss{‘spoiling’}  &  \\

                     \vernacular{y-}  &   
                     \vernacular{aá}  &   
                     \vernacular{
                    \ob [{\downstep}yábúkhányːá]\cb }  &   
                     \gloss{‘separating’}  &  \\
\end{tabular}
%\caption{\nocaption}
    
\z

 /Ø/ verbs have identical tonal properties in this
            context―a H tense prefix, followed by a downstepped H
            on all stem vowels. 

 
\ea\label{ex:xHabCØ} 
Habitual C-Initial /Ø/ \gloss{‘s/he is
              ever/always...’}


\begin{tabular}{lllll}  
  Subj  &   Tns  &   Stem  &   Gloss  &  \\

                     \vernacular{y-}  &   
                     \vernacular{aá}  &   
                     \vernacular{
                    \ob [{\downstep}kúa]\cb }  &   
                     \gloss{‘falling’}  &  \\

                     \vernacular{y-}  &   
                     \vernacular{aá}  &   
                     \vernacular{
                    \ob [{\downstep}lékhá]\cb }  &   
                   \gloss{
                  ‘leaving’}[SB] &  \\

                     \vernacular{y-}  &   
                     \vernacular{aá}  &   
                     \vernacular{
                    \ob [{\downstep}rééβá]\cb }  &   
                     \gloss{‘asking’}  &  \\

                     \vernacular{y-}  &   
                     \vernacular{aá}  &   
                     \vernacular{
                    \ob [{\downstep}kúlíkhá]\cb }  &   
                   \gloss{
                  ‘naming’}[SB] &  \\

                     \vernacular{y-}  &   
                     \vernacular{aá}  &   
                     \vernacular{
                    \ob [{\downstep}lákhúúlá]\cb }  &   
                     \gloss{‘releasing’}  &  \\

                     \vernacular{y-}  &   
                     \vernacular{aá}  &   
                     \vernacular{
                    \ob [{\downstep}sééβúlá]\cb }  &   
                     \gloss{‘saying goodbye
                    (to)’}  &  \\

                     \vernacular{y-}  &   
                     \vernacular{aá}  &   
                     \vernacular{
                    \ob [{\downstep}kálúshítsá]\cb }  &   
                   \gloss{
                  ‘returning’}[SB] &  \\

                     \vernacular{y-}  &   
                     \vernacular{aá}  &   
                     \vernacular{
                    \ob [{\downstep}síínjílítsá]\cb }  &   
                     \gloss{‘making
                    stand’}  &  \\

                     \vernacular{y-}  &   
                     \vernacular{aá}  &   
                     \vernacular{
                    \ob [{\downstep}séβúlúkháɲːá]\cb }  &   
                     \gloss{‘scattering’}  &  \\
\end{tabular}
%\caption{\nocaption}
    
\z

 
\ea\label{ex:xHabVØ} 
Habitual V-Initial /Ø/ \gloss{‘s/he is
              ever/always...’}


\begin{tabular}{lllll}  
  Subj  &   Tns  &   Stem  &   Gloss  &  \\

                     \vernacular{y-}  &   
                     \vernacular{aá}  &   
                     \vernacular{
                    \ob [{\downstep}yényá]\cb }  &   
                     \gloss{‘wanting’}  &  \\

                     \vernacular{y-}  &   
                     \vernacular{aá}  &   
                     \vernacular{
                    \ob [{\downstep}yílúúlá]\cb }  &   
                     \gloss{‘winnowing’}  &  \\

                     \vernacular{y-}  &   
                     \vernacular{aá}  &   
                     \vernacular{
                    \ob [{\downstep}yámbákháná]\cb }  &   
                     \gloss{‘refusing’}  &  \\

                     \vernacular{y-}  &   
                     \vernacular{aá}  &   
                     \vernacular{
                    \ob [{\downstep}yéléélítsá]\cb }  &   
                     \gloss{‘hanging s.t.
                    up’}  &  \\
\end{tabular}
%\caption{\nocaption}
    
\z


\subsubsection{Habitual with Object Prefixes}\label{sec:sP8xObjects}

When an object prefix is present, the root H
              surfaces \textit{in situ}and the
              melodic H span extends through the remainder of the
              stem. While the object prefix surfaces H, note the
              absence of downstep between the object prefix and the
              root.

 
\ea\label{ex:xHabCHOP} 
Habitual C-Initial /H/ + OP \gloss{‘s/he is
                ever/always...him/her’}


\begin{tabular}{llllll}  
  Subj  &   Tns  &   Obj  &   Stem  &   Gloss  &  \\

                       \vernacular{y-}  &   
                       \vernacular{aá}  &   
                       \vernacular{\ob {\downstep}mú}  &   
                       \vernacular{
                      [ráa]\cb }  &   
                       \gloss{‘burying’}  &  \\

                       \vernacular{y-}  &   
                       \vernacular{aá}  &   
                       \vernacular{\ob {\downstep}mú}  &   
                       \vernacular{
                      [βé{\downstep}ká]\cb }  &   
                       \gloss{‘shaving’}  &  \\

                       \vernacular{y-}  &   
                       \vernacular{aá}  &   
                       \vernacular{\ob {\downstep}mú}  &   
                       \vernacular{
                      [lé{\downstep}érá]\cb }  &   
                       \gloss{‘bringing’}  &  \\

                       \vernacular{y-}  &   
                       \vernacular{aá}  &   
                       \vernacular{\ob {\downstep}mú}  &   
                       \vernacular{
                      [khá{\downstep}láká]\cb }  &   
                       \gloss{‘cutting’}  &  \\

                       \vernacular{y-}  &   
                       \vernacular{aá}  &   
                       \vernacular{\ob {\downstep}mú}  &   
                       \vernacular{
                      [βó{\downstep}ólítsá]\cb }  &   
                       \gloss{‘seducing’}  &  \\

                       \vernacular{y-}  &   
                       \vernacular{aá}  &   
                       \vernacular{\ob {\downstep}mú}  &   
                       \vernacular{
                      [βó{\downstep}yóng’áná]\cb }  &   
                       \gloss{‘going
                      around’}  &  \\
\end{tabular}
%\caption{\nocaption}
    
\z

 
\ea\label{ex:xHabVHOP} 
Habitual V-Initial /H/ + OP \gloss{‘s/he is
                ever/always...him/her’}


\begin{tabular}{llllll}  
  Subj  &   Tns  &   Obj  &   Stem  &   Gloss  &  \\

                       \vernacular{y-}  &   
                       \vernacular{aá}  &   
                       \vernacular{
                      \ob {\downstep}mwí}  &   
                       \vernacular{
                      [í{\downstep}ra]\cb }  &   
                       \gloss{‘killing’}  &  \\

                       \vernacular{y-}  &   
                       \vernacular{aá}  &   
                       \vernacular{
                      \ob {\downstep}mwó}  &   
                       \vernacular{
                      [ó{\downstep}nónyːá]\cb }  &   
                       \gloss{‘spoiling’}  &  \\

                       \vernacular{y-}  &   
                       \vernacular{aá}  &   
                       \vernacular{
                      \ob {\downstep}mwá}  &   
                       \vernacular{
                      [á{\downstep}búkhányːá]\cb }  &   
                       \gloss{
                      ‘separating’}  &  \\
\end{tabular}
%\caption{\nocaption}
    
\z

 In /Ø/ stems, the tense prefix surfaces H as
              expected, and the melodic H spans from the object
              prefix through the FV. 

 
\ea\label{ex:xHabCØOP} 
Habitual C-Initial /Ø/ + OP \gloss{‘s/he is
                ever/always...him/her’}


\begin{tabular}{llllll}  
  Subj  &   Tns  &   Obj  &   Stem  &   Gloss  &  \\

                       \vernacular{y-}  &   
                       \vernacular{aá}  &   
                       \vernacular{\ob {\downstep}mú}  &   
                       \vernacular{
                      [tsía]\cb }  &   
                       \gloss{‘going
                      (for)’}  &  \\

                       \vernacular{y-}  &   
                       \vernacular{aá}  &   
                       \vernacular{\ob {\downstep}mú}  &   
                       \vernacular{
                      [lékhá]\cb }  &   
                       \gloss{‘leaving’}  &  \\

                       \vernacular{y-}  &   
                       \vernacular{aá}  &   
                       \vernacular{\ob {\downstep}mú}  &   
                       \vernacular{
                      [lóóndá]\cb }  &   
                       \gloss{
                      ‘following’}  &  \\

                       \vernacular{y-}  &   
                       \vernacular{aá}  &   
                       \vernacular{\ob {\downstep}mú}  &   
                       \vernacular{
                      [kúlíkhá]\cb }  &   
                       \gloss{‘naming’}  &  \\

                       \vernacular{y-}  &   
                       \vernacular{aá}  &   
                       \vernacular{\ob {\downstep}mú}  &   
                       \vernacular{
                      [sééβúlá]\cb }  &   
                       \gloss{‘saying
                      goodbye’}  &  \\

                       \vernacular{y-}  &   
                       \vernacular{aá}  &   
                       \vernacular{\ob {\downstep}mú}  &   
                       \vernacular{
                      [kálúshítsá]\cb }  &   
                     \gloss{
                    ‘returning’}[SB] &  \\
\end{tabular}
%\caption{\nocaption}
    
\z

 
\ea\label{ex:xHabVØOP} 
Habitual V-Initial /Ø/ + OP \gloss{‘s/he is
                ever/always...him/her’}


\begin{tabular}{llllll}  
  Subj  &   Tns  &   Obj  &   Stem  &   Gloss  &  \\

                       \vernacular{y-}  &   
                       \vernacular{aá}  &   
                       \vernacular{
                      \ob {\downstep}mwé}  &   
                       \vernacular{
                      [ényá]\cb }  &   
                       \gloss{‘wanting’}  &  \\

                       \vernacular{y-}  &   
                       \vernacular{aá}  &   
                       \vernacular{
                      \ob {\downstep}mwé}  &   
                       \vernacular{
                      [éyélá]\cb }  &   
                       \gloss{‘wiping
                      for’}  &  \\

                       \vernacular{y-}  &   
                       \vernacular{aá}  &   
                       \vernacular{
                      \ob {\downstep}mwá}  &   
                       \vernacular{
                      [ámbákháná]\cb }  &   
                       \gloss{‘refusing’}  &  \\
\end{tabular}
%\caption{\nocaption}
    
\z

 The tonal properties of Habitual forms involving
              1 \textsuperscript{st}sg object
              prefixes are identical to those involving CV- object
              prefixes. Both tonal classes realize the H of the
              tense prefix as expected. In /H/ verbs, a second H
              spans across the object prefix and stem-initial mora,
              and a third H spans throughout the remainder of the
              stem. /Ø/ verbs have instead a single H spanning from
              the object prefix through the FV.

 
\ea\label{ex:xHabCHOP1sg} 
Habitual C-Initial /H/ + OP \textsubscript{1sg} \gloss{‘s/he is
                ever/always...me’
                }


\begin{tabular}{llllll}  
  Subj  &   Tns  &   Obj  &   Stem  &   Gloss  &  \\

                       \vernacular{y-}  &   
                       \vernacular{aá}  &   
                       \vernacular{\ob {\downstep}á}  &   
                       \vernacular{
                      [khúa]\cb }  &   
                       \gloss{‘paying dowry
                      (for)’}  &  \\

                       \vernacular{y-}  &   
                       \vernacular{aá}  &   
                       \vernacular{\ob {\downstep}á}  &   
                       \vernacular{
                      [mbé{\downstep}ká]\cb }  &   
                       \gloss{‘shaving’}  &  \\

                       \vernacular{y-}  &   
                       \vernacular{aá}  &   
                       \vernacular{\ob {\downstep}á}  &   
                       \vernacular{
                      [ndé{\downstep}érá]\cb }  &   
                       \gloss{‘bringing’}  &  \\

                       \vernacular{y-}  &   
                       \vernacular{aá}  &   
                       \vernacular{\ob {\downstep}á}  &   
                       \vernacular{
                      [khá{\downstep}láká]\cb }  &   
                       \gloss{‘cutting’}  &  \\

                       \vernacular{y-}  &   
                       \vernacular{aá}  &   
                       \vernacular{\ob {\downstep}á}  &   
                       \vernacular{
                      [mbó{\downstep}ólítsá]\cb }  &   
                       \gloss{‘seducing’}  &  \\

                       \vernacular{y-}  &   
                       \vernacular{aá}  &   
                       \vernacular{\ob {\downstep}á}  &   
                       \vernacular{
                      [mbó{\downstep}yóng’áná]\cb }  &   
                       \gloss{‘going
                      around’}  &  \\
\end{tabular}
%\caption{\nocaption}
    
\z

 
\ea\label{ex:xHabCØOP1sg} 
Habitual C-Initial /Ø/ + OP \textsubscript{1sg} \gloss{‘s/he is
                ever/always...me’}


\begin{tabular}{llllll}  
  Subj  &   Tns  &   Obj  &   Stem  &   Gloss  &  \\

                       \vernacular{y-}  &   
                       \vernacular{aá}  &   
                       \vernacular{\ob {\downstep}á}  &   
                       \vernacular{
                      [sía]}  &   
                     \gloss{
                    ‘grinding’}[SB] &  \\

                       \vernacular{y-}  &   
                       \vernacular{aá}  &   
                       \vernacular{\ob {\downstep}á}  &   
                       \vernacular{
                      [ndékhá]}  &   
                       \gloss{‘leaving’}  &  \\

                       \vernacular{y-}  &   
                       \vernacular{aá}  &   
                       \vernacular{\ob {\downstep}á}  &   
                       \vernacular{
                      [nóóndá]}  &   
                       \gloss{
                      ‘following’}  &  \\

                       \vernacular{y-}  &   
                       \vernacular{aá}  &   
                       \vernacular{\ob {\downstep}á}  &   
                       \vernacular{
                      [ngúlíkhá]}  &   
                     \gloss{
                    ‘naming’}[SB] &  \\

                       \vernacular{y-}  &   
                       \vernacular{aá}  &   
                       \vernacular{\ob {\downstep}á}  &   
                       \vernacular{
                      [sééβúlá]}  &   
                       \gloss{‘saying goodbye
                      (to)’}  &  \\

                       \vernacular{y-}  &   
                       \vernacular{aá}  &   
                       \vernacular{\ob {\downstep}á}  &   
                       \vernacular{
                      [ngálúshítsá]}  &   
                     \gloss{
                    ‘returning’}[SB] &  \\
\end{tabular}
%\caption{\nocaption}
    
\z

 As in previous Habitual forms, nearly the entire
              word is H in forms which combine a CV- object prefix
              and a 1 \textsuperscript{st}sg object
              prefix. Downstep is observed following the tense
              prefix and the second object prefix.

 
\ea\label{ex:xHabCHOPOP1sg} 
Habitual C-Initial /H/ + OP + OP \textsubscript{1sg} \gloss{‘s/he is
                ever/always...him/her for
                me’}


\begin{tabular}{lllllll}  
  Subj  &   Tns  &   Obj
                     \textsubscript{CV} &   Obj
                     \textsubscript{1sg} &   Stem  &   Gloss  &  \\

                       \vernacular{y-}  &   
                       \vernacular{aá}  &   
                       \vernacular{
                      \ob {\downstep}mú-}  &   
                       \vernacular{ú}  &   
                       \vernacular{
                      [{\downstep}ndéélá]\cb }  &   
                       \gloss{‘burying’}  &  \\

                       \vernacular{y-}  &   
                       \vernacular{aá}  &   
                       \vernacular{
                      \ob {\downstep}mú-}  &   
                       \vernacular{ú}  &   
                       \vernacular{
                      [{\downstep}mbéchélá]\cb }  &   
                       \gloss{‘shaving’}  &  \\

                       \vernacular{y-}  &   
                       \vernacular{aá}  &   
                       \vernacular{
                      \ob {\downstep}mú-}  &   
                       \vernacular{ú}  &   
                       \vernacular{
                      [{\downstep}ndéérélá]\cb }  &   
                       \gloss{‘bringing’}  &  \\

                       \vernacular{y-}  &   
                       \vernacular{aá}  &   
                       \vernacular{
                      \ob {\downstep}mú-}  &   
                       \vernacular{ú}  &   
                       \vernacular{
                      [{\downstep}kháláchílá]\cb }  &   
                       \gloss{‘cutting’}  &  \\
\end{tabular}
%\caption{\nocaption}
    
\z

 
\ea\label{ex:xHabCØOPOP1sg} 
Habitual C-Initial /Ø/ + OP + OP \textsubscript{1sg} \gloss{‘s/he is
                ever/always...him/her for
                me’}


\begin{tabular}{lllllll}  
  Subj  &   Tns  &   Obj
                     \textsubscript{CV} &   Obj
                     \textsubscript{1sg} &   Stem  &   Gloss  &  \\

                       \vernacular{y-}  &   
                       \vernacular{aá}  &   
                       \vernacular{
                      \ob {\downstep}mú-}  &   
                       \vernacular{ú}  &   
                       \vernacular{
                      [{\downstep}nzíílá]\cb }  &   
                       \gloss{‘going
                      (for)’}  &  \\

                       \vernacular{y-}  &   
                       \vernacular{aá}  &   
                       \vernacular{
                      \ob {\downstep}mú-}  &   
                       \vernacular{ú}  &   
                       \vernacular{
                      [{\downstep}ndéshélá]\cb }  &   
                     \gloss{
                    ‘leaving’}[SB] &  \\

                       \vernacular{y-}  &   
                       \vernacular{aá}  &   
                       \vernacular{
                      \ob {\downstep}mú-}  &   
                       \vernacular{ú}  &   
                       \vernacular{
                      [{\downstep}nóóndélá]\cb }  &   
                       \gloss{
                      ‘following’}  &  \\

                       \vernacular{y-}  &   
                       \vernacular{aá}  &   
                       \vernacular{
                      \ob {\downstep}mú-}  &   
                       \vernacular{ú}  &   
                       \vernacular{
                      [{\downstep}ndákhúúlílá]\cb }  &   
                       \gloss{
                      ‘releasing’}  &  \\
\end{tabular}
%\caption{\nocaption}
    
\z

 The following are the core tonal properties of
              the Habitual: (i) underlying macrostem-initial Hs
              fail to surface, (ii) the pre-stem syllable has a
              rise, (iii) verbs from either tonal class without an
              object prefix and with two object prefixes have a
              single H which spans the full length of the stem, and
              (iv) /H/ verbs with an object prefix have a H on the
              initial mora followed by a downstepped H on the rest
              of the stem. The H span in /Ø/ verbs with an object
              prefix extends even onto the pre-stem syllable. These
              properties are summarized schematically in \REF{ex:xHabSchematic} . As before, the
              position of underlying Hs is indicated with a single
              underline, and the melodic H is indicated with double
              underlining.

 
\ea\label{ex:xHabSchematic} 
A Schematic Representation of the
                Habitual’s Tonal Properties 


\begin{tabular}{lllll}  
    &   \multicolumn{3}{l}{
                       \ul{/H/ Verbs} } &  \\
  &   
                       \textit{Subj + Tns}  &   \multicolumn{2}{l}{
                       \textit{Macrostem} } &  \\
OPsx0  &   
                       \vernacular{y-aá}  &   
                       \vernacular{\ob }  &   
                       \vernacular{[{\downstep}C
                      }  &  \\
OPsx1  &   
                       \vernacular{y-aá}  &   
                       \vernacular{\ob {\downstep}C
                      }  &   
                       \vernacular{[C
                      }  &  \\
OPsx2  &   
                       \vernacular{y-aá}  &   
                       \vernacular{\ob {\downstep}C
                      }  &   
                       \vernacular{[{\downstep}C
                      }  &  \\
  &   \multicolumn{2}{l}{ } &     &  \\
  &   \multicolumn{3}{l}{
                       \textbf{
                        } } &  \\
  &   
                       \textit{Subj + Tns}  &   \multicolumn{2}{l}{
                       \textit{Macrostem} } &  \\
OPsx0  &   
                       \vernacular{y-aá}  &   
                       \vernacular{\ob }  &   
                     \vernacular{[{\downstep}C
                    }\cb  &  \\
OPsx1  &   
                       \vernacular{y-aá}  &   
                       \vernacular{\ob {\downstep}C
                      }  &   
                       \vernacular{[C
                      }  &  \\
OPsx2  &   
                       \vernacular{y-aá}  &   
                       \vernacular{\ob {\downstep}C
                      }  &   
                       \vernacular{[{\downstep}C
                      }  &  \\
\end{tabular}
%\caption{\nocaption}
    
\z

 The level H span across the entire verb stem in
              both /H/ and /Ø/ verbs is accounted for via the
              following steps: the root H is lowered via \regel{Initial Lowering} \REF{ex:xInitialLowering} , a melodic H is
              assigned to the FV via \regel{Final MHA} \REF{ex:xFinalMHA} , and the melodic H spreads onto the
              initial stem mora via \regel{Leftward Spread} \REF{ex:xLeftwardSpread} . Note that this
              is the same analysis offered for the Hesternal
              Perfective in § \sectref{sec:sP7xObjects} , except
              that \regel{Default
              MHA}exceptionally does not apply in the
              Habitual, and the construction contributes only a
              single melodic H. \footnote{\label{fn:nNoDefaultMHAinP8} If \regel{Default MHA}were
                to apply in this context, it would assign the
                melodic H to the second stem mora before \regel{Final MHA}has a
                chance to apply.


}%


 
\ea\label{ex:xDerivHabH} 
 Derivation,
                  /H/ Habitual: \vernacular{
                  y-aá\ob [{\downstep}kháláká]\cb } \gloss{‘s/he is
                  always/ever cutting’} 

%\includegraphics[width=\textwidth]{InkScape%20Images/Derivations/DerivHabH.pdf}

\z

 
\ea\label{ex:xDerivHabØ} 
 Derivation,
                  /Ø/ Habitual: \vernacular{
                  y-aá\ob [{\downstep}kúlíkhá]\cb } \gloss{‘s/he is
                  always/ever naming’} 

%\includegraphics[width=\textwidth]{InkScape%20Images/Derivations/DerivHab0.pdf}

\z

 Turning now to forms with object prefixes, one
              observes downstep following the tense prefix and, in
              /H/ verbs, the stem-initial mora as well. I analyze
              these forms in the following manner: the H of the
              object prefix is lowered by \regel{Initial Lowering},
              but the object prefix surfaces H anyway due to the
              leftward spreading of the root H via \regel{Plateau} \REF{ex:xPlateau} . The melodic H, first assigned to the
              FV, also spreads left via \regel{Leftward
              Spread}.

 
\ea\label{ex:xDerivHabHOP} 
 Derivation,
                  /H/ Habitual + OP: \vernacular{
                  y-aá\ob {\downstep}mú[khá{\downstep}láká]\cb } \gloss{‘s/he is
                  always/ever cutting him/her’} 

%\includegraphics[width=\textwidth]{InkScape%20Images/Derivations/DerivHabHOP.pdf}

\z

 The analysis of /Ø/ verbs is similar, except that
              there is no root H on the initial mora of the stem to
              occasion a second instance of downstep. 

 
\ea\label{ex:xDerivHabØOP} 
 Derivation,
                  /Ø/ Habitual + OP: \vernacular{
                  y-aá\ob {\downstep}mú[kúlíkhá]\cb } \gloss{‘s/he is
                  always/ever naming him/her’} 

%\includegraphics[width=\textwidth]{InkScape%20Images/Derivations/DerivHab0OP.pdf}

\z

 Stem tone patterns in /H/ and /Ø/ verbs with two
              object prefixes are identical because the root H is
              deleted by \regel{Meeussen’s
              Rule}following the H of the second object
              prefix.

 
\ea\label{ex:xDerivHabHOPx2} 
 Derivation,
                  /H/ Habitual + OPx2: \vernacular{
                  y-aá\ob {\downstep}mú-ú[{\downstep}ndéélá]\cb } \gloss{‘s/he is
                  always/ever burying him/her for me’} 

%\includegraphics[width=\textwidth]{InkScape%20Images/Derivations/DerivHabHOPx2.pdf}

\z

 
\ea\label{ex:xDerivHabØOPx2} 
 Derivation,
                  /Ø/ Habitual + OPx2: \vernacular{
                  y-aá\ob {\downstep}mú-ú[{\downstep}nzíílá]\cb } \gloss{‘s/he is
                  always/ever going for him/her for me’} 

%\includegraphics[width=\textwidth]{InkScape%20Images/Derivations/DerivHab0OPx2.pdf}

\z



\subsubsection{Habitual: Phrase Medially}\label{sec:sP8xPhraseMed}

The melodic H is lost phrase-medially in the
              Habitual. Despite the deletion of the melodic H, the
              stem tone properties of Habitual verb forms in a
              phrase-medial context are often the same as they are
              phrase-finally when the verb is followed by a H-toned
              word― \regel{H Tone
              Anticipation}spreads post verbal Hs onto
              the stem. The leftward extent of the resulting H span
              might be limited in forms in which the root H is
              lowered by \regel{Initial
              Lowering}.

 Four pairs of /H/ and /Ø/ stems are provided
              below, half with and half without an object prefix.
              For each pair, the first member involves a H-toned
              complement, while the second involves a toneless
              complement. In each case, the stem tonal properties
              are the same as the pre-pausal counterparts. 

 
\ea\label{ex:xHabPhraseMedial} 
Habitual Phrase Medially \gloss{‘s/he is
                ever/always...(for
                him/her)’}


\begin{tabular}{lllll}  
  
                       %\includegraphics[width=\textwidth]{InkScape%20Images/H%20Stems.svg}
 &   
                       %\includegraphics[width=\textwidth]{InkScape%20Images/No%20OP.svg}
 &   
                       \vernacular{y-aá\ob [{\downstep}rá]
                      mú{\downstep}yáyi}  &   
                       \gloss{‘burying the
                      boy’}  &  \\

                       \vernacular{y-aá\ob [ra]
                      muundu}  &   
                       \gloss{‘burying
                      somebody’}  &  \\
  &     &  \\

                       \vernacular{y-aá\ob [khalaká]
                      mú{\downstep}yáyi}  &   
                       \gloss{‘cutting the
                      boy’}  &  \\

                       \vernacular{y-aá\ob [khalaka]
                      muundu}  &   
                       \gloss{‘cutting
                      somebody’}  &  \\
  &     &     &  \\

                       %\includegraphics[width=\textwidth]{InkScape%20Images/One%20OP.svg}
 &   
                       \vernacular{
                      y-aá\ob {\downstep}mú[ré{\downstep}élá] mú{\downstep}yáyi}  &   
                       \gloss{‘burying the
                      boy’}  &  \\

                       \vernacular{
                      y-aá\ob {\downstep}mú[réela] muundu}  &   
                       \gloss{‘burying
                      somebody’}  &  \\
  &     &  \\

                       \vernacular{
                      y-aá\ob {\downstep}mú[khá{\downstep}láchílá]
                      mú{\downstep}yáyi}  &   
                       \gloss{‘cutting the
                      boy’}  &  \\

                       \vernacular{
                      y-aá\ob {\downstep}mú[khálachila] muundu}  &   
                       \gloss{‘cutting
                      somebody’}  &  \\
  &     &     &  \\

                       %\includegraphics[width=\textwidth]{InkScape%20Images/0%20Stems.svg}
 &   
                       %\includegraphics[width=\textwidth]{InkScape%20Images/No%20OP.svg}
 &   
                       \vernacular{y-aá\ob [{\downstep}tsyá]
                      mú{\downstep}yáyi}  &   
                       \gloss{‘going for the
                      boy’}  &  \\

                       \vernacular{y-aá\ob [tsya]
                      muundu}  &   
                       \gloss{‘going for
                      somebody’}  &  \\
  &     &  \\

                       \vernacular{
                      y-aá\ob [{\downstep}sééβúlá] mú{\downstep}yáyi}  &   
                       \gloss{‘saying goodbye to
                      the boy’}  &  \\

                       \vernacular{y-aá\ob [seeβula]
                      muundu}  &   
                       \gloss{‘saying goodbye to
                      somebody’}  &  \\
  &     &     &  \\

                       %\includegraphics[width=\textwidth]{InkScape%20Images/One%20OP.svg}
 &   
                       \vernacular{
                      y-aá\ob {\downstep}mú[tsíílá] mú{\downstep}yáyi}  &   
                       \gloss{‘going for the
                      boy’}  &  \\

                       \vernacular{y-aá\ob mu[tsiila]
                      muundu}  &   
                       \gloss{‘going for
                      somebody’}  &  \\
  &     &  \\

                       \vernacular{
                      y-aá\ob {\downstep}mú[sééβúlilá]
                      mú{\downstep}yáyi}  &   
                       \gloss{‘saying goodbye to
                      the boy’}  &  \\

                       \vernacular{
                      y-aá\ob mu[seeβulila] muundu}  &   
                       \gloss{‘saying goodbye to
                      somebody’}  &  \\
\end{tabular}
%\caption{\nocaption}
    
\z



\subsubsection{Habitual: Impact of Subject
              Choice}\label{sec:sP8xSubjects}

The Habitual does not exhibit any subject-induced
              tonal alternations; verb stem tone is the same for
              all subjects. 

 
\ea\label{ex:xSubjHabH} 
Subject Choice in the Habitual /H/ \gloss{‘...am/are
                ever/always drunk 
                }[SB]


\begin{tabular}{llll}  
    &   Singular  &   Plural  &  \\
1
                     \textsuperscript{
                    st}Person &   
                       \vernacular{
                      n-aá\ob [{\downstep}ng’úa]\cb }  &   
                       \vernacular{
                      khw-aá\ob [{\downstep}ng’úa]\cb }  &  \\
2
                     \textsuperscript{
                    nd}Person &   
                       \vernacular{
                      w-aá\ob [{\downstep}téékhá]\cb }  &   
                       \vernacular{
                      mw-aá\ob [{\downstep}léérá]\cb }  &  \\
3
                     \textsuperscript{
                    rd}Person &   
                       \vernacular{
                      y-aá\ob [{\downstep}téékhá]\cb }  &   
                       \vernacular{
                      β-aá\ob [{\downstep}léérá]\cb }  &  \\
\end{tabular}
%\caption{\nocaption}
    
\z

 
\ea\label{ex:xSubjHabØ} 
Subject Choice in the Habitual /Ø/ \gloss{‘...is/are
                ever/always asking’}[SB]


\begin{tabular}{llll}  
    &   Singular  &   Plural  &  \\
1
                     \textsuperscript{
                    st}Person &   
                       \vernacular{
                      n-aá\ob [{\downstep}rééβá]\cb }  &   
                       \vernacular{
                      khw-aá\ob [{\downstep}rééβá]\cb }  &  \\
2
                     \textsuperscript{
                    nd}Person &   
                       \vernacular{
                      w-aá\ob [{\downstep}rééβá]\cb }  &   
                       \vernacular{
                      mw-aá\ob [{\downstep}rééβá]\cb }  &  \\
3
                     \textsuperscript{
                    rd}Person &   
                       \vernacular{
                      y-aa\ob ́[{\downstep}rééβá]\cb }  &   
                       \vernacular{
                      β-aá\ob [{\downstep}rééβá]\cb }  &  \\
\end{tabular}
%\caption{\nocaption}
    
\z

 The data below illustrate that forms with an
              object prefix are similarly unaffected by the choice
              of subject. 

 
\ea\label{ex:xSubjHabHOP} 
Subject Choice in the Habitual /H/
                + OP \gloss{‘...is/are
                ever/always cooking for
                him/her’}[SB]


\begin{tabular}{llll}  
    &   Singular  &   Plural  &  \\
2
                     \textsuperscript{
                    nd}Person &   
                       \vernacular{
                      w-aá\ob {\downstep}mú[té{\downstep}éshélá]\cb }  &   
                       \vernacular{
                      mw-aá\ob {\downstep}mú[té{\downstep}éshélá]\cb }  &  \\
3
                     \textsuperscript{
                    rd}Person &   
                       \vernacular{
                      y-aá\ob {\downstep}mú[té{\downstep}éshélá]\cb }  &   
                       \vernacular{
                      β-aá\ob {\downstep}mú[té{\downstep}éshélá]\cb }  &  \\
\end{tabular}
%\caption{\nocaption}
    
\z

 
\ea\label{ex:xSubjHabØOP} 
Subject Choice in the Habitual /Ø/
                + OP \gloss{‘...is/are
                ever/always asking
                him/her’}[SB]


\begin{tabular}{llll}  
    &   Singular  &   Plural  &  \\
1
                     \textsuperscript{
                    st}Person &   
                       \vernacular{
                      n-aá\ob {\downstep}mú[rééβá]\cb }  &   
                       \vernacular{
                      khw-aá\ob {\downstep}mú[rééβá]\cb }  &  \\
2
                     \textsuperscript{
                    nd}Person &   
                       \vernacular{
                      w-aá\ob {\downstep}mú[rééβá]\cb }  &   
                       \vernacular{
                      mw-aá\ob {\downstep}mú[rééβá]\cb }  &  \\
3
                     \textsuperscript{
                    rd}Person &   
                       \vernacular{
                      y-aá\ob {\downstep}mú[rééβá]\cb }  &   
                       \vernacular{
                      β-aá\ob {\downstep}mú[rééβá]\cb }  &  \\
\end{tabular}
%\caption{\nocaption}
    
\z



\subsubsection{Habitual: Passives}\label{sec:sP8xPassives}

Except for the fact that Habitual forms with
              passives have a final fall on a long final syllable
              where Habitual forms without passives have a level H
              on a short final syllable, the tonal properties of
              Habitual forms are identical with and without the
              passive suffix. 

 
\ea\label{ex:xHabPassives} 
Habitual: Passives \gloss{‘s/he is ever/always
                being...’}[SB]


\begin{tabular}{lllll}  
  \multicolumn{2}{l}{/H/ Stems } &   \multicolumn{2}{l}{/Ø/ Stems } &  \\

                       \vernacular{―――
                      }  &   
                       \gloss{‘cut’}  &   
                       \vernacular{
                      y-aá[{\downstep}lákhúúl-ú-a]}  &   
                       \gloss{‘released’}  &  \\

                       \vernacular{
                      y-aá[{\downstep}tsúúnzúún-ú-a]}  &   
                       \gloss{‘sucked’}  &   
                       \vernacular{
                      y-aá[{\downstep}kálúshíts-ú-a]}  &   
                       \gloss{‘returned
                      for’}  &  \\
\end{tabular}
%\caption{\nocaption}
    
\z

 The data in \REF{ex:xHabPassives} are consistent with the analysis of
              passive H distribution developed in § \sectref{sec:sP2aOtherTenses} . The
              melodic H is first assigned to the FV via \regel{Final MHA} \REF{ex:xFinalMHA} , thereby licensing the passive H. \regel{Passive H
              Assignment} \REF{ex:xPassiveHAssignmentFinal} then
              assigns the passive H to the penultimate mora within
              the final syllable, feeding \regel{Meeussen’s Rule} \REF{ex:xMeeussensRule} . Finally, the
              passive H spreads as far as it can via \regel{Leftward Spread} \REF{ex:xLeftwardSpread} .



\subsubsection{Pattern 8: Other Verbal Contexts}\label{sec:sP8xOtherTenses}

The surface tone pattern of Habitual Negative verb
              forms are identical to those of the affirmative. In
              the Habitual Negative, the melodic H surfaces across
              the full length of the verb stem in both /H/ and /Ø/
              verbs, with /H/ verbs also expressing the root H in
              constructions involving an object prefix. 

 Additionally, the Habitual Negative similarly
              shares tonal properties with parallel affirmative
              forms in phrase-medial position and when passivized.
              Despite these similarities, there is a large degree
              of overlap between the surface tonal patterns
              predicted by an analysis which assumes that a melodic
              H is assigned in the Habitual Negative via the same
              mechanisms through which the Habitual derives its
              surface tonal properties and an analysis in which the
              negative marker \vernacular{tá}is
              sufficient to trigger phrase-medial deletion of the
              melodic H, only to generate identical stem tone
              patterns via \regel{H Tone Anticipation} \REF{ex:xHToneAnticipation} . The present
              work remains agnostic regarding the best analysis of
              the Habitual Negative.

 
\ea\label{ex:xP8xTenses} 
Other Pattern 8 Verbal
                Contexts 


\begin{tabular}{llll}  
  a.  &   Habitual Negative  &   
                       \vernacular{SP-aá[ROOT-a]
                      tá(awe)}  &  \\
\end{tabular}
%\caption{\nocaption}
    
\z

 As shown below, both /H/ and /Ø/ verbs surface
              with a H spanning the full length of the verb stem,
              just as in the affirmative. 

 
\ea\label{ex:xP8xHStems} 
Morphologically Simple /H/ Stems
                [SB] \footnote{\label{fn:nP8xGlosses} The examples included in the current section
                  use \vernacular{
                  -khálak-} \gloss{‘cut’}and \vernacular{
                  -khóng’oond-} \gloss{‘knock’}to
                  illustrate the properties of /H/ verbs, and \vernacular{
                  -kulix-} \gloss{‘name’}and \vernacular{
                  -lakhuul-} \gloss{‘release’}as
                  representative of /H/ and /Ø/ verbal roots,
                  respectively. The basic gloss for the Habitual
                  Negative is \gloss{‘s/he is not
                  ever/always...ing’}.


}%



\begin{tabular}{llllll}  
    &   Subj  &   Tns  &   Stem  &   Neg  &  \\
Hab Neg  &   
                       \vernacular{y-}  &   
                       \vernacular{aá}  &   
                       \vernacular{
                      \ob [{\downstep}kháláká]\cb }  &   
                       \vernacular{tá}  &  \\

                       \vernacular{y-}  &   
                       \vernacular{aá}  &   
                       \vernacular{
                      \ob [{\downstep}khóng’óóndá]\cb }  &   
                       \vernacular{tá}  &  \\
\end{tabular}
%\caption{\nocaption}
    
\z

 
\ea\label{ex:xP8xØStems} 
Morphologically Simple /Ø/ Stems
                [SB] 


\begin{tabular}{llllll}  
    &   Subj  &   Tns  &   Stem  &   Neg  &  \\
Hab Neg  &   
                       \vernacular{y-}  &   
                       \vernacular{aá}  &   
                       \vernacular{
                      \ob [{\downstep}kúlíshá]\cb }  &   
                       \vernacular{tá}  &  \\

                       \vernacular{y-}  &   
                       \vernacular{aá}  &   
                       \vernacular{
                      \ob [{\downstep}lákhúúlá]\cb }  &   
                       \vernacular{tá}  &  \\
\end{tabular}
%\caption{\nocaption}
    
\z

  \regel{Initial Lowering}is
              useful in accounting for the failure of the root H to
              surface, though the surface stem tone properties may
              be derived in either of two ways: (i) a melodic H is
              assigned to the FV via \regel{Final MHA}, in turn
              spreading left via \regel{Leftward Spread}or
              (ii) the negative marker \vernacular{tá}triggers
              phrase-medial deletion of the melodic H, then spreads
              its H left via \regel{H Tone
              Anticipation}.

 Indeed, the absence of downstep between the FV of
              the verb and the negative marker \vernacular{
              tá}favors the latter approach. It is
              common for \vernacular{tá}to be
              downstepped relative to word final melodic Hs (e.g.
              Pattern 2a: \vernacular{a-kha\ob [lekhá]\cb 
              {\downstep}tá} \gloss{‘let him/her not leave
              (behind)’}), making it unusual that it is
              not observed in this case if the H span is, in fact,
              a melodic H span. On the other hand, the absence of
              downstep follows naturally from an analysis in which
              the FV and \vernacular{tá}are
              associated to the same H, as would result from the H
              of the negative marker undergoing \regel{H Tone
              Anticipation}.

 However, assuming that the H span appearing on the
              verb stem originates on the negative marker comes
              with its own set of challenges. First, we learned
              from Pattern 5a (§ \sectref{sec:sP5aOtherTenses} ) that
              the negative marker appears not to be sufficient to
              trigger phrase-medial melodic H deletion, even in
              contexts like the Present, where a full DP complement
              clearly triggers deletion of the melodic H: compare \vernacular{
              a\ob [sééβúlaanga]\cb } \gloss{‘s/he is saying
              bye’}and \vernacular{a\ob [seeβulaanga]\cb 
              muundu} \gloss{‘s/he is saying bye to
              someone’}. Additionally, Pattern 5b (§ \sectref{sec:sP5bOtherTenses} )
              showed that there are contexts in which the H of the
              negative marker is not downstepped even when
              following a H that is unambiguously inflectional,
              e.g. \vernacular{a-li\ob [khalaká]\cb 
              {\downstep}tá} \gloss{‘s/he will not
              cut’}.

 As in the Habitual, the H contributed by a single
              object prefix fails to surface, while the root H
              re-emerges in /H/ stems, spreading left through the
              object prefix via \regel{Plateau}. The
              melodic (or negative) H continues to be realized on
              all other moras of the macrostem.

 
\ea\label{ex:xP8xOPHStems} 
/H/ Stems with an Object Prefix
                [SB] 


\begin{tabular}{lllllll}  
    &   Subj  &   Tns  &   Obj  &   Stem  &   Neg  &  \\
Hab Neg  &   
                       \vernacular{y-}  &   
                       \vernacular{aá}  &   
                       \vernacular{\ob {\downstep}mú}  &   
                       \vernacular{
                      [khá{\downstep}láká]\cb }  &   
                       \vernacular{tá}  &  \\

                       \vernacular{y-}  &   
                       \vernacular{aá}  &   
                       \vernacular{\ob {\downstep}mú}  &   
                       \vernacular{
                      [khó{\downstep}ng’óóndá]\cb }  &   
                       \vernacular{tá}  &  \\
\end{tabular}
%\caption{\nocaption}
    
\z

 
\ea\label{ex:xP8xOPØStems} 
/Ø/ Stems with an Object Prefix
                [SB] 


\begin{tabular}{lllllll}  
    &   Part  &   Subj  &   Obj  &   Stem  &   Neg  &  \\
Hab Neg  &   
                       \vernacular{y-}  &   
                       \vernacular{aá}  &   
                       \vernacular{\ob {\downstep}mú}  &   
                       \vernacular{
                      [kúlíkhá]\cb }  &   
                       \vernacular{tá}  &  \\

                       \vernacular{y-}  &   
                       \vernacular{aá}  &   
                       \vernacular{\ob {\downstep}mú}  &   
                       \vernacular{
                      [lákhúúla]\cb }  &   
                       \vernacular{tá}  &  \\
\end{tabular}
%\caption{\nocaption}
    
\z

 Though the stem tone properties of the Habitual
              Negative forms in a phrase-medial context are again
              identical to parallel affirmative forms, the argument
              for phrase-position induced deletion of the melodic H
              in this context is stronger than in verbs followed
              only by \vernacular{tá}; while
              there is some precedent for lack of downstep between
              a melodic H and the negative marker \vernacular{tá},
              there are no cases in which post-verbal complements
              with an initial H (e.g. \vernacular{
              mú{\downstep}yáyi} \gloss{‘boy’}) are
              not downstepped following a verb-final melodic H.
              Furthermore, verbs in the affirmative Habitual
              unambiguously lose the melodic H phrase-medially; it
              would not be surprising that the Habitual Negative
              patterns with the Habitual in this respect. Examples
              are provided below of both /H/ and /Ø/ stems before a
              H-toned noun, \vernacular{
              mú{\downstep}yáyi} \gloss{‘boy’}, and a
              toneless noun \vernacular{muundu} \gloss{
              ‘person/somebody’}.

 
\ea\label{ex:xP8xPhraseMed} 
Constructions Like the Habitual
                Phrase Medially [SB] 


\begin{tabular}{llll}  
  
                       \textbf{Hab Neg}  &   
                    /H/  &   
                       \vernacular{
                      y-aá[{\downstep}kháláká] mú{\downstep}yá{\downstep}yí tá}  &  \\

                       \vernacular{
                      y-aá[{\downstep}kháláká] múúndú tá}  &  \\
  &     &  \\

                    /Ø/  &   
                       \vernacular{
                      y-aá[{\downstep}lákhúúlá] mú{\downstep}yá{\downstep}yí
                      tá}  &  \\

                       \vernacular{
                      y-aá[{\downstep}lákhúúlá] múúndú tá}  &  \\
\end{tabular}
%\caption{\nocaption}
    
\z

 An analysis of the above forms may take the
              following form: the melodic H is deleted in a
              phrase-final context and \regel{H Tone
              Anticipation}applies liberally, spreading
              the Hs of H-toned complements as far left as they can
              reach. The H of the negative marker spreads onto the
              initial mora of the stem, even through an
              underlyingly toneless complement.

 There is no data available relating to the effect
              of subject choice on the tonal properties of the
              Habitual Negative. 

 Finally, observe that Habitual Negative forms with
              the passive suffix have precisely the same tonal
              properties as those without: both /H/ and /Ø/ verbs
              surface with a H spanning the full length of the verb
              stem, and the H of the negative marker is not
              downstepped. 

 
\ea\label{ex:xP8xPassive} 
/H/ \& /Ø/ Stems with the
                Passive Suffix \gloss{‘s/he is not
                ever/always
                being...’}[SB]


\begin{tabular}{lllll}  
  
                       \textbf{Hab Neg}  &   /H/  &   
                       \vernacular{
                      y-aá[{\downstep}khálák-w-á] tá}  &   
                       \gloss{‘cut’}  &  \\

                       \vernacular{
                      y-aá[{\downstep}tsúúnzúún-w-á] tá}  &   
                       \gloss{‘sucked’}  &  \\
/Ø/  &   
                       \vernacular{
                      y-aá[{\downstep}lákhúúl-w-á] tá}  &   
                       \gloss{‘released’}  &  \\

                       \vernacular{
                      y-aá[{\downstep}kálúshíts-w-á] tá}  &   
                       \gloss{‘returned
                      for’}  &  \\
\end{tabular}
%\caption{\nocaption}
    
\z

 The lack of downstep is particularly interesting
              in the above data because such forms derive from
              intermediate representations which include a final
              fall on the final syllable. Recall that the final
              syllable is underlyingly bimoraic in the above forms,
              with one mora being contributed by the passive and a
              second by the FV. The final syllable is ultimately
              shortened before the negative marker due to \regel{Non-Final
              Shortening}(cf. \vernacular{
              y-aá\ob [{\downstep}khálákúa]\cb } \vernacular{‘s/he is
              always/ever being cut’}).



\section{Summary of Tonal Melodies}\label{sec:sSummaryOfTonalMelodies}

The preceding description identifies 8 primary
          Patterns. Among these, some further divide into
          sub-Patterns. Sub-Patterns differ from one another to
          varying degrees, with Patterns 1a and 1b and Patterns 2a
          and 2b differing only with respect to the tone of subject
          or tense prefixes. Patterns 5a, 5b, and 5c are distinct
          in more significant ways: (i) Pattern 5a constructions
          assign a melodic H to the third stem syllable, while
          Patterns 5b and 5c target the FV, and (ii) the melodic H
          is lost phrase-medially in Patterns 5a and 5b, but
          retained in Pattern 5c constructions. 

 Melodic Hs target a number of positions. Describing
          the position targeted requires reference to prosodic and
          morphological constituents at multiple hierarchical
          levels: both the mora and the syllable, and the stem
          boundary and the macrostem boundary. Targeted positions
          are as summarized in \REF{ex:xTargetsOfMHA} , reproduced below.

 
\ea\label{ex:xTargetsOfMHAReproduced} 
Targets of Melodic H
            Assignment 


\begin{tabular}{llll}  
  a.  &   Initial mora of the stem  &   Pattern 6  &  \\
b.  &   Second mora of the stem  &   Patterns 2, 4, 5, 6, 7  &  \\
c.  &   All moras of the third syllable of the
                stem  &   Pattern 5a  &  \\
d.  &   Final mora of the stem  &   Patterns 5b, 5c, 6, 7, 8  &  \\
e.  &   Initial mora of the macrostem  &   Pattern 4  &  \\
f.  &   2
                 \textsuperscript{nd}mora after
                the initial syllable of the macrostem &   Pattern 3  &  \\
\end{tabular}
%\caption{\nocaption}
    
\z

 Up to two melodic Hs may appear on the verb.
          Following the construction-specific assignment of melodic
          Hs to the verb, underlying and inflectional Hs are
          subject to a number of general and construction-specific
          tonal rules, listed in \REF{ex:xSummaryOfTonalRules} below in the
          order in which they apply. Not all ordering relationships
          are crucial.

 
\ea\label{ex:xSummaryOfTonalRules} 

\begin{tabular}{lll}  
  Rule  &   Pattern Restrictions  &  \\

                   \regel{Initial
                  Lowering} \REF{ex:xInitialLowering}   &   Requires a H
                 \textsubscript{M} &  \\

                   \regel{Macrostem-initial
                  MHA} \REF{ex:xMInitialMHA}   &   Pattern 4  &  \\

                   \regel{Initial
                  MHA} \REF{ex:xInitialMHA}   &   Pattern 6  &  \\

                   \regel{Default
                  MHA} \REF{ex:xDefaultMHA}   &   Not Pattern 8  &  \\

                   \regel{Pinball
                  Shift} \REF{ex:xPinballShift}   &   Pattern 2b  &  \\

                   \regel{Meeussen’s
                  Rule} \REF{ex:xMeeussensRule}   &     &  \\

                   \regel{L Spread
                  I} \REF{ex:xLSpreadI}   &   (Pattern 1a?)  &  \\

                   \regel{L Spread
                  II} \REF{ex:xLSpreadII}   &   Patterns 5 \& 6  &  \\

                   \regel{Final
                  Spread} \REF{ex:xFinalSpread}   &   Pattern 5a  &  \\

                   \regel{Final MHA} \REF{ex:xFinalMHA}   &   Patterns 5b, 5c, 6, \& 7  &  \\

                   \regel{Final Rise
                  Elimination} \REF{ex:xFinalRiseElimination}   &     &  \\

                   \regel{Non-Final
                  Shortening} \REF{ex:xInitialLowering}   &     &  \\

                   \regel{Subjunctive
                  MHA} \REF{ex:xSubjunctiveMHAFinal}   &   Pattern 3  &  \\

                   \regel{Third Syllable
                  MHA} \REF{ex:xThirdSyllableMHA}   &   Pattern 5a  &  \\

                   \regel{Pre-Penultimate
                  Doubling} \REF{ex:xPrePenultimateDoubling}   &     &  \\

                   \regel{Heavy
                  Shift} \REF{ex:xHeavyShift}   &   Pattern 6  &  \\

                   \regel{Rightward
                  Spread} \REF{ex:xRightwardSpread}   &   Pattern 6  &  \\

                   \regel{(Fall
                  Elimination)} \REF{ex:xFallElimination}   &   Pattern 7  &  \\

                   \regel{Passive H
                  Assignment} \REF{ex:xPassiveHAssignmentFinal}   &   Requires a H
                 \textsubscript{M}; Complex &  \\

                   \regel{Leftward
                  Spread} \REF{ex:xLeftwardSpread}   &   Pattern 7 \& 8  &  \\

                   \regel{Plateau} \REF{ex:xPlateau}   &   Not Pattern 2b  &  \\

                   \regel{H Tone
                  Anticipation} \REF{ex:xHToneAnticipation}   &     &  \\
\end{tabular}
%\caption{\nocaption}
    
\z

  \regel{Passive H Assignment}is
          designated as having ‘Complex’ pattern restrictions. The
          passive suffix \vernacular{-u
          }realizes
          a H under conditions which are not predicted by Pattern
          membership. To surface H, the passive suffix must appear
          in a context inflected with a melodic H (not Pattern 1).
          In addition, the verb form must satisfy one of the
          following conditions, either (i) a melodic H surfaces on
          the verb stem or (ii) the verb form includes the
          perfective suffix.

 Finally, the following display summarizes the pattern
          membership of all constructions discussed in the
          preceding description. An asterisk denotes constructions
          in which melodic Hs are lost phrase-medially. The
          affirmative Imperatives have the unusual property that
          melodic Hs are lost phrase-medially only in forms without
          object prefixes. 

 
\ea\label{ex:xTonalMelodiesSummary} 
Idakho Tonal Melodies:
            Summary 


\begin{tabular}{lllll}  
    &     &     &     &  \\

                   \textbf{Pattern 1a}  &   \multicolumn{2}{l}{(§
                 \sectref{sec:sPattern1a} )} &     &  \\
Near Future  &   /H/  &   
                   \vernacular{
                  a-la\ob [kálaanga]\cb }  &   
                   \gloss{‘s/he will
                  fry’}  &  \\
  &   /Ø/  &   
                   \vernacular{
                  a-la\ob [lakhuula]\cb }  &   
                   \gloss{‘s/he will
                  release’}  &  \\
Near Future  &   /H/  &   
                   \vernacular{
                  a-la\ob [ká{\downstep}láángá]\cb  tá}  &   
                   \gloss{‘s/he will not
                  fry’}  &  \\
Neg  &   /Ø/  &   
                   \vernacular{a-la\ob [lákhúúlá]\cb 
                  tá}  &   
                   \gloss{‘s/he will not
                  release’}  &  \\
Perfect  &   /H/  &   
                   \vernacular{
                  a-a\ob [kálaanji]\cb }  &   
                   \gloss{‘s/he has
                  fried’}  &  \\
  &   /Ø/  &   
                   \vernacular{
                  a-a\ob [lakhuuli]\cb }  &   
                   \gloss{‘s/he has
                  released’}  &  \\
Perfect  &   /H/  &   
                   \vernacular{a-a\ob [ká{\downstep}láánjí]\cb 
                  tá}  &   
                   \gloss{‘s/he has not
                  fried’}  &  \\
Neg  &   /Ø/  &   
                   \vernacular{a-a\ob [lákhúúlí]\cb 
                  tá}  &   
                   \gloss{‘s/he has not
                  released’}  &  \\
Infinitive  &   /H/  &   
                   \vernacular{
                  khu\ob [kálaanga]\cb }  &   
                   \gloss{‘to fry’}  &  \\
  &   /Ø/  &   
                   \vernacular{
                  khu\ob [lakhuula]\cb }  &   
                   \gloss{‘to release’}  &  \\
  &     &     &     &  \\

                   \textbf{Pattern 1b}  &   \multicolumn{2}{l}{(§
                 \sectref{sec:sPattern1b} )} &     &  \\
Imm. Past  &   /H/  &   
                   \vernacular{
                  y-á{\downstep}khá\ob [kálaanga]\cb }  &   
                   \gloss{‘s/he just
                  fried’}  &  \\
  &   /Ø/  &   
                   \vernacular{
                  y-ákha\ob [lakhuula]\cb }  &   
                   \gloss{‘s/he just
                  released’}  &  \\
Imm. Past  &   /H/  &   
                   \vernacular{
                  y-á{\downstep}khá\ob [ká{\downstep}láángá]\cb  tá}  &   
                   \gloss{‘s/he just
                  fried’}  &  \\
Neg  &   /Ø/  &   
                   \vernacular{
                  y-á{\downstep}khá\ob [lákhúúlá]\cb  tá}  &   
                   \gloss{‘s/he just
                  released’}  &  \\
Rem. Future  &   /H/  &   
                   \vernacular{
                  y-á{\downstep}khá\ob [kálaanjɛ]\cb }  &   
                   \gloss{‘s/he just
                  fried’}  &  \\
  &   /Ø/  &   
                   \vernacular{
                  y-ákha\ob [lakhuulɛ]\cb }  &   
                   \gloss{‘s/he just
                  released’}  &  \\
Rem. Future  &   /H/  &   
                   \vernacular{
                  y-á{\downstep}khá\ob [ká{\downstep}láánjɛ́]\cb  tá}  &   
                   \gloss{‘s/he just
                  fried’}  &  \\
Neg  &   /Ø/  &   
                   \vernacular{
                  y-á{\downstep}khá\ob [lákhúúlɛ́]\cb  tá}  &   
                   \gloss{‘s/he just
                  released’}  &  \\
  &     &     &     &  \\

                   \textbf{Pattern 2a}  &   \multicolumn{2}{l}{(§
                 \sectref{sec:sPattern2a} )} &  \\
Subjunctive  &   /H/  &   
                   \vernacular{a-kha\ob [kalaanga]\cb 
                  tá}  &   
                   \gloss{‘let him/her not
                  fry!’}  &  \\
Neg  &   /Ø/  &   
                   \vernacular{a-kha\ob [lakhúula]\cb 
                  tá}  &   
                   \gloss{‘let him/her not
                  release!’}  &  \\
Imperative
                 \textsubscript{sg} &   /H/  &   
                   \vernacular{u-kha\ob [kalaanga]\cb 
                  tá}  &   
                   \gloss{‘do not fry!’}  &  \\
Neg  &   /Ø/  &   
                   \vernacular{u-kha\ob [lakhúula]\cb 
                  tá}  &   
                   \gloss{‘do not
                  release!’}  &  \\
Imperative
                 \textsubscript{pl} &   /H/  &   
                   \vernacular{mu-kha\ob [kalaanji]\cb 
                  tá}  &   
                   \gloss{‘do not
                  seduce!’}  &  \\
Neg  &   /Ø/  &   
                   \vernacular{mu-kha\ob [lakhúuli]\cb 
                  tá}  &   
                   \gloss{‘do not name!’}  &  \\
Hod. Perf.  &   /H/  &   
                   \vernacular{
                  a\ob [kalaanji]\cb }  &   
                   \gloss{‘s/he fried’}  &  \\
  &   /Ø/  &   
                   \vernacular{
                  a\ob [lakhúuli]\cb }  &   
                   \gloss{‘s/he
                  released’}  &  \\
Hod. Perf.  &   /H/  &   
                   \vernacular{a\ob [kalaanji]\cb 
                  tá}  &   
                   \gloss{‘s/he did not
                  fry’}  &  \\
Neg  &   /Ø/  &   
                   \vernacular{a\ob [lakhúuli]\cb 
                  tá}  &   
                   \gloss{‘s/he did not
                  release’}  &  \\
Future  &   /H/  &   
                   \vernacular{sh-a\ob [khalaka]\cb 
                  tá}  &   
                   \gloss{‘s/he will not
                  cut’}  &  \\
Neg  &   /Ø/  &   
                   \vernacular{sh-a\ob [kulíkha]\cb 
                  tá}  &   
                   \gloss{‘s/he will not
                  name’}  &  \\
Present  &   /H/  &   
                   \vernacular{shi\ob [ndumulaa]\cb 
                  tá}  &   
                   \gloss{‘I do not hit’}  &  \\
Neg  &   /Ø/  &   
                   \vernacular{shi\ob [nomólómaa]\cb 
                  tá}  &   
                   \gloss{‘I do not
                  talk’}  &  \\
  &     &     &     &  \\

                   \textbf{Pattern 2b}  &   \multicolumn{2}{l}{(§
                 \sectref{sec:sPattern2b} )} &  \\
Conditional  &   /H/  &   
                   \vernacular{na-á-kha\ob [kalaka]\cb 
                  tá}  &   
                   \gloss{‘if s/he does not
                  cut’}  &  \\
Neg  &   /Ø/  &   
                   \vernacular{
                  na-á-kha\ob [kulíkha]\cb  tá}  &   
                   \gloss{‘if s/he does not
                  name’}  &  \\
  &     &     &     &  \\

                   \textbf{Pattern 3}  &   \multicolumn{2}{l}{(§
                 \sectref{sec:sPattern3} )} &  \\
Subjunctive  &   /H/  &   
                   \vernacular{
                  a\ob [kalaánjɛ]\cb }  &   
                   \gloss{‘let him/her
                  fry’}  &  \\
  &   /Ø/  &   
                   \vernacular{
                  a\ob [seeβulɪ́]\cb }  &   
                   \gloss{‘let him/her say
                  goodbye’}  &  \\
Crast. Fut.  &   /H/  &   
                   \vernacular{
                  na-a\ob [kalaánjɛ]\cb }  &   
                   \gloss{‘s/he will
                  fry’}  &  \\
  &   /Ø/  &   
                   \vernacular{
                  na-a\ob [seebulɪ́]\cb }  &   
                   \gloss{‘s/he will say
                  goodbye’}  &  \\
Crast. Fut.  &   /H/  &   
                   \vernacular{na-a\ob [kalaánjɛ]\cb 
                  tá}  &   
                   \gloss{‘s/he will not
                  fry’}  &  \\
Neg  &   /Ø/  &   
                   \vernacular{na-a\ob [kalushítsɪ]\cb 
                  tá}  &   
                   \gloss{‘s/he will not
                  return’}  &  \\
  &     &     &     &  \\

                   \textbf{Pattern 4}  &   \multicolumn{2}{l}{(§
                 \sectref{sec:sPattern4} )} &  \\
Rem. Past  &   /H/  &   
                   \vernacular{
                  y-aa\ob [βóolitsa]\cb }  &   
                   \gloss{‘s/he seduced’}  &  \\
  &   /Ø/  &   
                   \vernacular{
                  y-aa\ob [kúlikha]\cb }  &   
                   \gloss{‘s/he named’}  &  \\
Rem. Past  &   /H/  &   
                   \vernacular{y-aa\ob [βóolitsa]\cb 
                  tá}  &   
                   \gloss{‘s/he did not
                  seduce’}  &  \\
Neg  &   /Ø/  &   
                   \vernacular{y-aa\ob [kúlikha]\cb 
                  tá}  &   
                   \gloss{‘s/he did not
                  name’}  &  \\
  &     &     &     &  \\

                   \textbf{Pattern 5a}  &   \multicolumn{2}{l}{(§
                 \sectref{sec:sPattern5a} )} &  \\
*Present  &   /H/  &   
                   \vernacular{
                  a\ob [βoolitsáánga]\cb }  &   
                   \gloss{‘s/he is
                  seducing’}  &  \\
  &   /Ø/  &   
                   \vernacular{
                  a\ob [kulíkhaanga]\cb }  &   
                   \gloss{‘s/he is
                  naming’}  &  \\
*Present  &   /H/  &   
                   \vernacular{a\ob [βoolitsáánga]\cb 
                  tá}  &   
                   \gloss{‘s/he is not
                  seducing’}  &  \\
Neg  &   /Ø/  &   
                   \vernacular{a\ob [kulíkhaanga]\cb 
                  tá}  &   
                   \gloss{‘s/he is not
                  naming’}  &  \\
*Persistive  &   /H/  &   
                   \vernacular{
                  a-shi\ob [βoolitsáánga]\cb }  &   
                   \gloss{‘s/he is still
                  seducing’}  &  \\
  &   /Ø/  &   
                   \vernacular{
                  a-shi\ob [kulíkhaanga]\cb }  &   
                   \gloss{‘s/he is still
                  naming’}  &  \\
*Persistive  &   /H/  &   
                   \vernacular{
                  a-shi\ob [βoolitsáánga]\cb  tá}  &   
                   \gloss{‘s/he is not still
                  seducing’}  &  \\
Neg  &   /Ø/  &   
                   \vernacular{
                  a-shi\ob [kulíkháanga]\cb  tá}  &   
                   \gloss{‘s/he is not still
                  naming’}  &  \\
  &     &     &     &  \\

                   \textbf{Pattern 5b}  &   \multicolumn{2}{l}{(§
                 \sectref{sec:sPattern5b} )} &  \\
*Indef. Fut.  &   /H/  &   
                   \vernacular{
                  a-li\ob [βoolitsá]\cb }  &   
                   \gloss{‘s/he will
                  seduce’}  &  \\
  &   /Ø/  &   
                   \vernacular{
                  a-li\ob [kulíkha]\cb }  &   
                   \gloss{‘s/he will
                  name’}  &  \\
*Indef. Fut.  &   /H/  &   
                   \vernacular{a-li\ob [βoolitsá]\cb 
                  {\downstep}tá}  &   
                   \gloss{‘s/he will not
                  seduce’}  &  \\
Neg  &   /Ø/  &   
                   \vernacular{a-li\ob [kulíkha]\cb 
                  tá}  &   
                   \gloss{‘s/he will not
                  name’}  &  \\
  &     &     &     &  \\

                   \textbf{Pattern 5c}  &   \multicolumn{2}{l}{(§
                 \sectref{sec:sPattern5c} )} &  \\
Conditional  &   /H/  &   
                   \vernacular{
                  na-á\ob [{\downstep}βóólítsá]\cb }  &   
                   \gloss{‘if s/he
                  seduces’}  &  \\
  &   /Ø/  &   
                   \vernacular{
                  na-á\ob [{\downstep}kúlíkha]\cb }  &   
                   \gloss{‘if s/he
                  names’}  &  \\
  &     &     &     &  \\
  &     &     &     &  \\

                   \textbf{Pattern 6}  &   \multicolumn{2}{l}{(§
                 \sectref{sec:sPattern6} )} &  \\
(*)Imperative
                 \textsubscript{sg} &   /H/  &   
                   \vernacular{
                  \ob [βoolitsá]\cb }  &   
                   \gloss{‘seduce!’}  &  \\
  &   /Ø/  &   
                   \vernacular{
                  \ob [kúlí{\downstep}khá]\cb }  &   
                   \gloss{‘name!’}  &  \\
(*)Imperative
                 \textsubscript{pl} &   /H/  &   
                   \vernacular{
                  \ob [βoolitsí]\cb }  &   
                   \gloss{‘seduce!’}  &  \\
  &   /Ø/  &   
                   \vernacular{
                  \ob [kúlí{\downstep}shí]\cb }  &   
                   \gloss{‘name!’}  &  \\
  &     &     &     &  \\

                   \textbf{Pattern 7}  &   \multicolumn{2}{l}{(§
                 \sectref{sec:sPattern7} )} &  \\
*Hest. Perf.  &   /H/  &   
                   \vernacular{
                  y-a\ob [βóólíítsɪ́]\cb }  &   
                   \gloss{‘s/he seduced’}  &  \\
  &   /Ø/  &   
                   \vernacular{
                  y-a\ob [lákhú{\downstep}úlí]\cb }  &   
                   \gloss{‘s/he
                  released’}  &  \\
*Hest. Perf.  &   /H/  &   
                   \vernacular{
                  y-a\ob [βóólíítsɪ́]\cb  tá}  &   
                   \gloss{‘s/he did not
                  seduce’}  &  \\
Neg  &   /Ø/  &   
                   \vernacular{y-a\ob [lákhú{\downstep}úlí]\cb 
                  tá}  &   
                   \gloss{‘s/he did not
                  release’}  &  \\
  &     &     &     &  \\

                   \textbf{Pattern 8}  &   \multicolumn{2}{l}{(§
                 \sectref{sec:sPattern8} )} &  \\
*Habitual  &   /H/  &   
                   \vernacular{
                  y-aá\ob [{\downstep}βóólítsá]\cb }  &   
                   \gloss{‘s/he always
                  seduces’}  &  \\
  &   /Ø/  &   
                   \vernacular{
                  y-aá\ob [{\downstep}sééβúlá]\cb }  &   
                   \gloss{‘s/he always says
                  bye’}  &  \\
*Habitual  &   /H/  &   
                   \vernacular{
                  y-aá\ob [{\downstep}βóólítsá]\cb  tá}  &   
                   \gloss{‘s/he doesn’t always
                  seduce’}  &  \\
Neg  &   /Ø/  &   
                   \vernacular{
                  y-aá\ob [{\downstep}sééβúlá]\cb  tá}  &   
                   \gloss{‘s/he doesn’t always say
                  bye’}  &  \\
\end{tabular}
%\caption{\nocaption}
    
\z



\chapter{The Path to Predictability}\label{sec:cPathToPredictability}

The present chapter seeks to trace the course of a
        diachronic re-analysis of the lexical contrast between */H/
        vs. */Ø/ verbs reconstructed to Proto-Bantu ( \citealt{rStevick1969} )
        within the Luhya cluster of languages. In so doing, the
        chapter highlights a number of factors that catalyzed
        re-analysis of the lexical contrast and motivated
        associated tonal developments. A speaker preference for the
        position of morphological boundaries to be prosodically
        well-cued is identified as playing a central role in
        shaping the course of transitioning from a ‘conservative’
        verbal tone system to a ‘predictable’ one.

 Tonal melodies are commonly an exponent of tense,
        aspect, and mood in Bantu. Among the many factors which may
        influence the realization of tone in a particular verb form
        is the lexical class to which a verbal root belongs. Most
        contemporary Bantu languages may be characterized as
        ‘conservative’, maintaining the historical /H/ vs. /Ø/
        contrast reconstructed for Proto-Bantu and a set of
        morpho-syntactic contexts in which this contrast is
        directly revealed. In Idakho (Bantu, Kenya, ida, JE411)
        infinitives, for example, /H/ verbs realize a H on the
        stem-initial mora while /Ø/ verbs surface all L. The only H
        appearing in the infinitival forms in \REF{ex:xConservativeIdakho} is contributed by
        the verbal root.

 
\ea\label{ex:xConservativeIdakho} 
Lexical contrast preserved in
          conservative languages: \textit{Idakho}


\begin{tabular}{llllll}  
  \multicolumn{5}{l}{ } &  \\
\multicolumn{2}{l}{
                 \textbf{*/H/} } &     &   \multicolumn{2}{l}{
                 \textbf{*/Ø/} } &  \\

                 \vernacular{xu[léera]}  &   
                 \gloss{‘to bring’}  &     &   
                 \vernacular{xu[reeβa]}  &   
                 \gloss{‘to bring’}  &  \\

                 \vernacular{xu[βúkula]}  &   
                 \gloss{‘to take’}  &     &   
                 \vernacular{
                xu[lomaloma]}  &   
                 \gloss{‘to talk’}  &  \\

                 \vernacular{
                xu[βóyong’ana]}  &   
                 \gloss{‘to go around’}  &     &   
                 \vernacular{
                xu[kaβuluxaɲːa]}  &   
                 \gloss{‘to separate’}  &  \\
\end{tabular}
%\caption{\nocaption}
    
\z

 Not all contemporary Bantu languages have inherited the
        historical contrast as one between /H/ and /Ø/ verbs as in
        Idakho. \citealt{rMarlo2008a} , \citealt{rMarlo2009a} , \citealt{rMarlo2013} develops a typology of Bantu
        verbal tone systems based on reflexes of the Proto-Bantu
        */H/ vs. */Ø/ contrast, in which he identifies
        ‘conservative’, ‘predictable’, and ‘reversive’ verbal tone
        systems all within Luhya.

 ‘Predictable’ systems, in the sense of \citealt{rOdden1989} , have lost the
        lexical contrast; these languages have a single lexical
        tone class. An associated property of ‘predictable’ systems
        is the lack any morpho-syntactic context uninflected with a
        tonal melody. Cognate forms of the Idakho data above have
        strikingly different tonal properties in the Nyala-West
        variety of Luhya. Nyala-West infinitives are produced with
        an inflectional H spanning from the second stem syllable
        through the final.

 
\ea\label{ex:xPredictableNyalaWest} 
Lexical contrast lost in predictable
          languages: \textit{
          Nyala-West}


\begin{tabular}{llllll}  
  \multicolumn{5}{l}{
               \citealt{rMarlo2007} , \citealt{rEbarbEtAlInPrep} } &  \\
\multicolumn{2}{l}{
                 \textbf{*/H/} } &     &   \multicolumn{2}{l}{
                 \textbf{*/Ø/} } &  \\

                 \vernacular{
                o-xú[reerá]}  &   
                 \gloss{‘to bring’}  &     &   
                 \vernacular{
                o-xú[reeβá]}  &   
                 \gloss{‘to bring’}  &  \\

                 \vernacular{
                o-xú[βukúlá]}  &   
                 \gloss{‘to take’}  &     &   
                 \vernacular{
                o-xú[lomálómá]}  &   
                 \gloss{‘to talk’}  &  \\

                 \vernacular{
                o-xú[βodóxáná]}  &   
                 \gloss{‘to go around’}  &     &   
                 \vernacular{
                o-xú[kaβúlá]}  &   
                 \gloss{‘to separate’}  &  \\
\end{tabular}
%\caption{\nocaption}
    
\z

 ‘Reversive’ verbal tone systems are intermediate
        between ‘conservative’ and ‘predictable’: they maintain a
        contrast between two lexical tone classes, as in
        ‘conservative’ systems, but lack a morpho-syntactic context
        in which verbs are uninflected with a tonal melody.
        Additionally, details of the tonal melodies often suggest
        that the language has undergone a reanalysis whereby the
        lexical contrast is one between /L/ and /Ø/ verbs (hence,
        ‘reversive’). In \REF{ex:xReversiveNyalaEast} , it can be seen
        that */Ø/ verbs in an historically uninflected context have
        an inflectional H spanning the entire verb stem, while */H/
        verbs are all H on the stem except on the initial syllable,
        where the root L is expressed.

 
\ea\label{ex:xReversiveNyalaEast} 
Lexical contrast re-analyzed in reversive
          languages: \textit{
          Nyala-East}


\begin{tabular}{llllll}  
  \multicolumn{5}{l}{ } &  \\
\multicolumn{2}{l}{
                 \textbf{*/H/} } &     &   \multicolumn{2}{l}{
                 \textbf{*/Ø/} } &  \\

                 \vernacular{
                o-xú[leechá]}  &   
                 \gloss{‘to bring’}  &     &   
                 \vernacular{
                o-xú[chééβá]}  &   
                 \gloss{‘to bring’}  &  \\

                 \vernacular{
                o-xú[xamáchá]}  &   
                 \gloss{‘to take’}  &     &   
                 \vernacular{
                o-xú[láxúúlá]}  &   
                 \gloss{‘to talk’}  &  \\

                 \vernacular{
                o-xú[wotoóxáná]}  &   
                 \gloss{‘to go around’}  &     &   
                 \vernacular{
                o-xú[kííngúúlírá]}  &   
                 \gloss{‘to separate’}  &  \\
\end{tabular}
%\caption{\nocaption}
    
\z

 Note that the term ‘reversive’ refers only to the
        reversal of historically /H/ verbal roots into
        synchronically /L/. With the exception of certain tense
        prefixes, this reversal appears not to have had
        consequences for the tonal specifications of other
        morphemes. 

 ‘Conservative’ verbal tone systems are widespread within
        Bantu, and ‘predictable’ systems are well attested in parts
        of Kenya, the DRC, and Tanzania ( \citealt{rOdden1989} , \citealt{rMarlo2013} ).
        Reversive languages are not as well attested or discussed
        (consult \citealt{rMarlo2013} for a
        concise summary of research on ‘reversive’ systems). The
        Luhya macrolanguage spoken in western Kenya is special in
        having multiple examples of all three types of verbal tone
        systems, while most macrolanguages have only one or two
        types. As the map in \REF{ex:xDialectMapReproduced} shows,
        ‘conservative’ varieties are located to the east, and
        ‘predictable’ varieties to the west. Tonally intermediate
        ‘reversive’ system are also geographically intermediate.
        Comparing the synchronic systems of Luhya varieties from
        east to west sheds light on how the transition to from
        ‘conservative’to ‘predictable’ system types must have
        progressed.

 
\ea\label{ex:xDialectMapReproduced} 
Luhya dialect map; Reproduced from \REF{ex:xDialectMap} 

%\includegraphics[width=\textwidth]{InkScape%20Images/Luyia%20Language%20Map_ToneTypes_Aug_8_2014,%20Labeled,%20reduced.pdf}

\z

 The present study is motivated by the research
        questions in \REF{ex:xCh3ResearchQuestions} , which have been
        asked in various forms and explored to varying degrees in a
        series of talks given by Michael Marlo in \citealt{rMarlo2008a} , \citealt{rMarlo2009a} , \citealt{rMarloOdden2011b} (with David Odden), and \citealt{rMarlo2013b} .

 
\ea\label{ex:xCh3ResearchQuestions} 

\begin{tabular}{lll}  
  a.  &   What is the correspondence between
              conservative, reversive, and predictable
              melodies?  &  \\
b.  &   Why does every verbal context become
              inflected with a tonal melody?  &  \\
c.  &   Why is the lexical contrast
              re-analyzed as between /L/ and /Ø/?  &  \\
d.  &   Why is the lexical contrast
              neutralized?  &  \\
\end{tabular}
%\caption{\nocaption}
    
\z

 The central thesis of the chapter is that tonal
        developments within Luhya have been guided by a speaker
        preference for prosodically well-cued morphological
        boundaries—a natural preference within the context of a
        linguistic system in which a heavy inflectional burden is
        borne by verbal tone melodies necessarily defined with
        reference to morphological boundaries. § \sectref{sec:sTonalMelodiesExtended} seeks
        to demonstrate the role this preference has played in the
        extension of tonal melodies into historically uninflected
        contexts and § \sectref{sec:sLossOfLexicalContrast} documents its role in the loss of the lexical
        contrast between */H/ and */Ø/ classes of verbs. § \sectref{sec:sCh3Summary} concludes.


\section{Extension of Tonal Melodies into Historically
          Uninflected Contexts}\label{sec:sTonalMelodiesExtended}

Like infinitives, verbs in the Near Future are not
          inflected with a tonal melody in conservative varieties,
          but are so inflected in reversive and predictable
          varieties. The present section argues that the
          introduction of tonal melodies in constructions like the
          Near Future, for which there is no historical precedence
          of tonal inflection, proceeded in a series of steps; a
          crucial first among these steps was driven forward by a
          preference for clear prosodic demarcation of
          morphological boundaries. 

 The argument will proceed as follows: first, I will
          describe the basic tonal properties of verbs in the Near
          Future across several conservative, reversive, and
          predictable Luhya varieties. Second, I will sketch a
          proposal for the sequence of tonal developments which led
          to the predictable systems of western Luhya. I will then
          show how data from a context inflected with a melodic H,
          the Present tense, help explain the transition from one
          stage of development to the next. 

 In all conservative varieties, the root H surfaces on
          the initial syllable of the stem in */H/ verbs. Toneless
          verbs, on the other hand, surface all L. In Nyore and
          Logoori, the root H spreads leftward onto the tense
          prefix. 

 
\ea\label{ex:xConservativeNearFut} 
The Near Future in tonally conservative
            varieties: \gloss{‘s/he / they / we
            will...’}


\begin{tabular}{lllllll}  
  \multicolumn{6}{l}{Tiriki: 
                 \citealt{rMarloInPrepB} ; Tachoni: \citealt{rOdden2009} ; Logoori: \citealt{rLeung1991} } &  \\
  &   \multicolumn{2}{l}{
                   \textbf{*/H/} } &     &   \multicolumn{2}{l}{
                   \textbf{*/Ø/} } &  \\

                Idakho  &   
                   \vernacular{
                  a-la[xálaka]}  &   
                   \gloss{‘cut’}  &     &   
                   \vernacular{
                  a-la[kulixa]}  &   
                   \gloss{‘name’}  &  \\

                   \vernacular{
                  a-la[βóyong’ana]}  &   
                   \gloss{‘go around’}  &     &   
                   \vernacular{
                  a-la[kalushitsa]}  &   
                   \gloss{‘return’}  &  \\

                Tiriki  &   
                   \vernacular{
                  a-la[xálaka]}  &   
                   \gloss{‘cut’}  &     &   
                   \vernacular{
                  a-la[valitsa]}  &   
                   \gloss{‘count’}  &  \\

                   \vernacular{
                  a-la[βóyong’ana]}  &   
                   \gloss{‘go around’}  &     &   
                   \vernacular{
                  a-la[kaluxana]}  &   
                   \gloss{‘turn around’}  &  \\

                Tachoni  &   
                   \vernacular{
                  baa-la[béka]}  &   
                   \gloss{‘cut’}  &     &   
                   \vernacular{
                  baa-la[lima]}  &   
                   \gloss{‘cultivate’}  &  \\

                   \vernacular{
                  baa-la[téexa]}  &   
                   \gloss{‘go around’}  &     &   
                   \vernacular{
                  baa-la[chiinga]}  &   
                   \gloss{‘carry’}  &  \\
  &     &     &     &     &     &  \\

                Nyore  &   
                   \vernacular{
                  a-lá[xámara]}  &   
                   \gloss{‘grab’}  &     &   
                   \vernacular{
                  a-la[kulixa]}  &   
                   \gloss{‘name’}  &  \\

                   \vernacular{
                  a-lá[xálaanga]}  &   
                   \gloss{‘fry’}  &     &   
                   \vernacular{
                  a-la[kiinguula]}  &   
                   \gloss{‘raise up’}  &  \\

                Logoori  &   
                   \vernacular{
                  a-rá[véga]}  &   
                   \gloss{‘shave’}  &     &   
                   \vernacular{
                  kʊ-ra[saamba]}  &   
                   \gloss{‘burn’}  &  \\

                   \vernacular{
                  kʊ-rá[kálaga]}  &   
                   \gloss{‘cut’}  &     &   
                   \vernacular{
                  kʊ-ra[guriza]}  &   
                   \gloss{‘sell’}  &  \\
\end{tabular}
%\caption{\nocaption}
    
\z

 The same context in reversive systems has a similar
          tonal profile, except that an inflectional H surfaces on
          the verb stem as well. In historically */H/ verbs,
          synchronically /L/, the inflectional H surfaces from the
          second through the final syllable; the inflectional H
          occupies all syllables of the stem in */Ø/ verbs. 

 In Wanga and Marachi, the tonal class of the verb
          predicts the surface tone of the tense prefix: in */H/
          verbs it is L, in */Ø/ verbs, it is H. Both tonal classes
          are H throughout the enter stem, but */H/ verbs have
          downstep between the first and second syllables. 

 In the Nyala-East and Bukusu varieties, the tense
          prefix is always H. */H/ verbs are L on the initial
          syllable of the stem, but H on all other syllables. */Ø/
          verbs have a level H which spans the entire stem. 

 
\ea\label{ex:xReversiveNearFut} 
The Near Future in tonally reversive
            varieties: \gloss{‘s/he
            will...’}


\begin{tabular}{lllllll}  
  \multicolumn{6}{l}{Wanga: 
                 \citealt{rEbarbGreenMarlo2014} ; Marachi: \citealt{rMarlo2007} ; Bukusu: \citealt{rMutonyi2000} } &  \\
  &   \multicolumn{2}{l}{
                   \textbf{*/H/} } &     &   \multicolumn{2}{l}{
                   \textbf{*/Ø/} } &  \\

                Wanga  &   
                   \vernacular{
                  a-la[xó{\downstep}lólá]}  &   
                   \gloss{‘cough’}  &     &   
                   \vernacular{
                  a-lá[púrúxá]}  &   
                   \gloss{‘name’}  &  \\

                Marachi  &   
                   \vernacular{
                  a-la[βú{\downstep}kúlá]}  &   
                   \gloss{‘take’}  &     &   
                   \vernacular{
                  a-lá[βákálá]}  &   
                   \gloss{‘spread to
                  dry’}  &  \\

                   \vernacular{
                  a-la[βó{\downstep}dóxáná]}  &   
                   \gloss{‘go around’}  &     &   
                   \vernacular{
                  a-lá[lómálómá]}  &   
                   \gloss{‘turn around’}  &  \\
  &     &     &     &     &     &  \\

                Nyala-E  &   
                   \vernacular{
                  a-lá[siindíxá]}  &   
                   \gloss{‘push’}  &     &   
                   \vernacular{
                  a-lá[kúlíxá]}  &   
                   \gloss{‘name’}  &  \\

                   \vernacular{
                  a-lá[paangúlúlá]}  &   
                   \gloss{‘disarrange’}  &     &   
                   \vernacular{
                  a-lá[kííngúúlá]}  &   
                   \gloss{‘raise up’}  &  \\

                Bukusu
                 \footnote{\label{fn:nBukusuPhraseMedialNearFut} A tonal melody is not realized in */Ø/ Near
                  Future forms without an object prefix in Bukusu,
                  but will when there is an object prefix: \vernacular{
                  a-lá\ob lu[kálámá]\cb } \gloss{‘s/he will look
                  up’}( \citealt{rMutonyi2000} ).


}%
 &   
                   \vernacular{
                  a-lá[βukúlá]}  &   
                   \gloss{‘take’}  &     &   
                   \vernacular{
                  a-la[lima]}  &   
                   \gloss{‘cultivate’}  &  \\

                   \vernacular{
                  a-lá[xalákílá]}  &   
                   \gloss{‘cut for’}  &     &   
                   \vernacular{
                  a-la[kalama]}  &   
                   \gloss{‘look up’}  &  \\
\end{tabular}
%\caption{\nocaption}
    
\z

 Finally, Near Future forms in predictable varieties
          are invariably inflected with an inflectional H span from
          the second through the final stem syllable in both
          historically */H/ and */Ø/ verbs. 

 
\ea\label{ex:xPredictableNearFut} 
The Near Future in tonally predictable
            varieties: \gloss{‘s/he /
            will...’}


\begin{tabular}{lllllll}  
  \multicolumn{6}{l}{Nyala-West: 
                 \citealt{rMarlo2007} ; Khayo: \citealt{rMarlo2009b} ; Tura: \citealt{rMarlo2008b} } &  \\
  &   \multicolumn{2}{l}{
                   \textbf{*/H/} } &     &   \multicolumn{2}{l}{
                   \textbf{*/Ø/} } &  \\

                Nyala-W  &   
                   \vernacular{
                  a-ná[βukúlá]}  &   
                   \gloss{‘take’}  &     &   
                   \vernacular{
                  a-ná[kulá]}  &   
                   \gloss{‘buy’}  &  \\

                   \vernacular{
                  a-ná[siindíxá]}  &   
                   \gloss{‘push’}  &     &   
                   \vernacular{
                  a-ná[reeβá]}  &   
                   \gloss{‘ask’}  &  \\

                Khayo  &   
                   \vernacular{
                  a-ná[βukúlá]}  &   
                   \gloss{‘take’}  &     &   
                   \vernacular{
                  a-ná[reeβá]}  &   
                   \gloss{‘ask’}  &  \\

                   \vernacular{
                  a-ná[fuuníxá]}  &   
                   \gloss{‘cover’}  &     &   
                   \vernacular{
                  a-ná[burúxá]}  &   
                   \gloss{‘fly’}  &  \\

                Tura  &   
                   \vernacular{
                  a-lá[βukúlá]}  &   
                   \gloss{‘take’}  &     &   
                   \vernacular{
                  a-lá[βakálá]}  &   
                   \gloss{‘set out to
                  dry’}  &  \\

                   \vernacular{
                  a-lá[xaráángá]}  &   
                   \gloss{‘fry’}  &     &     &     &  \\
\end{tabular}
%\caption{\nocaption}
    
\z

 The first step in the transition to predictability
          begins with the observation that, although conservative
          varieties do not inflect the Near Future with a tonal
          melody, they all attest a rule of \regel{H Tone Anticipation} \REF{ex:xHToneAnticipation} which, in certain
          circumstances, generates stem tone patterns that are
          strikingly similar to those found in the same context in
          reversive varieties ( \citealt{rMarlo2008a} , \citealt{rMarlo2009a} ; \citealt{rMarloOdden2011b} ).

  \regel{H Tone Anticipation}is a
          cross-linguistically rare rule of leftward spreading by
          which H-toned post-verbal elements may spread onto the
          verb stem, as exemplified in the Tiriki data below. The H
          from \vernacular{mulína} \gloss{‘friend’}and \vernacular{kálaha} \gloss{‘slowly’}spread
          onto the verb; through the initial stem syllable in */Ø/
          verbs, through the second stem syllable in */H/
          verbs.

 
\ea\label{ex:xConservativeHTA} 
 \regel{H Tone
            Anticipation}in tonally conservative
            varieties: \gloss{‘s/he
            will...’}


\begin{tabular}{llllll}  
  \multicolumn{5}{l}{Tiriki: 
                 \citealt{rMarlo2012} , \citealt{rMarloInPrepB} } &  \\
\multicolumn{2}{l}{
                   \textbf{*/H/} } &     &   \multicolumn{2}{l}{
                   \textbf{*/Ø/} } &  \\

                   \vernacular{a-la[rhúmula]
                  mulimi}  &   
                   \gloss{‘beat a
                  farmer’}  &     &   
                   \vernacular{a-la[moloma]
                  vwaangu}  &   
                   \gloss{‘talk quickly’}  &  \\

                   \vernacular{a-la[rhú{\downstep}múlá]
                  múlína}  &   
                   \gloss{‘beat a
                  friend’}  &     &   
                   \vernacular{a-la[mólómá]
                  kálaha}  &   
                   \gloss{‘talk slowly’}  &  \\
\end{tabular}
%\caption{\nocaption}
    
\z

 One route by which a verbal tone system of the
          conservative type may develop into a system of the
          predictable type is sketched below. 

 Beginning from the Tiriki situation as depicted in \REF{ex:xConservativeNearFut} and \REF{ex:xConservativeHTA} , the first step is
          to innovate a rule whereby the root H spreads onto the
          tense prefix \vernacular{la-}. This is
          the synchronic situation in Logoori (and likely Nyore as
          well).

 
\ea\label{ex:xConservativeHTAPlusLeftwardDoubling} 
 \regel{Leftward
            Doubling}in a tonally conservative variety: \gloss{‘they
            will...’}


\begin{tabular}{llllll}  
  \multicolumn{5}{l}{Logoori: 
                 \citealt{rMarloOdden2011} } &  \\
\multicolumn{2}{l}{
                   \textbf{*/H/} } &     &   \multicolumn{2}{l}{
                   \textbf{*/Ø/} } &  \\

                   \vernacular{va-rá[vé{\downstep}gá]
                  gáráha}  &   
                   \gloss{‘shave slowly’}  &     &   
                   \vernacular{va-rá[mɔ́rɔ́má]
                  gáráha}  &   
                   \gloss{‘talk quickly’}  &  \\
\end{tabular}
%\caption{\nocaption}
    
\z

  \regel{Leftward
          Doubling}spreading root Hs onto the tense
          prefix in combination with \regel{H Tone Anticipation} \REF{ex:xHToneAnticipation} spreading
          post-verbal Hs onto the tense prefix in */Ø/ verbs \footnote{\label{fn:nLogooriHTAisDifferent} In this respect, \regel{H Tone Anticipation}in
            Logoori appears to differ from the same rule in Tiriki.
            In the Logoori, post-verbal Hs spread all the way
            through the tense prefix, but stops at the left edge of
            the stem in Tiriki.


}%
\vernacular{lá-}and a
          tonal melody which surfaces throughout most or all of the
          stem. Shortly, I will show how this reanalysis proceeds.
          Following this, I will argue that this re-analysis is
          reinforced or even primed by the emergence of a new tonal
          pattern in the Present tense, which comes to be inflected
          with a melodic H on the second syllable through the final
          in */H/ verbs and a melodic H from the initial syllable
          through the final in */Ø/ verbs.

 Reanalyzing the forms in \REF{ex:xConservativeHTAPlusLeftwardDoubling} above as involving a H-toned tense prefix
          and a melodic H as described above necessitates a
          concomitant re-analysis of downstep after the initial
          syllable in */H/ verbs. If one assumes that the tense
          prefix is toneless and the H span on the stem is not
          inflectional (assumptions which reflect the synchronic
          properties of conservative systems), downstep after the
          initial syllable is the natural outcome of spreading one
          H across several L-toned moras up until it abuts a
          phonologically distinct H via \regel{H Tone Anticipation}.
          The root H spreads left onto the tense prefix by an
          innovative rule. No downstep is observed in */Ø/ verbs,
          in which the post-verbal H spreads clear onto the tense
          prefix, unimpeded by a lexical H tone.

 Though no contemporary reversive system exhibits
          exactly this pattern, I hypothesize that the predictable
          tonal systems passed through a stage of development at
          which the surface stem tone patterns in \REF{ex:xConservativeHTAPlusLeftwardDoubling} could be observed in Near Future forms both
          phrase-medially as well as phrase-finally. In such a
          system, the H spans in Near Future forms must be analyzed
          as inflectional. \regel{H Tone
          Anticipation}cannot generate the H spans from
          the second and initial syllables through the final in
          */H/ and */Ø/ verbs, respectively, in phrase-final forms
          because there is no following H-toned word.

 
\ea\label{ex:xHypotheticalStage3} 
Hypothetical next stage, the initial
            state of reversiveness: \gloss{‘s/he
            will...’}


\begin{tabular}{llllll}  
  \multicolumn{5}{l}{ } &  \\
\multicolumn{2}{l}{
                   \textbf{*/H/} } &     &   \multicolumn{2}{l}{
                   \textbf{*/Ø/} } &  \\

                   \vernacular{a-lá[khá{\downstep}láká]
                  (...)}  &   
                   \gloss{‘cuts’}  &     &   
                   \vernacular{a-lá[kúlíkhá]
                  (...)}  &   
                   \gloss{‘names’}  &  \\
\end{tabular}
%\caption{\nocaption}
    
\z

 I argue that in */Ø/ verbs, the observation that the
          tense prefix is H necessitates re-analyzing tense
          prefixes as underlyingly /H/. Tonal melodies in Luhya do
          not target positions outside of the macrostem, and
          pattern-specific spreading rules never result in H spans
          that extend beyond the left stem boundary—a
          generalization that notably does not apply to \regel{H Tone Anticipation}in
          Logoori, as shown in \REF{ex:xConservativeHTAPlusLeftwardDoubling} .
          Once the H spanning the full length of the stem in */Ø/
          verbs is re-analyzed as inflectional, the H on the tense
          prefix may, and perhaps must, be re-analyzed as
          contributed by the tense prefix itself.

 Once the tense prefix is analyzed as /H/, accounting
          for the pattern of downstep in */H/ verbs requires a
          concomitant re-analysis of the lexical contrast as one
          between /L/ vs. /Ø/, rather than /H/ vs. /Ø/, if one
          assumes that downstep is observed only when a floating L
          intervenes between distinct phonological Hs. 

 The melodic H spans appearing in Near Future forms may
          be analyzed as triggering lowering of the root H via \regel{Initial Lowering} \REF{ex:xInitialLowering} —an effect which is
          observed within conservative Luhya in tonally inflected
          constructions. \footnote{\label{fn:nInitLoweringVsRMR}  \regel{Initial Lowering}may
            be unique to Idakho, with other Bantu languages within
            and outside of Luhya deriving similar tonal patterns
            through \regel{Reverse Meeussen’s
            Rule}, which deletes Hs before an adjacent H.
            See \citealt{rEbarbGreenMarlo2014} for
            some discussion how these rules play a role in Luhya
            tonology.


}%


 Downstep in */H/ verbs at the hypothesized stage may
          be analyzed in the following way. The principles of
          melodic H assignment generate a melodic H span from the
          second stem syllable through the final. The H of the
          tense prefix \vernacular{lá-}then
          spreads onto the initial syllable of the stem via a rule \regel{Prefix Spread},
          delinking the root L in the process. Downstep is observed
          between the H span of the tense prefix and that of the
          melodic H in this case because of the floating L that
          intervenes between the two. Downstep is not observed in
          */Ø/ verbs because no floating L intervenes between the H
          of the tense prefix on the pre-stem syllable and the
          melodic H which spans from the initial syllable through
          the final.

 The next step in the transition to predictability is
          one at which, as in the previous stage, the tense prefix
          is H-toned \vernacular{lá-}and the
          melodic H surfaces on the second stem syllable through
          the final in historically /H/, but synchronically /L/,
          verbs and on the initial syllable of the stem through the
          final in */Ø/ verbs. The difference is that the tense
          prefix H does not shift onto the stem-initial syllable at
          this stage. This is the synchronic situation in
          Nyala-East.

 
\ea\label{ex:xReversiveNoPrefixalSpread} 
No \regel{Prefixal
            Spread}in a tonally reversive variety: \gloss{‘s/he
            will...’}


\begin{tabular}{llllll}  
  \multicolumn{5}{l}{Nyala-East } &  \\
\multicolumn{2}{l}{
                   \textbf{*/H/} } &     &   \multicolumn{2}{l}{
                   \textbf{*/Ø/} } &  \\

                   \vernacular{
                  a-lá[siindíxá]}  &   
                   \gloss{‘push’}  &     &   
                   \vernacular{
                  a-lá[kúlíxá]}  &   
                   \gloss{‘name’}  &  \\
\end{tabular}
%\caption{\nocaption}
    
\z

 The final step in the transition to predictability in
          the Near Future is the neutralization of the lexical
          contrast. In all of Luhya’s documented predictable
          varieties, the melody has neutralized in favor of the
          pattern associated with */H/ verbs, namely: melodic H on
          the second stem syllable through the final. This is shown
          in the Tura data below. 

 
\ea\label{ex:xNeutralizeLexicalContrast} 
Neutralizing the lexical contrast in a
            tonally predictable variety: \gloss{‘s/he
            will...’}


\begin{tabular}{llllll}  
  \multicolumn{5}{l}{Tura: 
                 \citealt{rMarlo2008b} } &  \\
\multicolumn{2}{l}{
                   \textbf{*/H/} } &     &   \multicolumn{2}{l}{
                   \textbf{*/Ø/} } &  \\

                   \vernacular{
                  a-lá[fuuníxá]}  &   
                   \gloss{‘cover’}  &     &   
                   \vernacular{
                  a-lá[βakálá]}  &   
                   \gloss{‘set out to
                  dry’}  &  \\
\end{tabular}
%\caption{\nocaption}
    
\z

 The development of tonal properties in the Near
          Future in the transition from tonal conservativeness to
          predictability is summarized below. 

 
\ea\label{ex:xSummaryNearFutTPT} 
Summary: From no melody to σ2-FV \vernacular{‘s/he / they
            will...’}


\begin{tabular}{llllll}  
  \multicolumn{5}{l}{ } &  \\
\multicolumn{5}{l}{
                   \textbf{Stage I} \ul{H Tone Anticipation (
                  } } &  \\
\multicolumn{5}{l}{ } &  \\

                   \vernacular{a-la[rhú{\downstep}múlá]
                  múlína}  &     &   
                   \vernacular{a-la[mólómá]
                  múlína}  &  \\

                   \gloss{‘beat a
                  friend’}  &     &   
                   \gloss{‘talk to a
                  friend’}  &  \\
  &   \multicolumn{4}{l}{ } &  \\
\multicolumn{5}{l}{
                   \textbf{Stage 2} \ul{Hs spread onto the
                  tense prefix (
                  } } &  \\
\multicolumn{5}{l}{ } &  \\

                   \vernacular{va-rá[vé{\downstep}gá]
                  gáráha}  &     &   
                   \vernacular{va-rá[mɔ́lɔ́má]
                  gáráha}  &  \\

                   \gloss{‘shave slowly’}  &     &   
                   \gloss{‘talk slowly’}  &  \\
  &   \multicolumn{4}{l}{ } &  \\
\multicolumn{5}{l}{
                   \textbf{Stage 3} \ul{Extend pattern to
                  phrase-final verbs, *la- > lá- (Hypothetical,
                  reversive)} } &  \\
\multicolumn{5}{l}{ } &  \\

                   \vernacular{
                  a-lá[khá{\downstep}láká]}  &     &   
                   \vernacular{
                  a-lá[kúlíxá]}  &  \\

                   \gloss{‘cut’}  &     &   
                   \gloss{‘name’}  &  \\
  &   \multicolumn{4}{l}{ } &  \\
\multicolumn{5}{l}{
                   \textbf{Stage 4} \ul{Lose Prefixal Spreading
                  (
                  } } &  \\
\multicolumn{5}{l}{ } &  \\

                   \vernacular{
                  a-lá[siindíxá]}  &     &   
                   \vernacular{
                  a-lá[kúlíxá]}  &  \\

                   \gloss{‘push’}  &     &   
                   \gloss{‘name’}  &  \\
  &   \multicolumn{4}{l}{ } &  \\
\multicolumn{5}{l}{
                   \textbf{Stage 5} \ul{Neutralize the lexical
                  contrast (
                  } } &  \\
\multicolumn{5}{l}{ } &  \\

                   \vernacular{
                  a-lá[fuuníxá]}  &     &   
                   \vernacular{
                  a-lá[βakálá]}  &  \\

                   \gloss{‘cover’}  &     &   
                   \gloss{‘set to dry’}  &  \\
\end{tabular}
%\caption{\nocaption}
    
\z

 The central thesis of this chapter helps to motivate
          the transition between Stages 2 and 3. That is, a
          preference for prosodically well-cued morphological
          boundaries has catalyzed the extension of a stem tone
          pattern found only in phrase-medial forms in which the
          verb is followed by a H-toned word to other environments.
          These other environments include both phrase-final forms
          as well as phrase-medial forms in which the verb is
          followed by a toneless word. 

 I argue that the extension of the σ2-FV / σ1-FV
          pattern did not begin as a tonal melody in the Near
          Future nor in any other historically uninflected
          constructions. Instead, I propose that this melody first
          emerged as such in the Present. \footnote{\label{fn:nPTPTIndefFutToo} In Idakho, the generalizations asserted herein
            regarding the Present tense apply just as much to
            affirmative imperatives lacking object prefixes
            (Pattern 6, § \sectref{sec:sPattern6} ), Indefinite
            Future and Indefinite Future Negative forms (Pattern
            5b, § \sectref{sec:sPattern5b} ), and other
            constructions related to the present (Pattern 5a, § \sectref{sec:sPattern5a} ).


}%


 The Present tense is marked in Tiriki by a melodic H
          that targets the second stem syllable through the penult
          in /H/ verbs and the second stem mora in /Ø/ verbs. In
          addition, it is characterized by lowering of the root H
          in forms lacking an object prefix. 

 
\ea\label{ex:xConservativePresentPhraseFinal} 
The Present in tonally conservative
            varieties, Phrase-Final: \gloss{‘s/he
            is...’}


\begin{tabular}{llllll}  
  \multicolumn{5}{l}{Tiriki (
                 \citealt{rMarloInPrepB} ): */H/, H \textsubscript{M}on σ3; */Ø/, H \textsubscript{M}on μ2} &  \\
\multicolumn{2}{l}{
                   \textbf{*/H/} } &     &   \multicolumn{2}{l}{
                   \textbf{*/Ø/} } &  \\

                   \vernacular{
                  a[voyóng’ánáánga]}  &   
                   \gloss{‘going around’}  &     &   
                   \vernacular{
                  a[cheéndaanga]}  &   
                   \gloss{‘walking’}  &  \\
\end{tabular}
%\caption{\nocaption}
    
\z

 The Present in tonally conservative varieties like
          Tiriki has the unusual property whereby the melodic H is
          lost phrase-medially. \regel{Initial Lowering}is
          triggered by the presence of a melodic H. Though its
          trigger is lost phrase-medially, the effects of \regel{Initial Lowering}persist
          phrase-medially, and the H of post-verbal words spreads
          onto the stem via \regel{H Tone Anticipation}.
          Note that spreading terminates immediately right of the
          lowered root H in */H/ verbs rather than extending onto
          the stem-initial mora as in */Ø/ verbs.

 
\ea\label{ex:xConservativePresentPhraseMedial} 
The Present in tonally conservative
            varieties, Phrase-Medial: \gloss{‘s/he is...quickly (
            }


\begin{tabular}{llllll}  
  \multicolumn{5}{l}{Tiriki (
                 \citealt{rMarloInPrepB} )} &  \\
\multicolumn{2}{l}{
                   \textbf{*/H/} } &   \multicolumn{2}{l}{
                   \textbf{*/Ø/} } &  \\

                   \vernacular{a[voyong’anaa]
                  vwaangu}  &   
                   \gloss{‘going around’}  &   
                   \vernacular{a[cheendaa]
                  vwaangu}  &   
                   \gloss{‘walking’}  &  \\

                   \vernacular{a[voyóng’ánáá]
                  kálaha}  &   
                   \gloss{‘going around’}  &   
                   \vernacular{a[chééndáá]
                  kálaha}  &   
                   \gloss{‘walking’}  &  \\
\end{tabular}
%\caption{\nocaption}
    
\z

 In combination with the Present’s lowering of the
          root H, the loss of the melodic H in phrase-medial forms
          creates a scenario in which there is no tonal cue for the
          stem boundary in either tonal class. This is the case in
          particular when the post-verbal word is toneless. 

 The tonal pattern associated with phrase-medial forms
          followed by a H-toned word generalizes to all
          phrase-medial forms. I argue that this is because doing
          so improves the prosodic cuing of the left morphological
          stem boundary in phrase-medial forms followed by toneless
          words, in which there is no H whose position can be
          described with reference to the verb stem. 

 This gives rise to situations like what is found
          synchronically in Marachi. Marachi realizes one tonal
          pattern in all phrase-final forms, but another in all
          phrase-medial forms, regardless of the tonal properties
          of the following word. Note that the pattern
          characterizing phrase-medial forms is the same as
          described above for Present tense verbs followed by a
          H-toned word in Tiriki. 

 
\ea\label{ex:xMarachiPresentTense} 
The Present in (one) tonally reversive
            variety: \gloss{‘s/he
            is...’}


\begin{tabular}{llllll}  
  \multicolumn{5}{l}{Marachi (
                 \citealt{rMarlo2007} )} &  \\
\multicolumn{2}{l}{
                   \textbf{*/H/} } &     &   \multicolumn{2}{l}{
                   \textbf{*/Ø/} } &  \\

                   \vernacular{
                  a[karáángaanga]}  &   
                   \gloss{‘frying’}  &     &   
                   \vernacular{
                  a[léxúúlaanga]}  &   
                   \gloss{‘releasing’}  &  \\

                   \vernacular{
                  a[karáángáángá]...}  &   
                   \gloss{‘frying...’}  &     &   
                   \vernacular{
                  a[léxúúláángá]...}  &   
                   \gloss{‘releasing...’}  &  \\
\end{tabular}
%\caption{\nocaption}
    
\z

 The contrast between phrase-medial and phrase-final
          forms is subsequently leveled in favor of the
          phrase-medial pattern. This is the synchronic situation
          in the reversive Nyala-East variety. Note that the thesis
          of this chapter does not offer an explanation for why the
          phrase-medial pattern is generalized to phrase-final
          forms rather than the reverse. 

 
\ea\label{ex:xNyala-EastPresentTense} 
The Present in (one) tonally reversive
            variety: \gloss{‘s/he
            is...’}


\begin{tabular}{llllll}  
  \multicolumn{5}{l}{Nyala-East } &  \\
\multicolumn{2}{l}{
                   \textbf{*/H/} } &     &   \multicolumn{2}{l}{
                   \textbf{*/Ø/} } &  \\

                   \vernacular{
                  a[xamácháángá]}  &   
                   \gloss{‘taking’}  &     &   
                   \vernacular{
                  a[lóóndáángá]}  &   
                   \gloss{‘following’}  &  \\

                   \vernacular{
                  a[xamácháángá]...}  &   
                   \gloss{‘taking...’}  &     &   
                   \vernacular{
                  a[llóóndáángá]...}  &   
                   \gloss{‘following...’}  &  \\
\end{tabular}
%\caption{\nocaption}
    
\z

 Once the surface stem tone properties of */H/ and
          */Ø/ verbs described above are analyzed as a bona fide
          tonal melody, as it has been in Nyala-East, the final
          step is then to extend the novel tonal melodies into
          historically uninflected contexts like the Near Future.
          The H spans on */H/ and */Ø/ verb stems that result from
          anticipating post verbal Hs onto the verb stem are
          re-analyzed as expressing the same tonal melody exhibited
          by the Present. The result is hypothetical Stage 3
          proposed in \REF{ex:xSummaryNearFutTPT} above and repeated
          below.

 
\ea\label{ex:xHypothesizedStageIII} 
Hypothesized Stage 3 \gloss{‘s/he is cutting /
            naming’}


\begin{tabular}{llllll}  
  \multicolumn{5}{l}{ } &  \\
\multicolumn{2}{l}{
                   \textbf{*/H/} } &     &   \multicolumn{2}{l}{
                   \textbf{*/Ø/} } &  \\
\multicolumn{2}{l}{
                   %\includegraphics[width=\textwidth]{InkScape%20Images/Path%20to%20Predictability%20Displays/HypothesizedStage3HVerb.pdf}
} &     &   \multicolumn{2}{l}{
                   %\includegraphics[width=\textwidth]{InkScape%20Images/Path%20to%20Predictability%20Displays/HypothesizedStage30Verb.pdf}
} &  \\
\end{tabular}
%\caption{\nocaption}
    
\z

 Above, I have argued that a preference for
          prosodically well-cued morphological boundaries played a
          crucial role in re-analyzing the surface tonal patterns
          of Present tense verbs followed by H-toned words as an
          inflectional tonal melody. This set the stage for
          parallel stem tone patterns in historically uninflected
          contexts like the Near Future to be re-analyzed in the
          same way. 

 In the following section, I will show how the same
          preference is contributing to the loss of the lexical
          contrast between */H/ and */Ø/ within Luhya. 



\section{Loss of the Lexical Contrast}\label{sec:sLossOfLexicalContrast}

The present section describes two cases in which the
          Proto-Bantu lexical contrast between */H/ and */Ø/ verbs
          has been neutralized in vowel-initial stems. In one case,
          neutralization favors the tonal pattern associated with
          */H/ verbs; the other neutralizes in favor of the tonal
          pattern associated with */Ø/. It will be argued that a
          preference for clear prosodic cues of morphological
          boundaries accounts for the direction of the
          neutralization in each case. 

 As a corollary to the above, I argue that the
          proffered analysis correctly predicts that the
          neutralization should have originated in vowel-initial
          stems. The corollary is based upon the premise that
          morphological segmentation is made easier when
          morphological boundaries align with syllable boundaries
          (or, conversely, that morphological segmentation is made
          more difficult when morphological boundaries do not align
          with syllable boundaries). The premise is supported in
          the work of \citealt{rDowning1998a} , \citealt{rDowning1998b} , \citealt{rDowning1999c} , \citealt{rDowning2006} , \citealt{rDowning2009b} and \citealt{rMarlo2014Unpublished} , in
          which numerous examples of exceptional patterns of
          morphology and phonology in onsetless syllables are
          attributed to morpho-prosodic misalignment, particularly
          in the areas of reduplication and tone.

 The first case involves an asymmetry in the ratio
          between vowel-initial verbs that realize the tonal
          pattern associated with */H/ verbs and those which
          realize the melody associated with */Ø/ verbs in
          historically uninflected constructions. \citealt{rMarlo2007} ; \citealt{rMarlo2009a} observed the following asymmetry in the
          lexicon of the reversive Marachi variety.

 
\ea\label{ex:xMarachiLexicalAsymmetry} 
Lexical asymmetries in
            Marachi 


\begin{tabular}{lll}  
  a.  &   Some */Ø/ verbs are variably
                produced with the */H/ pattern.  &  \\
b.  &   Some */Ø/ verbs are categorically
                produced with the */H/ pattern.  &  \\
c.  &   All vowel-initial */Ø/ roots longer
                than -VC- take the */H/ pattern  &  \\
\end{tabular}
%\caption{\nocaption}
    
\z

 The observations above inspire the following
          questions: (i) why should */Ø/ verbs be taking on the
          */H/ pattern?, and (ii) why is it that vowel-initial
          roots are disproportionately affected? 

 The asymmetry in vowel-initial roots appears to be
          emerging in the tonally conservative varieties as well.
          The table below reports tallies of the number of
          synchronically /H/ and /Ø/ vowel-initial verbs I
          identified during my 2012-2013 fieldwork. The figures
          represented here are based on vocabulary lists generated
          using a compilation, appended in \citealt{rMarlo2013} , of
          approximately 700 Proto-Bantu reconstructions as
          elicitation prompts. The compilation includes 46 */H/ and
          37 */Ø/ reconstructions for *j initial roots, from whence
          many of Luhya’s synchronically vowel-initial roots
          descend. \footnote{\label{fn:nMorphologicalComponent} Morphology may be playing a larger role in this
            transition than the present work recognizes. In Idakho,
            /Ø/ vowel-initial verbs reliably surface all L in most
            tonally uninflected contexts (such as in Infinitives
            and the Near Future). However, the same verbs are all
            invariably produced with a H on the initial mora of the
            stem in the Perfect—this despite the fact that
            consonant-initial /Ø/ verbs consistently surface all L
            in the same context (cf. § \sectref{sec:sP1aOtherTenses} ).


}%


 
\ea\label{ex:xHvs0AsymmetryInConservativeLuhya} 
/H/ vs. /Ø/ asymmetry
              in conservative Luhya vowel-initial verbs 
\z

 
\begin{tabular}{lllllllllllll}  
    &   \multicolumn{2}{l}{Idakho } &     &   \multicolumn{2}{l}{Nyore } &     &   \multicolumn{2}{l}{Kabras } &     &   \multicolumn{2}{l}{Marama } &  \\
  &   /H/  &   /Ø/  &     &   /H/  &   /Ø/  &     &   /H/  &   /Ø/  &     &   /H/  &   /Ø/  &  \\
C-Initial  &   209  &   194  &     &   202  &   210  &     &   177  &   192  &     &   172  &   190  &  \\
V-Initial  &   
                 \textit{53}  &   
                 \textit{22}  &     &   
                 \textit{59}  &   
                 \textit{14}  &     &   
                 \textit{51}  &   
                 \textit{35}  &     &   
                 \textit{40}  &   
                 \textit{27}  &  \\
\end{tabular}
%\caption{\nocaption}
     The figures above show an even distribution between
          /H/ and /Ø/ in verbs with consonant initial verb roots.
          Contrastively, in verbs with vowel-initial roots, that
          ratio is strongly skewed in favor of /H/. 

 Consider that in contexts not inflected by a tonal
          melody, phrase-final /Ø/ verbs do not have any tonal cues
          signaling the position of the left stem boundary. On the
          other hand, /H/ verbs have a strong prosodic cue of that
          boundary, surfacing with a H on the initial mora of the
          stem. \footnote{\label{fn:nIdakhoAllowsRisingTone} Idakho is noteworthy in allowing rising tones in
            forms like \vernacular{
            a-la[áβuxaɲːa]} \gloss{‘s/he will
            separate’}. In Logoori, parallel forms would
            be produced with a level H as the result of a regular
            phonological rule whereby rising tones are eliminated ( \citealt{rLeung1991} , \citealt{rMarloOdden2011} ).


}%


 
\ea\label{ex:xProsodicCuingInConservativeNearFuture} 
Prosodic cuing in conservative
            varieties: Near Future \gloss{‘s/he
            will...’}


\begin{tabular}{llllll}  
  \multicolumn{5}{l}{Idakho } &  \\
\multicolumn{2}{l}{
                   \textbf{*/H/} } &     &   \multicolumn{2}{l}{
                   \textbf{*/Ø/} } &  \\

                   \vernacular{
                  a-la[xálaka]}  &   
                   \gloss{‘cut’}  &     &   
                   \vernacular{
                  a-la[kulixa]}  &   
                   \gloss{‘name’}  &  \\

                   \vernacular{
                  a-la[áβuxaɲːa]}  &   
                   \gloss{‘separate’}  &     &   
                   \vernacular{
                  a-la[ambaxana]}  &   
                   \gloss{‘refuse’}  &  \\
\end{tabular}
%\caption{\nocaption}
    
\z

 I argue that the need for a clear cue of stem
          boundary position is greater in /Ø/ vowel-initial stems
          than in consonant-initial verbs because, in the
          vowel-initial stems, the left edge of the morphological
          stem does not coincide with a syllable boundary.
          Consequently, the tonal pattern associated with /H/ verbs
          is extended to /Ø/ verbs to better cue the stem boundary
          position. 

 Analyzing the gradual neutralization of the lexical
          contrast as described above provides for a unified
          account of two relatable observations: (i) the
          neutralization favors the pattern associated with /H/
          verbs and (ii) vowel-initial stems are the first to
          exhibit the asymmetry. 

 The second case of diachronic change in vowel-initial
          verbs relates to asymmetric patterns of free variation in
          the conservative Marama variety. In Marama, the Present
          Negative is inflected with a tonal melody characterized
          by lowering of the root H and an inflectional H span
          aligned with the left stem boundary in /Ø/ verbs. 

 
\ea\label{ex:xMaramaPresNegCInitial} 
C-Initial Tonal Patterns in Marama:
            Present Negative \gloss{‘s/he is
            not...’}


\begin{tabular}{llllll}  
  \multicolumn{5}{l}{ } &  \\
\multicolumn{2}{l}{
                   \textbf{*/H/} } &     &   \multicolumn{2}{l}{
                   \textbf{*/Ø/} } &  \\

                   \vernacular{sh-a[βukulaanga]
                  tá}  &   
                   \gloss{‘taking’}  &     &   
                   \vernacular{sh-a[kúlíxaanga]
                  tá}  &   
                   \gloss{‘naming’}  &  \\

                   \vernacular{sh-a[teeshelaanga]
                  tá}  &   
                   \gloss{‘cooking for’}  &     &   
                   \vernacular{sh-a[sééβúlaanga]
                  tá}  &   
                   \gloss{‘saying bye’}  &  \\
\end{tabular}
%\caption{\nocaption}
    
\z

 The Present Negative is tonally stable in verbs with
          consonant-initial roots, but there is an asymmetric
          pattern of variation in vowel-initial verbs. The */H/
          vowel-initial verbs may take the surface tonal pattern
          associated with either tonal class, but */Ø/ verbs are
          invariantly produced with the historically appropriate
          tonal pattern. 

 
\ea\label{ex:xMaramaPresNegVInitial} 
V-Initial Tonal Patterns in Marama:
            Present Negative \gloss{‘s/he is
            not...’}


\begin{tabular}{llllll}  
  \multicolumn{5}{l}{ } &  \\
\multicolumn{2}{l}{
                   \textbf{*/H/} } &     &   \multicolumn{2}{l}{
                   \textbf{*/Ø/} } &  \\

                   \vernacular{
                  shy-a[akáánilaanga] tá}  &   
                   \gloss{‘meeting’}  &     &   
                   \vernacular{shy-e[eyáánga]
                  tá}  &   
                   \gloss{‘wanting’}  &  \\

                   \vernacular{~shy-a[akaanilaanga]
                  tá}  &     &     &   
                   \vernacular{(*shy-e[eyaanga]
                  tá)}  &     &  \\
\end{tabular}
%\caption{\nocaption}
    
\z

 I analyze the asymmetry above as extending the tonal
          pattern associated with */Ø/ to */H/ verbs. This is done
          because the tonal pattern associated with */Ø/ verbs
          provides a prosodic cue of stem boundary position, while
          the pattern associated with */H/ verbs does not. 

 The analysis above again provides a unified account of
          two relatable observations: (i) the neutralization favors
          the pattern associated with /Ø/ verbs, unlike the
          previous case, and (ii) vowel-initial stems are the first
          to exhibit the asymmetry. 

 In addition to extending to verbs with /H/
          vowel-initial roots, the pattern associated with */Ø/
          verbs subsequently generalized to consonant /H/ verbs in
          the predictable varieties. 

 
\ea\label{ex:xPredictablePresNeg} 
The Present Negative in tonally
            predictable varieties: 


\begin{tabular}{llllll}  
  \multicolumn{5}{l}{Nyala-West: 
                 \citealt{rMarlo2007} ; Khayo: \citealt{rMarlo2009b} ; Tura: \citealt{rMarlo2008b} } &  \\
  &     &     &   \multicolumn{2}{l}{ } &  \\

                Nyala-W  &     &     &   
                   \vernacular{
                  si-xu[paangúlula]}  &   
                   \gloss{‘we are not
                  disarranging’}  &  \\

                Khayo  &     &     &   
                   \vernacular{
                  sy-áa[siindíxa]}  &   
                   \gloss{‘s/he is not
                  pushing’}  &  \\

                Tura  &     &     &   
                   \vernacular{
                  sí-βa[fuundíxa]}  &   
                   \gloss{‘they are not
                  knotting’}  &  \\
\end{tabular}
%\caption{\nocaption}
    
\z

 In most conservative varieties other than Marama, /Ø/
          verbs take a melodic H either on the second stem mora or
          on both the first and second stem moras in the Present
          Negative. In consonant-initial stems, the melodic H
          surfaces on the second stem syllable in these varieties
          only when the initial syllable is long, cf. Idakho: \vernacular{sh-a[kulíxa]
          tá} \gloss{‘s/he is not
          naming’}vs. \vernacular{sh-a[seéβula]
          tá} \gloss{‘s/he is not saying
          goodbye’}. However, the melodic H invaribly
          surfaces on the second stem mora in vowel-initial stems,
          even in Idakho: \vernacular{a-khe[eyéla]
          tá} \gloss{‘let him/her not wipe
          for’}. \footnote{\label{fn:nDataFromSubjNeg} This verb form is from the Subjunctive Negative,
            rather than the Present Negative. My corpus of Idakho
            data does not include vowel-initial data for the
            version of the Present Negative that takes the relevant
            tonal melody, but the Subjunctive Negative selects the
            same tonal melody and so serves as a suitable
            substition. 


}%


 The observation that all known predictable Luhya
          varieties have cognate melodies that, as in Marama,
          invariably target the second stem syllable reinforces the
          notion that this tonal change within Luhya is emanating
          from verb forms with vowel-initial roots,
          morpho-prosodically misaligned as they are. 

 In this section, I have argued that two cases of
          diachronic change in vowel-initial stems within Luhya may
          be analyzed as extending tonal melodies where doing so
          enhances prosodic cuing of the left stem boundary. 



\section{Summary}\label{sec:sCh3Summary}

In this chapter, I have traced the course of how
          several surface tonal patterns have developed in the
          transition from ‘conservative’ Luhya varieties to
          ‘predictable’. In § \sectref{sec:sTonalMelodiesExtended} I
          explain how tonal melodies emerged in contexts not
          historically inflected with a tonal melody, and in § \sectref{sec:sLossOfLexicalContrast} , I
          identify vowel-initial verb roots as catalysts in the
          neutralization of the lexical contrast. I argue that
          aspects of both phenomena may be analyzed as extending
          tonal melodies to novel contexts in the service of better
          signalling morphological boundaries.



\chapter{Conclusions}\label{sec:cConclusions}

Luhya verbal tone systems are complex. One of their
        striking features is the rich variety of positions targeted
        for melodic H assignment, and describing those positions
        frequently requires reference to multiple levels of
        morphological and prosodic structure. Given the complexity
        of Luhya verbal tone systems, it may not be particularly
        surprising that some Luhya varieties have developed in such
        a way as to enhance synchronicity between the morphological
        and phonological component. This dissertation has
        endeavored to provide extensive, novel documentation of
        Idakho verbal tone (Chapter \sectref{sec:cVerbalTone} ) and account
        for a suite of diachronic changes evidenced in the
        diversity within the Luhya macrolanguage.

 In Chapter \sectref{sec:cVerbalTone} I show that
        Idakho has a particularly rich system of verbal tone
        inflection, exhibiting 12 distinct tonal melodies which may
        be organized into 8 broad primary patterns. Each pattern is
        distinguished by the number of melodic H tones contributed
        by the morpho-syntactic context (0, 1 or 2) and the set of
        tonal rules it motivates, some of which may be construction
        specific. For an overview of the tonal melodies and their
        basic properties, refer to § \sectref{sec:sOverviewOfIdakhoVerbalTone} .

 In addition, Idakho attests several notable phenomena of
        broad general interest, some of which relate to the
        licensing of passive Hs, the loss of melodic Hs in
        phrase-medial position, subject-induced lowering of lexical
        Hs, and the tonal effects of prefixal negation markers. 

 Passive Hs are licensed only under disjunctively defined
        conditions (§ \sectref{sec:sP2bPassives} ). Passive Hs
        may surface in contexts inflected with a melodic H either
        when the melodic H ultimately surfaces on the verb stem as
        well or, failing that, when the perfective suffix is
        present. The available passive data relating to Present
        tense forms may be at odds with this characterization,
        because passive Hs are expected to surface in this context,
        but do so only as one of two free tonal variants, the
        second of which excludes the passive H. It also remains
        unknown at this time how the phrase-medial deletion of
        melodic Hs in a particular set of constructions interacts
        with the licensing of passive Hs. The relevant
        constructions include those exhibiting the properties of
        Patterns 5a-b (§ \sectref{sec:sPattern5a} - \sectref{sec:sPattern5b} ) and, in forms
        with an object prefix, Pattern 6 (§ \sectref{sec:sP6xImpSg} ).

 An additional issue in Idakho verbal tone that merits
        continued investigation is the exceptional lowering of root
        and object prefix Hs following first and second person
        subject prefixes in several constructions not inflected
        with a tonal melody. § \sectref{sec:sP1aSubjects} proffers an
        analysis in which first and second person subject prefixes
        underlyingly bear a L which spreads rightward through the
        initial mora of the stem. This approach may be
        descriptively adequate, though additional data is needed to
        decide between this and other plausible analyses.

 Finally, during the course of my study, I identified two
        constructions in which negation may be expressed either
        with or without a prefixal marker: \vernacular{shi-}in the
        Near Future (cf. fn \fnref{fn:nNegativeNearFut} ) and \vernacular{kha-}in the
        Subjunctive (§ \sectref{sec:sP3xPassives} ). In both
        cases, the presence of the negative marker correlates with
        a shift to a default tonal melody (in particular, Pattern
        2a:§ \sectref{sec:sPattern2a} ). It remains to
        be seen if prefixal negative markers are optional in other
        negative contexts and whether the presence of the negative
        prefix always correlates with a shift to the default
        melody.

 In Chapter \sectref{sec:cPathToPredictability} , I
        demonstrate how a preference for strong prosodic cuing of
        morphological boundaries, the left stem boundary in
        particular, motivates a suite of historical developments
        within Luhya. In particular, I implicate this preference in
        the extension of tonal melodies into contexts with no
        historical precedence for tonal inflection (§ \sectref{sec:sTonalMelodiesExtended} ) and
        the neutralization of the contrast between */H/ and */Ø/
        verbs reconstructed for Proto-Bantu.

\backmatter
\appendix
\chapter{Idakho Verbal Tone Questionnaire}\label{sec:aIdakhoVerbalToneQuestionnaire}

Appendix \appref{sec:aIdakhoVerbalToneQuestionnaire} is composed of three parts: the primary
          verbal tone questionnaire which informs the overwhelming
          majority of the preceding description in Chapter \sectref{sec:cVerbalTone} , and two
          pilot surveys. The first surveys the tonal influence that
          two enclitics exert on verbal forms, and the second very
          briefly surveys the properties of each of the
          constructions considered in the primary questionnaire
          when passivized and relativized and when put in
          interrogative contexts. Both of the primary consultants
          for the study responded to the primary questionnaire,
          while only SB responded to the latter two surveys.

 This appendix is included as a reference for the
          companion audio archive. With support from Pomona
          College, the audio archive may currently be downloaded
          freely via \href{https://pomona.box.com/s/wqjx59t87opqg94qdbhf}{
          https://pomona.box.com/s/wqjx59t87opqg94qdbhf}. The
          audio files contained therein are named according to the
          following convention: "Speaker Name\_Paradigm \#\_Repetition
          \#.wav". Below, paradigm numbers are indicated in
          parentheses before each set of verbal forms. In the
          archive, repetition numbers are only indicated if the
          relevant paradigm was recorded multiple times.


\section{Primary Questionnaire}\label{sec:sPrimaryQuestionnaire}

The questionnaire below serves as the primary basis
            for the preceding description of Idakho verbal tone.
            The version presented below differs from the version
            presented to the study’s consultants in several ways,
            which are articulated presently. 

 The questionnaire presented to the study’s
            consultants includes only one verbal form prompt per
            line, with full glosses appearing to the right as shown
            in \REF{ex:xQuestionnaireScreenshot} . To
            conserve space, the questionnaire prompts below are
            presented in two columns, wherever possible, with a
            gloss only for the root appearing to the right of each
            prompt. Additionally, especially long glosses have been
            simplified or reduced, for example: what appeared as \gloss{‘ask
            repeatedly’}in the administered questionnaire
            appears below as \gloss{‘ask (iter)’},
            just as \gloss{‘bend to one’s
            will’}in the administered questionnaire
            appears simply as \gloss{‘bend’}below.
            Finally, many morpheme boundaries appear in the
            adminstered questionnaire, but are excluded below.

 The transcriptions for each construction largely
            reflect the characterization provided in Chapter \sectref{sec:cVerbalTone} , however
            other tonal variants are not represented except when
            they constitute a clearly more common variant. The
            preceding description, which acknowledges many such
            tonal variants, may be consulted for some discussion of
            how such variants may be derived from or are related to
            the tonal melodies characterized as "basic" above.
            Furthermore, transcriptions of phrase-medial forms may
            be inconsistent with respect to whether \regel{Pre-Penultimate
            Doubling}will apply when doing so would
            spread a melodic H onto a mora immediately preceding a
            post-verbal element with an initial H; such forms merit
            further verification.

 Phrase-medial prompts include one of three
            post-verbal words: \vernacular{mú{\downstep}yáyi} \gloss{‘boy’,} \vernacular{musáatsa} \gloss{‘man’}, and \vernacular{muundu} \gloss{‘person’}. Forms
            with each of these post-verbal words were presented
            separately during consultant interviews, but they are
            collapsed into a single prompt in this appendix, e.g. \vernacular{ala[béka]
            mú{\downstep}yáyi/músáatsa/muundu}. The effects of \regel{H Tone
            Anticipation}are not represented in the
            collapsed prompts.

 The verb forms presented below are representive of
            those appearing in the version of the questionnaire
            presented to SB. There are a number of regular
            differences between the version of the questionnaire
            presented to SB and that presented to JI. For instance,
            JI prefers \vernacular{
            khu[kalukhitsa]}for \gloss{‘to return’}and \vernacular{táawe}for
            the negative particle, while SB prefers \vernacular{
            khu[kalushitsa]}and \vernacular{tá},
            respectively, among other differences.

 Researchers using the appendix as a data source
            should also note the following: First, three items
            regularly appear in paradigms of the wrong lexical
            class, in particular, those meaning \gloss{‘admire’}, \gloss{‘smack’}, and \gloss{‘belch’}. Each
            of these appear to involve /Ø/ roots with a lexicalized
            reflexive object prefix, though their semantics are
            idiosyncratic and no longer transparently reflexive.
            These were erroneously included in paradigms involving
            /H/ vowel-initial roots. Second, long word-final
            syllables are not represented as such. Instead, the
            language community’s somewhat standard orthographic
            conventions are adhered to throughout, and so the
            distinction between, for example \vernacular{
            ala\ob [khúa]\cb } \gloss{‘s/he will pay
            dowry’}(bimoraic) and \vernacular{ala\ob [khwá]\cb 
            {\downstep}tá} \gloss{‘s/he will not pay
            dowry’}(monomoraic) is not represented.
            Third, [b] and [β] are both represented
            orthographically simply as <b>, though the
            distribution of these two allophones are predictable.
            In particular, [b] appears before nasals, while [β]
            appears elsewhere. Additionally, the root
            <lingakanyinya>, meaning \gloss{‘crumple’}, is
            almost always reduced to \vernacular{
            [lingaɲːa]}.

 Finally, forms which combine the verb meaning \gloss{‘grind’}with the
            third person singular object prefix convey a strange
            meaning, whose interpretation would have to be
            metaphorical and may not be immediately understood. It
            occurs to me now that it may have been better to simply
            substitute the third person singular object prefix for
            the Class 14 prefix, which may refer to readily
            grindable things, e.g., \vernacular{βusi} \gloss{‘flour’}and \vernacular{βulé} \gloss{‘millet’}.


\subsection{Near Future: Pattern 1a}\label{sec:sNearFut}


\begin{tabular}{llllll}  
  \multicolumn{5}{l}{
                     \vernacular{(1) /H/
                    C-Initial} \gloss{‘s/he
                    will...’} } &  \\
\multicolumn{5}{l}{ } &  \\

                     \vernacular{
                    ala[rá]}  &   
                     \gloss{‘bury’}  &     &   
                     \vernacular{
                    ala[ng’wá]}  &   
                     \gloss{‘drink’}  &  \\

                     \vernacular{
                    ala[lía]}  &   
                     \gloss{‘eat’}  &     &   
                     \vernacular{
                    ala[khwá]}  &   
                     \gloss{‘pay dowry’}  &  \\

                     \vernacular{
                    ala[lúma]}  &   
                     \gloss{‘bite’}  &     &   
                     \vernacular{
                    ala[béka]}  &   
                     \gloss{‘shave’}  &  \\

                     \vernacular{
                    ala[téekha]}  &   
                     \gloss{‘cook’}  &     &   
                     \vernacular{
                    ala[léera]}  &   
                     \gloss{‘bring’}  &  \\

                     \vernacular{
                    ala[khálaka]}  &   
                     \gloss{‘cut’}  &     &   
                     \vernacular{
                    ala[kálaanga]}  &   
                     \gloss{‘fry’}  &  \\

                     \vernacular{
                    ala[sítaaka]}  &   
                     \gloss{‘accuse’}  &     &   
                     \vernacular{
                    ala[bóolitsa]}  &   
                     \gloss{‘seduce’}  &  \\

                     \vernacular{
                    ala[saatsila]}  &   
                     \gloss{‘betray’}  &     &   
                     \vernacular{
                    ala[sáanditsa]}  &   
                     \gloss{‘thank’}  &  \\

                     \vernacular{
                    ala[khóng’oonda]}  &   
                     \gloss{‘knock’}  &     &   
                     \vernacular{
                    ala[bóholola]}  &   
                     \gloss{‘untie’}  &  \\

                     \vernacular{
                    ala[bóyong’ana]}  &   
                     \gloss{‘go around’}  &     &   
                     \vernacular{
                    ala[ng’óng’oolitsa]}  &   
                     \gloss{‘tease’}  &  \\

                     \vernacular{
                    ala[língakanyinya]}  &   
                     \gloss{‘crumple’}  &     &     &     &  \\
\end{tabular}
%\caption{\nocaption}
     
\begin{tabular}{llllll}  
  \multicolumn{5}{l}{
                     \vernacular{(2) /H/
                    V-Initial} \gloss{‘s/he
                    will...’} } &  \\
\multicolumn{5}{l}{ } &  \\

                     \vernacular{
                    al[iíra]}  &   
                     \gloss{‘kill’}  &     &   
                     \vernacular{
                    al[iíkoomba]}  &   
                     \gloss{‘admire’}  &  \\

                     \vernacular{
                    al[iísiaka]}  &   
                     \gloss{‘smack’}  &     &   
                     \vernacular{
                    al[iíkobola]}  &   
                     \gloss{‘belch’}  &  \\

                     \vernacular{
                    al[oónonyinya]}  &   
                     \gloss{‘spoil’}  &     &   
                     \vernacular{
                    al[aábukhanyinya]}  &   
                     \gloss{‘separate’}  &  \\
\end{tabular}
%\caption{\nocaption}
     
\begin{tabular}{llllll}  
  \multicolumn{5}{l}{
                     \vernacular{(3) /Ø/
                    C-Initial} \gloss{‘s/he
                    will...’} } &  \\
\multicolumn{5}{l}{ } &  \\

                     \vernacular{
                    ala[tsia]}  &   
                     \gloss{‘go’}  &     &   
                     \vernacular{
                    ala[kwa]}  &   
                     \gloss{‘fall’}  &  \\

                     \vernacular{
                    ala[lekha]}  &   
                     \gloss{‘leave’}  &     &   
                     \vernacular{
                    ala[reeba]}  &   
                     \gloss{‘ask’}  &  \\

                     \vernacular{
                    ala[loonda]}  &   
                     \gloss{‘follow’}  &     &   
                     \vernacular{
                    ala[sosana]}  &   
                     \gloss{‘resemble’}  &  \\

                     \vernacular{
                    ala[homoola]}  &   
                     \gloss{‘massage’}  &     &   
                     \vernacular{
                    ala[lakhuula]}  &   
                     \gloss{‘release’}  &  \\

                     \vernacular{
                    ala[seebula]}  &   
                     \gloss{‘say bye’}  &     &   
                     \vernacular{
                    ala[hoombelitsa]}  &   
                     \gloss{‘comfort’}  &  \\

                     \vernacular{
                    ala[kalushitsa]}  &   
                     \gloss{‘return’}  &     &   
                     \vernacular{
                    ala[siinjilitsa]}  &   
                     \gloss{‘make stand’}  &  \\

                     \vernacular{
                    ala[reebareeba]}  &   
                     \gloss{‘ask (iter)’}  &     &   
                     \vernacular{
                    ala[kalukhanyinya]}  &   
                     \gloss{‘turn over’}  &  \\

                     \vernacular{
                    ala[sebulukhanyinya]}  &   
                     \gloss{‘scatter’}  &     &     &     &  \\
\end{tabular}
%\caption{\nocaption}
     
\begin{tabular}{llllll}  
  \multicolumn{5}{l}{
                     \vernacular{(4) /Ø/
                    V-Initial} \gloss{‘s/he
                    will...’} } &  \\
\multicolumn{5}{l}{ } &  \\

                     \vernacular{
                    al[eenya]}  &   
                     \gloss{‘want’}  &     &   
                     \vernacular{
                    al[eeyela]}  &   
                     \gloss{‘wipe for’}  &  \\

                     \vernacular{
                    al[iiluula]}  &   
                     \gloss{‘winnow’}  &     &   
                     \vernacular{
                    al[aambakhana]}  &   
                     \gloss{‘refuse’}  &  \\

                     \vernacular{
                    al[eeleelitsa]}  &   
                     \gloss{‘hang up’}  &     &     &     &  \\
\end{tabular}
%\caption{\nocaption}
     
\begin{tabular}{llllll}  
  \multicolumn{5}{l}{
                     \vernacular{(5) /H/
                    C-Initial + OP} \gloss{‘s/he
                    will...him/her’} } &  \\
\multicolumn{5}{l}{ } &  \\

                     \vernacular{
                    alamú[ra]}  &   
                     \gloss{‘bury’}  &     &   
                     \vernacular{
                    alamú[khwa]}  &   
                     \gloss{‘pay dowry’}  &  \\

                     \vernacular{
                    alamú[beka]}  &   
                     \gloss{‘shave’}  &     &   
                     \vernacular{
                    alamú[leera]}  &   
                     \gloss{‘bring’}  &  \\

                     \vernacular{
                    alamú[khalaka]}  &   
                     \gloss{‘cut’}  &     &   
                     \vernacular{
                    alamú[sitaaka]}  &   
                     \gloss{‘accuse’}  &  \\

                     \vernacular{
                    alamú[boolitsa]}  &   
                     \gloss{‘seduce’}  &     &   
                     \vernacular{
                    alamú[khong’oonda]}  &   
                     \gloss{‘knock’}  &  \\

                     \vernacular{
                    alamú[tsuunzuuna]}  &   
                     \gloss{‘suck’}  &     &   
                     \vernacular{
                    alamú[boholola]}  &   
                     \gloss{‘untie’}  &  \\

                     \vernacular{
                    alamú[boyong’ana]}  &   
                     \gloss{‘go around’}  &     &   
                     \vernacular{
                    alamú[ng’ong’oolitsa]}  &   
                     \gloss{‘tease’}  &  \\

                     \vernacular{
                    alamú[lingakanyinya]}  &   
                     \gloss{‘bend’}  &     &     &     &  \\
\end{tabular}
%\caption{\nocaption}
     
\begin{tabular}{llllll}  
  \multicolumn{5}{l}{
                     \vernacular{(6) /H/
                    V-Initial + OP} \gloss{‘s/he
                    will...him/her’} } &  \\
\multicolumn{5}{l}{ } &  \\

                     \vernacular{
                    alamw[íira]}  &   
                     \gloss{‘kill’}  &     &   
                     \vernacular{
                    alamw[íikoomba]}  &   
                     \gloss{‘admire’}  &  \\

                     \vernacular{
                    alamw[íisiaka]}  &   
                     \gloss{‘smack’}  &     &   
                     \vernacular{
                    alamw[óononyinya]}  &   
                     \gloss{‘spoil’}  &  \\

                     \vernacular{
                    alamw[áabukhanyinya]}  &   
                     \gloss{‘separate’}  &     &     &     &  \\
\end{tabular}
%\caption{\nocaption}
     
\begin{tabular}{llllll}  
  \multicolumn{5}{l}{
                     \vernacular{(7) /Ø/
                    C-Initial + OP} \gloss{‘s/he
                    will...him/her \ob mu-\cb  / them \ob ba-\cb ’} } &  \\
\multicolumn{5}{l}{ } &  \\

                     \vernacular{
                    alamú[tsia]}  &   
                     \gloss{‘go for’}  &     &   
                     \vernacular{
                    alamú[lekha]}  &   
                     \gloss{‘leave’}  &  \\

                     \vernacular{
                    alamú[loonda]}  &   
                     \gloss{‘follow’}  &     &   
                     \vernacular{
                    alamú[chimila]}  &   
                     \gloss{‘hold’}  &  \\

                     \vernacular{
                    alamú[lakhuula]}  &   
                     \gloss{‘release’}  &     &   
                     \vernacular{
                    alamú[seebula]}  &   
                     \gloss{‘say bye to’}  &  \\

                     \vernacular{
                    alamú[hoombelitsa]}  &   
                     \gloss{‘comfort’}  &     &   
                     \vernacular{
                    alamú[kalushitsa]}  &   
                     \gloss{‘return’}  &  \\

                     \vernacular{
                    alamú[siinjilitsa]}  &   
                     \gloss{
                    ‘make...stand’}  &     &   
                     \vernacular{
                    alamú[reebareeba]}  &   
                     \gloss{‘ask (iter)’}  &  \\

                     \vernacular{
                    alamú[kalukhanyinya]}  &   
                     \gloss{
                    ‘turn...over’}  &     &   
                     \vernacular{
                    alabá[sebulukhanyinya]}  &   
                     \gloss{‘scatter’}  &  \\
\end{tabular}
%\caption{\nocaption}
     
\begin{tabular}{llllll}  
  \multicolumn{5}{l}{
                     \vernacular{(8) /Ø/
                    V-Initial + OP} \gloss{‘s/he
                    will...him/her \ob mw-\cb  / it
                    } } &  \\
\multicolumn{5}{l}{ } &  \\

                     \vernacular{
                    alamw[éenya]}  &   
                     \gloss{‘want’}  &     &   
                     \vernacular{
                    alamw[éeyela]}  &   
                     \gloss{‘wipe for’}  &  \\

                     \vernacular{
                    alabw[íiluula]}  &   
                     \gloss{‘winnow’}  &     &   
                     \vernacular{
                    alamw[áambakhana]}  &   
                     \gloss{‘refuse’}  &  \\

                     \vernacular{
                    alamw[éeleelitsa]}  &   
                     \gloss{‘hang...up’}  &  \\
\end{tabular}
%\caption{\nocaption}
     
\begin{tabular}{llllll}  
  \multicolumn{5}{l}{
                     \vernacular{(9) /H/
                    C-Initial + OP
                    } \gloss{‘s/he
                    will...me’} } &  \\
\multicolumn{5}{l}{ } &  \\

                     \vernacular{
                    alaá[ria]}  &   
                     \gloss{‘fear’}  &     &   
                     \vernacular{
                    alaá[khwa]}  &   
                     \gloss{‘pay dowry’}  &  \\

                     \vernacular{
                    alaá[mbeka]}  &   
                     \gloss{‘shave’}  &     &   
                     \vernacular{
                    alaá[ndeera]}  &   
                     \gloss{‘bring’}  &  \\

                     \vernacular{
                    alaá[khalaka]}  &   
                     \gloss{‘cut’}  &     &   
                     \vernacular{
                    alaá[sitaaka]}  &   
                     \gloss{‘accuse’}  &  \\

                     \vernacular{
                    alaá[mboolitsa]}  &   
                     \gloss{‘seduce’}  &     &   
                     \vernacular{
                    alaá[khong’oonda]}  &   
                     \gloss{‘knock’}  &  \\

                     \vernacular{
                    alaá[ndzuunzuuna]}  &   
                     \gloss{‘suck’}  &     &   
                     \vernacular{
                    alaá[mboholola]}  &   
                     \gloss{‘untie’}  &  \\

                     \vernacular{
                    alaá[mboyong’ana]}  &   
                     \gloss{‘go around’}  &     &   
                     \vernacular{
                    alaá[ng’ong’oolitsa]}  &   
                     \gloss{‘tease’}  &  \\

                     \vernacular{
                    alaá[ningakanyinya]}  &   
                     \gloss{‘bend’}  &     &     &     &  \\
\end{tabular}
%\caption{\nocaption}
     
\begin{tabular}{llllll}  
  \multicolumn{5}{l}{
                     \vernacular{(10) /H/
                    V-Initial + OP
                    } \gloss{‘s/he
                    will...me’} } &  \\
\multicolumn{5}{l}{ } &  \\

                     \vernacular{
                    alaá[nzira]}  &   
                     \gloss{‘kill’}  &     &   
                     \vernacular{
                    alaá[nzikoomba]}  &   
                     \gloss{‘admire’}  &  \\

                     \vernacular{
                    alaá[nzisiaka]}  &   
                     \gloss{‘smack’}  &     &   
                     \vernacular{
                    alaá[nzononyinya]}  &   
                     \gloss{‘spoil’}  &  \\

                     \vernacular{
                    alaá[nzabukhanyinya]}  &   
                     \gloss{‘separate’}  &     &     &     &  \\
\end{tabular}
%\caption{\nocaption}
     
\begin{tabular}{llllll}  
  \multicolumn{5}{l}{
                     \vernacular{(11) /Ø/
                    C-Initial + OP
                    } \gloss{‘s/he
                    will...me’} } &  \\
\multicolumn{5}{l}{ } &  \\

                     \vernacular{
                    alaá[ndekha]}  &   
                     \gloss{‘leave’}  &     &   
                     \vernacular{
                    alaá[noonda]}  &   
                     \gloss{‘follow’}  &  \\

                     \vernacular{
                    alaá[njimila]}  &   
                     \gloss{‘hold’}  &     &   
                     \vernacular{
                    alaá[ndakhuula]}  &   
                     \gloss{‘release’}  &  \\

                     \vernacular{
                    alaá[seebula]}  &   
                     \gloss{‘say bye to’}  &     &   
                     \vernacular{
                    alaá[mboombelitsa]}  &   
                     \gloss{‘comfort’}  &  \\

                     \vernacular{
                    alaá[siinjilitsa]}  &   
                     \gloss{
                    ‘make...stand’}  &     &   
                     \vernacular{
                    alaá[ndeebandeeba]}  &   
                     \gloss{‘ask (iter)’}  &  \\

                     \vernacular{
                    alaá[ngalukhanyinya]}  &   
                     \gloss{
                    ‘turn...over’}  &  \\
\end{tabular}
%\caption{\nocaption}
     
\begin{tabular}{llllll}  
  \multicolumn{5}{l}{
                     \vernacular{(12) /Ø/
                    V-Initial + OP
                    } \gloss{‘s/he
                    will...me’} } &  \\
\multicolumn{5}{l}{ } &  \\

                     \vernacular{
                    alaá[nzenya]}  &   
                     \gloss{‘want’}  &     &   
                     \vernacular{
                    alaá[nzeyela]}  &   
                     \gloss{‘wipe for’}  &  \\

                     \vernacular{
                    alaá[nyambakhana]}  &   
                     \gloss{‘refuse’}  &     &   
                     \vernacular{
                    alaá[nzeleelitsa]}  &   
                     \gloss{
                    ‘carry...hanging’}  &  \\
\end{tabular}
%\caption{\nocaption}
     
\begin{tabular}{llllll}  
  \multicolumn{5}{l}{
                     \vernacular{(13) /H/
                    C-Initial + OP
                    } \gloss{‘s/he
                    will...him/herself’} } &  \\
\multicolumn{5}{l}{ } &  \\

                     \vernacular{
                    alií[ra]}  &   
                     \gloss{‘bury’}  &     &   
                     \vernacular{
                    alií[khwa]}  &   
                     \gloss{‘pay dowry’}  &  \\

                     \vernacular{
                    alií[beka]}  &   
                     \gloss{‘shave’}  &     &   
                     \vernacular{
                    alií[suunga]}  &   
                     \gloss{‘hang’}  &  \\

                     \vernacular{
                    alií[khalaka]}  &   
                     \gloss{‘cut’}  &     &   
                     \vernacular{
                    alií[sitaaka]}  &   
                     \gloss{‘accuse’}  &  \\

                     \vernacular{
                    alií[saanditsa]}  &   
                     \gloss{‘thank’}  &     &   
                     \vernacular{
                    alií[khong’oonda]}  &   
                     \gloss{‘knock’}  &  \\

                     \vernacular{
                    alií[boholola]}  &   
                     \gloss{‘untie’}  &     &     &     &  \\
\end{tabular}
%\caption{\nocaption}
     
\begin{tabular}{llllll}  
  \multicolumn{5}{l}{
                     \vernacular{(14) /H/
                    V-Initial + OP
                    } \gloss{‘s/he
                    will...him/herself’} } &  \\
\multicolumn{5}{l}{ } &  \\

                     \vernacular{
                    alií[yira]}  &   
                     \gloss{‘kill’}  &     &   
                     \vernacular{
                    alií[yikoomba]}  &   
                     \gloss{‘admire’}  &  \\

                     \vernacular{
                    alií[yisiaka]}  &   
                     \gloss{‘smack’}  &     &   
                     \vernacular{
                    alií[yononyinya]}  &   
                     \gloss{‘spoil’}  &  \\

                     \vernacular{
                    alií[yabukhanyinya]}  &   
                     \gloss{‘separate’}  &     &     &     &  \\
\end{tabular}
%\caption{\nocaption}
     
\begin{tabular}{llllll}  
  \multicolumn{5}{l}{
                     \vernacular{(15) /Ø/
                    C-Initial + OP
                    } \gloss{‘s/he
                    will...him/herself’} } &  \\
\multicolumn{5}{l}{ } &  \\

                     \vernacular{
                    alií[kama]}  &   
                     \gloss{‘shelter’}  &     &   
                     \vernacular{
                    alií[siinga]}  &   
                     \gloss{‘bathe’}  &  \\

                     \vernacular{
                    alií[kulikha]}  &   
                     \gloss{‘name’}  &     &   
                     \vernacular{
                    alií[naabula]}  &   
                     \gloss{‘undress’}  &  \\

                     \vernacular{
                    alií[lakhuula]}  &   
                     \gloss{‘release’}  &     &   
                     \vernacular{
                    alií[hoombelitsa]}  &   
                     \gloss{‘comfort’}  &  \\

                     \vernacular{
                    alií[siinjilitsa]}  &   
                     \gloss{
                    ‘make...stand’}  &     &   
                     \vernacular{
                    alií[reebareeba]}  &   
                     \gloss{‘ask (iter)’}  &  \\

                     \vernacular{
                    alií[kalukhanyinya]}  &   
                     \gloss{
                    ‘turn...over’}  &  \\
\end{tabular}
%\caption{\nocaption}
     
\begin{tabular}{llllll}  
  \multicolumn{5}{l}{
                     \vernacular{(16) /Ø/
                    V-Initial + OP
                    } \gloss{‘s/he
                    will...him/herself’} } &  \\
\multicolumn{5}{l}{ } &  \\

                     \vernacular{
                    alií[yenya]}  &   
                     \gloss{‘want’}  &     &   
                     \vernacular{
                    alií[yeyela]}  &   
                     \gloss{‘wipe for’}  &  \\

                     \vernacular{
                    alií[yambakhana]}  &   
                     \gloss{‘refuse’}  &     &   
                     \vernacular{
                    alií[yeleelitsa]}  &   
                     \gloss{
                    ‘carry...hanging’}  &  \\
\end{tabular}
%\caption{\nocaption}
     
\begin{tabular}{llllll}  
  \multicolumn{5}{l}{
                     \vernacular{(17) /H/
                    C-Initial + OP + OP
                    } \gloss{‘s/he
                    will...him/her for me’} } &  \\
\multicolumn{5}{l}{ } &  \\

                     \vernacular{
                    alamúu[ndeela]}  &   
                     \gloss{‘bury’}  &     &   
                     \vernacular{
                    alamúu[mbechela]}  &   
                     \gloss{‘shave’}  &  \\

                     \vernacular{
                    alamúu[ndeerela]}  &   
                     \gloss{‘bring’}  &     &   
                     \vernacular{
                    alamúu[khalachila]}  &   
                     \gloss{‘cut’}  &  \\

                     \vernacular{
                    alamúu[sitaachila]}  &   
                     \gloss{‘accuse’}  &     &   
                     \vernacular{
                    alamúu[mboolitsila]}  &   
                     \gloss{‘seduce’}  &  \\

                     \vernacular{
                    alamúu[mbohololela]}  &   
                     \gloss{‘untie’}  &     &     &     &  \\
\end{tabular}
%\caption{\nocaption}
     
\begin{tabular}{llllll}  
  \multicolumn{5}{l}{
                     \vernacular{(18) /H/
                    V-Initial + OP + OP
                    } \gloss{‘s/he
                    will...him/her for me’} } &  \\
\multicolumn{5}{l}{ } &  \\

                     \vernacular{
                    alamúu[nzirila]}  &   
                     \gloss{‘kill’}  &     &   
                     \vernacular{
                    alamúu[nzechitsila]}  &   
                     \gloss{‘admire’}  &  \\

                     \vernacular{
                    alamúu[nzisiachila]}  &   
                     \gloss{‘smack’}  &     &   
                     \vernacular{
                    alamúu[nzononyinyila]}  &   
                     \gloss{‘spoil’}  &  \\

                     \vernacular{
                    alamúu[nzabukhanyinyila]}  &   
                     \gloss{‘separate’}  &     &     &     &  \\
\end{tabular}
%\caption{\nocaption}
     
\begin{tabular}{llllll}  
  \multicolumn{5}{l}{
                     \vernacular{(19) /Ø/
                    C-Initial + OP + OP
                    } \gloss{‘s/he
                    will...him/her for me’} } &  \\
\multicolumn{5}{l}{ } &  \\

                     \vernacular{
                    alamúu[nziila]}  &   
                     \gloss{‘go for’}  &     &   
                     \vernacular{
                    alamúu[ndeshela]}  &   
                     \gloss{‘leave’}  &  \\

                     \vernacular{
                    alamúu[noondela]}  &   
                     \gloss{‘follow’}  &     &   
                     \vernacular{
                    alamúu[ngulishila]}  &   
                     \gloss{‘name’}  &  \\

                     \vernacular{
                    alamúu[ndakhuulila]}  &   
                     \gloss{‘release’}  &     &   
                     \vernacular{
                    alamúu[seebulila]}  &   
                     \gloss{‘say bye to’}  &  \\

                     \vernacular{
                    alamúu[mboombelitsila]}  &   
                     \gloss{‘comfort’}  &     &   
                     \vernacular{
                    alamúu[siinjilitsila]}  &   
                     \gloss{
                    ‘make...stand’}  &  \\
\end{tabular}
%\caption{\nocaption}
     
\begin{tabular}{llllll}  
  \multicolumn{5}{l}{
                     \vernacular{(20) /Ø/
                    V-Initial + OP + OP
                    } \gloss{‘s/he
                    will...him/her \ob mu-\cb  / it
                    } } &  \\
\multicolumn{5}{l}{ } &  \\

                     \vernacular{
                    alamúu[nzeyela]}  &   
                     \gloss{‘wipe’}  &     &   
                     \vernacular{
                    alakúu[nzashitsila]}  &   
                     \gloss{‘light for’}  &  \\

                     \vernacular{
                    alabúu[nziluulila]}  &   
                     \gloss{‘refuse’}  &     &   
                     \vernacular{
                    alalúu[nzitsulitsila]}  &   
                     \gloss{‘fill’}  &  \\

                     \vernacular{
                    alakúu[nzeleelitsila]}  &   
                     \gloss{‘hang’}  &     &     &     &  \\
\end{tabular}
%\caption{\nocaption}
     
\begin{tabular}{lll}  
  \multicolumn{2}{l}{
                     \vernacular{(21) /H/
                    C-Initial Phrase-Medial} \gloss{‘s/he will...the
                    man \ob musáatsa\cb  /} } &  \\
\multicolumn{2}{l}{
                     \gloss{the boy
                    \ob mú{\downstep}yáyi\cb  / someone \ob muundu\cb ’} } &  \\

                     \vernacular{ala[rá]
                    musáatsa/mú{\downstep}yáyi/muundu}  &   
                     \gloss{‘bury’}  &  \\

                     \vernacular{ala[béka]
                    musáatsa/mú{\downstep}yáyi/muundu}  &   
                     \gloss{‘shave’}  &  \\

                     \vernacular{ala[léera]
                    musáatsa/mú{\downstep}yáyi/muundu}  &   
                     \gloss{‘bring’}  &  \\

                     \vernacular{ala[khálaka]
                    musáatsa/mú{\downstep}yáyi/muundu}  &   
                     \gloss{‘cut’}  &  \\

                     \vernacular{ala[sítaaka]
                    musáatsa/mú{\downstep}yáyi/muundu}  &   
                     \gloss{‘accuse’}  &  \\

                     \vernacular{ala[bóolitsa]
                    musáatsa/mú{\downstep}yáyi/muundu}  &   
                     \gloss{‘seduce’}  &  \\

                     \vernacular{ala[khóng’oonda]
                    musáatsa/mú{\downstep}yáyi/muundu}  &   
                     \gloss{‘knock’}  &  \\

                     \vernacular{ala[bóholola]
                    musáatsa/mú{\downstep}yáyi/muundu}  &   
                     \gloss{‘untie’}  &  \\

                     \vernacular{ala[búkaanila]
                    musáatsa/mú{\downstep}yáyi/muundu}  &   
                     \gloss{‘meet’}  &  \\

                     \vernacular{
                    ala[ng’óng’oolitsa]
                    musáatsa/mú{\downstep}yáyi/muundu}  &   
                     \gloss{‘tease’}  &  \\
\end{tabular}
%\caption{\nocaption}
     
\begin{tabular}{lll}  
  \multicolumn{2}{l}{
                     \vernacular{(22) /Ø/
                    C-Initial Phrase-Medial} \gloss{‘s/he will...the
                    man \ob musáatsa\cb  /} } &  \\
\multicolumn{2}{l}{
                     \gloss{the boy
                    \ob mú{\downstep}yáyi\cb  / someone \ob muundu\cb ’} } &  \\

                     \vernacular{ala[tsia]
                    musáatsa/mú{\downstep}yáyi/muundu}  &   
                     \gloss{‘go for’}  &  \\

                     \vernacular{ala[lekha]
                    musáatsa/mú{\downstep}yáyi/muundu}  &   
                     \gloss{‘leave’}  &  \\

                     \vernacular{ala[loonda]
                    musáatsa/mú{\downstep}yáyi/muundu}  &   
                     \gloss{‘follow’}  &  \\

                     \vernacular{ala[kulikha]
                    musáatsa/mú{\downstep}yáyi/muundu}  &   
                     \gloss{‘name’}  &  \\

                     \vernacular{ala[lakhuula]
                    musáatsa/mú{\downstep}yáyi/muundu}  &   
                     \gloss{‘release’}  &  \\

                     \vernacular{ala[seebula]
                    musáatsa/mú{\downstep}yáyi/muundu}  &   
                     \gloss{‘say bye to’}  &  \\

                     \vernacular{ala[hoombelitsa]
                    musáatsa/mú{\downstep}yáyi/muundu}  &   
                     \gloss{‘comfort’}  &  \\

                     \vernacular{ala[kalushitsa]
                    musáatsa/mú{\downstep}yáyi/muundu}  &   
                     \gloss{‘return’}  &  \\

                     \vernacular{ala[siinjilitsa]
                    musáatsa/mú{\downstep}yáyi/muundu}  &   
                     \gloss{
                    ‘make...stand’}  &  \\

                     \vernacular{ala[reebareeba]
                    musáatsa/mú{\downstep}yáyi/muundu}  &   
                     \gloss{‘ask (iter)’}  &  \\

                     \vernacular{ala[kalukhanyinya]
                    musáatsa/mú{\downstep}yáyi/muundu}  &   
                     \gloss{
                    ‘turn...over’}  &  \\
\end{tabular}
%\caption{\nocaption}
     
\begin{tabular}{lll}  
  \multicolumn{2}{l}{
                     \vernacular{(23) /H/
                    C-Initial +OP Phrase-Medial} \gloss{‘s/he will...the
                    man \ob musáatsa\cb  /} } &  \\
\multicolumn{2}{l}{
                     \gloss{the boy
                    \ob mú{\downstep}yáyi\cb  / someone \ob muundu\cb  for
                    him/her’} } &  \\

                     \vernacular{alamú[reela]
                    musáatsa/mú{\downstep}yáyi/muundu}  &   
                     \gloss{‘bury’}  &  \\

                     \vernacular{alamú[bechela]
                    musáatsa/mú{\downstep}yáyi/muundu}  &   
                     \gloss{‘shave’}  &  \\

                     \vernacular{alamú[leerela]
                    musáatsa/mú{\downstep}yáyi/muundu}  &   
                     \gloss{‘bring’}  &  \\

                     \vernacular{alamú[khalachila]
                    musáatsa/mú{\downstep}yáyi/muundu}  &   
                     \gloss{‘cut’}  &  \\

                     \vernacular{alamú[sitaachila]
                    musáatsa/mú{\downstep}yáyi/muundu}  &   
                     \gloss{‘accuse’}  &  \\

                     \vernacular{alamú[boolitsila]
                    musáatsa/mú{\downstep}yáyi/muundu}  &   
                     \gloss{‘seduce’}  &  \\

                     \vernacular{
                    alamú[khong’oondela]
                    musáatsa/mú{\downstep}yáyi/muundu}  &   
                     \gloss{‘knock’}  &  \\

                     \vernacular{alamú[bohololela]
                    musáatsa/mú{\downstep}yáyi/muundu}  &   
                     \gloss{‘untie’}  &  \\

                     \vernacular{alamú[bukaanila]
                    musáatsa/mú{\downstep}yáyi/muundu}  &   
                     \gloss{‘meet’}  &  \\

                     \vernacular{
                    alamú[ng’ong’oolitsila]
                    musáatsa/mú{\downstep}yáyi/muundu}  &   
                     \gloss{‘tease’}  &  \\

                     \vernacular{
                    alamú[lingakanyinyila]
                    musáatsa/mú{\downstep}yáyi/muundu}  &   
                     \gloss{‘bend’}  &  \\
\end{tabular}
%\caption{\nocaption}
     
\begin{tabular}{lll}  
  \multicolumn{2}{l}{
                     \vernacular{(24) /Ø/
                    C-Initial +OP Phrase-Medial} \gloss{‘s/he will...the
                    man \ob musáatsa\cb  /} } &  \\
\multicolumn{2}{l}{
                     \gloss{the boy
                    \ob mú{\downstep}yáyi\cb  / someone \ob muundu\cb  for
                    him/her’} } &  \\

                     \vernacular{alamú[tsiila]
                    musáatsa/mú{\downstep}yáyi/muundu}  &   
                     \gloss{‘go for’}  &  \\

                     \vernacular{alamú[leshela]
                    musáatsa/mú{\downstep}yáyi/muundu}  &   
                     \gloss{‘leave’}  &  \\

                     \vernacular{alamú[loondela]
                    musáatsa/mú{\downstep}yáyi/muundu}  &   
                     \gloss{‘follow’}  &  \\

                     \vernacular{alamú[kulishila]
                    musáatsa/mú{\downstep}yáyi/muundu}  &   
                     \gloss{‘name’}  &  \\

                     \vernacular{alamú[lakhuulila]
                    musáatsa/mú{\downstep}yáyi/muundu}  &   
                     \gloss{‘release’}  &  \\

                     \vernacular{alamú[seebulila]
                    musáatsa/mú{\downstep}yáyi/muundu}  &   
                     \gloss{‘say bye to’}  &  \\

                     \vernacular{
                    alamú[hoombelitsila]
                    musáatsa/mú{\downstep}yáyi/muundu}  &   
                     \gloss{‘comfort’}  &  \\

                     \vernacular{
                    alamú[kalushitsila]
                    musáatsa/mú{\downstep}yáyi/muundu}  &   
                     \gloss{‘return’}  &  \\

                     \vernacular{
                    alamú[siinjilitsila]
                    musáatsa/mú{\downstep}yáyi/muundu}  &   
                     \gloss{
                    ‘make...stand’}  &  \\

                     \vernacular{
                    alamú[reebareebela]
                    musáatsa/mú{\downstep}yáyi/muundu}  &   
                     \gloss{‘ask (iter)’}  &  \\

                     \vernacular{
                    alamú[kalukhanyinyila]
                    musáatsa/mú{\downstep}yáyi/muundu}  &   
                     \gloss{
                    ‘turn...over’}  &  \\
\end{tabular}
%\caption{\nocaption}
     
\begin{tabular}{lll}  
  \multicolumn{2}{l}{
                     \vernacular{(25) /H/
                    C-Initial +OP + OP
                    } \gloss{‘s/he will...the
                    man \ob musáatsa\cb  /} } &  \\
\multicolumn{2}{l}{
                     \gloss{the boy
                    \ob mú{\downstep}yáyi\cb  / someone \ob muundu\cb  for him/her for
                    me’} } &  \\

                     \vernacular{alamúu[ndeela]
                    musáatsa/mú{\downstep}yáyi/muundu}  &   
                     \gloss{‘bury’}  &  \\

                     \vernacular{alamúu[mbechela]
                    musáatsa/mú{\downstep}yáyi/muundu}  &   
                     \gloss{‘shave’}  &  \\

                     \vernacular{alamúu[ndeerela]
                    musáatsa/mú{\downstep}yáyi/muundu}  &   
                     \gloss{‘bring’}  &  \\

                     \vernacular{
                    alamúu[khalachila]
                    musáatsa/mú{\downstep}yáyi/muundu}  &   
                     \gloss{‘cut’}  &  \\

                     \vernacular{
                    alamúu[sitaachila]
                    musáatsa/mú{\downstep}yáyi/muundu}  &   
                     \gloss{‘accuse’}  &  \\

                     \vernacular{
                    alamúu[mboolitsila]
                    musáatsa/mú{\downstep}yáyi/muundu}  &   
                     \gloss{‘seduce’}  &  \\

                     \vernacular{
                    alamúu[mbohololela]
                    musáatsa/mú{\downstep}yáyi/muundu}  &   
                     \gloss{‘untie’}  &  \\
\end{tabular}
%\caption{\nocaption}
     
\begin{tabular}{lll}  
  \multicolumn{2}{l}{
                     \vernacular{(26) /Ø/
                    C-Initial +OP + OP
                    } \gloss{‘s/he will...the
                    man \ob musáatsa\cb  /} } &  \\
\multicolumn{2}{l}{
                     \gloss{the boy
                    \ob mú{\downstep}yáyi\cb  / someone \ob muundu\cb  for him/her for
                    me’} } &  \\

                     \vernacular{alamúu[nziila]
                    musáatsa/mú{\downstep}yáyi/muundu}  &   
                     \gloss{‘go for’}  &  \\

                     \vernacular{alamúu[ndeshela]
                    musáatsa/mú{\downstep}yáyi/muundu}  &   
                     \gloss{‘leave’}  &  \\

                     \vernacular{alamúu[noondela]
                    musáatsa/mú{\downstep}yáyi/muundu}  &   
                     \gloss{‘follow’}  &  \\

                     \vernacular{
                    alamúu[ngulishila]
                    musáatsa/mú{\downstep}yáyi/muundu}  &   
                     \gloss{‘name’}  &  \\

                     \vernacular{
                    alamúu[ndakhuulila]
                    musáatsa/mú{\downstep}yáyi/muundu}  &   
                     \gloss{‘release’}  &  \\

                     \vernacular{alamúu[seebulila]
                    musáatsa/mú{\downstep}yáyi/muundu}  &   
                     \gloss{‘say bye to’}  &  \\

                     \vernacular{
                    alamúu[mboombelitsila]
                    musáatsa/mú{\downstep}yáyi/muundu}  &   
                     \gloss{‘comfort’}  &  \\

                     \vernacular{
                    alamúu[siinjilitsila]
                    musáatsa/mú{\downstep}yáyi/muundu}  &   
                     \gloss{
                    ‘make...stand’}  &  \\
\end{tabular}
%\caption{\nocaption}
    

\subsection{Near Future Negative: Pattern 1a}\label{sec:sNearFutNeg}


\begin{tabular}{llllll}  
  \multicolumn{5}{l}{
                     \vernacular{(27) /H/
                    C-Initial} \gloss{‘s/he will
                    not...’} } &  \\
\multicolumn{5}{l}{ } &  \\

                     \vernacular{ala[rá]
                    {\downstep}tá}  &   
                     \gloss{‘bury’}  &     &   
                     \vernacular{ala[ng’wá]
                    tá}  &   
                     \gloss{‘drink’}  &  \\

                     \vernacular{ala[líá]
                    {\downstep}tá}  &   
                     \gloss{‘eat’}  &     &   
                     \vernacular{ala[lú{\downstep}má]
                    tá}  &   
                     \gloss{‘bite’}  &  \\

                     \vernacular{ala[bé{\downstep}ká]
                    tá}  &   
                     \gloss{‘shave’}  &     &   
                     \vernacular{ala[té{\downstep}ékhá]
                    tá}  &   
                     \gloss{‘cook’}  &  \\

                     \vernacular{ala[lé{\downstep}érá]
                    tá}  &   
                     \gloss{‘bring’}  &     &   
                     \vernacular{ala[khá{\downstep}láká]
                    tá}  &   
                     \gloss{‘cut’}  &  \\

                     \vernacular{ala[ká{\downstep}láángá]
                    tá}  &   
                     \gloss{‘fry’}  &     &   
                     \vernacular{ala[sí{\downstep}tááká]
                    tá}  &   
                     \gloss{‘accuse’}  &  \\

                     \vernacular{ala[bó{\downstep}ólítsá]
                    tá}  &   
                     \gloss{‘seduce’}  &     &   
                     \vernacular{
                    ala[sá{\downstep}ándítsá] tá}  &   
                     \gloss{‘thank’}  &  \\

                     \vernacular{
                    ala[tsú{\downstep}únzúúná] tá}  &   
                     \gloss{‘suck’}  &     &   
                     \vernacular{ala[bó{\downstep}hólólá]
                    tá}  &   
                     \gloss{‘untie’}  &  \\

                     \vernacular{
                    ala[bó{\downstep}yóng’áná] tá}  &   
                     \gloss{‘go around’}  &     &   
                     \vernacular{
                    ala[ng’ó{\downstep}ng’óólítsá] tá}  &   
                     \gloss{‘tease’}  &  \\

                     \vernacular{
                    ala[lí{\downstep}ngákányínyá] tá}  &   
                     \gloss{‘crumple’}  &     &     &     &  \\
\end{tabular}
%\caption{\nocaption}
     
\begin{tabular}{llllll}  
  \multicolumn{5}{l}{
                     \vernacular{(28) /H/
                    V-Initial} \gloss{‘s/he will
                    not...’} } &  \\
\multicolumn{5}{l}{ } &  \\

                     \vernacular{al[ií{\downstep}rá]
                    tá}  &   
                     \gloss{‘kill’}  &     &   
                     \vernacular{al[ií{\downstep}kóómbá]
                    tá}  &   
                     \gloss{‘admire’}  &  \\

                     \vernacular{al[ií{\downstep}síáká]
                    tá}  &   
                     \gloss{‘smack’}  &     &   
                     \vernacular{al[ií{\downstep}kóbólá]
                    tá}  &   
                     \gloss{‘belch’}  &  \\

                     \vernacular{
                    al[oó{\downstep}nónyínyá] tá}  &   
                     \gloss{‘spoil’}  &     &   
                     \vernacular{
                    al[aá{\downstep}búkhányínyá] tá}  &   
                     \gloss{‘separate’}  &  \\
\end{tabular}
%\caption{\nocaption}
     
\begin{tabular}{llllll}  
  \multicolumn{5}{l}{
                     \vernacular{(29) /Ø/
                    C-Initial} \gloss{‘s/he will
                    not...’} } &  \\
\multicolumn{5}{l}{ } &  \\

                     \vernacular{ala[tsíá]
                    tá}  &   
                     \gloss{‘go’}  &     &   
                     \vernacular{ala[kwá]
                    tá}  &   
                     \gloss{‘fall’}  &  \\

                     \vernacular{ala[lékhá]
                    tá}  &   
                     \gloss{‘leave’}  &     &   
                     \vernacular{ala[réébá]
                    tá}  &   
                     \gloss{‘ask’}  &  \\

                     \vernacular{ala[lóóndá]
                    tá}  &   
                     \gloss{‘follow’}  &     &   
                     \vernacular{ala[sósáná]
                    tá}  &   
                     \gloss{‘resemble’}  &  \\

                     \vernacular{ala[hómóólá]
                    tá}  &   
                     \gloss{‘massage’}  &     &   
                     \vernacular{ala[lákhúúlá]
                    tá}  &   
                     \gloss{‘release’}  &  \\

                     \vernacular{ala[séébúlá]
                    tá}  &   
                     \gloss{‘say bye’}  &     &   
                     \vernacular{
                    ala[hóómbélítsá] tá}  &   
                     \gloss{‘comfort’}  &  \\

                     \vernacular{
                    ala[kálúshítsá] tá}  &   
                     \gloss{‘return’}  &     &   
                     \vernacular{
                    ala[síínjílítsá] tá}  &   
                     \gloss{‘make stand’}  &  \\

                     \vernacular{
                    ala[réébáréébá] tá}  &   
                     \gloss{‘ask (iter)’}  &     &   
                     \vernacular{
                    ala[kálúkhányínyá] tá}  &   
                     \gloss{‘turn over’}  &  \\

                     \vernacular{
                    ala[sébúlúkhányínyá] tá}  &   
                     \gloss{‘scatter’}  &     &     &     &  \\
\end{tabular}
%\caption{\nocaption}
     
\begin{tabular}{llllll}  
  \multicolumn{5}{l}{
                     \vernacular{(30) /Ø/
                    V-Initial} \gloss{‘s/he will
                    not...’} } &  \\
\multicolumn{5}{l}{ } &  \\

                     \vernacular{al[eényá]
                    tá}  &   
                     \gloss{‘want’}  &     &   
                     \vernacular{al[eéyélá]
                    tá}  &   
                     \gloss{‘wipe for’}  &  \\

                     \vernacular{al[iílúúlá]
                    tá}  &   
                     \gloss{‘winnow’}  &     &   
                     \vernacular{al[aámbákháná]
                    tá}  &   
                     \gloss{‘refuse’}  &  \\

                     \vernacular{
                    al[eéléélítsá] tá}  &   
                     \gloss{‘hang up’}  &     &     &     &  \\
\end{tabular}
%\caption{\nocaption}
     
\begin{tabular}{llllll}  
  \multicolumn{5}{l}{
                     \vernacular{(31) /H/
                    C-Initial + OP} \gloss{‘s/he will
                    not...him/her’} } &  \\
\multicolumn{5}{l}{ } &  \\

                     \vernacular{alamú[{\downstep}rá]
                    tá}  &   
                     \gloss{‘bury’}  &     &   
                     \vernacular{alamú[{\downstep}khwá]
                    tá}  &   
                     \gloss{‘pay dowry’}  &  \\

                     \vernacular{alamú[{\downstep}béká]
                    tá}  &   
                     \gloss{‘shave’}  &     &   
                     \vernacular{alamú[{\downstep}léérá]
                    tá}  &   
                     \gloss{‘bring’}  &  \\

                     \vernacular{
                    alamú[{\downstep}kháláká] tá}  &   
                     \gloss{‘cut’}  &     &   
                     \vernacular{
                    alamú[{\downstep}sítááká] tá}  &   
                     \gloss{‘accuse’}  &  \\

                     \vernacular{
                    alamú[{\downstep}bóólítsá] tá}  &   
                     \gloss{‘seduce’}  &     &   
                     \vernacular{
                    alamú[{\downstep}tsúúnzúúná] tá}  &   
                     \gloss{‘suck’}  &  \\

                     \vernacular{
                    alamú[{\downstep}bóhólólá] tá}  &   
                     \gloss{‘untie’}  &     &   
                     \vernacular{
                    alamú[{\downstep}búkáánílá] tá}  &   
                     \gloss{‘meet’}  &  \\

                     \vernacular{
                    alamú[{\downstep}ng’óng’óólítsá] tá}  &   
                     \gloss{‘tease’}  &     &   
                     \vernacular{
                    alamú[{\downstep}língákányínyá] tá}  &   
                     \gloss{‘bend’}  &  \\
\end{tabular}
%\caption{\nocaption}
     
\begin{tabular}{llllll}  
  \multicolumn{5}{l}{
                     \vernacular{(32) /H/
                    V-Initial + OP} \gloss{‘s/he will
                    not...him/her’} } &  \\
\multicolumn{5}{l}{ } &  \\

                     \vernacular{alamw[í{\downstep}írá]
                    tá}  &   
                     \gloss{‘kill’}  &     &   
                     \vernacular{
                    alamw[í{\downstep}íkóómbá] tá}  &   
                     \gloss{‘admire’}  &  \\

                     \vernacular{
                    alamw[í{\downstep}ísíáká] tá}  &   
                     \gloss{‘smack’}  &     &   
                     \vernacular{
                    alamw[ó{\downstep}ónónyínyá] tá}  &   
                     \gloss{‘spoil’}  &  \\

                     \vernacular{
                    alamw[á{\downstep}ábúkhányínyá] tá}  &   
                     \gloss{‘separate’}  &     &     &     &  \\
\end{tabular}
%\caption{\nocaption}
     
\begin{tabular}{llllll}  
  \multicolumn{5}{l}{
                     \vernacular{(33) /Ø/
                    C-Initial + OP} \gloss{‘s/he will
                    not...him/her \ob mu-\cb  / them \ob ba-\cb ’} } &  \\
\multicolumn{5}{l}{ } &  \\

                     \vernacular{alamú[{\downstep}tsíá]
                    tá}  &   
                     \gloss{‘go for’}  &  \\

                     \vernacular{alamú[{\downstep}lékhá]
                    tá}  &   
                     \gloss{‘leave’}  &  \\

                     \vernacular{alamú[{\downstep}lóóndá]
                    tá}  &   
                     \gloss{‘follow’}  &  \\

                     \vernacular{
                    alamú[{\downstep}kúlíkhá] tá}  &   
                     \gloss{‘name’}  &  \\

                     \vernacular{
                    alamú[{\downstep}lákhúúlá] tá}  &   
                     \gloss{‘release’}  &  \\

                     \vernacular{
                    alamú[{\downstep}séébúlá] tá}  &   
                     \gloss{‘say bye to’}  &  \\

                     \vernacular{
                    alamú[{\downstep}hóómbélítsá] tá}  &   
                     \gloss{‘comfort’}  &  \\

                     \vernacular{
                    alamú[{\downstep}kálúshítsá] tá}  &   
                     \gloss{‘return’}  &  \\

                     \vernacular{
                    alamú[{\downstep}síínjílítsá] tá}  &   
                     \gloss{
                    ‘make...stand’}  &  \\

                     \vernacular{
                    alamú[{\downstep}réébáréébá] tá}  &   
                     \gloss{‘ask (iter)’}  &  \\

                     \vernacular{
                    alamú[{\downstep}kálúkhányínyá] tá}  &   
                     \gloss{
                    ‘turn...over’}  &  \\

                     \vernacular{
                    alabá[{\downstep}sébúlúkhányínyá] tá}  &   
                     \gloss{‘scatter’}  &  \\
\end{tabular}
%\caption{\nocaption}
     
\begin{tabular}{llllll}  
  \multicolumn{5}{l}{
                     \vernacular{(34) /Ø/
                    V-Initial + OP} \gloss{‘s/he will
                    not...him/her \ob mw-\cb  / it
                    } } &  \\
\multicolumn{5}{l}{ } &  \\

                     \vernacular{alamw[é{\downstep}ényá]
                    tá}  &   
                     \gloss{‘want’}  &     &   
                     \vernacular{alamw[é{\downstep}éyélá]
                    tá}  &   
                     \gloss{‘wipe for’}  &  \\

                     \vernacular{
                    alabw[í{\downstep}ílúúlá] tá}  &   
                     \gloss{‘winnow’}  &     &   
                     \vernacular{
                    alamw[á{\downstep}ámbákháná] tá}  &   
                     \gloss{‘refuse’}  &  \\

                     \vernacular{
                    alamw[é{\downstep}éléélítsá] tá}  &   
                     \gloss{‘hang...up’}  &  \\
\end{tabular}
%\caption{\nocaption}
     
\begin{tabular}{llllll}  
  \multicolumn{5}{l}{
                     \vernacular{(35) /H/
                    C-Initial + OP
                    } \gloss{‘s/he will
                    not...me’} } &  \\
\multicolumn{5}{l}{ } &  \\

                     \vernacular{alaá[{\downstep}ríá]
                    tá}  &   
                     \gloss{‘fear’}  &     &   
                     \vernacular{alaá[{\downstep}khwá]
                    tá}  &   
                     \gloss{‘pay dowry’}  &  \\

                     \vernacular{alaá[{\downstep}mbéká]
                    tá}  &   
                     \gloss{‘shave’}  &     &   
                     \vernacular{alaá[{\downstep}ndéérá]
                    tá}  &   
                     \gloss{‘bring’}  &  \\

                     \vernacular{alaá[{\downstep}kháláká]
                    tá}  &   
                     \gloss{‘cut’}  &     &   
                     \vernacular{
                    alaá[{\downstep}sítááká] tá}  &   
                     \gloss{‘accuse’}  &  \\

                     \vernacular{
                    alaá[{\downstep}mbóólítsá] tá}  &   
                     \gloss{‘seduce’}  &     &   
                     \vernacular{
                    alaá[{\downstep}tsúúnzúúná] tá}  &   
                     \gloss{‘suck’}  &  \\

                     \vernacular{
                    alaá[{\downstep}mbóhólólá] tá}  &   
                     \gloss{‘untie’}  &     &   
                     \vernacular{
                    alaá[{\downstep}búkáánílá] tá}  &   
                     \gloss{‘meet’}  &  \\

                     \vernacular{
                    alaá[{\downstep}ng’óng’óólítsá] tá}  &   
                     \gloss{‘tease’}  &     &   
                     \vernacular{
                    alaá[{\downstep}níngákányínyá] tá}  &   
                     \gloss{‘bend’}  &  \\
\end{tabular}
%\caption{\nocaption}
     
\begin{tabular}{llllll}  
  \multicolumn{5}{l}{
                     \vernacular{(36) /H/
                    V-Initial + OP
                    } \gloss{‘s/he will
                    not...me’} } &  \\
\multicolumn{5}{l}{ } &  \\

                     \vernacular{alaá{\downstep}[nzírá]
                    tá}  &   
                     \gloss{‘kill’}  &     &   
                     \vernacular{
                    alaá[{\downstep}nzíkóómbá] tá}  &   
                     \gloss{‘admire’}  &  \\

                     \vernacular{
                    alaá[{\downstep}nzísíáká] tá}  &   
                     \gloss{‘smack’}  &     &   
                     \vernacular{
                    alaá[{\downstep}nzónónyínyá] tá}  &   
                     \gloss{‘spoil’}  &  \\

                     \vernacular{
                    alaá[{\downstep}nzábúkhányínyá] tá}  &   
                     \gloss{‘separate’}  &     &     &     &  \\
\end{tabular}
%\caption{\nocaption}
     
\begin{tabular}{llllll}  
  \multicolumn{5}{l}{
                     \vernacular{(37) /Ø/
                    C-Initial + OP
                    } \gloss{‘s/he will
                    not...me’} } &  \\
\multicolumn{5}{l}{ } &  \\

                     \vernacular{alaá[{\downstep}ndékhá]
                    tá}  &   
                     \gloss{‘leave’}  &     &   
                     \vernacular{alaá[{\downstep}nóóndá]
                    tá}  &   
                     \gloss{‘follow’}  &  \\

                     \vernacular{
                    alaá[{\downstep}ngúlíkhá] tá}  &   
                     \gloss{‘name’}  &     &   
                     \vernacular{
                    alaá[{\downstep}ndákhúúlá] tá}  &   
                     \gloss{‘release’}  &  \\

                     \vernacular{
                    alaá[{\downstep}séébúlá] tá}  &   
                     \gloss{‘say bye to’}  &     &   
                     \vernacular{
                    alaá[{\downstep}mbóómbélítsá] tá}  &   
                     \gloss{‘comfort’}  &  \\

                     \vernacular{
                    alaá[{\downstep}síínjílítsá] tá}  &   
                     \gloss{
                    ‘make...stand’}  &     &   
                     \vernacular{
                    alaá[{\downstep}ndéébándéébá] tá}  &   
                     \gloss{‘ask (iter)’}  &  \\

                     \vernacular{
                    alaá[{\downstep}ngálúkhányínyá] tá}  &   
                     \gloss{
                    ‘turn...over’}  &  \\
\end{tabular}
%\caption{\nocaption}
     
\begin{tabular}{llllll}  
  \multicolumn{5}{l}{
                     \vernacular{(38) /Ø/
                    V-Initial + OP
                    } \gloss{‘s/he will
                    not...me’} } &  \\
\multicolumn{5}{l}{ } &  \\

                     \vernacular{alaá[{\downstep}nzényá]
                    tá}  &   
                     \gloss{‘want’}  &     &   
                     \vernacular{alaá[{\downstep}nzéyélá]
                    tá}  &   
                     \gloss{‘wipe for’}  &  \\

                     \vernacular{
                    alaá[{\downstep}nyámbákháná] tá}  &   
                     \gloss{‘refuse’}  &     &   
                     \vernacular{
                    alaá[{\downstep}nzéléélítsá] tá}  &   
                     \gloss{
                    ‘carry...hanging’}  &  \\
\end{tabular}
%\caption{\nocaption}
     
\begin{tabular}{llllll}  
  \multicolumn{5}{l}{
                     \vernacular{(39) /H/
                    C-Initial + OP
                    } \gloss{‘s/he will
                    not...him/herself’} } &  \\
\multicolumn{5}{l}{ } &  \\

                     \vernacular{alií[{\downstep}rá]
                    tá}  &   
                     \gloss{‘bury’}  &     &   
                     \vernacular{alií[{\downstep}khwá]
                    tá}  &   
                     \gloss{‘pay dowry’}  &  \\

                     \vernacular{alií[{\downstep}béká]
                    tá}  &   
                     \gloss{‘shave’}  &     &   
                     \vernacular{alií[{\downstep}súúngá]
                    tá}  &   
                     \gloss{‘hang’}  &  \\

                     \vernacular{alií[{\downstep}kháláká]
                    tá}  &   
                     \gloss{‘cut’}  &     &   
                     \vernacular{
                    alií[{\downstep}sítááká] tá}  &   
                     \gloss{‘accuse’}  &  \\

                     \vernacular{
                    alií[{\downstep}sáándítsá] tá}  &   
                     \gloss{‘thank’}  &     &   
                     \vernacular{
                    alií[{\downstep}khóng’óóndá] tá}  &   
                     \gloss{‘knock’}  &  \\

                     \vernacular{
                    alií[{\downstep}bóhólólá] tá}  &   
                     \gloss{‘untie’}  &     &     &     &  \\
\end{tabular}
%\caption{\nocaption}
     
\begin{tabular}{llllll}  
  \multicolumn{5}{l}{
                     \vernacular{(40) /H/
                    V-Initial + OP
                    } \gloss{‘s/he will
                    not...him/herself’} } &  \\
\multicolumn{5}{l}{ } &  \\

                     \vernacular{alií[{\downstep}yírá]
                    tá}  &   
                     \gloss{‘kill’}  &     &   
                     \vernacular{
                    alií[{\downstep}yíkóómbá] tá}  &   
                     \gloss{‘admire’}  &  \\

                     \vernacular{
                    alií[{\downstep}yísíáká] tá}  &   
                     \gloss{‘smack’}  &     &   
                     \vernacular{
                    alií[{\downstep}yónónyínyá] tá}  &   
                     \gloss{‘spoil’}  &  \\

                     \vernacular{
                    alií[{\downstep}yábúkhányínyá] tá}  &   
                     \gloss{‘separate’}  &     &     &     &  \\
\end{tabular}
%\caption{\nocaption}
     
\begin{tabular}{llllll}  
  \multicolumn{5}{l}{
                     \vernacular{(41) /Ø/
                    C-Initial + OP
                    } \gloss{‘s/he will
                    not...him/herself’} } &  \\
\multicolumn{5}{l}{ } &  \\

                     \vernacular{alií[{\downstep}lékhá]
                    tá}  &   
                     \gloss{‘leave’}  &     &   
                     \vernacular{alií[{\downstep}sííngá]
                    tá}  &   
                     \gloss{‘bathe’}  &  \\

                     \vernacular{alií[{\downstep}kúlíkhá]
                    tá}  &   
                     \gloss{‘name’}  &     &   
                     \vernacular{
                    alií[{\downstep}náábúlá] tá}  &   
                     \gloss{‘undress’}  &  \\

                     \vernacular{
                    alií[{\downstep}lákhúúlá] tá}  &   
                     \gloss{‘release’}  &     &   
                     \vernacular{
                    alií[{\downstep}hóómbélítsá] tá}  &   
                     \gloss{‘comfort’}  &  \\

                     \vernacular{
                    alií[{\downstep}síínjílítsá] tá}  &   
                     \gloss{
                    ‘make...stand’}  &     &   
                     \vernacular{
                    alií[{\downstep}réébáréébá] tá}  &   
                     \gloss{‘ask (iter)’}  &  \\

                     \vernacular{
                    alií[{\downstep}kálúkhányínyá] tá}  &   
                     \gloss{
                    ‘turn...over’}  &  \\
\end{tabular}
%\caption{\nocaption}
     
\begin{tabular}{llllll}  
  \multicolumn{5}{l}{
                     \vernacular{(42) /Ø/
                    V-Initial + OP
                    } \gloss{‘s/he will
                    not...him/herself’} } &  \\
\multicolumn{5}{l}{ } &  \\

                     \vernacular{alií[{\downstep}yényá]
                    tá}  &   
                     \gloss{‘want’}  &     &   
                     \vernacular{alií[{\downstep}yéyélá]
                    tá}  &   
                     \gloss{‘wipe for’}  &  \\

                     \vernacular{
                    alií[{\downstep}yámbákháná] tá}  &   
                     \gloss{‘refuse’}  &     &   
                     \vernacular{
                    alií[{\downstep}yéléélítsá] tá}  &   
                     \gloss{
                    ‘carry...hanging’}  &  \\
\end{tabular}
%\caption{\nocaption}
     
\begin{tabular}{llllll}  
  \multicolumn{5}{l}{
                     \vernacular{(43) /H/
                    C-Initial + OP + OP
                    } \gloss{‘s/he will
                    not...him/her for me’} } &  \\
\multicolumn{5}{l}{ } &  \\

                     \vernacular{
                    alamú{\downstep}ú[ndéélá] tá}  &   
                     \gloss{‘bury’}  &     &   
                     \vernacular{
                    alamú{\downstep}ú[mbéchélá] tá}  &   
                     \gloss{‘shave’}  &  \\

                     \vernacular{
                    alamú{\downstep}ú[ndéérélá] tá}  &   
                     \gloss{‘bring’}  &     &   
                     \vernacular{
                    alamú{\downstep}ú[kháláchílá] tá}  &   
                     \gloss{‘cut’}  &  \\

                     \vernacular{
                    alamú{\downstep}ú[sítááchílá] tá}  &   
                     \gloss{‘accuse’}  &     &   
                     \vernacular{
                    alamú{\downstep}ú[mbóólítsílá] tá}  &   
                     \gloss{‘seduce’}  &  \\

                     \vernacular{
                    alamú{\downstep}ú[mbóhólólélá] tá}  &   
                     \gloss{‘untie’}  &     &     &     &  \\
\end{tabular}
%\caption{\nocaption}
     
\begin{tabular}{llllll}  
  \multicolumn{5}{l}{
                     \vernacular{(44) /H/
                    V-Initial + OP + OP
                    } \gloss{‘s/he will
                    not...him/her for me’} } &  \\
\multicolumn{5}{l}{ } &  \\

                     \vernacular{
                    alamú{\downstep}ú[nzírílá] tá}  &   
                     \gloss{‘kill’}  &  \\

                     \vernacular{
                    alamú{\downstep}ú[nzéchítsílá] tá}  &   
                     \gloss{‘admire’}  &  \\

                     \vernacular{
                    alamú{\downstep}ú[nzísíáchílá] tá}  &   
                     \gloss{‘smack’}  &  \\

                     \vernacular{
                    alamú{\downstep}ú[nzónónyínyílá] tá}  &   
                     \gloss{‘spoil’}  &  \\

                     \vernacular{
                    alamú{\downstep}ú[nzábúkhányínyílá]
                    tá}  &   
                     \gloss{‘separate’}  &  \\
\end{tabular}
%\caption{\nocaption}
     
\begin{tabular}{llllll}  
  \multicolumn{5}{l}{
                     \vernacular{(45) /Ø/
                    C-Initial + OP + OP
                    } \gloss{‘s/he will
                    not...him/her for me’} } &  \\
\multicolumn{5}{l}{ } &  \\

                     \vernacular{
                    alamú{\downstep}ú[nzílílá] tá}  &   
                     \gloss{‘go for’}  &  \\

                     \vernacular{
                    alamú{\downstep}ú[ndéshélá] tá}  &   
                     \gloss{‘leave’}  &  \\

                     \vernacular{
                    alamú{\downstep}ú[nóóndélá] tá}  &   
                     \gloss{‘follow’}  &  \\

                     \vernacular{
                    alamú{\downstep}ú[ngúlíshílá] tá}  &   
                     \gloss{‘name’}  &  \\

                     \vernacular{
                    alamú{\downstep}ú[ndákhúúlílá] tá}  &   
                     \gloss{‘release’}  &  \\

                     \vernacular{
                    alamú{\downstep}ú[séébúlílá] tá}  &   
                     \gloss{‘say bye to’}  &  \\

                     \vernacular{
                    alamú{\downstep}ú[mbóómbélítsílá] tá}  &   
                     \gloss{‘comfort’}  &  \\

                     \vernacular{
                    alamú{\downstep}ú[síínjílítsílá] tá}  &   
                     \gloss{
                    ‘make...stand’}  &  \\
\end{tabular}
%\caption{\nocaption}
     
\begin{tabular}{llllll}  
  \multicolumn{5}{l}{
                     \vernacular{(46) /Ø/
                    V-Initial + OP + OP
                    } \gloss{‘s/he will
                    not...him/her \ob mu-\cb  / it
                    } } &  \\
\multicolumn{5}{l}{ } &  \\

                     \vernacular{
                    alamú{\downstep}ú[nyényélá] tá}  &   
                     \gloss{‘want’}  &     &   
                     \vernacular{
                    alamú{\downstep}ú[nzéyélá] tá}  &   
                     \gloss{‘wipe’}  &  \\

                     \vernacular{
                    alakú{\downstep}ú[nzáshítsílá] tá}  &   
                     \gloss{‘light’}  &     &   
                     \vernacular{
                    alabú{\downstep}ú[nzílúúlílá] tá}  &   
                     \gloss{‘fill’}  &  \\

                     \vernacular{
                    alalú{\downstep}ú[nzítsúlítsílá] tá}  &   
                     \gloss{‘fill’}  &     &   
                     \vernacular{
                    alakú{\downstep}ú[nzéléélítsílá] tá}  &   
                     \gloss{‘hang’}  &  \\
\end{tabular}
%\caption{\nocaption}
     
\begin{tabular}{lll}  
  \multicolumn{2}{l}{
                     \vernacular{(47) /H/
                    C-Initial Phrase-Medial} \gloss{‘s/he will
                    not...the man \ob musáatsa\cb  /} } &  \\
\multicolumn{2}{l}{
                     \gloss{the wife
                    \ob mukháli\cb  / the boy \ob mú{\downstep}yáyi\cb  / someone
                    \ob muundu\cb ’} } &  \\

                     \vernacular{ala[rá]
                    musáatsa/mú{\downstep}yáyi/muundu tá}  &   
                     \gloss{‘bury’}  &  \\

                     \vernacular{ala[khwá]
                    mukháli)/muundu tá}  &   
                     \gloss{‘pay dowry’}  &  \\

                     \vernacular{ala[béka]
                    musáatsa/mú{\downstep}yáyi/muundu tá}  &   
                     \gloss{‘shave’}  &  \\

                     \vernacular{ala[léera]
                    musáatsa/mú{\downstep}yáyi/muundu tá}  &   
                     \gloss{‘bring’}  &  \\

                     \vernacular{ala[khálaka]
                    musáatsa/mú{\downstep}yáyi/muundu tá}  &   
                     \gloss{‘cut’}  &  \\

                     \vernacular{ala[sítaaka]
                    musáatsa/mú{\downstep}yáyi/muundu tá}  &   
                     \gloss{‘accuse’}  &  \\

                     \vernacular{ala[bóolitsa]
                    musáatsa/mú{\downstep}yáyi/muundu tá}  &   
                     \gloss{‘seduce’}  &  \\

                     \vernacular{ala[tsúunzuuna]
                    musáatsa/mú{\downstep}yáyi/muundu tá}  &   
                     \gloss{‘suck’}  &  \\

                     \vernacular{ala[bóholola]
                    musáatsa/mú{\downstep}yáyi/muundu tá}  &   
                     \gloss{‘untie’}  &  \\

                     \vernacular{ala[búkaanila]
                    musáatsa/mú{\downstep}yáyi/muundu tá}  &   
                     \gloss{‘meet’}  &  \\

                     \vernacular{
                    ala[ng’óng’oolitsa] musáatsa/mú{\downstep}yáyi/muundu
                    tá}  &   
                     \gloss{‘tease’}  &  \\
\end{tabular}
%\caption{\nocaption}
     
\begin{tabular}{lll}  
  \multicolumn{2}{l}{
                     \vernacular{(48) /Ø/
                    C-Initial Phrase-Medial} \gloss{‘s/he will
                    not...the man \ob musáatsa\cb  /} } &  \\
\multicolumn{2}{l}{
                     \gloss{the boy
                    \ob mú{\downstep}yáyi\cb  / someone \ob muundu\cb ’} } &  \\

                     \vernacular{ala[tsia]
                    musáatsa/mú{\downstep}yáyi/muundu tá}  &   
                     \gloss{‘go for’}  &  \\

                     \vernacular{ala[lekha]
                    musáatsa/mú{\downstep}yáyi/muundu tá}  &   
                     \gloss{‘leave’}  &  \\

                     \vernacular{ala[loonda]
                    musáatsa/mú{\downstep}yáyi/muundu tá}  &   
                     \gloss{‘follow’}  &  \\

                     \vernacular{ala[kulikha]
                    musáatsa/mú{\downstep}yáyi/muundu tá}  &   
                     \gloss{‘name’}  &  \\

                     \vernacular{ala[lakhuula]
                    musáatsa/mú{\downstep}yáyi/muundu tá}  &   
                     \gloss{‘release’}  &  \\

                     \vernacular{ala[seebula]
                    musáatsa/mú{\downstep}yáyi/muundu tá}  &   
                     \gloss{‘say bye to’}  &  \\

                     \vernacular{ala[hoombelitsa]
                    musáatsa/mú{\downstep}yáyi/muundu tá}  &   
                     \gloss{‘comfort’}  &  \\

                     \vernacular{ala[kalushitsa]
                    musáatsa/mú{\downstep}yáyi/muundu tá}  &   
                     \gloss{‘return’}  &  \\

                     \vernacular{ala[siinjilitsa]
                    musáatsa/mú{\downstep}yáyi/muundu tá}  &   
                     \gloss{
                    ‘make...stand’}  &  \\

                     \vernacular{ala[reebareeba]
                    musáatsa/mú{\downstep}yáyi/muundu tá}  &   
                     \gloss{‘ask (iter)’}  &  \\

                     \vernacular{ala[kalukhanyinya]
                    musáatsa/mú{\downstep}yáyi/muundu tá}  &   
                     \gloss{
                    ‘turn...over’}  &  \\
\end{tabular}
%\caption{\nocaption}
     
\begin{tabular}{lll}  
  \multicolumn{2}{l}{
                     \vernacular{(49) /H/
                    C-Initial +OP Phrase-Medial} \gloss{‘s/he will
                    not...the man \ob musáatsa\cb  /} } &  \\
\multicolumn{2}{l}{
                     \gloss{the boy
                    \ob mú{\downstep}yáyi\cb  / someone \ob muundu\cb  for
                    him/her’} } &  \\

                     \vernacular{alamú[reela]
                    musáatsa/mú{\downstep}yáyi/muundu tá}  &   
                     \gloss{‘bury’}  &  \\

                     \vernacular{alamú[bechela]
                    musáatsa/mú{\downstep}yáyi/muundu tá}  &   
                     \gloss{‘shave’}  &  \\

                     \vernacular{alamú[leerela]
                    musáatsa/mú{\downstep}yáyi/muundu tá}  &   
                     \gloss{‘bring’}  &  \\

                     \vernacular{alamú[khalachila]
                    musáatsa/mú{\downstep}yáyi/muundu tá}  &   
                     \gloss{‘cut’}  &  \\

                     \vernacular{alamú[sitaachila]
                    musáatsa/mú{\downstep}yáyi/muundu tá}  &   
                     \gloss{‘accuse’}  &  \\

                     \vernacular{alamú[boolitsila]
                    musáatsa/mú{\downstep}yáyi/muundu tá}  &   
                     \gloss{‘seduce’}  &  \\

                     \vernacular{
                    alamú[tsuunzuunila] musáatsa/mú{\downstep}yáyi/muundu
                    tá}  &   
                     \gloss{‘suck’}  &  \\

                     \vernacular{alamú[bohololela]
                    musáatsa/mú{\downstep}yáyi/muundu tá}  &   
                     \gloss{‘untie’}  &  \\

                     \vernacular{alamú[bukaanila]
                    musáatsa/mú{\downstep}yáyi/muundu tá}  &   
                     \gloss{‘meet’}  &  \\

                     \vernacular{
                    alamú[ng’ong’oolitsila]
                    musáatsa/mú{\downstep}yáyi/muundu tá}  &   
                     \gloss{‘tease’}  &  \\

                     \vernacular{
                    alamú[lingakanyinyila]
                    musáatsa/mú{\downstep}yáyi/muundu tá}  &   
                     \gloss{‘bend’}  &  \\
\end{tabular}
%\caption{\nocaption}
     
\begin{tabular}{lll}  
  \multicolumn{2}{l}{
                     \vernacular{(50) /Ø/
                    C-Initial +OP Phrase-Medial} \gloss{‘s/he will
                    not...the man \ob musáatsa\cb  /} } &  \\
\multicolumn{2}{l}{
                     \gloss{the boy
                    \ob mú{\downstep}yáyi\cb  / someone \ob muundu\cb  for
                    him/her’} } &  \\

                     \vernacular{alamú[tsiila]
                    musáatsa/mú{\downstep}yáyi/muundu tá}  &   
                     \gloss{‘go for’}  &  \\

                     \vernacular{alamú[leshela]
                    musáatsa/mú{\downstep}yáyi/muundu tá}  &   
                     \gloss{‘leave’}  &  \\

                     \vernacular{alamú[loondela]
                    musáatsa/mú{\downstep}yáyi/muundu tá}  &   
                     \gloss{‘follow’}  &  \\

                     \vernacular{alamú[kulishila]
                    musáatsa/mú{\downstep}yáyi/muundu tá}  &   
                     \gloss{‘name’}  &  \\

                     \vernacular{alamú[lakhuulila]
                    musáatsa/mú{\downstep}yáyi/muundu tá}  &   
                     \gloss{‘release’}  &  \\

                     \vernacular{alamú[seebulila]
                    musáatsa/mú{\downstep}yáyi/muundu tá}  &   
                     \gloss{‘say bye to’}  &  \\

                     \vernacular{
                    alamú[hoombelitsila]
                    musáatsa/mú{\downstep}yáyi/muundu tá}  &   
                     \gloss{‘comfort’}  &  \\

                     \vernacular{
                    alamú[kalushitsila] musáatsa/mú{\downstep}yáyi/muundu
                    tá}  &   
                     \gloss{‘return’}  &  \\

                     \vernacular{
                    alamú[siinjilitsila]
                    musáatsa/mú{\downstep}yáyi/muundu tá}  &   
                     \gloss{
                    ‘make...stand’}  &  \\

                     \vernacular{
                    alamú[reebareebela] musáatsa/mú{\downstep}yáyi/muundu
                    tá}  &   
                     \gloss{‘ask (iter)’}  &  \\

                     \vernacular{
                    alamú[kalukhanyinyila]
                    musáatsa/mú{\downstep}yáyi/muundu tá}  &   
                     \gloss{
                    ‘turn...over’}  &  \\
\end{tabular}
%\caption{\nocaption}
     
\begin{tabular}{lll}  
  \multicolumn{2}{l}{
                     \vernacular{(51) /H/
                    C-Initial +OP + OP
                    } \gloss{‘s/he will
                    not...the man \ob musáatsa\cb  /} } &  \\
\multicolumn{2}{l}{
                     \gloss{the boy
                    \ob mú{\downstep}yáyi\cb  / someone \ob muundu\cb  for him/her for
                    me’} } &  \\

                     \vernacular{alamúu[ndeela]
                    musáatsa/mú{\downstep}yáyi/muundu tá}  &   
                     \gloss{‘bury’}  &  \\

                     \vernacular{alamúu[mbechela]
                    musáatsa/mú{\downstep}yáyi/muundu tá}  &   
                     \gloss{‘shave’}  &  \\

                     \vernacular{alamúu[ndeerela]
                    musáatsa/mú{\downstep}yáyi/muundu tá}  &   
                     \gloss{‘bring’}  &  \\

                     \vernacular{
                    alamúu[khalachila] musáatsa/mú{\downstep}yáyi/muundu
                    tá}  &   
                     \gloss{‘cut’}  &  \\

                     \vernacular{
                    alamúu[sitaachila] musáatsa/mú{\downstep}yáyi/muundu
                    tá}  &   
                     \gloss{‘accuse’}  &  \\

                     \vernacular{
                    alamúu[mboolitsila] musáatsa/mú{\downstep}yáyi/muundu
                    tá}  &   
                     \gloss{‘seduce’}  &  \\

                     \vernacular{
                    alamúu[mbohololela] musáatsa/mú{\downstep}yáyi/muundu
                    tá}  &   
                     \gloss{‘untie’}  &  \\
\end{tabular}
%\caption{\nocaption}
     
\begin{tabular}{lll}  
  \multicolumn{2}{l}{
                     \vernacular{(52) /Ø/
                    C-Initial +OP + OP
                    } \gloss{‘s/he will
                    not...the man \ob musáatsa\cb  /} } &  \\
\multicolumn{2}{l}{
                     \gloss{the boy
                    \ob mú{\downstep}yáyi\cb  / someone \ob muundu\cb  for him/her for
                    me’} } &  \\

                     \vernacular{alamúu[nziila]
                    musáatsa/mú{\downstep}yáyi/muundu tá}  &   
                     \gloss{‘go for’}  &  \\

                     \vernacular{alamúu[ndeshela]
                    musáatsa/mú{\downstep}yáyi/muundu tá}  &   
                     \gloss{‘leave’}  &  \\

                     \vernacular{alamúu[noondela]
                    musáatsa/mú{\downstep}yáyi/muundu tá}  &   
                     \gloss{‘follow’}  &  \\

                     \vernacular{
                    alamúu[ngulishila] musáatsa/mú{\downstep}yáyi/muundu
                    tá}  &   
                     \gloss{‘name’}  &  \\

                     \vernacular{
                    alamúu[ndakhuulila] musáatsa/mú{\downstep}yáyi/muundu
                    tá}  &   
                     \gloss{‘release’}  &  \\

                     \vernacular{alamúu[seebulila]
                    musáatsa/mú{\downstep}yáyi/muundu tá}  &   
                     \gloss{‘say bye to’}  &  \\

                     \vernacular{
                    alamúu[mboombelitsila]
                    musáatsa/mú{\downstep}yáyi/muundu tá}  &   
                     \gloss{‘comfort’}  &  \\

                     \vernacular{
                    alamúu[siinjilitsila]
                    musáatsa/mú{\downstep}yáyi/muundu tá}  &   
                     \gloss{
                    ‘make...stand’}  &  \\
\end{tabular}
%\caption{\nocaption}
    

\subsection{Remote Past: Pattern 4}\label{sec:sRemPast}


\begin{tabular}{llllll}  
  \multicolumn{5}{l}{
                     \vernacular{(53) /H/
                    C-Initial} \gloss{
                    ‘s/he...’} } &  \\
\multicolumn{5}{l}{ } &  \\

                     \vernacular{
                    yaa[rá]}  &   
                     \gloss{‘buried’}  &     &   
                     \vernacular{
                    yaa[ng’wá]}  &   
                     \gloss{‘drank’}  &  \\

                     \vernacular{
                    yaa[lía]}  &   
                     \gloss{‘ate’}  &     &   
                     \vernacular{
                    yaa[khwá]}  &   
                     \gloss{‘paid dowry’}  &  \\

                     \vernacular{
                    yaa[lúma]}  &   
                     \gloss{‘bit’}  &     &   
                     \vernacular{
                    yaa[béka]}  &   
                     \gloss{‘shaved’}  &  \\

                     \vernacular{
                    yaa[téekha]}  &   
                     \gloss{‘cooked’}  &     &   
                     \vernacular{
                    yaa[léera]}  &   
                     \gloss{‘brought’}  &  \\

                     \vernacular{
                    yaa[khálaka]}  &   
                     \gloss{‘cut’}  &     &   
                     \vernacular{
                    yaa[kálaanga]}  &   
                     \gloss{‘fried’}  &  \\

                     \vernacular{
                    yaa[sítaaka]}  &   
                     \gloss{‘accused’}  &     &   
                     \vernacular{
                    yaa[bóolitsa]}  &   
                     \gloss{‘seduced’}  &  \\

                     \vernacular{
                    yaa[sáanditsa]}  &   
                     \gloss{‘thanked’}  &     &   
                     \vernacular{
                    yaa[tsúunzuuna]}  &   
                     \gloss{‘sucked’}  &  \\

                     \vernacular{
                    yaa[bóholola]}  &   
                     \gloss{‘untied’}  &     &   
                     \vernacular{
                    yaa[bóyong’ana]}  &   
                     \gloss{‘went
                    around’}  &  \\

                     \vernacular{
                    yaa[ng’óng’oolitsa]}  &   
                     \gloss{‘teased’}  &     &   
                     \vernacular{
                    yaa[língakanyinya]}  &   
                     \gloss{‘crumpled’}  &  \\
\end{tabular}
%\caption{\nocaption}
     
\begin{tabular}{llllll}  
  \multicolumn{5}{l}{
                     \vernacular{(54) /H/
                    V-Initial} \gloss{
                    ‘s/he...’} } &  \\
\multicolumn{5}{l}{ } &  \\

                     \vernacular{
                    y[iíra]}  &   
                     \gloss{‘killed’}  &     &   
                     \vernacular{
                    y[ií{\downstep}kóómba]}  &   
                     \gloss{‘admired’}  &  \\

                     \vernacular{
                    y[ií{\downstep}síáka]}  &   
                     \gloss{‘smacked’}  &     &   
                     \vernacular{
                    y[ií{\downstep}kóbóla]}  &   
                     \gloss{‘belched’}  &  \\

                     \vernacular{
                    y[oónonyinya]}  &   
                     \gloss{‘spoiled’}  &     &   
                     \vernacular{
                    y[aábukhanyinya]}  &   
                     \gloss{‘separated’}  &  \\
\end{tabular}
%\caption{\nocaption}
     
\begin{tabular}{llllll}  
  \multicolumn{5}{l}{
                     \vernacular{(55) /Ø/
                    C-Initial} \gloss{
                    ‘s/he...’} } &  \\
\multicolumn{5}{l}{ } &  \\

                     \vernacular{
                    yaa[tsía]}  &   
                     \gloss{‘went’}  &     &   
                     \vernacular{
                    yaa[kwá]}  &   
                     \gloss{‘fell’}  &  \\

                     \vernacular{
                    yaa[lékha]}  &   
                     \gloss{‘left’}  &     &   
                     \vernacular{
                    yaa[réeba]}  &   
                     \gloss{‘asked’}  &  \\

                     \vernacular{
                    yaa[lóonda]}  &   
                     \gloss{‘followed’}  &     &   
                     \vernacular{
                    yaa[sósana]}  &   
                     \gloss{‘resembled’}  &  \\

                     \vernacular{
                    yaa[hómoola]}  &   
                     \gloss{‘massaged’}  &     &   
                     \vernacular{
                    yaa[lákhuula]}  &   
                     \gloss{‘released’}  &  \\

                     \vernacular{
                    yaa[séebula]}  &   
                     \gloss{‘said bye’}  &     &   
                     \vernacular{
                    yaa[hóombelitsa]}  &   
                     \gloss{‘comforted’}  &  \\

                     \vernacular{
                    yaa[kálushitsa]}  &   
                     \gloss{‘returned’}  &     &   
                     \vernacular{
                    yaa[síinjilitsa]}  &   
                     \gloss{‘made stand’}  &  \\

                     \vernacular{
                    yaa[réebareeba]}  &   
                     \gloss{‘asked
                    (iter)’}  &     &   
                     \vernacular{
                    yaa[kálukhanyinya]}  &   
                     \gloss{‘turned
                    over’}  &  \\

                     \vernacular{
                    yaa[sébulukhanyinya]}  &   
                     \gloss{‘scattered’}  &     &     &     &  \\
\end{tabular}
%\caption{\nocaption}
     
\begin{tabular}{llllll}  
  \multicolumn{5}{l}{
                     \vernacular{(56) /Ø/
                    V-Initial} \gloss{
                    ‘s/he...’} } &  \\
\multicolumn{5}{l}{ } &  \\

                     \vernacular{
                    y[eénya]}  &   
                     \gloss{‘wanted’}  &     &   
                     \vernacular{
                    y[aáshitsa]}  &   
                     \gloss{‘lit’}  &  \\

                     \vernacular{
                    y[iíluula]}  &   
                     \gloss{‘winnowed’}  &     &   
                     \vernacular{
                    y[aámbakhana]}  &   
                     \gloss{‘refused’}  &  \\

                     \vernacular{
                    y[eéleelitsa]}  &   
                     \gloss{‘hung up’}  &     &     &     &  \\
\end{tabular}
%\caption{\nocaption}
     
\begin{tabular}{llllll}  
  \multicolumn{5}{l}{
                     \vernacular{(57) /H/
                    C-Initial + OP} \gloss{
                    ‘s/he...him/her’} } &  \\
\multicolumn{5}{l}{ } &  \\

                     \vernacular{
                    yaamú[ra]}  &   
                     \gloss{‘buried’}  &     &   
                     \vernacular{
                    yaamú[khwa]}  &   
                     \gloss{‘paid dowry’}  &  \\

                     \vernacular{
                    yaamú[beka]}  &   
                     \gloss{‘shaved’}  &     &   
                     \vernacular{
                    yaamú[leera]}  &   
                     \gloss{‘brought’}  &  \\

                     \vernacular{
                    yaamú[khalaka]}  &   
                     \gloss{‘cut’}  &     &   
                     \vernacular{
                    yaamú[sitaaka]}  &   
                     \gloss{‘accused’}  &  \\

                     \vernacular{
                    yaamú[boolitsa]}  &   
                     \gloss{‘seduced’}  &     &   
                     \vernacular{
                    yaamú[tsuunzuuna]}  &   
                     \gloss{‘sucked’}  &  \\

                     \vernacular{
                    yaamú[boholola]}  &   
                     \gloss{‘untied’}  &     &   
                     \vernacular{
                    yaamú[bukaanila]}  &   
                     \gloss{‘met’}  &  \\

                     \vernacular{
                    yaamú[ng’ong’oolitsa]}  &   
                     \gloss{‘teased’}  &     &   
                     \vernacular{
                    yaamú[lingakanyinya]}  &   
                     \gloss{‘bent’}  &  \\
\end{tabular}
%\caption{\nocaption}
     
\begin{tabular}{llllll}  
  \multicolumn{5}{l}{
                     \vernacular{(58) /H/
                    V-Initial + OP} \gloss{
                    ‘s/he...him/her’} } &  \\
\multicolumn{5}{l}{ } &  \\

                     \vernacular{
                    yaamw[íira]}  &   
                     \gloss{‘killed’}  &     &   
                     \vernacular{
                    yaamw[í{\downstep}íkóómba]}  &   
                     \gloss{‘admired’}  &  \\

                     \vernacular{
                    yaamw[í{\downstep}ísíáka]}  &   
                     \gloss{‘smacked’}  &     &   
                     \vernacular{
                    yaamw[óononyinya]}  &   
                     \gloss{‘spoiled’}  &  \\

                     \vernacular{
                    yaamw[áabukhanyinya]}  &   
                     \gloss{‘separated’}  &     &     &     &  \\
\end{tabular}
%\caption{\nocaption}
     
\begin{tabular}{llllll}  
  \multicolumn{5}{l}{
                     \vernacular{(59) /Ø/
                    C-Initial + OP} \gloss{‘s/he...him/her
                    \ob mu-\cb  / them \ob ba-\cb ’} } &  \\
\multicolumn{5}{l}{ } &  \\

                     \vernacular{
                    yaamú[{\downstep}tsíá]}  &   
                     \gloss{‘went for’}  &  \\

                     \vernacular{
                    yaamú[{\downstep}lékhá]}  &   
                     \gloss{‘left’}  &  \\

                     \vernacular{
                    yaamú[{\downstep}lóónda]}  &   
                     \gloss{‘followed’}  &  \\

                     \vernacular{
                    yaamú[{\downstep}kúlíkha]}  &   
                     \gloss{‘named’}  &  \\

                     \vernacular{
                    yaamú[{\downstep}lákhúula]}  &   
                     \gloss{‘released’}  &  \\

                     \vernacular{
                    yaamú[{\downstep}séébula]}  &   
                     \gloss{‘said bye
                    to’}  &  \\

                     \vernacular{
                    yaamú[{\downstep}hóómbélitsa]}  &   
                     \gloss{‘comforted’}  &  \\

                     \vernacular{
                    yaamú[{\downstep}kálúshitsa]}  &   
                     \gloss{‘returned’}  &  \\

                     \vernacular{
                    yaamú[{\downstep}síínjílitsa]}  &   
                     \gloss{
                    ‘made...stand’}  &  \\

                     \vernacular{
                    yaamú[{\downstep}réébáreeba]}  &   
                     \gloss{‘asked
                    (iter)’}  &  \\

                     \vernacular{
                    yaamú[{\downstep}kálúkhányinya]}  &   
                     \gloss{
                    ‘turned...over’}  &  \\

                     \vernacular{
                    yaabá[{\downstep}sébúlúkhanyinya]}  &   
                     \gloss{‘scattered’}  &  \\
\end{tabular}
%\caption{\nocaption}
     
\begin{tabular}{llllll}  
  \multicolumn{5}{l}{
                     \vernacular{(60) /Ø/
                    V-Initial + OP} \gloss{‘s/he...him/her
                    \ob mw-\cb  / it
                    } } &  \\
\multicolumn{5}{l}{ } &  \\

                     \vernacular{
                    yaamw[é{\downstep}ényá]}  &   
                     \gloss{‘wanted’}  &     &   
                     \vernacular{
                    yaamw[é{\downstep}éyéla]}  &   
                     \gloss{‘wiped for’}  &  \\

                     \vernacular{
                    yaabw[í{\downstep}ílúula]}  &   
                     \gloss{‘winnowed’}  &     &   
                     \vernacular{
                    yaamw[á{\downstep}ámbákhana]}  &   
                     \gloss{‘refused’}  &  \\

                     \vernacular{
                    yaamw[é{\downstep}éléelitsa]}  &   
                     \gloss{‘hung...up’}  &  \\
\end{tabular}
%\caption{\nocaption}
     
\begin{tabular}{llllll}  
  \multicolumn{5}{l}{
                     \vernacular{(61) /H/
                    C-Initial + OP
                    } \gloss{
                    ‘s/he...me’} } &  \\
\multicolumn{5}{l}{ } &  \\

                     \vernacular{
                    yaá[ria]}  &   
                     \gloss{‘feared’}  &     &   
                     \vernacular{
                    yaá[khwa]}  &   
                     \gloss{‘paid dowry’}  &  \\

                     \vernacular{
                    yaá[mbeka]}  &   
                     \gloss{‘shaved’}  &     &   
                     \vernacular{
                    yaá[ndeera]}  &   
                     \gloss{‘brought’}  &  \\

                     \vernacular{
                    yaá[khalaka]}  &   
                     \gloss{‘cut’}  &     &   
                     \vernacular{
                    yaá[sitaaka]}  &   
                     \gloss{‘accused’}  &  \\

                     \vernacular{
                    yaá[mboolitsa]}  &   
                     \gloss{‘seduced’}  &     &   
                     \vernacular{
                    yaá[ndzuunzuuna]}  &   
                     \gloss{‘sucked’}  &  \\

                     \vernacular{
                    yaá[mboholola]}  &   
                     \gloss{‘untied’}  &     &   
                     \vernacular{
                    yaá[mboyong’ana]}  &   
                     \gloss{‘went
                    around’}  &  \\

                     \vernacular{
                    yaá[mbukaanila]}  &   
                     \gloss{‘met’}  &     &   
                     \vernacular{
                    yaá[ng’ong’oolitsa]}  &   
                     \gloss{‘teased’}  &  \\

                     \vernacular{
                    yaá[ningakanyinya]}  &   
                     \gloss{‘bent’}  &     &     &     &  \\
\end{tabular}
%\caption{\nocaption}
     
\begin{tabular}{llllll}  
  \multicolumn{5}{l}{
                     \vernacular{(62) /H/
                    V-Initial + OP
                    } \gloss{
                    ‘s/he...me’} } &  \\
\multicolumn{5}{l}{ } &  \\

                     \vernacular{
                    yaá[nzira]}  &   
                     \gloss{‘killed’}  &     &   
                     \vernacular{
                    yaa[nzí{\downstep}kóómba]}  &   
                     \gloss{‘admired’}  &  \\

                     \vernacular{
                    yaa[nzí{\downstep}síáka]}  &   
                     \gloss{‘smacked’}  &     &   
                     \vernacular{
                    yaá[nzononyinya]}  &   
                     \gloss{‘spoiled’}  &  \\

                     \vernacular{
                    yaá[nzabukhanyinya]}  &   
                     \gloss{‘separated’}  &     &     &     &  \\
\end{tabular}
%\caption{\nocaption}
     
\begin{tabular}{llllll}  
  \multicolumn{5}{l}{
                     \vernacular{(63) /Ø/
                    C-Initial + OP
                    } \gloss{
                    ‘s/he...me’} } &  \\
\multicolumn{5}{l}{ } &  \\

                     \vernacular{
                    yaá[{\downstep}ndékhá]}  &   
                     \gloss{‘left’}  &     &   
                     \vernacular{
                    yaá[{\downstep}nóónda]}  &   
                     \gloss{‘followed’}  &  \\

                     \vernacular{
                    yaá[{\downstep}ngúlíkha]}  &   
                     \gloss{‘named’}  &     &   
                     \vernacular{
                    yaá[{\downstep}ndákhúula]}  &   
                     \gloss{‘released’}  &  \\

                     \vernacular{
                    yaá[{\downstep}séébula]}  &   
                     \gloss{‘said bye
                    to’}  &     &   
                     \vernacular{
                    yaá[{\downstep}mbóómbélitsa]}  &   
                     \gloss{‘comforted’}  &  \\

                     \vernacular{
                    yaá[{\downstep}síínjílitsa]}  &   
                     \gloss{
                    ‘made...stand’}  &     &   
                     \vernacular{
                    yaá[{\downstep}ndéébándeeba]}  &   
                     \gloss{‘asked
                    (iter)’}  &  \\

                     \vernacular{
                    yaá[{\downstep}ngálúkhányinya]}  &   
                     \gloss{
                    ‘turned...over’}  &  \\
\end{tabular}
%\caption{\nocaption}
     
\begin{tabular}{llllll}  
  \multicolumn{5}{l}{
                     \vernacular{(64) /Ø/
                    V-Initial + OP
                    } \gloss{
                    ‘s/he...me’} } &  \\
\multicolumn{5}{l}{ } &  \\

                     \vernacular{
                    yaá[{\downstep}nzényá]}  &   
                     \gloss{‘wanted’}  &     &   
                     \vernacular{
                    yaá[{\downstep}nzéyéla]}  &   
                     \gloss{‘wiped for’}  &  \\

                     \vernacular{
                    yaá[{\downstep}nyámbákhana]}  &   
                     \gloss{‘refused’}  &     &   
                     \vernacular{
                    yaá[{\downstep}nzéléelitsa]}  &   
                     \gloss{
                    ‘carried...hanging’}  &  \\
\end{tabular}
%\caption{\nocaption}
     
\begin{tabular}{llllll}  
  \multicolumn{5}{l}{
                     \vernacular{(65) /H/
                    C-Initial + OP
                    } \gloss{
                    ‘s/he...him/herself’} } &  \\
\multicolumn{5}{l}{ } &  \\

                     \vernacular{
                    yií[ra]}  &   
                     \gloss{‘buried’}  &     &   
                     \vernacular{
                    yií[khwa]}  &   
                     \gloss{‘paid dowry’}  &  \\

                     \vernacular{
                    yií[beka]}  &   
                     \gloss{‘shaved’}  &     &   
                     \vernacular{
                    yií[suunga]}  &   
                     \gloss{‘hung’}  &  \\

                     \vernacular{
                    yií[khalaka]}  &   
                     \gloss{‘cut’}  &     &   
                     \vernacular{
                    yií[sitaaka]}  &   
                     \gloss{‘accused’}  &  \\

                     \vernacular{
                    yií[saanditsa]}  &   
                     \gloss{‘thanked’}  &     &   
                     \vernacular{
                    yií[khong’oonda]}  &   
                     \gloss{‘knocked’}  &  \\

                     \vernacular{
                    yií[boholola]}  &   
                     \gloss{‘untied’}  &     &     &     &  \\
\end{tabular}
%\caption{\nocaption}
     
\begin{tabular}{llllll}  
  \multicolumn{5}{l}{
                     \vernacular{(66) /H/
                    V-Initial + OP
                    } \gloss{
                    ‘s/he...him/herself’} } &  \\
\multicolumn{5}{l}{ } &  \\

                     \vernacular{
                    yií[yira]}  &   
                     \gloss{‘killed’}  &     &   
                     \vernacular{
                    yií[{\downstep}yíkóómba]}  &   
                     \gloss{‘admired’}  &  \\

                     \vernacular{
                    yií[{\downstep}yísíáka]}  &   
                     \gloss{‘smacked’}  &     &   
                     \vernacular{
                    yií[yononyinya]}  &   
                     \gloss{‘spoiled’}  &  \\

                     \vernacular{
                    yií[yabukhanyinya]}  &   
                     \gloss{‘separated’}  &     &     &     &  \\
\end{tabular}
%\caption{\nocaption}
     
\begin{tabular}{llllll}  
  \multicolumn{5}{l}{
                     \vernacular{(67) /Ø/
                    C-Initial + OP
                    } \gloss{
                    ‘s/he...him/herself’} } &  \\
\multicolumn{5}{l}{ } &  \\

                     \vernacular{
                    yií[{\downstep}lékhá]}  &   
                     \gloss{‘left’}  &     &   
                     \vernacular{
                    yií[{\downstep}síínga]}  &   
                     \gloss{‘bathed’}  &  \\

                     \vernacular{
                    yií[{\downstep}kúlíkha]}  &   
                     \gloss{‘named’}  &     &   
                     \vernacular{
                    yií[{\downstep}náábula]}  &   
                     \gloss{‘undressed’}  &  \\

                     \vernacular{
                    yií[{\downstep}lákhúula]}  &   
                     \gloss{‘released’}  &     &   
                     \vernacular{
                    yií[{\downstep}hóómbélitsa]}  &   
                     \gloss{‘comforted’}  &  \\

                     \vernacular{
                    yií[{\downstep}síínjílitsa]}  &   
                     \gloss{
                    ‘made...stand’}  &     &   
                     \vernacular{
                    yií[{\downstep}réébáreeba]}  &   
                     \gloss{‘asked
                    (iter)’}  &  \\

                     \vernacular{
                    yií[{\downstep}kálúkhányinya]}  &   
                     \gloss{
                    ‘turned...over’}  &  \\
\end{tabular}
%\caption{\nocaption}
     
\begin{tabular}{llllll}  
  \multicolumn{5}{l}{
                     \vernacular{(68) /Ø/
                    V-Initial + OP
                    } \gloss{
                    ‘s/he...him/herself’} } &  \\
\multicolumn{5}{l}{ } &  \\

                     \vernacular{
                    yií[yaala]}  &   
                     \gloss{‘exposed’}  &     &   
                     \vernacular{
                    yií[{\downstep}yéyá]}  &   
                     \gloss{‘wiped’}  &  \\

                     \vernacular{
                    yií[{\downstep}yéyéla]}  &   
                     \gloss{‘wiped for’}  &     &   
                     \vernacular{
                    yií[{\downstep}yámbákhana]}  &   
                     \gloss{‘despised’}  &  \\

                     \vernacular{
                    yií[{\downstep}yéléelitsa]}  &   
                     \gloss{‘hung’}  &     &     &     &  \\
\end{tabular}
%\caption{\nocaption}
     
\begin{tabular}{llllll}  
  \multicolumn{5}{l}{
                     \vernacular{(69) /H/
                    C-Initial + OP + OP
                    } \gloss{‘s/he...him/her
                    for me’} } &  \\
\multicolumn{5}{l}{ } &  \\

                     \vernacular{
                    yaamúu[ndeela]}  &   
                     \gloss{‘buried’}  &     &   
                     \vernacular{
                    yaamúu[mbechela]}  &   
                     \gloss{‘shaved’}  &  \\

                     \vernacular{
                    yaamúu[ndeerela]}  &   
                     \gloss{‘brought’}  &     &   
                     \vernacular{
                    yaamúu[khalachila]}  &   
                     \gloss{‘cut’}  &  \\

                     \vernacular{
                    yaamúu[sitaachila]}  &   
                     \gloss{‘accused’}  &     &   
                     \vernacular{
                    yaamúu[mboolitsila]}  &   
                     \gloss{‘seduced’}  &  \\

                     \vernacular{
                    yaamúu[mbohololela]}  &   
                     \gloss{‘untied’}  &     &     &     &  \\
\end{tabular}
%\caption{\nocaption}
     
\begin{tabular}{llllll}  
  \multicolumn{5}{l}{
                     \vernacular{(70) /H/
                    V-Initial + OP + OP
                    } \gloss{‘s/he...him/her
                    for me’} } &  \\
\multicolumn{5}{l}{ } &  \\

                     \vernacular{
                    yaamúu[nzirila]}  &   
                     \gloss{‘killed’}  &     &   
                     \vernacular{
                    yaamúu[nzechitsila]}  &   
                     \gloss{‘admired’}  &  \\

                     \vernacular{
                    yaamú{\downstep}ú[nzísíáchila]}  &   
                     \gloss{‘smacked’}  &     &   
                     \vernacular{
                    yaamúu[nzononyinyila]}  &   
                     \gloss{‘spoiled’}  &  \\

                     \vernacular{
                    yaamúu[nzabukhanyinyila]}  &   
                     \gloss{‘separated’}  &     &     &     &  \\
\end{tabular}
%\caption{\nocaption}
     
\begin{tabular}{llllll}  
  \multicolumn{5}{l}{
                     \vernacular{(71) /Ø/
                    C-Initial + OP + OP
                    } \gloss{‘s/he...him/her
                    for me’} } &  \\
\multicolumn{5}{l}{ } &  \\

                     \vernacular{
                    yaamú{\downstep}ú[nzííla]}  &   
                     \gloss{‘went for’}  &     &   
                     \vernacular{
                    yaamú{\downstep}ú[ndéshéla]}  &   
                     \gloss{‘left’}  &  \\

                     \vernacular{
                    yaamú{\downstep}ú[nóóndela]}  &   
                     \gloss{‘followed’}  &     &   
                     \vernacular{
                    yaamú{\downstep}ú[ngúlíshila]}  &   
                     \gloss{‘named’}  &  \\

                     \vernacular{
                    yaamú{\downstep}ú[ndákhúulila]}  &   
                     \gloss{‘released’}  &     &   
                     \vernacular{
                    yaamú{\downstep}ú[séébúlila]}  &   
                     \gloss{‘said bye
                    to’}  &  \\

                     \vernacular{
                    yaamú{\downstep}ú[mbóómbélitsila]}  &   
                     \gloss{‘comforted’}  &     &   
                     \vernacular{
                    yaamú{\downstep}ú[síínjílitsila]}  &   
                     \gloss{
                    ‘made...stand’}  &  \\
\end{tabular}
%\caption{\nocaption}
     
\begin{tabular}{llllll}  
  \multicolumn{5}{l}{
                     \vernacular{(72) /Ø/
                    V-Initial + OP + OP
                    } \gloss{‘s/he...him/her
                    \ob mu-\cb  / it
                    } } &  \\
\multicolumn{5}{l}{ } &  \\

                     \vernacular{
                    yaabú{\downstep}ú[nzálíla]}  &   
                     \gloss{‘displayed’}  &     &   
                     \vernacular{
                    yaakú{\downstep}ú[nzáshítsila]}  &   
                     \gloss{‘lit’}  &  \\

                     \vernacular{
                    yaabú{\downstep}ú[nzílúulila]}  &   
                     \gloss{‘winnowed’}  &     &   
                     \vernacular{
                    yaalú{\downstep}ú[nzítsúlitsila]}  &   
                     \gloss{‘filled’}  &  \\

                     \vernacular{
                    yaakú{\downstep}ú[nzéléelitsila]}  &   
                     \gloss{‘hung’}  &     &     &     &  \\
\end{tabular}
%\caption{\nocaption}
     
\begin{tabular}{lll}  
  \multicolumn{2}{l}{
                     \vernacular{(73) /H/
                    C-Initial Phrase-Medial} \gloss{‘s/he...the man
                    \ob musáatsa\cb  /} } &  \\
\multicolumn{2}{l}{
                     \gloss{the boy
                    \ob mú{\downstep}yáyi\cb  / someone \ob muundu\cb ’} } &  \\

                     \vernacular{yaa[rá]
                    musáatsa/mú{\downstep}yáyi/muundu}  &   
                     \gloss{‘buried’}  &  \\

                     \vernacular{yaa[béka]
                    musáatsa/mú{\downstep}yáyi/muundu}  &   
                     \gloss{‘shaved’}  &  \\

                     \vernacular{yaa[léera]
                    musáatsa/mú{\downstep}yáyi/muundu}  &   
                     \gloss{‘brought’}  &  \\

                     \vernacular{yaa[khálaka]
                    musáatsa/mú{\downstep}yáyi/muundu}  &   
                     \gloss{‘cut’}  &  \\

                     \vernacular{yaa[sítaaka]
                    musáatsa/mú{\downstep}yáyi/muundu}  &   
                     \gloss{‘accused’}  &  \\

                     \vernacular{yaa[bóolitsa]
                    musáatsa/mú{\downstep}yáyi/muundu}  &   
                     \gloss{‘seduced’}  &  \\

                     \vernacular{yaa[tsúunzuuna]
                    musáatsa/mú{\downstep}yáyi/muundu}  &   
                     \gloss{‘sucked’}  &  \\

                     \vernacular{yaa[bóholola]
                    musáatsa/mú{\downstep}yáyi/muundu}  &   
                     \gloss{‘untied’}  &  \\

                     \vernacular{yaa[búkaanila]
                    musáatsa/mú{\downstep}yáyi/muundu}  &   
                     \gloss{‘met’}  &  \\

                     \vernacular{
                    yaa[ng’óng’oolitsa]
                    musáatsa/mú{\downstep}yáyi/muundu}  &   
                     \gloss{‘teased’}  &  \\
\end{tabular}
%\caption{\nocaption}
     
\begin{tabular}{lll}  
  \multicolumn{2}{l}{
                     \vernacular{(74) /Ø/
                    C-Initial Phrase-Medial} \gloss{‘s/he...the man
                    \ob musáatsa\cb  /} } &  \\
\multicolumn{2}{l}{
                     \gloss{the boy
                    \ob mú{\downstep}yáyi\cb  / someone \ob muundu\cb ’} } &  \\

                     \vernacular{yaa[tsía]
                    musáatsa/mú{\downstep}yáyi/muundu}  &   
                     \gloss{‘went for’}  &  \\

                     \vernacular{yaa[lékha]
                    musáatsa/mú{\downstep}yáyi/muundu}  &   
                     \gloss{‘left’}  &  \\

                     \vernacular{yaa[lóonda]
                    musáatsa/mú{\downstep}yáyi/muundu}  &   
                     \gloss{‘followed’}  &  \\

                     \vernacular{yaa[kúlikha]
                    musáatsa/mú{\downstep}yáyi/muundu}  &   
                     \gloss{‘named’}  &  \\

                     \vernacular{yaa[lákhuula]
                    musáatsa/mú{\downstep}yáyi/muundu}  &   
                     \gloss{‘released’}  &  \\

                     \vernacular{yaa[séebula]
                    musáatsa/mú{\downstep}yáyi/muundu}  &   
                     \gloss{‘said bye
                    to’}  &  \\

                     \vernacular{yaa[kálushitsa]
                    musáatsa/mú{\downstep}yáyi/muundu}  &   
                     \gloss{‘returned’}  &  \\

                     \vernacular{yaa[síinjilitsa]
                    musáatsa/mú{\downstep}yáyi/muundu}  &   
                     \gloss{
                    ‘made...stand’}  &  \\

                     \vernacular{yaa[réebareeba]
                    musáatsa/mú{\downstep}yáyi/muundu}  &   
                     \gloss{‘asked
                    (iter)’}  &  \\

                     \vernacular{
                    yaa[kálukhanyinya]
                    musáatsa/mú{\downstep}yáyi/muundu}  &   
                     \gloss{
                    ‘turned...over’}  &  \\
\end{tabular}
%\caption{\nocaption}
     
\begin{tabular}{lll}  
  \multicolumn{2}{l}{
                     \vernacular{(75) /H/
                    C-Initial +OP Phrase-Medial} \gloss{‘s/he...the man
                    \ob musáatsa\cb  /} } &  \\
\multicolumn{2}{l}{
                     \gloss{the boy
                    \ob mú{\downstep}yáyi\cb  / someone \ob muundu\cb  for
                    him/her’} } &  \\

                     \vernacular{yaamú[reela]
                    musáatsa/mú{\downstep}yáyi/muundu}  &   
                     \gloss{‘buried’}  &  \\

                     \vernacular{yaamú[bechela]
                    musáatsa/mú{\downstep}yáyi/muundu}  &   
                     \gloss{‘shaved’}  &  \\

                     \vernacular{yaamú[leerela]
                    musáatsa/mú{\downstep}yáyi/muundu}  &   
                     \gloss{‘brought’}  &  \\

                     \vernacular{yaamú[khalachila]
                    musáatsa/mú{\downstep}yáyi/muundu}  &   
                     \gloss{‘cut’}  &  \\

                     \vernacular{yaamú[sitaachila]
                    musáatsa/mú{\downstep}yáyi/muundu}  &   
                     \gloss{‘accused’}  &  \\

                     \vernacular{yaamú[boolitsila]
                    musáatsa/mú{\downstep}yáyi/muundu}  &   
                     \gloss{‘seduced’}  &  \\

                     \vernacular{
                    yaamú[tsuunzuunila]
                    musáatsa/mú{\downstep}yáyi/muundu}  &   
                     \gloss{‘sucked’}  &  \\

                     \vernacular{yaamú[bohololela]
                    musáatsa/mú{\downstep}yáyi/muundu}  &   
                     \gloss{‘untied’}  &  \\

                     \vernacular{yaamú[bukaanila]
                    musáatsa/mú{\downstep}yáyi/muundu}  &   
                     \gloss{‘met’}  &  \\

                     \vernacular{
                    yaamú[ng’ong’oolitsila]
                    musáatsa/mú{\downstep}yáyi/muundu}  &   
                     \gloss{‘teased’}  &  \\
\end{tabular}
%\caption{\nocaption}
     
\begin{tabular}{lll}  
  \multicolumn{2}{l}{
                     \vernacular{(76) /Ø/
                    C-Initial +OP Phrase-Medial} \gloss{‘s/he...the man
                    \ob musáatsa\cb  /} } &  \\
\multicolumn{2}{l}{
                     \gloss{the boy
                    \ob mú{\downstep}yáyi\cb  / someone \ob muundu\cb  for
                    him/her’} } &  \\

                     \vernacular{yaamú[{\downstep}tsíílá]
                    musáatsa/mú{\downstep}yáyi/muundu}  &   
                     \gloss{‘went for’}  &  \\

                     \vernacular{
                    yaamú[{\downstep}léshélá]
                    musáatsa/mú{\downstep}yáyi/muundu}  &   
                     \gloss{‘left’}  &  \\

                     \vernacular{
                    yaamú[{\downstep}lóóndéla]
                    musáatsa/mú{\downstep}yáyi/muundu}  &   
                     \gloss{‘followed’}  &  \\

                     \vernacular{
                    yaamú[{\downstep}kúlíshíla]
                    musáatsa/mú{\downstep}yáyi/muundu}  &   
                     \gloss{‘named’}  &  \\

                     \vernacular{
                    yaamú[{\downstep}lákhúulila]
                    musáatsa/mú{\downstep}yáyi/muundu}  &   
                     \gloss{‘released’}  &  \\

                     \vernacular{
                    yaamú[{\downstep}séébúlila]
                    musáatsa/mú{\downstep}yáyi/muundu}  &   
                     \gloss{‘said bye
                    to’}  &  \\

                     \vernacular{
                    yaamú[{\downstep}réébáreebela]
                    musáatsa/mú{\downstep}yáyi/muundu}  &   
                     \gloss{‘asked
                    (iter)’}  &  \\
\end{tabular}
%\caption{\nocaption}
     
\begin{tabular}{lll}  
  \multicolumn{2}{l}{
                     \vernacular{(77) /H/
                    C-Initial +OP + OP
                    } \gloss{‘s/he...the man
                    \ob musáatsa\cb  /} } &  \\
\multicolumn{2}{l}{
                     \gloss{the boy
                    \ob mú{\downstep}yáyi\cb  / someone \ob muundu\cb  for him/her for
                    me’} } &  \\

                     \vernacular{yaamúu[ndeela]
                    musáatsa/mú{\downstep}yáyi/muundu}  &   
                     \gloss{‘buried’}  &  \\

                     \vernacular{yaamúu[mbechela]
                    musáatsa/mú{\downstep}yáyi/muundu}  &   
                     \gloss{‘shaved’}  &  \\

                     \vernacular{yaamúu[ndeerela]
                    musáatsa/mú{\downstep}yáyi/muundu}  &   
                     \gloss{‘brought’}  &  \\

                     \vernacular{
                    yaamúu[khalachila]
                    musáatsa/mú{\downstep}yáyi/muundu}  &   
                     \gloss{‘cut’}  &  \\

                     \vernacular{
                    yaamúu[sitaachila]
                    musáatsa/mú{\downstep}yáyi/muundu}  &   
                     \gloss{‘accused’}  &  \\

                     \vernacular{
                    yaamúu[mboolitsila]
                    musáatsa/mú{\downstep}yáyi/muundu}  &   
                     \gloss{‘seduced’}  &  \\

                     \vernacular{
                    yaamúu[mbohololela]
                    musáatsa/mú{\downstep}yáyi/muundu}  &   
                     \gloss{‘untied’}  &  \\
\end{tabular}
%\caption{\nocaption}
     
\begin{tabular}{lll}  
  \multicolumn{2}{l}{
                     \vernacular{(78) /Ø/
                    C-Initial +OP + OP
                    } \gloss{‘s/he...the man
                    \ob musáatsa\cb  /} } &  \\
\multicolumn{2}{l}{
                     \gloss{the boy
                    \ob mú{\downstep}yáyi\cb  / someone \ob muundu\cb  for him/her for
                    me’} } &  \\

                     \vernacular{
                    yaamú{\downstep}ú[nzíílá]
                    musáatsa/mú{\downstep}yáyi/muundu}  &   
                     \gloss{‘went for’}  &  \\

                     \vernacular{
                    yaamú{\downstep}ú[ndéshélá]
                    musáatsa/mú{\downstep}yáyi/muundu}  &   
                     \gloss{‘left’}  &  \\

                     \vernacular{
                    yaamú{\downstep}ú[nóóndéla]
                    musáatsa/mú{\downstep}yáyi/muundu}  &   
                     \gloss{‘followed’}  &  \\

                     \vernacular{
                    yaamú{\downstep}ú[ngúlíshíla]
                    musáatsa/mú{\downstep}yáyi/muundu}  &   
                     \gloss{‘named’}  &  \\

                     \vernacular{
                    yaamú{\downstep}ú[ndákhúulila]
                    musáatsa/mú{\downstep}yáyi/muundu}  &   
                     \gloss{‘released’}  &  \\

                     \vernacular{
                    yaamú{\downstep}ú[séébúlila]
                    musáatsa/mú{\downstep}yáyi/muundu}  &   
                     \gloss{‘said bye
                    to’}  &  \\

                     \vernacular{
                    yaamú{\downstep}ú[mbóómbélitsila]
                    musáatsa/mú{\downstep}yáyi/muundu}  &   
                     \gloss{‘comforted’}  &  \\

                     \vernacular{
                    yaamú{\downstep}ú[síínjílitsila]
                    musáatsa/mú{\downstep}yáyi/muundu}  &   
                     \gloss{
                    ‘made...stand’}  &  \\
\end{tabular}
%\caption{\nocaption}
    

\subsection{Remote Past Negative: Pattern 4}\label{sec:sRemPastNeg}


\begin{tabular}{llllll}  
  \multicolumn{5}{l}{
                     \vernacular{(79) /H/
                    C-Initial} \gloss{‘s/he did
                    not...’} } &  \\
\multicolumn{5}{l}{ } &  \\

                     \vernacular{yaa[rá]
                    {\downstep}tá}  &   
                     \gloss{‘bury’}  &     &   
                     \vernacular{yaa[ng’wá]
                    {\downstep}tá}  &   
                     \gloss{‘drink’}  &  \\

                     \vernacular{yaa[khwá]
                    {\downstep}tá}  &   
                     \gloss{‘pay dowry’}  &     &   
                     \vernacular{yaa[líá]
                    {\downstep}tá}  &   
                     \gloss{‘eat’}  &  \\

                     \vernacular{yaa[lúma]
                    tá}  &   
                     \gloss{‘bite’}  &     &   
                     \vernacular{yaa[béka]
                    tá}  &   
                     \gloss{‘shave’}  &  \\

                     \vernacular{yaa[téekha]
                    tá}  &   
                     \gloss{‘cook’}  &     &   
                     \vernacular{yaa[léera]
                    tá}  &   
                     \gloss{‘bring’}  &  \\

                     \vernacular{yaa[khálaka]
                    tá}  &   
                     \gloss{‘cut’}  &     &   
                     \vernacular{yaa[kálaanga]
                    tá}  &   
                     \gloss{‘fry’}  &  \\

                     \vernacular{yaa[sítaaka]
                    tá}  &   
                     \gloss{‘accuse’}  &     &   
                     \vernacular{yaa[bóolitsa]
                    tá}  &   
                     \gloss{‘seduce’}  &  \\

                     \vernacular{yaa[sáanditsa]
                    tá}  &   
                     \gloss{‘thank’}  &     &   
                     \vernacular{yaa[khóng’oonda]
                    tá}  &   
                     \gloss{‘knock’}  &  \\

                     \vernacular{yaa[tsúunzuuna]
                    tá}  &   
                     \gloss{‘suck’}  &     &   
                     \vernacular{yaa[boholola]
                    tá}  &   
                     \gloss{‘untie’}  &  \\

                     \vernacular{yaa[bóyong’ana]
                    tá}  &   
                     \gloss{‘go around’}  &     &   
                     \vernacular{
                    yaa[ng’óng’oolitsa] tá}  &   
                     \gloss{‘tease’}  &  \\

                     \vernacular{
                    yaa[língakanyinya] tá}  &   
                     \gloss{‘crumple’}  &     &     &     &  \\
\end{tabular}
%\caption{\nocaption}
     
\begin{tabular}{llllll}  
  \multicolumn{5}{l}{
                     \vernacular{(80) /H/
                    V-Initial} \gloss{‘s/he did
                    not...’} } &  \\
\multicolumn{5}{l}{ } &  \\

                     \vernacular{y[iíra]
                    tá}  &   
                     \gloss{‘kill’}  &     &   
                     \vernacular{y[ií{\downstep}kóómba]
                    tá}  &   
                     \gloss{‘admire’}  &  \\

                     \vernacular{y[ií{\downstep}síáka]
                    tá}  &   
                     \gloss{‘smack’}  &     &   
                     \vernacular{y[ií{\downstep}kóbóla]
                    tá}  &   
                     \gloss{‘belch’}  &  \\

                     \vernacular{y[oónonyinya]
                    tá}  &   
                     \gloss{‘spoil’}  &     &   
                     \vernacular{y[aábukhanyinya]
                    tá}  &   
                     \gloss{‘separate’}  &  \\
\end{tabular}
%\caption{\nocaption}
     
\begin{tabular}{llllll}  
  \multicolumn{5}{l}{
                     \vernacular{(81) /Ø/
                    C-Initial} \gloss{‘s/he did
                    not...’} } &  \\
\multicolumn{5}{l}{ } &  \\

                     \vernacular{yaa[tsíá]
                    {\downstep}tá}  &   
                     \gloss{‘go’}  &     &   
                     \vernacular{yaa[kwá]
                    {\downstep}tá}  &   
                     \gloss{‘fall’}  &  \\

                     \vernacular{yaa[lékha]
                    tá}  &   
                     \gloss{‘leave’}  &     &   
                     \vernacular{yaa[réeba]
                    tá}  &   
                     \gloss{‘ask’}  &  \\

                     \vernacular{yaa[lóonda]
                    tá}  &   
                     \gloss{‘follow’}  &     &   
                     \vernacular{yaa[kúlikha]
                    tá}  &   
                     \gloss{‘name’}  &  \\

                     \vernacular{yaa[sósana]
                    tá}  &   
                     \gloss{‘resemble’}  &     &   
                     \vernacular{yaa[hómoola]
                    tá}  &   
                     \gloss{‘massage’}  &  \\

                     \vernacular{yaa[lákhuula]
                    tá}  &   
                     \gloss{‘release’}  &     &   
                     \vernacular{yaa[séebula]
                    tá}  &   
                     \gloss{‘say bye’}  &  \\

                     \vernacular{yaa[hóombelitsa]
                    tá}  &   
                     \gloss{‘comfort’}  &     &   
                     \vernacular{yaa[kálushitsa]
                    tá}  &   
                     \gloss{‘return’}  &  \\

                     \vernacular{yaa[síinjilitsa]
                    tá}  &   
                     \gloss{‘make stand’}  &     &   
                     \vernacular{yaa[réebareeba]
                    tá}  &   
                     \gloss{‘ask (iter)’}  &  \\

                     \vernacular{
                    yaa[kálukhanyinya] tá}  &   
                     \gloss{‘turn over’}  &     &   
                     \vernacular{
                    yaa[sébulukhanyinya] tá}  &   
                     \gloss{‘scatter’}  &  \\
\end{tabular}
%\caption{\nocaption}
     
\begin{tabular}{llllll}  
  \multicolumn{5}{l}{
                     \vernacular{(82) /Ø/
                    V-Initial} \gloss{‘s/he did
                    not...’} } &  \\
\multicolumn{5}{l}{ } &  \\

                     \vernacular{y[eénya]
                    tá}  &   
                     \gloss{‘want’}  &     &   
                     \vernacular{y[aáshitsa]
                    tá}  &   
                     \gloss{‘light’}  &  \\

                     \vernacular{y[iíluula]
                    tá}  &   
                     \gloss{‘winnow’}  &     &   
                     \vernacular{y[aámbakhana]
                    tá}  &   
                     \gloss{‘refuse’}  &  \\

                     \vernacular{y[eéleelitsa]
                    tá}  &   
                     \gloss{‘hang up’}  &     &     &     &  \\
\end{tabular}
%\caption{\nocaption}
     
\begin{tabular}{llllll}  
  \multicolumn{5}{l}{
                     \vernacular{(83) /H/
                    C-Initial + OP} \gloss{‘s/he did
                    not...him/her’} } &  \\
\multicolumn{5}{l}{ } &  \\

                     \vernacular{yaamú[ra]
                    tá}  &   
                     \gloss{‘bury’}  &     &   
                     \vernacular{yaamú[beka]
                    tá}  &   
                     \gloss{‘shave’}  &  \\

                     \vernacular{yaamú[leera]
                    tá}  &   
                     \gloss{‘bring’}  &     &   
                     \vernacular{yaamú[khalaka]
                    tá}  &   
                     \gloss{‘cut’}  &  \\

                     \vernacular{yaamú[sitaaka]
                    tá}  &   
                     \gloss{‘accuse’}  &     &   
                     \vernacular{yaamú[boolitsa]
                    tá}  &   
                     \gloss{‘seduce’}  &  \\

                     \vernacular{
                    yaamú[khong’oonda] tá}  &   
                     \gloss{‘knock’}  &     &   
                     \vernacular{yaamú[tsuunzuuna]
                    tá}  &   
                     \gloss{‘suck’}  &  \\

                     \vernacular{yaamú[boholola]
                    tá}  &   
                     \gloss{‘untie’}  &     &   
                     \vernacular{yaamú[boyong’ana]
                    tá}  &   
                     \gloss{‘go around’}  &  \\

                     \vernacular{
                    yaamú[ng’ong’oolitsa] tá}  &   
                     \gloss{‘tease’}  &     &   
                     \vernacular{
                    yaamú[lingakanyinya] tá}  &   
                     \gloss{‘bend’}  &  \\
\end{tabular}
%\caption{\nocaption}
     
\begin{tabular}{llllll}  
  \multicolumn{5}{l}{
                     \vernacular{(84) /H/
                    V-Initial + OP} \gloss{‘s/he did
                    not...him/her’} } &  \\
\multicolumn{5}{l}{ } &  \\

                     \vernacular{yaamw[íira]
                    tá}  &   
                     \gloss{‘kill’}  &     &   
                     \vernacular{
                    yaamw[í{\downstep}íkóómba] tá}  &   
                     \gloss{‘admire’}  &  \\

                     \vernacular{
                    yaamw[í{\downstep}ísíáka] tá}  &   
                     \gloss{‘smack’}  &     &   
                     \vernacular{yaamw[óononyinya]
                    tá}  &   
                     \gloss{‘spoil’}  &  \\

                     \vernacular{
                    yaamw[áabukhanyinya] tá}  &   
                     \gloss{‘separate’}  &     &     &     &  \\
\end{tabular}
%\caption{\nocaption}
     
\begin{tabular}{llllll}  
  \multicolumn{5}{l}{
                     \vernacular{(85) /Ø/
                    C-Initial + OP} \gloss{‘s/he did
                    not...him/her \ob mu-\cb  / them \ob ba-\cb ’} } &  \\
\multicolumn{5}{l}{ } &  \\

                     \vernacular{yaamú[{\downstep}tsíá]
                    {\downstep}tá}  &   
                     \gloss{‘go for’}  &  \\

                     \vernacular{yaamú[{\downstep}lékhá]
                    {\downstep}tá}  &   
                     \gloss{‘leave’}  &  \\

                     \vernacular{yaamú[{\downstep}lóónda]
                    tá}  &   
                     \gloss{‘follow’}  &  \\

                     \vernacular{yaamú[{\downstep}kúlíkha]
                    tá}  &   
                     \gloss{‘name’}  &  \\

                     \vernacular{
                    yaamú[{\downstep}lákhúula] tá}  &   
                     \gloss{‘release’}  &  \\

                     \vernacular{yaamú[{\downstep}séébula]
                    tá}  &   
                     \gloss{‘say bye to’}  &  \\

                     \vernacular{
                    yaamú[{\downstep}hóómbélitsa] tá}  &   
                     \gloss{‘comfort’}  &  \\

                     \vernacular{
                    yaamú[{\downstep}kálúshitsa] tá}  &   
                     \gloss{‘return’}  &  \\

                     \vernacular{
                    yaamú[{\downstep}síínjílitsa] tá}  &   
                     \gloss{
                    ‘make...stand’}  &  \\

                     \vernacular{
                    yaamú[{\downstep}réébáreeba] tá}  &   
                     \gloss{‘ask (iter)’}  &  \\

                     \vernacular{
                    yaamú[{\downstep}kálúkhányinya] tá}  &   
                     \gloss{
                    ‘turn...over’}  &  \\

                     \vernacular{
                    yaabá[{\downstep}sébúlúkhanyinya] tá}  &   
                     \gloss{‘scatter’}  &  \\
\end{tabular}
%\caption{\nocaption}
     
\begin{tabular}{llllll}  
  \multicolumn{5}{l}{
                     \vernacular{(86) /Ø/
                    V-Initial + OP} \gloss{‘s/he did
                    not...him/her \ob mw-\cb  / it
                    } } &  \\
\multicolumn{5}{l}{ } &  \\

                     \vernacular{yaamw[é{\downstep}ényá]
                    {\downstep}tá}  &   
                     \gloss{‘want’}  &     &   
                     \vernacular{yaamw[é{\downstep}éyéla]
                    tá}  &   
                     \gloss{‘wipe for’}  &  \\

                     \vernacular{yaabw[í{\downstep}ílúula]
                    tá}  &   
                     \gloss{‘winnow’}  &     &   
                     \vernacular{
                    yaamw[á{\downstep}ámbákhana] tá}  &   
                     \gloss{‘refuse’}  &  \\

                     \vernacular{
                    yaamw[é{\downstep}éléelitsa] tá}  &   
                     \gloss{‘hang...up’}  &  \\
\end{tabular}
%\caption{\nocaption}
     
\begin{tabular}{llllll}  
  \multicolumn{5}{l}{
                     \vernacular{(87) /H/
                    C-Initial + OP
                    } \gloss{‘s/he did
                    not...me’} } &  \\
\multicolumn{5}{l}{ } &  \\

                     \vernacular{yaá[ria]
                    tá}  &   
                     \gloss{‘fear’}  &     &   
                     \vernacular{yaá[mbeka]
                    tá}  &   
                     \gloss{‘shave’}  &  \\

                     \vernacular{yaá[ndeera]
                    tá}  &   
                     \gloss{‘bring’}  &     &   
                     \vernacular{yaá[khalaka]
                    tá}  &   
                     \gloss{‘cut’}  &  \\

                     \vernacular{yaá[sitaaka]
                    tá}  &   
                     \gloss{‘accuse’}  &     &   
                     \vernacular{yaá[mboolitsa]
                    tá}  &   
                     \gloss{‘seduce’}  &  \\

                     \vernacular{yaá[ndzuunzuuna]
                    tá}  &   
                     \gloss{‘suck’}  &     &   
                     \vernacular{yaá[mboholola]
                    tá}  &   
                     \gloss{‘untie’}  &  \\

                     \vernacular{yaá[mboyong’ana]
                    tá}  &   
                     \gloss{‘go around’}  &     &   
                     \vernacular{
                    yaá[ng’ong’oolitsa] tá}  &   
                     \gloss{‘tease’}  &  \\

                     \vernacular{
                    yaá[ningakanyinya] tá}  &   
                     \gloss{‘bend’}  &     &     &     &  \\
\end{tabular}
%\caption{\nocaption}
     
\begin{tabular}{llllll}  
  \multicolumn{5}{l}{
                     \vernacular{(88) /H/
                    V-Initial + OP
                    } \gloss{‘s/he did
                    not...me’} } &  \\
\multicolumn{5}{l}{ } &  \\

                     \vernacular{yaá[nzira]
                    tá}  &   
                     \gloss{‘kill’}  &     &   
                     \vernacular{yaa[nzí{\downstep}kóómba]
                    tá}  &   
                     \gloss{‘admire’}  &  \\

                     \vernacular{yaa[nzí{\downstep}síáka]
                    tá}  &   
                     \gloss{‘smack’}  &     &   
                     \vernacular{yaá[nzononyinya]
                    tá}  &   
                     \gloss{‘spoil’}  &  \\

                     \vernacular{
                    yaá[nzabukhanyinya] tá}  &   
                     \gloss{‘separate’}  &     &     &     &  \\
\end{tabular}
%\caption{\nocaption}
     
\begin{tabular}{llllll}  
  \multicolumn{5}{l}{
                     \vernacular{(89) /Ø/
                    C-Initial + OP
                    } \gloss{‘s/he did
                    not...me’} } &  \\
\multicolumn{5}{l}{ } &  \\

                     \vernacular{yaá[{\downstep}ndékhá]
                    {\downstep}tá}  &   
                     \gloss{‘leave’}  &     &   
                     \vernacular{yaá[{\downstep}nóónda]
                    tá}  &   
                     \gloss{‘follow’}  &  \\

                     \vernacular{yaá[{\downstep}ngúmíla]
                    tá}  &   
                     \gloss{‘hold’}  &     &   
                     \vernacular{yaá[{\downstep}ndákhúula]
                    tá}  &   
                     \gloss{‘release’}  &  \\

                     \vernacular{yaá[{\downstep}séébula]
                    tá}  &   
                     \gloss{‘say bye to’}  &     &   
                     \vernacular{
                    yaá[{\downstep}mbóómbélitsa] tá}  &   
                     \gloss{‘comfort’}  &  \\

                     \vernacular{
                    yaá[{\downstep}síínjílitsa] tá}  &   
                     \gloss{
                    ‘make...stand’}  &     &   
                     \vernacular{
                    yaá[{\downstep}ndéébándeeba] tá}  &   
                     \gloss{‘ask (iter)’}  &  \\

                     \vernacular{
                    yaá[{\downstep}ngálúkhányinya] tá}  &   
                     \gloss{
                    ‘turn...over’}  &  \\
\end{tabular}
%\caption{\nocaption}
     
\begin{tabular}{llllll}  
  \multicolumn{5}{l}{
                     \vernacular{(90) /Ø/
                    V-Initial + OP
                    } \gloss{‘s/he did
                    not...me’} } &  \\
\multicolumn{5}{l}{ } &  \\

                     \vernacular{yaá[{\downstep}nzényá]
                    {\downstep}tá}  &   
                     \gloss{‘want’}  &     &   
                     \vernacular{yaá[{\downstep}nzéyéla]
                    tá}  &   
                     \gloss{‘wipe for’}  &  \\

                     \vernacular{
                    yaá[{\downstep}nyámbákhana] tá}  &   
                     \gloss{‘refuse’}  &     &   
                     \vernacular{
                    yaá[{\downstep}nzéléelitsa] tá}  &   
                     \gloss{
                    ‘carry...hanging’}  &  \\
\end{tabular}
%\caption{\nocaption}
     
\begin{tabular}{llllll}  
  \multicolumn{5}{l}{
                     \vernacular{(91) /H/
                    C-Initial + OP
                    } \gloss{‘s/he did
                    not...him/herself’} } &  \\
\multicolumn{5}{l}{ } &  \\

                     \vernacular{yií[ra]
                    tá}  &   
                     \gloss{‘bury’}  &     &   
                     \vernacular{yií[beka]
                    tá}  &   
                     \gloss{‘shave’}  &  \\

                     \vernacular{yií[suunga]
                    tá}  &   
                     \gloss{‘hang’}  &     &   
                     \vernacular{yií[khalaka]
                    tá}  &   
                     \gloss{‘cut’}  &  \\

                     \vernacular{yií[sitaaka]
                    tá}  &   
                     \gloss{‘accuse’}  &     &   
                     \vernacular{yií[saanditsa]
                    tá}  &   
                     \gloss{‘thank’}  &  \\

                     \vernacular{yií[tsuunzuuna]
                    tá}  &   
                     \gloss{‘suck’}  &     &   
                     \vernacular{yií[boholola]
                    tá}  &   
                     \gloss{‘untie’}  &  \\
\end{tabular}
%\caption{\nocaption}
     
\begin{tabular}{llllll}  
  \multicolumn{5}{l}{
                     \vernacular{(92) /H/
                    V-Initial + OP
                    } \gloss{‘s/he did
                    not...him/herself’} } &  \\
\multicolumn{5}{l}{ } &  \\

                     \vernacular{yií[yira]
                    tá}  &   
                     \gloss{‘kill’}  &     &   
                     \vernacular{yií[{\downstep}yíkóómba]
                    tá}  &   
                     \gloss{‘admire’}  &  \\

                     \vernacular{yií[{\downstep}yísíáka]
                    tá}  &   
                     \gloss{‘smack’}  &     &   
                     \vernacular{yií[yononyinya]
                    tá}  &   
                     \gloss{‘spoil’}  &  \\

                     \vernacular{
                    yií[yabukhanyinya] tá}  &   
                     \gloss{‘separate’}  &     &     &     &  \\
\end{tabular}
%\caption{\nocaption}
     
\begin{tabular}{llllll}  
  \multicolumn{5}{l}{
                     \vernacular{(93) /Ø/
                    C-Initial + OP
                    } \gloss{‘s/he did
                    not...him/herself’} } &  \\
\multicolumn{5}{l}{ } &  \\

                     \vernacular{yií[{\downstep}lékhá]
                    {\downstep}tá}  &   
                     \gloss{‘leave’}  &     &   
                     \vernacular{yií[{\downstep}síínga]
                    tá}  &   
                     \gloss{‘bathe’}  &  \\

                     \vernacular{yií[{\downstep}kúlíkha]
                    tá}  &   
                     \gloss{‘name’}  &     &   
                     \vernacular{yií[{\downstep}náábula]
                    tá}  &   
                     \gloss{‘undress’}  &  \\

                     \vernacular{yií[{\downstep}lákhúula]
                    tá}  &   
                     \gloss{‘release’}  &     &   
                     \vernacular{
                    yií[{\downstep}hóómbélitsa] tá}  &   
                     \gloss{‘comfort’}  &  \\

                     \vernacular{
                    yií[{\downstep}síínjílitsa] tá}  &   
                     \gloss{
                    ‘make...stand’}  &     &   
                     \vernacular{
                    yií[{\downstep}réébáreeba] tá}  &   
                     \gloss{‘ask (iter)’}  &  \\

                     \vernacular{
                    yií[{\downstep}kálúkhányinya] tá}  &   
                     \gloss{
                    ‘turn...over’}  &  \\
\end{tabular}
%\caption{\nocaption}
     
\begin{tabular}{llllll}  
  \multicolumn{5}{l}{
                     \vernacular{(94) /Ø/
                    V-Initial + OP
                    } \gloss{‘s/he did
                    not...him/herself’} } &  \\
\multicolumn{5}{l}{ } &  \\

                     \vernacular{yií[{\downstep}yéyá]
                    {\downstep}tá}  &   
                     \gloss{‘wipe’}  &     &   
                     \vernacular{yií[{\downstep}yéyéla]
                    {\downstep}tá}  &   
                     \gloss{‘wipe for’}  &  \\

                     \vernacular{
                    yií[{\downstep}yámbákhana] tá}  &   
                     \gloss{‘despise’}  &     &   
                     \vernacular{
                    yií[{\downstep}yéléelitsa] tá}  &   
                     \gloss{‘hang’}  &  \\
\end{tabular}
%\caption{\nocaption}
     
\begin{tabular}{llllll}  
  \multicolumn{5}{l}{
                     \vernacular{(95) /H/
                    C-Initial + OP + OP
                    } \gloss{‘s/he did
                    not...him/her for me’} } &  \\
\multicolumn{5}{l}{ } &  \\

                     \vernacular{yaamúu[ndeela]
                    tá}  &   
                     \gloss{‘bury’}  &     &   
                     \vernacular{yaamúu[mbechela]
                    tá}  &   
                     \gloss{‘shave’}  &  \\

                     \vernacular{yaamúu[ndeerela]
                    tá}  &   
                     \gloss{‘bring’}  &     &   
                     \vernacular{
                    yaamúu[khalachila] tá}  &   
                     \gloss{‘cut’}  &  \\

                     \vernacular{
                    yaamúu[sitaachila] tá}  &   
                     \gloss{‘accuse’}  &     &   
                     \vernacular{
                    yaamúu[mboolitsila] tá}  &   
                     \gloss{‘seduce’}  &  \\

                     \vernacular{
                    yaamúu[mbohololela] tá}  &   
                     \gloss{‘untie’}  &     &     &     &  \\
\end{tabular}
%\caption{\nocaption}
     
\begin{tabular}{llllll}  
  \multicolumn{5}{l}{
                     \vernacular{(96) /H/
                    V-Initial + OP + OP
                    } \gloss{‘s/he did
                    not...him/her for me’} } &  \\
\multicolumn{5}{l}{ } &  \\

                     \vernacular{yaamúu[nzirila]
                    tá}  &   
                     \gloss{‘kill’}  &  \\

                     \vernacular{
                    yaamúu[nzechitsila] tá}  &   
                     \gloss{‘admire’}  &  \\

                     \vernacular{
                    yaamú{\downstep}ú[nzísíáchila] tá}  &   
                     \gloss{‘smack’}  &  \\

                     \vernacular{
                    yaamúu[nzononyinyila] tá}  &   
                     \gloss{‘spoil’}  &  \\

                     \vernacular{
                    yaamúu[nzabukhanyinyila] tá}  &   
                     \gloss{‘separate’}  &  \\
\end{tabular}
%\caption{\nocaption}
     
\begin{tabular}{llllll}  
  \multicolumn{5}{l}{
                     \vernacular{(97) /Ø/
                    C-Initial + OP + OP
                    } \gloss{‘s/he did
                    not...him/her for me’} } &  \\
\multicolumn{5}{l}{ } &  \\

                     \vernacular{
                    yaamú{\downstep}ú[nzííla] tá}  &   
                     \gloss{‘go for’}  &  \\

                     \vernacular{
                    yaamú{\downstep}ú[ndéshéla] tá}  &   
                     \gloss{‘leave’}  &  \\

                     \vernacular{
                    yaamú{\downstep}ú[nóóndéla] tá}  &   
                     \gloss{‘follow’}  &  \\

                     \vernacular{
                    yaamú{\downstep}ú[ngúlíshíla] tá}  &   
                     \gloss{‘name’}  &  \\

                     \vernacular{
                    yaamú{\downstep}ú[ndákhúulila] tá}  &   
                     \gloss{‘release’}  &  \\

                     \vernacular{
                    yaamú{\downstep}ú[séébúlila] tá}  &   
                     \gloss{‘say bye to’}  &  \\

                     \vernacular{
                    yaamú{\downstep}ú[mbóómbélitsila] tá}  &   
                     \gloss{‘comfort’}  &  \\

                     \vernacular{
                    yaamú{\downstep}ú[síínjílitsila] tá}  &   
                     \gloss{
                    ‘make...stand’}  &  \\
\end{tabular}
%\caption{\nocaption}
     
\begin{tabular}{llllll}  
  \multicolumn{5}{l}{
                     \vernacular{(98) /Ø/
                    V-Initial + OP + OP
                    } \gloss{‘s/he did
                    not...him/her \ob mu-\cb  / it
                    } } &  \\
\multicolumn{5}{l}{ } &  \\

                     \vernacular{
                    yaabú{\downstep}ú[nzálíla] tá}  &   
                     \gloss{‘display’}  &     &   
                     \vernacular{
                    yaakú{\downstep}ú[nzáshítsíla] tá}  &   
                     \gloss{‘light’}  &  \\

                     \vernacular{
                    yaabú{\downstep}ú[nzílúulila] tá}  &   
                     \gloss{‘winnow’}  &     &   
                     \vernacular{
                    yaalú{\downstep}ú[nzítsúlítsila] tá}  &   
                     \gloss{‘fill’}  &  \\

                     \vernacular{
                    yaakú{\downstep}ú[nzéléelitsila] tá}  &   
                     \gloss{‘hang’}  &     &     &     &  \\
\end{tabular}
%\caption{\nocaption}
     
\begin{tabular}{lll}  
  \multicolumn{2}{l}{
                     \vernacular{(99) /H/
                    C-Initial Phrase-Medial} \gloss{‘s/he did
                    not...the man \ob musáatsa\cb  /} } &  \\
\multicolumn{2}{l}{
                     \gloss{the boy
                    \ob mú{\downstep}yáyi\cb  / someone \ob muundu\cb ’} } &  \\

                     \vernacular{yaa[rá]
                    musáatsa/mú{\downstep}yáyi/muundu tá}  &   
                     \gloss{‘bury’}  &  \\

                     \vernacular{yaa[béka]
                    musáatsa/mú{\downstep}yáyi/muundu tá}  &   
                     \gloss{‘shave’}  &  \\

                     \vernacular{yaa[léera]
                    musáatsa/mú{\downstep}yáyi/muundu tá}  &   
                     \gloss{‘bring’}  &  \\

                     \vernacular{yaa[khálaka]
                    musáatsa/mú{\downstep}yáyi/muundu tá}  &   
                     \gloss{‘cut’}  &  \\

                     \vernacular{yaa[sítaaka]
                    musáatsa/mú{\downstep}yáyi/muundu tá}  &   
                     \gloss{‘accuse’}  &  \\

                     \vernacular{yaa[bóolitsa]
                    musáatsa/mú{\downstep}yáyi/muundu tá}  &   
                     \gloss{‘seduce’}  &  \\

                     \vernacular{yaa[tsúunzuuna]
                    musáatsa/mú{\downstep}yáyi/muundu tá}  &   
                     \gloss{‘suck’}  &  \\

                     \vernacular{yaa[bóholola]
                    musáatsa/mú{\downstep}yáyi/muundu tá}  &   
                     \gloss{‘untie’}  &  \\

                     \vernacular{yaa[bóyong’ana]
                    musáatsa/mú{\downstep}yáyi/muundu tá}  &   
                     \gloss{‘go around’}  &  \\

                     \vernacular{
                    yaa[ng’óng’oolitsa] musáatsa/mú{\downstep}yáyi/muundu
                    tá}  &   
                     \gloss{‘tease’}  &  \\
\end{tabular}
%\caption{\nocaption}
     
\begin{tabular}{lll}  
  \multicolumn{2}{l}{
                     \vernacular{(100) /Ø/
                    C-Initial Phrase-Medial} \gloss{‘s/he did
                    not...the man \ob musáatsa\cb  /} } &  \\
\multicolumn{2}{l}{
                     \gloss{the boy
                    \ob mú{\downstep}yáyi\cb  / someone \ob muundu\cb ’} } &  \\

                     \vernacular{yaa[tsía]
                    musáatsa/mú{\downstep}yáyi/muundu tá}  &   
                     \gloss{‘go for’}  &  \\

                     \vernacular{yaa[lékha]
                    musáatsa/mú{\downstep}yáyi/muundu tá}  &   
                     \gloss{‘leave’}  &  \\

                     \vernacular{yaa[lóonda]
                    musáatsa/mú{\downstep}yáyi/muundu tá}  &   
                     \gloss{‘follow’}  &  \\

                     \vernacular{yaa[kúlikha]
                    musáatsa/mú{\downstep}yáyi/muundu tá}  &   
                     \gloss{‘name’}  &  \\

                     \vernacular{yaa[lákhuula]
                    musáatsa/mú{\downstep}yáyi/muundu tá}  &   
                     \gloss{‘release’}  &  \\

                     \vernacular{yaa[séebula]
                    musáatsa/mú{\downstep}yáyi/muundu tá}  &   
                     \gloss{‘say bye to’}  &  \\

                     \vernacular{yaa[kálushitsa]
                    musáatsa/mú{\downstep}yáyi/muundu tá}  &   
                     \gloss{‘return’}  &  \\

                     \vernacular{yaa[síinjilitsa]
                    musáatsa/mú{\downstep}yáyi/muundu tá}  &   
                     \gloss{
                    ‘make...stand’}  &  \\

                     \vernacular{yaa[réebareeba]
                    musáatsa/mú{\downstep}yáyi/muundu tá}  &   
                     \gloss{‘ask (iter)’}  &  \\

                     \vernacular{
                    yaa[kálukhanyinya] musáatsa/mú{\downstep}yáyi/muundu
                    tá}  &   
                     \gloss{
                    ‘turn...over’}  &  \\
\end{tabular}
%\caption{\nocaption}
     
\begin{tabular}{lll}  
  \multicolumn{2}{l}{
                     \vernacular{(101) /H/
                    C-Initial +OP Phrase-Medial} \gloss{‘s/he did
                    not...the man \ob musáatsa\cb  /} } &  \\
\multicolumn{2}{l}{
                     \gloss{the boy
                    \ob mú{\downstep}yáyi\cb  / someone \ob muundu\cb  for
                    him/her’} } &  \\

                     \vernacular{yaamú[reela]
                    musáatsa/mú{\downstep}yáyi/muundu tá}  &   
                     \gloss{‘bury’}  &  \\

                     \vernacular{yaamú[bechela]
                    musáatsa/mú{\downstep}yáyi/muundu tá}  &   
                     \gloss{‘shave’}  &  \\

                     \vernacular{yaamú[leerela]
                    musáatsa/mú{\downstep}yáyi/muundu tá}  &   
                     \gloss{‘bring’}  &  \\

                     \vernacular{yaamú[khalachila]
                    musáatsa/mú{\downstep}yáyi/muundu tá}  &   
                     \gloss{‘cut’}  &  \\

                     \vernacular{yaamú[sitaachila]
                    musáatsa/mú{\downstep}yáyi/muundu tá}  &   
                     \gloss{‘accuse’}  &  \\

                     \vernacular{yaamú[boolitsila]
                    musáatsa/mú{\downstep}yáyi/muundu tá}  &   
                     \gloss{‘seduce’}  &  \\

                     \vernacular{
                    yaamú[tsuunzuunila] musáatsa/mú{\downstep}yáyi/muundu
                    tá}  &   
                     \gloss{‘suck’}  &  \\

                     \vernacular{yaamú[bohololela]
                    musáatsa/mú{\downstep}yáyi/muundu tá}  &   
                     \gloss{‘untie’}  &  \\

                     \vernacular{
                    yaamú[boyong’anila] musáatsa/mú{\downstep}yáyi/muundu
                    tá}  &   
                     \gloss{‘go around’}  &  \\

                     \vernacular{
                    yaamú[ng’ong’oolitsila]
                    musáatsa/mú{\downstep}yáyi/muundu tá}  &   
                     \gloss{‘tease’}  &  \\
\end{tabular}
%\caption{\nocaption}
     
\begin{tabular}{lll}  
  \multicolumn{2}{l}{
                     \vernacular{(102) /Ø/
                    C-Initial +OP Phrase-Medial} \gloss{‘s/he did
                    not...the man \ob musáatsa\cb  /} } &  \\
\multicolumn{2}{l}{
                     \gloss{the boy
                    \ob mú{\downstep}yáyi\cb  / someone \ob muundu\cb  for
                    him/her’} } &  \\

                     \vernacular{yaamú[{\downstep}tsííla]
                    musáatsa/mú{\downstep}yáyi/muundu tá}  &   
                     \gloss{‘go for’}  &  \\

                     \vernacular{yaamú[{\downstep}léshéla]
                    musáatsa/mú{\downstep}yáyi/muundu tá}  &   
                     \gloss{‘leave’}  &  \\

                     \vernacular{
                    yaamú[{\downstep}lóóndéla] musáatsa/mú{\downstep}yáyi/muundu
                    tá}  &   
                     \gloss{‘follow’}  &  \\

                     \vernacular{
                    yaamú[{\downstep}kúlíshíla]
                    musáatsa/mú{\downstep}yáyi/muundu tá}  &   
                     \gloss{‘name’}  &  \\

                     \vernacular{
                    yaamú[{\downstep}lákhúulila]
                    musáatsa/mú{\downstep}yáyi/muundu tá}  &   
                     \gloss{‘release’}  &  \\

                     \vernacular{
                    yaamú[{\downstep}séébúlila]
                    musáatsa/mú{\downstep}yáyi/muundu tá}  &   
                     \gloss{‘say bye to’}  &  \\

                     \vernacular{
                    yaamú[{\downstep}réébáreebela]
                    musáatsa/mú{\downstep}yáyi/muundu tá}  &   
                     \gloss{‘ask (iter)’}  &  \\
\end{tabular}
%\caption{\nocaption}
     
\begin{tabular}{lll}  
  \multicolumn{2}{l}{
                     \vernacular{(103) /H/
                    C-Initial +OP + OP
                    } \gloss{‘s/he did
                    not...the man \ob musáatsa\cb  /} } &  \\
\multicolumn{2}{l}{
                     \gloss{the boy
                    \ob mú{\downstep}yáyi\cb  / someone \ob muundu\cb  for him/her for
                    me’} } &  \\

                     \vernacular{yaamúu[ndeela]
                    musáatsa/mú{\downstep}yáyi/muundu tá}  &   
                     \gloss{‘bury’}  &  \\

                     \vernacular{yaamúu[mbechela]
                    musáatsa/mú{\downstep}yáyi/muundu tá}  &   
                     \gloss{‘shave’}  &  \\

                     \vernacular{yaamúu[ndeerela]
                    musáatsa/mú{\downstep}yáyi/muundu tá}  &   
                     \gloss{‘bring’}  &  \\

                     \vernacular{
                    yaamúu[khalachila] musáatsa/mú{\downstep}yáyi/muundu
                    tá}  &   
                     \gloss{‘cut’}  &  \\

                     \vernacular{
                    yaamúu[sitaachila] musáatsa/mú{\downstep}yáyi/muundu
                    tá}  &   
                     \gloss{‘accuse’}  &  \\

                     \vernacular{
                    yaamúu[mboolitsila] musáatsa/mú{\downstep}yáyi/muundu
                    tá}  &   
                     \gloss{‘seduce’}  &  \\

                     \vernacular{
                    yaamúu[mbohololela] musáatsa/mú{\downstep}yáyi/muundu
                    tá}  &   
                     \gloss{‘untie’}  &  \\
\end{tabular}
%\caption{\nocaption}
     
\begin{tabular}{lll}  
  \multicolumn{2}{l}{
                     \vernacular{(104) /Ø/
                    C-Initial +OP + OP
                    } \gloss{‘s/he did
                    not...the man \ob musáatsa\cb  /} } &  \\
\multicolumn{2}{l}{
                     \gloss{the boy
                    \ob mú{\downstep}yáyi\cb  / someone \ob muundu\cb  for him/her for
                    me’} } &  \\

                     \vernacular{
                    yaamú{\downstep}ú[nzííla] musáatsa/mú{\downstep}yáyi/muundu
                    tá}  &   
                     \gloss{‘go for’}  &  \\

                     \vernacular{
                    yaamú{\downstep}ú[ndéshéla]
                    musáatsa/mú{\downstep}yáyi/muundu tá}  &   
                     \gloss{‘leave’}  &  \\

                     \vernacular{
                    yaamú{\downstep}ú[nóóndéla]
                    musáatsa/mú{\downstep}yáyi/muundu tá}  &   
                     \gloss{‘follow’}  &  \\

                     \vernacular{
                    yaamú{\downstep}ú[ngúlíshíla]
                    musáatsa/mú{\downstep}yáyi/muundu tá}  &   
                     \gloss{‘name’}  &  \\

                     \vernacular{
                    yaamú{\downstep}ú[ndákhúulila]
                    musáatsa/mú{\downstep}yáyi/muundu tá}  &   
                     \gloss{‘release’}  &  \\

                     \vernacular{
                    yaamú{\downstep}ú[séébúlila]
                    musáatsa/mú{\downstep}yáyi/muundu tá}  &   
                     \gloss{‘say bye to’}  &  \\

                     \vernacular{
                    yaamú{\downstep}ú[mbóómbélitsila]
                    musáatsa/mú{\downstep}yáyi/muundu tá}  &   
                     \gloss{‘comfort’}  &  \\

                     \vernacular{
                    yaamú{\downstep}ú[síínjílitsila]
                    musáatsa/mú{\downstep}yáyi/muundu tá}  &   
                     \gloss{
                    ‘make...stand’}  &  \\
\end{tabular}
%\caption{\nocaption}
    

\subsection{Immediate Past: Pattern 1b}\label{sec:sImmPast}


\begin{tabular}{llllll}  
  \multicolumn{5}{l}{
                     \vernacular{(105) /H/
                    C-Initial} \gloss{‘s/he
                    just...’} } &  \\
\multicolumn{5}{l}{ } &  \\

                     \vernacular{
                    yá{\downstep}khá[rá]}  &   
                     \gloss{‘buried’}  &     &   
                     \vernacular{
                    yá{\downstep}khá[ng’wá]}  &   
                     \gloss{‘drank’}  &  \\

                     \vernacular{
                    yá{\downstep}khá[khwá]}  &   
                     \gloss{‘paid dowry’}  &     &   
                     \vernacular{
                    yá{\downstep}khá[lía]}  &   
                     \gloss{‘ate’}  &  \\

                     \vernacular{
                    yá{\downstep}khá[lúma]}  &   
                     \gloss{‘bit’}  &     &   
                     \vernacular{
                    yá{\downstep}khá[béka]}  &   
                     \gloss{‘shaved’}  &  \\

                     \vernacular{
                    yá{\downstep}khá[téekha]}  &   
                     \gloss{‘cooked’}  &     &   
                     \vernacular{
                    yá{\downstep}khá[léera]}  &   
                     \gloss{‘brought’}  &  \\

                     \vernacular{
                    yá{\downstep}khá[khálaka]}  &   
                     \gloss{‘cut’}  &     &   
                     \vernacular{
                    yá{\downstep}khá[kálaanga]}  &   
                     \gloss{‘fried’}  &  \\

                     \vernacular{
                    yá{\downstep}khá[sítaaka]}  &   
                     \gloss{‘accused’}  &     &   
                     \vernacular{
                    yá{\downstep}khá[bóolitsa]}  &   
                     \gloss{‘seduced’}  &  \\

                     \vernacular{
                    yá{\downstep}khá[sáanditsa]}  &   
                     \gloss{‘thanked’}  &     &   
                     \vernacular{
                    yá{\downstep}khá[tsúunzuuna]}  &   
                     \gloss{‘sucked’}  &  \\

                     \vernacular{
                    yá{\downstep}khá[bóholola]}  &   
                     \gloss{‘untied’}  &     &   
                     \vernacular{
                    yá{\downstep}khá[bóyong’ana]}  &   
                     \gloss{‘went
                    around’}  &  \\

                     \vernacular{
                    yá{\downstep}khá[ng’óng’oolitsa]}  &   
                     \gloss{‘teased’}  &     &   
                     \vernacular{
                    yá{\downstep}khá[língakanyinya]}  &   
                     \gloss{‘crumpled’}  &  \\
\end{tabular}
%\caption{\nocaption}
     
\begin{tabular}{llllll}  
  \multicolumn{5}{l}{
                     \vernacular{(106) /H/
                    V-Initial} \gloss{‘s/he
                    just...’} } &  \\
\multicolumn{5}{l}{ } &  \\

                     \vernacular{
                    yá{\downstep}kh[ííra]}  &   
                     \gloss{‘killed’}  &     &   
                     \vernacular{
                    yá{\downstep}kh[ííkoomba]}  &   
                     \gloss{‘admired’}  &  \\

                     \vernacular{
                    yá{\downstep}kh[íísiaka]}  &   
                     \gloss{‘smacked’}  &     &   
                     \vernacular{
                    yá{\downstep}kh[ííkobola]}  &   
                     \gloss{‘belched’}  &  \\

                     \vernacular{
                    yá{\downstep}kh[óónonyinya]}  &   
                     \gloss{‘spoiled’}  &     &   
                     \vernacular{
                    yá{\downstep}kh[áábukhanyinya]}  &   
                     \gloss{‘separated’}  &  \\
\end{tabular}
%\caption{\nocaption}
     
\begin{tabular}{llllll}  
  \multicolumn{5}{l}{
                     \vernacular{(107) /Ø/
                    C-Initial} \gloss{‘s/he
                    just...’} } &  \\
\multicolumn{5}{l}{ } &  \\

                     \vernacular{
                    yákha[tsia]}  &   
                     \gloss{‘went’}  &     &   
                     \vernacular{
                    yákha[kwa]}  &   
                     \gloss{‘fell’}  &  \\

                     \vernacular{
                    yákha[lekha]}  &   
                     \gloss{‘left’}  &     &   
                     \vernacular{
                    yákha[reeba]}  &   
                     \gloss{‘asked’}  &  \\

                     \vernacular{
                    yákha[loonda]}  &   
                     \gloss{‘followed’}  &     &   
                     \vernacular{
                    yákha[kulikha]}  &   
                     \gloss{‘named’}  &  \\

                     \vernacular{
                    yákha[homoola]}  &   
                     \gloss{‘massaged’}  &     &   
                     \vernacular{
                    yákha[lakhuula]}  &   
                     \gloss{‘released’}  &  \\

                     \vernacular{
                    yákha[seebula]}  &   
                     \gloss{‘said bye’}  &     &   
                     \vernacular{
                    yákha[hoombelitsa]}  &   
                     \gloss{‘comforted’}  &  \\

                     \vernacular{
                    yákha[kalushitsa]}  &   
                     \gloss{‘returned’}  &     &   
                     \vernacular{
                    yákha[siinjilitsa]}  &   
                     \gloss{‘made stand’}  &  \\

                     \vernacular{
                    yákha[reebareeba]}  &   
                     \gloss{‘asked
                    (iter)’}  &     &   
                     \vernacular{
                    yákha[kalukhanyinya]}  &   
                     \gloss{‘turned
                    over’}  &  \\

                     \vernacular{
                    yákha[sebulukhanyinya]}  &   
                     \gloss{‘scattered’}  &     &     &     &  \\
\end{tabular}
%\caption{\nocaption}
     
\begin{tabular}{llllll}  
  \multicolumn{5}{l}{
                     \vernacular{(108) /Ø/
                    V-Initial} \gloss{‘s/he
                    just...’} } &  \\
\multicolumn{5}{l}{ } &  \\

                     \vernacular{
                    yákh[eenya]}  &   
                     \gloss{‘wanted’}  &     &   
                     \vernacular{
                    yákh[aashitsa]}  &   
                     \gloss{‘lit’}  &  \\

                     \vernacular{
                    yákh[iiluula]}  &   
                     \gloss{‘winnowed’}  &     &   
                     \vernacular{
                    yákh[aambakhana]}  &   
                     \gloss{‘refused’}  &  \\

                     \vernacular{
                    yákh[eeleelitsa]}  &   
                     \gloss{‘hung up’}  &     &     &     &  \\
\end{tabular}
%\caption{\nocaption}
     
\begin{tabular}{llllll}  
  \multicolumn{5}{l}{
                     \vernacular{(109) /H/
                    C-Initial + OP} \gloss{‘s/he
                    just...him/her’} } &  \\
\multicolumn{5}{l}{ } &  \\

                     \vernacular{
                    yá{\downstep}khámú[ra]}  &   
                     \gloss{‘buried’}  &     &   
                     \vernacular{
                    yá{\downstep}khámú[beka]}  &   
                     \gloss{‘shaved’}  &  \\

                     \vernacular{
                    yá{\downstep}khámú[leera]}  &   
                     \gloss{‘brought’}  &     &   
                     \vernacular{
                    yá{\downstep}khámú[khalaka]}  &   
                     \gloss{‘cut’}  &  \\

                     \vernacular{
                    yá{\downstep}khámú[sitaaka]}  &   
                     \gloss{‘accuse’}  &     &   
                     \vernacular{
                    yá{\downstep}khámú[boolitsa]}  &   
                     \gloss{‘seduced’}  &  \\

                     \vernacular{
                    yá{\downstep}khámú[tsuunzuuna]}  &   
                     \gloss{‘sucked’}  &     &   
                     \vernacular{
                    yá{\downstep}khámú[boholola]}  &   
                     \gloss{‘untied’}  &  \\

                     \vernacular{
                    yá{\downstep}khámú[bukaanila]}  &   
                     \gloss{‘met’}  &     &   
                     \vernacular{
                    yá{\downstep}khámú[ng’ong’oolitsa]}  &   
                     \gloss{‘teased’}  &  \\

                     \vernacular{
                    yá{\downstep}khámú[lingakanyinya]}  &   
                     \gloss{‘bent’}  &     &     &     &  \\
\end{tabular}
%\caption{\nocaption}
     
\begin{tabular}{llllll}  
  \multicolumn{5}{l}{
                     \vernacular{(110) /H/
                    V-Initial + OP} \gloss{‘s/he
                    just...him/her’} } &  \\
\multicolumn{5}{l}{ } &  \\

                     \vernacular{
                    yá{\downstep}khámw[íira]}  &   
                     \gloss{‘killed’}  &     &   
                     \vernacular{
                    yá{\downstep}khámw[íikoomba]}  &   
                     \gloss{‘admired’}  &  \\

                     \vernacular{
                    yá{\downstep}khámw[íisiaka]}  &   
                     \gloss{‘smacked’}  &     &   
                     \vernacular{
                    yá{\downstep}khámw[óononyinya]}  &   
                     \gloss{‘spoiled’}  &  \\

                     \vernacular{
                    yá{\downstep}khámw[áabukhanyinya]}  &   
                     \gloss{‘separated’}  &     &     &     &  \\
\end{tabular}
%\caption{\nocaption}
     
\begin{tabular}{llllll}  
  \multicolumn{5}{l}{
                     \vernacular{(111) /Ø/
                    C-Initial + OP} \gloss{‘s/he
                    just...him/her \ob mu-\cb  / them
                    } } &  \\
\multicolumn{5}{l}{ } &  \\

                     \vernacular{
                    yá{\downstep}khámú[tsia]}  &   
                     \gloss{‘went for’}  &  \\

                     \vernacular{
                    yá{\downstep}khámú[lekha]}  &   
                     \gloss{‘left’}  &  \\

                     \vernacular{
                    yá{\downstep}khámú[loonda]}  &   
                     \gloss{‘followed’}  &  \\

                     \vernacular{
                    yá{\downstep}khámú[kulikha]}  &   
                     \gloss{‘named’}  &  \\

                     \vernacular{
                    yá{\downstep}khámú[lakhuula]}  &   
                     \gloss{‘released’}  &  \\

                     \vernacular{
                    yá{\downstep}khámú[seebula]}  &   
                     \gloss{‘said bye
                    to’}  &  \\

                     \vernacular{
                    yá{\downstep}khámú[hoombelitsa]}  &   
                     \gloss{‘comforted’}  &  \\

                     \vernacular{
                    yá{\downstep}khámú[kalushitsa]}  &   
                     \gloss{‘returned’}  &  \\

                     \vernacular{
                    yá{\downstep}khámú[siinjilitsa]}  &   
                     \gloss{
                    ‘made...stand’}  &  \\

                     \vernacular{
                    yá{\downstep}khámú[reebareeba]}  &   
                     \gloss{‘asked
                    (iter)’}  &  \\

                     \vernacular{
                    yá{\downstep}khámú[kalukhanyinya]}  &   
                     \gloss{
                    ‘turned...over’}  &  \\

                     \vernacular{
                    yá{\downstep}khátsí[sebulukhanyinya]}  &   
                     \gloss{‘scattered’}  &  \\

                     \vernacular{
                    yá{\downstep}khábá[sebulukhanyinya]}  &   
                     \gloss{‘scattered’}  &  \\
\end{tabular}
%\caption{\nocaption}
     
\begin{tabular}{llllll}  
  \multicolumn{5}{l}{
                     \vernacular{(112) /Ø/
                    V-Initial + OP} \gloss{‘s/he
                    just...him/her \ob mw-\cb  / it
                    } } &  \\
\multicolumn{5}{l}{ } &  \\

                     \vernacular{
                    yá{\downstep}khámw[éenya]}  &   
                     \gloss{‘wanted’}  &     &   
                     \vernacular{
                    yá{\downstep}khákw[áashitsa]}  &   
                     \gloss{‘lit’}  &  \\

                     \vernacular{
                    yá{\downstep}khábw[íiluula]}  &   
                     \gloss{‘winnowed’}  &     &   
                     \vernacular{
                    yá{\downstep}khámw[áambakhana]}  &   
                     \gloss{‘refused’}  &  \\

                     \vernacular{
                    yá{\downstep}khámw[éeleelitsa]}  &   
                     \gloss{‘hung...up’}  &  \\
\end{tabular}
%\caption{\nocaption}
     
\begin{tabular}{llllll}  
  \multicolumn{5}{l}{
                     \vernacular{(113) /H/
                    C-Initial + OP
                    } \gloss{‘s/he
                    just...me’} } &  \\
\multicolumn{5}{l}{ } &  \\

                     \vernacular{
                    yá{\downstep}kháá[ria]}  &   
                     \gloss{‘feared’}  &     &   
                     \vernacular{
                    yá{\downstep}kháá[mbeka]}  &   
                     \gloss{‘shaved’}  &  \\

                     \vernacular{
                    yá{\downstep}kháá[ndeera]}  &   
                     \gloss{‘brought’}  &     &   
                     \vernacular{
                    yá{\downstep}kháá[khalaka]}  &   
                     \gloss{‘cut’}  &  \\

                     \vernacular{
                    yá{\downstep}kháá[sitaaka]}  &   
                     \gloss{‘accused’}  &     &   
                     \vernacular{
                    yá{\downstep}kháá[mboolitsa]}  &   
                     \gloss{‘seduced’}  &  \\

                     \vernacular{
                    yá{\downstep}kháá[ndzuunzuuna]}  &   
                     \gloss{‘sucked’}  &     &   
                     \vernacular{
                    yá{\downstep}kháá[mboholola]}  &   
                     \gloss{‘untied’}  &  \\

                     \vernacular{
                    yá{\downstep}kháá[mboyong’ana]}  &   
                     \gloss{‘went
                    around’}  &     &   
                     \vernacular{
                    yá{\downstep}kháá[ng’ong’oolitsa]}  &   
                     \gloss{‘teased’}  &  \\

                     \vernacular{
                    yá{\downstep}kháá[ningakanyinya]}  &   
                     \gloss{‘bent’}  &     &     &     &  \\
\end{tabular}
%\caption{\nocaption}
     
\begin{tabular}{llllll}  
  \multicolumn{5}{l}{
                     \vernacular{(114) /H/
                    V-Initial + OP
                    } \gloss{‘s/he
                    just...me’} } &  \\
\multicolumn{5}{l}{ } &  \\

                     \vernacular{
                    yá{\downstep}kháá[nzira]}  &   
                     \gloss{‘killed’}  &     &   
                     \vernacular{
                    yá{\downstep}kháá[nzikoomba]}  &   
                     \gloss{‘admired’}  &  \\

                     \vernacular{
                    yá{\downstep}kháá[nzisiaka]}  &   
                     \gloss{‘smacked’}  &     &   
                     \vernacular{
                    yá{\downstep}kháá[nzononyinya]}  &   
                     \gloss{‘spoiled’}  &  \\

                     \vernacular{
                    yá{\downstep}kháá[nzabukhanyinya]}  &   
                     \gloss{‘separated’}  &     &     &     &  \\
\end{tabular}
%\caption{\nocaption}
     
\begin{tabular}{llllll}  
  \multicolumn{5}{l}{
                     \vernacular{(115) /Ø/
                    C-Initial + OP
                    } \gloss{‘s/he
                    just...me’} } &  \\
\multicolumn{5}{l}{ } &  \\

                     \vernacular{
                    yá{\downstep}kháá[ndekha]}  &   
                     \gloss{‘left’}  &  \\

                     \vernacular{
                    yá{\downstep}kháá[noonda]}  &   
                     \gloss{‘followed’}  &  \\

                     \vernacular{
                    yá{\downstep}kháá[ngulikha]}  &   
                     \gloss{‘named’}  &  \\

                     \vernacular{
                    yá{\downstep}kháá[ndakhuula]}  &   
                     \gloss{‘released’}  &  \\

                     \vernacular{
                    yá{\downstep}kháá[seebula]}  &   
                     \gloss{‘said bye
                    to’}  &  \\

                     \vernacular{
                    yá{\downstep}kháá[mboombelitsa]}  &   
                     \gloss{‘comforted’}  &  \\

                     \vernacular{
                    yá{\downstep}kháá[siinjilitsa]}  &   
                     \gloss{
                    ‘made...stand’}  &  \\

                     \vernacular{
                    yá{\downstep}kháá[ndeebandeeba]}  &   
                     \gloss{‘asked
                    (iter)’}  &  \\

                     \vernacular{
                    yá{\downstep}kháá[ngalukhanyinya]}  &   
                     \gloss{
                    ‘turned...over’}  &  \\
\end{tabular}
%\caption{\nocaption}
     
\begin{tabular}{llllll}  
  \multicolumn{5}{l}{
                     \vernacular{(116) /Ø/
                    V-Initial + OP
                    } \gloss{‘s/he
                    just...me’} } &  \\
\multicolumn{5}{l}{ } &  \\

                     \vernacular{
                    yá{\downstep}kháá[nzenya]}  &   
                     \gloss{‘wanted’}  &     &   
                     \vernacular{
                    yá{\downstep}kháá[nzeyela]}  &   
                     \gloss{‘wiped for’}  &  \\

                     \vernacular{
                    yá{\downstep}kháá[nyambakhana]}  &   
                     \gloss{‘refused’}  &     &   
                     \vernacular{
                    yá{\downstep}kháá[nzeleelitsa]}  &   
                     \gloss{
                    ‘carried...hanging’}  &  \\
\end{tabular}
%\caption{\nocaption}
     
\begin{tabular}{llllll}  
  \multicolumn{5}{l}{
                     \vernacular{(117) /H/
                    C-Initial + OP
                    } \gloss{‘s/he
                    just...him/herself’} } &  \\
\multicolumn{5}{l}{ } &  \\

                     \vernacular{
                    yá{\downstep}khíí[ra]}  &   
                     \gloss{‘buried’}  &     &   
                     \vernacular{
                    yá{\downstep}khíí[beka]}  &   
                     \gloss{‘shaved’}  &  \\

                     \vernacular{
                    yá{\downstep}khíí[suunga]}  &   
                     \gloss{‘hung’}  &     &   
                     \vernacular{
                    yá{\downstep}khíí[khalaka]}  &   
                     \gloss{‘cut’}  &  \\

                     \vernacular{
                    yá{\downstep}khíí[sitaaka]}  &   
                     \gloss{‘accused’}  &     &   
                     \vernacular{
                    yá{\downstep}khíí[saanditsa]}  &   
                     \gloss{‘thanked’}  &  \\

                     \vernacular{
                    yá{\downstep}khíí[tsuunzuuna]}  &   
                     \gloss{‘sucked’}  &     &   
                     \vernacular{
                    yá{\downstep}khíí[boholola]}  &   
                     \gloss{‘untied’}  &  \\
\end{tabular}
%\caption{\nocaption}
     
\begin{tabular}{llllll}  
  \multicolumn{5}{l}{
                     \vernacular{(118) /H/
                    V-Initial + OP
                    } \gloss{‘s/he
                    just...him/herself’} } &  \\
\multicolumn{5}{l}{ } &  \\

                     \vernacular{
                    yá{\downstep}khíí[yira]}  &   
                     \gloss{‘killed’}  &     &   
                     \vernacular{
                    yá{\downstep}khíí[yikoomba]}  &   
                     \gloss{‘admired’}  &  \\

                     \vernacular{
                    yá{\downstep}khíí[yisiaka]}  &   
                     \gloss{‘smacked’}  &     &   
                     \vernacular{
                    yá{\downstep}khíí[yononyinya]}  &   
                     \gloss{‘spoiled’}  &  \\

                     \vernacular{
                    yá{\downstep}khíí[yabukhanyinya]}  &   
                     \gloss{‘separated’}  &     &     &     &  \\
\end{tabular}
%\caption{\nocaption}
     
\begin{tabular}{llllll}  
  \multicolumn{5}{l}{
                     \vernacular{(119) /Ø/
                    C-Initial + OP
                    } \gloss{‘s/he
                    just...him/herself’} } &  \\
\multicolumn{5}{l}{ } &  \\

                     \vernacular{
                    yá{\downstep}khíí[kama]}  &   
                     \gloss{‘sheltered’}  &     &   
                     \vernacular{
                    yá{\downstep}khíí[siinga]}  &   
                     \gloss{‘bathed’}  &  \\

                     \vernacular{
                    yá{\downstep}khíí[kulikha]}  &   
                     \gloss{‘named’}  &     &   
                     \vernacular{
                    yá{\downstep}khíí[naabula]}  &   
                     \gloss{‘undressed’}  &  \\

                     \vernacular{
                    yá{\downstep}khíí[lakhuula]}  &   
                     \gloss{‘released’}  &     &   
                     \vernacular{
                    yá{\downstep}khíí[hoombelitsa]}  &   
                     \gloss{‘comforted’}  &  \\

                     \vernacular{
                    yá{\downstep}khíí[siinjilitsa]}  &   
                     \gloss{
                    ‘made...stand’}  &     &   
                     \vernacular{
                    yá{\downstep}khíí[reebareeba]}  &   
                     \gloss{‘asked
                    (iter)’}  &  \\

                     \vernacular{
                    yá{\downstep}khíí[kalukhanyinya]}  &   
                     \gloss{
                    ‘turned...over’}  &  \\
\end{tabular}
%\caption{\nocaption}
     
\begin{tabular}{llllll}  
  \multicolumn{5}{l}{
                     \vernacular{(120) /Ø/
                    V-Initial + OP
                    } \gloss{‘s/he
                    just...him/herself’} } &  \\
\multicolumn{5}{l}{ } &  \\

                     \vernacular{
                    yá{\downstep}khíí[yeya]}  &   
                     \gloss{‘wiped’}  &     &   
                     \vernacular{
                    yá{\downstep}khíí[yeyela]}  &   
                     \gloss{‘wiped for’}  &  \\

                     \vernacular{
                    yá{\downstep}khíí[yambakhana]}  &   
                     \gloss{‘despised’}  &     &   
                     \vernacular{
                    yá{\downstep}khíí[yeleelitsa]}  &   
                     \gloss{‘hung’}  &  \\
\end{tabular}
%\caption{\nocaption}
     
\begin{tabular}{llllll}  
  \multicolumn{5}{l}{
                     \vernacular{(121) /H/
                    C-Initial + OP + OP
                    } \gloss{‘s/he
                    just...him/her for me’} } &  \\
\multicolumn{5}{l}{ } &  \\

                     \vernacular{
                    yá{\downstep}khámúu[ndeela]}  &   
                     \gloss{‘buried’}  &     &   
                     \vernacular{
                    yá{\downstep}khámúu[mbechela]}  &   
                     \gloss{‘shaved’}  &  \\

                     \vernacular{
                    yá{\downstep}khámúu[ndeerela]}  &   
                     \gloss{‘brought’}  &     &   
                     \vernacular{
                    yá{\downstep}khámúu[khalachila]}  &   
                     \gloss{‘cut’}  &  \\

                     \vernacular{
                    yá{\downstep}khámúu[sitaachila]}  &   
                     \gloss{‘accused’}  &     &   
                     \vernacular{
                    yá{\downstep}khámúu[mboolitsila]}  &   
                     \gloss{‘seduced’}  &  \\

                     \vernacular{
                    yá{\downstep}khámúu[mbohololela]}  &   
                     \gloss{‘untied’}  &     &     &     &  \\
\end{tabular}
%\caption{\nocaption}
     
\begin{tabular}{llllll}  
  \multicolumn{5}{l}{
                     \vernacular{(122) /H/
                    V-Initial + OP + OP
                    } \gloss{‘s/he
                    just...him/her for me’} } &  \\
\multicolumn{5}{l}{ } &  \\

                     \vernacular{
                    yá{\downstep}khámúu[nzirila]}  &   
                     \gloss{‘killed’}  &  \\

                     \vernacular{
                    yá{\downstep}khámúu[nzechitsila]}  &   
                     \gloss{‘admired’}  &  \\

                     \vernacular{
                    yá{\downstep}khámúu[nzisiachila]}  &   
                     \gloss{‘smacked’}  &  \\

                     \vernacular{
                    yá{\downstep}khámúu[nzononyinyila]}  &   
                     \gloss{‘spoiled’}  &  \\

                     \vernacular{
                    yá{\downstep}khámúu[nzabukhanyinyila]}  &   
                     \gloss{‘separated’}  &  \\
\end{tabular}
%\caption{\nocaption}
     
\begin{tabular}{llllll}  
  \multicolumn{5}{l}{
                     \vernacular{(123) /Ø/
                    C-Initial + OP + OP
                    } \gloss{‘s/he
                    just...him/her for me’} } &  \\
\multicolumn{5}{l}{ } &  \\

                     \vernacular{
                    yá{\downstep}khámúu[nziila]}  &   
                     \gloss{‘went for’}  &  \\

                     \vernacular{
                    yá{\downstep}khámúu[ndeshela]}  &   
                     \gloss{‘left’}  &  \\

                     \vernacular{
                    yá{\downstep}khámúu[noondela]}  &   
                     \gloss{‘followed’}  &  \\

                     \vernacular{
                    yá{\downstep}khámúu[ngulishila]}  &   
                     \gloss{‘named’}  &  \\

                     \vernacular{
                    yá{\downstep}khámúu[ndakhuulila]}  &   
                     \gloss{‘released’}  &  \\

                     \vernacular{
                    yá{\downstep}khámúu[seebulila]}  &   
                     \gloss{‘said bye
                    to’}  &  \\

                     \vernacular{
                    yá{\downstep}khámúu[mboombelitsila]}  &   
                     \gloss{‘comforted’}  &  \\

                     \vernacular{
                    yá{\downstep}khámúu[siinjilitsila]}  &   
                     \gloss{
                    ‘made...stand’}  &  \\
\end{tabular}
%\caption{\nocaption}
     
\begin{tabular}{llllll}  
  \multicolumn{5}{l}{
                     \vernacular{(124) /Ø/
                    V-Initial + OP + OP
                    } \gloss{‘s/he
                    just...him/her \ob mu-\cb  / it
                    } } &  \\
\multicolumn{5}{l}{ } &  \\

                     \vernacular{
                    yá{\downstep}khábúu[nzalila]}  &   
                     \gloss{‘displayed’}  &     &   
                     \vernacular{
                    yá{\downstep}khákúu[nzashitsila]}  &   
                     \gloss{‘lit’}  &  \\

                     \vernacular{
                    yá{\downstep}khábúu[nziluulila]}  &   
                     \gloss{‘winnowed’}  &     &   
                     \vernacular{
                    yá{\downstep}khálúu[nzitsulitsila]}  &   
                     \gloss{‘filled’}  &  \\

                     \vernacular{
                    yá{\downstep}khákúu[nzeleelitsila]}  &   
                     \gloss{‘hung’}  &     &     &     &  \\
\end{tabular}
%\caption{\nocaption}
     
\begin{tabular}{lll}  
  \multicolumn{2}{l}{
                     \vernacular{(125) /H/
                    C-Initial Phrase-Medial} \gloss{‘s/he just...the
                    man \ob musáatsa\cb  /} } &  \\
\multicolumn{2}{l}{
                     \gloss{the boy
                    \ob mú{\downstep}yáyi\cb  / someone \ob muundu\cb ’} } &  \\

                     \vernacular{yá{\downstep}khá[rá]
                    musáatsa/mú{\downstep}yáyi/muundu}  &   
                     \gloss{‘buried’}  &  \\

                     \vernacular{yá{\downstep}khá[béka]
                    musáatsa/mú{\downstep}yáyi/muundu}  &   
                     \gloss{‘shaved’}  &  \\

                     \vernacular{yá{\downstep}khá[léera]
                    musáatsa/mú{\downstep}yáyi/muundu}  &   
                     \gloss{‘brought’}  &  \\

                     \vernacular{yá{\downstep}khá[khálaka]
                    musáatsa/mú{\downstep}yáyi/muundu}  &   
                     \gloss{‘cut’}  &  \\

                     \vernacular{yá{\downstep}khá[sítaaka]
                    musáatsa/mú{\downstep}yáyi/muundu}  &   
                     \gloss{‘accused’}  &  \\

                     \vernacular{
                    yá{\downstep}khá[bóolitsa]
                    musáatsa/mú{\downstep}yáyi/muundu}  &   
                     \gloss{‘seduced’}  &  \\

                     \vernacular{
                    yá{\downstep}khá[tsúunzuuna]
                    musáatsa/mú{\downstep}yáyi/muundu}  &   
                     \gloss{‘sucked’}  &  \\

                     \vernacular{
                    yá{\downstep}khá[bóholola]
                    musáatsa/mú{\downstep}yáyi/muundu}  &   
                     \gloss{‘untied’}  &  \\

                     \vernacular{
                    yá{\downstep}khá[bóyong’ana]
                    musáatsa/mú{\downstep}yáyi/muundu}  &   
                     \gloss{‘went
                    around’}  &  \\

                     \vernacular{
                    yá{\downstep}khá[ng’óng’oolitsa]
                    musáatsa/mú{\downstep}yáyi/muundu}  &   
                     \gloss{‘teased’}  &  \\
\end{tabular}
%\caption{\nocaption}
     
\begin{tabular}{lll}  
  \multicolumn{2}{l}{
                     \vernacular{(126) /Ø/
                    C-Initial Phrase-Medial} \gloss{‘s/he just...the
                    man \ob musáatsa\cb  /} } &  \\
\multicolumn{2}{l}{
                     \gloss{the boy
                    \ob mú{\downstep}yáyi\cb  / someone \ob muundu\cb ’} } &  \\

                     \vernacular{yákha[tsia]
                    musáatsa/mú{\downstep}yáyi/muundu}  &   
                     \gloss{‘went for’}  &  \\

                     \vernacular{yákha[lekha]
                    musáatsa/mú{\downstep}yáyi/muundu}  &   
                     \gloss{‘left’}  &  \\

                     \vernacular{yákha[loonda]
                    musáatsa/mú{\downstep}yáyi/muundu}  &   
                     \gloss{‘followed’}  &  \\

                     \vernacular{yákha[kulikha]
                    musáatsa/mú{\downstep}yáyi/muundu}  &   
                     \gloss{‘named’}  &  \\

                     \vernacular{yákha[lakhuula]
                    musáatsa/mú{\downstep}yáyi/muundu}  &   
                     \gloss{‘released’}  &  \\

                     \vernacular{yákha[seebula]
                    musáatsa/mú{\downstep}yáyi/muundu}  &   
                     \gloss{‘said bye’}  &  \\

                     \vernacular{yákha[kalushitsa]
                    musáatsa/mú{\downstep}yáyi/muundu}  &   
                     \gloss{‘returned’}  &  \\

                     \vernacular{yákha[reebareeba]
                    musáatsa/mú{\downstep}yáyi/muundu}  &   
                     \gloss{‘asked
                    (iter)’}  &  \\

                     \vernacular{
                    yákha[kalukhanyinya]
                    musáatsa/mú{\downstep}yáyi/muundu}  &   
                     \gloss{
                    ‘turned...over’}  &  \\
\end{tabular}
%\caption{\nocaption}
     
\begin{tabular}{lll}  
  \multicolumn{2}{l}{
                     \vernacular{(127) /H/
                    C-Initial +OP Phrase-Medial} \gloss{‘s/he just...the
                    man \ob musáatsa\cb  /} } &  \\
\multicolumn{2}{l}{
                     \gloss{the boy
                    \ob mú{\downstep}yáyi\cb  / someone \ob muundu\cb  for
                    him/her’} } &  \\

                     \vernacular{yá{\downstep}khámú[reela]
                    musáatsa/mú{\downstep}yáyi/muundu}  &   
                     \gloss{‘buried’}  &  \\

                     \vernacular{
                    yá{\downstep}khámú[bechela]
                    musáatsa/mú{\downstep}yáyi/muundu}  &   
                     \gloss{‘shaved’}  &  \\

                     \vernacular{
                    yá{\downstep}khámú[leerela]
                    musáatsa/mú{\downstep}yáyi/muundu}  &   
                     \gloss{‘brought’}  &  \\

                     \vernacular{
                    yá{\downstep}khámú[khalachila]
                    musáatsa/mú{\downstep}yáyi/muundu}  &   
                     \gloss{‘cut’}  &  \\

                     \vernacular{
                    yá{\downstep}khámú[sitaachila]
                    musáatsa/mú{\downstep}yáyi/muundu}  &   
                     \gloss{‘accused’}  &  \\

                     \vernacular{
                    yá{\downstep}khámú[boolitsila]
                    musáatsa/mú{\downstep}yáyi/muundu}  &   
                     \gloss{‘seduced’}  &  \\

                     \vernacular{
                    yá{\downstep}khámú[tsuunzuunila]
                    musáatsa/mú{\downstep}yáyi/muundu}  &   
                     \gloss{‘sucked’}  &  \\

                     \vernacular{
                    yá{\downstep}khámú[bohololela]
                    musáatsa/mú{\downstep}yáyi/muundu}  &   
                     \gloss{‘untied’}  &  \\

                     \vernacular{
                    yá{\downstep}khámú[boyong’anila]
                    musáatsa/mú{\downstep}yáyi/muundu}  &   
                     \gloss{‘went
                    around’}  &  \\

                     \vernacular{
                    yá{\downstep}khámú[ng’ong’oolitsila]
                    musáatsa/mú{\downstep}yáyi/muundu}  &   
                     \gloss{‘teased’}  &  \\
\end{tabular}
%\caption{\nocaption}
     
\begin{tabular}{lll}  
  \multicolumn{2}{l}{
                     \vernacular{(128) /Ø/
                    C-Initial +OP Phrase-Medial} \gloss{‘s/he just...the
                    man \ob musáatsa\cb  /} } &  \\
\multicolumn{2}{l}{
                     \gloss{the boy
                    \ob mú{\downstep}yáyi\cb  / someone \ob muundu\cb  for
                    him/her’} } &  \\

                     \vernacular{
                    yá{\downstep}khámú[tsiila]
                    musáatsa/mú{\downstep}yáyi/muundu}  &   
                     \gloss{‘went for’}  &  \\

                     \vernacular{
                    yá{\downstep}khámú[leshela]
                    musáatsa/mú{\downstep}yáyi/muundu}  &   
                     \gloss{‘left’}  &  \\

                     \vernacular{
                    yá{\downstep}khámú[loondela]
                    musáatsa/mú{\downstep}yáyi/muundu}  &   
                     \gloss{‘followed’}  &  \\

                     \vernacular{
                    yá{\downstep}khámú[kulishila]
                    musáatsa/mú{\downstep}yáyi/muundu}  &   
                     \gloss{‘named’}  &  \\

                     \vernacular{
                    yá{\downstep}khámú[lakhuulila]
                    musáatsa/mú{\downstep}yáyi/muundu}  &   
                     \gloss{‘released’}  &  \\

                     \vernacular{
                    yá{\downstep}khámú[seebulila]
                    musáatsa/mú{\downstep}yáyi/muundu}  &   
                     \gloss{‘said bye
                    to’}  &  \\

                     \vernacular{
                    yá{\downstep}khámú[kalushitsila]
                    musáatsa/mú{\downstep}yáyi/muundu}  &   
                     \gloss{‘returned’}  &  \\

                     \vernacular{
                    yá{\downstep}khámú[reebareebela]
                    musáatsa/mú{\downstep}yáyi/muundu}  &   
                     \gloss{‘asked
                    (iter)’}  &  \\

                     \vernacular{
                    yá{\downstep}khámú[kalukhanyinyila]
                    musáatsa/mú{\downstep}yáyi/muundu}  &   
                     \gloss{
                    ‘turned...over’}  &  \\
\end{tabular}
%\caption{\nocaption}
     
\begin{tabular}{lll}  
  \multicolumn{2}{l}{
                     \vernacular{(129) /H/
                    C-Initial +OP + OP
                    } \gloss{‘s/he just...the
                    man \ob musáatsa\cb  /} } &  \\
\multicolumn{2}{l}{
                     \gloss{the boy
                    \ob mú{\downstep}yáyi\cb  / someone \ob muundu\cb  for him/her for
                    me’} } &  \\

                     \vernacular{
                    yá{\downstep}khámúu[ndeela]
                    musáatsa/mú{\downstep}yáyi/muundu}  &   
                     \gloss{‘buried’}  &  \\

                     \vernacular{
                    yá{\downstep}khámúu[mbechela]
                    musáatsa/mú{\downstep}yáyi/muundu}  &   
                     \gloss{‘shaved’}  &  \\

                     \vernacular{
                    yá{\downstep}khámúu[ndeerela]
                    musáatsa/mú{\downstep}yáyi/muundu}  &   
                     \gloss{‘brought’}  &  \\

                     \vernacular{
                    yá{\downstep}khámúu[sitaachila]
                    musáatsa/mú{\downstep}yáyi/muundu}  &   
                     \gloss{‘accused’}  &  \\

                     \vernacular{
                    yá{\downstep}khámúu[mbohololela]
                    musáatsa/mú{\downstep}yáyi/muundu}  &   
                     \gloss{‘untied’}  &  \\
\end{tabular}
%\caption{\nocaption}
     
\begin{tabular}{lll}  
  \multicolumn{2}{l}{
                     \vernacular{(130) /Ø/
                    C-Initial +OP + OP
                    } \gloss{‘s/he just...the
                    man \ob musáatsa\cb  /} } &  \\
\multicolumn{2}{l}{
                     \gloss{the boy
                    \ob mú{\downstep}yáyi\cb  / someone \ob muundu\cb  for him/her for
                    me’} } &  \\

                     \vernacular{
                    yá{\downstep}khámúu[nziila]
                    musáatsa/mú{\downstep}yáyi/muundu}  &   
                     \gloss{‘went for’}  &  \\

                     \vernacular{
                    yá{\downstep}khámúu[ndeshela]
                    musáatsa/mú{\downstep}yáyi/muundu}  &   
                     \gloss{‘left’}  &  \\

                     \vernacular{
                    yá{\downstep}khámúu[noondela]
                    musáatsa/mú{\downstep}yáyi/muundu}  &   
                     \gloss{‘followed’}  &  \\

                     \vernacular{
                    yá{\downstep}khámúu[ngulishila]
                    musáatsa/mú{\downstep}yáyi/muundu}  &   
                     \gloss{‘named’}  &  \\

                     \vernacular{
                    yá{\downstep}khámúu[ndakhuulila]
                    musáatsa/mú{\downstep}yáyi/muundu}  &   
                     \gloss{‘released’}  &  \\
\end{tabular}
%\caption{\nocaption}
    

\subsection{Immediate Past Negative: Pattern
              1b}\label{sec:sImmPastNeg}


\begin{tabular}{llllll}  
  \multicolumn{5}{l}{
                     \vernacular{(131) /H/
                    C-Initial} \gloss{‘s/he did not
                    just...’} } &  \\
\multicolumn{5}{l}{ } &  \\

                     \vernacular{yá{\downstep}khá[rá]
                    {\downstep}tá}  &   
                     \gloss{‘bury’}  &  \\

                     \vernacular{yá{\downstep}khá[ng’wá]
                    {\downstep}tá}  &   
                     \gloss{‘drink’}  &  \\

                     \vernacular{yá{\downstep}khá[khwá]
                    {\downstep}tá}  &   
                     \gloss{‘pay dowry’}  &  \\

                     \vernacular{yá{\downstep}khá[líá]
                    {\downstep}tá}  &   
                     \gloss{‘eat’}  &  \\

                     \vernacular{yá{\downstep}khá[lú{\downstep}má]
                    tá}  &   
                     \gloss{‘bite’}  &  \\

                     \vernacular{yá{\downstep}khá[bé{\downstep}ká]
                    tá}  &   
                     \gloss{‘shave’}  &  \\

                     \vernacular{
                    yá{\downstep}khá[té{\downstep}ékhá] tá}  &   
                     \gloss{‘cook’}  &  \\

                     \vernacular{
                    yá{\downstep}khá[lé{\downstep}érá] tá}  &   
                     \gloss{‘bring’}  &  \\

                     \vernacular{
                    yá{\downstep}khá[khá{\downstep}láká] tá}  &   
                     \gloss{‘cut’}  &  \\

                     \vernacular{
                    yá{\downstep}khá[ká{\downstep}láángá] tá}  &   
                     \gloss{‘fry’}  &  \\

                     \vernacular{
                    yá{\downstep}khá[sí{\downstep}tááká] tá}  &   
                     \gloss{‘accuse’}  &  \\

                     \vernacular{
                    yá{\downstep}khá[bó{\downstep}ólítsá] tá}  &   
                     \gloss{‘seduce’}  &  \\

                     \vernacular{
                    yá{\downstep}khá[sá{\downstep}ándítsá] tá}  &   
                     \gloss{‘thank’}  &  \\

                     \vernacular{
                    yá{\downstep}khá[tsú{\downstep}únzúúná] tá}  &   
                     \gloss{‘suck’}  &  \\

                     \vernacular{
                    yá{\downstep}khá[bó{\downstep}hólólá] tá}  &   
                     \gloss{‘untie’}  &  \\

                     \vernacular{
                    yá{\downstep}khá[bó{\downstep}yóng’áná] tá}  &   
                     \gloss{‘go around’}  &  \\

                     \vernacular{
                    yá{\downstep}khá[ng’ó{\downstep}ng’óólítsá] tá}  &   
                     \gloss{‘tease’}  &  \\

                     \vernacular{
                    yá{\downstep}khá[lí{\downstep}ngákányínyá] tá}  &   
                     \gloss{‘crumple’}  &  \\
\end{tabular}
%\caption{\nocaption}
     
\begin{tabular}{llllll}  
  \multicolumn{5}{l}{
                     \vernacular{(132) /Ø/
                    C-Initial} \gloss{‘s/he did not
                    just...’} } &  \\
\multicolumn{5}{l}{ } &  \\

                     \vernacular{yá{\downstep}khá[tsíá]
                    tá}  &   
                     \gloss{‘go’}  &  \\

                     \vernacular{yá{\downstep}khá[kwá]
                    tá}  &   
                     \gloss{‘fall’}  &  \\

                     \vernacular{yá{\downstep}khá[lékhá]
                    tá}  &   
                     \gloss{‘leave’}  &  \\

                     \vernacular{yá{\downstep}khá[réébá]
                    tá}  &   
                     \gloss{‘ask’}  &  \\

                     \vernacular{
                    yá{\downstep}khá[lóóndá] tá}  &   
                     \gloss{‘follow’}  &  \\

                     \vernacular{
                    yá{\downstep}khá[kúlíkhá] tá}  &   
                     \gloss{‘name’}  &  \\

                     \vernacular{
                    yá{\downstep}khá[hómóólá] tá}  &   
                     \gloss{‘massage’}  &  \\

                     \vernacular{
                    yá{\downstep}khá[lákhúúlá] tá}  &   
                     \gloss{‘release’}  &  \\

                     \vernacular{
                    yá{\downstep}khá[séébúlá] tá}  &   
                     \gloss{‘say bye’}  &  \\

                     \vernacular{
                    yá{\downstep}khá[hóómbélítsá] tá}  &   
                     \gloss{‘comfort’}  &  \\

                     \vernacular{
                    yá{\downstep}khá[kálúshítsá] tá}  &   
                     \gloss{‘return’}  &  \\

                     \vernacular{
                    yá{\downstep}khá[síínjílítsá] tá}  &   
                     \gloss{‘make stand’}  &  \\

                     \vernacular{
                    yá{\downstep}khá[réébáréébá] tá}  &   
                     \gloss{‘ask (iter)’}  &  \\

                     \vernacular{
                    yá{\downstep}khá[kálúkhányínyá] tá}  &   
                     \gloss{‘turn over’}  &  \\

                     \vernacular{
                    yá{\downstep}khá[sébúlúkhányínyá] tá}  &   
                     \gloss{‘scatter’}  &  \\
\end{tabular}
%\caption{\nocaption}
     
\begin{tabular}{llllll}  
  \multicolumn{5}{l}{
                     \vernacular{(133) /H/
                    C-Initial + OP} \gloss{‘s/he did not
                    just...him/her’} } &  \\
\multicolumn{5}{l}{ } &  \\

                     \vernacular{yá{\downstep}khámú[{\downstep}rá]
                    tá}  &   
                     \gloss{‘bury’}  &  \\

                     \vernacular{
                    yá{\downstep}khámú[{\downstep}béká] tá}  &   
                     \gloss{‘shave’}  &  \\

                     \vernacular{
                    yá{\downstep}khámú[{\downstep}léérá] tá}  &   
                     \gloss{‘bring’}  &  \\

                     \vernacular{
                    yá{\downstep}khámú[{\downstep}kháláká] tá}  &   
                     \gloss{‘cut’}  &  \\

                     \vernacular{
                    yá{\downstep}khámú[{\downstep}sítááká] tá}  &   
                     \gloss{‘accuse’}  &  \\

                     \vernacular{
                    yá{\downstep}khámú[{\downstep}bóólítsá] tá}  &   
                     \gloss{‘seduce’}  &  \\

                     \vernacular{
                    yá{\downstep}khámú[{\downstep}tsúúnzúúná] tá}  &   
                     \gloss{‘suck’}  &  \\

                     \vernacular{
                    yá{\downstep}khámú[{\downstep}bóhólólá] tá}  &   
                     \gloss{‘untie’}  &  \\

                     \vernacular{
                    yá{\downstep}khámú[{\downstep}bóyóng’áná] tá}  &   
                     \gloss{‘go around’}  &  \\

                     \vernacular{
                    yá{\downstep}khámú[{\downstep}ng’óng’óólítsá]
                    tá}  &   
                     \gloss{‘tease’}  &  \\

                     \vernacular{
                    yá{\downstep}khámú[{\downstep}língákányínyá] tá}  &   
                     \gloss{‘bend’}  &  \\
\end{tabular}
%\caption{\nocaption}
     
\begin{tabular}{llllll}  
  \multicolumn{5}{l}{
                     \vernacular{(134) /Ø/
                    C-Initial + OP} \gloss{‘s/he did not
                    just...him/her \ob mu-\cb  / them
                    } } &  \\
\multicolumn{5}{l}{ } &  \\

                     \vernacular{
                    yá{\downstep}khámú[{\downstep}tsíá] {\downstep}tá}  &   
                     \gloss{‘go for’}  &  \\

                     \vernacular{
                    yá{\downstep}khámú[{\downstep}lékhá] {\downstep}tá}  &   
                     \gloss{‘leave’}  &  \\

                     \vernacular{
                    yá{\downstep}khámú[{\downstep}lóóndá] tá}  &   
                     \gloss{‘follow’}  &  \\

                     \vernacular{
                    yá{\downstep}khámú[{\downstep}kúlíkhá] tá}  &   
                     \gloss{‘name’}  &  \\

                     \vernacular{
                    yá{\downstep}khámú[{\downstep}lákhúúlá] tá}  &   
                     \gloss{‘release’}  &  \\

                     \vernacular{
                    yá{\downstep}khámú[{\downstep}séébúlá] tá}  &   
                     \gloss{‘say bye to’}  &  \\

                     \vernacular{
                    yá{\downstep}khámú[{\downstep}hóómbélítsá] tá}  &   
                     \gloss{‘comfort’}  &  \\

                     \vernacular{
                    yá{\downstep}khámú[{\downstep}kálúshítsá] tá}  &   
                     \gloss{‘return’}  &  \\

                     \vernacular{
                    yá{\downstep}khámú[{\downstep}síínjílítsá] tá}  &   
                     \gloss{
                    ‘make...stand’}  &  \\

                     \vernacular{
                    yá{\downstep}khámú[{\downstep}réébáréébá] tá}  &   
                     \gloss{‘ask (iter)’}  &  \\

                     \vernacular{
                    yá{\downstep}khámú[{\downstep}kálúkhányínyá] tá}  &   
                     \gloss{
                    ‘turn...over’}  &  \\
\end{tabular}
%\caption{\nocaption}
     
\begin{tabular}{llllll}  
  \multicolumn{5}{l}{
                     \vernacular{(135) /H/
                    C-Initial + OP + OP
                    } \gloss{‘s/he did not
                    just...him/her for me’} } &  \\
\multicolumn{5}{l}{ } &  \\

                     \vernacular{
                    yá{\downstep}khámú{\downstep}ú[ndéélá] tá}  &   
                     \gloss{‘bury’}  &     &   
                     \vernacular{
                    yá{\downstep}khámú{\downstep}ú[mbéchélá] tá}  &   
                     \gloss{‘shave’}  &  \\

                     \vernacular{
                    yá{\downstep}khámú{\downstep}ú[ndéérélá] tá}  &   
                     \gloss{‘bring’}  &     &   
                     \vernacular{
                    yá{\downstep}khámú{\downstep}ú[kháláchílá] tá}  &   
                     \gloss{‘cut’}  &  \\

                     \vernacular{
                    yá{\downstep}khámú{\downstep}ú[sítááchílá] tá}  &   
                     \gloss{‘accuse’}  &     &   
                     \vernacular{
                    yá{\downstep}khámú{\downstep}ú[mbóólítsílá] tá}  &   
                     \gloss{‘seduce’}  &  \\

                     \vernacular{
                    yá{\downstep}khámú{\downstep}ú[mbóhólólélá] tá}  &   
                     \gloss{‘untie’}  &     &     &     &  \\
\end{tabular}
%\caption{\nocaption}
     
\begin{tabular}{llllll}  
  \multicolumn{5}{l}{
                     \vernacular{(136) /Ø/
                    C-Initial + OP + OP
                    } \gloss{‘s/he did not
                    just...him/her for me’} } &  \\
\multicolumn{5}{l}{ } &  \\

                     \vernacular{
                    yá{\downstep}khámú{\downstep}ú[nzíílá] tá}  &   
                     \gloss{‘go for’}  &  \\

                     \vernacular{
                    yá{\downstep}khámú{\downstep}ú[ndéshélá] tá}  &   
                     \gloss{‘leave’}  &  \\

                     \vernacular{
                    yá{\downstep}khámú{\downstep}ú[nóóndélá] tá}  &   
                     \gloss{‘follow’}  &  \\

                     \vernacular{
                    yá{\downstep}khámú{\downstep}ú[ngúlíshílá] tá}  &   
                     \gloss{‘name’}  &  \\

                     \vernacular{
                    yá{\downstep}khámú{\downstep}ú[ndákhúúlílá] tá}  &   
                     \gloss{‘release’}  &  \\

                     \vernacular{
                    yá{\downstep}khámú{\downstep}ú[séébúlílá] tá}  &   
                     \gloss{‘say bye to’}  &  \\

                     \vernacular{
                    yá{\downstep}khámú{\downstep}ú[mbóómbélítsílá]
                    tá}  &   
                     \gloss{‘comfort’}  &  \\

                     \vernacular{
                    yá{\downstep}khámú{\downstep}ú[síínjílítsílá]
                    tá}  &   
                     \gloss{
                    ‘make...stand’}  &  \\
\end{tabular}
%\caption{\nocaption}
     
\begin{tabular}{lll}  
  \multicolumn{2}{l}{
                     \vernacular{(137) /H/
                    C-Initial Phrase-Medial} \gloss{‘s/he did not
                    just...the man \ob musáatsa\cb  /} } &  \\
\multicolumn{2}{l}{
                     \gloss{the boy
                    \ob mú{\downstep}yáyi\cb  / someone \ob muundu\cb ’} } &  \\

                     \vernacular{yá{\downstep}khá[rá]
                    musáatsa/mú{\downstep}yáyi/muundu tá}  &   
                     \gloss{‘bury’}  &  \\

                     \vernacular{yá{\downstep}khá[béka]
                    musáatsa/mú{\downstep}yáyi/muundu tá}  &   
                     \gloss{‘shave’}  &  \\

                     \vernacular{yá{\downstep}khá[léera]
                    musáatsa/mú{\downstep}yáyi/muundu tá}  &   
                     \gloss{‘bring’}  &  \\

                     \vernacular{yá{\downstep}khá[khálaka]
                    musáatsa/mú{\downstep}yáyi/muundu tá}  &   
                     \gloss{‘cut’}  &  \\

                     \vernacular{yá{\downstep}khá[sítaaka]
                    musáatsa/mú{\downstep}yáyi/muundu tá}  &   
                     \gloss{‘accuse’}  &  \\

                     \vernacular{
                    yá{\downstep}khá[bóolitsa] musáatsa/mú{\downstep}yáyi/muundu
                    tá}  &   
                     \gloss{‘seduce’}  &  \\

                     \vernacular{
                    yá{\downstep}khá[tsúunzuuna]
                    musáatsa/mú{\downstep}yáyi/muundu tá}  &   
                     \gloss{‘suck’}  &  \\

                     \vernacular{
                    yá{\downstep}khá[bóholola] musáatsa/mú{\downstep}yáyi/muundu
                    tá}  &   
                     \gloss{‘untie’}  &  \\

                     \vernacular{
                    yá{\downstep}khá[bóyong’ana]
                    musáatsa/mú{\downstep}yáyi/muundu tá}  &   
                     \gloss{‘go around’}  &  \\

                     \vernacular{
                    yá{\downstep}khá[ng’óng’oolitsa]
                    musáatsa/mú{\downstep}yáyi/muundu tá}  &   
                     \gloss{‘tease’}  &  \\
\end{tabular}
%\caption{\nocaption}
     
\begin{tabular}{lll}  
  \multicolumn{2}{l}{
                     \vernacular{(138) /Ø/
                    C-Initial Phrase-Medial} \gloss{‘s/he did not
                    just...the man \ob musáatsa\cb  /} } &  \\
\multicolumn{2}{l}{
                     \gloss{the boy
                    \ob mú{\downstep}yáyi\cb  / someone \ob muundu\cb ’} } &  \\

                     \vernacular{yákha[tsia]
                    musáatsa/mú{\downstep}yáyi/muundu tá}  &   
                     \gloss{‘go for’}  &  \\

                     \vernacular{yákha[lekha]
                    musáatsa/mú{\downstep}yáyi/muundu tá}  &   
                     \gloss{‘leave’}  &  \\

                     \vernacular{yákha[reeba]
                    musáatsa/mú{\downstep}yáyi/muundu tá}  &   
                     \gloss{‘ask’}  &  \\

                     \vernacular{yákha[loonda]
                    musáatsa/mú{\downstep}yáyi/muundu tá}  &   
                     \gloss{‘follow’}  &  \\

                     \vernacular{yákha[kulikha]
                    musáatsa/mú{\downstep}yáyi/muundu tá}  &   
                     \gloss{‘name’}  &  \\

                     \vernacular{yákha[lakhuula]
                    musáatsa/mú{\downstep}yáyi/muundu tá}  &   
                     \gloss{‘release’}  &  \\

                     \vernacular{yákha[seebula]
                    musáatsa/mú{\downstep}yáyi/muundu tá}  &   
                     \gloss{‘say bye to’}  &  \\

                     \vernacular{yákha[kalushitsa]
                    musáatsa/mú{\downstep}yáyi/muundu tá}  &   
                     \gloss{‘return’}  &  \\

                     \vernacular{yákha[reebareeba]
                    musáatsa/mú{\downstep}yáyi/muundu tá}  &   
                     \gloss{‘ask (iter)’}  &  \\

                     \vernacular{
                    yákha[kalukhanyinya]
                    musáatsa/mú{\downstep}yáyi/muundu tá}  &   
                     \gloss{
                    ‘turn...over’}  &  \\
\end{tabular}
%\caption{\nocaption}
     
\begin{tabular}{lll}  
  \multicolumn{2}{l}{
                     \vernacular{(139) /H/
                    C-Initial +OP Phrase-Medial} \gloss{‘s/he did not
                    just...the man \ob musáatsa\cb  /} } &  \\
\multicolumn{2}{l}{
                     \gloss{the boy
                    \ob mú{\downstep}yáyi\cb  / someone \ob muundu\cb  for
                    him/her’} } &  \\

                     \vernacular{yá{\downstep}khámú[reela]
                    musáatsa/mú{\downstep}yáyi/muundu tá}  &   
                     \gloss{‘bury’}  &  \\

                     \vernacular{
                    yá{\downstep}khámú[bechela] musáatsa/mú{\downstep}yáyi/muundu
                    tá}  &   
                     \gloss{‘shave’}  &  \\

                     \vernacular{
                    yá{\downstep}khámú[leerela] musáatsa/mú{\downstep}yáyi/muundu
                    tá}  &   
                     \gloss{‘bring’}  &  \\

                     \vernacular{
                    yá{\downstep}khámú[khalachila]
                    musáatsa/mú{\downstep}yáyi/muundu tá}  &   
                     \gloss{‘cut’}  &  \\

                     \vernacular{
                    yá{\downstep}khámú[sitaachila]
                    musáatsa/mú{\downstep}yáyi/muundu tá}  &   
                     \gloss{‘accuse’}  &  \\

                     \vernacular{
                    yá{\downstep}khámú[boolitsila]
                    musáatsa/mú{\downstep}yáyi/muundu tá}  &   
                     \gloss{‘seduce’}  &  \\

                     \vernacular{
                    yá{\downstep}khámú[tsuunzuunila]
                    musáatsa/mú{\downstep}yáyi/muundu tá}  &   
                     \gloss{‘suck’}  &  \\

                     \vernacular{
                    yá{\downstep}khámú[bohololela]
                    musáatsa/mú{\downstep}yáyi/muundu tá}  &   
                     \gloss{‘untie’}  &  \\

                     \vernacular{
                    yá{\downstep}khámú[boyong’anila]
                    musáatsa/mú{\downstep}yáyi/muundu tá}  &   
                     \gloss{‘go around’}  &  \\

                     \vernacular{
                    yá{\downstep}khámú[ng’ong’oolitsila]
                    musáatsa/mú{\downstep}yáyi/muundu tá}  &   
                     \gloss{‘tease’}  &  \\
\end{tabular}
%\caption{\nocaption}
     
\begin{tabular}{lll}  
  \multicolumn{2}{l}{
                     \vernacular{(140) /Ø/
                    C-Initial +OP Phrase-Medial} \gloss{‘s/he did not
                    just...the man \ob musáatsa\cb  /} } &  \\
\multicolumn{2}{l}{
                     \gloss{the boy
                    \ob mú{\downstep}yáyi\cb  / someone \ob muundu\cb  for
                    him/her’} } &  \\

                     \vernacular{
                    yá{\downstep}khámú[tsiila] musáatsa/mú{\downstep}yáyi/muundu
                    tá}  &   
                     \gloss{‘go for’}  &  \\

                     \vernacular{
                    yá{\downstep}khámú[leshela] musáatsa/mú{\downstep}yáyi/muundu
                    tá}  &   
                     \gloss{‘leave’}  &  \\

                     \vernacular{
                    yá{\downstep}khámú[loondela]
                    musáatsa/mú{\downstep}yáyi/muundu tá}  &   
                     \gloss{‘follow’}  &  \\

                     \vernacular{
                    yá{\downstep}khámú[kulishila]
                    musáatsa/mú{\downstep}yáyi/muundu tá}  &   
                     \gloss{‘name’}  &  \\

                     \vernacular{
                    yá{\downstep}khámú[lakhuulila]
                    musáatsa/mú{\downstep}yáyi/muundu tá}  &   
                     \gloss{‘release’}  &  \\

                     \vernacular{
                    yá{\downstep}khámú[seebulila]
                    musáatsa/mú{\downstep}yáyi/muundu tá}  &   
                     \gloss{‘say bye to’}  &  \\

                     \vernacular{
                    yá{\downstep}khámú[kalushitsila]
                    musáatsa/mú{\downstep}yáyi/muundu tá}  &   
                     \gloss{‘return’}  &  \\

                     \vernacular{
                    yá{\downstep}khámú[reebareebela]
                    musáatsa/mú{\downstep}yáyi/muundu tá}  &   
                     \gloss{‘ask (iter)’}  &  \\

                     \vernacular{
                    yá{\downstep}khámú[kalukhanyinyila]
                    musáatsa/mú{\downstep}yáyi/muundu tá}  &   
                     \gloss{
                    ‘turn...over’}  &  \\
\end{tabular}
%\caption{\nocaption}
     
\begin{tabular}{lll}  
  \multicolumn{2}{l}{
                     \vernacular{(141) /H/
                    C-Initial +OP + OP
                    } \gloss{‘s/he did not
                    just...the man \ob musáatsa\cb  /} } &  \\
\multicolumn{2}{l}{
                     \gloss{the boy
                    \ob mú{\downstep}yáyi\cb  / someone \ob muundu\cb  for him/her for
                    me’} } &  \\

                     \vernacular{
                    yá{\downstep}khámúu[ndeela] musáatsa/mú{\downstep}yáyi/muundu
                    tá}  &   
                     \gloss{‘bury’}  &  \\

                     \vernacular{
                    yá{\downstep}khámúu[mbechela]
                    musáatsa/mú{\downstep}yáyi/muundu tá}  &   
                     \gloss{‘shave’}  &  \\

                     \vernacular{
                    yá{\downstep}khámúu[ndeerela]
                    musáatsa/mú{\downstep}yáyi/muundu tá}  &   
                     \gloss{‘bring’}  &  \\

                     \vernacular{
                    yá{\downstep}khámúu[sitaachila]
                    musáatsa/mú{\downstep}yáyi/muundu tá}  &   
                     \gloss{‘accuse’}  &  \\

                     \vernacular{
                    yá{\downstep}khámúu[mbohololela]
                    musáatsa/mú{\downstep}yáyi/muundu tá}  &   
                     \gloss{‘untie’}  &  \\
\end{tabular}
%\caption{\nocaption}
     
\begin{tabular}{lll}  
  \multicolumn{2}{l}{
                     \vernacular{(142) /Ø/
                    C-Initial +OP + OP
                    } \gloss{‘s/he did not
                    just...the man \ob musáatsa\cb  /} } &  \\
\multicolumn{2}{l}{
                     \gloss{the boy
                    \ob mú{\downstep}yáyi\cb  / someone \ob muundu\cb  for him/her for
                    me’} } &  \\

                     \vernacular{
                    yá{\downstep}khámúu[nziila] musáatsa/mú{\downstep}yáyi/muundu
                    tá}  &   
                     \gloss{‘go for’}  &  \\

                     \vernacular{
                    yá{\downstep}khámúu[ndeshela]
                    musáatsa/mú{\downstep}yáyi/muundu tá}  &   
                     \gloss{‘leave’}  &  \\

                     \vernacular{
                    yá{\downstep}khámúu[noondela]
                    musáatsa/mú{\downstep}yáyi/muundu tá}  &   
                     \gloss{‘follow’}  &  \\

                     \vernacular{
                    yá{\downstep}khámúu[ngulishila]
                    musáatsa/mú{\downstep}yáyi/muundu tá}  &   
                     \gloss{‘name’}  &  \\

                     \vernacular{
                    yá{\downstep}khámúu[ndakhuulila]
                    musáatsa/mú{\downstep}yáyi/muundu tá}  &   
                     \gloss{‘release’}  &  \\
\end{tabular}
%\caption{\nocaption}
    

\subsection{Remote Future: Pattern 1b [JI]
              }\label{sec:sRemFut}


\begin{tabular}{llllll}  
  \multicolumn{5}{l}{
                     \vernacular{(143) /H/
                    C-Initial} \gloss{‘s/he
                    will...’} } &  \\
\multicolumn{5}{l}{ } &  \\

                     \vernacular{
                    yá{\downstep}khá[rɛ́]}  &   
                     \gloss{‘bury’}  &     &   
                     \vernacular{
                    yá{\downstep}khá[ng’wí]}  &   
                     \gloss{‘drink’}  &  \\

                     \vernacular{
                    yá{\downstep}khá[khwí]}  &   
                     \gloss{‘pay dowry’}  &     &   
                     \vernacular{
                    yá{\downstep}khá[lí]}  &   
                     \gloss{‘eat’}  &  \\

                     \vernacular{
                    yá{\downstep}khá[lúmɛ]}  &   
                     \gloss{‘bite’}  &     &   
                     \vernacular{
                    yá{\downstep}khá[béchɛ]}  &   
                     \gloss{‘shave’}  &  \\

                     \vernacular{
                    yá{\downstep}khá[téeshɛ]}  &   
                     \gloss{‘cook’}  &     &   
                     \vernacular{
                    yá{\downstep}khá[léerɛ]}  &   
                     \gloss{‘bring’}  &  \\

                     \vernacular{
                    yá{\downstep}khá[khálachɛ]}  &   
                     \gloss{‘cut’}  &     &   
                     \vernacular{
                    yá{\downstep}khá[kálaangɛ]}  &   
                     \gloss{‘fry’}  &  \\

                     \vernacular{
                    yá{\downstep}khá[sítaachɛ]}  &   
                     \gloss{‘accuse’}  &     &   
                     \vernacular{
                    yá{\downstep}khá[bóolitsɛ]}  &   
                     \gloss{‘seduce’}  &  \\

                     \vernacular{
                    yá{\downstep}khá[sáanditsɛ]}  &   
                     \gloss{‘thank’}  &     &   
                     \vernacular{
                    yá{\downstep}khá[tsúunzuunɛ]}  &   
                     \gloss{‘suck’}  &  \\

                     \vernacular{
                    yá{\downstep}khá[bóhololɛ]}  &   
                     \gloss{‘untie’}  &     &   
                     \vernacular{
                    yá{\downstep}khá[bóyong’anɛ]}  &   
                     \gloss{‘go around’}  &  \\

                     \vernacular{
                    yá{\downstep}khá[ng’óng’oolitsɛ]}  &   
                     \gloss{‘tease’}  &     &   
                     \vernacular{
                    yá{\downstep}khá[shíling’anyinyɛ]}  &   
                     \gloss{‘silence’}  &  \\
\end{tabular}
%\caption{\nocaption}
     
\begin{tabular}{llllll}  
  \multicolumn{5}{l}{
                     \vernacular{(144) /H/
                    V-Initial} \gloss{‘s/he
                    will...’} } &  \\
\multicolumn{5}{l}{ } &  \\

                     \vernacular{
                    yá{\downstep}kh[ííri]}  &   
                     \gloss{‘kill’}  &     &   
                     \vernacular{
                    yá{\downstep}kh[ííkoombɛ]}  &   
                     \gloss{‘admire’}  &  \\

                     \vernacular{
                    yá{\downstep}kh[íísiachɛ]}  &   
                     \gloss{‘smack’}  &     &   
                     \vernacular{
                    yá{\downstep}kh[ííkobolɛ]}  &   
                     \gloss{‘belch’}  &  \\

                     \vernacular{
                    yá{\downstep}kh[óónonyinyɛ]}  &   
                     \gloss{‘spoil’}  &     &   
                     \vernacular{
                    yá{\downstep}kh[áábukhanyinyɛ]}  &   
                     \gloss{‘separate’}  &  \\
\end{tabular}
%\caption{\nocaption}
     
\begin{tabular}{llllll}  
  \multicolumn{5}{l}{
                     \vernacular{(145) /Ø/
                    C-Initial} \gloss{‘s/he
                    will...’} } &  \\
\multicolumn{5}{l}{ } &  \\

                     \vernacular{
                    yákha[tsi]}  &   
                     \gloss{‘go’}  &     &   
                     \vernacular{
                    yákha[kwi]}  &   
                     \gloss{‘fall’}  &  \\

                     \vernacular{
                    yákha[leshɛ]}  &   
                     \gloss{‘leave’}  &     &   
                     \vernacular{
                    yákha[reebɛ]}  &   
                     \gloss{‘ask’}  &  \\

                     \vernacular{
                    yákha[loondɛ]}  &   
                     \gloss{‘follow’}  &     &   
                     \vernacular{
                    yákha[kumilɛ]}  &   
                     \gloss{‘hold’}  &  \\

                     \vernacular{
                    yákha[sasanɛ]}  &   
                     \gloss{‘resemble’}  &     &   
                     \vernacular{
                    yákha[homoolɛ]}  &   
                     \gloss{‘massage’}  &  \\

                     \vernacular{
                    yákha[lakhuulɛ]}  &   
                     \gloss{‘release’}  &     &   
                     \vernacular{
                    yákha[seebulɛ]}  &   
                     \gloss{‘say bye’}  &  \\

                     \vernacular{
                    yákha[hoombelitsɛ]}  &   
                     \gloss{‘comfort’}  &     &   
                     \vernacular{
                    yákha[kalukhitsɛ]}  &   
                     \gloss{‘return’}  &  \\

                     \vernacular{
                    yákha[siinjilitsɛ]}  &   
                     \gloss{‘make stand’}  &     &   
                     \vernacular{
                    yákha[rootsarootsɛ]}  &   
                     \gloss{‘poke
                    (iter)’}  &  \\

                     \vernacular{
                    yákha[kalukhanyinyɛ]}  &   
                     \gloss{‘turn over’}  &     &   
                     \vernacular{
                    yákha[sabulukhanyinyɛ]}  &   
                     \gloss{‘scatter’}  &  \\
\end{tabular}
%\caption{\nocaption}
     
\begin{tabular}{llllll}  
  \multicolumn{5}{l}{
                     \vernacular{(146) /Ø/
                    V-Initial} \gloss{‘s/he
                    will...’} } &  \\
\multicolumn{5}{l}{ } &  \\

                     \vernacular{
                    yákh[eenyɛ]}  &   
                     \gloss{‘want’}  &     &   
                     \vernacular{
                    yákh[aakhitsɛ]}  &   
                     \gloss{‘light’}  &  \\

                     \vernacular{
                    yákh[aabaarɛ]}  &   
                     \gloss{‘grope’}  &     &   
                     \vernacular{
                    yákh[aambakhanɛ]}  &   
                     \gloss{‘refuse’}  &  \\

                     \vernacular{
                    yákh[eeleelitsɛ]}  &   
                     \gloss{‘hang up’}  &     &     &     &  \\
\end{tabular}
%\caption{\nocaption}
     
\begin{tabular}{llllll}  
  \multicolumn{5}{l}{
                     \vernacular{(147) /H/
                    C-Initial + OP} \gloss{‘s/he
                    will...him/her’} } &  \\
\multicolumn{5}{l}{ } &  \\

                     \vernacular{
                    yá{\downstep}khámú[rɛ]}  &   
                     \gloss{‘bury’}  &     &   
                     \vernacular{
                    yá{\downstep}khámú[bechɛ]}  &   
                     \gloss{‘shave’}  &  \\

                     \vernacular{
                    yá{\downstep}khámú[leerɛ]}  &   
                     \gloss{‘bring’}  &     &   
                     \vernacular{
                    yá{\downstep}khámú[khalachɛ]}  &   
                     \gloss{‘cut’}  &  \\

                     \vernacular{
                    yá{\downstep}khámú[sitaachɛ]}  &   
                     \gloss{‘accuse’}  &     &   
                     \vernacular{
                    yá{\downstep}khámú[boolitsɛ]}  &   
                     \gloss{‘seduce’}  &  \\

                     \vernacular{
                    yá{\downstep}khámú[tsuunzuunɛ]}  &   
                     \gloss{‘suck’}  &     &   
                     \vernacular{
                    yá{\downstep}khámú[bohololɛ]}  &   
                     \gloss{‘untie’}  &  \\

                     \vernacular{
                    yá{\downstep}khámú[boyong’anɛ]}  &   
                     \gloss{‘go around’}  &     &   
                     \vernacular{
                    yá{\downstep}khámú[ng’ong’oolitsɛ]}  &   
                     \gloss{‘tease’}  &  \\

                     \vernacular{
                    yá{\downstep}khámú[shiling’anyinyɛ]}  &   
                     \gloss{‘silence’}  &     &     &     &  \\
\end{tabular}
%\caption{\nocaption}
     
\begin{tabular}{llllll}  
  \multicolumn{5}{l}{
                     \vernacular{(148) /H/
                    V-Initial + OP} \gloss{‘s/he
                    will...him/her’} } &  \\
\multicolumn{5}{l}{ } &  \\

                     \vernacular{
                    yá{\downstep}khámw[íirɛ]}  &   
                     \gloss{‘kill’}  &     &   
                     \vernacular{
                    yá{\downstep}khámw[íikoombɛ]}  &   
                     \gloss{‘admire’}  &  \\

                     \vernacular{
                    yá{\downstep}khámw[íisiachɛ]}  &   
                     \gloss{‘smack’}  &     &   
                     \vernacular{
                    yá{\downstep}khámw[óononyinyɛ]}  &   
                     \gloss{‘spoil’}  &  \\

                     \vernacular{
                    yá{\downstep}khámw[áabukhanyinyɛ]}  &   
                     \gloss{‘separate’}  &     &     &     &  \\
\end{tabular}
%\caption{\nocaption}
     
\begin{tabular}{llllll}  
  \multicolumn{5}{l}{
                     \vernacular{(149) /Ø/
                    C-Initial + OP} \gloss{‘s/he
                    will...him/her \ob mu-\cb  / them
                    } } &  \\
\multicolumn{5}{l}{ } &  \\

                     \vernacular{
                    yá{\downstep}khámú[tsi]}  &   
                     \gloss{‘go for’}  &  \\

                     \vernacular{
                    yá{\downstep}khámú[leshɛ]}  &   
                     \gloss{‘leave’}  &  \\

                     \vernacular{
                    yá{\downstep}khámú[loondɛ]}  &   
                     \gloss{‘follow’}  &  \\

                     \vernacular{
                    yá{\downstep}khámú[kumilɛ]}  &   
                     \gloss{‘hold’}  &  \\

                     \vernacular{
                    yá{\downstep}khámú[lakhuulɛ]}  &   
                     \gloss{‘release’}  &  \\

                     \vernacular{
                    yá{\downstep}khámú[seebulɛ]}  &   
                     \gloss{‘say bye to’}  &  \\

                     \vernacular{
                    yá{\downstep}khámú[hoombelitsɛ]}  &   
                     \gloss{‘comfort’}  &  \\

                     \vernacular{
                    yá{\downstep}khámú[kalukhitsɛ]}  &   
                     \gloss{‘return’}  &  \\

                     \vernacular{
                    yá{\downstep}khámú[siinjilitsɛ]}  &   
                     \gloss{
                    ‘make...stand’}  &  \\

                     \vernacular{
                    yá{\downstep}khámú[rootsarootsɛ]}  &   
                     \gloss{‘poke
                    (iter)’}  &  \\

                     \vernacular{
                    yá{\downstep}khámú[kalukhanyinyɛ]}  &   
                     \gloss{
                    ‘turn...over’}  &  \\

                     \vernacular{
                    yá{\downstep}khátsí[sabulukhanyinyɛ]}  &   
                     \gloss{‘scatter’}  &  \\
\end{tabular}
%\caption{\nocaption}
     
\begin{tabular}{llllll}  
  \multicolumn{5}{l}{
                     \vernacular{(150) /Ø/
                    V-Initial + OP} \gloss{‘s/he
                    will...him/her \ob mw-\cb  / it
                    } } &  \\
\multicolumn{5}{l}{ } &  \\

                     \vernacular{
                    yá{\downstep}khámw[éenyɛ]}  &   
                     \gloss{‘want’}  &     &   
                     \vernacular{
                    yá{\downstep}khákw[áakhitsɛ]}  &   
                     \gloss{‘light’}  &  \\

                     \vernacular{
                    yá{\downstep}khábw[íiluulɛ]}  &   
                     \gloss{‘winnow’}  &     &   
                     \vernacular{
                    yá{\downstep}khámw[aabaarɛ]}  &   
                     \gloss{‘grope’}  &  \\

                     \vernacular{
                    yá{\downstep}khámw[áambakhanɛ]}  &   
                     \gloss{‘refuse’}  &     &   
                     \vernacular{
                    yá{\downstep}khámw[éeleelitsɛ]}  &   
                     \gloss{
                    ‘hold...hanging’}  &  \\
\end{tabular}
%\caption{\nocaption}
     
\begin{tabular}{llllll}  
  \multicolumn{5}{l}{
                     \vernacular{(151) /H/
                    C-Initial + OP
                    } \gloss{‘s/he
                    will...me’} } &  \\
\multicolumn{5}{l}{ } &  \\

                     \vernacular{
                    yá{\downstep}kháá[ndi]}  &   
                     \gloss{‘fear’}  &     &   
                     \vernacular{
                    yá{\downstep}kháá[mbechɛ]}  &   
                     \gloss{‘shave’}  &  \\

                     \vernacular{
                    yá{\downstep}kháá[ndeerɛ]}  &   
                     \gloss{‘bring’}  &     &   
                     \vernacular{
                    yá{\downstep}kháá[khalachɛ]}  &   
                     \gloss{‘cut’}  &  \\

                     \vernacular{
                    yá{\downstep}kháá[sitaachɛ]}  &   
                     \gloss{‘accuse’}  &     &   
                     \vernacular{
                    yá{\downstep}kháá[mboolitsɛ]}  &   
                     \gloss{‘seduce’}  &  \\

                     \vernacular{
                    yá{\downstep}kháá[ndzuunzuunɛ]}  &   
                     \gloss{‘suck’}  &     &   
                     \vernacular{
                    yá{\downstep}kháá[mbohololɛ]}  &   
                     \gloss{‘untie’}  &  \\

                     \vernacular{
                    yá{\downstep}kháá[mboyong’anɛ]}  &   
                     \gloss{‘go around’}  &     &   
                     \vernacular{
                    yá{\downstep}kháá[ng’ong’oolitsɛ]}  &   
                     \gloss{‘tease’}  &  \\

                     \vernacular{
                    yá{\downstep}kháá[shiling’anyinyɛ]}  &   
                     \gloss{‘bend’}  &     &     &     &  \\
\end{tabular}
%\caption{\nocaption}
     
\begin{tabular}{llllll}  
  \multicolumn{5}{l}{
                     \vernacular{(152) /H/
                    V-Initial + OP
                    } \gloss{‘s/he
                    will...me’} } &  \\
\multicolumn{5}{l}{ } &  \\

                     \vernacular{
                    yá{\downstep}kháá[nzirɛ]}  &   
                     \gloss{‘kill’}  &     &   
                     \vernacular{
                    yá{\downstep}kháá[nzikoombɛ]}  &   
                     \gloss{‘admire’}  &  \\

                     \vernacular{
                    yá{\downstep}kháá[nzisiachɛ]}  &   
                     \gloss{‘smack’}  &     &   
                     \vernacular{
                    yá{\downstep}kháá[nzononyinyɛ]}  &   
                     \gloss{‘spoil’}  &  \\

                     \vernacular{
                    yá{\downstep}kháá[nzabukhanyinyɛ]}  &   
                     \gloss{‘separate’}  &     &     &     &  \\
\end{tabular}
%\caption{\nocaption}
     
\begin{tabular}{llllll}  
  \multicolumn{5}{l}{
                     \vernacular{(153) /Ø/
                    C-Initial + OP
                    } \gloss{‘s/he
                    will...me’} } &  \\
\multicolumn{5}{l}{ } &  \\

                     \vernacular{
                    yá{\downstep}kháá[ndeshɛ]}  &   
                     \gloss{‘leave’}  &  \\

                     \vernacular{
                    yá{\downstep}kháá[noondɛ]}  &   
                     \gloss{‘follow’}  &  \\

                     \vernacular{
                    yá{\downstep}kháá[ngumilɛ]}  &   
                     \gloss{‘hold’}  &  \\

                     \vernacular{
                    yá{\downstep}kháá[ndakhuulɛ]}  &   
                     \gloss{‘release’}  &  \\

                     \vernacular{
                    yá{\downstep}kháá[seebulɛ]}  &   
                     \gloss{‘say bye to’}  &  \\

                     \vernacular{
                    yá{\downstep}kháá[mboombelitsɛ]}  &   
                     \gloss{‘comfort’}  &  \\

                     \vernacular{
                    yá{\downstep}kháá[siinjilitsɛ]}  &   
                     \gloss{
                    ‘make...stand’}  &  \\

                     \vernacular{
                    yá{\downstep}kháá[rootsarootsɛ]}  &   
                     \gloss{‘poke
                    (iter)’}  &  \\

                     \vernacular{
                    yá{\downstep}kháá[ngalukhanyinyɛ]}  &   
                     \gloss{
                    ‘turn...over’}  &  \\
\end{tabular}
%\caption{\nocaption}
     
\begin{tabular}{llllll}  
  \multicolumn{5}{l}{
                     \vernacular{(154) /Ø/
                    V-Initial + OP
                    } \gloss{‘s/he
                    will...me’} } &  \\
\multicolumn{5}{l}{ } &  \\

                     \vernacular{
                    yá{\downstep}kháá[nzenyɛ]}  &   
                     \gloss{‘want’}  &     &   
                     \vernacular{
                    yá{\downstep}kháá[nzeyelɛ]}  &   
                     \gloss{‘wipe for’}  &  \\

                     \vernacular{
                    yá{\downstep}kháá[nzabaarɛ]}  &   
                     \gloss{‘grope’}  &     &   
                     \vernacular{
                    yá{\downstep}kháá[nyambakhanɛ]}  &   
                     \gloss{‘refuse’}  &  \\

                     \vernacular{
                    yá{\downstep}kháá[nzeleelitsɛ]}  &   
                     \gloss{
                    ‘carry...hanging’}  &  \\
\end{tabular}
%\caption{\nocaption}
     
\begin{tabular}{llllll}  
  \multicolumn{5}{l}{
                     \vernacular{(155) /H/
                    C-Initial + OP
                    } \gloss{‘s/he
                    will...him/herself’} } &  \\
\multicolumn{5}{l}{ } &  \\

                     \vernacular{
                    yá{\downstep}khíí[rɛ]}  &   
                     \gloss{‘bury’}  &     &   
                     \vernacular{
                    yá{\downstep}khíí[bechɛ]}  &   
                     \gloss{‘shave’}  &  \\

                     \vernacular{
                    yá{\downstep}khíí[suunjɛ]}  &   
                     \gloss{‘hang’}  &     &   
                     \vernacular{
                    yá{\downstep}khíí[khalachɛ]}  &   
                     \gloss{‘cut’}  &  \\

                     \vernacular{
                    yá{\downstep}khíí[sitaachɛ]}  &   
                     \gloss{‘accuse’}  &     &   
                     \vernacular{
                    yá{\downstep}khíí[saanditsɛ]}  &   
                     \gloss{‘thank’}  &  \\

                     \vernacular{
                    yá{\downstep}khíí[tsuunzuunɛ]}  &   
                     \gloss{‘suck’}  &     &   
                     \vernacular{
                    yá{\downstep}khíí[bohololɛ]}  &   
                     \gloss{‘untie’}  &  \\
\end{tabular}
%\caption{\nocaption}
     
\begin{tabular}{llllll}  
  \multicolumn{5}{l}{
                     \vernacular{(156) /H/
                    V-Initial + OP
                    } \gloss{‘s/he
                    will...him/herself’} } &  \\
\multicolumn{5}{l}{ } &  \\

                     \vernacular{
                    yá{\downstep}khíí[yirɛ]}  &   
                     \gloss{‘kill’}  &     &   
                     \vernacular{
                    yá{\downstep}khíí[yikoombɛ]}  &   
                     \gloss{‘admire’}  &  \\

                     \vernacular{
                    yá{\downstep}khíí[yisiachɛ]}  &   
                     \gloss{‘smack’}  &     &   
                     \vernacular{
                    yá{\downstep}khíí[yononyinyɛ]}  &   
                     \gloss{‘spoil’}  &  \\

                     \vernacular{
                    yá{\downstep}khíí[yabukhanyinyɛ]}  &   
                     \gloss{‘separate’}  &     &     &     &  \\
\end{tabular}
%\caption{\nocaption}
     
\begin{tabular}{llllll}  
  \multicolumn{5}{l}{
                     \vernacular{(157) /Ø/
                    C-Initial + OP
                    } \gloss{‘s/he
                    will...him/herself’} } &  \\
\multicolumn{5}{l}{ } &  \\

                     \vernacular{
                    yá{\downstep}khíí[leshɛ]}  &   
                     \gloss{‘leave’}  &     &   
                     \vernacular{
                    yá{\downstep}khíí[siinjɛ]}  &   
                     \gloss{‘bathe’}  &  \\

                     \vernacular{
                    yá{\downstep}khíí[kumilɛ]}  &   
                     \gloss{‘hold’}  &     &   
                     \vernacular{
                    yá{\downstep}khíí[naabulɛ]}  &   
                     \gloss{‘undress’}  &  \\

                     \vernacular{
                    yá{\downstep}khíí[lakhuulɛ]}  &   
                     \gloss{‘release’}  &     &   
                     \vernacular{
                    yá{\downstep}khíí[hoombelitsɛ]}  &   
                     \gloss{‘comfort’}  &  \\

                     \vernacular{
                    yá{\downstep}khíí[siinjilitsɛ]}  &   
                     \gloss{
                    ‘make...stand’}  &     &   
                     \vernacular{
                    yá{\downstep}khíí[rootsarootsɛ]}  &   
                     \gloss{‘poke
                    (iter)’}  &  \\

                     \vernacular{
                    yá{\downstep}khíí[kalukhanyinyɛ]}  &   
                     \gloss{
                    ‘turn...over’}  &  \\
\end{tabular}
%\caption{\nocaption}
     
\begin{tabular}{llllll}  
  \multicolumn{5}{l}{
                     \vernacular{(158) /Ø/
                    V-Initial + OP
                    } \gloss{‘s/he
                    will...him/herself’} } &  \\
\multicolumn{5}{l}{ } &  \\

                     \vernacular{
                    yá{\downstep}khíí[yalɛ]}  &   
                     \gloss{‘expose’}  &     &   
                     \vernacular{
                    yá{\downstep}khíí[yeyelɛ]}  &   
                     \gloss{‘wipe for’}  &  \\

                     \vernacular{
                    yá{\downstep}khíí[yabaarɛ]}  &   
                     \gloss{‘grope’}  &     &   
                     \vernacular{
                    yá{\downstep}khíí[yambakhanɛ]}  &   
                     \gloss{‘despise’}  &  \\

                     \vernacular{
                    yá{\downstep}khíí[yeleelitsɛ]}  &   
                     \gloss{‘hang’}  &     &     &     &  \\
\end{tabular}
%\caption{\nocaption}
     
\begin{tabular}{llllll}  
  \multicolumn{5}{l}{
                     \vernacular{(159) /H/
                    C-Initial + OP + OP
                    } \gloss{‘s/he
                    will...him/her for me’} } &  \\
\multicolumn{5}{l}{ } &  \\

                     \vernacular{
                    yá{\downstep}khámúu[ndeelɛ]}  &   
                     \gloss{‘bury’}  &     &   
                     \vernacular{
                    yá{\downstep}khámúu[mbechelɛ]}  &   
                     \gloss{‘shave’}  &  \\

                     \vernacular{
                    yá{\downstep}khámúu[ndeelelɛ]}  &   
                     \gloss{‘bring’}  &     &   
                     \vernacular{
                    yá{\downstep}khámúu[khalachilɛ]}  &   
                     \gloss{‘cut’}  &  \\

                     \vernacular{
                    yá{\downstep}khámúu[sitaachilɛ]}  &   
                     \gloss{‘accuse’}  &     &   
                     \vernacular{
                    yá{\downstep}khámúu[mboolitsilɛ]}  &   
                     \gloss{‘seduce’}  &  \\

                     \vernacular{
                    yá{\downstep}khámúu[mbohololelɛ]}  &   
                     \gloss{‘untie’}  &     &     &     &  \\
\end{tabular}
%\caption{\nocaption}
     
\begin{tabular}{llllll}  
  \multicolumn{5}{l}{
                     \vernacular{(160) /H/
                    V-Initial + OP + OP
                    } \gloss{‘s/he
                    will...him/her for me’} } &  \\
\multicolumn{5}{l}{ } &  \\

                     \vernacular{
                    yá{\downstep}khámúu[nzirilɛ]}  &   
                     \gloss{‘kill’}  &  \\

                     \vernacular{
                    yá{\downstep}khámúu[nzechitsilɛ]}  &   
                     \gloss{‘admire’}  &  \\

                     \vernacular{
                    yá{\downstep}khámúu[nzisiachilɛ]}  &   
                     \gloss{‘smack’}  &  \\

                     \vernacular{
                    yá{\downstep}khámúu[nzononyinyilɛ]}  &   
                     \gloss{‘spoil’}  &  \\

                     \vernacular{
                    yá{\downstep}khámúu[nzabukhanyinyilɛ]}  &   
                     \gloss{‘separate’}  &  \\
\end{tabular}
%\caption{\nocaption}
     
\begin{tabular}{llllll}  
  \multicolumn{5}{l}{
                     \vernacular{(161) /Ø/
                    C-Initial + OP + OP
                    } \gloss{‘s/he
                    will...him/her for me’} } &  \\
\multicolumn{5}{l}{ } &  \\

                     \vernacular{
                    yá{\downstep}khámúu[nziilɛ]}  &   
                     \gloss{‘go for’}  &  \\

                     \vernacular{
                    yá{\downstep}khámúu[ndeshelɛ]}  &   
                     \gloss{‘leave’}  &  \\

                     \vernacular{
                    yá{\downstep}khámúu[noondelɛ]}  &   
                     \gloss{‘follow’}  &  \\

                     \vernacular{
                    yá{\downstep}khámúu[ngumililɛ]}  &   
                     \gloss{‘hold’}  &  \\

                     \vernacular{
                    yá{\downstep}khámúu[ndakhuulilɛ]}  &   
                     \gloss{‘release’}  &  \\

                     \vernacular{
                    yá{\downstep}khámúu[seebulilɛ]}  &   
                     \gloss{‘say bye to’}  &  \\

                     \vernacular{
                    yá{\downstep}khámúu[mboombelitsilɛ]}  &   
                     \gloss{‘comfort’}  &  \\

                     \vernacular{
                    yá{\downstep}khámúu[siinjilitsilɛ]}  &   
                     \gloss{
                    ‘make...stand’}  &  \\
\end{tabular}
%\caption{\nocaption}
     
\begin{tabular}{llllll}  
  \multicolumn{5}{l}{
                     \vernacular{(162) /Ø/
                    V-Initial + OP + OP
                    } \gloss{‘s/he
                    will...him/her \ob mu-\cb  / it
                    } } &  \\
\multicolumn{5}{l}{ } &  \\

                     \vernacular{
                    yá{\downstep}khábúu[nzalilɛ]}  &   
                     \gloss{‘display’}  &     &   
                     \vernacular{
                    yá{\downstep}khákúu[nzakhitsilɛ]}  &   
                     \gloss{‘light’}  &  \\

                     \vernacular{
                    yá{\downstep}khábúu[nziluulilɛ]}  &   
                     \gloss{‘winnow’}  &     &   
                     \vernacular{
                    yá{\downstep}khálúu[nzitsulitsilɛ]}  &   
                     \gloss{‘fill’}  &  \\

                     \vernacular{
                    yá{\downstep}khákúu[nzeleelitsilɛ]}  &   
                     \gloss{‘hang’}  &     &     &     &  \\
\end{tabular}
%\caption{\nocaption}
     
\begin{tabular}{lll}  
  \multicolumn{2}{l}{
                     \vernacular{(163) /H/
                    C-Initial Phrase-Medial} \gloss{‘s/he will...the
                    man \ob musáatsa\cb  /} } &  \\
\multicolumn{2}{l}{
                     \gloss{the girl
                    \ob mukháana\cb  / the boy \ob mú{\downstep}yáyi\cb  / someone
                    \ob muundu\cb ’} } &  \\

                     \vernacular{yá{\downstep}khá[rɛ́]
                    musáatsa/mú{\downstep}yáyi/muundu}  &   
                     \gloss{‘bury’}  &  \\

                     \vernacular{yá{\downstep}khá[béchɛ]
                    musáatsa/mú{\downstep}yáyi/muundu}  &   
                     \gloss{‘shave’}  &  \\

                     \vernacular{yá{\downstep}khá[léerɛ]
                    musáatsa/mú{\downstep}yáyi/muundu}  &   
                     \gloss{‘bring’}  &  \\

                     \vernacular{
                    yá{\downstep}khá[khálachɛ]
                    musáatsa/mú{\downstep}yáyi/muundu}  &   
                     \gloss{‘cut’}  &  \\

                     \vernacular{
                    yá{\downstep}khá[sítaachɛ]
                    musáatsa/mú{\downstep}yáyi/muundu}  &   
                     \gloss{‘accuse’}  &  \\

                     \vernacular{
                    yá{\downstep}khá[bóolitsɛ]
                    mukháana/mú{\downstep}yáyi/muundu}  &   
                     \gloss{‘seduce’}  &  \\

                     \vernacular{
                    yá{\downstep}khá[tsúunzuunɛ]
                    musáatsa/mú{\downstep}yáyi/muundu}  &   
                     \gloss{‘suck’}  &  \\

                     \vernacular{
                    yá{\downstep}khá[bóhololɛ]
                    musáatsa/mú{\downstep}yáyi/muundu}  &   
                     \gloss{‘untie’}  &  \\

                     \vernacular{
                    yá{\downstep}khá[bóyong’anɛ]
                    musáatsa/mú{\downstep}yáyi/muundu}  &   
                     \gloss{‘go around’}  &  \\
\end{tabular}
%\caption{\nocaption}
     
\begin{tabular}{lll}  
  \multicolumn{2}{l}{
                     \vernacular{(164) /Ø/
                    C-Initial Phrase-Medial} \gloss{‘s/he will...the
                    man \ob musáatsa\cb  /} } &  \\
\multicolumn{2}{l}{
                     \gloss{the boy
                    \ob mú{\downstep}yáyi\cb  / someone \ob muundu\cb ’} } &  \\

                     \vernacular{yákha[tsi]
                    musáatsa/mú{\downstep}yáyi/muundu}  &   
                     \gloss{‘go for’}  &  \\

                     \vernacular{yákha[leshɛ]
                    musáatsa/mú{\downstep}yáyi/muundu}  &   
                     \gloss{‘leave’}  &  \\

                     \vernacular{yákha[loondɛ]
                    musáatsa/mú{\downstep}yáyi/muundu}  &   
                     \gloss{‘follow’}  &  \\

                     \vernacular{yákha[kumilɛ]
                    musáatsa/mú{\downstep}yáyi/muundu}  &   
                     \gloss{‘hold’}  &  \\

                     \vernacular{yákha[lakhuulɛ]
                    musáatsa/mú{\downstep}yáyi/muundu}  &   
                     \gloss{‘release’}  &  \\

                     \vernacular{yákha[seebulɛ]
                    musáatsa/mú{\downstep}yáyi/muundu}  &   
                     \gloss{‘say bye to’}  &  \\

                     \vernacular{yákha[kalukhitsɛ]
                    musáatsa/mú{\downstep}yáyi/muundu}  &   
                     \gloss{‘return’}  &  \\

                     \vernacular{
                    yákha[roostarootsɛ]
                    musáatsa/mú{\downstep}yáyi/muundu}  &   
                     \gloss{‘poke
                    (iter)’}  &  \\
\end{tabular}
%\caption{\nocaption}
     
\begin{tabular}{lll}  
  \multicolumn{2}{l}{
                     \vernacular{(165) /H/
                    C-Initial +OP Phrase-Medial} \gloss{‘s/he will...the
                    man \ob musáatsa\cb  /} } &  \\
\multicolumn{2}{l}{
                     \gloss{the girl /
                    \ob mukháana\cb  / the boy \ob mú{\downstep}yáyi\cb  / someone
                    \ob muundu\cb  for him/her’} } &  \\

                     \vernacular{yá{\downstep}khámú[rɛ]
                    musáatsa/mú{\downstep}yáyi/muundu}  &   
                     \gloss{‘bury’}  &  \\

                     \vernacular{yá{\downstep}khámú[bechɛ]
                    musáatsa/mú{\downstep}yáyi/muundu}  &   
                     \gloss{‘shave’}  &  \\

                     \vernacular{yá{\downstep}khámú[leerɛ]
                    musáatsa/mú{\downstep}yáyi/muundu}  &   
                     \gloss{‘bring’}  &  \\

                     \vernacular{
                    yá{\downstep}khámú[khalachɛ]
                    musáatsa/mú{\downstep}yáyi/muundu}  &   
                     \gloss{‘cut’}  &  \\

                     \vernacular{
                    yá{\downstep}khámú[sitaachɛ]
                    musáatsa/mú{\downstep}yáyi/muundu}  &   
                     \gloss{‘accuse’}  &  \\

                     \vernacular{
                    yá{\downstep}khámú[boolitsɛ]
                    mukáana/mú{\downstep}yáyi/muundu}  &   
                     \gloss{‘seduce’}  &  \\

                     \vernacular{
                    yá{\downstep}khámú[tsuunzuunɛ]
                    musáatsa/mú{\downstep}yáyi/muundu}  &   
                     \gloss{‘suck’}  &  \\

                     \vernacular{
                    yá{\downstep}khámú[bohololɛ]
                    musáatsa/mú{\downstep}yáyi/muundu}  &   
                     \gloss{‘untie’}  &  \\

                     \vernacular{
                    yá{\downstep}khámú[boyong’anɛ]
                    musáatsa/mú{\downstep}yáyi/muundu}  &   
                     \gloss{‘go around’}  &  \\

                     \vernacular{
                    yá{\downstep}khámú[shiling’anyinyɛ]
                    musáatsa/mú{\downstep}yáyi/muundu}  &   
                     \gloss{‘silence’}  &  \\
\end{tabular}
%\caption{\nocaption}
     
\begin{tabular}{lll}  
  \multicolumn{2}{l}{
                     \vernacular{(166) /Ø/
                    C-Initial +OP Phrase-Medial} \gloss{‘s/he will...the
                    man \ob musáatsa\cb  /} } &  \\
\multicolumn{2}{l}{
                     \gloss{the boy
                    \ob mú{\downstep}yáyi\cb  / someone \ob muundu\cb  for
                    him/her’} } &  \\

                     \vernacular{yá{\downstep}khámú[tsi]
                    musáatsa/mú{\downstep}yáyi/muundu}  &   
                     \gloss{‘go for’}  &  \\

                     \vernacular{yá{\downstep}khámú[leshɛ]
                    musáatsa/mú{\downstep}yáyi/muundu}  &   
                     \gloss{‘leave’}  &  \\

                     \vernacular{
                    yá{\downstep}khámú[loondɛ]
                    musáatsa/mú{\downstep}yáyi/muundu}  &   
                     \gloss{‘follow’}  &  \\

                     \vernacular{
                    yá{\downstep}khámú[kumilɛ]
                    musáatsa/mú{\downstep}yáyi/muundu}  &   
                     \gloss{‘hold’}  &  \\

                     \vernacular{
                    yá{\downstep}khámú[lakhuulɛ]
                    musáatsa/mú{\downstep}yáyi/muundu}  &   
                     \gloss{‘release’}  &  \\

                     \vernacular{
                    yá{\downstep}khámú[seebulɛ]
                    musáatsa/mú{\downstep}yáyi/muundu}  &   
                     \gloss{‘say bye to’}  &  \\

                     \vernacular{
                    yá{\downstep}khámú[kalukhitsɛ]
                    musáatsa/mú{\downstep}yáyi/muundu}  &   
                     \gloss{‘return’}  &  \\

                     \vernacular{
                    yá{\downstep}khámú[rootsarootsɛ]
                    musáatsa/mú{\downstep}yáyi/muundu}  &   
                     \gloss{‘poke
                    (iter)’}  &  \\
\end{tabular}
%\caption{\nocaption}
     
\begin{tabular}{lll}  
  \multicolumn{2}{l}{
                     \vernacular{(167) /H/
                    C-Initial +OP + OP
                    } \gloss{‘s/he will...the
                    man \ob musáatsa\cb  /} } &  \\
\multicolumn{2}{l}{
                     \gloss{the boy
                    \ob mú{\downstep}yáyi\cb  / someone \ob muundu\cb  for him/her for
                    me’} } &  \\

                     \vernacular{
                    yá{\downstep}khámúu[ndeelɛ]
                    musáatsa/mú{\downstep}yáyi/muundu}  &   
                     \gloss{‘bury’}  &  \\

                     \vernacular{
                    yá{\downstep}khámúu[mbechelɛ]
                    musáatsa/mú{\downstep}yáyi/muundu}  &   
                     \gloss{‘shave’}  &  \\

                     \vernacular{
                    yá{\downstep}khámúu[ndeelelɛ]
                    musáatsa/mú{\downstep}yáyi/muundu}  &   
                     \gloss{‘bring’}  &  \\

                     \vernacular{
                    yá{\downstep}khámúu[khalachilɛ]
                    musáatsa/mú{\downstep}yáyi/muundu}  &   
                     \gloss{‘cut’}  &  \\

                     \vernacular{
                    yá{\downstep}khámúu[sitaachilɛ]
                    musáatsa/mú{\downstep}yáyi/muundu}  &   
                     \gloss{‘accuse’}  &  \\

                     \vernacular{
                    yá{\downstep}khámúu[mbohololelɛ]
                    musáatsa/mú{\downstep}yáyi/muundu}  &   
                     \gloss{‘untie’}  &  \\
\end{tabular}
%\caption{\nocaption}
     
\begin{tabular}{lll}  
  \multicolumn{2}{l}{
                     \vernacular{(168) /Ø/
                    C-Initial +OP + OP
                    } \gloss{‘s/he will...the
                    man \ob musáatsa\cb  /} } &  \\
\multicolumn{2}{l}{
                     \gloss{the boy
                    \ob mú{\downstep}yáyi\cb  / someone \ob muundu\cb  for him/her for
                    me’} } &  \\

                     \vernacular{
                    yá{\downstep}khámúu[nziilɛ]
                    musáatsa/mú{\downstep}yáyi/muundu}  &   
                     \gloss{‘go for’}  &  \\

                     \vernacular{
                    yá{\downstep}khámúu[ndeshelɛ]
                    musáatsa/mú{\downstep}yáyi/muundu}  &   
                     \gloss{‘leave’}  &  \\

                     \vernacular{
                    yá{\downstep}khámúu[noondelɛ]
                    musáatsa/mú{\downstep}yáyi/muundu}  &   
                     \gloss{‘follow’}  &  \\

                     \vernacular{
                    yá{\downstep}khámúu[ngumililɛ]
                    musáatsa/mú{\downstep}yáyi/muundu}  &   
                     \gloss{‘hold’}  &  \\

                     \vernacular{
                    yá{\downstep}khámúu[ndakhuulilɛ]
                    musáatsa/mú{\downstep}yáyi/muundu}  &   
                     \gloss{‘release’}  &  \\

                     \vernacular{
                    yá{\downstep}khámúu[seebulilɛ]
                    musáatsa/mú{\downstep}yáyi/muundu}  &   
                     \gloss{‘say bye to’}  &  \\
\end{tabular}
%\caption{\nocaption}
    

\subsection{Remote Future Negative: Pattern 1b [JI]
              }\label{sec:sRemFutNeg}


\begin{tabular}{llllll}  
  \multicolumn{5}{l}{
                     \vernacular{(169) /H/
                    C-Initial} \gloss{‘s/he will
                    not...’} } &  \\
\multicolumn{5}{l}{ } &  \\

                     \vernacular{yá{\downstep}khá[ngwí]
                    {\downstep}tá}  &   
                     \gloss{‘drink’}  &     &   
                     \vernacular{yá{\downstep}khá[bé{\downstep}chɛ́]
                    tá}  &   
                     \gloss{‘shave’}  &  \\

                     \vernacular{
                    yá{\downstep}khá[lé{\downstep}érɛ́] tá}  &   
                     \gloss{‘bring’}  &     &   
                     \vernacular{
                    yá{\downstep}khá[khá{\downstep}láchɛ́] tá}  &   
                     \gloss{‘cut’}  &  \\

                     \vernacular{
                    yá{\downstep}khá[sí{\downstep}tááchɛ́] tá}  &   
                     \gloss{‘accuse’}  &     &   
                     \vernacular{
                    yá{\downstep}khá[bó{\downstep}ólítsɛ́] tá}  &   
                     \gloss{‘seduce’}  &  \\

                     \vernacular{
                    yá{\downstep}khá[tsú{\downstep}únzúúnɛ́] tá}  &   
                     \gloss{‘suck’}  &     &   
                     \vernacular{
                    yá{\downstep}khá[bó{\downstep}hólólɛ́] tá}  &   
                     \gloss{‘untie’}  &  \\

                     \vernacular{
                    yá{\downstep}khá[bó{\downstep}yóng’ánɛ́] tá}  &   
                     \gloss{‘go around’}  &     &   
                     \vernacular{
                    yá{\downstep}khá[ng’ó{\downstep}ng’óólítsɛ́] tá}  &   
                     \gloss{‘tease’}  &  \\
\end{tabular}
%\caption{\nocaption}
     
\begin{tabular}{llllll}  
  \multicolumn{5}{l}{
                     \vernacular{(170) /Ø/
                    C-Initial} \gloss{‘s/he will
                    not...’} } &  \\
\multicolumn{5}{l}{ } &  \\

                     \vernacular{yá{\downstep}khá[tsí]
                    tá}  &   
                     \gloss{‘go’}  &     &   
                     \vernacular{yá{\downstep}khá[leshɛ́]
                    tá}  &   
                     \gloss{‘leave’}  &  \\

                     \vernacular{yá{\downstep}khá[réébɛ́]
                    tá}  &   
                     \gloss{‘ask’}  &     &   
                     \vernacular{
                    yá{\downstep}khá[kúmílɛ́] tá}  &   
                     \gloss{‘hold’}  &  \\

                     \vernacular{
                    yá{\downstep}khá[sásánɛ́] tá}  &   
                     \gloss{‘resemble’}  &     &   
                     \vernacular{
                    yá{\downstep}khá[lákhúúlɛ́] tá}  &   
                     \gloss{‘release’}  &  \\

                     \vernacular{
                    yá{\downstep}khá[séébúlɛ́] tá}  &   
                     \gloss{‘say bye’}  &     &   
                     \vernacular{
                    yá{\downstep}khá[kálúkhítsɛ́] tá}  &   
                     \gloss{‘return’}  &  \\

                     \vernacular{
                    yá{\downstep}khá[róótsɛ́róótsɛ́] tá}  &   
                     \gloss{‘poke
                    (iter)’}  &     &   
                     \vernacular{
                    yá{\downstep}khá[kálúkhányínyɛ́] tá}  &   
                     \gloss{‘turn over’}  &  \\
\end{tabular}
%\caption{\nocaption}
     
\begin{tabular}{llllll}  
  \multicolumn{5}{l}{
                     \vernacular{(171) /H/
                    C-Initial + OP} \gloss{‘s/he will
                    not...him/her’} } &  \\
\multicolumn{5}{l}{ } &  \\

                     \vernacular{yá{\downstep}khámú[{\downstep}rɛ́]
                    tá}  &   
                     \gloss{‘bury’}  &  \\

                     \vernacular{
                    yá{\downstep}khámú[{\downstep}béchɛ́] tá}  &   
                     \gloss{‘shave’}  &  \\

                     \vernacular{
                    yá{\downstep}khámú[{\downstep}léérɛ́] tá}  &   
                     \gloss{‘bring’}  &  \\

                     \vernacular{
                    yá{\downstep}khámú[{\downstep}kháláchɛ́] tá}  &   
                     \gloss{‘cut’}  &  \\

                     \vernacular{
                    yá{\downstep}khámú[{\downstep}sítááchɛ́] tá}  &   
                     \gloss{‘accuse’}  &  \\

                     \vernacular{
                    yá{\downstep}khámú[{\downstep}bóólítsɛ́] tá}  &   
                     \gloss{‘seduce’}  &  \\

                     \vernacular{
                    yá{\downstep}khámú[{\downstep}tsúúnzúúnɛ́] tá}  &   
                     \gloss{‘suck’}  &  \\

                     \vernacular{
                    yá{\downstep}khámú[{\downstep}bóhólólɛ́] tá}  &   
                     \gloss{‘untie’}  &  \\

                     \vernacular{
                    yá{\downstep}khámú[{\downstep}bóyóng’ánɛ́] tá}  &   
                     \gloss{‘go around’}  &  \\

                     \vernacular{
                    yá{\downstep}khámú[{\downstep}ng’óng’óólítsɛ́]
                    tá}  &   
                     \gloss{‘tease’}  &  \\

                     \vernacular{
                    yá{\downstep}khámú[{\downstep}shílíng’ányínyɛ́]
                    tá}  &   
                     \gloss{‘silence’}  &  \\
\end{tabular}
%\caption{\nocaption}
     
\begin{tabular}{llllll}  
  \multicolumn{5}{l}{
                     \vernacular{(172) /Ø/
                    C-Initial + OP} \gloss{‘s/he will
                    not...him/her \ob mu-\cb  / them
                    } } &  \\
\multicolumn{5}{l}{ } &  \\

                     \vernacular{yá{\downstep}khámú[{\downstep}tsí]
                    {\downstep}tá}  &   
                     \gloss{‘go for’}  &  \\

                     \vernacular{
                    yá{\downstep}khámú[{\downstep}léshɛ́] {\downstep}tá}  &   
                     \gloss{‘leave’}  &  \\

                     \vernacular{
                    yá{\downstep}khámú[{\downstep}lóóndɛ́] tá}  &   
                     \gloss{‘follow’}  &  \\

                     \vernacular{
                    yá{\downstep}khámú[{\downstep}kúmílɛ́] tá}  &   
                     \gloss{‘hold’}  &  \\

                     \vernacular{
                    yá{\downstep}khámú[{\downstep}lákhúúlɛ́] tá}  &   
                     \gloss{‘release’}  &  \\

                     \vernacular{
                    yá{\downstep}khámú[{\downstep}séébúlɛ́] tá}  &   
                     \gloss{‘say bye to’}  &  \\

                     \vernacular{
                    yá{\downstep}khámú[{\downstep}kálúkhítsɛ́] tá}  &   
                     \gloss{‘return’}  &  \\

                     \vernacular{
                    yá{\downstep}khámú[{\downstep}síínjílítsɛ́] tá}  &   
                     \gloss{
                    ‘make...stand’}  &  \\

                     \vernacular{
                    yá{\downstep}khámú[{\downstep}róótsɛ́róótsɛ́] tá}  &   
                     \gloss{‘poke
                    (iter)’}  &  \\

                     \vernacular{
                    yá{\downstep}khámú[{\downstep}kálúkhányínyɛ́] tá}  &   
                     \gloss{
                    ‘turn...over’}  &  \\
\end{tabular}
%\caption{\nocaption}
     
\begin{tabular}{llllll}  
  \multicolumn{5}{l}{
                     \vernacular{(173) /H/
                    C-Initial + OP + OP
                    } \gloss{‘s/he will
                    not...him/her for me’} } &  \\
\multicolumn{5}{l}{ } &  \\

                     \vernacular{
                    yá{\downstep}khámú{\downstep}ú[ndéélɛ́] tá}  &   
                     \gloss{‘bury’}  &     &   
                     \vernacular{
                    yá{\downstep}khámú{\downstep}ú[mbéchélɛ́] tá}  &   
                     \gloss{‘shave’}  &  \\

                     \vernacular{
                    yá{\downstep}khámú{\downstep}ú[ndéelélɛ́] tá}  &   
                     \gloss{‘bring’}  &     &   
                     \vernacular{
                    yá{\downstep}khámú{\downstep}ú[kháláchílɛ́] tá}  &   
                     \gloss{‘cut’}  &  \\

                     \vernacular{
                    yá{\downstep}khámú{\downstep}ú[sítááchílɛ́] tá}  &   
                     \gloss{‘accuse’}  &     &   
                     \vernacular{
                    yá{\downstep}khámú{\downstep}ú[mbóólítsílɛ́] tá}  &   
                     \gloss{‘seduce’}  &  \\

                     \vernacular{
                    yá{\downstep}khámú{\downstep}ú[mbóhólólélɛ́] tá}  &   
                     \gloss{‘untie’}  &     &     &     &  \\
\end{tabular}
%\caption{\nocaption}
     
\begin{tabular}{llllll}  
  \multicolumn{5}{l}{
                     \vernacular{(174) /Ø/
                    C-Initial + OP + OP
                    } \gloss{‘s/he will
                    not...him/her for me’} } &  \\
\multicolumn{5}{l}{ } &  \\

                     \vernacular{
                    yá{\downstep}khámú{\downstep}ú[nzíílɛ́] tá}  &   
                     \gloss{‘go for’}  &  \\

                     \vernacular{
                    yá{\downstep}khámú{\downstep}ú[ndéshélɛ́] tá}  &   
                     \gloss{‘leave’}  &  \\

                     \vernacular{
                    yá{\downstep}khámú{\downstep}ú[nóóndélɛ́] tá}  &   
                     \gloss{‘follow’}  &  \\

                     \vernacular{
                    yá{\downstep}khámú{\downstep}ú[ngúmílílɛ́] tá}  &   
                     \gloss{‘hold’}  &  \\

                     \vernacular{
                    yá{\downstep}khámú{\downstep}ú[ndákhúúlílɛ́] tá}  &   
                     \gloss{‘release’}  &  \\

                     \vernacular{
                    yá{\downstep}khámú{\downstep}ú[séébúlílɛ́] tá}  &   
                     \gloss{‘say bye to’}  &  \\

                     \vernacular{
                    yá{\downstep}khámú{\downstep}ú[síínjílítsílɛ́]
                    tá}  &   
                     \gloss{
                    ‘make...stand’}  &  \\
\end{tabular}
%\caption{\nocaption}
     
\begin{tabular}{lll}  
  \multicolumn{2}{l}{
                     \vernacular{(175) /H/
                    C-Initial Phrase-Medial} \gloss{‘s/he will
                    not...the man \ob musáatsa\cb  /} } &  \\
\multicolumn{2}{l}{
                     \gloss{the girl
                    \ob mukháana\cb  / the wife \ob mukháli\cb  / the boy
                    \ob mú{\downstep}yáyi\cb  / someone \ob muundu\cb ’} } &  \\

                     \vernacular{yá{\downstep}khá[rɛ́]
                    musáatsa/mú{\downstep}yáyi/muundu tá}  &   
                     \gloss{‘bury’}  &  \\

                     \vernacular{yá{\downstep}khá[béchɛ]
                    musáatsa/mú{\downstep}yáyi/muundu tá}  &   
                     \gloss{‘shave’}  &  \\

                     \vernacular{yá{\downstep}khá[léerɛ]
                    musáatsa/mú{\downstep}yáyi/muundu tá}  &   
                     \gloss{‘bring’}  &  \\

                     \vernacular{
                    yá{\downstep}khá[khálachɛ] musáatsa/mú{\downstep}yáyi/muundu
                    tá}  &   
                     \gloss{‘cut’}  &  \\

                     \vernacular{
                    yá{\downstep}khá[sítaachɛ] musáatsa/mú{\downstep}yáyi/muundu
                    tá}  &   
                     \gloss{‘accuse’}  &  \\

                     \vernacular{
                    yá{\downstep}khá[bóolitsɛ] mukháana/mú{\downstep}yáyi/muundu
                    tá}  &   
                     \gloss{‘seduce’}  &  \\

                     \vernacular{
                    yá{\downstep}khá[tsúunzuunɛ]
                    musáatsa/mú{\downstep}yáyi/muundu tá}  &   
                     \gloss{‘suck’}  &  \\

                     \vernacular{
                    yá{\downstep}khá[bóhololɛ] musáatsa/mú{\downstep}yáyi/muundu
                    tá}  &   
                     \gloss{‘untie’}  &  \\

                     \vernacular{
                    yá{\downstep}khá[bóyong’anɛ]
                    musáatsa/mú{\downstep}yáyi/muundu tá}  &   
                     \gloss{‘go around’}  &  \\
\end{tabular}
%\caption{\nocaption}
     
\begin{tabular}{lll}  
  \multicolumn{2}{l}{
                     \vernacular{(176) /Ø/
                    C-Initial Phrase-Medial} \gloss{‘s/he will
                    not...the man \ob musáatsa\cb  /} } &  \\
\multicolumn{2}{l}{
                     \gloss{the boy
                    \ob mú{\downstep}yáyi\cb  / someone \ob muundu\cb ’} } &  \\

                     \vernacular{yákha[tsi]
                    musáatsa/mú{\downstep}yáyi/muundu tá}  &   
                     \gloss{‘go for’}  &  \\

                     \vernacular{yákha[leshɛ]
                    musáatsa/mú{\downstep}yáyi/muundu tá}  &   
                     \gloss{‘leave’}  &  \\

                     \vernacular{yákha[loondɛ]
                    musáatsa/mú{\downstep}yáyi/muundu tá}  &   
                     \gloss{‘follow’}  &  \\

                     \vernacular{yákha[kumilɛ]
                    musáatsa/mú{\downstep}yáyi/muundu tá}  &   
                     \gloss{‘hold’}  &  \\

                     \vernacular{yákha[lakhuulɛ]
                    musáatsa/mú{\downstep}yáyi/muundu tá}  &   
                     \gloss{‘release’}  &  \\

                     \vernacular{yákha[seebulɛ]
                    musáatsa/mú{\downstep}yáyi/muundu tá}  &   
                     \gloss{‘say bye to’}  &  \\

                     \vernacular{yákha[kalukhitsɛ]
                    musáatsa/mú{\downstep}yáyi/muundu tá}  &   
                     \gloss{‘return’}  &  \\

                     \vernacular{
                    yákha[rootsɛrootsɛ] musáatsa/mú{\downstep}yáyi/muundu
                    tá}  &   
                     \gloss{‘poke
                    (iter)’}  &  \\
\end{tabular}
%\caption{\nocaption}
     
\begin{tabular}{lll}  
  \multicolumn{2}{l}{
                     \vernacular{(177) /H/
                    C-Initial +OP Phrase-Medial} \gloss{‘s/he will
                    not...the man \ob musáatsa\cb } } &  \\
\multicolumn{2}{l}{
                     \gloss{the girl
                    \ob mukháana\cb  / the boy \ob mú{\downstep}yáyi\cb  / someone
                    \ob muundu\cb  for him/her’} } &  \\

                     \vernacular{yá{\downstep}khámú[rɛ]
                    musáatsa/mú{\downstep}yáyi/muundu tá}  &   
                     \gloss{‘bury’}  &  \\

                     \vernacular{yá{\downstep}khámú[bechɛ]
                    musáatsa/mú{\downstep}yáyi/muundu tá}  &   
                     \gloss{‘shave’}  &  \\

                     \vernacular{yá{\downstep}khámú[leerɛ]
                    musáatsa/mú{\downstep}yáyi/muundu tá}  &   
                     \gloss{‘bring’}  &  \\

                     \vernacular{
                    yá{\downstep}khámú[khalachɛ]
                    musáatsa/mú{\downstep}yáyi/muundu tá}  &   
                     \gloss{‘cut’}  &  \\

                     \vernacular{
                    yá{\downstep}khámú[sitaachɛ]
                    musáatsa/mú{\downstep}yáyi/muundu tá}  &   
                     \gloss{‘accuse’}  &  \\

                     \vernacular{
                    yá{\downstep}khámú[boolitsɛ]
                    mukháana/mú{\downstep}yáyi/muundu tá}  &   
                     \gloss{‘seduce’}  &  \\

                     \vernacular{
                    yá{\downstep}khámú[tsuunzuunɛ]
                    musáatsa/mú{\downstep}yáyi/muundu tá}  &   
                     \gloss{‘suck’}  &  \\

                     \vernacular{
                    yá{\downstep}khámú[bohololɛ]
                    musáatsa/mú{\downstep}yáyi/muundu tá}  &   
                     \gloss{‘untie’}  &  \\

                     \vernacular{
                    yá{\downstep}khámú[boyong’anɛ]
                    musáatsa/mú{\downstep}yáyi/muundu tá}  &   
                     \gloss{‘go around’}  &  \\
\end{tabular}
%\caption{\nocaption}
     
\begin{tabular}{lll}  
  \multicolumn{2}{l}{
                     \vernacular{(178) /Ø/
                    C-Initial +OP Phrase-Medial} \gloss{‘s/he will
                    not...the man \ob musáatsa\cb  /} } &  \\
\multicolumn{2}{l}{
                     \gloss{the boy
                    \ob mú{\downstep}yáyi\cb  / someone \ob muundu\cb  for
                    him/her’} } &  \\

                     \vernacular{yá{\downstep}khámú[tsi]
                    musáatsa/mú{\downstep}yáyi/muundu tá}  &   
                     \gloss{‘go for’}  &  \\

                     \vernacular{yá{\downstep}khámú[leshɛ]
                    musáatsa/mú{\downstep}yáyi/muundu tá}  &   
                     \gloss{‘leave’}  &  \\

                     \vernacular{
                    yá{\downstep}khámú[loondɛ] musáatsa/mú{\downstep}yáyi/muundu
                    tá}  &   
                     \gloss{‘follow’}  &  \\

                     \vernacular{
                    yá{\downstep}khámú[kumilɛ] musáatsa/mú{\downstep}yáyi/muundu
                    tá}  &   
                     \gloss{‘hold’}  &  \\

                     \vernacular{
                    yá{\downstep}khámú[lakhuulɛ]
                    musáatsa/mú{\downstep}yáyi/muundu tá}  &   
                     \gloss{‘release’}  &  \\

                     \vernacular{
                    yá{\downstep}khámú[seebulɛ] musáatsa/mú{\downstep}yáyi/muundu
                    tá}  &   
                     \gloss{‘say bye to’}  &  \\

                     \vernacular{
                    yá{\downstep}khámú[kalukhitsɛ]
                    musáatsa/mú{\downstep}yáyi/muundu tá}  &   
                     \gloss{‘return’}  &  \\

                     \vernacular{
                    yá{\downstep}khámú[rootsɛrootsɛlɛ]
                    musáatsa/mú{\downstep}yáyi/muundu tá}  &   
                     \gloss{‘poke
                    (iter)’}  &  \\
\end{tabular}
%\caption{\nocaption}
     
\begin{tabular}{lll}  
  \multicolumn{2}{l}{
                     \vernacular{(179) /H/
                    C-Initial +OP + OP
                    } \gloss{‘s/he will
                    not...the man \ob musáatsa\cb  /} } &  \\
\multicolumn{2}{l}{
                     \gloss{the boy
                    \ob mú{\downstep}yáyi\cb  / someone \ob muundu\cb  for him/her for
                    me’} } &  \\

                     \vernacular{
                    yá{\downstep}khámúu[ndeelɛ] musáatsa/mú{\downstep}yáyi/muundu
                    tá}  &   
                     \gloss{‘bury’}  &  \\

                     \vernacular{
                    yá{\downstep}khámúu[mbechelɛ]
                    musáatsa/mú{\downstep}yáyi/muundu tá}  &   
                     \gloss{‘shave’}  &  \\

                     \vernacular{
                    yá{\downstep}khámúu[ndeelelɛ]
                    musáatsa/mú{\downstep}yáyi/muundu tá}  &   
                     \gloss{‘bring’}  &  \\

                     \vernacular{
                    yá{\downstep}khámúu[khalachilɛ]
                    musáatsa/mú{\downstep}yáyi/muundu tá}  &   
                     \gloss{‘cut’}  &  \\

                     \vernacular{
                    yá{\downstep}khámúu[sitaachilɛ]
                    musáatsa/mú{\downstep}yáyi/muundu tá}  &   
                     \gloss{‘accuse’}  &  \\

                     \vernacular{
                    yá{\downstep}khámúu[mbohololelɛ]
                    musáatsa/mú{\downstep}yáyi/muundu tá}  &   
                     \gloss{‘untie’}  &  \\
\end{tabular}
%\caption{\nocaption}
     
\begin{tabular}{lll}  
  \multicolumn{2}{l}{
                     \vernacular{(180) /Ø/
                    C-Initial +OP + OP
                    } \gloss{‘s/he will
                    not...the man \ob musáatsa\cb  /} } &  \\
\multicolumn{2}{l}{
                     \gloss{the boy
                    \ob mú{\downstep}yáyi\cb  / someone \ob muundu\cb  for him/her for
                    me’} } &  \\

                     \vernacular{
                    yá{\downstep}khámúu[nziilɛ] musáatsa/mú{\downstep}yáyi/muundu
                    tá}  &   
                     \gloss{‘go for’}  &  \\

                     \vernacular{
                    yá{\downstep}khámúu[ndeshelɛ]
                    musáatsa/mú{\downstep}yáyi/muundu tá}  &   
                     \gloss{‘leave’}  &  \\

                     \vernacular{
                    yá{\downstep}khámúu[noondelɛ]
                    musáatsa/mú{\downstep}yáyi/muundu tá}  &   
                     \gloss{‘follow’}  &  \\

                     \vernacular{
                    yá{\downstep}khámúu[ngumililɛ]
                    musáatsa/mú{\downstep}yáyi/muundu tá}  &   
                     \gloss{‘hold’}  &  \\

                     \vernacular{
                    yá{\downstep}khámúu[ndakhuulilɛ]
                    musáatsa/mú{\downstep}yáyi/muundu tá}  &   
                     \gloss{‘release’}  &  \\

                     \vernacular{
                    yá{\downstep}khámúu[seebulilɛ]
                    musáatsa/mú{\downstep}yáyi/muundu tá}  &   
                     \gloss{‘say bye to’}  &  \\
\end{tabular}
%\caption{\nocaption}
    

\subsection{Present: Pattern 5a
              }\label{sec:sPres}


\begin{tabular}{llllll}  
  \multicolumn{5}{l}{
                     \vernacular{(181) /H/
                    C-Initial} \gloss{‘s/he
                    is...’} } &  \\
\multicolumn{5}{l}{ } &  \\

                     \vernacular{
                    a[reetsáángá]}  &   
                     \gloss{‘burying’}  &  \\

                     \vernacular{
                    a[ng’weetsáángá]}  &   
                     \gloss{‘drinking’}  &  \\

                     \vernacular{
                    a[khweetsáángá]}  &   
                     \gloss{‘paying
                    dowry’}  &  \\

                     \vernacular{
                    a[liitsáángá]}  &   
                     \gloss{‘eating’}  &  \\

                     \vernacular{
                    a[lumaángá]}  &   
                     \gloss{‘biting’}  &  \\

                     \vernacular{
                    a[bekaángá]}  &   
                     \gloss{‘shaving’}  &  \\

                     \vernacular{
                    a[teekháángá]}  &   
                     \gloss{‘cooking’}  &  \\

                     \vernacular{
                    a[leeráángá]}  &   
                     \gloss{‘bringing’}  &  \\

                     \vernacular{
                    a[khalakáánga]}  &   
                     \gloss{‘cutting’}  &  \\

                     \vernacular{
                    a[kalaangáánga]}  &   
                     \gloss{‘frying’}  &  \\

                     \vernacular{
                    a[sitaakáánga]}  &   
                     \gloss{‘accusing’}  &  \\

                     \vernacular{
                    a[boolitsáánga]}  &   
                     \gloss{‘seducing’}  &  \\

                     \vernacular{
                    a[saanditsáánga]}  &   
                     \gloss{‘thanking’}  &  \\

                     \vernacular{
                    a[tsuunzuunáánga]}  &   
                     \gloss{‘sucking’}  &  \\

                     \vernacular{
                    a[boholóláanga]}  &   
                     \gloss{‘untying’}  &  \\

                     \vernacular{
                    a[boyong’ánáanga]}  &   
                     \gloss{‘going
                    around’}  &  \\

                     \vernacular{
                    a[ng’ong’oolítsáanga]}  &   
                     \gloss{‘teasing’}  &  \\

                     \vernacular{
                    a[linga(ka)nyínyáanga]}  &   
                     \gloss{‘crumpling’}  &  \\
\end{tabular}
%\caption{\nocaption}
     
\begin{tabular}{llllll}  
  \multicolumn{5}{l}{
                     \vernacular{(182) /H/
                    V-Initial} \gloss{‘s/he
                    is...’} } &  \\
\multicolumn{5}{l}{ } &  \\

                     \vernacular{
                    y[iraángá]}  &   
                     \gloss{‘killing’}  &     &   
                     \vernacular{
                    y[ikóómbáanga]}  &   
                     \gloss{‘admiring’}  &  \\

                     \vernacular{
                    y[isíákáanga]}  &   
                     \gloss{‘smacking’}  &     &   
                     \vernacular{
                    y[ikobóláanga]}  &   
                     \gloss{‘belching’}  &  \\

                     \vernacular{
                    y[ononyínyáanga]}  &   
                     \gloss{‘spoiling’}  &     &   
                     \vernacular{
                    y[abukhányínyaanga]}  &   
                     \gloss{‘separating’}  &  \\
\end{tabular}
%\caption{\nocaption}
     
\begin{tabular}{llllll}  
  \multicolumn{5}{l}{
                     \vernacular{(183) /Ø/
                    C-Initial} \gloss{‘s/he
                    is...’} } &  \\
\multicolumn{5}{l}{ } &  \\

                     \vernacular{
                    a[tsiítsaanga]}  &   
                     \gloss{‘going’}  &  \\

                     \vernacular{
                    a[kwiítsaanga]}  &   
                     \gloss{‘falling’}  &  \\

                     \vernacular{
                    a[lekháanga]}  &   
                     \gloss{‘leaving’}  &  \\

                     \vernacular{
                    a[reébáanga]}  &   
                     \gloss{‘asking’}  &  \\

                     \vernacular{
                    a[loóndáanga]}  &   
                     \gloss{‘following’}  &  \\

                     \vernacular{
                    a[kulíkháanga]}  &   
                     \gloss{‘naming’}  &  \\

                     \vernacular{
                    a[homóolaanga]}  &   
                     \gloss{‘massaging’}  &  \\

                     \vernacular{
                    a[lakhúulaanga]}  &   
                     \gloss{‘releasing’}  &  \\

                     \vernacular{
                    a[seébúlaanga]}  &   
                     \gloss{‘saying bye’}  &  \\

                     \vernacular{
                    a[hoómbélitsaanga]}  &   
                     \gloss{‘comforting’}  &  \\

                     \vernacular{
                    a[kalúshítsaanga]}  &   
                     \gloss{‘returning’}  &  \\

                     \vernacular{
                    a[siínjílitsaanga]}  &   
                     \gloss{‘making
                    stand’}  &  \\

                     \vernacular{
                    a[reébáreebaanga]}  &   
                     \gloss{‘asking
                    (iter)’}  &  \\

                     \vernacular{
                    a[kalúkhányinyaanga]}  &   
                     \gloss{‘turning
                    over’}  &  \\

                     \vernacular{
                    a[sebúlúkhanyinyaanga]}  &   
                     \gloss{‘scattering’}  &  \\
\end{tabular}
%\caption{\nocaption}
     
\begin{tabular}{llllll}  
  \multicolumn{5}{l}{
                     \vernacular{(184) /Ø/
                    V-Initial} \gloss{‘s/he
                    is...’} } &  \\
\multicolumn{5}{l}{ } &  \\

                     \vernacular{
                    y[enyáanga]}  &   
                     \gloss{‘wanting’}  &     &   
                     \vernacular{
                    y[eyéláanga]}  &   
                     \gloss{‘wiping for’}  &  \\

                     \vernacular{
                    y[ilúulaanga]}  &   
                     \gloss{‘winnowing’}  &     &   
                     \vernacular{
                    y[ambákhánaanga]}  &   
                     \gloss{‘refusing’}  &  \\

                     \vernacular{
                    y[eléelitsaanga]}  &   
                     \gloss{‘hanging up’}  &     &     &     &  \\
\end{tabular}
%\caption{\nocaption}
     
\begin{tabular}{llllll}  
  \multicolumn{5}{l}{
                     \vernacular{(185) /H/
                    C-Initial + OP} \gloss{‘s/he
                    is...him/her’} } &  \\
\multicolumn{5}{l}{ } &  \\

                     \vernacular{
                    amu[ré{\downstep}étsáángá]}  &   
                     \gloss{‘burying’}  &  \\

                     \vernacular{
                    amu[bé{\downstep}káángá]}  &   
                     \gloss{‘shaving’}  &  \\

                     \vernacular{
                    amu[lé{\downstep}éráángá]}  &   
                     \gloss{‘bringing’}  &  \\

                     \vernacular{
                    amu[khá{\downstep}lákáánga]}  &   
                     \gloss{‘cutting’}  &  \\

                     \vernacular{
                    amu[sí{\downstep}táákáánga]}  &   
                     \gloss{‘accusing’}  &  \\

                     \vernacular{
                    amu[bó{\downstep}ólítsáánga]}  &   
                     \gloss{‘seducing’}  &  \\

                     \vernacular{
                    amu[tsú{\downstep}únzúúnáánga]}  &   
                     \gloss{‘sucking’}  &  \\

                     \vernacular{
                    amu[bó{\downstep}hólóláanga]}  &   
                     \gloss{‘untying’}  &  \\

                     \vernacular{
                    amu[bó{\downstep}yóng’ánáanga]}  &   
                     \gloss{‘going
                    around’}  &  \\

                     \vernacular{
                    amu[ng’ó{\downstep}ng’óólítsáanga]}  &   
                     \gloss{‘teasing’}  &  \\

                     \vernacular{
                    amu[lí{\downstep}ngá(ka)nyínyáanga]}  &   
                     \gloss{‘bending’}  &  \\
\end{tabular}
%\caption{\nocaption}
     
\begin{tabular}{llllll}  
  \multicolumn{5}{l}{
                     \vernacular{(186) /H/
                    V-Initial + OP} \gloss{‘s/he
                    is...him/her’} } &  \\
\multicolumn{5}{l}{ } &  \\

                     \vernacular{
                    amw[ií{\downstep}ráángá]}  &   
                     \gloss{‘killing’}  &     &   
                     \vernacular{
                    amw[ií{\downstep}kóómbáánga]}  &   
                     \gloss{‘admiring’}  &  \\

                     \vernacular{
                    amw[ií{\downstep}síákáánga]}  &   
                     \gloss{‘smacking’}  &     &   
                     \vernacular{
                    amw[oó{\downstep}nónyínyáanga]}  &   
                     \gloss{‘spoiling’}  &  \\

                     \vernacular{
                    amw[aá{\downstep}búkhányínyaanga]}  &   
                     \gloss{‘separating’}  &     &     &     &  \\
\end{tabular}
%\caption{\nocaption}
     
\begin{tabular}{llllll}  
  \multicolumn{5}{l}{
                     \vernacular{(187) /Ø/
                    C-Initial + OP} \gloss{‘s/he
                    is...him/her \ob mu-\cb  / them
                    } } &  \\
\multicolumn{5}{l}{ } &  \\

                     \vernacular{
                    amu[tsiítsaanga]}  &   
                     \gloss{‘going for’}  &  \\

                     \vernacular{
                    amu[lekháanga]}  &   
                     \gloss{‘leaving’}  &  \\

                     \vernacular{
                    amu[loóndáanga]}  &   
                     \gloss{‘following’}  &  \\

                     \vernacular{
                    amu[kulíkháanga]}  &   
                     \gloss{‘naming’}  &  \\

                     \vernacular{
                    amu[lakhúulaanga]}  &   
                     \gloss{‘releasing’}  &  \\

                     \vernacular{
                    amu[seébúlaanga]}  &   
                     \gloss{‘saying bye
                    to’}  &  \\

                     \vernacular{
                    amu[hoómbélitsaanga]}  &   
                     \gloss{‘comforting’}  &  \\

                     \vernacular{
                    amu[kalúshítsaanga]}  &   
                     \gloss{‘returning’}  &  \\

                     \vernacular{
                    amu[siínjílitsaanga]}  &   
                     \gloss{
                    ‘making...stand’}  &  \\

                     \vernacular{
                    amu[reébáreebaanga]}  &   
                     \gloss{‘asking
                    (iter)’}  &  \\

                     \vernacular{
                    amu[kalúkhányinyaanga]}  &   
                     \gloss{
                    ‘turning...over’}  &  \\

                     \vernacular{
                    abi[sebúlúkhanyinyaanga]}  &   
                     \gloss{‘scattering’}  &  \\
\end{tabular}
%\caption{\nocaption}
     
\begin{tabular}{llllll}  
  \multicolumn{5}{l}{
                     \vernacular{(188) /Ø/
                    V-Initial + OP} \gloss{‘s/he
                    is...him/her \ob mw-\cb  / it
                    } } &  \\
\multicolumn{5}{l}{ } &  \\

                     \vernacular{
                    amw[eenyáanga]}  &   
                     \gloss{‘wanting’}  &     &   
                     \vernacular{
                    amw[eeyéláanga]}  &   
                     \gloss{‘lighting’}  &  \\

                     \vernacular{
                    abw[iilúulaanga]}  &   
                     \gloss{‘winnowing’}  &     &   
                     \vernacular{
                    amw[aambákhánaanga]}  &   
                     \gloss{‘refusing’}  &  \\

                     \vernacular{
                    amw[eeléelitsaanga]}  &   
                     \gloss{
                    ‘hanging...up’}  &  \\
\end{tabular}
%\caption{\nocaption}
     
\begin{tabular}{llllll}  
  \multicolumn{5}{l}{
                     \vernacular{(189) /H/
                    C-Initial + OP
                    } \gloss{‘s/he
                    is...me’} } &  \\
\multicolumn{5}{l}{ } &  \\

                     \vernacular{
                    aa[ndí{\downstep}ítsáángá]}  &   
                     \gloss{‘fearing’}  &  \\

                     \vernacular{
                    aa[mbé{\downstep}káángá]}  &   
                     \gloss{‘shaving’}  &  \\

                     \vernacular{
                    aa[ndé{\downstep}éráángá]}  &   
                     \gloss{‘bringing’}  &  \\

                     \vernacular{
                    aa[khá{\downstep}lákáánga]}  &   
                     \gloss{‘cutting’}  &  \\

                     \vernacular{
                    aa[sí{\downstep}táákáánga]}  &   
                     \gloss{‘accusing’}  &  \\

                     \vernacular{
                    aa[mbó{\downstep}ólítsáánga]}  &   
                     \gloss{‘seducing’}  &  \\

                     \vernacular{
                    aa[ndzú{\downstep}únzúúnáánga]}  &   
                     \gloss{‘sucking’}  &  \\

                     \vernacular{
                    aa[mbó{\downstep}hólóláanga]}  &   
                     \gloss{‘untying’}  &  \\

                     \vernacular{
                    aa[mbó{\downstep}yóng’ánáanga]}  &   
                     \gloss{‘going
                    around’}  &  \\

                     \vernacular{
                    aa[ng’ó{\downstep}ng’óólítsáanga]}  &   
                     \gloss{‘teasing’}  &  \\

                     \vernacular{
                    aa[ní{\downstep}ngá(ká)nyínyáanga]}  &   
                     \gloss{‘bending’}  &  \\
\end{tabular}
%\caption{\nocaption}
     
\begin{tabular}{llllll}  
  \multicolumn{5}{l}{
                     \vernacular{(190) /H/
                    V-Initial + OP
                    } \gloss{‘s/he
                    is...me’} } &  \\
\multicolumn{5}{l}{ } &  \\

                     \vernacular{
                    aa[nzí{\downstep}ráángá]}  &   
                     \gloss{‘killing’}  &     &   
                     \vernacular{
                    aa[nzí{\downstep}kóómbaanga]}  &   
                     \gloss{‘admiring’}  &  \\

                     \vernacular{
                    aa[nzí{\downstep}síákaanga]}  &   
                     \gloss{‘smacking’}  &     &   
                     \vernacular{
                    aa[nzó{\downstep}nónyínyáanga]}  &   
                     \gloss{‘spoiling’}  &  \\

                     \vernacular{
                    aa[nzá{\downstep}búkhányínyaanga]}  &   
                     \gloss{‘separating’}  &     &     &     &  \\
\end{tabular}
%\caption{\nocaption}
     
\begin{tabular}{llllll}  
  \multicolumn{5}{l}{
                     \vernacular{(191) /Ø/
                    C-Initial + OP
                    } \gloss{‘s/he
                    is...me’} } &  \\
\multicolumn{5}{l}{ } &  \\

                     \vernacular{
                    aa[ndekháanga]}  &   
                     \gloss{‘leaving’}  &  \\

                     \vernacular{
                    aa[noóndáanga]}  &   
                     \gloss{‘following’}  &  \\

                     \vernacular{
                    aa[ngulíkháanga]}  &   
                     \gloss{‘naming’}  &  \\

                     \vernacular{
                    aa[ndakhúulaanga]}  &   
                     \gloss{‘releasing’}  &  \\

                     \vernacular{
                    aa[seébúlaanga]}  &   
                     \gloss{‘saying bye
                    to’}  &  \\

                     \vernacular{
                    aa[mboómbélitsaanga]}  &   
                     \gloss{‘comforting’}  &  \\

                     \vernacular{
                    aa[siínjílitsaanga]}  &   
                     \gloss{
                    ‘making...stand’}  &  \\

                     \vernacular{
                    aa[ndeébándeebaanga]}  &   
                     \gloss{‘asking
                    (iter)’}  &  \\

                     \vernacular{
                    aa[ngalúkhányinyaanga]}  &   
                     \gloss{
                    ‘turning...over’}  &  \\
\end{tabular}
%\caption{\nocaption}
     
\begin{tabular}{llllll}  
  \multicolumn{5}{l}{
                     \vernacular{(192) /Ø/
                    V-Initial + OP
                    } \gloss{‘s/he
                    is...me’} } &  \\
\multicolumn{5}{l}{ } &  \\

                     \vernacular{
                    aa[nzenyáanga]}  &   
                     \gloss{‘wanting’}  &     &   
                     \vernacular{
                    aa[nzeyéláanga]}  &   
                     \gloss{‘wiping for’}  &  \\

                     \vernacular{
                    aa[nyambákhánaanga]}  &   
                     \gloss{‘refusing’}  &     &   
                     \vernacular{
                    aa[nzeléelitsaanga]}  &   
                     \gloss{
                    ‘carrying...hanging’}  &  \\
\end{tabular}
%\caption{\nocaption}
     
\begin{tabular}{llllll}  
  \multicolumn{5}{l}{
                     \vernacular{(193) /H/
                    C-Initial + OP
                    } \gloss{‘s/he
                    is...him/herself’} } &  \\
\multicolumn{5}{l}{ } &  \\

                     \vernacular{
                    yii[ré{\downstep}étsáángá]}  &   
                     \gloss{‘burying’}  &     &   
                     \vernacular{
                    yii[bé{\downstep}káángá]}  &   
                     \gloss{‘shaving’}  &  \\

                     \vernacular{
                    yii[sú{\downstep}úngáángá]}  &   
                     \gloss{‘hanging’}  &     &   
                     \vernacular{
                    yii[khá{\downstep}lákáánga]}  &   
                     \gloss{‘cutting’}  &  \\

                     \vernacular{
                    yii[sí{\downstep}táákáánga]}  &   
                     \gloss{‘accusing’}  &     &   
                     \vernacular{
                    yii[sá{\downstep}ándítsáánga]}  &   
                     \gloss{‘thanking’}  &  \\

                     \vernacular{
                    yii[tsú{\downstep}únzúúnáánga]}  &   
                     \gloss{‘sucking’}  &     &   
                     \vernacular{
                    yii[bó{\downstep}hólóláanga]}  &   
                     \gloss{‘untying’}  &  \\
\end{tabular}
%\caption{\nocaption}
     
\begin{tabular}{llllll}  
  \multicolumn{5}{l}{
                     \vernacular{(194) /H/
                    V-Initial + OP
                    } \gloss{‘s/he
                    is...him/herself’} } &  \\
\multicolumn{5}{l}{ } &  \\

                     \vernacular{
                    yii[yí{\downstep}ráángá]}  &   
                     \gloss{‘killing’}  &     &   
                     \vernacular{
                    yii[yí{\downstep}kóómbáánga]}  &   
                     \gloss{‘admiring’}  &  \\

                     \vernacular{
                    yii[yí{\downstep}síákáánga]}  &   
                     \gloss{‘smacking’}  &     &   
                     \vernacular{
                    yii[yó{\downstep}nónyínyáanga]}  &   
                     \gloss{‘spoiling’}  &  \\

                     \vernacular{
                    yii[yá{\downstep}búkhányínyaanga]}  &   
                     \gloss{‘separating’}  &     &     &     &  \\
\end{tabular}
%\caption{\nocaption}
     
\begin{tabular}{llllll}  
  \multicolumn{5}{l}{
                     \vernacular{(195) /Ø/
                    C-Initial + OP
                    } \gloss{‘s/he
                    is...him/herself’} } &  \\
\multicolumn{5}{l}{ } &  \\

                     \vernacular{
                    yii[lekháanga]}  &   
                     \gloss{‘leaving’}  &  \\

                     \vernacular{
                    yii[siíngáanga]}  &   
                     \gloss{‘bathing’}  &  \\

                     \vernacular{
                    yii[kulíkháanga]}  &   
                     \gloss{‘naming’}  &  \\

                     \vernacular{
                    yii[naábúlaanga]}  &   
                     \gloss{‘undressing’}  &  \\

                     \vernacular{
                    yii[lakhúulaanga]}  &   
                     \gloss{‘releasing’}  &  \\

                     \vernacular{
                    yii[hoómbélitsaanga]}  &   
                     \gloss{‘comforting’}  &  \\

                     \vernacular{
                    yii[siínjílitsaanga]}  &   
                     \gloss{
                    ‘making...stand’}  &  \\

                     \vernacular{
                    yii[reébáreebaanga]}  &   
                     \gloss{‘asking
                    (iter)’}  &  \\

                     \vernacular{
                    yii[kalúkhányinyaanga]}  &   
                     \gloss{
                    ‘turning...over’}  &  \\
\end{tabular}
%\caption{\nocaption}
     
\begin{tabular}{llllll}  
  \multicolumn{5}{l}{
                     \vernacular{(196) /Ø/
                    V-Initial + OP
                    } \gloss{‘s/he
                    is...him/herself’} } &  \\
\multicolumn{5}{l}{ } &  \\

                     \vernacular{
                    yii[yaláanga]}  &   
                     \gloss{‘exposing’}  &     &   
                     \vernacular{
                    yii[yeyéláanga]}  &   
                     \gloss{‘wiping for’}  &  \\

                     \vernacular{
                    yii[yambákhánaanga]}  &   
                     \gloss{‘despising’}  &     &   
                     \vernacular{
                    yii[yeléelitsaanga]}  &   
                     \gloss{‘hanging’}  &  \\
\end{tabular}
%\caption{\nocaption}
     
\begin{tabular}{llllll}  
  \multicolumn{5}{l}{
                     \vernacular{(197) /H/
                    C-Initial + OP + OP
                    } \gloss{‘s/he
                    is...him/her for me’} } &  \\
\multicolumn{5}{l}{ } &  \\

                     \vernacular{
                    amuú[{\downstep}ndééláángá]}  &   
                     \gloss{‘burying’}  &  \\

                     \vernacular{
                    amuú[{\downstep}mbéchéláánga]}  &   
                     \gloss{‘shaving’}  &  \\

                     \vernacular{
                    amuú[{\downstep}ndééréláánga]}  &   
                     \gloss{‘bringing’}  &  \\

                     \vernacular{
                    amuú[{\downstep}kháláchíláanga]}  &   
                     \gloss{‘cutting’}  &  \\

                     \vernacular{
                    amuú[{\downstep}sítááchíláanga]}  &   
                     \gloss{‘accusing’}  &  \\

                     \vernacular{
                    amuú[{\downstep}mbóólítsíláanga]}  &   
                     \gloss{‘seducing’}  &  \\

                     \vernacular{
                    amuú[{\downstep}mbóhólólélaanga]}  &   
                     \gloss{‘untying’}  &  \\
\end{tabular}
%\caption{\nocaption}
     
\begin{tabular}{llllll}  
  \multicolumn{5}{l}{
                     \vernacular{(198) /H/
                    V-Initial + OP + OP
                    } \gloss{‘s/he
                    is...him/her for me’} } &  \\
\multicolumn{5}{l}{ } &  \\

                     \vernacular{
                    amuú[{\downstep}nzíríláánga]}  &   
                     \gloss{‘killing’}  &  \\

                     \vernacular{
                    amuú[{\downstep}nzéchítsíláanga]}  &   
                     \gloss{‘admiring’}  &  \\

                     \vernacular{
                    amuú[{\downstep}nzísíáchíláánga]}  &   
                     \gloss{‘smacking’}  &  \\

                     \vernacular{
                    amuú[{\downstep}nzónónyínyílaanga]}  &   
                     \gloss{‘spoiling’}  &  \\

                     \vernacular{
                    amuú[{\downstep}nzábúkhányínyilaanga]}  &   
                     \gloss{‘separating’}  &  \\
\end{tabular}
%\caption{\nocaption}
     
\begin{tabular}{llllll}  
  \multicolumn{5}{l}{
                     \vernacular{(199) /Ø/
                    C-Initial + OP + OP
                    } \gloss{‘s/he
                    is...him/her for me’} } &  \\
\multicolumn{5}{l}{ } &  \\

                     \vernacular{
                    amuú[{\downstep}nzííláanga]}  &   
                     \gloss{‘going for’}  &  \\

                     \vernacular{
                    amuú[{\downstep}ndéshéláanga]}  &   
                     \gloss{‘leaving’}  &  \\

                     \vernacular{
                    amuú[{\downstep}nóóndélaanga]}  &   
                     \gloss{‘following’}  &  \\

                     \vernacular{
                    amuú[{\downstep}ngúlíshílaanga]}  &   
                     \gloss{‘naming’}  &  \\

                     \vernacular{
                    amuú[{\downstep}ndákhúulilaanga]}  &   
                     \gloss{‘releasing’}  &  \\

                     \vernacular{
                    amuú[{\downstep}séébúlilaanga]}  &   
                     \gloss{‘saying bye
                    to’}  &  \\

                     \vernacular{
                    amuú[{\downstep}mbóómbélitsilaanga]}  &   
                     \gloss{‘comforting’}  &  \\

                     \vernacular{
                    amuú[{\downstep}síínjílitsilaanga]}  &   
                     \gloss{
                    ‘making...stand’}  &  \\
\end{tabular}
%\caption{\nocaption}
     
\begin{tabular}{llllll}  
  \multicolumn{5}{l}{
                     \vernacular{(200) /Ø/
                    V-Initial + OP + OP
                    } \gloss{‘s/he
                    is...him/her \ob mu-\cb  / it
                    } } &  \\
\multicolumn{5}{l}{ } &  \\

                     \vernacular{
                    amuú[{\downstep}nzéyéláanga]}  &   
                     \gloss{‘wiping for’}  &     &   
                     \vernacular{
                    akuú[nzáshítsílaanga]}  &   
                     \gloss{‘lighting’}  &  \\

                     \vernacular{
                    abuú[{\downstep}nzílúulilaanga]}  &   
                     \gloss{‘winnowing’}  &     &   
                     \vernacular{
                    aluú[nzítsúlítsilaanga]}  &   
                     \gloss{‘filling’}  &  \\

                     \vernacular{
                    akuú[nzéléelitsilaanga]}  &   
                     \gloss{‘hanging’}  &     &     &     &  \\
\end{tabular}
%\caption{\nocaption}
     
\begin{tabular}{lll}  
  \multicolumn{2}{l}{
                     \vernacular{(201) /H/
                    C-Initial Phrase-Medial} \gloss{‘s/he is...the
                    man \ob musáatsa\cb  /} } &  \\
\multicolumn{2}{l}{
                     \gloss{the boy
                    \ob mú{\downstep}yáyi\cb  / someone \ob muundu\cb ’} } &  \\

                     \vernacular{a[reetsaanga]
                    musáatsa/mú{\downstep}yáyi/muundu}  &   
                     \gloss{‘burying’}  &  \\

                     \vernacular{a[bekaanga]
                    musáatsa/mú{\downstep}yáyi/muundu}  &   
                     \gloss{‘shaving’}  &  \\

                     \vernacular{a[leeraanga]
                    musáatsa/mú{\downstep}yáyi/muundu}  &   
                     \gloss{‘bringing’}  &  \\

                     \vernacular{a[khalakaanga]
                    musáatsa/mú{\downstep}yáyi/muundu}  &   
                     \gloss{‘cutting’}  &  \\

                     \vernacular{a[sitaakaanga]
                    musáatsa/mú{\downstep}yáyi/muundu}  &   
                     \gloss{‘accusing’}  &  \\

                     \vernacular{a[boolitsaanga]
                    musáatsa/mú{\downstep}yáyi/muundu}  &   
                     \gloss{‘seducing’}  &  \\

                     \vernacular{a[tsuunzuunaanga]
                    musáatsa/mú{\downstep}yáyi/muundu}  &   
                     \gloss{‘sucking’}  &  \\

                     \vernacular{a[bohololaanga]
                    musáatsa/mú{\downstep}yáyi/muundu}  &   
                     \gloss{‘untying’}  &  \\

                     \vernacular{a[boyong’anaanga]
                    musáatsa/mú{\downstep}yáyi/muundu}  &   
                     \gloss{‘going
                    around’}  &  \\
\end{tabular}
%\caption{\nocaption}
     
\begin{tabular}{lll}  
  \multicolumn{2}{l}{
                     \vernacular{(202) /Ø/
                    C-Initial Phrase-Medial} \gloss{‘s/he is...the
                    man \ob musáatsa\cb  /} } &  \\
\multicolumn{2}{l}{
                     \gloss{the boy
                    \ob mú{\downstep}yáyi\cb  / someone \ob muundu\cb ’} } &  \\

                     \vernacular{a[tsiitsaanga]
                    musáatsa/mú{\downstep}yáyi/muundu}  &   
                     \gloss{‘going for’}  &  \\

                     \vernacular{a[lekhaanga]
                    musáatsa/mú{\downstep}yáyi/muundu}  &   
                     \gloss{‘leaving’}  &  \\

                     \vernacular{a[loondaanga]
                    musáatsa/mú{\downstep}yáyi/muundu}  &   
                     \gloss{‘following’}  &  \\

                     \vernacular{a[kulikhaanga]
                    musáatsa/mú{\downstep}yáyi/muundu}  &   
                     \gloss{‘naming’}  &  \\

                     \vernacular{a[lakhuulaanga]
                    musáatsa/mú{\downstep}yáyi/muundu}  &   
                     \gloss{‘releasing’}  &  \\

                     \vernacular{a[seebulaanga]
                    musáatsa/mú{\downstep}yáyi/muundu}  &   
                     \gloss{‘saying bye
                    to’}  &  \\

                     \vernacular{a[kalushitsaanga]
                    musáatsa/mú{\downstep}yáyi/muundu}  &   
                     \gloss{‘returning’}  &  \\

                     \vernacular{a[reebareebaanga]
                    musáatsa/mú{\downstep}yáyi/muundu}  &   
                     \gloss{‘asking
                    (iter)’}  &  \\
\end{tabular}
%\caption{\nocaption}
     
\begin{tabular}{lll}  
  \multicolumn{2}{l}{
                     \vernacular{(203) /H/
                    C-Initial +OP Phrase-Medial} \gloss{‘s/he is...the
                    man \ob musáatsa\cb  /} } &  \\
\multicolumn{2}{l}{
                     \gloss{the boy
                    \ob mú{\downstep}yáyi\cb  / someone \ob muundu\cb  for
                    him/her’} } &  \\

                     \vernacular{amu[réelaanga]
                    musáatsa/mú{\downstep}yáyi/muundu}  &   
                     \gloss{‘burying’}  &  \\

                     \vernacular{amu[béchelaanga]
                    musáatsa/mú{\downstep}yáyi/muundu}  &   
                     \gloss{‘shaving’}  &  \\

                     \vernacular{amu[léerelaanga]
                    musáatsa/mú{\downstep}yáyi/muundu}  &   
                     \gloss{‘bringing’}  &  \\

                     \vernacular{
                    amu[khálachilaanga]
                    musáatsa/mú{\downstep}yáyi/muundu}  &   
                     \gloss{‘cutting’}  &  \\

                     \vernacular{
                    amu[sítaachilaanga]
                    musáatsa/mú{\downstep}yáyi/muundu}  &   
                     \gloss{‘accusing’}  &  \\

                     \vernacular{
                    amu[bóolitsilaanga]
                    musáatsa/mú{\downstep}yáyi/muundu}  &   
                     \gloss{‘seducing’}  &  \\

                     \vernacular{
                    amu[tsúunzuunilaanga]
                    musáatsa/mú{\downstep}yáyi/muundu}  &   
                     \gloss{‘sucking’}  &  \\

                     \vernacular{
                    amu[bóhololelaanga]
                    musáatsa/mú{\downstep}yáyi/muundu}  &   
                     \gloss{‘untying’}  &  \\

                     \vernacular{
                    amu[bóyong’anilaanga]
                    musáatsa/mú{\downstep}yáyi/muundu}  &   
                     \gloss{‘going
                    around’}  &  \\
\end{tabular}
%\caption{\nocaption}
     
\begin{tabular}{lll}  
  \multicolumn{2}{l}{
                     \vernacular{(204) /Ø/
                    C-Initial +OP Phrase-Medial} \gloss{‘s/he is...the
                    man \ob musáatsa\cb  /} } &  \\
\multicolumn{2}{l}{
                     \gloss{the boy
                    \ob mú{\downstep}yáyi\cb  / someone \ob muundu\cb  for
                    him/her’} } &  \\

                     \vernacular{amu[tsiilaanga]
                    musáatsa/mú{\downstep}yáyi/muundu}  &   
                     \gloss{‘going for’}  &  \\

                     \vernacular{amu[leshelaanga]
                    musáatsa/mú{\downstep}yáyi/muundu}  &   
                     \gloss{‘leaving’}  &  \\

                     \vernacular{amu[loondelaanga]
                    musáatsa/mú{\downstep}yáyi/muundu}  &   
                     \gloss{‘following’}  &  \\

                     \vernacular{amu[kulishilaanga]
                    musáatsa/mú{\downstep}yáyi/muundu}  &   
                     \gloss{‘naming’}  &  \\

                     \vernacular{
                    amu[lakhuulilaanga]
                    musáatsa/mú{\downstep}yáyi/muundu}  &   
                     \gloss{‘releasing’}  &  \\

                     \vernacular{amu[seebulilaanga]
                    musáatsa/mú{\downstep}yáyi/muundu}  &   
                     \gloss{‘saying bye
                    to’}  &  \\

                     \vernacular{
                    amu[kalushitsilaanga]
                    musáatsa/mú{\downstep}yáyi/muundu}  &   
                     \gloss{‘returning’}  &  \\

                     \vernacular{
                    amu[reebareebelaanga]
                    musáatsa/mú{\downstep}yáyi/muundu}  &   
                     \gloss{‘asking
                    (iter)’}  &  \\
\end{tabular}
%\caption{\nocaption}
     
\begin{tabular}{lll}  
  \multicolumn{2}{l}{
                     \vernacular{(205) /H/
                    C-Initial +OP + OP
                    } \gloss{‘s/he is...the
                    man \ob musáatsa\cb  /} } &  \\
\multicolumn{2}{l}{
                     \gloss{the boy
                    \ob mú{\downstep}yáyi\cb  / someone \ob muundu\cb  for him/her for
                    me’} } &  \\

                     \vernacular{amuú[ndeelaanga]
                    musáatsa/mú{\downstep}yáyi/muundu}  &   
                     \gloss{‘buring’}  &  \\

                     \vernacular{
                    amuú[mbechelaanga]
                    musáatsa/mú{\downstep}yáyi/muundu}  &   
                     \gloss{‘shaving’}  &  \\

                     \vernacular{
                    amuú[ndeerelaanga]
                    musáatsa/mú{\downstep}yáyi/muundu}  &   
                     \gloss{‘bringing’}  &  \\

                     \vernacular{
                    amuú[khalachilaanga]
                    musáatsa/mú{\downstep}yáyi/muundu}  &   
                     \gloss{‘cutting’}  &  \\

                     \vernacular{
                    amuú[sitaachilaanga]
                    musáatsa/mú{\downstep}yáyi/muundu}  &   
                     \gloss{‘accusing’}  &  \\

                     \vernacular{
                    amuú[mboolitsilaanga]
                    musáatsa/mú{\downstep}yáyi/muundu}  &   
                     \gloss{‘seducing’}  &  \\

                     \vernacular{
                    amuú[mbohololelaanga]
                    musáatsa/mú{\downstep}yáyi/muundu}  &   
                     \gloss{‘untying’}  &  \\
\end{tabular}
%\caption{\nocaption}
     
\begin{tabular}{lll}  
  \multicolumn{2}{l}{
                     \vernacular{(206) /Ø/
                    C-Initial +OP + OP
                    } \gloss{‘s/he is...the
                    man \ob musáatsa\cb  /} } &  \\
\multicolumn{2}{l}{
                     \gloss{the boy
                    \ob mú{\downstep}yáyi\cb  / someone \ob muundu\cb  for him/her for
                    me’} } &  \\

                     \vernacular{amuú[nziilaanga]
                    musáatsa/mú{\downstep}yáyi/muundu}  &   
                     \gloss{‘going for’}  &  \\

                     \vernacular{
                    amuú[ndeshelaanga]
                    musáatsa/mú{\downstep}yáyi/muundu}  &   
                     \gloss{‘leaving’}  &  \\

                     \vernacular{
                    amuú[noondelaanga]
                    musáatsa/mú{\downstep}yáyi/muundu}  &   
                     \gloss{‘following’}  &  \\

                     \vernacular{
                    amuú[ngulishilaanga]
                    musáatsa/mú{\downstep}yáyi/muundu}  &   
                     \gloss{‘naming’}  &  \\

                     \vernacular{
                    amuú[ndakhuulilaanga]
                    musáatsa/mú{\downstep}yáyi/muundu}  &   
                     \gloss{‘releasing’}  &  \\

                     \vernacular{
                    amuú[seebulilaanga]
                    musáatsa/mú{\downstep}yáyi/muundu}  &   
                     \gloss{‘saying bye
                    to’}  &  \\
\end{tabular}
%\caption{\nocaption}
    

\subsection{Present Negative: Pattern 5a
              }\label{sec:sPresNeg}


\begin{tabular}{lll}  
  \multicolumn{2}{l}{
                     \vernacular{(207) /H/
                    C-Initial} \gloss{‘s/he is
                    not...’} } &  \\
\multicolumn{2}{l}{ } &  \\

                     \vernacular{a[reetsáángá]
                    {\downstep}tá}  &   
                     \gloss{‘burying’}  &  \\

                     \vernacular{
                    a[ng’weetsáángá] {\downstep}tá}  &   
                     \gloss{‘drinking’}  &  \\

                     \vernacular{a[khweetsáángá]
                    {\downstep}tá}  &   
                     \gloss{‘paying
                    dowry’}  &  \\

                     \vernacular{a[liitsáángá]
                    {\downstep}tá}  &   
                     \gloss{‘eating’}  &  \\

                     \vernacular{a[lumaángá]
                    {\downstep}tá}  &   
                     \gloss{‘biting’}  &  \\

                     \vernacular{a[bekaángá]
                    {\downstep}tá}  &   
                     \gloss{‘shaving’}  &  \\

                     \vernacular{a[teekháángá]
                    {\downstep}tá}  &   
                     \gloss{‘cooking’}  &  \\

                     \vernacular{a[leeráángá]
                    {\downstep}tá}  &   
                     \gloss{‘bringing’}  &  \\

                     \vernacular{a[khalakáá{\downstep}ngá]
                    tá}  &   
                     \gloss{‘cutting’}  &  \\

                     \vernacular{
                    a[kalaangáá{\downstep}ngá] tá}  &   
                     \gloss{‘frying’}  &  \\

                     \vernacular{a[sitaakáá{\downstep}ngá]
                    tá}  &   
                     \gloss{‘accusing’}  &  \\

                     \vernacular{
                    a[boolitsáá{\downstep}ngá] tá}  &   
                     \gloss{‘seducing’}  &  \\

                     \vernacular{
                    a[saanditsáá{\downstep}ngá] tá}  &   
                     \gloss{‘thanking’}  &  \\

                     \vernacular{
                    a[tsuunzuunáá{\downstep}ngá] tá}  &   
                     \gloss{‘sucking’}  &  \\

                     \vernacular{
                    a[boholólá{\downstep}ángá] tá}  &   
                     \gloss{‘untying’}  &  \\

                     \vernacular{
                    a[boyong’áná{\downstep}ángá] tá}  &   
                     \gloss{‘going
                    around’}  &  \\

                     \vernacular{
                    a[ng’ong’oolítsá{\downstep}ángá] tá}  &   
                     \gloss{‘teasing’}  &  \\

                     \vernacular{
                    a[linga(ka)nyínyá{\downstep}ángá] tá}  &   
                     \gloss{‘crumpling’}  &  \\
\end{tabular}
%\caption{\nocaption}
     
\begin{tabular}{llllll}  
  \multicolumn{5}{l}{
                     \vernacular{(208) /H/
                    V-Initial} \gloss{‘s/he is
                    not...’} } &  \\
\multicolumn{5}{l}{ } &  \\

                     \vernacular{y[iraángá]
                    {\downstep}tá}  &   
                     \gloss{‘killing’}  &     &   
                     \vernacular{
                    y[ikoómbá{\downstep}ángá] tá}  &   
                     \gloss{‘admiring’}  &  \\

                     \vernacular{
                    y[isíáká{\downstep}ángá] tá}  &   
                     \gloss{‘smacking’}  &     &   
                     \vernacular{
                    y[ikobólá{\downstep}ángá] tá}  &   
                     \gloss{‘belching’}  &  \\

                     \vernacular{
                    y[ononyínyá{\downstep}ángá] tá}  &   
                     \gloss{‘spoiling’}  &     &   
                     \vernacular{
                    y[abukhányí{\downstep}nyáángá] tá}  &   
                     \gloss{‘separating’}  &  \\
\end{tabular}
%\caption{\nocaption}
     
\begin{tabular}{llllll}  
  \multicolumn{5}{l}{
                     \vernacular{(209) /Ø/
                    C-Initial} \gloss{‘s/he is
                    not...’} } &  \\
\multicolumn{5}{l}{ } &  \\

                     \vernacular{
                    a[tsií{\downstep}tsáángá] tá}  &   
                     \gloss{‘going’}  &  \\

                     \vernacular{
                    a[kwií{\downstep}tsáángá] tá}  &   
                     \gloss{‘falling’}  &  \\

                     \vernacular{a[lekhá{\downstep}ángá]
                    tá}  &   
                     \gloss{‘leaving’}  &  \\

                     \vernacular{a[reébá{\downstep}ángá]
                    tá}  &   
                     \gloss{‘asking’}  &  \\

                     \vernacular{a[loóndá{\downstep}ángá]
                    tá}  &   
                     \gloss{‘following’}  &  \\

                     \vernacular{
                    a[kulíkhá{\downstep}ángá] tá}  &   
                     \gloss{‘naming’}  &  \\

                     \vernacular{
                    a[homó{\downstep}óláángá] tá}  &   
                     \gloss{‘massaging’}  &  \\

                     \vernacular{
                    a[lakhú{\downstep}úláángá] tá}  &   
                     \gloss{‘releasing’}  &  \\

                     \vernacular{
                    a[seébú{\downstep}láángá] tá}  &   
                     \gloss{‘saying bye’}  &  \\

                     \vernacular{
                    a[hoómbé{\downstep}lítsáángá] tá}  &   
                     \gloss{‘comforting’}  &  \\

                     \vernacular{
                    a[kalúshí{\downstep}tsáángá] tá}  &   
                     \gloss{‘returning’}  &  \\

                     \vernacular{
                    a[siínjí{\downstep}lítsáángá] tá}  &   
                     \gloss{‘making
                    stand’}  &  \\

                     \vernacular{
                    a[reébá{\downstep}réébáángá] tá}  &   
                     \gloss{‘asking
                    (iter)’}  &  \\

                     \vernacular{
                    a[kalúkhá{\downstep}nyínyáángá] tá}  &   
                     \gloss{‘turning
                    over’}  &  \\

                     \vernacular{
                    a[sebúlú{\downstep}khányínyáángá] tá}  &   
                     \gloss{‘scattering’}  &  \\
\end{tabular}
%\caption{\nocaption}
     
\begin{tabular}{llllll}  
  \multicolumn{5}{l}{
                     \vernacular{(210) /Ø/
                    V-Initial} \gloss{‘s/he is
                    not...’} } &  \\
\multicolumn{5}{l}{ } &  \\

                     \vernacular{y[enyá{\downstep}ángá]
                    tá}  &   
                     \gloss{‘wanting’}  &     &   
                     \vernacular{y[eyélá{\downstep}ángá]
                    tá}  &   
                     \gloss{‘wiping for’}  &  \\

                     \vernacular{
                    y[ilú{\downstep}úláángá] tá}  &   
                     \gloss{‘winnowing’}  &     &   
                     \vernacular{
                    y[ambákhá{\downstep}náángá] tá}  &   
                     \gloss{‘refusing’}  &  \\

                     \vernacular{
                    y[elé{\downstep}élítsáángá] tá}  &   
                     \gloss{‘hanging up’}  &     &     &     &  \\
\end{tabular}
%\caption{\nocaption}
     
\begin{tabular}{llllll}  
  \multicolumn{5}{l}{
                     \vernacular{(211) /H/
                    C-Initial + OP} \gloss{‘s/he is
                    not...him/her’} } &  \\
\multicolumn{5}{l}{ } &  \\

                     \vernacular{
                    amu[ré{\downstep}étsáángá] {\downstep}tá}  &   
                     \gloss{‘burying’}  &  \\

                     \vernacular{amu[bé{\downstep}káángá]
                    {\downstep}tá}  &   
                     \gloss{‘shaving’}  &  \\

                     \vernacular{
                    amu[lé{\downstep}éráángá] {\downstep}tá}  &   
                     \gloss{‘bringing’}  &  \\

                     \vernacular{
                    amu[khá{\downstep}lákáá{\downstep}ngá] tá}  &   
                     \gloss{‘cutting’}  &  \\

                     \vernacular{
                    amu[sí{\downstep}táákáá{\downstep}ngá] tá}  &   
                     \gloss{‘accusing’}  &  \\

                     \vernacular{
                    amu[bó{\downstep}ólítsáá{\downstep}ngá] tá}  &   
                     \gloss{‘seducing’}  &  \\

                     \vernacular{
                    amu[tsú{\downstep}únzúúnáá{\downstep}ngá] tá}  &   
                     \gloss{‘sucking’}  &  \\

                     \vernacular{
                    amu[bó{\downstep}hólólá{\downstep}ángá] tá}  &   
                     \gloss{‘untying’}  &  \\

                     \vernacular{
                    amu[bó{\downstep}yóng’áná{\downstep}ángá] tá}  &   
                     \gloss{‘going
                    around’}  &  \\

                     \vernacular{
                    amu[ng’ó{\downstep}ng’óólítsá{\downstep}ángá] tá}  &   
                     \gloss{‘teasing’}  &  \\

                     \vernacular{
                    amu[lí{\downstep}ngá(ka)nyínyá{\downstep}ángá] tá}  &   
                     \gloss{‘bending’}  &  \\
\end{tabular}
%\caption{\nocaption}
     
\begin{tabular}{llllll}  
  \multicolumn{5}{l}{
                     \vernacular{(212) /H/
                    V-Initial + OP} \gloss{‘s/he is
                    not...him/her’} } &  \\
\multicolumn{5}{l}{ } &  \\

                     \vernacular{amw[ií{\downstep}ráángá]
                    {\downstep}tá}  &   
                     \gloss{‘killing’}  &  \\

                     \vernacular{
                    amw[ií{\downstep}kóómbáá{\downstep}ngá] tá}  &   
                     \gloss{‘admiring’}  &  \\

                     \vernacular{
                    amw[ií{\downstep}síákáá{\downstep}ngá] tá}  &   
                     \gloss{‘smacking’}  &  \\

                     \vernacular{
                    amw[oó{\downstep}nónyínyá{\downstep}ángá] tá}  &   
                     \gloss{‘spoiling’}  &  \\

                     \vernacular{
                    amw[aá{\downstep}búkhányí{\downstep}nyáángá] tá}  &   
                     \gloss{‘separating’}  &  \\
\end{tabular}
%\caption{\nocaption}
     
\begin{tabular}{llllll}  
  \multicolumn{5}{l}{
                     \vernacular{(213) /Ø/
                    C-Initial + OP} \gloss{‘s/he is
                    not...him/her \ob mu-\cb  / them
                    } } &  \\
\multicolumn{5}{l}{ } &  \\

                     \vernacular{
                    amu[tsií{\downstep}tsáángá] tá}  &   
                     \gloss{‘going for’}  &  \\

                     \vernacular{amu[lekhá{\downstep}ángá]
                    tá}  &   
                     \gloss{‘leaving’}  &  \\

                     \vernacular{
                    amu[loóndá{\downstep}ángá] tá}  &   
                     \gloss{‘following’}  &  \\

                     \vernacular{
                    amu[kulíkhá{\downstep}ángá] tá}  &   
                     \gloss{‘naming’}  &  \\

                     \vernacular{
                    amu[lakhú{\downstep}úláángá] tá}  &   
                     \gloss{‘releasing’}  &  \\

                     \vernacular{
                    amu[seébú{\downstep}láángá] tá}  &   
                     \gloss{‘saying bye
                    to’}  &  \\

                     \vernacular{
                    amu[hoómbé{\downstep}lítsáángá] tá}  &   
                     \gloss{‘comforting’}  &  \\

                     \vernacular{
                    amu[kalúshí{\downstep}tsáángá] tá}  &   
                     \gloss{‘returning’}  &  \\

                     \vernacular{
                    amu[siínjí{\downstep}lítsáángá] tá}  &   
                     \gloss{
                    ‘making...stand’}  &  \\

                     \vernacular{
                    amu[reébá{\downstep}réébáángá] tá}  &   
                     \gloss{‘asking
                    (iter)’}  &  \\

                     \vernacular{
                    amu[kalúkhá{\downstep}nyínyáángá] tá}  &   
                     \gloss{
                    ‘turning...over’}  &  \\

                     \vernacular{
                    abi[sebúlú{\downstep}khányínyáángá] tá}  &   
                     \gloss{‘scattering’}  &  \\
\end{tabular}
%\caption{\nocaption}
     
\begin{tabular}{llllll}  
  \multicolumn{5}{l}{
                     \vernacular{(214) /Ø/
                    V-Initial + OP} \gloss{‘s/he is
                    not...him/her \ob mw-\cb  / it
                    } } &  \\
\multicolumn{5}{l}{ } &  \\

                     \vernacular{amw[eenyá{\downstep}ángá]
                    tá}  &   
                     \gloss{‘wanting’}  &     &   
                     \vernacular{
                    amw[eeyélá{\downstep}ángá] tá}  &   
                     \gloss{‘lighting’}  &  \\

                     \vernacular{
                    abw[iilú{\downstep}úláángá] tá}  &   
                     \gloss{‘winnowing’}  &     &   
                     \vernacular{
                    amw[aambákhá{\downstep}náángá] tá}  &   
                     \gloss{‘refusing’}  &  \\

                     \vernacular{
                    amw[eelé{\downstep}élítsáángá] tá}  &   
                     \gloss{
                    ‘hanging...up’}  &  \\
\end{tabular}
%\caption{\nocaption}
     
\begin{tabular}{llllll}  
  \multicolumn{5}{l}{
                     \vernacular{(215) /H/
                    C-Initial + OP
                    } \gloss{‘s/he is
                    not...me’} } &  \\
\multicolumn{5}{l}{ } &  \\

                     \vernacular{
                    aa[ndí{\downstep}ítsáángá] {\downstep}tá}  &   
                     \gloss{‘fearing’}  &  \\

                     \vernacular{aa[mbé{\downstep}káángá]
                    {\downstep}tá}  &   
                     \gloss{‘shaving’}  &  \\

                     \vernacular{
                    aa[ndé{\downstep}éráángá] {\downstep}tá}  &   
                     \gloss{‘bringing’}  &  \\

                     \vernacular{
                    aa[khá{\downstep}lákáá{\downstep}ngá] tá}  &   
                     \gloss{‘cutting’}  &  \\

                     \vernacular{
                    aa[sí{\downstep}táákáá{\downstep}ngá] tá}  &   
                     \gloss{‘accusing’}  &  \\

                     \vernacular{
                    aa[mbó{\downstep}ólítsáá{\downstep}ngá] tá}  &   
                     \gloss{‘seducing’}  &  \\

                     \vernacular{
                    aa[ndzú{\downstep}únzúúnáá{\downstep}ngá] tá}  &   
                     \gloss{‘sucking’}  &  \\

                     \vernacular{
                    aa[mbó{\downstep}hólólá{\downstep}ángá] tá}  &   
                     \gloss{‘untying’}  &  \\

                     \vernacular{
                    aa[mbó{\downstep}yóng’áná{\downstep}ángá] tá}  &   
                     \gloss{‘going
                    around’}  &  \\

                     \vernacular{
                    aa[ng’ó{\downstep}ng’óólítsá{\downstep}ángá] tá}  &   
                     \gloss{‘teasing’}  &  \\

                     \vernacular{
                    aa[ní{\downstep}ngá(ká)nyínyá{\downstep}ángá] tá}  &   
                     \gloss{‘bending’}  &  \\
\end{tabular}
%\caption{\nocaption}
     
\begin{tabular}{llllll}  
  \multicolumn{5}{l}{
                     \vernacular{(216) /H/
                    V-Initial + OP
                    } \gloss{‘s/he is
                    not...me’} } &  \\
\multicolumn{5}{l}{ } &  \\

                     \vernacular{aa[nzí{\downstep}ráángá]
                    {\downstep}tá}  &   
                     \gloss{‘killing’}  &  \\

                     \vernacular{
                    aa[nzí{\downstep}kóómbáá{\downstep}ngá] tá}  &   
                     \gloss{‘admiring’}  &  \\

                     \vernacular{
                    aa[nzí{\downstep}síákáá{\downstep}ngá] tá}  &   
                     \gloss{‘smacking’}  &  \\

                     \vernacular{
                    aa[nzó{\downstep}nónyínyá{\downstep}ángá] tá}  &   
                     \gloss{‘spoiling’}  &  \\

                     \vernacular{
                    aa[nzá{\downstep}búkhányí{\downstep}nyáángá] tá}  &   
                     \gloss{‘separating’}  &  \\
\end{tabular}
%\caption{\nocaption}
     
\begin{tabular}{llllll}  
  \multicolumn{5}{l}{
                     \vernacular{(217) /Ø/
                    C-Initial + OP
                    } \gloss{‘s/he is
                    not...me’} } &  \\
\multicolumn{5}{l}{ } &  \\

                     \vernacular{aa[ndekhá{\downstep}ángá]
                    tá}  &   
                     \gloss{‘leaving’}  &  \\

                     \vernacular{
                    aa[noóndá{\downstep}ángá] tá}  &   
                     \gloss{‘following’}  &  \\

                     \vernacular{
                    aa[ngulíkhá{\downstep}ángá] tá}  &   
                     \gloss{‘naming’}  &  \\

                     \vernacular{
                    aa[ndakhú{\downstep}úláángá] tá}  &   
                     \gloss{‘releasing’}  &  \\

                     \vernacular{
                    aa[seébúl{\downstep}áángá] tá}  &   
                     \gloss{‘saying bye
                    to’}  &  \\

                     \vernacular{
                    aa[mboómbé{\downstep}lítsáángá] tá}  &   
                     \gloss{‘comforting’}  &  \\

                     \vernacular{
                    aa[siínjí{\downstep}lítsáángá] tá}  &   
                     \gloss{
                    ‘making...stand’}  &  \\

                     \vernacular{
                    aa[ndeébá{\downstep}ndéébáángá] tá}  &   
                     \gloss{‘asking
                    (iter)’}  &  \\

                     \vernacular{
                    aa[ngalúkhá{\downstep}nyínyáángá] tá}  &   
                     \gloss{
                    ‘turning...over’}  &  \\
\end{tabular}
%\caption{\nocaption}
     
\begin{tabular}{llllll}  
  \multicolumn{5}{l}{
                     \vernacular{(218) /Ø/
                    V-Initial + OP
                    } \gloss{‘s/he is
                    not...me’} } &  \\
\multicolumn{5}{l}{ } &  \\

                     \vernacular{aa[nzenyá{\downstep}ángá]
                    tá}  &   
                     \gloss{‘wanting’}  &  \\

                     \vernacular{
                    aa[nzeyélá{\downstep}ángá] tá}  &   
                     \gloss{‘wiping for’}  &  \\

                     \vernacular{
                    aa[nyambákhá{\downstep}náángá] tá}  &   
                     \gloss{‘refusing’}  &  \\

                     \vernacular{
                    aa[nzelé{\downstep}élítsáángá] tá}  &   
                     \gloss{
                    ‘carrying...hanging’}  &  \\
\end{tabular}
%\caption{\nocaption}
     
\begin{tabular}{llllll}  
  \multicolumn{5}{l}{
                     \vernacular{(219) /H/
                    C-Initial + OP
                    } \gloss{‘s/he is
                    not...him/herself’} } &  \\
\multicolumn{5}{l}{ } &  \\

                     \vernacular{
                    yii[ré{\downstep}étsáángá] {\downstep}tá}  &   
                     \gloss{‘burying’}  &     &   
                     \vernacular{yii[bé{\downstep}káángá]
                    {\downstep}tá}  &   
                     \gloss{‘shaving’}  &  \\

                     \vernacular{
                    yii[sú{\downstep}úngáángá] {\downstep}tá}  &   
                     \gloss{‘hanging’}  &     &   
                     \vernacular{
                    yii[khá{\downstep}lákáá{\downstep}ngá] tá}  &   
                     \gloss{‘cutting’}  &  \\

                     \vernacular{
                    yii[sí{\downstep}táákáá{\downstep}ngá] tá}  &   
                     \gloss{‘accusing’}  &     &   
                     \vernacular{
                    yii[sá{\downstep}ándítsáá{\downstep}ngá] tá}  &   
                     \gloss{‘thanking’}  &  \\

                     \vernacular{
                    yii[tsú{\downstep}únzúúnáá{\downstep}ngá] tá}  &   
                     \gloss{‘sucking’}  &     &   
                     \vernacular{
                    yii[bó{\downstep}hólólá{\downstep}ángá] tá}  &   
                     \gloss{‘untying’}  &  \\
\end{tabular}
%\caption{\nocaption}
     
\begin{tabular}{llllll}  
  \multicolumn{5}{l}{
                     \vernacular{(220) /H/
                    V-Initial + OP
                    } \gloss{‘s/he is
                    not...him/herself’} } &  \\
\multicolumn{5}{l}{ } &  \\

                     \vernacular{yii[yí{\downstep}ráángá]
                    {\downstep}tá}  &   
                     \gloss{‘killing’}  &  \\

                     \vernacular{
                    yii[yí{\downstep}kóómbáá{\downstep}ngá] tá}  &   
                     \gloss{‘admiring’}  &  \\

                     \vernacular{
                    yii[yí{\downstep}síákáá{\downstep}ngá] tá}  &   
                     \gloss{‘smacking’}  &  \\

                     \vernacular{
                    yii[yó{\downstep}nónyínyá{\downstep}ángá] tá}  &   
                     \gloss{‘spoiling’}  &  \\

                     \vernacular{
                    yii[yá{\downstep}búkhányí{\downstep}nyáángá] tá}  &   
                     \gloss{‘separating’}  &  \\
\end{tabular}
%\caption{\nocaption}
     
\begin{tabular}{llllll}  
  \multicolumn{5}{l}{
                     \vernacular{(221) /Ø/
                    C-Initial + OP
                    } \gloss{‘s/he is
                    not...him/herself’} } &  \\
\multicolumn{5}{l}{ } &  \\

                     \vernacular{yii[lekhá{\downstep}ángá]
                    tá}  &   
                     \gloss{‘leaving’}  &  \\

                     \vernacular{
                    yii[siíngá{\downstep}ángá] tá}  &   
                     \gloss{‘bathing’}  &  \\

                     \vernacular{
                    yii[kulíkhá{\downstep}ángá] tá}  &   
                     \gloss{‘naming’}  &  \\

                     \vernacular{
                    yii[naábú{\downstep}láángá] tá}  &   
                     \gloss{‘undressing’}  &  \\

                     \vernacular{
                    yii[lakhú{\downstep}úláángá] tá}  &   
                     \gloss{‘releasing’}  &  \\

                     \vernacular{
                    yii[hoómbé{\downstep}lítsáángá] tá}  &   
                     \gloss{‘comforting’}  &  \\

                     \vernacular{
                    yii[siínjí{\downstep}lítsáángá] tá}  &   
                     \gloss{
                    ‘making...stand’}  &  \\

                     \vernacular{
                    yii[reébá{\downstep}réébáángá] tá}  &   
                     \gloss{‘asking
                    (iter)’}  &  \\

                     \vernacular{
                    yii[kalúkhá{\downstep}nyínyáángá] tá}  &   
                     \gloss{
                    ‘turning...over’}  &  \\
\end{tabular}
%\caption{\nocaption}
     
\begin{tabular}{llllll}  
  \multicolumn{5}{l}{
                     \vernacular{(222) /Ø/
                    V-Initial + OP
                    } \gloss{‘s/he is
                    not...him/herself’} } &  \\
\multicolumn{5}{l}{ } &  \\

                     \vernacular{yii[yalá{\downstep}ángá]
                    tá}  &   
                     \gloss{‘exposing’}  &     &   
                     \vernacular{
                    yii[yeyélá{\downstep}ángá] tá}  &   
                     \gloss{‘wiping for’}  &  \\

                     \vernacular{
                    yii[yambákhá{\downstep}náángá] tá}  &   
                     \gloss{‘despising’}  &     &   
                     \vernacular{
                    yii[yelé{\downstep}élítsáángá] tá}  &   
                     \gloss{‘hanging’}  &  \\
\end{tabular}
%\caption{\nocaption}
     
\begin{tabular}{llllll}  
  \multicolumn{5}{l}{
                     \vernacular{(223) /H/
                    C-Initial + OP + OP
                    } \gloss{‘s/he is
                    not...him/her for me’} } &  \\
\multicolumn{5}{l}{ } &  \\

                     \vernacular{
                    amuú[{\downstep}ndééláángá] {\downstep}tá}  &   
                     \gloss{‘burying’}  &  \\

                     \vernacular{
                    amuú[{\downstep}mbéchéláá{\downstep}ngá] tá}  &   
                     \gloss{‘shaving’}  &  \\

                     \vernacular{
                    amuú[{\downstep}ndééréláá{\downstep}ngá] tá}  &   
                     \gloss{‘bringing’}  &  \\

                     \vernacular{
                    amuú[{\downstep}kháláchílá{\downstep}ángá] tá}  &   
                     \gloss{‘cutting’}  &  \\

                     \vernacular{
                    amuú[{\downstep}sítááchílá{\downstep}ángá] tá}  &   
                     \gloss{‘accusing’}  &  \\

                     \vernacular{
                    amuú[{\downstep}mbóólítsílá{\downstep}ángá] tá}  &   
                     \gloss{‘seducing’}  &  \\

                     \vernacular{
                    amuú[{\downstep}mbóhólólé{\downstep}láángá] tá}  &   
                     \gloss{‘untying’}  &  \\
\end{tabular}
%\caption{\nocaption}
     
\begin{tabular}{llllll}  
  \multicolumn{5}{l}{
                     \vernacular{(224) /H/
                    V-Initial + OP + OP
                    } \gloss{‘s/he is
                    not...him/her for me’} } &  \\
\multicolumn{5}{l}{ } &  \\

                     \vernacular{
                    amuú[{\downstep}nzíríláá{\downstep}ngá] tá}  &   
                     \gloss{‘killing’}  &  \\

                     \vernacular{
                    amuú[{\downstep}nzéchítsílá{\downstep}ángá] tá}  &   
                     \gloss{‘admiring’}  &  \\

                     \vernacular{
                    amuú[{\downstep}nzísíáchíláá{\downstep}ngá] tá}  &   
                     \gloss{‘smacking’}  &  \\

                     \vernacular{
                    amuú[{\downstep}nzónónyínyí{\downstep}láángá]
                    tá}  &   
                     \gloss{‘spoiling’}  &  \\

                     \vernacular{
                    amuú[{\downstep}nzábúkhányí{\downstep}nyíláángá]
                    tá}  &   
                     \gloss{‘separating’}  &  \\
\end{tabular}
%\caption{\nocaption}
     
\begin{tabular}{llllll}  
  \multicolumn{5}{l}{
                     \vernacular{(225) /Ø/
                    C-Initial + OP + OP
                    } \gloss{‘s/he is
                    not...him/her for me’} } &  \\
\multicolumn{5}{l}{ } &  \\

                     \vernacular{
                    amuú[{\downstep}nzíílá{\downstep}ángá] tá}  &   
                     \gloss{‘going for’}  &  \\

                     \vernacular{
                    amuú[{\downstep}ndéshélá{\downstep}ángá] tá}  &   
                     \gloss{‘leaving’}  &  \\

                     \vernacular{
                    amuú[{\downstep}nóóndé{\downstep}láángá] tá}  &   
                     \gloss{‘following’}  &  \\

                     \vernacular{
                    amuú[{\downstep}ngúlíshí{\downstep}láángá] tá}  &   
                     \gloss{‘naming’}  &  \\

                     \vernacular{
                    amuú[{\downstep}ndákhú{\downstep}úlíláángá] tá}  &   
                     \gloss{‘releasing’}  &  \\

                     \vernacular{
                    amuú[{\downstep}séébú{\downstep}líláángá] tá}  &   
                     \gloss{‘saying bye
                    to’}  &  \\

                     \vernacular{
                    amuú[{\downstep}mbóómbé{\downstep}lítsíláángá]
                    tá}  &   
                     \gloss{‘comforting’}  &  \\

                     \vernacular{
                    amuú[{\downstep}síínjí{\downstep}lítsíláángá]
                    tá}  &   
                     \gloss{
                    ‘making...stand’}  &  \\
\end{tabular}
%\caption{\nocaption}
     
\begin{tabular}{llllll}  
  \multicolumn{5}{l}{
                     \vernacular{(226) /Ø/
                    V-Initial + OP + OP
                    } \gloss{‘s/he is
                    not...him/her \ob mu-\cb  / it
                    } } &  \\
\multicolumn{5}{l}{ } &  \\

                     \vernacular{
                    amuú[{\downstep}nzéyélá{\downstep}ángá] tá}  &   
                     \gloss{‘wiping for’}  &  \\

                     \vernacular{
                    akuú[nzáshítsí{\downstep}láángá] tá}  &   
                     \gloss{‘lighting’}  &  \\

                     \vernacular{
                    abuú[{\downstep}nzílú{\downstep}úlíláángá] tá}  &   
                     \gloss{‘winnowing’}  &  \\

                     \vernacular{
                    aluú[nzítsúlí{\downstep}tsíláángá] tá}  &   
                     \gloss{‘filling’}  &  \\

                     \vernacular{
                    akuú[nzélé{\downstep}élítsíláángá]
                    tá}  &   
                     \gloss{‘hanging’}  &  \\
\end{tabular}
%\caption{\nocaption}
     
\begin{tabular}{lll}  
  \multicolumn{2}{l}{
                     \vernacular{(227) /H/
                    C-Initial Phrase-Medial} \gloss{‘s/he is
                    not...the man \ob musáatsa\cb  /} } &  \\
\multicolumn{2}{l}{
                     \gloss{the boy
                    \ob mú{\downstep}yáyi\cb  / someone \ob muundu\cb ’} } &  \\

                     \vernacular{a[reetsaanga]
                    musáatsa/mú{\downstep}yáyi/muundu tá}  &   
                     \gloss{‘burying’}  &  \\

                     \vernacular{a[bekaanga]
                    musáatsa/mú{\downstep}yáyi/muundu tá}  &   
                     \gloss{‘shaving’}  &  \\

                     \vernacular{a[leeraanga]
                    musáatsa/mú{\downstep}yáyi/muundu tá}  &   
                     \gloss{‘bringing’}  &  \\

                     \vernacular{a[khalakaanga]
                    musáatsa/mú{\downstep}yáyi/muundu tá}  &   
                     \gloss{‘cutting’}  &  \\

                     \vernacular{a[sitaakaanga]
                    musáatsa/mú{\downstep}yáyi/muundu tá}  &   
                     \gloss{‘accusing’}  &  \\

                     \vernacular{a[boolitsaanga]
                    musáatsa/mú{\downstep}yáyi/muundu tá}  &   
                     \gloss{‘seducing’}  &  \\

                     \vernacular{a[tsuunzuunaanga]
                    musáatsa/mú{\downstep}yáyi/muundu tá}  &   
                     \gloss{‘sucking’}  &  \\

                     \vernacular{a[bohololaanga]
                    musáatsa/mú{\downstep}yáyi/muundu tá}  &   
                     \gloss{‘untying’}  &  \\

                     \vernacular{a[boyong’anaanga]
                    musáatsa/mú{\downstep}yáyi/muundu tá}  &   
                     \gloss{‘going
                    around’}  &  \\
\end{tabular}
%\caption{\nocaption}
     
\begin{tabular}{lll}  
  \multicolumn{2}{l}{
                     \vernacular{(228) /Ø/
                    C-Initial Phrase-Medial} \gloss{‘s/he is
                    not...the man \ob musáatsa\cb  /} } &  \\
\multicolumn{2}{l}{
                     \gloss{the boy
                    \ob mú{\downstep}yáyi\cb  / someone \ob muundu\cb ’} } &  \\

                     \vernacular{a[tsiitsaanga]
                    musáatsa/mú{\downstep}yáyi/muundu tá}  &   
                     \gloss{‘going for’}  &  \\

                     \vernacular{a[lekhaanga]
                    musáatsa/mú{\downstep}yáyi/muundu tá}  &   
                     \gloss{‘leaving’}  &  \\

                     \vernacular{a[loondaanga]
                    musáatsa/mú{\downstep}yáyi/muundu tá}  &   
                     \gloss{‘following’}  &  \\

                     \vernacular{a[kulikhaanga]
                    musáatsa/mú{\downstep}yáyi/muundu tá}  &   
                     \gloss{‘naming’}  &  \\

                     \vernacular{a[lakhuulaanga]
                    musáatsa/mú{\downstep}yáyi/muundu tá}  &   
                     \gloss{‘releasing’}  &  \\

                     \vernacular{a[seebulaanga]
                    musáatsa/mú{\downstep}yáyi/muundu tá}  &   
                     \gloss{‘saying bye
                    to’}  &  \\

                     \vernacular{a[kalushitsaanga]
                    musáatsa/mú{\downstep}yáyi/muundu tá}  &   
                     \gloss{‘returning’}  &  \\

                     \vernacular{a[reebareebaanga]
                    musáatsa/mú{\downstep}yáyi/muundu tá}  &   
                     \gloss{‘asking
                    (iter)’}  &  \\
\end{tabular}
%\caption{\nocaption}
     
\begin{tabular}{lll}  
  \multicolumn{2}{l}{
                     \vernacular{(229) /H/
                    C-Initial +OP Phrase-Medial} \gloss{‘s/he is
                    not...the man \ob musáatsa\cb  /} } &  \\
\multicolumn{2}{l}{
                     \gloss{the boy
                    \ob mú{\downstep}yáyi\cb  / someone \ob muundu\cb  for
                    him/her’} } &  \\

                     \vernacular{amu[réelaanga]
                    musáatsa/mú{\downstep}yáyi/muundu tá}  &   
                     \gloss{‘burying’}  &  \\

                     \vernacular{amu[béchelaanga]
                    musáatsa/mú{\downstep}yáyi/muundu tá}  &   
                     \gloss{‘shaving’}  &  \\

                     \vernacular{amu[léerelaanga]
                    musáatsa/mú{\downstep}yáyi/muundu tá}  &   
                     \gloss{‘bringing’}  &  \\

                     \vernacular{
                    amu[khálachilaanga] musáatsa/mú{\downstep}yáyi/muundu
                    tá}  &   
                     \gloss{‘cutting’}  &  \\

                     \vernacular{
                    amu[sítaachilaanga] musáatsa/mú{\downstep}yáyi/muundu
                    tá}  &   
                     \gloss{‘accusing’}  &  \\

                     \vernacular{
                    amu[bóolitsilaanga] musáatsa/mú{\downstep}yáyi/muundu
                    tá}  &   
                     \gloss{‘seducing’}  &  \\

                     \vernacular{
                    amu[tsúunzuunilaanga]
                    musáatsa/mú{\downstep}yáyi/muundu tá}  &   
                     \gloss{‘sucking’}  &  \\

                     \vernacular{
                    amu[bóhololelaanga] musáatsa/mú{\downstep}yáyi/muundu
                    tá}  &   
                     \gloss{‘untying’}  &  \\

                     \vernacular{
                    amu[bóyong’anilaanga]
                    musáatsa/mú{\downstep}yáyi/muundu tá}  &   
                     \gloss{‘going
                    around’}  &  \\
\end{tabular}
%\caption{\nocaption}
     
\begin{tabular}{lll}  
  \multicolumn{2}{l}{
                     \vernacular{(230) /Ø/
                    C-Initial +OP Phrase-Medial} \gloss{‘s/he is
                    not...the man \ob musáatsa\cb  /} } &  \\
\multicolumn{2}{l}{
                     \gloss{the boy
                    \ob mú{\downstep}yáyi\cb  / someone \ob muundu\cb  for
                    him/her’} } &  \\

                     \vernacular{amu[tsiilaanga]
                    musáatsa/mú{\downstep}yáyi/muundu tá}  &   
                     \gloss{‘going for’}  &  \\

                     \vernacular{amu[leshelaanga]
                    musáatsa/mú{\downstep}yáyi/muundu tá}  &   
                     \gloss{‘leaving’}  &  \\

                     \vernacular{amu[loondelaanga]
                    musáatsa/mú{\downstep}yáyi/muundu tá}  &   
                     \gloss{‘following’}  &  \\

                     \vernacular{amu[kulishilaanga]
                    musáatsa/mú{\downstep}yáyi/muundu tá}  &   
                     \gloss{‘naming’}  &  \\

                     \vernacular{
                    amu[lakhuulilaanga] musáatsa/mú{\downstep}yáyi/muundu
                    tá}  &   
                     \gloss{‘releasing’}  &  \\

                     \vernacular{amu[seebulilaanga]
                    musáatsa/mú{\downstep}yáyi/muundu tá}  &   
                     \gloss{‘saying bye
                    to’}  &  \\

                     \vernacular{
                    amu[kalushitsilaanga]
                    musáatsa/mú{\downstep}yáyi/muundu tá}  &   
                     \gloss{‘returning’}  &  \\

                     \vernacular{
                    amu[reebareebelaanga]
                    musáatsa/mú{\downstep}yáyi/muundu tá}  &   
                     \gloss{‘asking
                    (iter)’}  &  \\
\end{tabular}
%\caption{\nocaption}
     
\begin{tabular}{lll}  
  \multicolumn{2}{l}{
                     \vernacular{(231) /H/
                    C-Initial +OP + OP
                    } \gloss{‘s/he is
                    not...the man \ob musáatsa\cb  /} } &  \\
\multicolumn{2}{l}{
                     \gloss{the boy
                    \ob mú{\downstep}yáyi\cb  / someone \ob muundu\cb  for him/her for
                    me’} } &  \\

                     \vernacular{amuú[ndeelaanga]
                    musáatsa/mú{\downstep}yáyi/muundu tá}  &   
                     \gloss{‘buring’}  &  \\

                     \vernacular{
                    amuú[mbechelaanga] musáatsa/mú{\downstep}yáyi/muundu
                    tá}  &   
                     \gloss{‘shaving’}  &  \\

                     \vernacular{
                    amuú[ndeerelaanga] musáatsa/mú{\downstep}yáyi/muundu
                    tá}  &   
                     \gloss{‘bringing’}  &  \\

                     \vernacular{
                    amuú[khalachilaanga]
                    musáatsa/mú{\downstep}yáyi/muundu tá}  &   
                     \gloss{‘cutting’}  &  \\

                     \vernacular{
                    amuú[sitaachilaanga]
                    musáatsa/mú{\downstep}yáyi/muundu tá}  &   
                     \gloss{‘accusing’}  &  \\

                     \vernacular{
                    amuú[mboolitsilaanga]
                    musáatsa/mú{\downstep}yáyi/muundu tá}  &   
                     \gloss{‘seducing’}  &  \\

                     \vernacular{
                    amuú[mbohololelaanga]
                    musáatsa/mú{\downstep}yáyi/muundu tá}  &   
                     \gloss{‘untying’}  &  \\
\end{tabular}
%\caption{\nocaption}
     
\begin{tabular}{lll}  
  \multicolumn{2}{l}{
                     \vernacular{(232) /Ø/
                    C-Initial +OP + OP
                    } \gloss{‘s/he is
                    not...the man \ob musáatsa\cb  /} } &  \\
\multicolumn{2}{l}{
                     \gloss{the boy
                    \ob mú{\downstep}yáyi\cb  / someone \ob muundu\cb  for him/her for
                    me’} } &  \\

                     \vernacular{amuú[nziilaanga]
                    musáatsa/mú{\downstep}yáyi/muundu tá}  &   
                     \gloss{‘going for’}  &  \\

                     \vernacular{
                    amuú[ndeshelaanga] musáatsa/mú{\downstep}yáyi/muundu
                    tá}  &   
                     \gloss{‘leaving’}  &  \\

                     \vernacular{
                    amuú[noondelaanga] musáatsa/mú{\downstep}yáyi/muundu
                    tá}  &   
                     \gloss{‘following’}  &  \\

                     \vernacular{
                    amuú[ngulishilaanga]
                    musáatsa/mú{\downstep}yáyi/muundu tá}  &   
                     \gloss{‘naming’}  &  \\

                     \vernacular{
                    amuú[ndakhuulilaanga]
                    musáatsa/mú{\downstep}yáyi/muundu tá}  &   
                     \gloss{‘releasing’}  &  \\

                     \vernacular{
                    amuú[seebulilaanga] musáatsa/mú{\downstep}yáyi/muundu
                    tá}  &   
                     \gloss{‘saying bye
                    to’}  &  \\
\end{tabular}
%\caption{\nocaption}
    

\subsection{Indefinite Future: Pattern 5b}\label{sec:sIndefFut}


\begin{tabular}{llllll}  
  \multicolumn{5}{l}{
                     \vernacular{(233) /H/
                    C-Initial} \gloss{‘s/he
                    will...’} } &  \\
\multicolumn{5}{l}{ } &  \\

                     \vernacular{ali[ra]}  &   
                     \gloss{‘bury’}  &     &   
                     \vernacular{
                    ali[ng’wa]}  &   
                     \gloss{‘drink’}  &  \\

                     \vernacular{
                    ali[khwa]}  &   
                     \gloss{‘eat’}  &     &   
                     \vernacular{
                    ali[lia]}  &   
                     \gloss{‘pay dowry’}  &  \\

                     \vernacular{
                    ali[luma]}  &   
                     \gloss{‘bite’}  &     &   
                     \vernacular{
                    ali[beka]}  &   
                     \gloss{‘shave’}  &  \\

                     \vernacular{
                    ali[teekhá]}  &   
                     \gloss{‘cook’}  &     &   
                     \vernacular{
                    ali[leerá]}  &   
                     \gloss{‘bring’}  &  \\

                     \vernacular{
                    ali[khalaká]}  &   
                     \gloss{‘cut’}  &     &   
                     \vernacular{
                    ali[kalaangá]}  &   
                     \gloss{‘fry’}  &  \\

                     \vernacular{
                    ali[sitaaká]}  &   
                     \gloss{‘accuse’}  &     &   
                     \vernacular{
                    ali[boolitsá]}  &   
                     \gloss{‘seduce’}  &  \\

                     \vernacular{
                    ali[saanditsá]}  &   
                     \gloss{‘thank’}  &     &   
                     \vernacular{
                    ali[khong’oondá]}  &   
                     \gloss{‘knock’}  &  \\

                     \vernacular{
                    ali[bohololá]}  &   
                     \gloss{‘untie’}  &     &   
                     \vernacular{
                    ali[boyong’aná]}  &   
                     \gloss{‘go around’}  &  \\

                     \vernacular{
                    ali[ng’ong’oolitsá]}  &   
                     \gloss{‘tease’}  &     &   
                     \vernacular{
                    ali[lingakanyinyá]}  &   
                     \gloss{‘crumple’}  &  \\
\end{tabular}
%\caption{\nocaption}
     
\begin{tabular}{llllll}  
  \multicolumn{5}{l}{
                     \vernacular{(234) /H/
                    V-Initial} \gloss{‘s/he will...’
                    } } &  \\
\multicolumn{5}{l}{ } &  \\

                     \vernacular{
                    al[iirá]}  &   
                     \gloss{‘kill’}  &     &   
                     \vernacular{
                    al[iikoómba]}  &   
                     \gloss{‘admire’}  &  \\

                     \vernacular{
                    al[iisiáka]}  &   
                     \gloss{‘smack’}  &     &   
                     \vernacular{
                    al[iikobóla]}  &   
                     \gloss{‘belch’}  &  \\

                     \vernacular{
                    al[iononyínya]}  &   
                     \gloss{‘spoil’}  &     &   
                     \vernacular{
                    al[iabukhanyínya]}  &   
                     \gloss{‘separate’}  &  \\
\end{tabular}
%\caption{\nocaption}
     
\begin{tabular}{llllll}  
  \multicolumn{5}{l}{
                     \vernacular{(235) /Ø/
                    C-Initial} \gloss{‘s/he
                    will...’} } &  \\
\multicolumn{5}{l}{ } &  \\

                     \vernacular{
                    ali[tsía]}  &   
                     \gloss{‘go’}  &     &   
                     \vernacular{
                    ali[kwá]}  &   
                     \gloss{‘fall’}  &  \\

                     \vernacular{
                    ali[lekhá]}  &   
                     \gloss{‘leave’}  &     &   
                     \vernacular{
                    ali[reéba]}  &   
                     \gloss{‘ask’}  &  \\

                     \vernacular{
                    ali[loónda]}  &   
                     \gloss{‘follow’}  &     &   
                     \vernacular{
                    ali[kumíla]}  &   
                     \gloss{‘hold’}  &  \\

                     \vernacular{
                    ali[kulíkha]}  &   
                     \gloss{‘name’}  &     &   
                     \vernacular{
                    ali[homóola]}  &   
                     \gloss{‘massage’}  &  \\

                     \vernacular{
                    ali[lakhúula]}  &   
                     \gloss{‘release’}  &     &   
                     \vernacular{
                    ali[seébula]}  &   
                     \gloss{‘say bye’}  &  \\

                     \vernacular{
                    ali[hoómbélitsa]}  &   
                     \gloss{‘comfort’}  &     &   
                     \vernacular{
                    ali[kalúshitsa]}  &   
                     \gloss{‘return’}  &  \\

                     \vernacular{
                    ali[siínjílitsa]}  &   
                     \gloss{‘make stand’}  &     &   
                     \vernacular{
                    ali[reébáreeba]}  &   
                     \gloss{‘ask (iter)’}  &  \\

                     \vernacular{
                    ali[kalúkhányinya]}  &   
                     \gloss{‘turn over’}  &     &   
                     \vernacular{
                    ali[sebúlúkhanyinya]}  &   
                     \gloss{‘scatter’}  &  \\
\end{tabular}
%\caption{\nocaption}
     
\begin{tabular}{llllll}  
  \multicolumn{5}{l}{
                     \vernacular{(236) /Ø/
                    V-Initial} \gloss{‘s/he
                    will...’} } &  \\
\multicolumn{5}{l}{ } &  \\

                     \vernacular{
                    al[ienyá]}  &   
                     \gloss{‘want’}  &     &   
                     \vernacular{
                    al[ieyéla]}  &   
                     \gloss{‘wipe for’}  &  \\

                     \vernacular{
                    al[iilúula]}  &   
                     \gloss{‘winnow’}  &     &   
                     \vernacular{
                    al[iambákhana]}  &   
                     \gloss{‘refuse’}  &  \\

                     \vernacular{
                    al[ieléelitsa]}  &   
                     \gloss{‘hang up’}  &     &     &     &  \\
\end{tabular}
%\caption{\nocaption}
     
\begin{tabular}{llllll}  
  \multicolumn{5}{l}{
                     \vernacular{(237) /H/
                    C-Initial + OP} \gloss{‘s/he
                    will...him/her’} } &  \\
\multicolumn{5}{l}{ } &  \\

                     \vernacular{
                    alimu[rá]}  &   
                     \gloss{‘bury’}  &     &   
                     \vernacular{
                    alimu[bé{\downstep}ká]}  &   
                     \gloss{‘shave’}  &  \\

                     \vernacular{
                    alimu[lé{\downstep}érá]}  &   
                     \gloss{‘bring’}  &     &   
                     \vernacular{
                    alimu[khá{\downstep}láká]}  &   
                     \gloss{‘cut’}  &  \\

                     \vernacular{
                    alimu[sí{\downstep}tááká]}  &   
                     \gloss{‘accuse’}  &     &   
                     \vernacular{
                    alimu[bó{\downstep}ólítsá]}  &   
                     \gloss{‘seduce’}  &  \\

                     \vernacular{
                    alimu[khó{\downstep}ng’óóndá]}  &   
                     \gloss{‘knock’}  &     &   
                     \vernacular{
                    alimu[bó{\downstep}hólólá]}  &   
                     \gloss{‘untie’}  &  \\

                     \vernacular{
                    alimu[bó{\downstep}yóng’áná]}  &   
                     \gloss{‘go around’}  &     &   
                     \vernacular{
                    alimu[ng’ó{\downstep}ng’óólítsá]}  &   
                     \gloss{‘tease’}  &  \\

                     \vernacular{
                    alimu[lí{\downstep}ngákányínyá]}  &   
                     \gloss{‘bend’}  &     &     &     &  \\
  &  \\

                     \vernacular{
                    balimu[rá]}  &   
                     \gloss{‘buried’}  &     &   
                     \vernacular{
                    balimu[bé{\downstep}ká]}  &   
                     \gloss{‘shave’}  &  \\

                     \vernacular{
                    balimu[lé{\downstep}érá]}  &   
                     \gloss{‘bring’}  &     &   
                     \vernacular{
                    balimu[khá{\downstep}láká]}  &   
                     \gloss{‘cut’}  &  \\

                     \vernacular{
                    balimu[sí{\downstep}tááká]}  &   
                     \gloss{‘accuse’}  &     &   
                     \vernacular{
                    balimu[bó{\downstep}ólítsá]}  &   
                     \gloss{‘seduce’}  &  \\

                     \vernacular{
                    balimu[khó{\downstep}ng’óóndá]}  &   
                     \gloss{‘knock’}  &     &   
                     \vernacular{
                    balimu[bó{\downstep}hólólá]}  &   
                     \gloss{‘untie’}  &  \\

                     \vernacular{
                    balimu[bó{\downstep}yóng’áná]}  &   
                     \gloss{‘go around’}  &     &   
                     \vernacular{
                    balimu[ng’ó{\downstep}ng’óólítsá]}  &   
                     \gloss{‘tease’}  &  \\

                     \vernacular{
                    balimu[lí{\downstep}ngákányínyá]}  &   
                     \gloss{‘bend’}  &     &     &  \\
\end{tabular}
%\caption{\nocaption}
     
\begin{tabular}{llllll}  
  \multicolumn{5}{l}{
                     \vernacular{(238) /H/
                    V-Initial + OP} \gloss{‘s/he
                    will...him/her’} } &  \\
\multicolumn{5}{l}{ } &  \\

                     \vernacular{
                    alimw[ií{\downstep}rá]}  &   
                     \gloss{‘kill’}  &     &   
                     \vernacular{
                    alimw[ií{\downstep}kóómbá]}  &   
                     \gloss{‘admire’}  &  \\

                     \vernacular{
                    alimw[ií{\downstep}síáká]}  &   
                     \gloss{‘smack’}  &     &   
                     \vernacular{
                    alimw[oó{\downstep}nónyínya]}  &   
                     \gloss{‘spoil’}  &  \\

                     \vernacular{
                    alimw[aá{\downstep}búkhányínya]}  &   
                     \gloss{‘separate’}  &     &   
                     \vernacular{
                    alimw[ií{\downstep}rílá]}  &   
                     \gloss{‘kill for’}  &  \\
\end{tabular}
%\caption{\nocaption}
     
\begin{tabular}{llllll}  
  \multicolumn{5}{l}{
                     \vernacular{(239) /Ø/
                    C-Initial + OP} \gloss{‘s/he
                    will...him/her \ob mu-\cb  / them
                    } } &  \\
\multicolumn{5}{l}{ } &  \\

                     \vernacular{
                    alimu[tsía]}  &   
                     \gloss{‘go for’}  &  \\

                     \vernacular{
                    alimu[lekhá]}  &   
                     \gloss{‘leave’}  &  \\

                     \vernacular{
                    alimu[loónda]}  &   
                     \gloss{‘follow’}  &  \\

                     \vernacular{
                    alimu[kulíkha]}  &   
                     \gloss{‘name’}  &  \\

                     \vernacular{
                    alimu[lakhúula]}  &   
                     \gloss{‘release’}  &  \\

                     \vernacular{
                    alimu[seébula]}  &   
                     \gloss{‘say bye to’}  &  \\

                     \vernacular{
                    alimu[hoómbélitsa]}  &   
                     \gloss{‘comfort’}  &  \\

                     \vernacular{
                    alimu[kalúshitsa]}  &   
                     \gloss{‘return’}  &  \\

                     \vernacular{
                    alimu[siínjílitsa]}  &   
                     \gloss{
                    ‘make...stand’}  &  \\

                     \vernacular{
                    alimu[reébáreeba]}  &   
                     \gloss{‘ask (iter)’}  &  \\

                     \vernacular{
                    alimu[kalúkhányinya]}  &   
                     \gloss{
                    ‘turn...over’}  &  \\

                     \vernacular{
                    alibi[sebúlúkhanyinya]}  &   
                     \gloss{‘scatter’}  &  \\
\end{tabular}
%\caption{\nocaption}
     
\begin{tabular}{llllll}  
  \multicolumn{5}{l}{
                     \vernacular{(240) /Ø/
                    V-Initial + OP} \gloss{‘s/he
                    will...him/her \ob mw-\cb  / it
                    } } &  \\
\multicolumn{5}{l}{ } &  \\

                     \vernacular{
                    alimw[eenyá]}  &   
                     \gloss{‘want’}  &     &   
                     \vernacular{
                    alimw[eeyéla]}  &   
                     \gloss{‘wipe for’}  &  \\

                     \vernacular{
                    alibw[iilúula]}  &   
                     \gloss{‘winnow’}  &     &   
                     \vernacular{
                    alimw[aambákhana]}  &   
                     \gloss{‘refuse’}  &  \\

                     \vernacular{
                    alimw[eeléelitsa]}  &   
                     \gloss{
                    ‘carry...hanging’}  &  \\
\end{tabular}
%\caption{\nocaption}
     
\begin{tabular}{llllll}  
  \multicolumn{5}{l}{
                     \vernacular{(241) /H/
                    C-Initial + OP
                    } \gloss{‘s/he
                    will...me’} } &  \\
\multicolumn{5}{l}{ } &  \\

                     \vernacular{
                    alii[ría]}  &   
                     \gloss{‘fear’}  &     &   
                     \vernacular{
                    alii[mbé{\downstep}ká]}  &   
                     \gloss{‘shave’}  &  \\

                     \vernacular{
                    alii[ndé{\downstep}érá]}  &   
                     \gloss{‘bring’}  &     &   
                     \vernacular{
                    alii[khá{\downstep}láká]}  &   
                     \gloss{‘cut’}  &  \\

                     \vernacular{
                    alii[sí{\downstep}tááká]}  &   
                     \gloss{‘accuse’}  &     &   
                     \vernacular{
                    alii[mbó{\downstep}ólítsá]}  &   
                     \gloss{‘seduce’}  &  \\

                     \vernacular{
                    alii[khó{\downstep}ng’óóndá]}  &   
                     \gloss{‘knock’}  &     &   
                     \vernacular{
                    alii[mbó{\downstep}hólólá]}  &   
                     \gloss{‘untie’}  &  \\

                     \vernacular{
                    alii[mbó{\downstep}yóng’áná]}  &   
                     \gloss{‘go around’}  &     &   
                     \vernacular{
                    alii[ng’ó{\downstep}ng’óólítsá]}  &   
                     \gloss{‘tease’}  &  \\

                     \vernacular{
                    alii[ní{\downstep}ngákányínyá]}  &   
                     \gloss{‘bend’}  &  \\
\end{tabular}
%\caption{\nocaption}
     
\begin{tabular}{llllll}  
  \multicolumn{5}{l}{
                     \vernacular{(242) /H/
                    V-Initial + OP
                    } \gloss{‘s/he
                    will...me’} } &  \\
\multicolumn{5}{l}{ } &  \\

                     \vernacular{
                    alii[nzí{\downstep}rá]}  &   
                     \gloss{‘kill’}  &     &   
                     \vernacular{
                    alii[nzí{\downstep}kóómbá]}  &   
                     \gloss{‘admire’}  &  \\

                     \vernacular{
                    alii[nzí{\downstep}síáká]}  &   
                     \gloss{‘smack’}  &     &   
                     \vernacular{
                    alii[nzó{\downstep}nónyínyá]}  &   
                     \gloss{‘spoil’}  &  \\

                     \vernacular{
                    alii[nzá{\downstep}búkhányínyá]}  &   
                     \gloss{‘separate’}  &     &   
                     \vernacular{
                    alii[nzí{\downstep}rílá]}  &   
                     \gloss{‘kill for’}  &  \\
\end{tabular}
%\caption{\nocaption}
     
\begin{tabular}{llllll}  
  \multicolumn{5}{l}{
                     \vernacular{(243) /Ø/
                    C-Initial + OP
                    } \gloss{‘s/he
                    will...me’} } &  \\
\multicolumn{5}{l}{ } &  \\

                     \vernacular{
                    alii[sía]}  &   
                     \gloss{‘grind’}  &     &   
                     \vernacular{
                    alii[ndekhá]}  &   
                     \gloss{‘leave’}  &  \\

                     \vernacular{
                    alii[noónda]}  &   
                     \gloss{‘follow’}  &     &   
                     \vernacular{
                    alii[ngulíkha]}  &   
                     \gloss{‘name’}  &  \\

                     \vernacular{
                    alii[ndakhúula]}  &   
                     \gloss{‘release’}  &     &   
                     \vernacular{
                    alii[seébula]}  &   
                     \gloss{‘say bye to’}  &  \\

                     \vernacular{
                    alii[mboómbélitsa]}  &   
                     \gloss{‘comfort’}  &     &   
                     \vernacular{
                    alii[siínjílitsa]}  &   
                     \gloss{
                    ‘make..stand’}  &  \\

                     \vernacular{
                    alii[ndeébándeeba]}  &   
                     \gloss{‘ask (iter)’}  &     &   
                     \vernacular{
                    alii[ngalúkhányinya]}  &   
                     \gloss{
                    ‘turn...over’}  &  \\
\end{tabular}
%\caption{\nocaption}
     
\begin{tabular}{llllll}  
  \multicolumn{5}{l}{
                     \vernacular{(244) /Ø/
                    V-Initial + OP
                    } \gloss{‘s/he
                    will...me’} } &  \\
\multicolumn{5}{l}{ } &  \\

                     \vernacular{
                    alii[nzenyá]}  &   
                     \gloss{‘want’}  &     &   
                     \vernacular{
                    alii[nzeyéla]}  &   
                     \gloss{‘wipe for’}  &  \\

                     \vernacular{
                    alii[nyambákhana]}  &   
                     \gloss{‘refuse’}  &     &   
                     \vernacular{
                    alii[nzeléelitsa]}  &   
                     \gloss{
                    ‘carry...hanging’}  &  \\
\end{tabular}
%\caption{\nocaption}
     
\begin{tabular}{llllll}  
  \multicolumn{5}{l}{
                     \vernacular{(245) /H/
                    C-Initial + OP
                    } \gloss{‘s/he
                    will...him/herself’} } &  \\
\multicolumn{5}{l}{ } &  \\

                     \vernacular{
                    alii[rá]}  &   
                     \gloss{‘bury’}  &     &   
                     \vernacular{
                    alii[bé{\downstep}ká]}  &   
                     \gloss{‘shave’}  &  \\

                     \vernacular{
                    alii[sú{\downstep}úngá]}  &   
                     \gloss{‘hang’}  &     &   
                     \vernacular{
                    alii[khá{\downstep}láká]}  &   
                     \gloss{‘cut’}  &  \\

                     \vernacular{
                    alii[sí{\downstep}tááká]}  &   
                     \gloss{‘accuse’}  &     &   
                     \vernacular{
                    alii[sá{\downstep}ándítsá]}  &   
                     \gloss{‘thank’}  &  \\

                     \vernacular{
                    alii[khó{\downstep}ng’óóndá]}  &   
                     \gloss{‘knock’}  &     &   
                     \vernacular{
                    alii[bó{\downstep}hólólá]}  &   
                     \gloss{‘untie’}  &  \\
\end{tabular}
%\caption{\nocaption}
     
\begin{tabular}{llllll}  
  \multicolumn{5}{l}{
                     \vernacular{(246) /H/
                    V-Initial + OP
                    } \gloss{‘s/he
                    will...him/herself’} } &  \\
\multicolumn{5}{l}{ } &  \\

                     \vernacular{
                    alii[yí{\downstep}rá]}  &   
                     \gloss{‘kill’}  &     &   
                     \vernacular{
                    alii[yí{\downstep}kóómbá]}  &   
                     \gloss{‘admire’}  &  \\

                     \vernacular{
                    alii[yí{\downstep}síáká]}  &   
                     \gloss{‘smack’}  &     &   
                     \vernacular{
                    alii[yó{\downstep}nónyínyá]}  &   
                     \gloss{‘spoil’}  &  \\

                     \vernacular{
                    alii[yá{\downstep}búkhányínyá]}  &   
                     \gloss{‘separate’}  &     &   
                     \vernacular{
                    alii[yí{\downstep}rílá]}  &   
                     \gloss{‘kill for’}  &  \\
\end{tabular}
%\caption{\nocaption}
     
\begin{tabular}{llllll}  
  \multicolumn{5}{l}{
                     \vernacular{(247) /Ø/
                    C-Initial + OP
                    } \gloss{‘s/he
                    will...him/herself’} } &  \\
\multicolumn{5}{l}{ } &  \\

                     \vernacular{
                    alii[sía]}  &   
                     \gloss{‘grind’}  &     &   
                     \vernacular{
                    alii[lekhá]}  &   
                     \gloss{‘leave’}  &  \\

                     \vernacular{
                    alii[siínga]}  &   
                     \gloss{‘bathe’}  &     &   
                     \vernacular{
                    alii[kulíkha]}  &   
                     \gloss{‘name’}  &  \\

                     \vernacular{
                    alii[naábula]}  &   
                     \gloss{‘undress’}  &     &   
                     \vernacular{
                    alii[lakhúula]}  &   
                     \gloss{‘release’}  &  \\

                     \vernacular{
                    alii[hoómbélitsa]}  &   
                     \gloss{‘comfort’}  &     &   
                     \vernacular{
                    alii[siínjílitsa]}  &   
                     \gloss{
                    ‘make...stand’}  &  \\

                     \vernacular{
                    alii[reébáreeba]}  &   
                     \gloss{‘ask (iter)’}  &     &   
                     \vernacular{
                    alii[kalúkhányinya]}  &   
                     \gloss{
                    ‘turn...over’}  &  \\
\end{tabular}
%\caption{\nocaption}
     
\begin{tabular}{llllll}  
  \multicolumn{5}{l}{
                     \vernacular{(248) /Ø/
                    V-Initial + OP
                    } \gloss{‘s/he
                    will...him/herself’} } &  \\
\multicolumn{5}{l}{ } &  \\

                     \vernacular{
                    alii[yenyá]}  &   
                     \gloss{‘want’}  &     &   
                     \vernacular{
                    alii[yeyéla]}  &   
                     \gloss{‘wipe for’}  &  \\

                     \vernacular{
                    alii[yambákhana]}  &   
                     \gloss{‘refuse’}  &     &   
                     \vernacular{
                    alii[yeléelitsa]}  &   
                     \gloss{‘hang...up’}  &  \\
\end{tabular}
%\caption{\nocaption}
     
\begin{tabular}{llllll}  
  \multicolumn{5}{l}{
                     \vernacular{(249) /H/
                    C-Initial + OP + OP
                    } \gloss{‘s/he
                    will...him/her for me’} } &  \\
\multicolumn{5}{l}{ } &  \\

                     \vernacular{
                    alimuu[ndé{\downstep}élá]}  &   
                     \gloss{‘bury’}  &     &   
                     \vernacular{
                    alimuu[mbé{\downstep}chélá]}  &   
                     \gloss{‘shave’}  &  \\

                     \vernacular{
                    alimuu[ndé{\downstep}érélá]}  &   
                     \gloss{‘bring’}  &     &   
                     \vernacular{
                    alimuu[khá{\downstep}láchílá]}  &   
                     \gloss{‘cut’}  &  \\

                     \vernacular{
                    alimuu[sí{\downstep}tááchílá]}  &   
                     \gloss{‘accuse’}  &     &   
                     \vernacular{
                    alimuu[mbó{\downstep}ólítsílá]}  &   
                     \gloss{‘seduce’}  &  \\

                     \vernacular{
                    alimuu[mbó{\downstep}hólólélá]}  &   
                     \gloss{‘untie’}  &     &     &     &  \\
\end{tabular}
%\caption{\nocaption}
     
\begin{tabular}{llllll}  
  \multicolumn{5}{l}{
                     \vernacular{(250) /H/
                    V-Initial + OP + OP
                    } \gloss{‘s/he
                    will...him/her for me’} } &  \\
\multicolumn{5}{l}{ } &  \\

                     \vernacular{
                    alimuu[nzí{\downstep}rílá]}  &   
                     \gloss{‘kill’}  &     &   
                     \vernacular{
                    alimuu[nzé{\downstep}chítsílá]}  &   
                     \gloss{‘admire’}  &  \\

                     \vernacular{
                    alimuu[nzí{\downstep}síáchílá]}  &   
                     \gloss{‘smack’}  &     &   
                     \vernacular{
                    alimuu[nzó{\downstep}nónyínyílá]}  &   
                     \gloss{‘spoil’}  &  \\

                     \vernacular{
                    alimuu[nzá{\downstep}búkhányínyílá]}  &   
                     \gloss{‘separate’}  &     &     &     &  \\
\end{tabular}
%\caption{\nocaption}
     
\begin{tabular}{llllll}  
  \multicolumn{5}{l}{
                     \vernacular{(251) /Ø/
                    C-Initial + OP + OP
                    } \gloss{‘s/he
                    will...him/her for me’} } &  \\
\multicolumn{5}{l}{ } &  \\

                     \vernacular{
                    alimuu[nziíla]}  &   
                     \gloss{‘go for’}  &     &   
                     \vernacular{
                    alimuu[ndeshéla]}  &   
                     \gloss{‘leave’}  &  \\

                     \vernacular{
                    alimuu[noóndela]}  &   
                     \gloss{‘follow’}  &     &   
                     \vernacular{
                    alimuu[ngulíshila]}  &   
                     \gloss{‘name’}  &  \\

                     \vernacular{
                    alimuu[ndakhúulila]}  &   
                     \gloss{‘release’}  &     &   
                     \vernacular{
                    alimuu[seébúlila]}  &   
                     \gloss{‘say bye to’}  &  \\

                     \vernacular{
                    alimuu[mboómbélitsila]}  &   
                     \gloss{‘comfort’}  &     &   
                     \vernacular{
                    alimuu[siínjílitsila]}  &   
                     \gloss{
                    ‘make...stand’}  &  \\
\end{tabular}
%\caption{\nocaption}
     
\begin{tabular}{llllll}  
  \multicolumn{5}{l}{
                     \vernacular{(252) /Ø/
                    V-Initial + OP + OP
                    } \gloss{‘s/he
                    will...him/her \ob mu-\cb  / it
                    } } &  \\
\multicolumn{5}{l}{ } &  \\

                     \vernacular{
                    alimuu[nzeyéla]}  &   
                     \gloss{‘wipe’}  &     &   
                     \vernacular{
                    alikuu[nzashítsila]}  &   
                     \gloss{‘light’}  &  \\

                     \vernacular{
                    alibuu[nzilúulila]}  &   
                     \gloss{‘winnow’}  &     &   
                     \vernacular{
                    aliluu[nzitsúlitsila]}  &   
                     \gloss{‘fill’}  &  \\

                     \vernacular{
                    alikuu[nzeléelitsila]}  &   
                     \gloss{‘hang’}  &     &     &     &  \\
\end{tabular}
%\caption{\nocaption}
     
\begin{tabular}{lll}  
  \multicolumn{2}{l}{
                     \vernacular{(253) /H/
                    C-Initial Phrase-Medial} \gloss{‘s/he will...the
                    man \ob musáatsa\cb  /} } &  \\
\multicolumn{2}{l}{
                     \gloss{the boy
                    \ob mú{\downstep}yáyi\cb  / someone \ob muundu\cb ’} } &  \\

                     \vernacular{ali[ra]
                    musáatsa/mú{\downstep}yáyi/muundu}  &   
                     \gloss{‘bury’}  &  \\

                     \vernacular{ali[beka]
                    musáatsa/mú{\downstep}yáyi/muundu}  &   
                     \gloss{‘shave’}  &  \\

                     \vernacular{ali[leera]
                    musáatsa/mú{\downstep}yáyi/muundu}  &   
                     \gloss{‘bring’}  &  \\

                     \vernacular{ali[khalaka]
                    musáatsa/mú{\downstep}yáyi/muundu}  &   
                     \gloss{‘cut’}  &  \\

                     \vernacular{ali[sitaaka]
                    musáatsa/mú{\downstep}yáyi/muundu}  &   
                     \gloss{‘accuse’}  &  \\

                     \vernacular{ali[boolitsa]
                    musáatsa/mú{\downstep}yáyi/muundu}  &   
                     \gloss{‘seduce’}  &  \\

                     \vernacular{ali[khong’oonda]
                    musáatsa/mú{\downstep}yáyi/muundu}  &   
                     \gloss{‘knock’}  &  \\

                     \vernacular{ali[boholola]
                    musáatsa/mú{\downstep}yáyi/muundu}  &   
                     \gloss{‘untie’}  &  \\

                     \vernacular{ali[boyong’ana]
                    musáatsa/mú{\downstep}yáyi/muundu}  &   
                     \gloss{‘go around’}  &  \\

                     \vernacular{ali[lingakanyinya]
                    musáatsa/mú{\downstep}yáyi/muundu}  &   
                     \gloss{‘bend’}  &  \\
\end{tabular}
%\caption{\nocaption}
     
\begin{tabular}{lll}  
  \multicolumn{2}{l}{
                     \vernacular{(254) /Ø/
                    C-Initial Phrase-Medial} \gloss{‘s/he will...the
                    man \ob musáatsa\cb  /} } &  \\
\multicolumn{2}{l}{
                     \gloss{the boy
                    \ob mú{\downstep}yáyi\cb  / someone \ob muundu\cb ’} } &  \\

                     \vernacular{ali[tsia]
                    musáatsa/mú{\downstep}yáyi/muundu}  &   
                     \gloss{‘go for’}  &  \\

                     \vernacular{ali[lekha]
                    musáatsa/mú{\downstep}yáyi/muundu}  &   
                     \gloss{‘leave’}  &  \\

                     \vernacular{ali[loonda]
                    musáatsa/mú{\downstep}yáyi/muundu}  &   
                     \gloss{‘follow’}  &  \\

                     \vernacular{ali[kulikha]
                    musáatsa/mú{\downstep}yáyi/muundu}  &   
                     \gloss{‘name’}  &  \\

                     \vernacular{ali[lakhuula]
                    musáatsa/mú{\downstep}yáyi/muundu}  &   
                     \gloss{‘release’}  &  \\

                     \vernacular{ali[seebula]
                    musáatsa/mú{\downstep}yáyi/muundu}  &   
                     \gloss{‘say bye to’}  &  \\

                     \vernacular{ali[kalushitsa]
                    musáatsa/mú{\downstep}yáyi/muundu}  &   
                     \gloss{‘return’}  &  \\

                     \vernacular{ali[reebareeba]
                    musáatsa/mú{\downstep}yáyi/muundu}  &   
                     \gloss{‘ask (iter)’}  &  \\

                     \vernacular{ali[kalukhanyinya]
                    musáatsa/mú{\downstep}yáyi/muundu}  &   
                     \gloss{
                    ‘turn...over’}  &  \\
\end{tabular}
%\caption{\nocaption}
     
\begin{tabular}{lll}  
  \multicolumn{2}{l}{
                     \vernacular{(255) /H/
                    C-Initial +OP Phrase-Medial} \gloss{‘s/he will...the
                    man \ob musáatsa\cb  /} } &  \\
\multicolumn{2}{l}{
                     \gloss{the boy
                    \ob mú{\downstep}yáyi\cb  / someone \ob muundu\cb  for
                    him/her’} } &  \\

                     \vernacular{alimu[réela]
                    musáatsa/mú{\downstep}yáyi/muundu}  &   
                     \gloss{‘bury’}  &  \\

                     \vernacular{alimu[béchela]
                    musáatsa/mú{\downstep}yáyi/muundu}  &   
                     \gloss{‘shave’}  &  \\

                     \vernacular{alimu[léerela]
                    musáatsa/mú{\downstep}yáyi/muundu}  &   
                     \gloss{‘bring’}  &  \\

                     \vernacular{alimu[khálachila]
                    musáatsa/mú{\downstep}yáyi/muundu}  &   
                     \gloss{‘cut’}  &  \\

                     \vernacular{alimu[sítaachila]
                    musáatsa/mú{\downstep}yáyi/muundu}  &   
                     \gloss{‘accuse’}  &  \\

                     \vernacular{alimu[bóolitsila]
                    musáatsa/mú{\downstep}yáyi/muundu}  &   
                     \gloss{‘seduce’}  &  \\

                     \vernacular{
                    alimu[khóng’oondela]
                    musáatsa/mú{\downstep}yáyi/muundu}  &   
                     \gloss{‘knock’}  &  \\

                     \vernacular{alimu[bóhololela]
                    musáatsa/mú{\downstep}yáyi/muundu}  &   
                     \gloss{‘untie’}  &  \\

                     \vernacular{
                    alimu[bóyong’anila]
                    musáatsa/mú{\downstep}yáyi/muundu}  &   
                     \gloss{‘go around’}  &  \\

                     \vernacular{
                    alimu[língakanyinyila]
                    musáatsa/mú{\downstep}yáyi/muundu}  &   
                     \gloss{‘bend’}  &  \\
\end{tabular}
%\caption{\nocaption}
     
\begin{tabular}{lll}  
  \multicolumn{2}{l}{
                     \vernacular{(256) /Ø/
                    C-Initial +OP Phrase-Medial} \gloss{‘s/he will...the
                    man \ob musáatsa\cb  /} } &  \\
\multicolumn{2}{l}{
                     \gloss{the boy
                    \ob mú{\downstep}yáyi\cb  / someone \ob muundu\cb  for
                    him/her’} } &  \\

                     \vernacular{alimu[tsiila]
                    musáatsa/mú{\downstep}yáyi/muundu}  &   
                     \gloss{‘go for’}  &  \\

                     \vernacular{alimu[leshela]
                    musáatsa/mú{\downstep}yáyi/muundu}  &   
                     \gloss{‘leave’}  &  \\

                     \vernacular{alimu[loondela]
                    musáatsa/mú{\downstep}yáyi/muundu}  &   
                     \gloss{‘follow’}  &  \\

                     \vernacular{alimu[kulishila]
                    musáatsa/mú{\downstep}yáyi/muundu}  &   
                     \gloss{‘name’}  &  \\

                     \vernacular{alimu[lakhuulila]
                    musáatsa/mú{\downstep}yáyi/muundu}  &   
                     \gloss{‘release’}  &  \\

                     \vernacular{alimu[seebulila]
                    musáatsa/mú{\downstep}yáyi/muundu}  &   
                     \gloss{‘say bye to’}  &  \\

                     \vernacular{
                    alimu[kalushitsila]
                    musáatsa/mú{\downstep}yáyi/muundu}  &   
                     \gloss{‘return’}  &  \\

                     \vernacular{
                    alimu[reebareebela]
                    musáatsa/mú{\downstep}yáyi/muundu}  &   
                     \gloss{‘ask (iter)’}  &  \\

                     \vernacular{
                    alimu[kalukhanyinyila]
                    musáatsa/mú{\downstep}yáyi/muundu}  &   
                     \gloss{
                    ‘turn...over’}  &  \\
\end{tabular}
%\caption{\nocaption}
     
\begin{tabular}{lll}  
  \multicolumn{2}{l}{
                     \vernacular{(257) /H/
                    C-Initial +OP + OP
                    } \gloss{‘s/he will...the
                    man \ob musáatsa\cb  /} } &  \\
\multicolumn{2}{l}{
                     \gloss{the boy
                    \ob mú{\downstep}yáyi\cb  / someone \ob muundu\cb  for him/her for
                    me’
                    } } &  \\

                     \vernacular{alimuu[ndéela]
                    musáatsa/mú{\downstep}yáyi/muundu}  &   
                     \gloss{‘bury’}  &  \\

                     \vernacular{alimuu[mbéchela]
                    musáatsa/mú{\downstep}yáyi/muundu}  &   
                     \gloss{‘shave’}  &  \\

                     \vernacular{alimuu[ndéerela]
                    musáatsa/mú{\downstep}yáyi/muundu}  &   
                     \gloss{‘bring’}  &  \\

                     \vernacular{
                    alimuu[khálachila]
                    musáatsa/mú{\downstep}yáyi/muundu}  &   
                     \gloss{‘cut’}  &  \\

                     \vernacular{
                    alimuu[sítaachila]
                    musáatsa/mú{\downstep}yáyi/muundu}  &   
                     \gloss{‘accuse’}  &  \\

                     \vernacular{
                    alimuu[mbóolitsila]
                    musáatsa/mú{\downstep}yáyi/muundu}  &   
                     \gloss{‘seduce’}  &  \\

                     \vernacular{
                    alimuu[mbóhololela]
                    musáatsa/mú{\downstep}yáyi/muundu}  &   
                     \gloss{‘untie’}  &  \\
\end{tabular}
%\caption{\nocaption}
     
\begin{tabular}{lll}  
  \multicolumn{2}{l}{
                     \vernacular{(258) /Ø/
                    C-Initial +OP + OP
                    } \gloss{‘s/he will...the
                    man \ob musáatsa\cb  /} } &  \\
\multicolumn{2}{l}{
                     \gloss{the boy
                    \ob mú{\downstep}yáyi\cb  / someone \ob muundu\cb  for him/her for
                    me’} } &  \\

                     \vernacular{alimuu[nziila]
                    musáatsa/mú{\downstep}yáyi/muundu}  &   
                     \gloss{‘go for’}  &  \\

                     \vernacular{alimuu[ndeshela]
                    musáatsa/mú{\downstep}yáyi/muundu}  &   
                     \gloss{‘leave’}  &  \\

                     \vernacular{alimuu[noondela]
                    musáatsa/mú{\downstep}yáyi/muundu}  &   
                     \gloss{‘follow’}  &  \\

                     \vernacular{alimuu[ngulishila]
                    musáatsa/mú{\downstep}yáyi/muundu}  &   
                     \gloss{‘name’}  &  \\

                     \vernacular{
                    alimuu[ndakhuulila]
                    musáatsa/mú{\downstep}yáyi/muundu}  &   
                     \gloss{‘release’}  &  \\

                     \vernacular{alimuu[seebulila]
                    musáatsa/mú{\downstep}yáyi/muundu}  &   
                     \gloss{‘say bye to’}  &  \\

                     \vernacular{
                    alimuu[mboombelitsila]
                    musáatsa/mú{\downstep}yáyi/muundu}  &   
                     \gloss{‘comfort’}  &  \\

                     \vernacular{
                    alimuu[siinjilitsila]
                    musáatsa/mú{\downstep}yáyi/muundu}  &   
                     \gloss{
                    ‘make...stand’}  &  \\
\end{tabular}
%\caption{\nocaption}
    

\subsection{Indefinite Future Negative: Pattern
              5b}\label{sec:sIndefFutNeg}


\begin{tabular}{llllll}  
  \multicolumn{5}{l}{
                     \vernacular{(259) /H/
                    C-Initial} \gloss{‘s/he will
                    not...’} } &  \\
\multicolumn{5}{l}{ } &  \\

                     \vernacular{ali[ra]
                    tá}  &   
                     \gloss{‘bury’}  &     &   
                     \vernacular{ali[ng’wa]
                    tá}  &   
                     \gloss{‘drink’}  &  \\

                     \vernacular{ali[khwa]
                    tá}  &   
                     \gloss{‘eat’}  &     &   
                     \vernacular{ali[lia]
                    tá}  &   
                     \gloss{‘pay dowry’}  &  \\

                     \vernacular{ali[luma]
                    tá}  &   
                     \gloss{‘bite’}  &     &   
                     \vernacular{ali[beka]
                    tá}  &   
                     \gloss{‘shave’}  &  \\

                     \vernacular{ali[teekhá]
                    {\downstep}tá}  &   
                     \gloss{‘cook’}  &     &   
                     \vernacular{ali[leerá]
                    {\downstep}tá}  &   
                     \gloss{‘bring’}  &  \\

                     \vernacular{ali[khalaká]
                    {\downstep}tá}  &   
                     \gloss{‘cut’}  &     &   
                     \vernacular{ali[kalaangá]
                    {\downstep}tá}  &   
                     \gloss{‘fry’}  &  \\

                     \vernacular{ali[sitaaká]
                    {\downstep}tá}  &   
                     \gloss{‘accuse’}  &     &   
                     \vernacular{ali[boolitsá]
                    {\downstep}tá}  &   
                     \gloss{‘seduce’}  &  \\

                     \vernacular{ali[saanditsá]
                    {\downstep}tá}  &   
                     \gloss{‘thank’}  &     &   
                     \vernacular{ali[khong’oondá]
                    {\downstep}tá}  &   
                     \gloss{‘knock’}  &  \\

                     \vernacular{ali[bohololá]
                    {\downstep}tá}  &   
                     \gloss{‘untie’}  &     &   
                     \vernacular{ali[boyong’aná]
                    {\downstep}tá}  &   
                     \gloss{‘go around’}  &  \\

                     \vernacular{
                    ali[ng’ong’oolitsá] {\downstep}tá}  &   
                     \gloss{‘tease’}  &     &   
                     \vernacular{
                    ali[lingakanyinyá] {\downstep}tá}  &   
                     \gloss{‘crumple’}  &  \\
\end{tabular}
%\caption{\nocaption}
     
\begin{tabular}{llllll}  
  \multicolumn{5}{l}{
                     \vernacular{(260) /Ø/
                    C-Initial} \gloss{‘s/he will
                    not...’} } &  \\
\multicolumn{5}{l}{ } &  \\

                     \vernacular{ali[tsíá]
                    {\downstep}tá}  &   
                     \gloss{‘go’}  &     &   
                     \vernacular{ali[kwá]
                    {\downstep}tá}  &   
                     \gloss{‘fall’}  &  \\

                     \vernacular{ali[lekhá]
                    {\downstep}tá}  &   
                     \gloss{‘leave’}  &     &   
                     \vernacular{ali[reéba]
                    tá}  &   
                     \gloss{‘ask’}  &  \\

                     \vernacular{ali[loónda]
                    tá}  &   
                     \gloss{‘follow’}  &     &   
                     \vernacular{ali[kulíkha]
                    tá}  &   
                     \gloss{‘name’}  &  \\

                     \vernacular{ali[homóola]
                    tá}  &   
                     \gloss{‘massage’}  &     &   
                     \vernacular{ali[lakhúula]
                    tá}  &   
                     \gloss{‘release’}  &  \\

                     \vernacular{ali[seébúla]
                    tá}  &   
                     \gloss{‘say bye’}  &     &   
                     \vernacular{ali[hoómbélitsa]
                    tá}  &   
                     \gloss{‘comfort’}  &  \\

                     \vernacular{ali[kalúshítsa]
                    tá}  &   
                     \gloss{‘return’}  &     &   
                     \vernacular{ali[siínjílitsa]
                    tá}  &   
                     \gloss{‘make stand’}  &  \\

                     \vernacular{ali[reébáreeba]
                    tá}  &   
                     \gloss{‘ask (iter)’}  &     &   
                     \vernacular{
                    ali[kalúkhányinya] tá}  &   
                     \gloss{‘turn over’}  &  \\

                     \vernacular{
                    ali[sebúlúkhanyinya] tá}  &   
                     \gloss{‘scatter’}  &  \\
\end{tabular}
%\caption{\nocaption}
     
\begin{tabular}{llllll}  
  \multicolumn{5}{l}{
                     \vernacular{(261) /H/
                    C-Initial + OP} \gloss{‘s/he will
                    not...him/her’} } &  \\
\multicolumn{5}{l}{ } &  \\

                     \vernacular{alimu[rá]
                    {\downstep}tá}  &   
                     \gloss{‘bury’}  &     &   
                     \vernacular{alimu[bé{\downstep}ká]
                    {\downstep}tá}  &   
                     \gloss{‘shave’}  &  \\

                     \vernacular{alimu[lé{\downstep}érá]
                    {\downstep}tá}  &   
                     \gloss{‘bring’}  &     &   
                     \vernacular{alimu[khá{\downstep}láká]
                    {\downstep}tá}  &   
                     \gloss{‘cut’}  &  \\

                     \vernacular{
                    alimu[sí{\downstep}tááká] {\downstep}tá}  &   
                     \gloss{‘accuse’}  &     &   
                     \vernacular{
                    alimu[bó{\downstep}ólítsá] {\downstep}tá}  &   
                     \gloss{‘seduce’}  &  \\

                     \vernacular{
                    alimu[khó{\downstep}ng’óóndá] {\downstep}tá}  &   
                     \gloss{‘knock’}  &     &   
                     \vernacular{
                    alimu[bó{\downstep}hólólá] {\downstep}tá}  &   
                     \gloss{‘untie’}  &  \\

                     \vernacular{
                    alimu[bó{\downstep}yóng’áná] {\downstep}tá}  &   
                     \gloss{‘go around’}  &     &   
                     \vernacular{
                    alimu[ng’ó{\downstep}ng’óólítsá] {\downstep}tá}  &   
                     \gloss{‘tease’}  &  \\

                     \vernacular{
                    alimu[lí{\downstep}ngákányínyá] {\downstep}tá}  &   
                     \gloss{‘bend’}  &     &     &     &  \\
\end{tabular}
%\caption{\nocaption}
     
\begin{tabular}{llllll}  
  \multicolumn{5}{l}{
                     \vernacular{(262) /Ø/
                    C-Initial + OP} \gloss{‘s/he will
                    not...him/her \ob mu-\cb  / them
                    } } &  \\
\multicolumn{5}{l}{ } &  \\

                     \vernacular{alimu[tsíá]
                    {\downstep}tá}  &   
                     \gloss{‘go for’}  &  \\

                     \vernacular{alimu[lekhá]
                    {\downstep}tá}  &   
                     \gloss{‘leave’}  &  \\

                     \vernacular{alimu[loónda]
                    tá}  &   
                     \gloss{‘follow’}  &  \\

                     \vernacular{alimu[kulíkha]
                    tá}  &   
                     \gloss{‘name’}  &  \\

                     \vernacular{alimu[lakhúula]
                    tá}  &   
                     \gloss{‘release’}  &  \\

                     \vernacular{alimu[seébúla]
                    tá}  &   
                     \gloss{‘say bye to’}  &  \\

                     \vernacular{
                    alimu[hoómbélitsa] tá}  &   
                     \gloss{‘comfort’}  &  \\

                     \vernacular{
                    alimu[kalúshítsa] tá}  &   
                     \gloss{‘return’}  &  \\

                     \vernacular{
                    alimu[siínjílitsa] tá}  &   
                     \gloss{
                    ‘make...stand’}  &  \\

                     \vernacular{
                    alimu[reébáreeba] tá}  &   
                     \gloss{‘ask (iter)’}  &  \\

                     \vernacular{
                    alimu[kalúkhányinya] tá}  &   
                     \gloss{
                    ‘turn...over’}  &  \\

                     \vernacular{
                    alibi[sebúlúkhanyinya] tá}  &   
                     \gloss{‘scatter’}  &  \\
\end{tabular}
%\caption{\nocaption}
     
\begin{tabular}{llllll}  
  \multicolumn{5}{l}{
                     \vernacular{(263) /H/
                    C-Initial + OP + OP
                    } \gloss{‘s/he will
                    not...him/her for me’} } &  \\
\multicolumn{5}{l}{ } &  \\

                     \vernacular{alimuu[ndé{\downstep}élá]
                    {\downstep}tá}  &   
                     \gloss{‘bury’}  &     &   
                     \vernacular{
                    alimuu[mbé{\downstep}chélá] {\downstep}tá}  &   
                     \gloss{‘shave’}  &  \\

                     \vernacular{
                    alimuu[ndé{\downstep}érélá] {\downstep}tá}  &   
                     \gloss{‘bring’}  &     &   
                     \vernacular{
                    alimuu[khá{\downstep}láchílá] {\downstep}tá}  &   
                     \gloss{‘cut’}  &  \\

                     \vernacular{
                    alimuu[sí{\downstep}tááchílá] {\downstep}tá}  &   
                     \gloss{‘accuse’}  &     &   
                     \vernacular{
                    alimuu[mbó{\downstep}ólítsílá] {\downstep}tá}  &   
                     \gloss{‘seduce’}  &  \\

                     \vernacular{
                    alimuu[mbó{\downstep}hólólélá] {\downstep}tá}  &   
                     \gloss{‘untie’}  &     &     &     &  \\
\end{tabular}
%\caption{\nocaption}
     
\begin{tabular}{llllll}  
  \multicolumn{5}{l}{
                     \vernacular{(264) /Ø/
                    C-Initial + OP + OP
                    } \gloss{‘s/he will
                    not...him/her for me’} } &  \\
\multicolumn{5}{l}{ } &  \\

                     \vernacular{alimuu[nziíla]
                    tá}  &   
                     \gloss{‘go for’}  &     &   
                     \vernacular{alimuu[ndeshéla]
                    tá}  &   
                     \gloss{‘leave’}  &  \\

                     \vernacular{alimuu[noóndéla]
                    tá}  &   
                     \gloss{‘follow’}  &     &   
                     \vernacular{
                    alimuu[ngulíshíla] tá}  &   
                     \gloss{‘name’}  &  \\

                     \vernacular{
                    alimuu[ndakhúulila] tá}  &   
                     \gloss{‘release’}  &     &   
                     \vernacular{
                    alimuu[seébúlila] tá}  &   
                     \gloss{‘say bye to’}  &  \\

                     \vernacular{
                    alimuu[mboómbélitsila] tá}  &   
                     \gloss{‘comfort’}  &     &   
                     \vernacular{
                    alimuu[siínjílitsila] tá}  &   
                     \gloss{
                    ‘make...stand’}  &  \\
\end{tabular}
%\caption{\nocaption}
     
\begin{tabular}{lll}  
  \multicolumn{2}{l}{
                     \vernacular{(265) /H/
                    C-Initial Phrase-Medial} \gloss{‘s/he will
                    not...the man \ob musáatsa\cb  /} } &  \\
\multicolumn{2}{l}{
                     \gloss{the boy
                    \ob mú{\downstep}yáyi\cb  / someone \ob muundu\cb ’} } &  \\

                     \vernacular{ali[ra]
                    musáatsa/mú{\downstep}yáyi/muundu}  &   
                     \gloss{‘bury’}  &  \\

                     \vernacular{ali[beka]
                    musáatsa/mú{\downstep}yáyi/muundu tá}  &   
                     \gloss{‘shave’}  &  \\

                     \vernacular{ali[leera]
                    musáatsa/mú{\downstep}yáyi/muundu tá}  &   
                     \gloss{‘bring’}  &  \\

                     \vernacular{ali[khalaka]
                    musáatsa/mú{\downstep}yáyi/muundu tá}  &   
                     \gloss{‘cut’}  &  \\

                     \vernacular{ali[sitaaka]
                    musáatsa/mú{\downstep}yáyi/muundu tá}  &   
                     \gloss{‘accuse’}  &  \\

                     \vernacular{ali[boolitsa]
                    musáatsa/mú{\downstep}yáyi/muundu tá}  &   
                     \gloss{‘seduce’}  &  \\

                     \vernacular{ali[khong’oonda]
                    musáatsa/mú{\downstep}yáyi/muundu tá}  &   
                     \gloss{‘knock’}  &  \\

                     \vernacular{ali[boholola]
                    musáatsa/mú{\downstep}yáyi/muundu tá}  &   
                     \gloss{‘untie’}  &  \\

                     \vernacular{ali[boyong’ana]
                    musáatsa/mú{\downstep}yáyi/muundu tá}  &   
                     \gloss{‘go around’}  &  \\

                     \vernacular{ali[lingakanyinya]
                    musáatsa/mú{\downstep}yáyi/muundu tá}  &   
                     \gloss{‘bend’}  &  \\
\end{tabular}
%\caption{\nocaption}
     
\begin{tabular}{lll}  
  \multicolumn{2}{l}{
                     \vernacular{(266) /Ø/
                    C-Initial Phrase-Medial} \gloss{‘s/he will
                    not...the man \ob musáatsa\cb  /} } &  \\
\multicolumn{2}{l}{
                     \gloss{the boy
                    \ob mú{\downstep}yáyi\cb  / someone \ob muundu\cb ’} } &  \\

                     \vernacular{ali[tsia]
                    musáatsa/mú{\downstep}yáyi/muundu tá}  &   
                     \gloss{‘go for’}  &  \\

                     \vernacular{ali[lekha]
                    musáatsa/mú{\downstep}yáyi/muundu tá}  &   
                     \gloss{‘leave’}  &  \\

                     \vernacular{ali[loonda]
                    musáatsa/mú{\downstep}yáyi/muundu tá}  &   
                     \gloss{‘follow’}  &  \\

                     \vernacular{ali[kulikha]
                    musáatsa/mú{\downstep}yáyi/muundu tá}  &   
                     \gloss{‘name’}  &  \\

                     \vernacular{ali[lakhuula]
                    musáatsa/mú{\downstep}yáyi/muundu tá}  &   
                     \gloss{‘release’}  &  \\

                     \vernacular{ali[seebula]
                    musáatsa/mú{\downstep}yáyi/muundu tá}  &   
                     \gloss{‘say bye to’}  &  \\

                     \vernacular{ali[kalushitsa]
                    musáatsa/mú{\downstep}yáyi/muundu tá}  &   
                     \gloss{‘return’}  &  \\

                     \vernacular{ali[reebareeba]
                    musáatsa/mú{\downstep}yáyi/muundu tá}  &   
                     \gloss{‘ask (iter)’}  &  \\

                     \vernacular{ali[kalukhanyinya]
                    musáatsa/mú{\downstep}yáyi/muundu tá}  &   
                     \gloss{
                    ‘turn...over’}  &  \\
\end{tabular}
%\caption{\nocaption}
     
\begin{tabular}{lll}  
  \multicolumn{2}{l}{
                     \vernacular{(267) /H/
                    C-Initial +OP Phrase-Medial} \gloss{‘s/he will
                    not...the man \ob musáatsa\cb  /} } &  \\
\multicolumn{2}{l}{
                     \gloss{the boy
                    \ob mú{\downstep}yáyi\cb  / someone \ob muundu\cb  for
                    him/her’} } &  \\

                     \vernacular{alimu[réela]
                    musáatsa/mú{\downstep}yáyi/muundu tá}  &   
                     \gloss{‘bury’}  &  \\

                     \vernacular{alimu[béchela]
                    musáatsa/mú{\downstep}yáyi/muundu tá}  &   
                     \gloss{‘shave’}  &  \\

                     \vernacular{alimu[léerela]
                    musáatsa/mú{\downstep}yáyi/muundu tá}  &   
                     \gloss{‘bring’}  &  \\

                     \vernacular{alimu[khálachila]
                    musáatsa/mú{\downstep}yáyi/muundu tá}  &   
                     \gloss{‘cut’}  &  \\

                     \vernacular{alimu[sítaachila]
                    musáatsa/mú{\downstep}yáyi/muundu tá}  &   
                     \gloss{‘accuse’}  &  \\

                     \vernacular{alimu[bóolitsila]
                    musáatsa/mú{\downstep}yáyi/muundu tá}  &   
                     \gloss{‘seduce’}  &  \\

                     \vernacular{
                    alimu[tsúunzuunila] musáatsa/mú{\downstep}yáyi/muundu
                    tá}  &   
                     \gloss{‘suck’}  &  \\

                     \vernacular{alimu[bóhololela]
                    musáatsa/mú{\downstep}yáyi/muundu tá}  &   
                     \gloss{‘untie’}  &  \\

                     \vernacular{
                    alimu[bóyong’anila] musáatsa/mú{\downstep}yáyi/muundu
                    tá}  &   
                     \gloss{‘go around’}  &  \\

                     \vernacular{
                    alimu[língakanyinyila]
                    musáatsa/mú{\downstep}yáyi/muundu tá}  &   
                     \gloss{‘bend’}  &  \\
\end{tabular}
%\caption{\nocaption}
     
\begin{tabular}{lll}  
  \multicolumn{2}{l}{
                     \vernacular{(268) /Ø/
                    C-Initial +OP Phrase-Medial} \gloss{‘s/he will
                    not...the man \ob musáatsa\cb  /} } &  \\
\multicolumn{2}{l}{
                     \gloss{the boy
                    \ob mú{\downstep}yáyi\cb  / someone \ob muundu\cb  for
                    him/her’} } &  \\

                     \vernacular{alimu[siela]
                    musáatsa/mú{\downstep}yáyi/muundu tá}  &   
                     \gloss{‘grind’}  &  \\

                     \vernacular{alimu[leshela]
                    musáatsa/mú{\downstep}yáyi/muundu tá}  &   
                     \gloss{‘leave’}  &  \\

                     \vernacular{alimu[loondela]
                    musáatsa/mú{\downstep}yáyi/muundu tá}  &   
                     \gloss{‘follow’}  &  \\

                     \vernacular{alimu[kulishila]
                    musáatsa/mú{\downstep}yáyi/muundu tá}  &   
                     \gloss{‘name’}  &  \\

                     \vernacular{alimu[lakhuulila]
                    musáatsa/mú{\downstep}yáyi/muundu tá}  &   
                     \gloss{‘release’}  &  \\

                     \vernacular{alimu[seebulila]
                    musáatsa/mú{\downstep}yáyi/muundu tá}  &   
                     \gloss{‘say bye to’}  &  \\

                     \vernacular{
                    alimu[kalushitsila] musáatsa/mú{\downstep}yáyi/muundu
                    tá}  &   
                     \gloss{‘return’}  &  \\

                     \vernacular{
                    alimu[reebareebela] musáatsa/mú{\downstep}yáyi/muundu
                    tá}  &   
                     \gloss{‘ask (iter)’}  &  \\

                     \vernacular{
                    alimu[kalukhanyinyila]
                    musáatsa/mú{\downstep}yáyi/muundu tá}  &   
                     \gloss{
                    ‘turn...over’}  &  \\
\end{tabular}
%\caption{\nocaption}
     
\begin{tabular}{lll}  
  \multicolumn{2}{l}{
                     \vernacular{(269) /H/
                    C-Initial +OP + OP
                    } \gloss{‘s/he will
                    not...the man \ob musáatsa\cb  /} } &  \\
\multicolumn{2}{l}{
                     \gloss{the boy
                    \ob mú{\downstep}yáyi\cb  / someone \ob muundu\cb  for him/her for
                    me’} } &  \\

                     \vernacular{alimuu[ndéela]
                    musáatsa/mú{\downstep}yáyi/muundu tá}  &   
                     \gloss{‘bury’}  &  \\

                     \vernacular{alimuu[mbéchela]
                    musáatsa/mú{\downstep}yáyi/muundu tá}  &   
                     \gloss{‘shave’}  &  \\

                     \vernacular{alimuu[ndéerela]
                    musáatsa/mú{\downstep}yáyi/muundu tá}  &   
                     \gloss{‘bring’}  &  \\

                     \vernacular{
                    alimuu[khálachila] musáatsa/mú{\downstep}yáyi/muundu
                    tá}  &   
                     \gloss{‘cut’}  &  \\

                     \vernacular{
                    alimuu[sítaachila] musáatsa/mú{\downstep}yáyi/muundu
                    tá}  &   
                     \gloss{‘accuse’}  &  \\

                     \vernacular{
                    alimuu[mbóolitsila] musáatsa/mú{\downstep}yáyi/muundu
                    tá}  &   
                     \gloss{‘seduce’}  &  \\

                     \vernacular{
                    alimuu[mbóhololela] musáatsa/mú{\downstep}yáyi/muundu
                    tá}  &   
                     \gloss{‘untie’}  &  \\
\end{tabular}
%\caption{\nocaption}
     
\begin{tabular}{lll}  
  \multicolumn{2}{l}{
                     \vernacular{(270) /Ø/
                    C-Initial +OP + OP
                    } \gloss{‘s/he will
                    not...the man \ob musáatsa\cb  /} } &  \\
\multicolumn{2}{l}{
                     \gloss{the boy
                    \ob mú{\downstep}yáyi\cb  / someone \ob muundu\cb  for him/her for
                    me’} } &  \\

                     \vernacular{alimuu[nziila]
                    musáatsa/mú{\downstep}yáyi/muundu tá}  &   
                     \gloss{‘go for’}  &  \\

                     \vernacular{alimuu[ndeshela]
                    musáatsa/mú{\downstep}yáyi/muundu tá}  &   
                     \gloss{‘leave’}  &  \\

                     \vernacular{alimuu[noondela]
                    musáatsa/mú{\downstep}yáyi/muundu tá}  &   
                     \gloss{‘follow’}  &  \\

                     \vernacular{alimuu[ngulishila]
                    musáatsa/mú{\downstep}yáyi/muundu tá}  &   
                     \gloss{‘name’}  &  \\

                     \vernacular{
                    alimuu[ndakhuulila] musáatsa/mú{\downstep}yáyi/muundu
                    tá}  &   
                     \gloss{‘release’}  &  \\

                     \vernacular{alimuu[seebulila]
                    musáatsa/mú{\downstep}yáyi/muundu tá}  &   
                     \gloss{‘say bye to’}  &  \\
\end{tabular}
%\caption{\nocaption}
    

\subsection{Imperative
              }\label{sec:sImpSg}


\begin{tabular}{llllll}  
  \multicolumn{5}{l}{
                     \vernacular{(271) /H/
                    C-Initial} \gloss{‘...!’} } &  \\
\multicolumn{5}{l}{ } &  \\

                     \vernacular{[ra]}  &   
                     \gloss{‘bury’}  &     &   
                     \vernacular{[ng’wa]}  &   
                     \gloss{‘drink’}  &  \\

                     \vernacular{[khwa]}  &   
                     \gloss{‘eat’}  &     &   
                     \vernacular{[lia]}  &   
                     \gloss{‘pay dowry’}  &  \\

                     \vernacular{[luma]}  &   
                     \gloss{‘bite’}  &     &   
                     \vernacular{[beka]}  &   
                     \gloss{‘shave’}  &  \\

                     \vernacular{
                    [teekhá]}  &   
                     \gloss{‘cook’}  &     &   
                     \vernacular{
                    [leerá]}  &   
                     \gloss{‘bring’}  &  \\

                     \vernacular{
                    [khalaká]}  &   
                     \gloss{‘cut’}  &     &   
                     \vernacular{
                    [kalaangá]}  &   
                     \gloss{‘fry’}  &  \\

                     \vernacular{
                    [sitaaká]}  &   
                     \gloss{‘accuse’}  &     &   
                     \vernacular{
                    [boolitsá]}  &   
                     \gloss{‘seduce’}  &  \\

                     \vernacular{
                    [saanditsá]}  &   
                     \gloss{‘thank’}  &     &   
                     \vernacular{
                    [khong’oondá]}  &   
                     \gloss{‘knock’}  &  \\

                     \vernacular{
                    [bohololá]}  &   
                     \gloss{‘untie’}  &     &   
                     \vernacular{
                    [boyong’aná]}  &   
                     \gloss{‘go around’}  &  \\

                     \vernacular{
                    [ng’ong’oolitsá]}  &   
                     \gloss{‘tease’}  &     &   
                     \vernacular{
                    [lingakanyinyá]}  &   
                     \gloss{‘crumple’}  &  \\
\end{tabular}
%\caption{\nocaption}
     
\begin{tabular}{llllll}  
  \multicolumn{5}{l}{
                     \vernacular{(272) /H/
                    V-Initial} \gloss{‘...!’} } &  \\
\multicolumn{5}{l}{ } &  \\

                     \vernacular{[yira]}  &   
                     \gloss{‘kill’}  &     &   
                     \vernacular{
                    [yikoó{\downstep}mbɛ́]}  &   
                     \gloss{‘admire’}  &  \\

                     \vernacular{
                    [yisiá{\downstep}chɛ́]}  &   
                     \gloss{‘smack’}  &     &   
                     \vernacular{
                    [yikobó{\downstep}lɛ́]}  &   
                     \gloss{‘belch’}  &  \\

                     \vernacular{
                    [yononyinyá]}  &   
                     \gloss{‘spoil’}  &     &   
                     \vernacular{
                    [yabukhanyinyá]}  &   
                     \gloss{‘separate’}  &  \\
\end{tabular}
%\caption{\nocaption}
     
\begin{tabular}{llllll}  
  \multicolumn{5}{l}{
                     \vernacular{(273) /Ø/
                    C-Initial} \gloss{‘...!’} } &  \\
\multicolumn{5}{l}{ } &  \\

                     \vernacular{[tsía]}  &   
                     \gloss{‘go’}  &     &   
                     \vernacular{[kwá]}  &   
                     \gloss{‘fall’}  &  \\

                     \vernacular{
                    [lé{\downstep}khá]}  &   
                     \gloss{‘leave’}  &     &   
                     \vernacular{
                    [réé{\downstep}bá]}  &   
                     \gloss{‘ask’}  &  \\

                     \vernacular{
                    [lóó{\downstep}ndá]}  &   
                     \gloss{‘follow’}  &     &   
                     \vernacular{
                    [kúmí{\downstep}lá]}  &   
                     \gloss{‘hold’}  &  \\

                     \vernacular{
                    [kúlí{\downstep}khá]}  &   
                     \gloss{‘name’}  &     &   
                     \vernacular{
                    [hómóó{\downstep}lá]}  &   
                     \gloss{‘massage’}  &  \\

                     \vernacular{
                    [lákhúú{\downstep}lá]}  &   
                     \gloss{‘release’}  &     &   
                     \vernacular{
                    [séébú{\downstep}lá]}  &   
                     \gloss{‘say bye’}  &  \\

                     \vernacular{
                    [hóómbélí{\downstep}tsá]}  &   
                     \gloss{‘comfort’}  &     &   
                     \vernacular{
                    [kálúshí{\downstep}tsá]}  &   
                     \gloss{‘return’}  &  \\

                     \vernacular{
                    [síínjílí{\downstep}tsá]}  &   
                     \gloss{‘make stand’}  &     &   
                     \vernacular{
                    [réébáréé{\downstep}bá]}  &   
                     \gloss{‘ask (iter)’}  &  \\

                     \vernacular{
                    [kálúkhányí{\downstep}nyá]}  &   
                     \gloss{‘turn over’}  &     &   
                     \vernacular{
                    [sébúlúkhányí{\downstep}nyá]}  &   
                     \gloss{‘scatter’}  &  \\
\end{tabular}
%\caption{\nocaption}
     
\begin{tabular}{llllll}  
  \multicolumn{5}{l}{
                     \vernacular{(274) /Ø/
                    V-Initial} \gloss{‘...!’} } &  \\
\multicolumn{5}{l}{ } &  \\

                     \vernacular{
                    [yé{\downstep}nyá]}  &   
                     \gloss{‘want’}  &     &   
                     \vernacular{
                    [yéyé{\downstep}lá]}  &   
                     \gloss{‘wipe for’}  &  \\

                     \vernacular{
                    [yílúú{\downstep}lá]}  &   
                     \gloss{‘winnow’}  &     &   
                     \vernacular{
                    [yámbákhá{\downstep}ná]}  &   
                     \gloss{‘refuse’}  &  \\

                     \vernacular{
                    [yéléélí{\downstep}tsá]}  &   
                     \gloss{‘hang up’}  &     &     &     &  \\
\end{tabular}
%\caption{\nocaption}
     
\begin{tabular}{llllll}  
  \multicolumn{5}{l}{
                     \vernacular{(275) /H/
                    C-Initial + OP} \gloss{
                    ‘...him/her!’} } &  \\
\multicolumn{5}{l}{ } &  \\

                     \vernacular{mu[rɛ́]}  &   
                     \gloss{‘bury’}  &     &   
                     \vernacular{
                    mu[bé{\downstep}chɛ́]}  &   
                     \gloss{‘shave’}  &  \\

                     \vernacular{
                    mu[leé{\downstep}rɛ́]}  &   
                     \gloss{‘bring’}  &     &   
                     \vernacular{
                    mu[khá{\downstep}láchɛ́]}  &   
                     \gloss{‘cut’}  &  \\

                     \vernacular{
                    mu[sí{\downstep}tááchɛ́]}  &   
                     \gloss{‘accuse’}  &     &   
                     \vernacular{
                    mu[boó{\downstep}lítsɪ́]}  &   
                     \gloss{‘seduce’}  &  \\

                     \vernacular{
                    mu[khó{\downstep}ng’óóndɛ́]}  &   
                     \gloss{‘knock’}  &     &   
                     \vernacular{
                    mu[bó{\downstep}hólólɛ́]}  &   
                     \gloss{‘untie’}  &  \\

                     \vernacular{
                    mu[bó{\downstep}yóng’ánɛ́]}  &   
                     \gloss{‘go around’}  &     &   
                     \vernacular{
                    mu[ng’ó{\downstep}ng’óólítsɪ́]}  &   
                     \gloss{‘tease’}  &  \\

                     \vernacular{
                    mu[lí{\downstep}ngákányínyɪ́]}  &   
                     \gloss{‘bend’}  &     &     &     &  \\
\end{tabular}
%\caption{\nocaption}
     
\begin{tabular}{llllll}  
  \multicolumn{5}{l}{
                     \vernacular{(276) /H/
                    V-Initial + OP} \gloss{
                    ‘...him/her!’} } &  \\
\multicolumn{5}{l}{ } &  \\

                     \vernacular{
                    mw[ií{\downstep}rɪ́]}  &   
                     \gloss{‘kill’}  &     &   
                     \vernacular{
                    mw[ií{\downstep}kóó{\downstep}mbɛ́]}  &   
                     \gloss{‘admire’}  &  \\

                     \vernacular{
                    mw[ií{\downstep}síá{\downstep}chɛ́]}  &   
                     \gloss{‘smack’}  &     &   
                     \vernacular{
                    mw[oó{\downstep}nónyínyɪ́]}  &   
                     \gloss{‘spoil’}  &  \\

                     \vernacular{
                    mw[aá{\downstep}búkhányínyɪ́]}  &   
                     \gloss{‘separate’}  &  \\
\end{tabular}
%\caption{\nocaption}
     
\begin{tabular}{llllll}  
  \multicolumn{5}{l}{
                     \vernacular{(277) /Ø/
                    C-Initial + OP} \gloss{‘...him/her \ob mu-\cb 
                    / them
                    } } &  \\
\multicolumn{5}{l}{ } &  \\

                     \vernacular{
                    mu[tsí]}  &   
                     \gloss{‘go for’}  &  \\

                     \vernacular{
                    mu[leshɛ́]}  &   
                     \gloss{‘leave’}  &  \\

                     \vernacular{
                    mu[loó{\downstep}ndɛ́]}  &   
                     \gloss{‘follow’}  &  \\

                     \vernacular{
                    mu[kulí{\downstep}shɪ́]}  &   
                     \gloss{‘name’}  &  \\

                     \vernacular{
                    mu[lá{\downstep}khúúlɪ́]}  &   
                     \gloss{‘release’}  &  \\

                     \vernacular{
                    mu[seé{\downstep}búlɪ́]}  &   
                     \gloss{‘say bye to’}  &  \\

                     \vernacular{
                    mu[hoómbé{\downstep}lítsɪ́]}  &   
                     \gloss{‘comfort’}  &  \\

                     \vernacular{
                    mu[kalú{\downstep}shítsɪ́]}  &   
                     \gloss{‘return’}  &  \\

                     \vernacular{
                    mu[siínjí{\downstep}lítsɪ́]}  &   
                     \gloss{
                    ‘make...stand’}  &  \\

                     \vernacular{
                    mu[reébɛ́{\downstep}réébɛ́]}  &   
                     \gloss{‘ask (iter)’}  &  \\

                     \vernacular{
                    mu[kalúkhá{\downstep}nyínyɪ́]}  &   
                     \gloss{
                    ‘turn...over’}  &  \\

                     \vernacular{
                    bi[sebúlú{\downstep}khányínyɪ́]}  &   
                     \gloss{‘scatter’}  &  \\
\end{tabular}
%\caption{\nocaption}
     
\begin{tabular}{llllll}  
  \multicolumn{5}{l}{
                     \vernacular{(278) /Ø/
                    V-Initial + OP} \gloss{‘...him/her \ob mw-\cb 
                    / it
                    } } &  \\
\multicolumn{5}{l}{ } &  \\

                     \vernacular{
                    mw[eé{\downstep}nyɛ́]}  &   
                     \gloss{‘want’}  &     &   
                     \vernacular{
                    mw[eé{\downstep}yɛ́lɛ́]}  &   
                     \gloss{‘wipe for’}  &  \\

                     \vernacular{
                    bw[ií{\downstep}lúúlɪ́]}  &   
                     \gloss{‘winnow’}  &     &   
                     \vernacular{
                    mw[aá{\downstep}mbákhánɛ́]}  &   
                     \gloss{‘refuse’}  &  \\

                     \vernacular{
                    mw[eé{\downstep}léélítsɪ́]}  &   
                     \gloss{
                    ‘carry...hanging’}  &  \\
\end{tabular}
%\caption{\nocaption}
     
\begin{tabular}{llllll}  
  \multicolumn{5}{l}{
                     \vernacular{(279) /H/
                    C-Initial + OP
                    } \gloss{
                    ‘...me!’} } &  \\
\multicolumn{5}{l}{ } &  \\

                     \vernacular{u[rí]}  &   
                     \gloss{‘fear’}  &     &   
                     \vernacular{
                    [mbé{\downstep}ká]}  &   
                     \gloss{‘shave’}  &  \\

                     \vernacular{
                    ([ndeé{\downstep}rá]}  &   
                     \gloss{‘bring’)
                    }  &     &   
                     \vernacular{
                    [khá{\downstep}láká]}  &   
                     \gloss{‘cut’}  &  \\

                     \vernacular{
                    [sí{\downstep}tááká]}  &   
                     \gloss{‘accuse’}  &     &   
                     \vernacular{
                    [mboó{\downstep}lítsá]}  &   
                     \gloss{‘seduce’}  &  \\

                     \vernacular{
                    [khó{\downstep}ng’óóndá]}  &   
                     \gloss{‘knock’}  &     &   
                     \vernacular{
                    [mbó{\downstep}hólólá]}  &   
                     \gloss{‘untie’}  &  \\

                     \vernacular{
                    [bó{\downstep}yóng’áná]}  &   
                     \gloss{‘go around’}  &     &   
                     \vernacular{
                    [ng’ó{\downstep}ng’óólítsá]}  &   
                     \gloss{‘tease’}  &  \\

                     \vernacular{
                    [ní{\downstep}ngákányínyá]}  &   
                     \gloss{‘bend’}  &  \\
\end{tabular}
%\caption{\nocaption}
     
\begin{tabular}{llllll}  
  \multicolumn{5}{l}{
                     \vernacular{(280) /H/
                    V-Initial + OP
                    } \gloss{
                    ‘...me!’} } &  \\
\multicolumn{5}{l}{ } &  \\

                     \vernacular{
                    [nzí{\downstep}rá]}  &   
                     \gloss{‘kill’}  &     &   
                     \vernacular{
                    [nzí{\downstep}kóómbá]}  &   
                     \gloss{‘admire’}  &  \\

                     \vernacular{
                    [nzí{\downstep}síáchɛ́]}  &   
                     \gloss{‘smack’}  &     &   
                     \vernacular{
                    [nzó{\downstep}nónyínyá]}  &   
                     \gloss{‘spoil’}  &  \\

                     \vernacular{
                    [nzá{\downstep}búkhányínyá]}  &   
                     \gloss{‘separate’}  &  \\
\end{tabular}
%\caption{\nocaption}
     
\begin{tabular}{llllll}  
  \multicolumn{5}{l}{
                     \vernacular{(281) /Ø/
                    C-Initial + OP
                    } \gloss{
                    ‘...me!’} } &  \\
\multicolumn{5}{l}{ } &  \\

                     \vernacular{([sía]}  &   
                     \gloss{‘grind’)}  &     &   
                     \vernacular{
                    [ndé{\downstep}khá]}  &   
                     \gloss{‘leave’}  &  \\

                     \vernacular{
                    [noó{\downstep}ndá]}  &   
                     \gloss{‘follow’}  &     &   
                     \vernacular{
                    [ngú{\downstep}líkhá]}  &   
                     \gloss{‘name’}  &  \\

                     \vernacular{
                    [ndá{\downstep}khúúlá]}  &   
                     \gloss{‘release’}  &     &   
                     \vernacular{
                    [seé{\downstep}búlá]}  &   
                     \gloss{‘say bye to’}  &  \\

                     \vernacular{
                    [mboómbé{\downstep}lítsá]}  &   
                     \gloss{‘comfort’}  &     &   
                     \vernacular{
                    [siínjí{\downstep}lítsá]}  &   
                     \gloss{
                    ‘make..stand’}  &  \\

                     \vernacular{
                    [ndeébá{\downstep}ndéébá]}  &   
                     \gloss{‘ask (iter)’}  &     &   
                     \vernacular{
                    [ngalúkhá{\downstep}nyínyá]}  &   
                     \gloss{
                    ‘turn...over’}  &  \\
\end{tabular}
%\caption{\nocaption}
     
\begin{tabular}{llllll}  
  \multicolumn{5}{l}{
                     \vernacular{(282) /Ø/
                    V-Initial + OP
                    } \gloss{
                    ‘...me!’} } &  \\
\multicolumn{5}{l}{ } &  \\

                     \vernacular{
                    [nzé{\downstep}nyá]}  &   
                     \gloss{‘want’}  &     &   
                     \vernacular{
                    [nzé{\downstep}yélá]}  &   
                     \gloss{‘wipe for’}  &  \\

                     \vernacular{
                    [nyá{\downstep}mbákháná]}  &   
                     \gloss{‘refuse’}  &     &   
                     \vernacular{
                    [nzé{\downstep}léélítsá]}  &   
                     \gloss{
                    ‘carry...hanging’}  &  \\
\end{tabular}
%\caption{\nocaption}
     
\begin{tabular}{llllll}  
  \multicolumn{5}{l}{
                     \vernacular{(283) /H/
                    C-Initial + OP
                    } \gloss{
                    ‘...yourself!’} } &  \\
\multicolumn{5}{l}{ } &  \\

                     \vernacular{yi[rɛ́]}  &   
                     \gloss{‘bury’}  &     &   
                     \vernacular{
                    yi[bé{\downstep}chɛ́]}  &   
                     \gloss{‘shave’}  &  \\

                     \vernacular{
                    yi[suú{\downstep}njɪ́]}  &   
                     \gloss{‘hang’}  &     &   
                     \vernacular{
                    yi[khá{\downstep}láchɛ́]}  &   
                     \gloss{‘cut’}  &  \\

                     \vernacular{
                    yi[sí{\downstep}tááchɛ́]}  &   
                     \gloss{‘accuse’}  &     &   
                     \vernacular{
                    yi[saá{\downstep}ndítsɪ́]}  &   
                     \gloss{‘thank’}  &  \\

                     \vernacular{
                    yi[khó{\downstep}ng’óóndɛ́]}  &   
                     \gloss{‘knock’}  &     &   
                     \vernacular{
                    yi[bó{\downstep}hólólɛ́]}  &   
                     \gloss{‘untie’}  &  \\

                     \vernacular{
                    yi[khú{\downstep}píkhúpílɪ́]}  &   
                     \gloss{‘knock’}  &  \\
\end{tabular}
%\caption{\nocaption}
     
\begin{tabular}{llllll}  
  \multicolumn{5}{l}{
                     \vernacular{(284) /H/
                    V-Initial + OP
                    } \gloss{
                    ‘...yourself!’} } &  \\
\multicolumn{5}{l}{ } &  \\

                     \vernacular{
                    yi[yí{\downstep}rɪ́]}  &   
                     \gloss{‘kill’}  &     &   
                     \vernacular{
                    yi[yí{\downstep}kóómbɛ́]}  &   
                     \gloss{‘admire’}  &  \\

                     \vernacular{
                    yi[yí{\downstep}síáchɛ́]}  &   
                     \gloss{‘smack’}  &     &   
                     \vernacular{
                    yi[yó{\downstep}nónyínyɪ́]}  &   
                     \gloss{‘spoil’}  &  \\

                     \vernacular{
                    yi[yá{\downstep}búkhányínyɪ́]}  &   
                     \gloss{‘separate’}  &  \\
\end{tabular}
%\caption{\nocaption}
     
\begin{tabular}{llllll}  
  \multicolumn{5}{l}{
                     \vernacular{(285) /Ø/
                    C-Initial + OP
                    } \gloss{
                    ‘...yourself!’} } &  \\
\multicolumn{5}{l}{ } &  \\

                     \vernacular{yi[sí]}  &   
                     \gloss{‘grind’}  &     &   
                     \vernacular{
                    yi[leshɛ́]}  &   
                     \gloss{‘leave’}  &  \\

                     \vernacular{
                    yi[sií{\downstep}njɪ́]}  &   
                     \gloss{‘bathe’}  &     &   
                     \vernacular{
                    yi[kulí{\downstep}shɪ́]}  &   
                     \gloss{‘name’}  &  \\

                     \vernacular{
                    yi[naá{\downstep}búlɪ́]}  &   
                     \gloss{‘undress’}  &     &   
                     \vernacular{
                    yi[lakhú{\downstep}úlɪ́]}  &   
                     \gloss{‘release’}  &  \\

                     \vernacular{
                    yi[hoómbé{\downstep}lítsɪ́]}  &   
                     \gloss{‘comfort’}  &     &   
                     \vernacular{
                    yi[siínjí{\downstep}lítsɪ́]}  &   
                     \gloss{
                    ‘make...stand’}  &  \\

                     \vernacular{
                    yi[reébɛ́{\downstep}réébɛ́]}  &   
                     \gloss{‘ask (iter)’}  &     &   
                     \vernacular{
                    yi[kalúkhá{\downstep}nyínyɪ́]}  &   
                     \gloss{
                    ‘turn...over’}  &  \\
\end{tabular}
%\caption{\nocaption}
     
\begin{tabular}{llllll}  
  \multicolumn{5}{l}{
                     \vernacular{(286) /Ø/
                    V-Initial + OP
                    } \gloss{
                    ‘...yourself!’} } &  \\
\multicolumn{5}{l}{ } &  \\

                     \vernacular{
                    yi[yá{\downstep}lɛ́]}  &   
                     \gloss{‘spread’}  &     &   
                     \vernacular{
                    yi[yé{\downstep}yɛ́lɛ́]}  &   
                     \gloss{‘wipe for’}  &  \\

                     \vernacular{
                    yi[yá{\downstep}mbákhánɛ́]}  &   
                     \gloss{‘refuse’}  &     &   
                     \vernacular{
                    yi[yé{\downstep}léélítsɪ́]}  &   
                     \gloss{‘hang...up’}  &  \\
\end{tabular}
%\caption{\nocaption}
     
\begin{tabular}{llllll}  
  \multicolumn{5}{l}{
                     \vernacular{(287) /H/
                    C-Initial + OP + OP
                    } \gloss{‘...him/her for
                    me!’} } &  \\
\multicolumn{5}{l}{ } &  \\

                     \vernacular{
                    muú[{\downstep}ndéélɛ́]}  &   
                     \gloss{‘bury’}  &     &   
                     \vernacular{
                    muú[{\downstep}mbéchélɛ́]}  &   
                     \gloss{‘shave’}  &  \\

                     \vernacular{
                    muú[{\downstep}ndéérélɛ́]}  &   
                     \gloss{‘bring’}  &     &   
                     \vernacular{
                    muú[{\downstep}kháláchílɪ́]}  &   
                     \gloss{‘cut’}  &  \\

                     \vernacular{
                    muú[{\downstep}sítááchílɪ́]}  &   
                     \gloss{‘accuse’}  &     &   
                     \vernacular{
                    muú[{\downstep}mbóólítsílɪ́]}  &   
                     \gloss{‘seduce’}  &  \\

                     \vernacular{
                    muú[{\downstep}mbóhólólélɛ́]}  &   
                     \gloss{‘untie’}  &     &     &     &  \\
\end{tabular}
%\caption{\nocaption}
     
\begin{tabular}{llllll}  
  \multicolumn{5}{l}{
                     \vernacular{(288) /H/
                    V-Initial + OP + OP
                    } \gloss{‘...him/her for
                    me!’} } &  \\
\multicolumn{5}{l}{ } &  \\

                     \vernacular{
                    muú[{\downstep}nzírílɪ́]}  &   
                     \gloss{‘kill’}  &     &   
                     \vernacular{
                    muú[{\downstep}nzéchítsílɪ́]}  &   
                     \gloss{‘admire’}  &  \\

                     \vernacular{
                    muú[{\downstep}nzísíáchílɪ́]}  &   
                     \gloss{‘smack’}  &     &   
                     \vernacular{
                    muú[{\downstep}nzónónyínyílɪ́]}  &   
                     \gloss{‘spoil’}  &  \\

                     \vernacular{
                    muú[{\downstep}nzábúkhányínyílɪ́]}  &   
                     \gloss{‘separate’}  &     &     &     &  \\
\end{tabular}
%\caption{\nocaption}
     
\begin{tabular}{llllll}  
  \multicolumn{5}{l}{
                     \vernacular{(289) /Ø/
                    C-Initial + OP + OP
                    } \gloss{‘...him/her for
                    me!’} } &  \\
\multicolumn{5}{l}{ } &  \\

                     \vernacular{
                    muú[{\downstep}nzíí{\downstep}lɪ́]}  &   
                     \gloss{‘go for’}  &     &   
                     \vernacular{
                    muú[{\downstep}ndéshé{\downstep}lɛ́]}  &   
                     \gloss{‘leave’}  &  \\

                     \vernacular{
                    muú[{\downstep}nóó{\downstep}ndélɛ́]}  &   
                     \gloss{‘follow’}  &     &   
                     \vernacular{
                    muú[{\downstep}ngúlí{\downstep}shílɪ́]}  &   
                     \gloss{‘name’}  &  \\

                     \vernacular{
                    muú[{\downstep}ndákhú{\downstep}úlílɪ́]}  &   
                     \gloss{‘release’}  &     &   
                     \vernacular{
                    muú[{\downstep}séébú{\downstep}lílɪ́]}  &   
                     \gloss{‘say bye to’}  &  \\

                     \vernacular{
                    muú[{\downstep}mbóómbé{\downstep}lítsílɪ́]}  &   
                     \gloss{‘comfort’}  &     &   
                     \vernacular{
                    muú[{\downstep}síínjí{\downstep}lítsílɪ́]}  &   
                     \gloss{
                    ‘make...stand’}  &  \\
\end{tabular}
%\caption{\nocaption}
     
\begin{tabular}{llllll}  
  \multicolumn{5}{l}{
                     \vernacular{(290) /Ø/
                    V-Initial + OP + OP
                    } \gloss{‘...him/her \ob mu-\cb 
                    / it
                    } } &  \\
\multicolumn{5}{l}{ } &  \\

                     \vernacular{
                    muú[{\downstep}nzéyé{\downstep}lɛ́]}  &   
                     \gloss{‘wipe’}  &     &   
                     \vernacular{
                    kuú[{\downstep}nzáshí{\downstep}tsílɪ́]}  &   
                     \gloss{‘light’}  &  \\

                     \vernacular{
                    buú[{\downstep}nzílú{\downstep}úlílɪ́]}  &   
                     \gloss{‘winnow’}  &     &   
                     \vernacular{
                    luú[{\downstep}nzítsú{\downstep}lítsílɪ́]}  &   
                     \gloss{‘fill’}  &  \\

                     \vernacular{
                    kuú[{\downstep}nzélé{\downstep}élítsílɪ́]}  &   
                     \gloss{‘hang’}  &     &     &     &  \\
\end{tabular}
%\caption{\nocaption}
     
\begin{tabular}{lll}  
  \multicolumn{2}{l}{
                     \vernacular{(291) /H/
                    C-Initial Phrase-Medial} \gloss{‘...the man
                    \ob musáatsa\cb  /} } &  \\
\multicolumn{2}{l}{
                     \gloss{someone
                    \ob muundu\cb !’} } &  \\

                     \vernacular{[ra]
                    musáatsa/muundu}  &   
                     \gloss{‘bury’}  &  \\

                     \vernacular{[beka]
                    musáatsa/muundu}  &   
                     \gloss{‘shave’}  &  \\

                     \vernacular{[leera]
                    musáatsa/muundu}  &   
                     \gloss{‘bring’}  &  \\

                     \vernacular{[khalaka]
                    musáatsa/muundu}  &   
                     \gloss{‘cut’}  &  \\

                     \vernacular{[sitaaka]
                    musáatsa/muundu}  &   
                     \gloss{‘accuse’}  &  \\

                     \vernacular{[boolitsa]
                    musáatsa/muundu}  &   
                     \gloss{‘seduce’}  &  \\

                     \vernacular{[khong’oonda]
                    musáatsa/muundu}  &   
                     \gloss{‘knock’}  &  \\

                     \vernacular{[boholola]
                    musáatsa/muundu}  &   
                     \gloss{‘untie’}  &  \\

                     \vernacular{[boyong’ana]
                    musáatsa/muundu}  &   
                     \gloss{‘go around’}  &  \\

                     \vernacular{[lingakanyinya]
                    musáatsa/muundu}  &   
                     \gloss{‘bend’}  &  \\
\end{tabular}
%\caption{\nocaption}
     
\begin{tabular}{lll}  
  \multicolumn{2}{l}{
                     \vernacular{(292) /Ø/
                    C-Initial Phrase-Medial} \gloss{‘...the man
                    \ob musáatsa\cb  /} } &  \\
\multicolumn{2}{l}{
                     \gloss{someone
                    \ob muundu\cb !’} } &  \\

                     \vernacular{[tsia]
                    musáatsa/muundu}  &   
                     \gloss{‘go for’}  &  \\

                     \vernacular{[lekha]
                    musáatsa/muundu}  &   
                     \gloss{‘leave’}  &  \\

                     \vernacular{[loonda]
                    musáatsa/muundu}  &   
                     \gloss{‘follow’}  &  \\

                     \vernacular{[kulikha]
                    musáatsa/muundu}  &   
                     \gloss{‘name’}  &  \\

                     \vernacular{[lakhuula]
                    musáatsa/muundu}  &   
                     \gloss{‘release’}  &  \\

                     \vernacular{[seebula]
                    musáatsa/muundu}  &   
                     \gloss{‘say bye to’}  &  \\

                     \vernacular{[kalushitsa]
                    musáatsa/muundu}  &   
                     \gloss{‘return’}  &  \\

                     \vernacular{[reebareeba]
                    musáatsa/muundu}  &   
                     \gloss{‘ask (iter)’}  &  \\

                     \vernacular{[kalukhanyinya]
                    musáatsa/muundu}  &   
                     \gloss{
                    ‘turn...over’}  &  \\
\end{tabular}
%\caption{\nocaption}
     
\begin{tabular}{lll}  
  \multicolumn{2}{l}{
                     \vernacular{(293) /H/
                    C-Initial +OP Phrase-Medial} \gloss{‘...the man
                    \ob musáatsa\cb  /} } &  \\
\multicolumn{2}{l}{
                     \gloss{someone \ob muundu\cb 
                    for him/her!’} } &  \\

                     \vernacular{mu[reé{\downstep}lélɛ́]
                    musáatsa/muundu}  &   
                     \gloss{‘bury’}  &  \\

                     \vernacular{mu[bé{\downstep}chélɛ́]
                    musáatsa/muundu}  &   
                     \gloss{‘shave’}  &  \\

                     \vernacular{mu[leé{\downstep}rélɛ́]
                    musáatsa/muundu}  &   
                     \gloss{‘bring’}  &  \\

                     \vernacular{
                    mu[khá{\downstep}láchílɪ́] musáatsa/muundu}  &   
                     \gloss{‘cut’}  &  \\

                     \vernacular{
                    mu[sí{\downstep}tááchílɪ́]
                    musáatsa/muundu}  &   
                     \gloss{‘accuse’}  &  \\

                     \vernacular{
                    mu[boó{\downstep}lítsílɪ́] musáatsa/muundu}  &   
                     \gloss{‘seduce’}  &  \\

                     \vernacular{
                    mu[khó{\downstep}ng’óóndélɛ́]
                    musáatsa/muundu}  &   
                     \gloss{‘knock’}  &  \\

                     \vernacular{
                    mu[bó{\downstep}hólólélɛ́]
                    musáatsa/muundu}  &   
                     \gloss{‘untie’}  &  \\

                     \vernacular{
                    mu[bó{\downstep}yóng’ánílɪ́]
                    musáatsa/muundu}  &   
                     \gloss{‘go around’}  &  \\
\end{tabular}
%\caption{\nocaption}
     
\begin{tabular}{lll}  
  \multicolumn{2}{l}{
                     \vernacular{(294) /Ø/
                    C-Initial +OP Phrase-Medial} \gloss{‘...the man
                    \ob musáatsa\cb  /} } &  \\
\multicolumn{2}{l}{
                     \gloss{someone \ob muundu\cb 
                    for him/her!’} \footnote{\label{fn:nNotImperatives} The first recording of this paradigm, with
                      musáatsa ‘the man’, were produced as
                      subjunctives rather than imperatives. 


}%
} &  \\

                     \vernacular{mu[tsií{\downstep}lɪ́]
                    musáatsa/muundu}  &   
                     \gloss{‘go for’}  &  \\

                     \vernacular{mu[leshé{\downstep}lɛ́]
                    musáatsa/muundu}  &   
                     \gloss{‘leave’}  &  \\

                     \vernacular{mu[loóndé{\downstep}lɛ́]
                    musáatsa/muundu}  &   
                     \gloss{‘follow’}  &  \\

                     \vernacular{mu[kulíshí{\downstep}lɪ́]
                    musáatsa/muundu}  &   
                     \gloss{‘name’}  &  \\

                     \vernacular{
                    mu[lakhú{\downstep}úlílɪ́] musáatsa/muundu}  &   
                     \gloss{‘release’}  &  \\

                     \vernacular{mu[seébú{\downstep}lílɪ́]
                    musáatsa/muundu}  &   
                     \gloss{‘say bye to’}  &  \\

                     \vernacular{
                    mu[kalúshí{\downstep}tsílɪ́]
                    musáatsa/muundu}  &   
                     \gloss{‘return’}  &  \\

                     \vernacular{
                    mu[reébɛ́{\downstep}réébélɛ́]
                    musáatsa/muundu}  &   
                     \gloss{‘ask (iter)’}  &  \\
\end{tabular}
%\caption{\nocaption}
     
\begin{tabular}{lll}  
  \multicolumn{2}{l}{
                     \vernacular{(295) /H/
                    C-Initial +OP + OP
                    } \gloss{‘...the man
                    \ob musáatsa\cb  /} } &  \\
\multicolumn{2}{l}{
                     \gloss{someone \ob muundu\cb 
                    for him/her for me!’} } &  \\

                     \vernacular{muú[{\downstep}ndéélɛ́]
                    musáatsa/muundu}  &   
                     \gloss{‘bury’}  &  \\

                     \vernacular{muú[{\downstep}mbéchélɛ́]
                    musáatsa/muundu}  &   
                     \gloss{‘shave’}  &  \\

                     \vernacular{
                    muú[{\downstep}ndéérélɛ́] musáatsa/muundu}  &   
                     \gloss{‘bring’}  &  \\

                     \vernacular{
                    muú[{\downstep}kháláchílɪ́]
                    musáatsa/muundu}  &   
                     \gloss{‘cut’}  &  \\

                     \vernacular{
                    muú[{\downstep}sítááchílɪ́]
                    musáatsa/muundu}  &   
                     \gloss{‘accuse’}  &  \\

                     \vernacular{
                    muú[{\downstep}mbóólítsílɪ́]
                    musáatsa/muundu}  &   
                     \gloss{‘seduce’}  &  \\
\end{tabular}
%\caption{\nocaption}
     
\begin{tabular}{lll}  
  \multicolumn{2}{l}{
                     \vernacular{(296) /Ø/
                    C-Initial +OP + OP
                    } \gloss{‘...the man
                    \ob musáatsa\cb  /} } &  \\
\multicolumn{2}{l}{
                     \gloss{someone \ob muundu\cb 
                    for him/her for me!’} } &  \\

                     \vernacular{muú[{\downstep}nzíí{\downstep}lɪ́]
                    musáatsa/muundu}  &   
                     \gloss{‘go for’}  &  \\

                     \vernacular{
                    muú[{\downstep}ndéshé{\downstep}lɛ́] musáatsa/muundu}  &   
                     \gloss{‘leave’}  &  \\

                     \vernacular{
                    muú[{\downstep}nóóndé{\downstep}lɛ́]
                    musáatsa/muundu}  &   
                     \gloss{‘follow’}  &  \\

                     \vernacular{
                    muú[{\downstep}ngúlíshí{\downstep}lɪ́]
                    musáatsa/muundu}  &   
                     \gloss{‘name’}  &  \\

                     \vernacular{
                    muú[{\downstep}ndákhú{\downstep}úlílɪ́]
                    musáatsa/muundu}  &   
                     \gloss{‘release’}  &  \\

                     \vernacular{
                    muú[{\downstep}séébú{\downstep}lílɪ́]
                    musáatsa/muundu}  &   
                     \gloss{‘say bye to’}  &  \\
\end{tabular}
%\caption{\nocaption}
    

\subsection{Imperative
              }\label{sec:sImpSgNeg}


\begin{tabular}{llllll}  
  \multicolumn{5}{l}{
                     \vernacular{(297) /H/
                    C-Initial} \gloss{‘do
                    not...!’} } &  \\
\multicolumn{5}{l}{ } &  \\

                     \vernacular{ukha[ra]
                    tá}  &   
                     \gloss{‘bury’}  &     &   
                     \vernacular{ukha[beka]
                    tá}  &   
                     \gloss{‘shave’}  &  \\

                     \vernacular{ukha[leera]
                    tá}  &   
                     \gloss{‘bring’}  &     &   
                     \vernacular{ukha[khalaka]
                    tá}  &   
                     \gloss{‘cut’}  &  \\

                     \vernacular{ukha[sitaaka]
                    {\downstep}tá}  &   
                     \gloss{‘accuse’}  &     &   
                     \vernacular{ukha[boolitsa]
                    {\downstep}tá}  &   
                     \gloss{‘seduce’}  &  \\

                     \vernacular{ukha[khong’oonda]
                    {\downstep}tá}  &   
                     \gloss{‘knock’}  &     &   
                     \vernacular{ukha[boholola]
                    {\downstep}tá}  &   
                     \gloss{‘untie’}  &  \\

                     \vernacular{ukha[boyong’ana]
                    {\downstep}tá}  &   
                     \gloss{‘go around’}  &  \\
\end{tabular}
%\caption{\nocaption}
     
\begin{tabular}{llllll}  
  \multicolumn{5}{l}{
                     \vernacular{(298) /Ø/
                    C-Initial} \gloss{‘do
                    not...!’} } &  \\
\multicolumn{5}{l}{ } &  \\

                     \vernacular{ukha[tsíá]
                    {\downstep}tá}  &   
                     \gloss{‘go’}  &     &   
                     \vernacular{ukha[lekhá]
                    {\downstep}tá}  &   
                     \gloss{‘leave’}  &  \\

                     \vernacular{ukha[loónda]
                    tá}  &   
                     \gloss{‘follow’}  &     &   
                     \vernacular{ukha[kumíla]
                    tá}  &   
                     \gloss{‘hold’}  &  \\

                     \vernacular{ukha[kulíkha]
                    tá}  &   
                     \gloss{‘name’}  &     &   
                     \vernacular{ukha[lakhúula]
                    tá}  &   
                     \gloss{‘release’}  &  \\

                     \vernacular{ukha[seébúla]
                    tá}  &   
                     \gloss{‘say bye’}  &     &   
                     \vernacular{ukha[kalúshítsa]
                    tá}  &   
                     \gloss{‘return’}  &  \\

                     \vernacular{ukha[reébáreeba]
                    tá}  &   
                     \gloss{‘ask (iter)’}  &     &   
                     \vernacular{
                    ukha[kalúkhányinya] tá}  &   
                     \gloss{‘turn over’}  &  \\
\end{tabular}
%\caption{\nocaption}
     
\begin{tabular}{llllll}  
  \multicolumn{5}{l}{
                     \vernacular{(299) /H/
                    C-Initial + OP} \gloss{‘do
                    not...him/her!’} } &  \\
\multicolumn{5}{l}{ } &  \\

                     \vernacular{ukhamu[rá]
                    {\downstep}tá}  &   
                     \gloss{‘bury’}  &     &   
                     \vernacular{ukhamu[béka]
                    tá}  &   
                     \gloss{‘shave’}  &  \\

                     \vernacular{ukhamu[léera]
                    tá}  &   
                     \gloss{‘bring’}  &     &   
                     \vernacular{ukhamu[khálaka]
                    tá}  &   
                     \gloss{‘cut’}  &  \\

                     \vernacular{ukhamu[sítaaka]
                    tá}  &   
                     \gloss{‘accuse’}  &     &   
                     \vernacular{ukhamu[bóolitsa]
                    tá}  &   
                     \gloss{‘seduce’}  &  \\

                     \vernacular{
                    ukhamu[khóng’oonda] tá}  &   
                     \gloss{‘knock’}  &     &   
                     \vernacular{ukhamu[bóholola]
                    tá}  &   
                     \gloss{‘untie’}  &  \\

                     \vernacular{
                    ukhamu[bóyong’ana] tá}  &   
                     \gloss{‘go around’}  &  \\
\end{tabular}
%\caption{\nocaption}
     
\begin{tabular}{llllll}  
  \multicolumn{5}{l}{
                     \vernacular{(300) /Ø/
                    C-Initial + OP} \gloss{‘do
                    not...him/her!’} } &  \\
\multicolumn{5}{l}{ } &  \\

                     \vernacular{ukhamu[tsíá]
                    {\downstep}tá}  &   
                     \gloss{‘go for’}  &  \\

                     \vernacular{ukhamu[lekhá]
                    {\downstep}tá}  &   
                     \gloss{‘leave’}  &  \\

                     \vernacular{ukhamu[loónda]
                    tá}  &   
                     \gloss{‘follow’}  &  \\

                     \vernacular{ukhamu[kulíkha]
                    tá}  &   
                     \gloss{‘name’}  &  \\

                     \vernacular{ukhamu[lakhúula]
                    tá}  &   
                     \gloss{‘release’}  &  \\

                     \vernacular{ukhamu[seébúla]
                    tá}  &   
                     \gloss{‘say bye to’}  &  \\

                     \vernacular{
                    ukhamu[kalúshítsa] tá}  &   
                     \gloss{‘return’}  &  \\

                     \vernacular{
                    ukhamu[reébáreeba] tá}  &   
                     \gloss{‘ask (iter)’}  &  \\

                     \vernacular{
                    ukhamu[kalúkhányinya] tá}  &   
                     \gloss{
                    ‘turn...over’}  &  \\
\end{tabular}
%\caption{\nocaption}
     
\begin{tabular}{llllll}  
  \multicolumn{5}{l}{
                     \vernacular{(301) /H/
                    C-Initial + OP + OP
                    } \gloss{‘do not...him/her
                    for me!’} } &  \\
\multicolumn{5}{l}{ } &  \\

                     \vernacular{ukhamuú[ndeela]
                    {\downstep}tá}  &   
                     \gloss{‘bury’}  &     &   
                     \vernacular{ukhamuú[mbechela]
                    {\downstep}tá}  &   
                     \gloss{‘shave’}  &  \\

                     \vernacular{ukhamuú[ndeerela]
                    {\downstep}tá}  &   
                     \gloss{‘bring’}  &     &   
                     \vernacular{
                    ukhamuú[khalachila] {\downstep}tá}  &   
                     \gloss{‘cut’}  &  \\

                     \vernacular{
                    ukhamuú[sitaachila] {\downstep}tá}  &   
                     \gloss{‘accuse’}  &     &   
                     \vernacular{
                    ukhamuú[mboolitsila] {\downstep}tá}  &   
                     \gloss{‘seduce’}  &  \\

                     \vernacular{
                    ukhamuú[mbohololela] {\downstep}tá}  &   
                     \gloss{‘untie’}  &     &     &     &  \\
\end{tabular}
%\caption{\nocaption}
     
\begin{tabular}{llllll}  
  \multicolumn{5}{l}{
                     \vernacular{(302) /Ø/
                    C-Initial + OP + OP
                    } \gloss{‘do not...him/her
                    for me!’} } &  \\
\multicolumn{5}{l}{ } &  \\

                     \vernacular{
                    ukhamuú[{\downstep}nzííla] tá}  &   
                     \gloss{‘go for’}  &  \\

                     \vernacular{
                    ukhamuú[{\downstep}ndéshéla] tá}  &   
                     \gloss{‘leave’}  &  \\

                     \vernacular{
                    ukhamuú[{\downstep}nóóndéla] tá}  &   
                     \gloss{‘follow’}  &  \\

                     \vernacular{
                    ukhamuú[{\downstep}ngúlíshíla] tá}  &   
                     \gloss{‘name’}  &  \\

                     \vernacular{
                    ukhamuú[{\downstep}ndákhúulila] tá}  &   
                     \gloss{‘release’}  &  \\

                     \vernacular{
                    ukhamuú[{\downstep}séébúlila] tá}  &   
                     \gloss{‘say bye to’}  &  \\

                     \vernacular{
                    ukhamuú[{\downstep}mbóómbélitsila] tá}  &   
                     \gloss{‘comfort’}  &  \\

                     \vernacular{
                    ukhamuú[{\downstep}síínjílitsila] tá}  &   
                     \gloss{
                    ‘make...stand’}  &  \\
\end{tabular}
%\caption{\nocaption}
     
\begin{tabular}{lll}  
  \multicolumn{2}{l}{
                     \vernacular{(303) /H/
                    C-Initial Phrase-Medial} \gloss{‘do not...the boy
                    \ob mú{\downstep}yáyi\cb  /} } &  \\
\multicolumn{2}{l}{
                     \gloss{someone
                    \ob muundu\cb !’} } &  \\

                     \vernacular{ukha[ra]
                    mú{\downstep}yáyi/muundu}  &   
                     \gloss{‘bury’}  &  \\

                     \vernacular{ukha[beka]
                    mú{\downstep}yáyi/muundu tá}  &   
                     \gloss{‘shave’}  &  \\

                     \vernacular{ukha[leera]
                    mú{\downstep}yáyi/muundu tá}  &   
                     \gloss{‘bring’}  &  \\

                     \vernacular{ukha[khalaka]
                    mú{\downstep}yáyi/muundu tá}  &   
                     \gloss{‘cut’}  &  \\

                     \vernacular{ukha[sitaaka]
                    mú{\downstep}yáyi/muundu tá}  &   
                     \gloss{‘accuse’}  &  \\

                     \vernacular{ukha[boolitsa]
                    mú{\downstep}yáyi/muundu tá}  &   
                     \gloss{‘seduce’}  &  \\

                     \vernacular{ukha[khong’oonda]
                    mú{\downstep}yáyi/muundu tá}  &   
                     \gloss{‘knock’}  &  \\

                     \vernacular{ukha[boholola]
                    mú{\downstep}yáyi/muundu tá}  &   
                     \gloss{‘untie’}  &  \\

                     \vernacular{ukha[boyong’ana]
                    mú{\downstep}yáyi/muundu tá}  &   
                     \gloss{‘go around’}  &  \\
\end{tabular}
%\caption{\nocaption}
     
\begin{tabular}{lll}  
  \multicolumn{2}{l}{
                     \vernacular{(304) /Ø/
                    C-Initial Phrase-Medial} \gloss{‘do not...the boy
                    \ob mú{\downstep}yáyi\cb  /} } &  \\
\multicolumn{2}{l}{
                     \gloss{someone
                    \ob muundu\cb !’} } &  \\

                     \vernacular{ukha[tsíá]
                    {\downstep}mú{\downstep}yáyi/muundu tá}  &   
                     \gloss{‘go for’}  &  \\

                     \vernacular{ukha[lekhá]
                    {\downstep}mú{\downstep}yáyi/muundu tá}  &   
                     \gloss{‘leave’}  &  \\

                     \vernacular{ukha[loóndá]
                    {\downstep}mú{\downstep}yáyi/muundu tá}  &   
                     \gloss{‘follow’}  &  \\

                     \vernacular{ukha[kulíkhá]
                    {\downstep}mú{\downstep}yáyi/muundu tá}  &   
                     \gloss{‘name’}  &  \\

                     \vernacular{ukha[lakhúula]
                    mú{\downstep}yáyi/muundu tá}  &   
                     \gloss{‘release’}  &  \\

                     \vernacular{ukha[seébúla]
                    mú{\downstep}yáyi/muundu tá}  &   
                     \gloss{‘say bye to’}  &  \\

                     \vernacular{ukha[kalúshítsa]
                    mú{\downstep}yáyi/muundu tá}  &   
                     \gloss{‘return’}  &  \\

                     \vernacular{ukha[reébáreeba]
                    mú{\downstep}yáyi/muundu tá}  &   
                     \gloss{‘ask (iter)’}  &  \\

                     \vernacular{
                    ukha[kalúkhányinya] mú{\downstep}yáyi/muundu
                    tá}  &   
                     \gloss{
                    ‘turn...over’}  &  \\
\end{tabular}
%\caption{\nocaption}
     
\begin{tabular}{lll}  
  \multicolumn{2}{l}{
                     \vernacular{(305) /H/
                    C-Initial +OP Phrase-Medial} \gloss{‘do not...the boy
                    \ob mú{\downstep}yáyi\cb  /} } &  \\
\multicolumn{2}{l}{
                     \gloss{someone \ob muundu\cb 
                    for him/her!’} } &  \\

                     \vernacular{ukhamu[réela]
                    mú{\downstep}yáyi/muundu tá}  &   
                     \gloss{‘bury’}  &  \\

                     \vernacular{ukhamu[béchela]
                    mú{\downstep}yáyi/muundu tá}  &   
                     \gloss{‘shave’}  &  \\

                     \vernacular{ukhamu[léerela]
                    mú{\downstep}yáyi/muundu tá}  &   
                     \gloss{‘bring’}  &  \\

                     \vernacular{
                    ukhamu[khálachila] mú{\downstep}yáyi/muundu
                    tá}  &   
                     \gloss{‘cut’}  &  \\

                     \vernacular{
                    ukhamu[sítaachila] mú{\downstep}yáyi/muundu
                    tá}  &   
                     \gloss{‘accuse’}  &  \\

                     \vernacular{
                    ukhamu[bóolitsila] mú{\downstep}yáyi/muundu
                    tá}  &   
                     \gloss{‘seduce’}  &  \\

                     \vernacular{
                    ukhamu[khóng’oondela] mú{\downstep}yáyi/muundu
                    tá}  &   
                     \gloss{‘knock’}  &  \\

                     \vernacular{
                    ukhamu[bóhololela] mú{\downstep}yáyi/muundu
                    tá}  &   
                     \gloss{‘untie’}  &  \\

                     \vernacular{
                    ukhamu[bóyong’anila] mú{\downstep}yáyi/muundu
                    tá}  &   
                     \gloss{‘go around’}  &  \\
\end{tabular}
%\caption{\nocaption}
     
\begin{tabular}{lll}  
  \multicolumn{2}{l}{
                     \vernacular{(306) /Ø/
                    C-Initial +OP Phrase-Medial} \gloss{‘do not...the boy
                    \ob mú{\downstep}yáyi\cb  /} } &  \\
\multicolumn{2}{l}{
                     \gloss{someone \ob muundu\cb 
                    for him/her!’} } &  \\

                     \vernacular{ukhamu[tsiílá]
                    {\downstep}mú{\downstep}yáyi/muundu tá}  &   
                     \gloss{‘grind’}  &  \\

                     \vernacular{ukhamu[leshélá]
                    {\downstep}mú{\downstep}yáyi/muundu tá}  &   
                     \gloss{‘leave’}  &  \\

                     \vernacular{ukhamu[loóndéla]
                    mú{\downstep}yáyi/muundu tá}  &   
                     \gloss{‘follow’}  &  \\

                     \vernacular{
                    ukhamu[kulíshíla] mú{\downstep}yáyi/muundu
                    tá}  &   
                     \gloss{‘name’}  &  \\

                     \vernacular{
                    ukhamu[lakhúulila] mú{\downstep}yáyi/muundu
                    tá}  &   
                     \gloss{‘release’}  &  \\

                     \vernacular{
                    ukhamu[seébúlila] mú{\downstep}yáyi/muundu
                    tá}  &   
                     \gloss{‘say bye to’}  &  \\

                     \vernacular{
                    ukhamu[kalúshítsila] mú{\downstep}yáyi/muundu
                    tá}  &   
                     \gloss{‘return’}  &  \\

                     \vernacular{
                    ukhamu[reébáreebela] mú{\downstep}yáyi/muundu
                    tá}  &   
                     \gloss{‘ask (iter)’}  &  \\
\end{tabular}
%\caption{\nocaption}
     
\begin{tabular}{lll}  
  \multicolumn{2}{l}{
                     \vernacular{(307) /H/
                    C-Initial +OP + OP
                    } \gloss{‘do not...the boy
                    \ob mú{\downstep}yáyi\cb  /} } &  \\
\multicolumn{2}{l}{
                     \gloss{someone \ob muundu\cb 
                    for him/her for me!’
                    } } &  \\

                     \vernacular{ukhamuú[ndeela]
                    mú{\downstep}yáyi/muundu tá}  &   
                     \gloss{‘bury’}  &  \\

                     \vernacular{ukhamuú[mbechela]
                    mú{\downstep}yáyi/muundu tá}  &   
                     \gloss{‘shave’}  &  \\

                     \vernacular{ukhamuú[ndeerela]
                    mú{\downstep}yáyi/muundu tá}  &   
                     \gloss{‘bring’}  &  \\

                     \vernacular{
                    ukhamuú[khalachila] mú{\downstep}yáyi/muundu
                    tá}  &   
                     \gloss{‘cut’}  &  \\

                     \vernacular{
                    ukhamuú[sitaachila] mú{\downstep}yáyi/muundu
                    tá}  &   
                     \gloss{‘accuse’}  &  \\

                     \vernacular{
                    ukhamuú[mboolitsila] mú{\downstep}yáyi/muundu
                    tá}  &   
                     \gloss{‘seduce’}  &  \\
\end{tabular}
%\caption{\nocaption}
     
\begin{tabular}{lll}  
  \multicolumn{2}{l}{
                     \vernacular{(308) /Ø/
                    C-Initial +OP + OP
                    } \gloss{‘do not...the boy
                    \ob mú{\downstep}yáyi\cb  /} } &  \\
\multicolumn{2}{l}{
                     \gloss{someone \ob muundu\cb 
                    for him/her for me!’} } &  \\

                     \vernacular{
                    ukhamuú[{\downstep}nzíílá] {\downstep}mú{\downstep}yáyi/muundu
                    tá}  &   
                     \gloss{‘go for’}  &  \\

                     \vernacular{
                    ukhamuú[{\downstep}ndéshélá] {\downstep}mú{\downstep}yáyi/muundu
                    tá}  &   
                     \gloss{‘leave’}  &  \\

                     \vernacular{
                    ukhamuú[{\downstep}nóóndéla] mú{\downstep}yáyi/muundu
                    tá}  &   
                     \gloss{‘follow’}  &  \\

                     \vernacular{
                    ukhamuú[{\downstep}ngúlíshíla] mú{\downstep}yáyi/muundu
                    tá}  &   
                     \gloss{‘name’}  &  \\

                     \vernacular{
                    ukhamuú[{\downstep}ndákhúulila] mú{\downstep}yáyi/muundu
                    tá}  &   
                     \gloss{‘release’}  &  \\

                     \vernacular{
                    ukhamuú[{\downstep}séébúlila] mú{\downstep}yáyi/muundu
                    tá}  &   
                     \gloss{‘say bye to’}  &  \\
\end{tabular}
%\caption{\nocaption}
    

\subsection{Crastinal Future: Pattern 3}\label{sec:sCrastFut}


\begin{tabular}{llllll}  
  \multicolumn{5}{l}{
                     \vernacular{(309) /H/
                    C-Initial} \gloss{‘s/he
                    will...’} } &  \\
\multicolumn{5}{l}{ } &  \\

                     \vernacular{
                    naa[rɛ́]}  &   
                     \gloss{‘bury’}  &     &   
                     \vernacular{
                    naa[ng’wí]}  &   
                     \gloss{‘drink’}  &  \\

                     \vernacular{
                    naa[khwí]}  &   
                     \gloss{‘eat’}  &     &   
                     \vernacular{
                    naa[lí]}  &   
                     \gloss{‘pay dowry’}  &  \\

                     \vernacular{
                    naa[lumɪ́]}  &   
                     \gloss{‘bite’}  &     &   
                     \vernacular{
                    naa[bechɛ́]}  &   
                     \gloss{‘shave’}  &  \\

                     \vernacular{
                    naa[teeshɛ́]}  &   
                     \gloss{‘cook’}  &     &   
                     \vernacular{
                    naa[leerɛ́]}  &   
                     \gloss{‘bring’}  &  \\

                     \vernacular{
                    naa[khalachɛ́]}  &   
                     \gloss{‘cut’}  &     &   
                     \vernacular{
                    naa[kalaánjɛ]}  &   
                     \gloss{‘fry’}  &  \\

                     \vernacular{
                    naa[sitaáchɛ]}  &   
                     \gloss{‘accuse’}  &     &   
                     \vernacular{
                    naa[boolitsɪ́]}  &   
                     \gloss{‘seduce’}  &  \\

                     \vernacular{
                    naa[saanditsɪ́]}  &   
                     \gloss{‘thank’}  &     &   
                     \vernacular{
                    naa[khong’oóndɛ]}  &   
                     \gloss{‘knock’}  &  \\

                     \vernacular{
                    naa[boholólɛ]}  &   
                     \gloss{‘untie’}  &     &   
                     \vernacular{
                    naa[boyong’ánɛ]}  &   
                     \gloss{‘go around’}  &  \\

                     \vernacular{
                    naa[boyoóng’anɛ]}  &   
                     \gloss{‘be
                    confused’}  &     &   
                     \vernacular{
                    naa[ng’ong’oólitsɪ]}  &   
                     \gloss{‘tease’}  &  \\

                     \vernacular{
                    naa[ling(ak)anyínyɪ]}  &   
                     \gloss{‘crumple’}  &  \\
\end{tabular}
%\caption{\nocaption}
     
\begin{tabular}{llllll}  
  \multicolumn{5}{l}{
                     \vernacular{(310) /H/
                    V-Initial} \gloss{‘s/he
                    will...’} } &  \\
\multicolumn{5}{l}{ } &  \\

                     \vernacular{
                    niy[irɪ́]}  &   
                     \gloss{‘kill’}  &     &   
                     \vernacular{
                    niy[ikoómbɛ]}  &   
                     \gloss{‘admire’}  &  \\

                     \vernacular{
                    niy[isiáchɛ]}  &   
                     \gloss{‘smack’}  &     &   
                     \vernacular{
                    niy[ikobólɛ]}  &   
                     \gloss{‘belch’}  &  \\

                     \vernacular{
                    niy[ononyínyɪ]}  &   
                     \gloss{‘spoil’}  &     &   
                     \vernacular{
                    niy[abukhányinyɪ]}  &   
                     \gloss{‘separate’}  &  \\
\end{tabular}
%\caption{\nocaption}
     
\begin{tabular}{llllll}  
  \multicolumn{5}{l}{
                     \vernacular{(311) /Ø/
                    C-Initial} \gloss{‘s/he
                    will...’} } &  \\
\multicolumn{5}{l}{ } &  \\

                     \vernacular{
                    naa[tsí]}  &   
                     \gloss{‘go’}  &     &   
                     \vernacular{
                    naa[kwí]}  &   
                     \gloss{‘fall’}  &  \\

                     \vernacular{
                    naa[leshɛ́]}  &   
                     \gloss{‘leave’}  &     &   
                     \vernacular{
                    naa[reebɛ́]}  &   
                     \gloss{‘ask’}  &  \\

                     \vernacular{
                    naa[loondɛ́]}  &   
                     \gloss{‘follow’}  &     &   
                     \vernacular{
                    naa[kumilɪ́]}  &   
                     \gloss{‘hold’}  &  \\

                     \vernacular{
                    naa[kulishɪ́]}  &   
                     \gloss{‘name’}  &     &   
                     \vernacular{
                    naa[homoólɛ]}  &   
                     \gloss{‘massage’}  &  \\

                     \vernacular{
                    naa[lakhuúlɪ]}  &   
                     \gloss{‘release’}  &     &   
                     \vernacular{
                    naa[seebulɪ́]}  &   
                     \gloss{‘say bye’}  &  \\

                     \vernacular{
                    naa[hoombelítsɪ]}  &   
                     \gloss{‘comfort’}  &     &   
                     \vernacular{
                    naa[kalushítsɪ]}  &   
                     \gloss{‘return’}  &  \\

                     \vernacular{
                    naa[siinjilítsɪ]}  &   
                     \gloss{‘make stand’}  &     &   
                     \vernacular{
                    naa[reebaréebɛ]}  &   
                     \gloss{‘ask (iter)’}  &  \\

                     \vernacular{
                    naa[kalukhányinyɪ]}  &   
                     \gloss{‘turn over’}  &     &   
                     \vernacular{
                    naa[sebulúkhányinyɪ]}  &   
                     \gloss{‘scatter’}  &  \\
\end{tabular}
%\caption{\nocaption}
     
\begin{tabular}{llllll}  
  \multicolumn{5}{l}{
                     \vernacular{(312) /Ø/
                    V-Initial} \gloss{‘s/he
                    will...’} } &  \\
\multicolumn{5}{l}{ } &  \\

                     \vernacular{
                    niy[enyɛ́]}  &   
                     \gloss{‘want’}  &     &   
                     \vernacular{
                    niy[eyelɛ́]}  &   
                     \gloss{‘wipe for’}  &  \\

                     \vernacular{
                    niy[iluúlɪ]}  &   
                     \gloss{‘winnow’}  &     &   
                     \vernacular{
                    niy[ambakhánɛ]}  &   
                     \gloss{‘refuse’}  &  \\

                     \vernacular{
                    niy[eleélitsɪ]}  &   
                     \gloss{‘hang up’}  &     &     &     &  \\
\end{tabular}
%\caption{\nocaption}
     
\begin{tabular}{llllll}  
  \multicolumn{5}{l}{
                     \vernacular{(313) /H/
                    C-Initial + OP} \gloss{‘s/he
                    will...him/her’} } &  \\
\multicolumn{5}{l}{ } &  \\

                     \vernacular{
                    naamu[rɛ́]}  &   
                     \gloss{‘bury’}  &     &   
                     \vernacular{
                    naamu[béchɛ]}  &   
                     \gloss{‘shave’}  &  \\

                     \vernacular{
                    naamu[léerɛ]}  &   
                     \gloss{‘bring’}  &     &   
                     \vernacular{
                    naamu[khálachɛ]}  &   
                     \gloss{‘cut’}  &  \\

                     \vernacular{
                    naamu[sítaachɛ]}  &   
                     \gloss{‘accuse’}  &     &   
                     \vernacular{
                    naamu[bóolitsɪ]}  &   
                     \gloss{‘seduce’}  &  \\

                     \vernacular{
                    naamu[khóng’oondɛ]}  &   
                     \gloss{‘knock’}  &     &   
                     \vernacular{
                    naamu[bóhololɛ]}  &   
                     \gloss{‘untie’}  &  \\

                     \vernacular{
                    naamu[bóyong’anɛ]}  &   
                     \gloss{‘go around’}  &     &   
                     \vernacular{
                    naamu[ng’óng’oolitsɪ]}  &   
                     \gloss{‘tease’}  &  \\

                     \vernacular{
                    naamu[língakanyinyɪ]}  &   
                     \gloss{‘bend’}  &     &     &     &  \\
\end{tabular}
%\caption{\nocaption}
     
\begin{tabular}{llllll}  
  \multicolumn{5}{l}{
                     \vernacular{(314) /H/
                    V-Initial + OP} \gloss{‘s/he
                    will...him/her’} } &  \\
\multicolumn{5}{l}{ } &  \\

                     \vernacular{
                    naamw[iírɪ]}  &   
                     \gloss{‘kill’}  &     &   
                     \vernacular{
                    naamw[ií{\downstep}kóómbɛ]}  &   
                     \gloss{‘admire’}  &  \\

                     \vernacular{
                    naamw[ií{\downstep}síáchɛ]}  &   
                     \gloss{‘smack’}  &     &   
                     \vernacular{
                    naamw[oónonyinyɪ]}  &   
                     \gloss{‘spoil’}  &  \\

                     \vernacular{
                    naamw[aábukhanyinyɪ]}  &   
                     \gloss{‘separate’}  &  \\
\end{tabular}
%\caption{\nocaption}
     
\begin{tabular}{llllll}  
  \multicolumn{5}{l}{
                     \vernacular{(315) /Ø/
                    C-Initial + OP} \gloss{‘s/he
                    will...him/her \ob mu-\cb  / them
                    } } &  \\
\multicolumn{5}{l}{ } &  \\

                     \vernacular{
                    naamu[tsí]}  &   
                     \gloss{‘go for’}  &  \\

                     \vernacular{
                    naamu[leshɛ́]}  &   
                     \gloss{‘leave’}  &  \\

                     \vernacular{
                    naamu[loóndɛ]}  &   
                     \gloss{‘follow’}  &  \\

                     \vernacular{
                    naamu[kulíshɪ]}  &   
                     \gloss{‘name’}  &  \\

                     \vernacular{
                    naamu[lakhúulɪ]}  &   
                     \gloss{‘release’}  &  \\

                     \vernacular{
                    naamu[seébulɪ]}  &   
                     \gloss{‘say bye to’}  &  \\

                     \vernacular{
                    naamu[hoómbélitsɪ]}  &   
                     \gloss{‘comfort’}  &  \\

                     \vernacular{
                    naamu[kalúshitsɪ]}  &   
                     \gloss{‘return’}  &  \\

                     \vernacular{
                    naamu[siínjílitsɪ]}  &   
                     \gloss{
                    ‘make...stand’}  &  \\

                     \vernacular{
                    naamu[reébɛ́reebɛ]}  &   
                     \gloss{‘ask (iter)’}  &  \\

                     \vernacular{
                    naamu[kalúkhányinyɪ]}  &   
                     \gloss{
                    ‘turn...over’}  &  \\

                     \vernacular{
                    naabi[sebúlúkhanyinyɪ]}  &   
                     \gloss{‘scatter’}  &  \\
\end{tabular}
%\caption{\nocaption}
     
\begin{tabular}{llllll}  
  \multicolumn{5}{l}{
                     \vernacular{(316) /Ø/
                    V-Initial + OP} \gloss{‘s/he
                    will...him/her \ob mw-\cb  / it
                    } } &  \\
\multicolumn{5}{l}{ } &  \\

                     \vernacular{
                    naamw[eenyɛ́]}  &   
                     \gloss{‘want’}  &     &   
                     \vernacular{
                    naamw[eeyɛ́lɛ]}  &   
                     \gloss{‘wipe for’}  &  \\

                     \vernacular{
                    naabw[iilúulɪ]}  &   
                     \gloss{‘winnow’}  &     &   
                     \vernacular{
                    naamw[aambákhanɛ]}  &   
                     \gloss{‘refuse’}  &  \\

                     \vernacular{
                    naamw[eeléelitsɪ]}  &   
                     \gloss{
                    ‘carry...hanging’}  &  \\
\end{tabular}
%\caption{\nocaption}
     
\begin{tabular}{llllll}  
  \multicolumn{5}{l}{
                     \vernacular{(317) /H/
                    C-Initial + OP
                    } \gloss{‘s/he
                    will...me’} } &  \\
\multicolumn{5}{l}{ } &  \\

                     \vernacular{
                    naa[rí]}  &   
                     \gloss{‘fear’}  &     &   
                     \vernacular{
                    naa[mbéchɛ]}  &   
                     \gloss{‘shave’}  &  \\

                     \vernacular{
                    naa[ndéerɛ]}  &   
                     \gloss{‘bring’}  &     &   
                     \vernacular{
                    naa[khálachɛ]}  &   
                     \gloss{‘cut’}  &  \\

                     \vernacular{
                    naa[sítaachɛ]}  &   
                     \gloss{‘accuse’}  &     &   
                     \vernacular{
                    naa[mbóolitsɪ]}  &   
                     \gloss{‘seduce’}  &  \\

                     \vernacular{
                    naa[khóng’oondɛ]}  &   
                     \gloss{‘knock’}  &     &   
                     \vernacular{
                    naa[mbóhololɛ]}  &   
                     \gloss{‘untie’}  &  \\

                     \vernacular{
                    naa[mbóyong’anɛ]}  &   
                     \gloss{‘go around’}  &     &   
                     \vernacular{
                    naa[ng’óng’oolitsɪ]}  &   
                     \gloss{‘tease’}  &  \\

                     \vernacular{
                    naa[níngakanyinyɪ]}  &   
                     \gloss{‘bend’}  &  \\
\end{tabular}
%\caption{\nocaption}
     
\begin{tabular}{llllll}  
  \multicolumn{5}{l}{
                     \vernacular{(318) /H/
                    V-Initial + OP
                    } \gloss{‘s/he
                    will...me’} } &  \\
\multicolumn{5}{l}{ } &  \\

                     \vernacular{
                    naa[nzírɪ]}  &   
                     \gloss{‘kill’}  &     &   
                     \vernacular{
                    naa[nzí{\downstep}kóómbɛ]}  &   
                     \gloss{‘admire’}  &  \\

                     \vernacular{
                    naa[nzí{\downstep}síáchɛ]}  &   
                     \gloss{‘smack’}  &     &   
                     \vernacular{
                    naa[nzónonyinyɪ]}  &   
                     \gloss{‘spoil’}  &  \\

                     \vernacular{
                    naa[nzábukhanyinyɪ]}  &   
                     \gloss{‘separate’}  &  \\
\end{tabular}
%\caption{\nocaption}
     
\begin{tabular}{llllll}  
  \multicolumn{5}{l}{
                     \vernacular{(319) /Ø/
                    C-Initial + OP
                    } \gloss{‘s/he
                    will...me’} } &  \\
\multicolumn{5}{l}{ } &  \\

                     \vernacular{
                    naa[sí]}  &   
                     \gloss{‘grind’}  &     &   
                     \vernacular{
                    naa[ndeshɛ́]}  &   
                     \gloss{‘leave’}  &  \\

                     \vernacular{
                    naa[noóndɛ]}  &   
                     \gloss{‘follow’}  &     &   
                     \vernacular{
                    naa[ngulíshɪ]}  &   
                     \gloss{‘name’}  &  \\

                     \vernacular{
                    naa[ndakhúulɪ]}  &   
                     \gloss{‘release’}  &     &   
                     \vernacular{
                    naa[seébulɪ]}  &   
                     \gloss{‘say bye to’}  &  \\

                     \vernacular{
                    naa[mboómbélitsɪ]}  &   
                     \gloss{‘comfort’}  &     &   
                     \vernacular{
                    naa[siínjílitsɪ]}  &   
                     \gloss{
                    ‘make..stand’}  &  \\

                     \vernacular{
                    naa[ndeébɛ́ndeebɛ]}  &   
                     \gloss{‘ask (iter)’}  &     &   
                     \vernacular{
                    naa[ngalúkhányinyɪ]}  &   
                     \gloss{
                    ‘turn...over’}  &  \\
\end{tabular}
%\caption{\nocaption}
     
\begin{tabular}{llllll}  
  \multicolumn{5}{l}{
                     \vernacular{(320) /Ø/
                    V-Initial + OP
                    } \gloss{‘s/he
                    will...me’} } &  \\
\multicolumn{5}{l}{ } &  \\

                     \vernacular{
                    naa[nzenyɛ́]}  &   
                     \gloss{‘want’}  &     &   
                     \vernacular{
                    naa[nzeyélɛ]}  &   
                     \gloss{‘wipe for’}  &  \\

                     \vernacular{
                    naa[nyambákhanɛ]}  &   
                     \gloss{‘refuse’}  &     &   
                     \vernacular{
                    naa[nzeléelitsɪ]}  &   
                     \gloss{
                    ‘carry...hanging’}  &  \\
\end{tabular}
%\caption{\nocaption}
     
\begin{tabular}{llllll}  
  \multicolumn{5}{l}{
                     \vernacular{(321) /H/
                    C-Initial + OP
                    } \gloss{‘s/he
                    will...yourself’} } &  \\
\multicolumn{5}{l}{ } &  \\

                     \vernacular{
                    niyii[rɛ́]}  &   
                     \gloss{‘bury’}  &     &   
                     \vernacular{
                    niyii[béchɛ]}  &   
                     \gloss{‘shave’}  &  \\

                     \vernacular{
                    niyii[súunjɪ]}  &   
                     \gloss{‘hang’}  &     &   
                     \vernacular{
                    niyii[khálachɛ]}  &   
                     \gloss{‘cut’}  &  \\

                     \vernacular{
                    niyii[sítaachɛ]}  &   
                     \gloss{‘accuse’}  &     &   
                     \vernacular{
                    niyii[sáanditsɪ]}  &   
                     \gloss{‘thank’}  &  \\

                     \vernacular{
                    niyii[khóng’oondɛ]}  &   
                     \gloss{‘knock’}  &     &   
                     \vernacular{
                    niyii[bóhololɛ]}  &   
                     \gloss{‘untie’}  &  \\
\end{tabular}
%\caption{\nocaption}
     
\begin{tabular}{llllll}  
  \multicolumn{5}{l}{
                     \vernacular{(322) /H/
                    V-Initial + OP
                    } \gloss{‘s/he
                    will...yourself’} } &  \\
\multicolumn{5}{l}{ } &  \\

                     \vernacular{
                    niyii[yírɪ]}  &   
                     \gloss{‘kill’}  &     &   
                     \vernacular{
                    niyii[yí{\downstep}kóómbɛ]}  &   
                     \gloss{‘admire’}  &  \\

                     \vernacular{
                    niyii[yí{\downstep}síáchɛ]}  &   
                     \gloss{‘smack’}  &     &   
                     \vernacular{
                    niyii[yónonyinyɪ]}  &   
                     \gloss{‘spoil’}  &  \\

                     \vernacular{
                    niyii[yábukhanyinyɪ]}  &   
                     \gloss{‘separate’}  &  \\
\end{tabular}
%\caption{\nocaption}
     
\begin{tabular}{llllll}  
  \multicolumn{5}{l}{
                     \vernacular{(323) /Ø/
                    C-Initial + OP
                    } \gloss{‘s/he
                    will...yourself’} } &  \\
\multicolumn{5}{l}{ } &  \\

                     \vernacular{
                    niyii[sí]}  &   
                     \gloss{‘grind’}  &     &   
                     \vernacular{
                    niyii[leshɛ́]}  &   
                     \gloss{‘leave’}  &  \\

                     \vernacular{
                    niyii[siínjɪ]}  &   
                     \gloss{‘bathe’}  &     &   
                     \vernacular{
                    niyii[kulíshɪ]}  &   
                     \gloss{‘name’}  &  \\

                     \vernacular{
                    niyii[naábulɪ]}  &   
                     \gloss{‘undress’}  &     &   
                     \vernacular{
                    niyii[lakhúulɪ]}  &   
                     \gloss{‘release’}  &  \\

                     \vernacular{
                    niyii[hoómbélitsɪ]}  &   
                     \gloss{‘comfort’}  &     &   
                     \vernacular{
                    niyii[siínjílitsɪ]}  &   
                     \gloss{
                    ‘make...stand’}  &  \\

                     \vernacular{
                    niyii[reébɛ́reebɛ]}  &   
                     \gloss{‘ask (iter)’}  &     &   
                     \vernacular{
                    niyii[kalúkhányinyɪ]}  &   
                     \gloss{
                    ‘turn...over’}  &  \\
\end{tabular}
%\caption{\nocaption}
     
\begin{tabular}{llllll}  
  \multicolumn{5}{l}{
                     \vernacular{(324) /Ø/
                    V-Initial + OP
                    } \gloss{‘s/he
                    will...yourself’} } &  \\
\multicolumn{5}{l}{ } &  \\

                     \vernacular{
                    niyii[yalɛ́]}  &   
                     \gloss{‘expose’}  &     &   
                     \vernacular{
                    niyii[yeyélɛ]}  &   
                     \gloss{‘wipe for’}  &  \\

                     \vernacular{
                    niyii[yambákhanɛ]}  &   
                     \gloss{‘refuse’}  &     &   
                     \vernacular{
                    niyii[yeléelitsɪ]}  &   
                     \gloss{‘hang...up’}  &  \\
\end{tabular}
%\caption{\nocaption}
     
\begin{tabular}{llllll}  
  \multicolumn{5}{l}{
                     \vernacular{(325) /H/
                    C-Initial + OP + OP
                    } \gloss{‘s/he
                    will...him/her for me’} } &  \\
\multicolumn{5}{l}{ } &  \\

                     \vernacular{
                    naamuú[ndeelɛ]}  &   
                     \gloss{‘bury’}  &     &   
                     \vernacular{
                    naamuú[mbechelɛ]}  &   
                     \gloss{‘shave’}  &  \\

                     \vernacular{
                    naamuú[ndeerelɛ]}  &   
                     \gloss{‘bring’}  &     &   
                     \vernacular{
                    naamuú[khalachilɪ]}  &   
                     \gloss{‘cut’}  &  \\

                     \vernacular{
                    naamuú[sitaachilɪ]}  &   
                     \gloss{‘accuse’}  &     &   
                     \vernacular{
                    naamuú[mboolitsilɪ]}  &   
                     \gloss{‘seduce’}  &  \\

                     \vernacular{
                    naamuú[mbohololelɛ]}  &   
                     \gloss{‘untie’}  &     &     &     &  \\
\end{tabular}
%\caption{\nocaption}
     
\begin{tabular}{llllll}  
  \multicolumn{5}{l}{
                     \vernacular{(326) /H/
                    V-Initial + OP + OP
                    } \gloss{‘s/he
                    will...him/her for me’} } &  \\
\multicolumn{5}{l}{ } &  \\

                     \vernacular{
                    naamuú[nzirilɪ]}  &   
                     \gloss{‘kill’}  &     &   
                     \vernacular{
                    naamuú[nzechitsilɪ]}  &   
                     \gloss{‘admire’}  &  \\

                     \vernacular{
                    naamuú[{\downstep}nzísíáchilɪ]}  &   
                     \gloss{‘smack’}  &     &   
                     \vernacular{
                    naamuú[nzononyinyilɪ]}  &   
                     \gloss{‘spoil’}  &  \\

                     \vernacular{
                    naamuú[nzabukhanyinyilɪ]}  &   
                     \gloss{‘separate’}  &     &     &     &  \\
\end{tabular}
%\caption{\nocaption}
     
\begin{tabular}{llllll}  
  \multicolumn{5}{l}{
                     \vernacular{(327) /Ø/
                    C-Initial + OP + OP
                    } \gloss{‘s/he
                    will...him/her for me’} } &  \\
\multicolumn{5}{l}{ } &  \\

                     \vernacular{
                    naamuú[{\downstep}zíílɪ]}  &   
                     \gloss{‘go for’}  &     &   
                     \vernacular{
                    naamuú[{\downstep}ndéshélɛ]}  &   
                     \gloss{‘leave’}  &  \\

                     \vernacular{
                    naamuú[{\downstep}nóóndelɛ]}  &   
                     \gloss{‘follow’}  &     &   
                     \vernacular{
                    naamuú[{\downstep}ngúlíshili]}  &   
                     \gloss{‘name’}  &  \\

                     \vernacular{
                    naamuú[{\downstep}ndákhúulilɪ]}  &   
                     \gloss{‘release’}  &     &   
                     \vernacular{
                    naamuú[{\downstep}séébúlilɪ]}  &   
                     \gloss{‘say bye to’}  &  \\

                     \vernacular{
                    naamuú[{\downstep}hóómbélitsɪ]}  &   
                     \gloss{‘comfort’}  &     &   
                     \vernacular{
                    naamuú[{\downstep}síínjílitsilɪ]}  &   
                     \gloss{
                    ‘make...stand’}  &  \\
\end{tabular}
%\caption{\nocaption}
     
\begin{tabular}{llllll}  
  \multicolumn{5}{l}{
                     \vernacular{(328) /Ø/
                    V-Initial + OP + OP
                    } \gloss{‘s/he
                    will...him/her \ob mu-\cb  / it
                    } } &  \\
\multicolumn{5}{l}{ } &  \\

                     \vernacular{
                    naamuú[{\downstep}nzéyélɛ]}  &   
                     \gloss{‘wipe’}  &     &   
                     \vernacular{
                    naakuú[{\downstep}nzáshítsilɪ]}  &   
                     \gloss{‘light’}  &  \\

                     \vernacular{
                    naabuú[{\downstep}nzílúulilɪ]}  &   
                     \gloss{‘winnow’}  &     &   
                     \vernacular{
                    naakuú[{\downstep}nzéléelitsilɪ]}  &   
                     \gloss{‘hang’}  &  \\
\end{tabular}
%\caption{\nocaption}
     
\begin{tabular}{lll}  
  \multicolumn{2}{l}{
                     \vernacular{(329) /H/
                    C-Initial Phrase-Medial} \gloss{‘s/he will...the
                    boy \ob mú{\downstep}yáyi\cb  /} } &  \\
\multicolumn{2}{l}{
                     \gloss{someone
                    \ob muundu\cb ’} } &  \\

                     \vernacular{naa[rɛ́]
                    {\downstep}mú{\downstep}yáyi/muundu}  &   
                     \gloss{‘bury’}  &  \\

                     \vernacular{naa[bechɛ́]
                    {\downstep}mú{\downstep}yáyi/muundu}  &   
                     \gloss{‘shave’}  &  \\

                     \vernacular{naa[leerɛ́]
                    {\downstep}mú{\downstep}yáyi/muundu}  &   
                     \gloss{‘bring’}  &  \\

                     \vernacular{naa[khalachɛ́]
                    {\downstep}mú{\downstep}yáyi/muundu}  &   
                     \gloss{‘cut’}  &  \\

                     \vernacular{naa[sitaáchɛ]
                    mú{\downstep}yáyi/muundu}  &   
                     \gloss{‘accuse’}  &  \\

                     \vernacular{naa[boolitsɪ́]
                    {\downstep}mú{\downstep}yáyi/muundu}  &   
                     \gloss{‘seduce’}  &  \\

                     \vernacular{naa[khong’oóndɛ]
                    mú{\downstep}yáyi/muundu}  &   
                     \gloss{‘knock’}  &  \\

                     \vernacular{naa[boholólɛ]
                    mú{\downstep}yáyi/muundu}  &   
                     \gloss{‘untie’}  &  \\

                     \vernacular{naa[boyong’ánɛ]
                    mú{\downstep}yáyi/muundu}  &   
                     \gloss{‘go around’}  &  \\

                     \vernacular{
                    naa[ling(ak)anyínyɪ]
                    mú{\downstep}yáyi/muundu}  &   
                     \gloss{‘bend’}  &  \\
\end{tabular}
%\caption{\nocaption}
     
\begin{tabular}{lll}  
  \multicolumn{2}{l}{
                     \vernacular{(330) /Ø/
                    C-Initial Phrase-Medial} \gloss{‘s/he will...the
                    boy \ob mú{\downstep}yáyi\cb  /} } &  \\
\multicolumn{2}{l}{
                     \gloss{someone
                    \ob muundu\cb ’} } &  \\

                     \vernacular{naa[tsí]
                    {\downstep}mú{\downstep}yáyi/muundu}  &   
                     \gloss{‘go for’}  &  \\

                     \vernacular{naa[leshɛ́]
                    {\downstep}mú{\downstep}yáyi/muundu}  &   
                     \gloss{‘leave’}  &  \\

                     \vernacular{naa[loondɛ́]
                    {\downstep}mú{\downstep}yáyi/muundu}  &   
                     \gloss{‘follow’}  &  \\

                     \vernacular{naa[kulishɪ́]
                    {\downstep}mú{\downstep}yáyi/muundu}  &   
                     \gloss{‘name’}  &  \\

                     \vernacular{naa[lakhuúlɪ]
                    mú{\downstep}yáyi/muundu}  &   
                     \gloss{‘release’}  &  \\

                     \vernacular{naa[seebulɪ́]
                    {\downstep}mú{\downstep}yáyi/muundu}  &   
                     \gloss{‘say bye to’}  &  \\

                     \vernacular{naa[kalushítsɪ]
                    mú{\downstep}yáyi/muundu}  &   
                     \gloss{‘return’}  &  \\

                     \vernacular{naa[reebɛréebɛ]
                    mú{\downstep}yáyi/muundu}  &   
                     \gloss{‘ask (iter)’}  &  \\

                     \vernacular{
                    naa[kalukhányinyɪ] mú{\downstep}yáyi/muundu}  &   
                     \gloss{
                    ‘turn...over’}  &  \\
\end{tabular}
%\caption{\nocaption}
     
\begin{tabular}{lll}  
  \multicolumn{2}{l}{
                     \vernacular{(331) /H/
                    C-Initial +OP Phrase-Medial} \gloss{‘s/he will...the
                    boy \ob mú{\downstep}yáyi\cb  /} } &  \\
\multicolumn{2}{l}{
                     \gloss{someone \ob muundu\cb 
                    for him/her’} } &  \\

                     \vernacular{naamu[réelɛ]
                    mú{\downstep}yáyi/muundu}  &   
                     \gloss{‘bury’}  &  \\

                     \vernacular{naamu[béchelɛ]
                    mú{\downstep}yáyi/muundu}  &   
                     \gloss{‘shave’}  &  \\

                     \vernacular{naamu[léerelɛ]
                    mú{\downstep}yáyi/muundu}  &   
                     \gloss{‘bring’}  &  \\

                     \vernacular{naamu[khálachilɪ]
                    mú{\downstep}yáyi/muundu}  &   
                     \gloss{‘cut’}  &  \\

                     \vernacular{naamu[sítaachilɪ]
                    mú{\downstep}yáyi/muundu}  &   
                     \gloss{‘accuse’}  &  \\

                     \vernacular{naamu[bóolitsilɪ]
                    mú{\downstep}yáyi/muundu}  &   
                     \gloss{‘seduce’}  &  \\

                     \vernacular{
                    naamu[khóng’oondelɛ]
                    mú{\downstep}yáyi/muundu}  &   
                     \gloss{‘knock’}  &  \\

                     \vernacular{naamu[bóhololelɛ]
                    mú{\downstep}yáyi/muundu}  &   
                     \gloss{‘untie’}  &  \\

                     \vernacular{
                    naamu[bóyong’anilɪ]
                    mú{\downstep}yáyi/muundu}  &   
                     \gloss{‘go around’}  &  \\

                     \vernacular{
                    naamu[língakanyinyilɪ]
                    mú{\downstep}yáyi/muundu}  &   
                     \gloss{‘bend’}  &  \\
\end{tabular}
%\caption{\nocaption}
     
\begin{tabular}{lll}  
  \multicolumn{2}{l}{
                     \vernacular{(332) /Ø/
                    C-Initial +OP Phrase-Medial} \gloss{‘s/he will...the
                    boy \ob mú{\downstep}yáyi\cb  /} } &  \\
\multicolumn{2}{l}{
                     \gloss{someone \ob muundu\cb 
                    for him/her’} } &  \\

                     \vernacular{naamu[tsiílɪ́]
                    {\downstep}mú{\downstep}yáyi/muundu}  &   
                     \gloss{‘go for’}  &  \\

                     \vernacular{naamu[leshélɛ́]
                    {\downstep}mú{\downstep}yáyi/muundu}  &   
                     \gloss{‘leave’}  &  \\

                     \vernacular{naamu[loóndélɛ]
                    mú{\downstep}yáyi/muundu}  &   
                     \gloss{‘follow’}  &  \\

                     \vernacular{naamu[kulíshílɪ]
                    mú{\downstep}yáyi/muundu}  &   
                     \gloss{‘name’}  &  \\

                     \vernacular{naamu[lakhúulilɪ]
                    mú{\downstep}yáyi/muundu}  &   
                     \gloss{‘release’}  &  \\

                     \vernacular{naamu[seébúlilɪ]
                    mú{\downstep}yáyi/muundu}  &   
                     \gloss{‘say bye to’}  &  \\

                     \vernacular{
                    naamu[kalúshítsilɪ]
                    mú{\downstep}yáyi/muundu}  &   
                     \gloss{‘return’}  &  \\

                     \vernacular{
                    naamu[reébɛ́reebelɛ]
                    mú{\downstep}yáyi/muundu}  &   
                     \gloss{‘ask (iter)’}  &  \\

                     \vernacular{
                    naamu[kalúkhányinyilɪ]
                    mú{\downstep}yáyi/muundu}  &   
                     \gloss{
                    ‘turn...over’}  &  \\
\end{tabular}
%\caption{\nocaption}
     
\begin{tabular}{lll}  
  \multicolumn{2}{l}{
                     \vernacular{(333) /H/
                    C-Initial +OP + OP
                    } \gloss{‘s/he will...the
                    boy \ob mú{\downstep}yáyi\cb  /} } &  \\
\multicolumn{2}{l}{
                     \gloss{someone \ob muundu\cb 
                    for him/her for me’} } &  \\

                     \vernacular{naamuú[ndeelɛ]
                    mú{\downstep}yáyi/muundu}  &   
                     \gloss{‘bury’}  &  \\

                     \vernacular{naamuú[mbechelɛ]
                    mú{\downstep}yáyi/muundu}  &   
                     \gloss{‘shave’}  &  \\

                     \vernacular{naamuú[ndeerelɛ]
                    mú{\downstep}yáyi/muundu}  &   
                     \gloss{‘bring’}  &  \\

                     \vernacular{
                    naamuú[khalachilɪ] mú{\downstep}yáyi/muundu}  &   
                     \gloss{‘cut’}  &  \\

                     \vernacular{
                    naamuú[sitaachilɪ] mú{\downstep}yáyi/muundu}  &   
                     \gloss{‘accuse’}  &  \\

                     \vernacular{
                    naamuú[mboolitsilɪ]
                    mú{\downstep}yáyi/muundu}  &   
                     \gloss{‘seduce’}  &  \\

                     \vernacular{
                    naamuú[mbohololelɛ]
                    mú{\downstep}yáyi/muundu}  &   
                     \gloss{‘untie’}  &  \\
\end{tabular}
%\caption{\nocaption}
     
\begin{tabular}{lll}  
  \multicolumn{2}{l}{
                     \vernacular{(334) /Ø/
                    C-Initial +OP + OP
                    } \gloss{‘s/he will...the
                    boy \ob mú{\downstep}yáyi\cb  /} } &  \\
\multicolumn{2}{l}{
                     \gloss{someone \ob muundu\cb 
                    for him/her for me’} } &  \\

                     \vernacular{
                    naamuú[{\downstep}nzíílɪ́]
                    {\downstep}mú{\downstep}yáyi/muundu}  &   
                     \gloss{‘go for’}  &  \\

                     \vernacular{
                    naamuú[{\downstep}ndéshélɛ́]
                    {\downstep}mú{\downstep}yáyi/muundu}  &   
                     \gloss{‘leave’}  &  \\

                     \vernacular{
                    naamuú[{\downstep}nóóndélɛ]
                    mú{\downstep}yáyi/muundu}  &   
                     \gloss{‘follow’}  &  \\

                     \vernacular{
                    naamuú[{\downstep}ngúlíshílɪ]
                    mú{\downstep}yáyi/muundu}  &   
                     \gloss{‘name’}  &  \\

                     \vernacular{
                    naamuú[{\downstep}ndákhúulilɪ]
                    mú{\downstep}yáyi/muundu}  &   
                     \gloss{‘release’}  &  \\

                     \vernacular{
                    naamuú[{\downstep}séébúlilɪ]
                    mú{\downstep}yáyi/muundu}  &   
                     \gloss{‘say bye to’}  &  \\
\end{tabular}
%\caption{\nocaption}
    

\subsection{Crastinal Future Negative: Pattern
              3}\label{sec:sCrastFutNeg}


\begin{tabular}{llllll}  
  \multicolumn{5}{l}{
                     \vernacular{(335) /H/
                    C-Initial} \gloss{‘s/he will
                    not...’} } &  \\
\multicolumn{5}{l}{ } &  \\

                     \vernacular{naa[rɛ́]
                    {\downstep}tá}  &   
                     \gloss{‘bury’}  &     &   
                     \vernacular{naa[ng’wí]
                    {\downstep}tá}  &   
                     \gloss{‘drink’}  &  \\

                     \vernacular{naa[khwí]
                    {\downstep}tá}  &   
                     \gloss{‘eat’}  &     &   
                     \vernacular{naa[lí]
                    {\downstep}tá}  &   
                     \gloss{‘pay dowry’}  &  \\

                     \vernacular{naa[lumɪ́]
                    {\downstep}tá}  &   
                     \gloss{‘bite’}  &     &   
                     \vernacular{naa[bechɛ́]
                    {\downstep}tá}  &   
                     \gloss{‘shave’}  &  \\

                     \vernacular{naa[teeshɛ́]
                    {\downstep}tá}  &   
                     \gloss{‘cook’}  &     &   
                     \vernacular{naa[leerɛ́]
                    {\downstep}tá}  &   
                     \gloss{‘bring’}  &  \\

                     \vernacular{naa[khalachɛ́]
                    {\downstep}tá}  &   
                     \gloss{‘cut’}  &     &   
                     \vernacular{naa[kalaánjɛ]
                    tá}  &   
                     \gloss{‘fry’}  &  \\

                     \vernacular{naa[sitaáchɛ]
                    tá}  &   
                     \gloss{‘accuse’}  &     &   
                     \vernacular{naa[boolitsɪ́]
                    {\downstep}tá}  &   
                     \gloss{‘seduce’}  &  \\

                     \vernacular{naa[saanditsɪ́]
                    {\downstep}tá}  &   
                     \gloss{‘thank’}  &     &   
                     \vernacular{naa[khong’oóndɛ]
                    tá}  &   
                     \gloss{‘knock’}  &  \\

                     \vernacular{naa[boholólɛ]
                    tá}  &   
                     \gloss{‘untie’}  &     &   
                     \vernacular{naa[boyong’ánɛ]
                    tá}  &   
                     \gloss{‘go around’}  &  \\

                     \vernacular{
                    naa[ng’ong’oólitsɪ] tá}  &   
                     \gloss{‘tease’}  &     &   
                     \vernacular{
                    naa[ling(ak)anyínyɪ] tá}  &   
                     \gloss{‘crumple’}  &  \\
\end{tabular}
%\caption{\nocaption}
     
\begin{tabular}{llllll}  
  \multicolumn{5}{l}{
                     \vernacular{(336) /Ø/
                    C-Initial} \gloss{‘s/he will
                    not...’} } &  \\
\multicolumn{5}{l}{ } &  \\

                     \vernacular{naa[tsí]
                    {\downstep}tá}  &   
                     \gloss{‘go’}  &     &   
                     \vernacular{naa[kwí]
                    {\downstep}tá}  &   
                     \gloss{‘fall’}  &  \\

                     \vernacular{naa[leshɛ́]
                    {\downstep}tá}  &   
                     \gloss{‘leave’}  &     &   
                     \vernacular{naa[reebɛ́]
                    {\downstep}tá}  &   
                     \gloss{‘ask’}  &  \\

                     \vernacular{naa[loondɛ́]
                    {\downstep}tá}  &   
                     \gloss{‘follow’}  &     &   
                     \vernacular{naa[kumilɪ́]
                    {\downstep}tá}  &   
                     \gloss{‘hold’}  &  \\

                     \vernacular{naa[kulishɪ́]
                    {\downstep}tá}  &   
                     \gloss{‘name’}  &     &   
                     \vernacular{naa[homoólɛ]
                    tá}  &   
                     \gloss{‘massage’}  &  \\

                     \vernacular{naa[lakhuúlɪ]
                    tá}  &   
                     \gloss{‘release’}  &     &   
                     \vernacular{naa[seebulɪ́]
                    {\downstep}tá}  &   
                     \gloss{‘say bye’}  &  \\

                     \vernacular{naa[hoombelítsɪ]
                    tá}  &   
                     \gloss{‘comfort’}  &     &   
                     \vernacular{naa[kalushítsɪ]
                    tá}  &   
                     \gloss{‘return’}  &  \\

                     \vernacular{naa[siinjilítsɪ]
                    tá}  &   
                     \gloss{‘make stand’}  &     &   
                     \vernacular{naa[reebaréebɛ]
                    tá}  &   
                     \gloss{‘ask (iter)’}  &  \\

                     \vernacular{
                    naa[kalukhányínyɪ] tá}  &   
                     \gloss{‘turn over’}  &     &   
                     \vernacular{
                    naa[sebulúkhányinyɪ] tá}  &   
                     \gloss{‘scatter’}  &  \\
\end{tabular}
%\caption{\nocaption}
     
\begin{tabular}{llllll}  
  \multicolumn{5}{l}{
                     \vernacular{(337) /H/
                    C-Initial + OP} \gloss{‘s/he will
                    not...him/her’} } &  \\
\multicolumn{5}{l}{ } &  \\

                     \vernacular{naamu[rɛ́]
                    {\downstep}tá}  &   
                     \gloss{‘bury’}  &     &   
                     \vernacular{naamu[béchɛ]
                    tá}  &   
                     \gloss{‘shave’}  &  \\

                     \vernacular{naamu[léerɛ]
                    tá}  &   
                     \gloss{‘bring’}  &     &   
                     \vernacular{naamu[khálachɛ]
                    tá}  &   
                     \gloss{‘cut’}  &  \\

                     \vernacular{naamu[sítaachɛ]
                    tá}  &   
                     \gloss{‘accuse’}  &     &   
                     \vernacular{naamu[bóolitsɪ]
                    tá}  &   
                     \gloss{‘seduce’}  &  \\

                     \vernacular{
                    naamu[khóng’oondɛ] tá}  &   
                     \gloss{‘knock’}  &     &   
                     \vernacular{naamu[bóhololɛ]
                    tá}  &   
                     \gloss{‘untie’}  &  \\

                     \vernacular{naamu[bóyong’anɛ]
                    tá}  &   
                     \gloss{‘go around’}  &     &   
                     \vernacular{
                    naamu[ng’óng’oolitsɪ] tá}  &   
                     \gloss{‘tease’}  &  \\

                     \vernacular{
                    naamu[língakanyinyɪ] tá}  &   
                     \gloss{‘bend’}  &     &     &     &  \\
\end{tabular}
%\caption{\nocaption}
     
\begin{tabular}{llllll}  
  \multicolumn{5}{l}{
                     \vernacular{(338) /Ø/
                    C-Initial + OP} \gloss{‘s/he will
                    not...him/her \ob mu-\cb  / them
                    } } &  \\
\multicolumn{5}{l}{ } &  \\

                     \vernacular{naamu[tsí]
                    {\downstep}tá}  &   
                     \gloss{‘go for’}  &  \\

                     \vernacular{naamu[leshɛ́]
                    {\downstep}tá}  &   
                     \gloss{‘leave’}  &  \\

                     \vernacular{naamu[loóndɛ]
                    tá}  &   
                     \gloss{‘follow’}  &  \\

                     \vernacular{naamu[kulíshɪ]
                    tá}  &   
                     \gloss{‘name’}  &  \\

                     \vernacular{naamu[lakhúulɪ]
                    tá}  &   
                     \gloss{‘release’}  &  \\

                     \vernacular{naamu[seébúlɪ]
                    tá}  &   
                     \gloss{‘say bye to’}  &  \\

                     \vernacular{
                    naamu[hoómbélitsɪ] tá}  &   
                     \gloss{‘comfort’}  &  \\

                     \vernacular{
                    naamu[kalúshítsɪ] tá}  &   
                     \gloss{‘return’}  &  \\

                     \vernacular{
                    naamu[siínjílitsɪ] tá}  &   
                     \gloss{
                    ‘make...stand’}  &  \\

                     \vernacular{
                    naamu[reébɛ́reebɛ] tá}  &   
                     \gloss{‘ask (iter)’}  &  \\

                     \vernacular{
                    naamu[kalúkhányinyɪ] tá}  &   
                     \gloss{
                    ‘turn...over’}  &  \\

                     \vernacular{
                    naabi[sebúlúkhanyinyɪ] tá}  &   
                     \gloss{‘scatter’}  &  \\
\end{tabular}
%\caption{\nocaption}
     
\begin{tabular}{llllll}  
  \multicolumn{5}{l}{
                     \vernacular{(339) /H/
                    C-Initial + OP + OP
                    } \gloss{‘s/he will
                    not...him/her for me’} } &  \\
\multicolumn{5}{l}{ } &  \\

                     \vernacular{naamuú[ndeelɛ]
                    tá}  &   
                     \gloss{‘bury’}  &     &   
                     \vernacular{naamuú[mbechelɛ]
                    tá}  &   
                     \gloss{‘shave’}  &  \\

                     \vernacular{naamuú[ndeerelɛ]
                    tá}  &   
                     \gloss{‘bring’}  &     &   
                     \vernacular{
                    naamuú[khalachilɪ] tá}  &   
                     \gloss{‘cut’}  &  \\

                     \vernacular{
                    naamuú[sitaachilɪ] tá}  &   
                     \gloss{‘accuse’}  &     &   
                     \vernacular{
                    naamuú[mboolitsilɪ] tá}  &   
                     \gloss{‘seduce’}  &  \\

                     \vernacular{
                    naamuú[mbohololelɛ] tá}  &   
                     \gloss{‘untie’}  &     &     &     &  \\
\end{tabular}
%\caption{\nocaption}
     
\begin{tabular}{llllll}  
  \multicolumn{5}{l}{
                     \vernacular{(340) /Ø/
                    C-Initial + OP + OP
                    } \gloss{‘s/he will
                    not...him/her for me’} } &  \\
\multicolumn{5}{l}{ } &  \\

                     \vernacular{naamuú[{\downstep}zíílɪ́]
                    {\downstep}tá}  &   
                     \gloss{‘go for’}  &     &   
                     \vernacular{
                    naamuú[{\downstep}ndéshélɛ́] {\downstep}tá}  &   
                     \gloss{‘leave’}  &  \\

                     \vernacular{
                    naamuú[{\downstep}nóóndélɛ] tá}  &   
                     \gloss{‘follow’}  &     &   
                     \vernacular{
                    naamuú[{\downstep}ngúlíshíli] tá}  &   
                     \gloss{‘name’}  &  \\

                     \vernacular{
                    naamuú[{\downstep}ndákhúulilɪ] tá}  &   
                     \gloss{‘release’}  &     &   
                     \vernacular{
                    naamuú[{\downstep}séébúlilɪ] tá}  &   
                     \gloss{‘say bye to’}  &  \\

                     \vernacular{
                    naamuú[{\downstep}hóómbélitsɪ] tá}  &   
                     \gloss{‘comfort’}  &     &   
                     \vernacular{
                    naamuú[{\downstep}síínjílitsilɪ] tá}  &   
                     \gloss{
                    ‘make...stand’}  &  \\
\end{tabular}
%\caption{\nocaption}
     
\begin{tabular}{lll}  
  \multicolumn{2}{l}{
                     \vernacular{(341) /H/
                    C-Initial Phrase-Medial} \gloss{‘s/he will
                    not...the boy \ob mú{\downstep}yáyi\cb  /} } &  \\
\multicolumn{2}{l}{
                     \gloss{someone
                    \ob muundu\cb ’} } &  \\

                     \vernacular{naa[rɛ́]
                    {\downstep}mú{\downstep}yáyi/muundu tá}  &   
                     \gloss{‘bury’}  &  \\

                     \vernacular{naa[bechɛ́]
                    {\downstep}mú{\downstep}yáyi/muundu tá}  &   
                     \gloss{‘shave’}  &  \\

                     \vernacular{naa[leerɛ́]
                    {\downstep}mú{\downstep}yáyi/muundu tá}  &   
                     \gloss{‘bring’}  &  \\

                     \vernacular{naa[khalachɛ́]
                    {\downstep}mú{\downstep}yáyi/muundu tá}  &   
                     \gloss{‘cut’}  &  \\

                     \vernacular{naa[sitaáchɛ]
                    mú{\downstep}yáyi/muundu tá}  &   
                     \gloss{‘accuse’}  &  \\

                     \vernacular{naa[boolitsɪ́]
                    {\downstep}mú{\downstep}yáyi/muundu tá}  &   
                     \gloss{‘seduce’}  &  \\

                     \vernacular{naa[khong’oóndɛ]
                    mú{\downstep}yáyi/muundu tá}  &   
                     \gloss{‘knock’}  &  \\

                     \vernacular{naa[boholólɛ]
                    mú{\downstep}yáyi/muundu tá}  &   
                     \gloss{‘untie’}  &  \\

                     \vernacular{naa[boyong’ánɛ]
                    mú{\downstep}yáyi/muundu tá}  &   
                     \gloss{‘go around’}  &  \\

                     \vernacular{
                    naa[ling(ak)anyínyɪ] mú{\downstep}yáyi/muundu
                    tá}  &   
                     \gloss{‘bend’}  &  \\
\end{tabular}
%\caption{\nocaption}
     
\begin{tabular}{lll}  
  \multicolumn{2}{l}{
                     \vernacular{(342) /Ø/
                    C-Initial Phrase-Medial} \gloss{‘s/he will
                    not...the boy \ob mú{\downstep}yáyi\cb  /} } &  \\
\multicolumn{2}{l}{
                     \gloss{someone
                    \ob muundu\cb ’} } &  \\

                     \vernacular{naa[tsí]
                    {\downstep}mú{\downstep}yáyi/muundu tá}  &   
                     \gloss{‘go for’}  &  \\

                     \vernacular{naa[leshɛ́]
                    {\downstep}mú{\downstep}yáyi/muundu tá}  &   
                     \gloss{‘leave’}  &  \\

                     \vernacular{naa[loondɛ́]
                    {\downstep}mú{\downstep}yáyi/muundu tá}  &   
                     \gloss{‘follow’}  &  \\

                     \vernacular{naa[kulishɪ́]
                    {\downstep}mú{\downstep}yáyi/muundu tá}  &   
                     \gloss{‘name’}  &  \\

                     \vernacular{naa[lakhuúlɪ]
                    mú{\downstep}yáyi/muundu tá}  &   
                     \gloss{‘release’}  &  \\

                     \vernacular{naa[seebulɪ́]
                    {\downstep}mú{\downstep}yáyi/muundu tá}  &   
                     \gloss{‘say bye to’}  &  \\

                     \vernacular{naa[kalushítsɪ]
                    mú{\downstep}yáyi/muundu tá}  &   
                     \gloss{‘return’}  &  \\

                     \vernacular{naa[reebɛréebɛ]
                    mú{\downstep}yáyi/muundu tá}  &   
                     \gloss{‘ask (iter)’}  &  \\

                     \vernacular{
                    naa[kalukhányínyɪ] mú{\downstep}yáyi/muundu
                    tá}  &   
                     \gloss{
                    ‘turn...over’}  &  \\
\end{tabular}
%\caption{\nocaption}
     
\begin{tabular}{lll}  
  \multicolumn{2}{l}{
                     \vernacular{(343) /H/
                    C-Initial +OP Phrase-Medial} \gloss{‘s/he will
                    not...the boy \ob mú{\downstep}yáyi\cb  /} } &  \\
\multicolumn{2}{l}{
                     \gloss{someone \ob muundu\cb 
                    for him/her’} } &  \\

                     \vernacular{naamu[réelɛ]
                    mú{\downstep}yáyi/muundu tá}  &   
                     \gloss{‘bury’}  &  \\

                     \vernacular{naamu[béchelɛ]
                    mú{\downstep}yáyi/muundu tá}  &   
                     \gloss{‘shave’}  &  \\

                     \vernacular{naamu[léerelɛ]
                    mú{\downstep}yáyi/muundu tá}  &   
                     \gloss{‘bring’}  &  \\

                     \vernacular{naamu[khálachilɪ]
                    mú{\downstep}yáyi/muundu tá}  &   
                     \gloss{‘cut’}  &  \\

                     \vernacular{naamu[sítaachilɪ]
                    mú{\downstep}yáyi/muundu tá}  &   
                     \gloss{‘accuse’}  &  \\

                     \vernacular{naamu[bóolitsilɪ]
                    mú{\downstep}yáyi/muundu tá}  &   
                     \gloss{‘seduce’}  &  \\

                     \vernacular{
                    naamu[khóng’oondelɛ] mú{\downstep}yáyi/muundu
                    tá}  &   
                     \gloss{‘knock’}  &  \\

                     \vernacular{naamu[bóhololelɛ]
                    mú{\downstep}yáyi/muundu tá}  &   
                     \gloss{‘untie’}  &  \\

                     \vernacular{
                    naamu[bóyong’anilɪ] mú{\downstep}yáyi/muundu
                    tá}  &   
                     \gloss{‘go around’}  &  \\

                     \vernacular{
                    naamu[língakanyinyilɪ] mú{\downstep}yáyi/muundu
                    tá}  &   
                     \gloss{‘bend’}  &  \\
\end{tabular}
%\caption{\nocaption}
     
\begin{tabular}{lll}  
  \multicolumn{2}{l}{
                     \vernacular{(344) /Ø/
                    C-Initial +OP Phrase-Medial} \gloss{‘s/he will
                    not...the boy \ob mú{\downstep}yáyi\cb  /} } &  \\
\multicolumn{2}{l}{
                     \gloss{someone \ob muundu\cb 
                    for him/her’} } &  \\

                     \vernacular{naamu[tsiílɪ́]
                    {\downstep}mú{\downstep}yáyi/muundu tá}  &   
                     \gloss{‘go for’}  &  \\

                     \vernacular{naamu[leshélɛ́]
                    {\downstep}mú{\downstep}yáyi/muundu tá}  &   
                     \gloss{‘leave’}  &  \\

                     \vernacular{naamu[loóndélɛ]
                    mú{\downstep}yáyi/muundu tá}  &   
                     \gloss{‘follow’}  &  \\

                     \vernacular{naamu[kulíshílɪ]
                    mú{\downstep}yáyi/muundu tá}  &   
                     \gloss{‘name’}  &  \\

                     \vernacular{naamu[lakhúulilɪ]
                    mú{\downstep}yáyi/muundu tá}  &   
                     \gloss{‘release’}  &  \\

                     \vernacular{naamu[seébúlilɪ]
                    mú{\downstep}yáyi/muundu tá}  &   
                     \gloss{‘say bye to’}  &  \\

                     \vernacular{
                    naamu[kalúshítsilɪ] mú{\downstep}yáyi/muundu
                    tá}  &   
                     \gloss{‘return’}  &  \\

                     \vernacular{
                    naamu[reébɛ́reebelɛ] mú{\downstep}yáyi/muundu
                    tá}  &   
                     \gloss{‘ask (iter)’}  &  \\

                     \vernacular{
                    naamu[kalúkhányinyilɪ] mú{\downstep}yáyi/muundu
                    tá}  &   
                     \gloss{
                    ‘turn...over’}  &  \\
\end{tabular}
%\caption{\nocaption}
     
\begin{tabular}{lll}  
  \multicolumn{2}{l}{
                     \vernacular{(345) /H/
                    C-Initial +OP + OP
                    } \gloss{‘s/he will
                    not...the boy \ob mú{\downstep}yáyi\cb  /} } &  \\
\multicolumn{2}{l}{
                     \gloss{someone \ob muundu\cb 
                    for him/her for me’} } &  \\

                     \vernacular{naamuú[ndeelɛ]
                    mú{\downstep}yáyi/muundu tá}  &   
                     \gloss{‘bury’}  &  \\

                     \vernacular{naamuú[mbechelɛ]
                    mú{\downstep}yáyi/muundu tá}  &   
                     \gloss{‘shave’}  &  \\

                     \vernacular{naamuú[ndeerelɛ]
                    mú{\downstep}yáyi/muundu tá}  &   
                     \gloss{‘bring’}  &  \\

                     \vernacular{
                    naamuú[khalachilɪ] mú{\downstep}yáyi/muundu
                    tá}  &   
                     \gloss{‘cut’}  &  \\

                     \vernacular{
                    naamuú[sitaachilɪ] mú{\downstep}yáyi/muundu
                    tá}  &   
                     \gloss{‘accuse’}  &  \\

                     \vernacular{
                    naamuú[mboolitsilɪ] mú{\downstep}yáyi/muundu
                    tá}  &   
                     \gloss{‘seduce’}  &  \\

                     \vernacular{
                    naamuú[mbohololelɛ] mú{\downstep}yáyi/muundu
                    tá}  &   
                     \gloss{‘untie’}  &  \\
\end{tabular}
%\caption{\nocaption}
     
\begin{tabular}{lll}  
  \multicolumn{2}{l}{
                     \vernacular{(346) /Ø/
                    C-Initial +OP + OP
                    } \gloss{‘s/he will
                    not...the boy \ob mú{\downstep}yáyi\cb  /} } &  \\
\multicolumn{2}{l}{
                     \gloss{someone \ob muundu\cb 
                    for him/her for me’} } &  \\

                     \vernacular{
                    naamuú[{\downstep}nzíílɪ́] {\downstep}mú{\downstep}yáyi/muundu
                    tá}  &   
                     \gloss{‘go for’}  &  \\

                     \vernacular{
                    naamuú[{\downstep}ndéshélɛ́] {\downstep}mú{\downstep}yáyi/muundu
                    tá}  &   
                     \gloss{‘leave’}  &  \\

                     \vernacular{
                    naamuú[{\downstep}nóóndélɛ] mú{\downstep}yáyi/muundu
                    tá}  &   
                     \gloss{‘follow’}  &  \\

                     \vernacular{
                    naamuú[{\downstep}ngúlíshílɪ] mú{\downstep}yáyi/muundu
                    tá}  &   
                     \gloss{‘name’}  &  \\

                     \vernacular{
                    naamuú[{\downstep}ndákhúulilɪ] mú{\downstep}yáyi/muundu
                    tá}  &   
                     \gloss{‘release’}  &  \\

                     \vernacular{
                    naamuú[{\downstep}séébúlilɪ] mú{\downstep}yáyi/muundu
                    tá}  &   
                     \gloss{‘say bye to’}  &  \\
\end{tabular}
%\caption{\nocaption}
    

\subsection{Imperative
              }\label{sec:sImpPl}


\begin{tabular}{llllll}  
  \multicolumn{5}{l}{
                     \vernacular{(347) /H/
                    C-Initial} \gloss{‘...!’} } &  \\
\multicolumn{5}{l}{ } &  \\

                     \vernacular{[ri]}  &   
                     \gloss{‘bury’}  &     &   
                     \vernacular{[ng’wi]}  &   
                     \gloss{‘drink’}  &  \\

                     \vernacular{[khwi]}  &   
                     \gloss{‘eat’}  &     &   
                     \vernacular{[li]}  &   
                     \gloss{‘pay dowry’}  &  \\

                     \vernacular{[lumi]}  &   
                     \gloss{‘bite’}  &     &   
                     \vernacular{[bechi]}  &   
                     \gloss{‘shave’}  &  \\

                     \vernacular{
                    [teeshí]}  &   
                     \gloss{‘cook’}  &     &   
                     \vernacular{
                    [leerí]}  &   
                     \gloss{‘bring’}  &  \\

                     \vernacular{
                    [khalachí]}  &   
                     \gloss{‘cut’}  &     &   
                     \vernacular{
                    [kalaanjí]}  &   
                     \gloss{‘fry’}  &  \\

                     \vernacular{
                    [sitaachí]}  &   
                     \gloss{‘accuse’}  &     &   
                     \vernacular{
                    [boolitsí]}  &   
                     \gloss{‘seduce’}  &  \\

                     \vernacular{
                    [saanditsí]}  &   
                     \gloss{‘thank’}  &     &   
                     \vernacular{
                    [khong’oondí]}  &   
                     \gloss{‘knock’}  &  \\

                     \vernacular{
                    [bohololí]}  &   
                     \gloss{‘untie’}  &     &   
                     \vernacular{
                    [boyong’aní]}  &   
                     \gloss{‘go around’}  &  \\

                     \vernacular{
                    [ng’ong’oolitsí]}  &   
                     \gloss{‘tease’}  &     &   
                     \vernacular{
                    [lingakanyinyí]}  &   
                     \gloss{‘crumple’}  &  \\
\end{tabular}
%\caption{\nocaption}
     
\begin{tabular}{llllll}  
  \multicolumn{5}{l}{
                     \vernacular{(348) /H/
                    V-Initial} \gloss{‘...!’} } &  \\
\multicolumn{5}{l}{ } &  \\

                     \vernacular{[yiri]}  &   
                     \gloss{‘kill’}  &     &   
                     \vernacular{
                    [yikoó{\downstep}mbí]}  &   
                     \gloss{‘admire’}  &  \\

                     \vernacular{
                    [yisiá{\downstep}chí]}  &   
                     \gloss{‘smack’}  &     &   
                     \vernacular{
                    [yikobó{\downstep}lí]}  &   
                     \gloss{‘belch’}  &  \\

                     \vernacular{
                    [yononyinyí]}  &   
                     \gloss{‘spoil’}  &     &   
                     \vernacular{
                    [yabukhanyinyí]}  &   
                     \gloss{‘separate’}  &  \\
\end{tabular}
%\caption{\nocaption}
     
\begin{tabular}{llllll}  
  \multicolumn{5}{l}{
                     \vernacular{(349) /Ø/
                    C-Initial} \gloss{‘...!’} } &  \\
\multicolumn{5}{l}{ } &  \\

                     \vernacular{[tsí]}  &   
                     \gloss{‘go’}  &     &   
                     \vernacular{[kwí]}  &   
                     \gloss{‘fall’}  &  \\

                     \vernacular{
                    [lé{\downstep}shí]}  &   
                     \gloss{‘leave’}  &     &   
                     \vernacular{
                    [réé{\downstep}bí]}  &   
                     \gloss{‘ask’}  &  \\

                     \vernacular{
                    [lóó{\downstep}ndí]}  &   
                     \gloss{‘follow’}  &     &   
                     \vernacular{
                    [kúmí{\downstep}lí]}  &   
                     \gloss{‘hold’}  &  \\

                     \vernacular{
                    [kúlí{\downstep}shí]}  &   
                     \gloss{‘name’}  &     &   
                     \vernacular{
                    [hómóó{\downstep}lí]}  &   
                     \gloss{‘massage’}  &  \\

                     \vernacular{
                    [lákhúú{\downstep}lí]}  &   
                     \gloss{‘release’}  &     &   
                     \vernacular{
                    [séébú{\downstep}lí]}  &   
                     \gloss{‘say bye’}  &  \\

                     \vernacular{
                    [hóómbélí{\downstep}tsí]}  &   
                     \gloss{‘comfort’}  &     &   
                     \vernacular{
                    [kálúshí{\downstep}tsí]}  &   
                     \gloss{‘return’}  &  \\

                     \vernacular{
                    [síínjílí{\downstep}tsí]}  &   
                     \gloss{‘make stand’}  &     &   
                     \vernacular{
                    [réébáréé{\downstep}bí]}  &   
                     \gloss{‘ask (iter)’}  &  \\

                     \vernacular{
                    [kálúkhányí{\downstep}nyí]}  &   
                     \gloss{‘turn over’}  &     &   
                     \vernacular{
                    [sébúlúkhányí{\downstep}nyí]}  &   
                     \gloss{‘scatter’}  &  \\
\end{tabular}
%\caption{\nocaption}
     
\begin{tabular}{llllll}  
  \multicolumn{5}{l}{
                     \vernacular{(350) /Ø/
                    V-Initial} \gloss{‘...!’} } &  \\
\multicolumn{5}{l}{ } &  \\

                     \vernacular{
                    [yé{\downstep}nyí]}  &   
                     \gloss{‘want’}  &     &   
                     \vernacular{
                    [yéyé{\downstep}lí]}  &   
                     \gloss{‘wipe for’}  &  \\

                     \vernacular{
                    [yílúú{\downstep}lí]}  &   
                     \gloss{‘winnow’}  &     &   
                     \vernacular{
                    [yámbákhá{\downstep}ní]}  &   
                     \gloss{‘refuse’}  &  \\

                     \vernacular{
                    [yéléélí{\downstep}tsí]}  &   
                     \gloss{‘hang up’}  &     &     &     &  \\
\end{tabular}
%\caption{\nocaption}
     
\begin{tabular}{llllll}  
  \multicolumn{5}{l}{
                     \vernacular{(351) /H/
                    C-Initial + OP} \gloss{
                    ‘...him/her!’} } &  \\
\multicolumn{5}{l}{ } &  \\

                     \vernacular{mu[rí]}  &   
                     \gloss{‘bury’}  &     &   
                     \vernacular{
                    mu[bé{\downstep}chí]}  &   
                     \gloss{‘shave’}  &  \\

                     \vernacular{
                    mu[leé{\downstep}rí]}  &   
                     \gloss{‘bring’}  &     &   
                     \vernacular{
                    mu[khá{\downstep}láchí]}  &   
                     \gloss{‘cut’}  &  \\

                     \vernacular{
                    mu[sí{\downstep}tááchí]}  &   
                     \gloss{‘accuse’}  &     &   
                     \vernacular{
                    mu[boó{\downstep}lítsí]}  &   
                     \gloss{‘seduce’}  &  \\

                     \vernacular{
                    mu[khó{\downstep}ng’óóndí]}  &   
                     \gloss{‘knock’}  &     &   
                     \vernacular{
                    mu[bó{\downstep}hólólí]}  &   
                     \gloss{‘untie’}  &  \\

                     \vernacular{
                    mu[bó{\downstep}yóng’ání]}  &   
                     \gloss{‘go around’}  &     &   
                     \vernacular{
                    mu[ng’ó{\downstep}ng’óólítsí]}  &   
                     \gloss{‘tease’}  &  \\

                     \vernacular{
                    mu[lí{\downstep}ngákányínyí]}  &   
                     \gloss{‘bend’}  &     &     &     &  \\
\end{tabular}
%\caption{\nocaption}
     
\begin{tabular}{llllll}  
  \multicolumn{5}{l}{
                     \vernacular{(352) /H/
                    V-Initial + OP} \gloss{
                    ‘...him/her!’} } &  \\
\multicolumn{5}{l}{ } &  \\

                     \vernacular{
                    mw[ií{\downstep}rí]}  &   
                     \gloss{‘kill’}  &     &   
                     \vernacular{
                    mw[ií{\downstep}kóómbí]}  &   
                     \gloss{‘admire’}  &  \\

                     \vernacular{
                    mw[ií{\downstep}síáchí]}  &   
                     \gloss{‘smack’}  &     &   
                     \vernacular{
                    mw[oó{\downstep}nónyínyí]}  &   
                     \gloss{‘spoil’}  &  \\

                     \vernacular{
                    mw[aá{\downstep}búkhányínyí]}  &   
                     \gloss{‘separate’}  &  \\
\end{tabular}
%\caption{\nocaption}
     
\begin{tabular}{llllll}  
  \multicolumn{5}{l}{
                     \vernacular{(353) /Ø/
                    C-Initial + OP} \gloss{‘...him/her \ob mu-\cb 
                    / them
                    } } &  \\
\multicolumn{5}{l}{ } &  \\

                     \vernacular{
                    mu[tsí]}  &   
                     \gloss{‘go for’}  &  \\

                     \vernacular{
                    mu[leshí]}  &   
                     \gloss{‘leave’}  &  \\

                     \vernacular{
                    mu[loó{\downstep}ndí]}  &   
                     \gloss{‘follow’}  &  \\

                     \vernacular{
                    mu[kulí{\downstep}shí]}  &   
                     \gloss{‘name’}  &  \\

                     \vernacular{
                    mu[lakhú{\downstep}úlí]}  &   
                     \gloss{‘release’}  &  \\

                     \vernacular{
                    mu[seé{\downstep}búlí]}  &   
                     \gloss{‘say bye to’}  &  \\

                     \vernacular{
                    mu[hoómbé{\downstep}lítsí]}  &   
                     \gloss{‘comfort’}  &  \\

                     \vernacular{
                    mu[kalú{\downstep}shítsí]}  &   
                     \gloss{‘return’}  &  \\

                     \vernacular{
                    mu[siínjí{\downstep}lítsí]}  &   
                     \gloss{
                    ‘make...stand’}  &  \\

                     \vernacular{
                    mu[reébí{\downstep}réébí]}  &   
                     \gloss{‘ask (iter)’}  &  \\

                     \vernacular{
                    mu[kalúkhá{\downstep}nyínyí]}  &   
                     \gloss{
                    ‘turn...over’}  &  \\

                     \vernacular{
                    bi[sebúlú{\downstep}khányínyí]}  &   
                     \gloss{‘scatter’}  &  \\
\end{tabular}
%\caption{\nocaption}
     
\begin{tabular}{llllll}  
  \multicolumn{5}{l}{
                     \vernacular{(354) /Ø/
                    V-Initial + OP} \gloss{‘...him/her \ob mw-\cb 
                    / it
                    } } &  \\
\multicolumn{5}{l}{ } &  \\

                     \vernacular{
                    mw[eenyí]}  &   
                     \gloss{‘want’}  &     &   
                     \vernacular{
                    mw[eeyé{\downstep}lí]}  &   
                     \gloss{‘wipe for’}  &  \\

                     \vernacular{
                    bw[iilú{\downstep}úlí]}  &   
                     \gloss{‘winnow’}  &     &   
                     \vernacular{
                    mw[aambá{\downstep}khání]}  &   
                     \gloss{‘refuse’}  &  \\

                     \vernacular{
                    mw[eelé{\downstep}élítsí]}  &   
                     \gloss{
                    ‘carry...hanging’}  &  \\
\end{tabular}
%\caption{\nocaption}
     
\begin{tabular}{llllll}  
  \multicolumn{5}{l}{
                     \vernacular{(355) /H/
                    C-Initial + OP
                    } \gloss{
                    ‘...me!’} } &  \\
\multicolumn{5}{l}{ } &  \\

                     \vernacular{[ríi]}  &   
                     \gloss{‘fear’}  &     &   
                     \vernacular{
                    [mbé{\downstep}chí]}  &   
                     \gloss{‘shave’}  &  \\

                     \vernacular{
                    ([ndeé{\downstep}rí]}  &   
                     \gloss{‘bring’)}  &     &   
                     \vernacular{
                    [khá{\downstep}láchí]}  &   
                     \gloss{‘cut’}  &  \\

                     \vernacular{
                    [sí{\downstep}tááchí]}  &   
                     \gloss{‘accuse’}  &     &   
                     \vernacular{
                    [mboó{\downstep}lítsí]}  &   
                     \gloss{‘seduce’}  &  \\

                     \vernacular{
                    [khó{\downstep}ng’óóndí]}  &   
                     \gloss{‘knock’}  &     &   
                     \vernacular{
                    [mbó{\downstep}hólólí]}  &   
                     \gloss{‘untie’}  &  \\

                     \vernacular{
                    [bó{\downstep}yóng’ání]}  &   
                     \gloss{‘go around’}  &     &   
                     \vernacular{
                    [ng’ó{\downstep}ng’óólítsí]}  &   
                     \gloss{‘tease’}  &  \\

                     \vernacular{
                    [ní{\downstep}ngákányínyí]}  &   
                     \gloss{‘bend’}  &  \\
\end{tabular}
%\caption{\nocaption}
     
\begin{tabular}{llllll}  
  \multicolumn{5}{l}{
                     \vernacular{(356) /H/
                    V-Initial + OP
                    } \gloss{
                    ‘...me!’} } &  \\
\multicolumn{5}{l}{ } &  \\

                     \vernacular{
                    [nzí{\downstep}rí]}  &   
                     \gloss{‘kill’}  &     &   
                     \vernacular{
                    [nzí{\downstep}kóó{\downstep}mbí]}  &   
                     \gloss{‘admire’}  &  \\

                     \vernacular{
                    [nzí{\downstep}síá{\downstep}chí]}  &   
                     \gloss{‘smack’}  &     &   
                     \vernacular{
                    [nzó{\downstep}nónyínyí]}  &   
                     \gloss{‘spoil’}  &  \\

                     \vernacular{
                    [nzá{\downstep}búkhányínyí]}  &   
                     \gloss{‘separate’}  &  \\
\end{tabular}
%\caption{\nocaption}
     
\begin{tabular}{llllll}  
  \multicolumn{5}{l}{
                     \vernacular{(357) /Ø/
                    C-Initial + OP
                    } \gloss{‘...me!’
                    } } &  \\
\multicolumn{5}{l}{ } &  \\

                     \vernacular{
                    [ndeshí]}  &   
                     \gloss{‘leave’}  &     &   
                     \vernacular{
                    [noó{\downstep}ndí]}  &   
                     \gloss{‘follow’}  &  \\

                     \vernacular{
                    [ngulí{\downstep}shí]}  &   
                     \gloss{‘name’}  &     &   
                     \vernacular{
                    [ndakhú{\downstep}úlí]}  &   
                     \gloss{‘release’}  &  \\

                     \vernacular{
                    [seé{\downstep}búlí]}  &   
                     \gloss{‘say bye to’}  &     &   
                     \vernacular{
                    [mboómbé{\downstep}lítsí]}  &   
                     \gloss{‘comfort’}  &  \\

                     \vernacular{
                    [siínjí{\downstep}lítsí]}  &   
                     \gloss{
                    ‘make..stand’}  &     &   
                     \vernacular{
                    [ndeébí{\downstep}ndéébí]}  &   
                     \gloss{‘ask (iter)’}  &  \\

                     \vernacular{
                    [ngalúkhá{\downstep}nyínyí]}  &   
                     \gloss{
                    ‘turn...over’}  &  \\
\end{tabular}
%\caption{\nocaption}
     
\begin{tabular}{llllll}  
  \multicolumn{5}{l}{
                     \vernacular{(358) /Ø/
                    V-Initial + OP
                    } \gloss{
                    ‘...me!’} } &  \\
\multicolumn{5}{l}{ } &  \\

                     \vernacular{
                    [nzenyí]}  &   
                     \gloss{‘want’}  &     &   
                     \vernacular{
                    [nzeyé{\downstep}lí]}  &   
                     \gloss{‘wipe for’}  &  \\

                     \vernacular{
                    [nyambá{\downstep}khání]}  &   
                     \gloss{‘refuse’}  &     &   
                     \vernacular{
                    [nzelé{\downstep}élítsí]}  &   
                     \gloss{
                    ‘carry...hanging’}  &  \\
\end{tabular}
%\caption{\nocaption}
     
\begin{tabular}{llllll}  
  \multicolumn{5}{l}{
                     \vernacular{(359) /H/
                    C-Initial + OP
                    } \gloss{
                    ‘...yourselves!’} } &  \\
\multicolumn{5}{l}{ } &  \\

                     \vernacular{yi[rí]}  &   
                     \gloss{‘bury’}  &     &   
                     \vernacular{
                    yi[bé{\downstep}chí]}  &   
                     \gloss{‘shave’}  &  \\

                     \vernacular{
                    yi[suú{\downstep}njí]}  &   
                     \gloss{‘hang’}  &     &   
                     \vernacular{
                    yi[khá{\downstep}láchí]}  &   
                     \gloss{‘cut’}  &  \\

                     \vernacular{
                    yi[sí{\downstep}tááchí]}  &   
                     \gloss{‘accuse’}  &     &   
                     \vernacular{
                    yi[saá{\downstep}ndítsí]}  &   
                     \gloss{‘thank’}  &  \\

                     \vernacular{
                    yi[khó{\downstep}ng’óóndí]}  &   
                     \gloss{‘knock’}  &     &   
                     \vernacular{
                    yi[bó{\downstep}hólólí]}  &   
                     \gloss{‘untie’}  &  \\
\end{tabular}
%\caption{\nocaption}
     
\begin{tabular}{llllll}  
  \multicolumn{5}{l}{
                     \vernacular{(360) /H/
                    V-Initial + OP
                    } \gloss{
                    ‘...yourselves!’} } &  \\
\multicolumn{5}{l}{ } &  \\

                     \vernacular{
                    yi[yí{\downstep}rí]}  &   
                     \gloss{‘kill’}  &     &   
                     \vernacular{
                    yi[yikoó{\downstep}mbí]}  &   
                     \gloss{‘admire’}  &  \\

                     \vernacular{
                    yi[yisiá{\downstep}chí]}  &   
                     \gloss{‘smack’}  &     &   
                     \vernacular{
                    yi[yó{\downstep}nónyínyí]}  &   
                     \gloss{‘spoil’}  &  \\

                     \vernacular{
                    yi[yá{\downstep}búkhányínyí]}  &   
                     \gloss{‘separate’}  &  \\
\end{tabular}
%\caption{\nocaption}
     
\begin{tabular}{llllll}  
  \multicolumn{5}{l}{
                     \vernacular{(361) /Ø/
                    C-Initial + OP
                    } \gloss{
                    ‘...yourselves!’} } &  \\
\multicolumn{5}{l}{ } &  \\

                     \vernacular{
                    yi[leshí]}  &   
                     \gloss{‘leave’}  &     &   
                     \vernacular{
                    yi[sií{\downstep}njí]}  &   
                     \gloss{‘bathe’}  &  \\

                     \vernacular{
                    yi[kulí{\downstep}shí]}  &   
                     \gloss{‘name’}  &     &   
                     \vernacular{
                    yi[naá{\downstep}búlí]}  &   
                     \gloss{‘undress’}  &  \\

                     \vernacular{
                    yi[lakhú{\downstep}úlí]}  &   
                     \gloss{‘release’}  &     &   
                     \vernacular{
                    yi[hoómbé{\downstep}lítsí]}  &   
                     \gloss{‘comfort’}  &  \\

                     \vernacular{
                    yi[siínjí{\downstep}lítsí]}  &   
                     \gloss{
                    ‘make...stand’}  &     &   
                     \vernacular{
                    yi[reébí{\downstep}réébí]}  &   
                     \gloss{‘ask (iter)’}  &  \\

                     \vernacular{
                    yi[kalúkhá{\downstep}nyínyí]}  &   
                     \gloss{
                    ‘turn...over’}  &  \\
\end{tabular}
%\caption{\nocaption}
     
\begin{tabular}{llllll}  
  \multicolumn{5}{l}{
                     \vernacular{(362) /Ø/
                    V-Initial + OP
                    } \gloss{
                    ‘...yourselves!’} } &  \\
\multicolumn{5}{l}{ } &  \\

                     \vernacular{
                    yi[yalí]}  &   
                     \gloss{‘spread’}  &     &   
                     \vernacular{
                    yi[yeyɛ́{\downstep}lí]}  &   
                     \gloss{‘wipe for’}  &  \\

                     \vernacular{
                    yi[yambá{\downstep}khání]}  &   
                     \gloss{‘refuse’}  &     &   
                     \vernacular{
                    yi[yelé{\downstep}élítsí]}  &   
                     \gloss{‘hang...up’}  &  \\
\end{tabular}
%\caption{\nocaption}
     
\begin{tabular}{llllll}  
  \multicolumn{5}{l}{
                     \vernacular{(363) /H/
                    C-Initial + OP + OP
                    } \gloss{‘...him/her for
                    me!’} } &  \\
\multicolumn{5}{l}{ } &  \\

                     \vernacular{
                    muú[{\downstep}ndéélí]}  &   
                     \gloss{‘bury’}  &     &   
                     \vernacular{
                    muú[{\downstep}mbéchélí]}  &   
                     \gloss{‘shave’}  &  \\

                     \vernacular{
                    muú[{\downstep}ndéérélí]}  &   
                     \gloss{‘bring’}  &     &   
                     \vernacular{
                    muú[{\downstep}kháláchílí]}  &   
                     \gloss{‘cut’}  &  \\

                     \vernacular{
                    muú[{\downstep}sítááchílí]}  &   
                     \gloss{‘accuse’}  &     &   
                     \vernacular{
                    muú[{\downstep}mbóólítsílí]}  &   
                     \gloss{‘seduce’}  &  \\

                     \vernacular{
                    muú[{\downstep}mbóhólólélí]}  &   
                     \gloss{‘untie’}  &  \\
\end{tabular}
%\caption{\nocaption}
     
\begin{tabular}{llllll}  
  \multicolumn{5}{l}{
                     \vernacular{(364) /H/
                    V-Initial + OP + OP
                    } \gloss{‘...him/her for
                    me!’} } &  \\
\multicolumn{5}{l}{ } &  \\

                     \vernacular{
                    muú[{\downstep}nzírílí]}  &   
                     \gloss{‘kill’}  &     &   
                     \vernacular{
                    muú[{\downstep}nzéchítsílí]}  &   
                     \gloss{‘admire’}  &  \\

                     \vernacular{
                    muú[{\downstep}nzísíá{\downstep}chílí]}  &   
                     \gloss{‘smack’}  &     &   
                     \vernacular{
                    muú[{\downstep}nzónónyínyílí]}  &   
                     \gloss{‘spoil’}  &  \\

                     \vernacular{
                    muú[{\downstep}nzábúkhányínyílí]}  &   
                     \gloss{‘separate’}  &  \\
\end{tabular}
%\caption{\nocaption}
     
\begin{tabular}{llllll}  
  \multicolumn{5}{l}{
                     \vernacular{(365) /Ø/
                    C-Initial + OP + OP
                    } \gloss{‘...him/her for
                    me!’} } &  \\
\multicolumn{5}{l}{ } &  \\

                     \vernacular{
                    muú[{\downstep}nzíí{\downstep}lí]}  &   
                     \gloss{‘go for’}  &     &   
                     \vernacular{
                    muú[{\downstep}ndéshé{\downstep}lí]}  &   
                     \gloss{‘leave’}  &  \\

                     \vernacular{
                    muú[{\downstep}nóó{\downstep}ndélí]}  &   
                     \gloss{‘follow’}  &     &   
                     \vernacular{
                    muú[{\downstep}ngúlí{\downstep}shílí]}  &   
                     \gloss{‘name’}  &  \\

                     \vernacular{
                    muú[{\downstep}ndákhú{\downstep}úlílí]}  &   
                     \gloss{‘release’}  &     &   
                     \vernacular{
                    muú[{\downstep}séébú{\downstep}lílí]}  &   
                     \gloss{‘say bye to’}  &  \\

                     \vernacular{
                    muú[{\downstep}mbóómbé{\downstep}lítsílí]}  &   
                     \gloss{‘comfort’}  &     &   
                     \vernacular{
                    muú[{\downstep}síínjí{\downstep}lítsílí]}  &   
                     \gloss{
                    ‘make...stand’}  &  \\
\end{tabular}
%\caption{\nocaption}
     
\begin{tabular}{llllll}  
  \multicolumn{5}{l}{
                     \vernacular{(366) /Ø/
                    V-Initial + OP + OP
                    } \gloss{‘...him/her \ob mu-\cb 
                    / it
                    } } &  \\
\multicolumn{5}{l}{ } &  \\

                     \vernacular{
                    muú[{\downstep}nzéyé{\downstep}lí]}  &   
                     \gloss{‘wipe’}  &     &   
                     \vernacular{
                    kuú[{\downstep}nzáshí{\downstep}tsílí]}  &   
                     \gloss{‘light’}  &  \\

                     \vernacular{
                    buú[{\downstep}nzílú{\downstep}úlílí]}  &   
                     \gloss{‘winnow’}  &     &   
                     \vernacular{
                    luú[{\downstep}nzítsú{\downstep}lítsílí]}  &   
                     \gloss{‘fill’}  &  \\

                     \vernacular{
                    kuú[{\downstep}nzélé{\downstep}élítsílí]}  &   
                     \gloss{‘hang’}  &     &     &     &  \\
\end{tabular}
%\caption{\nocaption}
     
\begin{tabular}{lll}  
  \multicolumn{2}{l}{
                     \vernacular{(367) /H/
                    C-Initial Phrase-Medial} \gloss{‘...the boy
                    \ob mú{\downstep}yáyi\cb  /} } &  \\
\multicolumn{2}{l}{
                     \gloss{someone
                    \ob muundu\cb !’} } &  \\

                     \vernacular{[ri]
                    mú{\downstep}yáyi/muundu}  &   
                     \gloss{‘bury’}  &  \\

                     \vernacular{[bechi]
                    mú{\downstep}yáyi/muundu}  &   
                     \gloss{‘shave’}  &  \\

                     \vernacular{[leeri]
                    mú{\downstep}yáyi/muundu}  &   
                     \gloss{‘bring’}  &  \\

                     \vernacular{[khalachi]
                    mú{\downstep}yáyi/muundu}  &   
                     \gloss{‘cut’}  &  \\

                     \vernacular{[sitaachi]
                    mú{\downstep}yáyi/muundu}  &   
                     \gloss{‘accuse’}  &  \\

                     \vernacular{[boolitsi]
                    mú{\downstep}yáyi/muundu}  &   
                     \gloss{‘seduce’}  &  \\

                     \vernacular{[khong’oondi]
                    mú{\downstep}yáyi/muundu}  &   
                     \gloss{‘knock’}  &  \\

                     \vernacular{[bohololi]
                    mú{\downstep}yáyi/muundu}  &   
                     \gloss{‘untie’}  &  \\

                     \vernacular{[boyong’ani]
                    mú{\downstep}yáyi/muundu}  &   
                     \gloss{‘go around’}  &  \\

                     \vernacular{[lingakanyinyi]
                    mú{\downstep}yáyi/muundu}  &   
                     \gloss{‘bend’}  &  \\
\end{tabular}
%\caption{\nocaption}
     
\begin{tabular}{lll}  
  \multicolumn{2}{l}{
                     \vernacular{(368) /Ø/
                    C-Initial Phrase-Medial} \gloss{‘...the boy
                    \ob mú{\downstep}yáyi\cb  /} } &  \\
\multicolumn{2}{l}{
                     \gloss{someone
                    \ob muundu\cb !’} } &  \\

                     \vernacular{[tsi]
                    mú{\downstep}yáyi/muundu}  &   
                     \gloss{‘go for’}  &  \\

                     \vernacular{[leshi]
                    mú{\downstep}yáyi/muundu}  &   
                     \gloss{‘leave’}  &  \\

                     \vernacular{[loondi]
                    mú{\downstep}yáyi/muundu}  &   
                     \gloss{‘follow’}  &  \\

                     \vernacular{[kulishi]
                    mú{\downstep}yáyi/muundu}  &   
                     \gloss{‘name’}  &  \\

                     \vernacular{[lakhuuli]
                    mú{\downstep}yáyi/muundu}  &   
                     \gloss{‘release’}  &  \\

                     \vernacular{[seebuli]
                    mú{\downstep}yáyi/muundu}  &   
                     \gloss{‘say bye to’}  &  \\

                     \vernacular{[kalushitsi]
                    mú{\downstep}yáyi/muundu}  &   
                     \gloss{‘return’}  &  \\

                     \vernacular{[reebireebi]
                    mú{\downstep}yáyi/muundu}  &   
                     \gloss{‘ask (iter)’}  &  \\

                     \vernacular{[kalukhanyinyi]
                    mú{\downstep}yáyi/muundu}  &   
                     \gloss{
                    ‘turn...over’}  &  \\
\end{tabular}
%\caption{\nocaption}
     
\begin{tabular}{lll}  
  \multicolumn{2}{l}{
                     \vernacular{(369) /H/
                    C-Initial +OP Phrase-Medial} \gloss{‘...the boy
                    \ob mú{\downstep}yáyi\cb  /} } &  \\
\multicolumn{2}{l}{
                     \gloss{someone \ob muundu\cb 
                    for him/her!’} } &  \\

                     \vernacular{mu[reé{\downstep}lí]
                    {\downstep}mú{\downstep}yáyi/muundu}  &   
                     \gloss{‘bury’}  &  \\

                     \vernacular{mu[bé{\downstep}chélí]
                    {\downstep}mú{\downstep}yáyi/muundu}  &   
                     \gloss{‘shave’}  &  \\

                     \vernacular{mu[leé{\downstep}rélí]
                    {\downstep}mú{\downstep}yáyi/muundu}  &   
                     \gloss{‘bring’}  &  \\

                     \vernacular{
                    mu[khá{\downstep}láchílí]
                    {\downstep}mú{\downstep}yáyi/muundu}  &   
                     \gloss{‘cut’}  &  \\

                     \vernacular{
                    mu[sí{\downstep}tááchílí]
                    {\downstep}mú{\downstep}yáyi/muundu}  &   
                     \gloss{‘accuse’}  &  \\

                     \vernacular{
                    mu[boó{\downstep}lítsílí]
                    {\downstep}mú{\downstep}yáyi/muundu}  &   
                     \gloss{‘seduce’}  &  \\

                     \vernacular{
                    mu[khó{\downstep}ng’óóndélí]
                    {\downstep}mú{\downstep}yáyi/muundu}  &   
                     \gloss{‘knock’}  &  \\

                     \vernacular{
                    mu[bó{\downstep}hólólélí]
                    {\downstep}mú{\downstep}yáyi/muundu}  &   
                     \gloss{‘untie’}  &  \\

                     \vernacular{
                    mu[bó{\downstep}yóng’ánílí]
                    {\downstep}mú{\downstep}yáyi/muundu}  &   
                     \gloss{‘go around’}  &  \\

                     \vernacular{
                    mu[ng’ó{\downstep}ng’óólítsílí]
                    {\downstep}mú{\downstep}yáyi/muundu}  &   
                     \gloss{‘tease’}  &  \\

                     \vernacular{
                    mu[lí{\downstep}ngákányínyílí]
                    {\downstep}mú{\downstep}yáyi/muundu}  &   
                     \gloss{‘bend’}  &  \\
\end{tabular}
%\caption{\nocaption}
     
\begin{tabular}{lll}  
  \multicolumn{2}{l}{
                     \vernacular{(370) /Ø/
                    C-Initial +OP Phrase-Medial} \gloss{‘...the boy
                    \ob mú{\downstep}yáyi\cb  /} } &  \\
\multicolumn{2}{l}{
                     \gloss{someone \ob muundu\cb 
                    for him/her!’} } &  \\

                     \vernacular{mu[tsií{\downstep}lí]
                    {\downstep}mú{\downstep}yáyi/muundu}  &   
                     \gloss{‘go for’}  &  \\

                     \vernacular{mu[leshé{\downstep}lí]
                    {\downstep}mú{\downstep}yáyi/muundu}  &   
                     \gloss{‘leave’}  &  \\

                     \vernacular{mu[loóndé{\downstep}lí]
                    {\downstep}mú{\downstep}yáyi/muundu}  &   
                     \gloss{‘follow’}  &  \\

                     \vernacular{mu[kulíshí{\downstep}lí]
                    {\downstep}mú{\downstep}yáyi/muundu}  &   
                     \gloss{‘name’}  &  \\

                     \vernacular{
                    mu[lakhú{\downstep}úlílí]
                    {\downstep}mú{\downstep}yáyi/muundu}  &   
                     \gloss{‘release’}  &  \\

                     \vernacular{mu[seébú{\downstep}lílí]
                    {\downstep}mú{\downstep}yáyi/muundu}  &   
                     \gloss{‘say bye to’}  &  \\

                     \vernacular{
                    mu[kalúshí{\downstep}tsílí]
                    {\downstep}mú{\downstep}yáyi/muundu}  &   
                     \gloss{‘return’}  &  \\

                     \vernacular{
                    mu[reébí{\downstep}réébélí]
                    {\downstep}mú{\downstep}yáyi/muundu}  &   
                     \gloss{‘ask (iter)’}  &  \\

                     \vernacular{
                    mu[kalúkhá{\downstep}nyínyílí]
                    {\downstep}mú{\downstep}yáyi/muundu}  &   
                     \gloss{
                    ‘turn...over’}  &  \\
\end{tabular}
%\caption{\nocaption}
     
\begin{tabular}{lll}  
  \multicolumn{2}{l}{
                     \vernacular{(371) /H/
                    C-Initial +OP + OP
                    } \gloss{‘...the boy
                    \ob mú{\downstep}yáyi\cb  /} } &  \\
\multicolumn{2}{l}{
                     \gloss{someone \ob muundu\cb 
                    for him/her for me!’} } &  \\

                     \vernacular{muú[{\downstep}ndéélí]
                    {\downstep}mú{\downstep}yáyi/muundu}  &   
                     \gloss{‘bury’}  &  \\

                     \vernacular{muú[{\downstep}mbéchélí]
                    {\downstep}mú{\downstep}yáyi/muundu}  &   
                     \gloss{‘shave’}  &  \\

                     \vernacular{
                    muú[{\downstep}ndéérélí]
                    {\downstep}mú{\downstep}yáyi/muundu}  &   
                     \gloss{‘bring’}  &  \\

                     \vernacular{
                    muú[{\downstep}sítááchílí]
                    {\downstep}mú{\downstep}yáyi/muundu}  &   
                     \gloss{‘accuse’}  &  \\

                     \vernacular{
                    muú[{\downstep}mbóólítsílí]
                    {\downstep}mú{\downstep}yáyi/muundu}  &   
                     \gloss{‘seduce’}  &  \\

                     \vernacular{
                    muú[{\downstep}mbóhólólélí]
                    {\downstep}mú{\downstep}yáyi/muundu}  &   
                     \gloss{‘untie’}  &  \\
\end{tabular}
%\caption{\nocaption}
     
\begin{tabular}{lll}  
  \multicolumn{2}{l}{
                     \vernacular{(372) /Ø/
                    C-Initial +OP + OP
                    } \gloss{‘...the boy
                    \ob mú{\downstep}yáyi\cb  /} } &  \\
\multicolumn{2}{l}{
                     \gloss{someone \ob muundu\cb 
                    for him/her for me!’} } &  \\

                     \vernacular{muú[{\downstep}nzíí{\downstep}lí]
                    {\downstep}mú{\downstep}yáyi/muundu}  &   
                     \gloss{‘go for’}  &  \\

                     \vernacular{
                    muú[{\downstep}ndéshé{\downstep}lí]
                    {\downstep}mú{\downstep}yáyi/muundu}  &   
                     \gloss{‘leave’}  &  \\

                     \vernacular{
                    muú[{\downstep}nóóndé{\downstep}lí]
                    {\downstep}mú{\downstep}yáyi/muundu}  &   
                     \gloss{‘follow’}  &  \\

                     \vernacular{
                    muú[{\downstep}ngúlíshí{\downstep}lí]
                    {\downstep}mú{\downstep}yáyi/muundu}  &   
                     \gloss{‘name’}  &  \\

                     \vernacular{
                    muú[{\downstep}ndákhú{\downstep}úlílí]
                    {\downstep}mú{\downstep}yáyi/muundu}  &   
                     \gloss{‘release’}  &  \\

                     \vernacular{
                    muú[{\downstep}séébú{\downstep}lílí]
                    {\downstep}mú{\downstep}yáyi/muundu}  &   
                     \gloss{‘say bye to’}  &  \\
\end{tabular}
%\caption{\nocaption}
    

\subsection{Imperative
              }\label{sec:sImpPlNeg}


\begin{tabular}{llllll}  
  \multicolumn{5}{l}{
                     \vernacular{(373) /H/
                    C-Initial} \gloss{‘do
                    not...!’} } &  \\
\multicolumn{5}{l}{ } &  \\

                     \vernacular{mukha[ri]
                    tá}  &   
                     \gloss{‘bury’}  &  \\

                     \vernacular{mukha[ng’wi]
                    tá}  &   
                     \gloss{‘drink’}  &  \\

                     \vernacular{mukha[khwi]
                    tá}  &   
                     \gloss{‘pay dowry’}  &  \\

                     \vernacular{mukha[li]
                    tá}  &   
                     \gloss{‘eat’}  &  \\

                     \vernacular{mukha[lumi]
                    tá}  &   
                     \gloss{‘bite’}  &  \\

                     \vernacular{mukha[bechi]
                    tá}  &   
                     \gloss{‘shave’}  &  \\

                     \vernacular{mukha[teeshi]
                    tá}  &   
                     \gloss{‘cook’}  &  \\

                     \vernacular{mukha[leeri]
                    tá}  &   
                     \gloss{‘bring’}  &  \\

                     \vernacular{mukha[khalachi]
                    tá}  &   
                     \gloss{‘cut’}  &  \\

                     \vernacular{mukha[kalaanji]
                    tá}  &   
                     \gloss{‘fry’}  &  \\

                     \vernacular{mukha[sitaachi]
                    {\downstep}tá}  &   
                     \gloss{‘accuse’}  &  \\

                     \vernacular{mukha[boolitsi]
                    {\downstep}tá}  &   
                     \gloss{‘seduce’}  &  \\

                     \vernacular{mukha[saanditsi]
                    {\downstep}tá}  &   
                     \gloss{‘thank’}  &  \\

                     \vernacular{mukha[khong’oondi]
                    {\downstep}tá}  &   
                     \gloss{‘knock’}  &  \\

                     \vernacular{mukha[bohololi]
                    {\downstep}tá}  &   
                     \gloss{‘untie’}  &  \\

                     \vernacular{mukha[boyong’ani]
                    {\downstep}tá}  &   
                     \gloss{‘go around’}  &  \\

                     \vernacular{
                    mukha[ng’ong’oolitsi] {\downstep}tá}  &   
                     \gloss{‘tease’}  &  \\

                     \vernacular{
                    mukha[lingakanyinyi] {\downstep}tá}  &   
                     \gloss{‘crumple’}  &  \\
\end{tabular}
%\caption{\nocaption}
     
\begin{tabular}{llllll}  
  \multicolumn{5}{l}{
                     \vernacular{(374) /Ø/
                    C-Initial} \gloss{‘do
                    not...!’} } &  \\
\multicolumn{5}{l}{ } &  \\

                     \vernacular{mukha[tsí]
                    {\downstep}tá}  &   
                     \gloss{‘go’}  &  \\

                     \vernacular{mukha[kwí]
                    {\downstep}tá}  &   
                     \gloss{‘fall’}  &  \\

                     \vernacular{mukha[leshí]
                    {\downstep}tá}  &   
                     \gloss{‘leave’}  &  \\

                     \vernacular{mukha[reébi]
                    tá}  &   
                     \gloss{‘ask’}  &  \\

                     \vernacular{mukha[loóndi]
                    tá}  &   
                     \gloss{‘follow’}  &  \\

                     \vernacular{mukha[kumíli]
                    tá}  &   
                     \gloss{‘hold’}  &  \\

                     \vernacular{mukha[kulíshi]
                    tá}  &   
                     \gloss{‘name’}  &  \\

                     \vernacular{mukha[homóoli]
                    tá}  &   
                     \gloss{‘massage’}  &  \\

                     \vernacular{mukha[lakhúuli]
                    tá}  &   
                     \gloss{‘release’}  &  \\

                     \vernacular{mukha[seébuli]
                    tá}  &   
                     \gloss{‘say bye’}  &  \\

                     \vernacular{
                    mukha[hoómbélitsi] tá}  &   
                     \gloss{‘comfort’}  &  \\

                     \vernacular{
                    mukha[kalúshítsi] tá}  &   
                     \gloss{‘return’}  &  \\

                     \vernacular{
                    mukha[siínjílitsi] tá}  &   
                     \gloss{‘make stand’}  &  \\

                     \vernacular{mukha[reébireebi]
                    tá}  &   
                     \gloss{‘turn over’}  &  \\

                     \vernacular{
                    mukha[kalúkhányinyi] tá}  &   
                     \gloss{‘turn over’}  &  \\

                     \vernacular{
                    mukha[sebúlúkhanyinyi] tá}  &   
                     \gloss{‘scatter’}  &  \\
\end{tabular}
%\caption{\nocaption}
     
\begin{tabular}{llllll}  
  \multicolumn{5}{l}{
                     \vernacular{(375) /H/
                    C-Initial + OP} \gloss{‘do
                    not...him/her!’} } &  \\
\multicolumn{5}{l}{ } &  \\

                     \vernacular{mukhamu[rí]
                    {\downstep}tá}  &   
                     \gloss{‘bury’}  &  \\

                     \vernacular{mukhamu[béchi]
                    tá}  &   
                     \gloss{‘shave’}  &  \\

                     \vernacular{mukhamu[léeri]
                    tá}  &   
                     \gloss{‘bring’}  &  \\

                     \vernacular{mukhamu[khálachi]
                    tá}  &   
                     \gloss{‘cut’}  &  \\

                     \vernacular{mukhamu[sítaachi]
                    tá}  &   
                     \gloss{‘accuse’}  &  \\

                     \vernacular{mukhamu[bóolitsi]
                    tá}  &   
                     \gloss{‘seduce’}  &  \\

                     \vernacular{
                    mukhamu[khóng’oondi] tá}  &   
                     \gloss{‘knock’}  &  \\

                     \vernacular{mukhamu[bóhololi]
                    tá}  &   
                     \gloss{‘untie’}  &  \\

                     \vernacular{
                    mukhamu[bóyong’ani] tá}  &   
                     \gloss{‘go around’}  &  \\

                     \vernacular{
                    mukhamu[ng’óng’oolitsi] tá}  &   
                     \gloss{‘tease’}  &  \\

                     \vernacular{
                    mukhamu[língakanyinyi] tá}  &   
                     \gloss{‘bend’}  &  \\
\end{tabular}
%\caption{\nocaption}
     
\begin{tabular}{llllll}  
  \multicolumn{5}{l}{
                     \vernacular{(376) /Ø/
                    C-Initial + OP} \gloss{‘do not...him/her
                    \ob mu-\cb  / them
                    } } &  \\
\multicolumn{5}{l}{ } &  \\

                     \vernacular{mukhamu[tsí]
                    {\downstep}tá}  &   
                     \gloss{‘go for’}  &  \\

                     \vernacular{mukhamu[leshí]
                    {\downstep}tá}  &   
                     \gloss{‘leave’}  &  \\

                     \vernacular{mukhamu[loóndi]
                    tá}  &   
                     \gloss{‘follow’}  &  \\

                     \vernacular{mukhamu[kulíshi]
                    tá}  &   
                     \gloss{‘name’}  &  \\

                     \vernacular{mukhamu[lakhúuli]
                    tá}  &   
                     \gloss{‘release’}  &  \\

                     \vernacular{mukhamu[seébúli]
                    tá}  &   
                     \gloss{‘say bye to’}  &  \\

                     \vernacular{
                    mukhamu[hoómbélitsi] tá}  &   
                     \gloss{‘comfort’}  &  \\

                     \vernacular{
                    mukhamu[kalúshítsi] tá}  &   
                     \gloss{‘return’}  &  \\

                     \vernacular{
                    mukhamu[siínjílitsi] tá}  &   
                     \gloss{
                    ‘make...stand’}  &  \\

                     \vernacular{
                    mukhamu[reébíreebi] tá}  &   
                     \gloss{‘ask (iter)’}  &  \\

                     \vernacular{
                    mukhamu[kalúkhányinyi] tá}  &   
                     \gloss{
                    ‘turn...over’}  &  \\

                     \vernacular{
                    mukhabi[sebúlúkhanyinyi] tá}  &   
                     \gloss{‘scatter’}  &  \\
\end{tabular}
%\caption{\nocaption}
     
\begin{tabular}{llllll}  
  \multicolumn{5}{l}{
                     \vernacular{(377) /H/
                    C-Initial + OP + OP
                    } \gloss{‘do not...him/her
                    for me!’} } &  \\
\multicolumn{5}{l}{ } &  \\

                     \vernacular{mukhamuú[ndeeli]
                    {\downstep}tá}  &   
                     \gloss{‘bury’}  &     &   
                     \vernacular{
                    mukhamuú[mbecheli] {\downstep}tá}  &   
                     \gloss{‘shave’}  &  \\

                     \vernacular{
                    mukhamuú[ndeereli] {\downstep}tá}  &   
                     \gloss{‘bring’}  &     &   
                     \vernacular{
                    mukhamuú[khalachili] {\downstep}tá}  &   
                     \gloss{‘cut’}  &  \\

                     \vernacular{
                    mukhamuú[sitaachili] {\downstep}tá}  &   
                     \gloss{‘accuse’}  &     &   
                     \vernacular{
                    mukhamuú[mboolitsili] {\downstep}tá}  &   
                     \gloss{‘seduce’}  &  \\

                     \vernacular{
                    mukhamuú[mbohololeli] {\downstep}tá}  &   
                     \gloss{‘untie’}  &     &     &     &  \\
\end{tabular}
%\caption{\nocaption}
     
\begin{tabular}{llllll}  
  \multicolumn{5}{l}{
                     \vernacular{(378) /Ø/
                    C-Initial + OP + OP
                    } \gloss{‘do not...him/her
                    for me!’} } &  \\
\multicolumn{5}{l}{ } &  \\

                     \vernacular{
                    mukhamuú[{\downstep}nzííli] tá}  &   
                     \gloss{‘go for’}  &  \\

                     \vernacular{
                    mukhamuú[{\downstep}ndéshéli] tá}  &   
                     \gloss{‘leave’}  &  \\

                     \vernacular{
                    mukhamuú[{\downstep}nóóndéli] tá}  &   
                     \gloss{‘follow’}  &  \\

                     \vernacular{
                    mukhamuú[{\downstep}ngúlíshíli] tá}  &   
                     \gloss{‘name’}  &  \\

                     \vernacular{
                    mukhamuú[{\downstep}ndákhúulili] tá}  &   
                     \gloss{‘release’}  &  \\

                     \vernacular{
                    mukhamuú[{\downstep}séébúlili] tá}  &   
                     \gloss{‘say bye to’}  &  \\

                     \vernacular{
                    mukhamuú[{\downstep}mbóómbélitsili] tá}  &   
                     \gloss{‘comfort’}  &  \\

                     \vernacular{
                    mukhamuú[{\downstep}síínjílitsili] tá}  &   
                     \gloss{
                    ‘make...stand’}  &  \\
\end{tabular}
%\caption{\nocaption}
     
\begin{tabular}{lll}  
  \multicolumn{2}{l}{
                     \vernacular{(379) /H/
                    C-Initial Phrase-Medial} \gloss{‘do not...the boy
                    \ob mú{\downstep}yáyi\cb  /} } &  \\
\multicolumn{2}{l}{
                     \gloss{someone
                    \ob muundu\cb !’} } &  \\

                     \vernacular{mukha[ri]
                    mú{\downstep}yáyi/muundu}  &   
                     \gloss{‘bury’}  &  \\

                     \vernacular{mukha[bechi]
                    mú{\downstep}yáyi/muundu tá}  &   
                     \gloss{‘shave’}  &  \\

                     \vernacular{mukha[leeri]
                    mú{\downstep}yáyi/muundu tá}  &   
                     \gloss{‘bring’}  &  \\

                     \vernacular{mukha[khalachi]
                    mú{\downstep}yáyi/muundu tá}  &   
                     \gloss{‘cut’}  &  \\

                     \vernacular{mukha[sitaachi]
                    mú{\downstep}yáyi/muundu tá}  &   
                     \gloss{‘accuse’}  &  \\

                     \vernacular{mukha[boolitsi]
                    mú{\downstep}yáyi/muundu tá}  &   
                     \gloss{‘seduce’}  &  \\

                     \vernacular{mukha[khong’oondi]
                    mú{\downstep}yáyi/muundu tá}  &   
                     \gloss{‘knock’}  &  \\

                     \vernacular{mukha[bohololi]
                    mú{\downstep}yáyi/muundu tá}  &   
                     \gloss{‘untie’}  &  \\

                     \vernacular{mukha[boyong’ani]
                    mú{\downstep}yáyi/muundu tá}  &   
                     \gloss{‘go around’}  &  \\
\end{tabular}
%\caption{\nocaption}
     
\begin{tabular}{lll}  
  \multicolumn{2}{l}{
                     \vernacular{(380) /Ø/
                    C-Initial Phrase-Medial} \gloss{‘do not...the boy
                    \ob mú{\downstep}yáyi\cb  /} } &  \\
\multicolumn{2}{l}{
                     \gloss{someone
                    \ob muundu\cb !’} } &  \\

                     \vernacular{mukha[tsí]
                    {\downstep}mú{\downstep}yáyi/muundu tá}  &   
                     \gloss{‘go for’}  &  \\

                     \vernacular{mukha[leshí]
                    {\downstep}mú{\downstep}yáyi/muundu tá}  &   
                     \gloss{‘leave’}  &  \\

                     \vernacular{mukha[loóndí]
                    {\downstep}mú{\downstep}yáyi/muundu tá}  &   
                     \gloss{‘follow’}  &  \\

                     \vernacular{mukha[kulíshí]
                    {\downstep}mú{\downstep}yáyi/muundu tá}  &   
                     \gloss{‘name’}  &  \\

                     \vernacular{mukha[lakhúuli]
                    mú{\downstep}yáyi/muundu tá}  &   
                     \gloss{‘release’}  &  \\

                     \vernacular{mukha[seébúli]
                    mú{\downstep}yáyi/muundu tá}  &   
                     \gloss{‘say bye to’}  &  \\

                     \vernacular{
                    mukha[kalúshítsi] mú{\downstep}yáyi/muundu
                    tá}  &   
                     \gloss{‘return’}  &  \\

                     \vernacular{
                    mukha[reébíreebi] mú{\downstep}yáyi/muundu
                    tá}  &   
                     \gloss{‘ask (iter)’}  &  \\

                     \vernacular{
                    mukha[kalúkhányinyi] mú{\downstep}yáyi/muundu
                    tá}  &   
                     \gloss{
                    ‘turn...over’}  &  \\
\end{tabular}
%\caption{\nocaption}
     
\begin{tabular}{lll}  
  \multicolumn{2}{l}{
                     \vernacular{(381) /H/
                    C-Initial +OP Phrase-Medial} \gloss{‘do not...the boy
                    \ob mú{\downstep}yáyi\cb  /} } &  \\
\multicolumn{2}{l}{
                     \gloss{someone \ob muundu\cb 
                    for him/her!’} } &  \\

                     \vernacular{mukhamu[réeli]
                    mú{\downstep}yáyi/muundu tá}  &   
                     \gloss{‘bury’}  &  \\

                     \vernacular{mukhamu[bécheli]
                    mú{\downstep}yáyi/muundu tá}  &   
                     \gloss{‘shave’}  &  \\

                     \vernacular{mukhamu[léereli]
                    mú{\downstep}yáyi/muundu tá}  &   
                     \gloss{‘bring’}  &  \\

                     \vernacular{
                    mukhamu[khálachili] mú{\downstep}yáyi/muundu
                    tá}  &   
                     \gloss{‘cut’}  &  \\

                     \vernacular{
                    mukhamu[sítaachili] mú{\downstep}yáyi/muundu
                    tá}  &   
                     \gloss{‘accuse’}  &  \\

                     \vernacular{
                    mukhamu[bóolitsili] mú{\downstep}yáyi/muundu
                    tá}  &   
                     \gloss{‘seduce’}  &  \\

                     \vernacular{
                    mukhamu[khóng’oondeli] mú{\downstep}yáyi/muundu
                    tá}  &   
                     \gloss{‘knock’}  &  \\

                     \vernacular{
                    mukhamu[bóhololeli] mú{\downstep}yáyi/muundu
                    tá}  &   
                     \gloss{‘untie’}  &  \\

                     \vernacular{
                    mukhamu[bóyong’anili] mú{\downstep}yáyi/muundu
                    tá}  &   
                     \gloss{‘go around’}  &  \\

                     \vernacular{
                    mukhamu[ng’óng’oolitsili] mú{\downstep}yáyi/muundu
                    tá}  &   
                     \gloss{‘tease’}  &  \\

                     \vernacular{
                    mukhamu[língakanyinyili] mú{\downstep}yáyi/muundu
                    tá}  &   
                     \gloss{‘bend’}  &  \\
\end{tabular}
%\caption{\nocaption}
     
\begin{tabular}{lll}  
  \multicolumn{2}{l}{
                     \vernacular{(382) /Ø/
                    C-Initial +OP Phrase-Medial} \gloss{‘do not...the boy
                    \ob mú{\downstep}yáyi\cb  /} } &  \\
\multicolumn{2}{l}{
                     \gloss{someone \ob muundu\cb 
                    for him/her!’} } &  \\

                     \vernacular{mukhamu[tsiílí]
                    {\downstep}mú{\downstep}yáyi/muundu tá}  &   
                     \gloss{‘grind’}  &  \\

                     \vernacular{mukhamu[leshélí]
                    {\downstep}mú{\downstep}yáyi/muundu tá}  &   
                     \gloss{‘leave’}  &  \\

                     \vernacular{
                    mukhamu[loóndéli] mú{\downstep}yáyi/muundu
                    tá}  &   
                     \gloss{‘follow’}  &  \\

                     \vernacular{
                    mukhamu[kulíshíli] mú{\downstep}yáyi/muundu
                    tá}  &   
                     \gloss{‘name’}  &  \\

                     \vernacular{
                    mukhamu[lakhúulili] mú{\downstep}yáyi/muundu
                    tá}  &   
                     \gloss{‘release’}  &  \\

                     \vernacular{
                    mukhamu[seébúlili] mú{\downstep}yáyi/muundu
                    tá}  &   
                     \gloss{‘say bye to’}  &  \\

                     \vernacular{
                    mukhamu[kalúshítsili] mú{\downstep}yáyi/muundu
                    tá}  &   
                     \gloss{‘return’}  &  \\

                     \vernacular{
                    mukhamu[reébíreebeli] mú{\downstep}yáyi/muundu
                    tá}  &   
                     \gloss{‘ask (iter)’}  &  \\

                     \vernacular{
                    mukhamu[kalúkhányinyili] mú{\downstep}yáyi/muundu
                    tá}  &   
                     \gloss{
                    ‘turn...over’}  &  \\
\end{tabular}
%\caption{\nocaption}
     
\begin{tabular}{lll}  
  \multicolumn{2}{l}{
                     \vernacular{(383) /H/
                    C-Initial +OP + OP
                    } \gloss{‘do not...the boy
                    \ob mú{\downstep}yáyi\cb  /} } &  \\
\multicolumn{2}{l}{
                     \gloss{someone \ob muundu\cb 
                    for him/her for me!’} } &  \\

                     \vernacular{mukhamuú[ndeeli]
                    mú{\downstep}yáyi/muundu tá}  &   
                     \gloss{‘bury’}  &  \\

                     \vernacular{
                    mukhamuú[mbecheli] mú{\downstep}yáyi/muundu
                    tá}  &   
                     \gloss{‘shave’}  &  \\

                     \vernacular{
                    mukhamuú[ndeereli] mú{\downstep}yáyi/muundu
                    tá}  &   
                     \gloss{‘bring’}  &  \\

                     \vernacular{
                    mukhamuú[sitaachili] mú{\downstep}yáyi/muundu
                    tá}  &   
                     \gloss{‘accuse’}  &  \\

                     \vernacular{
                    mukhamuú[mboolitsili] mú{\downstep}yáyi/muundu
                    tá}  &   
                     \gloss{‘seduce’}  &  \\

                     \vernacular{
                    mukhamuú[mbohololeli] mú{\downstep}yáyi/muundu
                    tá}  &   
                     \gloss{‘untie’}  &  \\
\end{tabular}
%\caption{\nocaption}
     
\begin{tabular}{lll}  
  \multicolumn{2}{l}{
                     \vernacular{(384) /Ø/
                    C-Initial +OP + OP
                    } \gloss{‘do not...the boy
                    \ob mú{\downstep}yáyi\cb  /} } &  \\
\multicolumn{2}{l}{
                     \gloss{someone \ob muundu\cb 
                    for him/her for me!’} } &  \\

                     \vernacular{
                    mukhamuú[{\downstep}nzíílí] {\downstep}mú{\downstep}yáyi/muundu
                    tá}  &   
                     \gloss{‘go for’}  &  \\

                     \vernacular{
                    mukhamuú[{\downstep}ndéshélí] {\downstep}mú{\downstep}yáyi/muundu
                    tá}  &   
                     \gloss{‘leave’}  &  \\

                     \vernacular{
                    mukhamuú[{\downstep}nóóndéli] mú{\downstep}yáyi/muundu
                    tá}  &   
                     \gloss{‘follow’}  &  \\

                     \vernacular{
                    mukhamuú[{\downstep}ngúlíshíli] mú{\downstep}yáyi/muundu
                    tá}  &   
                     \gloss{‘name’}  &  \\

                     \vernacular{
                    mukhamuú[{\downstep}ndákhúulili] mú{\downstep}yáyi/muundu
                    tá}  &   
                     \gloss{‘release’}  &  \\

                     \vernacular{
                    mukhamuú[{\downstep}séébúlili] mú{\downstep}yáyi/muundu
                    tá}  &   
                     \gloss{‘say bye to’}  &  \\
\end{tabular}
%\caption{\nocaption}
    

\subsection{Subjunctive: Pattern 3}\label{sec:sSubj}


\begin{tabular}{llllll}  
  \multicolumn{5}{l}{
                     \vernacular{(385) /H/
                    C-Initial} \gloss{‘let
                    him/her...’} } &  \\
\multicolumn{5}{l}{ } &  \\

                     \vernacular{a[rɛ́]}  &   
                     \gloss{‘bury’}  &     &   
                     \vernacular{
                    a[ng’wí]}  &   
                     \gloss{‘drink’}  &  \\

                     \vernacular{
                    a[khwí]}  &   
                     \gloss{‘eat’}  &     &   
                     \vernacular{a[lí]}  &   
                     \gloss{‘pay dowry’}  &  \\

                     \vernacular{
                    a[lumɪ́]}  &   
                     \gloss{‘bite’}  &     &   
                     \vernacular{
                    a[bechɛ́]}  &   
                     \gloss{‘shave’}  &  \\

                     \vernacular{
                    a[teeshɛ́]}  &   
                     \gloss{‘cook’}  &     &   
                     \vernacular{
                    a[leerɛ́]}  &   
                     \gloss{‘bring’}  &  \\

                     \vernacular{
                    a[khalachɛ́]}  &   
                     \gloss{‘cut’}  &     &   
                     \vernacular{
                    a[kalaánjɛ]}  &   
                     \gloss{‘fry’}  &  \\

                     \vernacular{
                    a[sitaáchɛ]}  &   
                     \gloss{‘accuse’}  &     &   
                     \vernacular{
                    a[boolitsɪ́]}  &   
                     \gloss{‘seduce’}  &  \\

                     \vernacular{
                    a[saanditsɪ́]}  &   
                     \gloss{‘thank’}  &     &   
                     \vernacular{
                    a[khong’oóndɛ́]}  &   
                     \gloss{‘knock’}  &  \\

                     \vernacular{
                    a[boholólɛ]}  &   
                     \gloss{‘untie’}  &     &   
                     \vernacular{
                    a[boyong’ánɛ]}  &   
                     \gloss{‘go around’}  &  \\

                     \vernacular{
                    a[ng’ong’oólitsɪ]}  &   
                     \gloss{‘tease’}  &     &   
                     \vernacular{
                    a[ling(ak)anyínyɪ]}  &   
                     \gloss{‘crumple’}  &  \\
\end{tabular}
%\caption{\nocaption}
     
\begin{tabular}{llllll}  
  \multicolumn{5}{l}{
                     \vernacular{(386) /H/
                    V-Initial} \gloss{‘let
                    him/her...’} } &  \\
\multicolumn{5}{l}{ } &  \\

                     \vernacular{y[irɪ]}  &   
                     \gloss{‘kill’}  &     &   
                     \vernacular{
                    y[ikóómbɛ]}  &   
                     \gloss{‘admire’}  &  \\

                     \vernacular{
                    y[isíáchɛ]}  &   
                     \gloss{‘smack’}  &     &   
                     \vernacular{
                    y[ikobólɛ]}  &   
                     \gloss{‘belch’}  &  \\

                     \vernacular{
                    y[ononyínyɪ]}  &   
                     \gloss{‘spoil’}  &     &   
                     \vernacular{
                    y[abukhányinyɪ]}  &   
                     \gloss{‘separate’}  &  \\
\end{tabular}
%\caption{\nocaption}
     
\begin{tabular}{llllll}  
  \multicolumn{5}{l}{
                     \vernacular{(387) /Ø/
                    C-Initial} \gloss{‘let
                    him/her...’} } &  \\
\multicolumn{5}{l}{ } &  \\

                     \vernacular{a[tsí]}  &   
                     \gloss{‘go’}  &     &   
                     \vernacular{a[kwí]}  &   
                     \gloss{‘fall’}  &  \\

                     \vernacular{
                    a[leshɛ́]}  &   
                     \gloss{‘leave’}  &     &   
                     \vernacular{
                    a[reebɛ́]}  &   
                     \gloss{‘ask’}  &  \\

                     \vernacular{
                    a[loondɛ́]}  &   
                     \gloss{‘follow’}  &     &   
                     \vernacular{
                    a[kumilɪ́]}  &   
                     \gloss{‘hold’}  &  \\

                     \vernacular{
                    a[kulishɪ́]}  &   
                     \gloss{‘name’}  &     &   
                     \vernacular{
                    a[homoólɛ]}  &   
                     \gloss{‘massage’}  &  \\

                     \vernacular{
                    a[lakhuúlɪ]}  &   
                     \gloss{‘release’}  &     &   
                     \vernacular{
                    a[seebulɪ́]}  &   
                     \gloss{‘say bye’}  &  \\

                     \vernacular{
                    a[hoombelítsɪ]}  &   
                     \gloss{‘comfort’}  &     &   
                     \vernacular{
                    a[kalushítsɪ]}  &   
                     \gloss{‘return’}  &  \\

                     \vernacular{
                    a[siinjilítsɪ]}  &   
                     \gloss{‘make stand’}  &     &   
                     \vernacular{
                    a[reebaréebɛ]}  &   
                     \gloss{‘ask (iter)’}  &  \\

                     \vernacular{
                    a[kalukhányinyɪ]}  &   
                     \gloss{‘turn over’}  &     &   
                     \vernacular{
                    a[sebulúkhányinyɪ]}  &   
                     \gloss{‘scatter’}  &  \\
\end{tabular}
%\caption{\nocaption}
     
\begin{tabular}{llllll}  
  \multicolumn{5}{l}{
                     \vernacular{(388) /Ø/
                    V-Initial} \gloss{‘let
                    him/her...’} } &  \\
\multicolumn{5}{l}{ } &  \\

                     \vernacular{
                    y[enyɛ́]}  &   
                     \gloss{‘want’}  &     &   
                     \vernacular{
                    y[eyelɛ́]}  &   
                     \gloss{‘wipe for’}  &  \\

                     \vernacular{
                    y[iluúlɪ]}  &   
                     \gloss{‘winnow’}  &     &   
                     \vernacular{
                    y[ambakhánɛ]}  &   
                     \gloss{‘refuse’}  &  \\

                     \vernacular{
                    y[eleélitsɪ]}  &   
                     \gloss{‘hang up’}  &     &     &     &  \\
\end{tabular}
%\caption{\nocaption}
     
\begin{tabular}{llllll}  
  \multicolumn{5}{l}{
                     \vernacular{(389) /H/
                    C-Initial + OP} \gloss{‘let
                    him/her...him/her’} } &  \\
\multicolumn{5}{l}{ } &  \\

                     \vernacular{
                    amu[rɛ́]}  &   
                     \gloss{‘bury’}  &     &   
                     \vernacular{
                    amu[béchɛ]}  &   
                     \gloss{‘shave’}  &  \\

                     \vernacular{
                    amu[léerɛ]}  &   
                     \gloss{‘bring’}  &     &   
                     \vernacular{
                    amu[khálachɛ]}  &   
                     \gloss{‘cut’}  &  \\

                     \vernacular{
                    amu[sítaachɛ]}  &   
                     \gloss{‘accuse’}  &     &   
                     \vernacular{
                    amu[bóolitsɪ]}  &   
                     \gloss{‘seduce’}  &  \\

                     \vernacular{
                    amu[khóng’oondɛ]}  &   
                     \gloss{‘knock’}  &     &   
                     \vernacular{
                    amu[bóhololɛ]}  &   
                     \gloss{‘untie’}  &  \\

                     \vernacular{
                    amu[bóyong’anɛ]}  &   
                     \gloss{‘go around’}  &     &   
                     \vernacular{
                    amu[ng’óng’oolitsɪ]}  &   
                     \gloss{‘tease’}  &  \\

                     \vernacular{
                    amu[língakanyinyɪ]}  &   
                     \gloss{‘bend’}  &     &     &     &  \\
\end{tabular}
%\caption{\nocaption}
     
\begin{tabular}{llllll}  
  \multicolumn{5}{l}{
                     \vernacular{(390) /H/
                    V-Initial + OP} \gloss{‘let
                    him/her...him/her’} } &  \\
\multicolumn{5}{l}{ } &  \\

                     \vernacular{
                    amw[iírɪ]}  &   
                     \gloss{‘kill’}  &     &   
                     \vernacular{
                    amw[ií{\downstep}kóómbɛ]}  &   
                     \gloss{‘admire’}  &  \\

                     \vernacular{
                    amw[ií{\downstep}síáchɛ]}  &   
                     \gloss{‘smack’}  &     &   
                     \vernacular{
                    amw[oónonyinyɪ]}  &   
                     \gloss{‘spoil’}  &  \\

                     \vernacular{
                    amw[aábukhanyinyɪ]}  &   
                     \gloss{‘separate’}  &  \\
\end{tabular}
%\caption{\nocaption}
     
\begin{tabular}{llllll}  
  \multicolumn{5}{l}{
                     \vernacular{(391) /Ø/
                    C-Initial + OP} \gloss{‘let
                    him/her...him/her \ob mu-\cb  / them
                    } } &  \\
\multicolumn{5}{l}{ } &  \\

                     \vernacular{
                    amu[tsí]}  &   
                     \gloss{‘go for’}  &  \\

                     \vernacular{
                    amu[leshɛ́]}  &   
                     \gloss{‘leave’}  &  \\

                     \vernacular{
                    amu[loóndɛ]}  &   
                     \gloss{‘follow’}  &  \\

                     \vernacular{
                    amu[kulíshɪ]}  &   
                     \gloss{‘name’}  &  \\

                     \vernacular{
                    amu[lakhúulɪ]}  &   
                     \gloss{‘release’}  &  \\

                     \vernacular{
                    amu[seébulɪ]}  &   
                     \gloss{‘say bye to’}  &  \\

                     \vernacular{
                    amu[hoómbélitsɪ]}  &   
                     \gloss{‘comfort’}  &  \\

                     \vernacular{
                    amu[kalúshitsɪ]}  &   
                     \gloss{‘return’}  &  \\

                     \vernacular{
                    amu[siínjílitsɪ]}  &   
                     \gloss{
                    ‘make...stand’}  &  \\

                     \vernacular{
                    amu[reébɛ́reebɛ]}  &   
                     \gloss{‘ask (iter)’}  &  \\

                     \vernacular{
                    amu[kalúkhányinyɪ]}  &   
                     \gloss{
                    ‘turn...over’}  &  \\

                     \vernacular{
                    abi[sebúlúkhanyinyɪ]}  &   
                     \gloss{‘scatter’}  &  \\
\end{tabular}
%\caption{\nocaption}
     
\begin{tabular}{llllll}  
  \multicolumn{5}{l}{
                     \vernacular{(392) /Ø/
                    V-Initial + OP} \gloss{‘let
                    him/her...him/her \ob mw-\cb  / it
                    } } &  \\
\multicolumn{5}{l}{ } &  \\

                     \vernacular{
                    amw[eenyɛ́]}  &   
                     \gloss{‘want’}  &     &   
                     \vernacular{
                    amw[eeyɛ́lɛ]}  &   
                     \gloss{‘wipe for’}  &  \\

                     \vernacular{
                    abw[iilúulɪ]}  &   
                     \gloss{‘winnow’}  &     &   
                     \vernacular{
                    amw[aambákhanɛ]}  &   
                     \gloss{‘refuse’}  &  \\

                     \vernacular{
                    amw[eeléelitsɪ]}  &   
                     \gloss{
                    ‘carry...hanging’}  &  \\
\end{tabular}
%\caption{\nocaption}
     
\begin{tabular}{llllll}  
  \multicolumn{5}{l}{
                     \vernacular{(393) /H/
                    C-Initial + OP
                    } \gloss{‘let
                    him/her...me’} } &  \\
\multicolumn{5}{l}{ } &  \\

                     \vernacular{aa[rí]}  &   
                     \gloss{‘fear’}  &     &   
                     \vernacular{
                    aa[mbéchɛ]}  &   
                     \gloss{‘shave’}  &  \\

                     \vernacular{
                    aa[ndéerɛ]}  &   
                     \gloss{‘bring’}  &     &   
                     \vernacular{
                    aa[khálachɛ]}  &   
                     \gloss{‘cut’}  &  \\

                     \vernacular{
                    aa[sítaachɛ]}  &   
                     \gloss{‘accuse’}  &     &   
                     \vernacular{
                    aa[mbóolitsɪ]}  &   
                     \gloss{‘seduce’}  &  \\

                     \vernacular{
                    aa[khóng’oondɛ]}  &   
                     \gloss{‘knock’}  &     &   
                     \vernacular{
                    aa[mbóhololɛ]}  &   
                     \gloss{‘untie’}  &  \\

                     \vernacular{
                    aa[mbóyong’anɛ]}  &   
                     \gloss{‘go around’}  &     &   
                     \vernacular{
                    aa[ng’óng’oolitsɪ]}  &   
                     \gloss{‘tease’}  &  \\

                     \vernacular{
                    aa[níngakanyinyɪ]}  &   
                     \gloss{‘bend’}  &  \\
\end{tabular}
%\caption{\nocaption}
     
\begin{tabular}{llllll}  
  \multicolumn{5}{l}{
                     \vernacular{(394) /H/
                    V-Initial + OP
                    } \gloss{‘let
                    him/her...me’} } &  \\
\multicolumn{5}{l}{ } &  \\

                     \vernacular{
                    aa[nzírɪ]}  &   
                     \gloss{‘kill’}  &     &   
                     \vernacular{
                    aa[nzí{\downstep}kóómbɛ]}  &   
                     \gloss{‘admire’}  &  \\

                     \vernacular{
                    aa[nzí{\downstep}síáchɛ]}  &   
                     \gloss{‘smack’}  &     &   
                     \vernacular{
                    aa[nzónonyinyɪ]}  &   
                     \gloss{‘spoil’}  &  \\

                     \vernacular{
                    aa[nzábukhanyinyɪ]}  &   
                     \gloss{‘separate’}  &  \\
\end{tabular}
%\caption{\nocaption}
     
\begin{tabular}{llllll}  
  \multicolumn{5}{l}{
                     \vernacular{(395) /Ø/
                    C-Initial + OP
                    } \gloss{‘let
                    him/her...me’} } &  \\
\multicolumn{5}{l}{ } &  \\

                     \vernacular{aa[sí]}  &   
                     \gloss{‘grind’}  &     &   
                     \vernacular{
                    aa[ndeshɛ́]}  &   
                     \gloss{‘leave’}  &  \\

                     \vernacular{
                    aa[noóndɛ]}  &   
                     \gloss{‘follow’}  &     &   
                     \vernacular{
                    aa[ngulíshɪ]}  &   
                     \gloss{‘name’}  &  \\

                     \vernacular{
                    aa[ndakhúulɪ]}  &   
                     \gloss{‘release’}  &     &   
                     \vernacular{
                    aa[seébulɪ]}  &   
                     \gloss{‘say bye to’}  &  \\

                     \vernacular{
                    aa[mboómbélitsɪ]}  &   
                     \gloss{‘comfort’}  &     &   
                     \vernacular{
                    aa[siínjílitsɪ]}  &   
                     \gloss{
                    ‘make..stand’}  &  \\

                     \vernacular{
                    aa[ndeébɛ́ndeebɛ]}  &   
                     \gloss{‘ask (iter)’}  &     &   
                     \vernacular{
                    aa[ngalúkhányinyɪ]}  &   
                     \gloss{
                    ‘turn...over’}  &  \\
\end{tabular}
%\caption{\nocaption}
     
\begin{tabular}{llllll}  
  \multicolumn{5}{l}{
                     \vernacular{(396) /Ø/
                    V-Initial + OP
                    } \gloss{‘let
                    him/her...me’} } &  \\
\multicolumn{5}{l}{ } &  \\

                     \vernacular{
                    aa[nzenyɛ́]}  &   
                     \gloss{‘want’}  &     &   
                     \vernacular{
                    aa[nzeyélɛ]}  &   
                     \gloss{‘wipe for’}  &  \\

                     \vernacular{
                    aa[nyambákhanɛ]}  &   
                     \gloss{‘refuse’}  &     &   
                     \vernacular{
                    aa[nzeléelitsɪ]}  &   
                     \gloss{
                    ‘carry...hanging’}  &  \\
\end{tabular}
%\caption{\nocaption}
     
\begin{tabular}{llllll}  
  \multicolumn{5}{l}{
                     \vernacular{(397) /H/
                    C-Initial + OP
                    } \gloss{‘let
                    him/her...yourself’} } &  \\
\multicolumn{5}{l}{ } &  \\

                     \vernacular{
                    yii[rɛ́]}  &   
                     \gloss{‘bury’}  &     &   
                     \vernacular{
                    yii[béchɛ]}  &   
                     \gloss{‘shave’}  &  \\

                     \vernacular{
                    yii[súunjɪ]}  &   
                     \gloss{‘hang’}  &     &   
                     \vernacular{
                    yii[khálachɛ]}  &   
                     \gloss{‘cut’}  &  \\

                     \vernacular{
                    yii[sítaachɛ]}  &   
                     \gloss{‘accuse’}  &     &   
                     \vernacular{
                    yii[sáanditsɪ]}  &   
                     \gloss{‘thank’}  &  \\

                     \vernacular{
                    yii[khóng’oondɛ]}  &   
                     \gloss{‘knock’}  &     &   
                     \vernacular{
                    yii[bóhololɛ]}  &   
                     \gloss{‘untie’}  &  \\
\end{tabular}
%\caption{\nocaption}
     
\begin{tabular}{llllll}  
  \multicolumn{5}{l}{
                     \vernacular{(398) /H/
                    V-Initial + OP
                    } \gloss{‘let
                    him/her...yourself’} } &  \\
\multicolumn{5}{l}{ } &  \\

                     \vernacular{
                    yii[yírɪ]}  &   
                     \gloss{‘kill’}  &     &   
                     \vernacular{
                    yii[yí{\downstep}kóómbɛ]}  &   
                     \gloss{‘admire’}  &  \\

                     \vernacular{
                    yii[yí{\downstep}síáchɛ]}  &   
                     \gloss{‘smack’}  &     &   
                     \vernacular{
                    yii[yónonyinyɪ]}  &   
                     \gloss{‘spoil’}  &  \\

                     \vernacular{
                    yii[yábukhanyinyɪ]}  &   
                     \gloss{‘separate’}  &  \\
\end{tabular}
%\caption{\nocaption}
     
\begin{tabular}{llllll}  
  \multicolumn{5}{l}{
                     \vernacular{(399) /Ø/
                    C-Initial + OP
                    } \gloss{‘let
                    him/her...yourself’} } &  \\
\multicolumn{5}{l}{ } &  \\

                     \vernacular{
                    yii[sí]}  &   
                     \gloss{‘grind’}  &     &   
                     \vernacular{
                    yii[leshɛ́]}  &   
                     \gloss{‘leave’}  &  \\

                     \vernacular{
                    yii[siínjɪ]}  &   
                     \gloss{‘bathe’}  &     &   
                     \vernacular{
                    yii[kulíshɪ]}  &   
                     \gloss{‘name’}  &  \\

                     \vernacular{
                    yii[naábulɪ]}  &   
                     \gloss{‘undress’}  &     &   
                     \vernacular{
                    yii[lakhúulɪ]}  &   
                     \gloss{‘release’}  &  \\

                     \vernacular{
                    yii[hoómbélitsɪ]}  &   
                     \gloss{‘comfort’}  &     &   
                     \vernacular{
                    yii[siínjílitsɪ]}  &   
                     \gloss{
                    ‘make...stand’}  &  \\

                     \vernacular{
                    yii[reébɛ́reebɛ]}  &   
                     \gloss{‘ask (iter)’}  &     &   
                     \vernacular{
                    yii[kalúkhányinyɪ]}  &   
                     \gloss{
                    ‘turn...over’}  &  \\
\end{tabular}
%\caption{\nocaption}
     
\begin{tabular}{llllll}  
  \multicolumn{5}{l}{
                     \vernacular{(400) /Ø/
                    V-Initial + OP
                    } \gloss{‘let
                    him/her...yourself’} } &  \\
\multicolumn{5}{l}{ } &  \\

                     \vernacular{
                    yii[yalɛ́]}  &   
                     \gloss{‘expose’}  &     &   
                     \vernacular{
                    yii[yeyélɛ]}  &   
                     \gloss{‘wipe for’}  &  \\

                     \vernacular{
                    yii[yambákhanɛ]}  &   
                     \gloss{‘refuse’}  &     &   
                     \vernacular{
                    yii[yeléelitsɪ]}  &   
                     \gloss{‘hang...up’}  &  \\
\end{tabular}
%\caption{\nocaption}
     
\begin{tabular}{llllll}  
  \multicolumn{5}{l}{
                     \vernacular{(401) /H/
                    C-Initial + OP + OP
                    } \gloss{‘let
                    him/her...him/her for me’} } &  \\
\multicolumn{5}{l}{ } &  \\

                     \vernacular{
                    amuú[ndeelɛ]}  &   
                     \gloss{‘bury’}  &     &   
                     \vernacular{
                    amuú[mbechelɛ]}  &   
                     \gloss{‘shave’}  &  \\

                     \vernacular{
                    amuú[ndeerelɛ]}  &   
                     \gloss{‘bring’}  &     &   
                     \vernacular{
                    amuú[khalachilɪ]}  &   
                     \gloss{‘cut’}  &  \\

                     \vernacular{
                    amuú[sitaachilɪ]}  &   
                     \gloss{‘accuse’}  &     &   
                     \vernacular{
                    amuú[mboolitsilɪ]}  &   
                     \gloss{‘seduce’}  &  \\

                     \vernacular{
                    amuú[mbohololelɛ]}  &   
                     \gloss{‘untie’}  &     &     &     &  \\
\end{tabular}
%\caption{\nocaption}
     
\begin{tabular}{llllll}  
  \multicolumn{5}{l}{
                     \vernacular{(402) /H/
                    V-Initial + OP + OP
                    } \gloss{‘let
                    him/her...him/her for me’} } &  \\
\multicolumn{5}{l}{ } &  \\

                     \vernacular{
                    amuú[nzirilɪ]}  &   
                     \gloss{‘kill’}  &     &   
                     \vernacular{
                    amuú[nzechitsilɪ]}  &   
                     \gloss{‘admire’}  &  \\

                     \vernacular{
                    amuú[{\downstep}nzísíáchilɪ]}  &   
                     \gloss{‘smack’}  &     &   
                     \vernacular{
                    amuú[nzononyinyilɪ]}  &   
                     \gloss{‘spoil’}  &  \\

                     \vernacular{
                    amuú[nzabukhanyinyilɪ]}  &   
                     \gloss{‘separate’}  &     &     &     &  \\
\end{tabular}
%\caption{\nocaption}
     
\begin{tabular}{llllll}  
  \multicolumn{5}{l}{
                     \vernacular{(403) /Ø/
                    C-Initial + OP + OP
                    } \gloss{‘let
                    him/her...him/her for me’} } &  \\
\multicolumn{5}{l}{ } &  \\

                     \vernacular{
                    amuú[{\downstep}síélɛ]}  &   
                     \gloss{‘grind’}  &     &   
                     \vernacular{
                    amuú[{\downstep}nzíílɪ]}  &   
                     \gloss{‘go for’}  &  \\

                     \vernacular{
                    amuú[{\downstep}ndéshélɛ]}  &   
                     \gloss{‘leave’}  &     &   
                     \vernacular{
                    amuú[{\downstep}nóóndelɛ]}  &   
                     \gloss{‘follow’}  &  \\

                     \vernacular{
                    amuú[{\downstep}ngúlíshili]}  &   
                     \gloss{‘name’}  &     &   
                     \vernacular{
                    amuú[{\downstep}ndákhúulilɪ]}  &   
                     \gloss{‘release’}  &  \\

                     \vernacular{
                    amuú[{\downstep}séébúlilɪ]}  &   
                     \gloss{‘say bye to’}  &     &   
                     \vernacular{
                    amuú[{\downstep}hóómbélitsɪ]}  &   
                     \gloss{‘comfort’}  &  \\

                     \vernacular{
                    amuú[{\downstep}síínjí{\downstep}lítsílɪ́]}  &   
                     \gloss{
                    ‘make...stand’}  &  \\
\end{tabular}
%\caption{\nocaption}
     
\begin{tabular}{llllll}  
  \multicolumn{5}{l}{
                     \vernacular{(404) /Ø/
                    V-Initial + OP + OP
                    } \gloss{‘let
                    him/her...him/her \ob mu-\cb  / it
                    } } &  \\
\multicolumn{5}{l}{ } &  \\

                     \vernacular{
                    amuú[{\downstep}nzéyélɛ]}  &   
                     \gloss{‘wipe’}  &     &   
                     \vernacular{
                    akuú[{\downstep}nzáshítsilɪ]}  &   
                     \gloss{‘light’}  &  \\

                     \vernacular{
                    abuú[{\downstep}nzílúulilɪ]}  &   
                     \gloss{‘winnow’}  &     &   
                     \vernacular{
                    akuú[{\downstep}nzéléelitsilɪ]}  &   
                     \gloss{‘hang’}  &  \\
\end{tabular}
%\caption{\nocaption}
     
\begin{tabular}{lll}  
  \multicolumn{2}{l}{
                     \vernacular{(405) /H/
                    C-Initial Phrase-Medial} \gloss{‘let
                    him/her...the boy \ob mú{\downstep}yáyi\cb  /} } &  \\
\multicolumn{2}{l}{
                     \gloss{someone
                    \ob muundu\cb ’} } &  \\

                     \vernacular{a[rɛ́]
                    {\downstep}mú{\downstep}yáyi/muundu}  &   
                     \gloss{‘bury’}  &  \\

                     \vernacular{a[bechɛ́]
                    {\downstep}mú{\downstep}yáyi/muundu}  &   
                     \gloss{‘shave’}  &  \\

                     \vernacular{a[leerɛ́]
                    {\downstep}mú{\downstep}yáyi/muundu}  &   
                     \gloss{‘bring’}  &  \\

                     \vernacular{a[khalachɛ́]
                    {\downstep}mú{\downstep}yáyi/muundu}  &   
                     \gloss{‘cut’}  &  \\

                     \vernacular{a[sitaáchɛ]
                    mú{\downstep}yáyi/muundu}  &   
                     \gloss{‘accuse’}  &  \\

                     \vernacular{a[boolitsɪ́]
                    {\downstep}mú{\downstep}yáyi/muundu}  &   
                     \gloss{‘seduce’}  &  \\

                     \vernacular{a[khong’oóndɛ]
                    mú{\downstep}yáyi/muundu}  &   
                     \gloss{‘knock’}  &  \\

                     \vernacular{a[boholólɛ]
                    mú{\downstep}yáyi/muundu}  &   
                     \gloss{‘untie’}  &  \\

                     \vernacular{a[boyong’ánɛ]
                    mú{\downstep}yáyi/muundu}  &   
                     \gloss{‘go around’}  &  \\

                     \vernacular{
                    a[ling(ak)anyínyɪ] mú{\downstep}yáyi/muundu}  &   
                     \gloss{‘bend’}  &  \\
\end{tabular}
%\caption{\nocaption}
     
\begin{tabular}{lll}  
  \multicolumn{2}{l}{
                     \vernacular{(406) /Ø/
                    C-Initial Phrase-Medial} \gloss{‘let
                    him/her...the boy \ob mú{\downstep}yáyi\cb  /} } &  \\
\multicolumn{2}{l}{
                     \gloss{someone
                    \ob muundu\cb ’} } &  \\

                     \vernacular{a[tsí]
                    {\downstep}mú{\downstep}yáyi/muundu}  &   
                     \gloss{‘go for’}  &  \\

                     \vernacular{a[leshɛ́]
                    {\downstep}mú{\downstep}yáyi/muundu}  &   
                     \gloss{‘leave’}  &  \\

                     \vernacular{a[loondɛ́]
                    {\downstep}mú{\downstep}yáyi/muundu}  &   
                     \gloss{‘follow’}  &  \\

                     \vernacular{a[kulishɪ́]
                    {\downstep}mú{\downstep}yáyi/muundu}  &   
                     \gloss{‘name’}  &  \\

                     \vernacular{a[lakhuúlɪ]
                    mú{\downstep}yáyi/muundu}  &   
                     \gloss{‘release’}  &  \\

                     \vernacular{a[seebulɪ́]
                    {\downstep}mú{\downstep}yáyi/muundu}  &   
                     \gloss{‘say bye to’}  &  \\

                     \vernacular{a[kalushítsɪ]
                    mú{\downstep}yáyi/muundu}  &   
                     \gloss{‘return’}  &  \\

                     \vernacular{a[siinjilítsɪ]
                    mú{\downstep}yáyi/muundu}  &   
                     \gloss{‘make stand’}  &  \\

                     \vernacular{a[reebɛréebɛ]
                    mú{\downstep}yáyi/muundu}  &   
                     \gloss{‘ask (iter)’}  &  \\

                     \vernacular{a[kalukhányinyɪ]
                    mú{\downstep}yáyi/muundu}  &   
                     \gloss{
                    ‘turn...over’}  &  \\
\end{tabular}
%\caption{\nocaption}
     
\begin{tabular}{lll}  
  \multicolumn{2}{l}{
                     \vernacular{(407) /H/
                    C-Initial +OP Phrase-Medial} \gloss{‘let
                    him/her...the boy \ob mú{\downstep}yáyi\cb  /} } &  \\
\multicolumn{2}{l}{
                     \gloss{someone \ob muundu\cb 
                    for him/her’} } &  \\

                     \vernacular{amu[rɛ́]
                    mú{\downstep}yáyi/muundu}  &   
                     \gloss{‘bury’}  &  \\

                     \vernacular{amu[béchɛ]
                    mú{\downstep}yáyi/muundu}  &   
                     \gloss{‘shave’}  &  \\

                     \vernacular{amu[léerɛ]
                    mú{\downstep}yáyi/muundu}  &   
                     \gloss{‘bring’}  &  \\

                     \vernacular{amu[khálachɛ]
                    mú{\downstep}yáyi/muundu}  &   
                     \gloss{‘cut’}  &  \\

                     \vernacular{amu[sítaachɛ]
                    mú{\downstep}yáyi/muundu}  &   
                     \gloss{‘accuse’}  &  \\

                     \vernacular{amu[bóolitsɪ]
                    mú{\downstep}yáyi/muundu}  &   
                     \gloss{‘seduce’}  &  \\

                     \vernacular{amu[khóng’oondɛ]
                    mú{\downstep}yáyi/muundu}  &   
                     \gloss{‘knock’}  &  \\

                     \vernacular{amu[bóhololɛ]
                    mú{\downstep}yáyi/muundu}  &   
                     \gloss{‘untie’}  &  \\

                     \vernacular{amu[bóyong’anɛ]
                    mú{\downstep}yáyi/muundu}  &   
                     \gloss{‘go around’}  &  \\

                     \vernacular{
                    amu[ng’óng’oolitsɪ]
                    mú{\downstep}yáyi/muundu}  &   
                     \gloss{‘tease’}  &  \\
\end{tabular}
%\caption{\nocaption}
     
\begin{tabular}{lll}  
  \multicolumn{2}{l}{
                     \vernacular{(408) /Ø/
                    C-Initial +OP Phrase-Medial} \gloss{‘let
                    him/her...the boy \ob mú{\downstep}yáyi\cb  /} } &  \\
\multicolumn{2}{l}{
                     \gloss{someone \ob muundu\cb 
                    for him/her’} } &  \\

                     \vernacular{amu[tsí]
                    {\downstep}mú{\downstep}yáyi/muundu}  &   
                     \gloss{‘go for’}  &  \\

                     \vernacular{amu[leshɛ́]
                    {\downstep}mú{\downstep}yáyi/muundu}  &   
                     \gloss{‘leave’}  &  \\

                     \vernacular{amu[loóndɛ]
                    mú{\downstep}yáyi/muundu}  &   
                     \gloss{‘follow’}  &  \\

                     \vernacular{amu[kulíshɪ]
                    mú{\downstep}yáyi/muundu}  &   
                     \gloss{‘name’}  &  \\

                     \vernacular{amu[lakhúulɪ]
                    mú{\downstep}yáyi/muundu}  &   
                     \gloss{‘release’}  &  \\

                     \vernacular{amu[seébúlɪ]
                    mú{\downstep}yáyi/muundu}  &   
                     \gloss{‘say bye to’}  &  \\

                     \vernacular{amu[kalúshítsɪ]
                    mú{\downstep}yáyi/muundu}  &   
                     \gloss{‘return’}  &  \\

                     \vernacular{amu[siínjílitsɪ]
                    mú{\downstep}yáyi/muundu}  &   
                     \gloss{
                    ‘make...stand’}  &  \\

                     \vernacular{amu[reébɛ́reebɛ]
                    mú{\downstep}yáyi/muundu}  &   
                     \gloss{‘ask (iter)’}  &  \\

                     \vernacular{
                    amu[kalúkhányinyɪ]
                    mú{\downstep}yáyi/muundu}  &   
                     \gloss{
                    ‘turn...over’}  &  \\
\end{tabular}
%\caption{\nocaption}
     
\begin{tabular}{lll}  
  \multicolumn{2}{l}{
                     \vernacular{(409) /H/
                    C-Initial +OP + OP
                    } \gloss{‘let
                    him/her...the boy \ob mú{\downstep}yáyi\cb  /} } &  \\
\multicolumn{2}{l}{
                     \gloss{someone \ob muundu\cb 
                    for him/her for me’} } &  \\

                     \vernacular{amuú[ndeelɛ]
                    mú{\downstep}yáyi/muundu}  &   
                     \gloss{‘bury’}  &  \\

                     \vernacular{amuú[mbechelɛ]
                    mú{\downstep}yáyi/muundu}  &   
                     \gloss{‘shave’}  &  \\

                     \vernacular{amuú[ndeerelɛ]
                    mú{\downstep}yáyi/muundu}  &   
                     \gloss{‘bring’}  &  \\

                     \vernacular{amuú[khalachilɪ]
                    mú{\downstep}yáyi/muundu}  &   
                     \gloss{‘cut’}  &  \\

                     \vernacular{amuú[sitaachilɪ]
                    mú{\downstep}yáyi/muundu}  &   
                     \gloss{‘accuse’}  &  \\

                     \vernacular{amuú[mboolitsilɪ]
                    mú{\downstep}yáyi/muundu}  &   
                     \gloss{‘seduce’}  &  \\

                     \vernacular{amuú[mbohololelɛ]
                    mú{\downstep}yáyi/muundu}  &   
                     \gloss{‘untie’}  &  \\
\end{tabular}
%\caption{\nocaption}
     
\begin{tabular}{lll}  
  \multicolumn{2}{l}{
                     \vernacular{(410) /Ø/
                    C-Initial +OP + OP
                    } \gloss{‘let
                    him/her...the boy \ob mú{\downstep}yáyi\cb  /} } &  \\
\multicolumn{2}{l}{
                     \gloss{someone \ob muundu\cb 
                    for him/her for me’} } &  \\

                     \vernacular{amuú[{\downstep}nzíílɪ́]
                    {\downstep}mú{\downstep}yáyi/muundu}  &   
                     \gloss{‘go for’}  &  \\

                     \vernacular{
                    amuú[{\downstep}ndéshélɛ́]
                    {\downstep}mú{\downstep}yáyi/muundu}  &   
                     \gloss{‘leave’}  &  \\

                     \vernacular{
                    amuú[{\downstep}nóóndélɛ] mú{\downstep}yáyi/muundu}  &   
                     \gloss{‘follow’}  &  \\

                     \vernacular{
                    amuú[{\downstep}ngúlíshílɪ]
                    mú{\downstep}yáyi/muundu}  &   
                     \gloss{‘name’}  &  \\

                     \vernacular{
                    amuú[{\downstep}ndákhúulilɪ]
                    mú{\downstep}yáyi/muundu}  &   
                     \gloss{‘release’}  &  \\

                     \vernacular{
                    amuú[{\downstep}séébúlilɪ]
                    mú{\downstep}yáyi/muundu}  &   
                     \gloss{‘say bye to’}  &  \\

                     \vernacular{
                    amuú[{\downstep}síínjílitsilɪ]
                    mú{\downstep}yáyi/muundu}  &   
                     \gloss{
                    ‘make...stand’}  &  \\
\end{tabular}
%\caption{\nocaption}
    

\subsection{Subjunctive Negative: Pattern 2a}\label{sec:sSubjNeg}


\begin{tabular}{llllll}  
  \multicolumn{5}{l}{
                     \vernacular{(411) /H/
                    C-Initial} \gloss{‘let him/her
                    not...’} } &  \\
\multicolumn{5}{l}{ } &  \\

                     \vernacular{akha[ra]
                    tá}  &   
                     \gloss{‘bury’}  &     &   
                     \vernacular{akha[ng’wa]
                    tá}  &   
                     \gloss{‘drink’}  &  \\

                     \vernacular{akha[khwa]
                    tá}  &   
                     \gloss{‘eat’}  &     &   
                     \vernacular{akha[lia]
                    tá}  &   
                     \gloss{‘pay dowry’}  &  \\

                     \vernacular{akha[luma]
                    tá}  &   
                     \gloss{‘bite’}  &     &   
                     \vernacular{akha[beka]
                    tá}  &   
                     \gloss{‘shave’}  &  \\

                     \vernacular{akha[teekha]
                    tá}  &   
                     \gloss{‘cook’}  &     &   
                     \vernacular{akha[leera]
                    tá}  &   
                     \gloss{‘bring’}  &  \\

                     \vernacular{akha[khalaka]
                    tá}  &   
                     \gloss{‘cut’}  &     &   
                     \vernacular{akha[kalaanga]
                    tá}  &   
                     \gloss{‘fry’}  &  \\

                     \vernacular{akha[sitaaka]
                    tá}  &   
                     \gloss{‘accuse’}  &     &   
                     \vernacular{akha[boolitsa]
                    tá}  &   
                     \gloss{‘seduce’}  &  \\

                     \vernacular{akha[saanditsa]
                    tá}  &   
                     \gloss{‘thank’}  &     &   
                     \vernacular{akha[khong’oonda]
                    tá}  &   
                     \gloss{‘knock’}  &  \\

                     \vernacular{akha[boholola]
                    tá}  &   
                     \gloss{‘untie’}  &     &   
                     \vernacular{akha[boyong’ana]
                    tá}  &   
                     \gloss{‘go around’}  &  \\

                     \vernacular{
                    akha[ng’ong’oolitsa] tá}  &   
                     \gloss{‘tease’}  &     &   
                     \vernacular{
                    akha[lingakanyinya] tá}  &   
                     \gloss{‘crumple’}  &  \\
\end{tabular}
%\caption{\nocaption}
     
\begin{tabular}{llllll}  
  \multicolumn{5}{l}{
                     \vernacular{(412) /H/
                    V-Initial} \gloss{‘let him/her
                    not...’} } &  \\
\multicolumn{5}{l}{ } &  \\

                     \vernacular{akh[iira]
                    tá}  &   
                     \gloss{‘kill’}  &     &   
                     \vernacular{akh[iikoó{\downstep}mbá]
                    tá}  &   
                     \gloss{‘admire’}  &  \\

                     \vernacular{akh[iisiá{\downstep}ká]
                    tá}  &   
                     \gloss{‘smack’}  &     &   
                     \vernacular{akh[iikobó{\downstep}lá]
                    tá}  &   
                     \gloss{‘belch’}  &  \\

                     \vernacular{akh[oononyinya]
                    tá}  &   
                     \gloss{‘spoil’}  &     &   
                     \vernacular{akh[aabukhanyinya]
                    tá}  &   
                     \gloss{‘separate’}  &  \\
\end{tabular}
%\caption{\nocaption}
     
\begin{tabular}{llllll}  
  \multicolumn{5}{l}{
                     \vernacular{(413) /Ø/
                    C-Initial} \gloss{‘let him/her
                    not...’} } &  \\
\multicolumn{5}{l}{ } &  \\

                     \vernacular{akha[tsíá]
                    {\downstep}tá}  &   
                     \gloss{‘go’}  &     &   
                     \vernacular{akha[kwá]
                    {\downstep}tá}  &   
                     \gloss{‘fall’}  &  \\

                     \vernacular{akha[lekhá]
                    {\downstep}tá}  &   
                     \gloss{‘leave’}  &     &   
                     \vernacular{akha[reéba]
                    tá}  &   
                     \gloss{‘ask’}  &  \\

                     \vernacular{akha[loónda]
                    tá}  &   
                     \gloss{‘follow’}  &     &   
                     \vernacular{akha[kumíla]
                    tá}  &   
                     \gloss{‘hold’}  &  \\

                     \vernacular{akha[kulíkha]
                    tá}  &   
                     \gloss{‘name’}  &     &   
                     \vernacular{akha[homóola]
                    tá}  &   
                     \gloss{‘massage’}  &  \\

                     \vernacular{akha[lakhúula]
                    tá}  &   
                     \gloss{‘release’}  &     &   
                     \vernacular{akha[seébúla]
                    tá}  &   
                     \gloss{‘say bye’}  &  \\

                     \vernacular{
                    akha[hoómbélitsa] tá}  &   
                     \gloss{‘comfort’}  &     &   
                     \vernacular{akha[kalúshitsa]
                    tá}  &   
                     \gloss{‘return’}  &  \\

                     \vernacular{
                    akha[siínjílitsa] tá}  &   
                     \gloss{‘make stand’}  &     &   
                     \vernacular{akha[reébáreeba]
                    tá}  &   
                     \gloss{‘ask (iter)’}  &  \\

                     \vernacular{
                    akha[kalúkhányinya] tá}  &   
                     \gloss{‘turn over’}  &     &   
                     \vernacular{
                    akha[sebúlúkhanyinya] tá}  &   
                     \gloss{‘scatter’}  &  \\
\end{tabular}
%\caption{\nocaption}
     
\begin{tabular}{llllll}  
  \multicolumn{5}{l}{
                     \vernacular{(414) /Ø/
                    V-Initial} \gloss{‘let him/her
                    not...’} } &  \\
\multicolumn{5}{l}{ } &  \\

                     \vernacular{akh[eenyá]
                    {\downstep}tá}  &   
                     \gloss{‘want’}  &     &   
                     \vernacular{akh[eeyéla]
                    tá}  &   
                     \gloss{‘wipe for’}  &  \\

                     \vernacular{akh[iilúula]
                    tá}  &   
                     \gloss{‘winnow’}  &     &   
                     \vernacular{akh[aambákhána]
                    tá}  &   
                     \gloss{‘refuse’}  &  \\

                     \vernacular{akh[eeléelitsa]
                    tá}  &   
                     \gloss{‘hang up’}  &     &     &     &  \\
\end{tabular}
%\caption{\nocaption}
     
\begin{tabular}{llllll}  
  \multicolumn{5}{l}{
                     \vernacular{(415) /H/
                    C-Initial + OP} \gloss{‘let him/her
                    not...him/her’} } &  \\
\multicolumn{5}{l}{ } &  \\

                     \vernacular{akhamu[rá]
                    {\downstep}tá}  &   
                     \gloss{‘bury’}  &     &   
                     \vernacular{akhamu[béka]
                    tá}  &   
                     \gloss{‘shave’}  &  \\

                     \vernacular{akhamu[léera]
                    tá}  &   
                     \gloss{‘bring’}  &     &   
                     \vernacular{akhamu[khálaka]
                    tá}  &   
                     \gloss{‘cut’}  &  \\

                     \vernacular{akhamu[sítaaka]
                    tá}  &   
                     \gloss{‘accuse’}  &     &   
                     \vernacular{akhamu[bóolitsa]
                    tá}  &   
                     \gloss{‘seduce’}  &  \\

                     \vernacular{
                    akhamu[khóng’oonda] tá}  &   
                     \gloss{‘knock’}  &     &   
                     \vernacular{akhamu[bóholola]
                    tá}  &   
                     \gloss{‘untie’}  &  \\

                     \vernacular{
                    akhamu[bóyong’ana] tá}  &   
                     \gloss{‘go around’}  &     &   
                     \vernacular{
                    akhamu[ng’óng’oolitsa] tá}  &   
                     \gloss{‘tease’}  &  \\

                     \vernacular{
                    akhamu[língakanyinya] tá}  &   
                     \gloss{‘bend’}  &     &     &     &  \\
\end{tabular}
%\caption{\nocaption}
     
\begin{tabular}{llllll}  
  \multicolumn{5}{l}{
                     \vernacular{(416) /H/
                    V-Initial + OP} \gloss{‘let him/her
                    not...him/her’} } &  \\
\multicolumn{5}{l}{ } &  \\

                     \vernacular{akhamw[iíra]
                    tá}  &   
                     \gloss{‘kill’}  &     &   
                     \vernacular{
                    akhamw[ií{\downstep}kóómba] tá}  &   
                     \gloss{‘admire’}  &  \\

                     \vernacular{
                    akhamw[ií{\downstep}síáka] tá}  &   
                     \gloss{‘smack’}  &     &   
                     \vernacular{
                    akhamw[oónonyinya] tá}  &   
                     \gloss{‘spoil’}  &  \\

                     \vernacular{
                    akhamw[aábukhanyinya] tá}  &   
                     \gloss{‘separate’}  &  \\
\end{tabular}
%\caption{\nocaption}
     
\begin{tabular}{llllll}  
  \multicolumn{5}{l}{
                     \vernacular{(417) /Ø/
                    C-Initial + OP} \gloss{‘let him/her
                    not...him/her \ob mu-\cb  / them
                    } } &  \\
\multicolumn{5}{l}{ } &  \\

                     \vernacular{akhamu[tsíá]
                    {\downstep}tá}  &   
                     \gloss{‘go for’}  &  \\

                     \vernacular{akhamu[lekhá]
                    {\downstep}tá}  &   
                     \gloss{‘leave’}  &  \\

                     \vernacular{akhamu[loónda]
                    tá}  &   
                     \gloss{‘follow’}  &  \\

                     \vernacular{akhamu[kulíkha]
                    tá}  &   
                     \gloss{‘name’}  &  \\

                     \vernacular{akhamu[lakhúula]
                    tá}  &   
                     \gloss{‘release’}  &  \\

                     \vernacular{akhamu[seébúla]
                    tá}  &   
                     \gloss{‘say bye to’}  &  \\

                     \vernacular{
                    akhamu[hoómbélitsa] tá}  &   
                     \gloss{‘comfort’}  &  \\

                     \vernacular{
                    akhamu[kalúshitsa] tá}  &   
                     \gloss{‘return’}  &  \\

                     \vernacular{
                    akhamu[siínjílitsa] tá}  &   
                     \gloss{
                    ‘make...stand’}  &  \\

                     \vernacular{
                    akhamu[reébáreeba] tá}  &   
                     \gloss{‘ask (iter)’}  &  \\

                     \vernacular{
                    akhamu[kalúkhányinya] tá}  &   
                     \gloss{
                    ‘turn...over’}  &  \\

                     \vernacular{
                    akhabi[sebúlúkhanyinya] tá}  &   
                     \gloss{‘scatter’}  &  \\
\end{tabular}
%\caption{\nocaption}
     
\begin{tabular}{llllll}  
  \multicolumn{5}{l}{
                     \vernacular{(418) /Ø/
                    V-Initial + OP} \gloss{‘let him/her
                    not...him/her \ob mw-\cb  / it
                    } } &  \\
\multicolumn{5}{l}{ } &  \\

                     \vernacular{akhamw[eenyá]
                    {\downstep}tá}  &   
                     \gloss{‘want’}  &     &   
                     \vernacular{akhamw[eeyéla]
                    tá}  &   
                     \gloss{‘wipe for’}  &  \\

                     \vernacular{akhabw[iilúula]
                    tá}  &   
                     \gloss{‘winnow’}  &     &   
                     \vernacular{
                    akhamw[aambákhana] tá}  &   
                     \gloss{‘refuse’}  &  \\

                     \vernacular{
                    akhamw[eeléelitsa] tá}  &   
                     \gloss{
                    ‘carry...hanging’}  &  \\
\end{tabular}
%\caption{\nocaption}
     
\begin{tabular}{llllll}  
  \multicolumn{5}{l}{
                     \vernacular{(419) /H/
                    C-Initial + OP
                    } \gloss{‘let him/her
                    not...me’} } &  \\
\multicolumn{5}{l}{ } &  \\

                     \vernacular{akhaa[ríá]
                    {\downstep}tá}  &   
                     \gloss{‘fear’}  &     &   
                     \vernacular{akhaa[mbéka]
                    tá}  &   
                     \gloss{‘shave’}  &  \\

                     \vernacular{akhaa[ndéera]
                    tá}  &   
                     \gloss{‘bring’}  &     &   
                     \vernacular{akhaa[khálaka]
                    tá}  &   
                     \gloss{‘cut’}  &  \\

                     \vernacular{akhaa[sítaaka]
                    tá}  &   
                     \gloss{‘accuse’}  &     &   
                     \vernacular{akhaa[mbóolitsa]
                    tá}  &   
                     \gloss{‘seduce’}  &  \\

                     \vernacular{
                    akhaa[khóng’oonda] tá}  &   
                     \gloss{‘knock’}  &     &   
                     \vernacular{akhaa[mbóholola]
                    tá}  &   
                     \gloss{‘untie’}  &  \\

                     \vernacular{
                    akhaa[mbóyong’ana] tá}  &   
                     \gloss{‘go around’}  &     &   
                     \vernacular{
                    akhaa[ng’óng’oolitsa] tá}  &   
                     \gloss{‘tease’}  &  \\

                     \vernacular{
                    akhaa[níngakanyinya] tá}  &   
                     \gloss{‘bend’}  &  \\
\end{tabular}
%\caption{\nocaption}
     
\begin{tabular}{llllll}  
  \multicolumn{5}{l}{
                     \vernacular{(420) /H/
                    V-Initial + OP
                    } \gloss{‘let him/her
                    not...me’} } &  \\
\multicolumn{5}{l}{ } &  \\

                     \vernacular{akhaa[nzíra]
                    tá}  &   
                     \gloss{‘kill’}  &     &   
                     \vernacular{
                    akhaa[nzí{\downstep}kóómba] tá}  &   
                     \gloss{‘admire’}  &  \\

                     \vernacular{
                    akhaa[nzí{\downstep}síáka] tá}  &   
                     \gloss{‘smack’}  &     &   
                     \vernacular{
                    akhaa[nzónonyinya] tá}  &   
                     \gloss{‘spoil’}  &  \\

                     \vernacular{
                    akhaa[nzábukhanyinya] tá}  &   
                     \gloss{‘separate’}  &  \\
\end{tabular}
%\caption{\nocaption}
     
\begin{tabular}{llllll}  
  \multicolumn{5}{l}{
                     \vernacular{(421) /Ø/
                    C-Initial + OP
                    } \gloss{‘let him/her
                    not...me’} } &  \\
\multicolumn{5}{l}{ } &  \\

                     \vernacular{akhaa[síá]
                    {\downstep}tá}  &   
                     \gloss{‘grind’}  &     &   
                     \vernacular{akhaa[ndekhá]
                    {\downstep}tá}  &   
                     \gloss{‘leave’}  &  \\

                     \vernacular{akhaa[noónda]
                    tá}  &   
                     \gloss{‘follow’}  &     &   
                     \vernacular{akhaa[ngulíkha]
                    tá}  &   
                     \gloss{‘name’}  &  \\

                     \vernacular{akhaa[ndakhúula]
                    tá}  &   
                     \gloss{‘release’}  &     &   
                     \vernacular{akhaa[seébúla]
                    tá}  &   
                     \gloss{‘say bye to’}  &  \\

                     \vernacular{
                    akhaa[mboómbélitsa] tá}  &   
                     \gloss{‘comfort’}  &     &   
                     \vernacular{
                    akhaa[siínjílitsa] tá}  &   
                     \gloss{
                    ‘make..stand’}  &  \\

                     \vernacular{
                    akhaa[ndeébándeeba] tá}  &   
                     \gloss{‘ask (iter)’}  &     &   
                     \vernacular{
                    akhaa[ngalúkhányinya] tá}  &   
                     \gloss{
                    ‘turn...over’}  &  \\
\end{tabular}
%\caption{\nocaption}
     
\begin{tabular}{llllll}  
  \multicolumn{5}{l}{
                     \vernacular{(422) /Ø/
                    V-Initial + OP
                    } \gloss{‘let him/her
                    not...me’} } &  \\
\multicolumn{5}{l}{ } &  \\

                     \vernacular{akhaa[nzenyá]
                    {\downstep}tá}  &   
                     \gloss{‘want’}  &     &   
                     \vernacular{akhaa[nzeyéla]
                    tá}  &   
                     \gloss{‘wipe for’}  &  \\

                     \vernacular{
                    akhaa[nyambákhana] tá}  &   
                     \gloss{‘refuse’}  &     &   
                     \vernacular{
                    akhaa[nzeléelitsa] tá}  &   
                     \gloss{
                    ‘carry...hanging’}  &  \\
\end{tabular}
%\caption{\nocaption}
     
\begin{tabular}{llllll}  
  \multicolumn{5}{l}{
                     \vernacular{(423) /H/
                    C-Initial + OP
                    } \gloss{‘let him/her
                    not...him/herself’} } &  \\
\multicolumn{5}{l}{ } &  \\

                     \vernacular{akhii[rá]
                    {\downstep}tá}  &   
                     \gloss{‘bury’}  &     &   
                     \vernacular{akhii[béka]
                    tá}  &   
                     \gloss{‘shave’}  &  \\

                     \vernacular{akhii[súunga]
                    tá}  &   
                     \gloss{‘hang’}  &     &   
                     \vernacular{akhii[khálaka]
                    tá}  &   
                     \gloss{‘cut’}  &  \\

                     \vernacular{akhii[sítaaka]
                    tá}  &   
                     \gloss{‘accuse’}  &     &   
                     \vernacular{akhii[sáanditsa]
                    tá}  &   
                     \gloss{‘thank’}  &  \\

                     \vernacular{
                    akhii[khóng’oonda] tá}  &   
                     \gloss{‘knock’}  &     &   
                     \vernacular{akhii[bóholola]
                    tá}  &   
                     \gloss{‘untie’}  &  \\
\end{tabular}
%\caption{\nocaption}
     
\begin{tabular}{llllll}  
  \multicolumn{5}{l}{
                     \vernacular{(424) /H/
                    V-Initial + OP
                    } \gloss{‘let him/her
                    not...him/herself’} } &  \\
\multicolumn{5}{l}{ } &  \\

                     \vernacular{akhii[yíra]
                    tá}  &   
                     \gloss{‘kill’}  &     &   
                     \vernacular{
                    akhii[yí{\downstep}kóómba] tá}  &   
                     \gloss{‘admire’}  &  \\

                     \vernacular{akhii[yí{\downstep}síáka]
                    tá}  &   
                     \gloss{‘smack’}  &     &   
                     \vernacular{akhii[yónonyinya]
                    tá}  &   
                     \gloss{‘spoil’}  &  \\

                     \vernacular{
                    akhii[yábukhanyinya] tá}  &   
                     \gloss{‘separate’}  &  \\
\end{tabular}
%\caption{\nocaption}
     
\begin{tabular}{llllll}  
  \multicolumn{5}{l}{
                     \vernacular{(425) /Ø/
                    C-Initial + OP
                    } \gloss{‘let him/her
                    not...him/herself’} } &  \\
\multicolumn{5}{l}{ } &  \\

                     \vernacular{akhii[síá]
                    {\downstep}tá}  &   
                     \gloss{‘grind’}  &     &   
                     \vernacular{akhii[lekhá]
                    {\downstep}tá}  &   
                     \gloss{‘leave’}  &  \\

                     \vernacular{akhii[siínga]
                    tá}  &   
                     \gloss{‘bathe’}  &     &   
                     \vernacular{akhii[kulíkha]
                    tá}  &   
                     \gloss{‘name’}  &  \\

                     \vernacular{akhii[naábúla]
                    tá}  &   
                     \gloss{‘undress’}  &     &   
                     \vernacular{akhii[lakhúula]
                    tá}  &   
                     \gloss{‘release’}  &  \\

                     \vernacular{
                    akhii[hoómbélitsa] tá}  &   
                     \gloss{‘comfort’}  &     &   
                     \vernacular{
                    akhii[siínjílitsa] tá}  &   
                     \gloss{
                    ‘make...stand’}  &  \\

                     \vernacular{
                    akhii[reébáreeba] tá}  &   
                     \gloss{‘ask (iter)’}  &     &   
                     \vernacular{
                    akhii[kalúkhányinya] tá}  &   
                     \gloss{
                    ‘turn...over’}  &  \\
\end{tabular}
%\caption{\nocaption}
     
\begin{tabular}{llllll}  
  \multicolumn{5}{l}{
                     \vernacular{(426) /Ø/
                    V-Initial + OP
                    } \gloss{‘let him/her
                    not...him/herself’} } &  \\
\multicolumn{5}{l}{ } &  \\

                     \vernacular{akhii[yalá]
                    {\downstep}tá}  &   
                     \gloss{‘expose’}  &     &   
                     \vernacular{akhii[yeyéla]
                    tá}  &   
                     \gloss{‘wipe for’}  &  \\

                     \vernacular{akhii[yambákhana]
                    tá}  &   
                     \gloss{‘refuse’}  &     &   
                     \vernacular{akhii[yeléelitsa]
                    tá}  &   
                     \gloss{‘hang...up’}  &  \\
\end{tabular}
%\caption{\nocaption}
     
\begin{tabular}{llllll}  
  \multicolumn{5}{l}{
                     \vernacular{(427) /H/
                    C-Initial + OP + OP
                    } \gloss{‘let him/her
                    not...him/her for me’} } &  \\
\multicolumn{5}{l}{ } &  \\

                     \vernacular{akhamuu[ndéela]
                    tá}  &   
                     \gloss{‘bury’}  &     &   
                     \vernacular{akhamuu[mbéchela]
                    tá}  &   
                     \gloss{‘shave’}  &  \\

                     \vernacular{akhamuu[ndéerela]
                    tá}  &   
                     \gloss{‘bring’}  &     &   
                     \vernacular{
                    akhamuu[khálachila] tá}  &   
                     \gloss{‘cut’}  &  \\

                     \vernacular{
                    akhamuu[sítaachila] tá}  &   
                     \gloss{‘accuse’}  &     &   
                     \vernacular{
                    akhamuu[mbóolitsila] tá}  &   
                     \gloss{‘seduce’}  &  \\

                     \vernacular{
                    akhamuu[mbóhololela] tá}  &   
                     \gloss{‘untie’}  &     &     &     &  \\
\end{tabular}
%\caption{\nocaption}
     
\begin{tabular}{llllll}  
  \multicolumn{5}{l}{
                     \vernacular{(428) /H/
                    V-Initial + OP + OP
                    } \gloss{‘let him/her
                    not...him/her for me’} } &  \\
\multicolumn{5}{l}{ } &  \\

                     \vernacular{akhamuu[nzírila]
                    tá}  &   
                     \gloss{‘kill’}  &  \\

                     \vernacular{
                    akhamuu[nzéchitsila] tá}  &   
                     \gloss{‘admire’}  &  \\

                     \vernacular{
                    akhamuu[nzí{\downstep}síáchila] tá}  &   
                     \gloss{‘smack’}  &  \\

                     \vernacular{
                    akhamuu[nzónonyinyila] tá}  &   
                     \gloss{‘spoil’}  &  \\

                     \vernacular{
                    akhamuu[nzábukhanyinyila] tá}  &   
                     \gloss{‘separate’}  &  \\
\end{tabular}
%\caption{\nocaption}
     
\begin{tabular}{llllll}  
  \multicolumn{5}{l}{
                     \vernacular{(429) /Ø/
                    C-Initial + OP + OP
                    } \gloss{‘let him/her
                    not...him/her for me’} } &  \\
\multicolumn{5}{l}{ } &  \\

                     \vernacular{akhamuu[nziíla]
                    tá}  &   
                     \gloss{‘go for’}  &  \\

                     \vernacular{akhamuu[ndeshéla]
                    tá}  &   
                     \gloss{‘leave’}  &  \\

                     \vernacular{
                    akhamuu[noóndéla] tá}  &   
                     \gloss{‘follow’}  &  \\

                     \vernacular{
                    akhamuu[ngulíshíla] tá}  &   
                     \gloss{‘name’}  &  \\

                     \vernacular{
                    akhamuu[ndakhúulila] tá}  &   
                     \gloss{‘release’}  &  \\

                     \vernacular{
                    akhamuu[seébúlila] tá}  &   
                     \gloss{‘say bye to’}  &  \\

                     \vernacular{
                    akhamuu[mboómbélitsila] tá}  &   
                     \gloss{‘comfort’}  &  \\

                     \vernacular{
                    akhamuu[siínjílitsila] tá}  &   
                     \gloss{
                    ‘make...stand’}  &  \\
\end{tabular}
%\caption{\nocaption}
     
\begin{tabular}{llllll}  
  \multicolumn{5}{l}{
                     \vernacular{(430) /Ø/
                    V-Initial + OP + OP
                    } \gloss{‘let him/her
                    not...him/her \ob mu-\cb  / it
                    } } &  \\
\multicolumn{5}{l}{ } &  \\

                     \vernacular{akhamuu[nzeyéla]
                    tá}  &   
                     \gloss{‘wipe’}  &     &   
                     \vernacular{
                    akhakuu[nzashítsila] tá}  &   
                     \gloss{‘light’}  &  \\

                     \vernacular{
                    akhabuu[nzilúulila] tá}  &   
                     \gloss{‘winnow’}  &     &   
                     \vernacular{
                    akhakuu[nzeléelitsila] tá}  &   
                     \gloss{‘hang’}  &  \\
\end{tabular}
%\caption{\nocaption}
     
\begin{tabular}{lll}  
  \multicolumn{2}{l}{
                     \vernacular{(431) /H/
                    C-Initial Phrase-Medial} \gloss{‘let him/her
                    not...the boy \ob mú{\downstep}yáyi\cb  /} } &  \\
\multicolumn{2}{l}{
                     \gloss{someone
                    \ob muundu\cb ’} } &  \\

                     \vernacular{akha[ra]
                    mú{\downstep}yáyi/muundu tá}  &   
                     \gloss{‘bury’}  &  \\

                     \vernacular{akha[beka]
                    mú{\downstep}yáyi/muundu tá}  &   
                     \gloss{‘shave’}  &  \\

                     \vernacular{akha[leera]
                    mú{\downstep}yáyi/muundu tá}  &   
                     \gloss{‘bring’}  &  \\

                     \vernacular{akha[khalaka]
                    mú{\downstep}yáyi/muundu tá}  &   
                     \gloss{‘cut’}  &  \\

                     \vernacular{akha[sitaaka]
                    mú{\downstep}yáyi/muundu tá}  &   
                     \gloss{‘accuse’}  &  \\

                     \vernacular{akha[boolitsa]
                    mú{\downstep}yáyi/muundu tá}  &   
                     \gloss{‘seduce’}  &  \\

                     \vernacular{akha[khong’oonda]
                    mú{\downstep}yáyi/muundu tá}  &   
                     \gloss{‘knock’}  &  \\

                     \vernacular{akha[boholola]
                    mú{\downstep}yáyi/muundu tá}  &   
                     \gloss{‘untie’}  &  \\

                     \vernacular{akha[boyong’ana]
                    mú{\downstep}yáyi/muundu tá}  &   
                     \gloss{‘go around’}  &  \\

                     \vernacular{
                    akha[lingakanyinya] mú{\downstep}yáyi/muundu
                    tá}  &   
                     \gloss{‘bend’}  &  \\
\end{tabular}
%\caption{\nocaption}
     
\begin{tabular}{lll}  
  \multicolumn{2}{l}{
                     \vernacular{(432) /Ø/
                    C-Initial Phrase-Medial} \gloss{‘let him/her
                    not...the boy \ob mú{\downstep}yáyi\cb  /} } &  \\
\multicolumn{2}{l}{
                     \gloss{someone
                    \ob muundu\cb ’} } &  \\

                     \vernacular{akha[tsíá]
                    {\downstep}mú{\downstep}yáyi/muundu tá}  &   
                     \gloss{‘go for’}  &  \\

                     \vernacular{akha[lekhá]
                    {\downstep}mú{\downstep}yáyi/muundu tá}  &   
                     \gloss{‘leave’}  &  \\

                     \vernacular{akha[loónda]
                    mú{\downstep}yáyi/muundu tá}  &   
                     \gloss{‘follow’}  &  \\

                     \vernacular{akha[kulíkha]
                    mú{\downstep}yáyi/muundu tá}  &   
                     \gloss{‘name’}  &  \\

                     \vernacular{akha[lakhúula]
                    mú{\downstep}yáyi/muundu tá}  &   
                     \gloss{‘release’}  &  \\

                     \vernacular{akha[seébúla]
                    mú{\downstep}yáyi/muundu tá}  &   
                     \gloss{‘say bye to’}  &  \\

                     \vernacular{akha[kalúshítsa]
                    mú{\downstep}yáyi/muundu tá}  &   
                     \gloss{‘return’}  &  \\

                     \vernacular{
                    akha[siínjílitsa] mú{\downstep}yáyi/muundu
                    tá}  &   
                     \gloss{
                    ‘make...stand’}  &  \\

                     \vernacular{akha[reébáreeba]
                    mú{\downstep}yáyi/muundu tá}  &   
                     \gloss{‘ask (iter)’}  &  \\

                     \vernacular{
                    akha[kalúkhányinya] mú{\downstep}yáyi/muundu
                    tá}  &   
                     \gloss{
                    ‘turn...over’}  &  \\
\end{tabular}
%\caption{\nocaption}
     
\begin{tabular}{lll}  
  \multicolumn{2}{l}{
                     \vernacular{(433) /H/
                    C-Initial +OP Phrase-Medial} \gloss{‘let him/her
                    not...the boy \ob mú{\downstep}yáyi\cb  /} } &  \\
\multicolumn{2}{l}{
                     \gloss{someone \ob muundu\cb 
                    for him/her’} } &  \\

                     \vernacular{akhamu[rá]
                    {\downstep}mú{\downstep}yáyi/muundu tá}  &   
                     \gloss{‘bury’}  &  \\

                     \vernacular{akhamu[béka]
                    mú{\downstep}yáyi/muundu tá}  &   
                     \gloss{‘shave’}  &  \\

                     \vernacular{akhamu[léera]
                    mú{\downstep}yáyi/muundu tá}  &   
                     \gloss{‘bring’}  &  \\

                     \vernacular{akhamu[khálaka]
                    mú{\downstep}yáyi/muundu tá}  &   
                     \gloss{‘cut’}  &  \\

                     \vernacular{akhamu[sítaaka]
                    mú{\downstep}yáyi/muundu tá}  &   
                     \gloss{‘accuse’}  &  \\

                     \vernacular{akhamu[bóolitsa]
                    mú{\downstep}yáyi/muundu tá}  &   
                     \gloss{‘seduce’}  &  \\

                     \vernacular{
                    akhamu[khóng’oonda] mú{\downstep}yáyi/muundu
                    tá}  &   
                     \gloss{‘knock’}  &  \\

                     \vernacular{akhamu[bóholola]
                    mú{\downstep}yáyi/muundu tá}  &   
                     \gloss{‘untie’}  &  \\

                     \vernacular{
                    akhamu[bóyong’ana] mú{\downstep}yáyi/muundu
                    tá}  &   
                     \gloss{‘go around’}  &  \\

                     \vernacular{
                    akhamu[ng’óng’oolitsa] mú{\downstep}yáyi/muundu
                    tá}  &   
                     \gloss{‘tease’}  &  \\
\end{tabular}
%\caption{\nocaption}
     
\begin{tabular}{lll}  
  \multicolumn{2}{l}{
                     \vernacular{(434) /Ø/
                    C-Initial +OP Phrase-Medial} \gloss{‘let him/her
                    not...the boy \ob mú{\downstep}yáyi\cb  /} } &  \\
\multicolumn{2}{l}{
                     \gloss{someone \ob muundu\cb 
                    for him/her’} } &  \\

                     \vernacular{akhamu[tsíá]
                    {\downstep}mú{\downstep}yáyi/muundu tá}  &   
                     \gloss{‘go for’}  &  \\

                     \vernacular{akhamu[lekhá]
                    {\downstep}mú{\downstep}yáyi/muundu tá}  &   
                     \gloss{‘leave’}  &  \\

                     \vernacular{akhamu[loónda]
                    mú{\downstep}yáyi/muundu tá}  &   
                     \gloss{‘follow’}  &  \\

                     \vernacular{akhamu[kulíkha]
                    mú{\downstep}yáyi/muundu tá}  &   
                     \gloss{‘name’}  &  \\

                     \vernacular{akhamu[lakhúula]
                    mú{\downstep}yáyi/muundu tá}  &   
                     \gloss{‘release’}  &  \\

                     \vernacular{akhamu[seébúla]
                    mú{\downstep}yáyi/muundu tá}  &   
                     \gloss{‘say bye to’}  &  \\

                     \vernacular{
                    akhamu[kalúshítsa] mú{\downstep}yáyi/muundu
                    tá}  &   
                     \gloss{‘return’}  &  \\

                     \vernacular{
                    akhamu[siínjílitsa] mú{\downstep}yáyi/muundu
                    tá}  &   
                     \gloss{
                    ‘make...stand’}  &  \\

                     \vernacular{
                    akhamu[reébáreeba] mú{\downstep}yáyi/muundu
                    tá}  &   
                     \gloss{‘ask (iter)’}  &  \\

                     \vernacular{
                    akhamu[kalúkhányinya] mú{\downstep}yáyi/muundu
                    tá}  &   
                     \gloss{
                    ‘turn...over’}  &  \\
\end{tabular}
%\caption{\nocaption}
     
\begin{tabular}{lll}  
  \multicolumn{2}{l}{
                     \vernacular{(435) /H/
                    C-Initial +OP + OP
                    } \gloss{‘let him/her
                    not...the boy \ob mú{\downstep}yáyi\cb  /} } &  \\
\multicolumn{2}{l}{
                     \gloss{someone \ob muundu\cb 
                    for him/her for me’} } &  \\

                     \vernacular{akhamuu[ndéela]
                    mú{\downstep}yáyi/muundu tá}  &   
                     \gloss{‘bury’}  &  \\

                     \vernacular{akhamuu[mbéchela]
                    mú{\downstep}yáyi/muundu tá}  &   
                     \gloss{‘shave’}  &  \\

                     \vernacular{akhamuu[ndéerela]
                    mú{\downstep}yáyi/muundu tá}  &   
                     \gloss{‘bring’}  &  \\

                     \vernacular{
                    akhamuu[khálachila] mú{\downstep}yáyi/muundu
                    tá}  &   
                     \gloss{‘cut’}  &  \\

                     \vernacular{
                    akhamuu[sítaachila] mú{\downstep}yáyi/muundu
                    tá}  &   
                     \gloss{‘accuse’}  &  \\

                     \vernacular{
                    akhamuu[mbóolitsila] mú{\downstep}yáyi/muundu
                    tá}  &   
                     \gloss{‘seduce’}  &  \\

                     \vernacular{
                    akhamuu[mbóhololela] mú{\downstep}yáyi/muundu
                    tá}  &   
                     \gloss{‘untie’}  &  \\
\end{tabular}
%\caption{\nocaption}
     
\begin{tabular}{lll}  
  \multicolumn{2}{l}{
                     \vernacular{(436) /Ø/
                    C-Initial +OP + OP
                    } \gloss{‘let him/her
                    not...the boy \ob mú{\downstep}yáyi\cb  /} } &  \\
\multicolumn{2}{l}{
                     \gloss{someone \ob muundu\cb 
                    for him/her for me’} } &  \\

                     \vernacular{akhamuu[nzííla]
                    mú{\downstep}yáyi/muundu tá}  &   
                     \gloss{‘go for’}  &  \\

                     \vernacular{akhamuu[ndeshéla]
                    mú{\downstep}yáyi/muundu tá}  &   
                     \gloss{‘leave’}  &  \\

                     \vernacular{
                    akhamuu[noóndéla] mú{\downstep}yáyi/muundu
                    tá}  &   
                     \gloss{‘follow’}  &  \\

                     \vernacular{
                    akhamuu[ngulíshíla] mú{\downstep}yáyi/muundu
                    tá}  &   
                     \gloss{‘name’}  &  \\

                     \vernacular{
                    akhamuu[ndakhúulila] mú{\downstep}yáyi/muundu
                    tá}  &   
                     \gloss{‘release’}  &  \\

                     \vernacular{
                    akhamuu[seébúlila] mú{\downstep}yáyi/muundu
                    tá}  &   
                     \gloss{‘say bye to’}  &  \\

                     \vernacular{
                    akhamuu[siínjílitsila] mú{\downstep}yáyi/muundu
                    tá}  &   
                     \gloss{
                    ‘make...stand’}  &  \\
\end{tabular}
%\caption{\nocaption}
    

\subsection{Hesternal Perfective: Pattern 7}\label{sec:sHestPerf}


\begin{tabular}{llllll}  
  \multicolumn{5}{l}{
                     \vernacular{(437) /H/
                    C-Initial} \gloss{
                    ‘s/he...’} } &  \\
\multicolumn{5}{l}{ } &  \\

                     \vernacular{
                    ya[réélé]}  &   
                     \gloss{‘buried’}  &     &   
                     \vernacular{
                    ya[ng’wéélé]}  &   
                     \gloss{‘drank’}  &  \\

                     \vernacular{
                    ya[khwéélé]}  &   
                     \gloss{‘ate’}  &     &   
                     \vernacular{
                    ya[líílí]}  &   
                     \gloss{‘paid dowry’}  &  \\

                     \vernacular{
                    ya[lúmí]}  &   
                     \gloss{‘bit’}  &     &   
                     \vernacular{
                    ya[béchí]}  &   
                     \gloss{‘shaved’}  &  \\

                     \vernacular{
                    ya[tééshí]}  &   
                     \gloss{‘cooked’}  &     &   
                     \vernacular{
                    ya[léérí]}  &   
                     \gloss{‘brought’}  &  \\

                     \vernacular{
                    ya[khálááchɛ́]}  &   
                     \gloss{‘cut’}  &     &   
                     \vernacular{
                    ya[káláánjí]}  &   
                     \gloss{‘fried’}  &  \\

                     \vernacular{
                    ya[sítááchí]}  &   
                     \gloss{‘accused’}  &     &   
                     \vernacular{
                    ya[bóólíítsɪ́]}  &   
                     \gloss{‘seduced’}  &  \\

                     \vernacular{
                    ya[sáándíítsɪ́]}  &   
                     \gloss{‘thanked’}  &     &   
                     \vernacular{
                    ya[khóng’óóndí]}  &   
                     \gloss{‘knocked’}  &  \\

                     \vernacular{
                    ya[bóhólóólɛ́]}  &   
                     \gloss{‘untied’}  &     &   
                     \vernacular{
                    ya[bóyóng’áánɛ́]}  &   
                     \gloss{‘went
                    around’}  &  \\

                     \vernacular{
                    ya[ng’óng’óólíítsɪ́]}  &   
                     \gloss{‘teased’}  &     &   
                     \vernacular{
                    ya[líng(ák)ányíínyɪ́]}  &   
                     \gloss{‘crumpled’}  &  \\

                     \vernacular{
                    wa[ng’wéélé]}  &   
                     \gloss{‘(you)
                    drank’}  &  \\
\end{tabular}
%\caption{\nocaption}
     
\begin{tabular}{llllll}  
  \multicolumn{5}{l}{
                     \vernacular{(438) /H/
                    V-Initial} \gloss{
                    ‘s/he...’} } &  \\
\multicolumn{5}{l}{ } &  \\

                     \vernacular{
                    ya[yírí]}  &   
                     \gloss{‘killed’}  &     &   
                     \vernacular{
                    ya[yikóómbí]}  &   
                     \gloss{‘admired’}  &  \\

                     \vernacular{
                    ya[yisíáchí]}  &   
                     \gloss{‘smacked’}  &     &   
                     \vernacular{
                    ya[yikóbó{\downstep}ólɛ́]}  &   
                     \gloss{‘belched’}  &  \\

                     \vernacular{
                    ya[yónónyíínyɪ́]}  &   
                     \gloss{‘spoiled’}  &     &   
                     \vernacular{
                    ya[yábúkhányíínyɪ́]}  &   
                     \gloss{‘separated’}  &  \\
\end{tabular}
%\caption{\nocaption}
     
\begin{tabular}{llllll}  
  \multicolumn{5}{l}{
                     \vernacular{(439) /Ø/
                    C-Initial} \gloss{
                    ‘s/he...’} } &  \\
\multicolumn{5}{l}{ } &  \\

                     \vernacular{
                    ya[tsíí{\downstep}lí]}  &   
                     \gloss{‘went’}  &     &   
                     \vernacular{
                    ya[kwíí{\downstep}lí]}  &   
                     \gloss{‘fell’}  &  \\

                     \vernacular{
                    ya[léshí]}  &   
                     \gloss{‘left’}  &     &   
                     \vernacular{
                    ya[réé{\downstep}bí]}  &   
                     \gloss{‘asked’}  &  \\

                     \vernacular{
                    ya[lóó{\downstep}ndí]}  &   
                     \gloss{‘followed’}  &     &   
                     \vernacular{
                    ya[kúmíí{\downstep}lɪ́]}  &   
                     \gloss{‘held’}  &  \\

                     \vernacular{
                    ya[kúlíí{\downstep}shɪ́]}  &   
                     \gloss{‘named’}  &     &   
                     \vernacular{
                    ya[hómó{\downstep}ólí]}  &   
                     \gloss{‘massaged’}  &  \\

                     \vernacular{
                    ya[lákhú{\downstep}úlí]}  &   
                     \gloss{‘released’}  &     &   
                     \vernacular{
                    ya[séé{\downstep}búúlɪ́]}  &   
                     \gloss{‘said bye’}  &  \\

                     \vernacular{
                    ya[hóómbé{\downstep}líítsɪ́]}  &   
                     \gloss{‘comforted’}  &     &   
                     \vernacular{
                    ya[kálú{\downstep}shíítsɪ́]}  &   
                     \gloss{‘returned’}  &  \\

                     \vernacular{
                    ya[síínjí{\downstep}líítsɪ́]}  &   
                     \gloss{‘made stand’}  &     &   
                     \vernacular{
                    ya[réébí{\downstep}réébí]}  &   
                     \gloss{‘asked
                    (iter)’}  &  \\

                     \vernacular{
                    ya[kálúkhá{\downstep}nyíínyɪ́]}  &   
                     \gloss{‘turned
                    over’}  &     &   
                     \vernacular{
                    ya[sébúlú{\downstep}khányíínyɪ́]}  &   
                     \gloss{‘scattered’}  &  \\
\end{tabular}
%\caption{\nocaption}
     
\begin{tabular}{llllll}  
  \multicolumn{5}{l}{
                     \vernacular{(440) /Ø/
                    V-Initial} \gloss{
                    ‘s/he...’} } &  \\
\multicolumn{5}{l}{ } &  \\

                     \vernacular{
                    ya[yényí]}  &   
                     \gloss{‘wanted’}  &     &   
                     \vernacular{
                    ya[yéyéé{\downstep}lɛ́]}  &   
                     \gloss{‘wiped for’}  &  \\

                     \vernacular{
                    ya[yílúú{\downstep}lí]}  &   
                     \gloss{‘winnowed’}  &     &   
                     \vernacular{
                    ya[yámbá{\downstep}kháánɛ́]}  &   
                     \gloss{‘refused’}  &  \\

                     \vernacular{
                    ya[yéléé{\downstep}líítsɪ́]}  &   
                     \gloss{‘hung up’}  &     &     &     &  \\
\end{tabular}
%\caption{\nocaption}
     
\begin{tabular}{llllll}  
  \multicolumn{5}{l}{
                     \vernacular{(441) /H/
                    C-Initial + OP} \gloss{
                    ‘s/he...him/her’} } &  \\
\multicolumn{5}{l}{ } &  \\

                     \vernacular{
                    yamu[ré{\downstep}élé]}  &   
                     \gloss{‘buried’}  &     &   
                     \vernacular{
                    yamu[bé{\downstep}chí]}  &   
                     \gloss{‘shaved’}  &  \\

                     \vernacular{
                    yamu[lé{\downstep}érí]}  &   
                     \gloss{‘brought’}  &     &   
                     \vernacular{
                    yamu[khá{\downstep}lááchɛ́]}  &   
                     \gloss{‘cut’}  &  \\

                     \vernacular{
                    yamu[sí{\downstep}tááchí]}  &   
                     \gloss{‘accused’}  &     &   
                     \vernacular{
                    yamu[bó{\downstep}ólíítsɪ́]}  &   
                     \gloss{‘seduced’}  &  \\

                     \vernacular{
                    yamu[khó{\downstep}ng’óóndí]}  &   
                     \gloss{‘knocked’}  &     &   
                     \vernacular{
                    yamu[bó{\downstep}hólóólɛ́]}  &   
                     \gloss{‘untied’}  &  \\

                     \vernacular{
                    yamu[bó{\downstep}yóng’áánɛ́]}  &   
                     \gloss{‘went
                    around’}  &     &   
                     \vernacular{
                    yamu[ng’ó{\downstep}ng’óólíítsɪ́]}  &   
                     \gloss{‘teased’}  &  \\

                     \vernacular{
                    yamu[lí{\downstep}ngákányíínyɪ́]}  &   
                     \gloss{‘bent’}  &     &     &     &  \\
\end{tabular}
%\caption{\nocaption}
     
\begin{tabular}{llllll}  
  \multicolumn{5}{l}{
                     \vernacular{(442) /H/
                    V-Initial + OP} \gloss{
                    ‘s/he...him/her’} } &  \\
\multicolumn{5}{l}{ } &  \\

                     \vernacular{
                    yamw[ií{\downstep}rí]}  &   
                     \gloss{‘killed’}  &     &   
                     \vernacular{
                    yamw[ií{\downstep}kóómbí]}  &   
                     \gloss{‘admired’}  &  \\

                     \vernacular{
                    yamw[ií{\downstep}síáchí]}  &   
                     \gloss{‘smacked’}  &     &   
                     \vernacular{
                    yamw[oó{\downstep}nónyíínyɪ́]}  &   
                     \gloss{‘spoiled’}  &  \\

                     \vernacular{
                    yamw[aá{\downstep}búkhányíínyɪ́]}  &   
                     \gloss{‘separated’}  &  \\
\end{tabular}
%\caption{\nocaption}
     
\begin{tabular}{llllll}  
  \multicolumn{5}{l}{
                     \vernacular{(443) /Ø/
                    C-Initial + OP} \gloss{‘s/he...him/her
                    \ob mu-\cb  / them
                    } } &  \\
\multicolumn{5}{l}{ } &  \\

                     \vernacular{
                    yamu[tsíí{\downstep}lí]}  &   
                     \gloss{‘went for’}  &  \\

                     \vernacular{
                    yamu[léshí]}  &   
                     \gloss{‘left’}  &  \\

                     \vernacular{
                    yamu[lóó{\downstep}ndí]}  &   
                     \gloss{‘followed’}  &  \\

                     \vernacular{
                    yamu[kúlíí{\downstep}shɪ́]}  &   
                     \gloss{‘named’}  &  \\

                     \vernacular{
                    yamu[lákhú{\downstep}úlí]}  &   
                     \gloss{‘released’}  &  \\

                     \vernacular{
                    yamu[séé{\downstep}búúlɪ́]}  &   
                     \gloss{‘said bye
                    to’}  &  \\

                     \vernacular{
                    yamu[hóómbé{\downstep}líítsɪ́]}  &   
                     \gloss{‘comforted’}  &  \\

                     \vernacular{
                    yamu[kálú{\downstep}shíítsɪ́]}  &   
                     \gloss{‘returned’}  &  \\

                     \vernacular{
                    yamu[síínjí{\downstep}líítsɪ́]}  &   
                     \gloss{
                    ‘made...stand’}  &  \\

                     \vernacular{
                    yamu[réébí{\downstep}réébí]}  &   
                     \gloss{‘asked
                    (iter)’}  &  \\

                     \vernacular{
                    yamu[kálúkhá{\downstep}nyíínyɪ́]}  &   
                     \gloss{
                    ‘turned...over’}  &  \\

                     \vernacular{
                    yabi[sébúlú{\downstep}khányíínyɪ́]}  &   
                     \gloss{‘scattered’}  &  \\
\end{tabular}
%\caption{\nocaption}
     
\begin{tabular}{llllll}  
  \multicolumn{5}{l}{
                     \vernacular{(444) /Ø/
                    V-Initial + OP} \gloss{‘s/he...him/her
                    \ob mw-\cb  / it
                    } } &  \\
\multicolumn{5}{l}{ } &  \\

                     \vernacular{
                    yamw[eé{\downstep}nyí]}  &   
                     \gloss{‘wanted’}  &     &   
                     \vernacular{
                    yamw[eé{\downstep}yéélɛ́]}  &   
                     \gloss{‘wiped for’}  &  \\

                     \vernacular{
                    yabw[ií{\downstep}lúúlí]}  &   
                     \gloss{‘winnowed’}  &     &   
                     \vernacular{
                    yamw[aá{\downstep}mbákháánɛ́]}  &   
                     \gloss{‘refused’}  &  \\

                     \vernacular{
                    yamw[eé{\downstep}léélíítsɪ́]}  &   
                     \gloss{
                    ‘carried...hanging’}  &  \\
\end{tabular}
%\caption{\nocaption}
     
\begin{tabular}{llllll}  
  \multicolumn{5}{l}{
                     \vernacular{(445) /H/
                    C-Initial + OP
                    } \gloss{
                    ‘s/he...me’} } &  \\
\multicolumn{5}{l}{ } &  \\

                     \vernacular{
                    yaa[rí{\downstep}ílí]}  &   
                     \gloss{‘feared’}  &     &   
                     \vernacular{
                    yaa[mbé{\downstep}chí]}  &   
                     \gloss{‘shaved’}  &  \\

                     \vernacular{
                    yaa[ndé{\downstep}érí]}  &   
                     \gloss{‘brought’}  &     &   
                     \vernacular{
                    yaa[khá{\downstep}lááchɛ́]}  &   
                     \gloss{‘cut’}  &  \\

                     \vernacular{
                    yaa[sí{\downstep}tááchí]}  &   
                     \gloss{‘accused’}  &     &   
                     \vernacular{
                    yaa[mbó{\downstep}ólíítsɪ́]}  &   
                     \gloss{‘seduced’}  &  \\

                     \vernacular{
                    yaa[khó{\downstep}ng’óóndí]}  &   
                     \gloss{‘knocked’}  &     &   
                     \vernacular{
                    yaa[mbó{\downstep}hólóólɛ́]}  &   
                     \gloss{‘untied’}  &  \\

                     \vernacular{
                    yaa[mbó{\downstep}yóng’áánɛ́]}  &   
                     \gloss{‘went
                    around’}  &     &   
                     \vernacular{
                    yaa[ng’ó{\downstep}ng’óólíítsɪ́]}  &   
                     \gloss{‘teased’}  &  \\

                     \vernacular{
                    yaa[ní{\downstep}ngákányíínyɪ́]}  &   
                     \gloss{‘bent’}  &  \\
\end{tabular}
%\caption{\nocaption}
     
\begin{tabular}{llllll}  
  \multicolumn{5}{l}{
                     \vernacular{(446) /H/
                    V-Initial + OP
                    } \gloss{
                    ‘s/he...me’} } &  \\
\multicolumn{5}{l}{ } &  \\

                     \vernacular{
                    yaa[nzí{\downstep}rí]}  &   
                     \gloss{‘killed’}  &     &   
                     \vernacular{
                    yaa[nzí{\downstep}kóó{\downstep}mbí]}  &   
                     \gloss{‘admired’}  &  \\

                     \vernacular{
                    yaa[nzí{\downstep}síá{\downstep}chí]}  &   
                     \gloss{‘smacked’}  &     &   
                     \vernacular{
                    yaa[nzó{\downstep}nónyíínyɪ́]}  &   
                     \gloss{‘spoiled’}  &  \\

                     \vernacular{
                    yaa[nzá{\downstep}búkhányíínyɪ́]}  &   
                     \gloss{‘separated’}  &  \\
\end{tabular}
%\caption{\nocaption}
     
\begin{tabular}{llllll}  
  \multicolumn{5}{l}{
                     \vernacular{(447) /Ø/
                    C-Initial + OP
                    } \gloss{
                    ‘s/he...me’} } &  \\
\multicolumn{5}{l}{ } &  \\

                     \vernacular{
                    yaa[síé{\downstep}lé]}  &   
                     \gloss{‘ground’}  &     &   
                     \vernacular{
                    yaa[ndéshí]}  &   
                     \gloss{‘left’}  &  \\

                     \vernacular{
                    yaa[nóó{\downstep}ndí]}  &   
                     \gloss{‘followed’}  &     &   
                     \vernacular{
                    yaa[ngúlíí{\downstep}shɪ́]}  &   
                     \gloss{‘named’}  &  \\

                     \vernacular{
                    yaa[ndákhú{\downstep}úlí]}  &   
                     \gloss{‘released’}  &     &   
                     \vernacular{
                    yaa[séé{\downstep}búúlɪ́]}  &   
                     \gloss{‘said bye
                    to’}  &  \\

                     \vernacular{
                    yaa[mbóómbé{\downstep}líítsɪ́]}  &   
                     \gloss{‘comforted’}  &     &   
                     \vernacular{
                    yaa[síínjí{\downstep}líítsɪ́]}  &   
                     \gloss{
                    ‘made..stand’}  &  \\

                     \vernacular{
                    yaa[ndéébí{\downstep}ndéébí]}  &   
                     \gloss{‘asked
                    (iter)’}  &     &   
                     \vernacular{
                    yaa[ngálúkhá{\downstep}nyíínyɪ́]}  &   
                     \gloss{
                    ‘turned...over’}  &  \\
\end{tabular}
%\caption{\nocaption}
     
\begin{tabular}{llllll}  
  \multicolumn{5}{l}{
                     \vernacular{(448) /Ø/
                    V-Initial + OP
                    } \gloss{
                    ‘s/he...me’} } &  \\
\multicolumn{5}{l}{ } &  \\

                     \vernacular{
                    yaa[nzényí]}  &   
                     \gloss{‘wanted’}  &     &   
                     \vernacular{
                    yaa[nzéyé{\downstep}élɛ́]}  &   
                     \gloss{‘wiped for’}  &  \\

                     \vernacular{
                    yaa[nyámbákhá{\downstep}ánɛ́]}  &   
                     \gloss{‘refused’}  &     &   
                     \vernacular{
                    yaa[nzélé{\downstep}élíítsɪ́]}  &   
                     \gloss{
                    ‘carried...hanging’}  &  \\
\end{tabular}
%\caption{\nocaption}
     
\begin{tabular}{llllll}  
  \multicolumn{5}{l}{
                     \vernacular{(449) /H/
                    C-Initial + OP
                    } \gloss{
                    ‘s/he...him/herself’} } &  \\
\multicolumn{5}{l}{ } &  \\

                     \vernacular{
                    yayi[ré{\downstep}élé]}  &   
                     \gloss{‘buried’}  &     &   
                     \vernacular{
                    yayi[bé{\downstep}chí]}  &   
                     \gloss{‘shaved’}  &  \\

                     \vernacular{
                    yayi[sú{\downstep}únjí]}  &   
                     \gloss{‘hung’}  &     &   
                     \vernacular{
                    yayi[khá{\downstep}lááchɛ́]}  &   
                     \gloss{‘cut’}  &  \\

                     \vernacular{
                    yayi[sí{\downstep}tááchí]}  &   
                     \gloss{‘accused’}  &     &   
                     \vernacular{
                    yayi[sá{\downstep}ándíítsɪ́]}  &   
                     \gloss{‘thanked’}  &  \\

                     \vernacular{
                    yayi[khó{\downstep}ng’óóndí]}  &   
                     \gloss{‘knocked’}  &     &   
                     \vernacular{
                    yayi[bó{\downstep}hólóólɛ́]}  &   
                     \gloss{‘untied’}  &  \\
\end{tabular}
%\caption{\nocaption}
     
\begin{tabular}{llllll}  
  \multicolumn{5}{l}{
                     \vernacular{(450) /H/
                    V-Initial + OP
                    } \gloss{
                    ‘s/he...him/herself’} } &  \\
\multicolumn{5}{l}{ } &  \\

                     \vernacular{
                    yayi[yí{\downstep}rí]}  &   
                     \gloss{‘killed’}  &     &   
                     \vernacular{
                    yayi[yí{\downstep}kóó{\downstep}mbí]}  &   
                     \gloss{‘admired’}  &  \\

                     \vernacular{
                    yayi[yí{\downstep}síá{\downstep}chí]}  &   
                     \gloss{‘smacked’}  &     &   
                     \vernacular{
                    yayi[yó{\downstep}nónyíínyɪ́]}  &   
                     \gloss{‘spoiled’}  &  \\

                     \vernacular{
                    yayi[yá{\downstep}búkhányíínyɪ́]}  &   
                     \gloss{‘separated’}  &  \\
\end{tabular}
%\caption{\nocaption}
     
\begin{tabular}{llllll}  
  \multicolumn{5}{l}{
                     \vernacular{(451) /Ø/
                    C-Initial + OP
                    } \gloss{
                    ‘s/he...him/herself’} } &  \\
\multicolumn{5}{l}{ } &  \\

                     \vernacular{
                    yayi[síé{\downstep}lé]}  &   
                     \gloss{‘ground’}  &     &   
                     \vernacular{
                    yayi[léshí]}  &   
                     \gloss{‘left’}  &  \\

                     \vernacular{
                    yayi[síí{\downstep}njí]}  &   
                     \gloss{‘bathed’}  &     &   
                     \vernacular{
                    yayi[kúlíí{\downstep}shɪ́]}  &   
                     \gloss{‘named’}  &  \\

                     \vernacular{
                    yayi[náá{\downstep}búúlɪ́]}  &   
                     \gloss{‘undressed’}  &     &   
                     \vernacular{
                    yayi[lákhú{\downstep}úlí]}  &   
                     \gloss{‘released’}  &  \\

                     \vernacular{
                    yayi[hóómbé{\downstep}líítsɪ́]}  &   
                     \gloss{‘comforted’}  &     &   
                     \vernacular{
                    yayi[síínjí{\downstep}líítsɪ́]}  &   
                     \gloss{
                    ‘made...stand’}  &  \\

                     \vernacular{
                    yayi[réébí{\downstep}réébí]}  &   
                     \gloss{‘asked
                    (iter)’}  &     &   
                     \vernacular{
                    yayi[kálúkhá{\downstep}nyíínyɪ́]}  &   
                     \gloss{
                    ‘turned...over’}  &  \\
\end{tabular}
%\caption{\nocaption}
     
\begin{tabular}{llllll}  
  \multicolumn{5}{l}{
                     \vernacular{(452) /Ø/
                    V-Initial + OP
                    } \gloss{
                    ‘s/he...him/herself’} } &  \\
\multicolumn{5}{l}{ } &  \\

                     \vernacular{
                    yayi[yálí]}  &   
                     \gloss{‘exposed’}  &     &   
                     \vernacular{
                    yayi[yéyéé{\downstep}lɛ́]}  &   
                     \gloss{‘wiped for’}  &  \\

                     \vernacular{
                    yayi[yámbákhá{\downstep}ánɛ́]}  &   
                     \gloss{‘refused’}  &     &   
                     \vernacular{
                    yayi[yélé{\downstep}élíítsɪ́]}  &   
                     \gloss{‘hung...up’}  &  \\
\end{tabular}
%\caption{\nocaption}
     
\begin{tabular}{llllll}  
  \multicolumn{5}{l}{
                     \vernacular{(453) /H/
                    C-Initial + OP + OP
                    } \gloss{‘s/he...him/her
                    for me’} } &  \\
\multicolumn{5}{l}{ } &  \\

                     \vernacular{
                    yamuu[ndé{\downstep}éléélɛ́]}  &   
                     \gloss{‘buried’}  &     &   
                     \vernacular{
                    yamuu[mbé{\downstep}chéélɛ́]}  &   
                     \gloss{‘shaved’}  &  \\

                     \vernacular{
                    yamuu[ndé{\downstep}éréélɛ́]}  &   
                     \gloss{‘brought’}  &     &   
                     \vernacular{
                    yamuu[khá{\downstep}láchíílɪ́]}  &   
                     \gloss{‘cut’}  &  \\

                     \vernacular{
                    yamuu[sí{\downstep}tááchíílɪ́]}  &   
                     \gloss{‘accused’}  &     &   
                     \vernacular{
                    yamuu[mbó{\downstep}ólítsíílɪ́]}  &   
                     \gloss{‘seduced’}  &  \\

                     \vernacular{
                    yamuu[mbó{\downstep}hólóléélɛ́]}  &   
                     \gloss{‘untied’}  &     &     &     &  \\
\end{tabular}
%\caption{\nocaption}
     
\begin{tabular}{llllll}  
  \multicolumn{5}{l}{
                     \vernacular{(454) /H/
                    V-Initial + OP + OP
                    } \gloss{‘s/he...him/her
                    for me’} } &  \\
\multicolumn{5}{l}{ } &  \\

                     \vernacular{
                    yamuú[{\downstep}nzííríílɪ́]}  &   
                     \gloss{‘killed’}  &     &   
                     \vernacular{
                    yamuú[{\downstep}nzéchítsíílɪ́]}  &   
                     \gloss{‘admired’}  &  \\

                     \vernacular{
                    yamuú[{\downstep}nzísíá{\downstep}chíílɪ́]}  &   
                     \gloss{‘smacked’}  &     &   
                     \vernacular{
                    yamuú[{\downstep}nzónónyínyíílɪ́]}  &   
                     \gloss{‘spoiled’}  &  \\

                     \vernacular{
                    yamuú[{\downstep}nzábúkhányínyíílɪ́]}  &   
                     \gloss{‘separated’}  &     &     &     &  \\
\end{tabular}
%\caption{\nocaption}
     
\begin{tabular}{llllll}  
  \multicolumn{5}{l}{
                     \vernacular{(455) /Ø/
                    C-Initial + OP + OP
                    } \gloss{‘s/he...him/her
                    for me’} } &  \\
\multicolumn{5}{l}{ } &  \\

                     \vernacular{
                    yamuú[{\downstep}nzíílí{\downstep}ílɪ́]}  &   
                     \gloss{‘went for’}  &     &   
                     \vernacular{
                    yamuú[{\downstep}ndéshé{\downstep}élɛ́]}  &   
                     \gloss{‘went for’}  &  \\

                     \vernacular{
                    yamuú[{\downstep}nóóndé{\downstep}élɛ́]}  &   
                     \gloss{‘left’}  &     &   
                     \vernacular{
                    yamuú[{\downstep}ngúlíshí{\downstep}ílɪ́]}  &   
                     \gloss{‘followed’}  &  \\

                     \vernacular{
                    yamuú[{\downstep}ndákhú{\downstep}úlíílɪ́]}  &   
                     \gloss{‘named’}  &     &   
                     \vernacular{
                    yamuú[{\downstep}séébú{\downstep}líílɪ́]}  &   
                     \gloss{‘released’}  &  \\

                     \vernacular{
                    yamuú[{\downstep}mbóómbé{\downstep}lítsíílɪ́]}  &   
                     \gloss{‘said bye
                    to’}  &     &   
                     \vernacular{
                    yamuú[{\downstep}síínjí{\downstep}lítsíílɪ́]}  &   
                     \gloss{‘comforted’}  &  \\
\end{tabular}
%\caption{\nocaption}
     
\begin{tabular}{llllll}  
  \multicolumn{5}{l}{
                     \vernacular{(456) /Ø/
                    V-Initial + OP + OP
                    } \gloss{‘s/he...him/her
                    \ob mu-\cb  / it
                    } } &  \\
\multicolumn{5}{l}{ } &  \\

                     \vernacular{
                    yamuú[{\downstep}nzéyéé{\downstep}lɛ́]}  &   
                     \gloss{‘wiped’}  &     &   
                     \vernacular{
                    yakuú[{\downstep}nzáshítsí{\downstep}ílɪ́]}  &   
                     \gloss{‘lit’}  &  \\

                     \vernacular{
                    yabuú[{\downstep}nzílú{\downstep}úlíílɪ́]}  &   
                     \gloss{‘winnowed’}  &     &   
                     \vernacular{
                    yakuú[{\downstep}nzélé{\downstep}élítsíílɪ́]}  &   
                     \gloss{‘hung’}  &  \\
\end{tabular}
%\caption{\nocaption}
     
\begin{tabular}{lll}  
  \multicolumn{2}{l}{
                     \vernacular{(457) /H/
                    C-Initial Phrase-Medial} \gloss{‘s/he...the boy
                    \ob mú{\downstep}yáyi\cb  /} } &  \\
\multicolumn{2}{l}{
                     \gloss{someone
                    \ob muundu\cb ’} } &  \\

                     \vernacular{ya[reele]
                    mú{\downstep}yáyi/muundu}  &   
                     \gloss{‘buried’}  &  \\

                     \vernacular{ya[bechi]
                    mú{\downstep}yáyi/muundu}  &   
                     \gloss{‘shaved’}  &  \\

                     \vernacular{ya[leeri]
                    mú{\downstep}yáyi/muundu}  &   
                     \gloss{‘brought’}  &  \\

                     \vernacular{ya[khalaachɛ]
                    mú{\downstep}yáyi/muundu}  &   
                     \gloss{‘cut’}  &  \\

                     \vernacular{ya[sitaachi]
                    mú{\downstep}yáyi/muundu}  &   
                     \gloss{‘accused’}  &  \\

                     \vernacular{ya[booliitsɪ]
                    mú{\downstep}yáyi/muundu}  &   
                     \gloss{‘seduced’}  &  \\

                     \vernacular{ya[khong’oondi]
                    mú{\downstep}yáyi/muundu}  &   
                     \gloss{‘knocked’}  &  \\

                     \vernacular{ya[boholoolɛ]
                    mú{\downstep}yáyi/muundu}  &   
                     \gloss{‘untied’}  &  \\

                     \vernacular{ya[boyong’aanɛ]
                    mú{\downstep}yáyi/muundu}  &   
                     \gloss{‘went
                    around’}  &  \\
\end{tabular}
%\caption{\nocaption}
     
\begin{tabular}{lll}  
  \multicolumn{2}{l}{
                     \vernacular{(458) /Ø/
                    C-Initial Phrase-Medial} \gloss{‘s/he...the boy
                    \ob mú{\downstep}yáyi\cb  / the man \ob musáatsa\cb  /} } &  \\
\multicolumn{2}{l}{
                     \gloss{the cook
                    \ob mutéeshi\cb  / someone \ob muundu\cb ’} } &  \\

                     \vernacular{ya[tsiili]
                    mú{\downstep}yáyi/musáatsa/mutéeshi/muundu}  &   
                     \gloss{‘went for’}  &  \\

                     \vernacular{ya[leshi]
                    mú{\downstep}yáyi/musáatsa/mutéeshi/muundu}  &   
                     \gloss{‘left’}  &  \\

                     \vernacular{ya[loondi]
                    mú{\downstep}yáyi/musáatsa/mutéeshi/muundu}  &   
                     \gloss{‘followed’}  &  \\

                     \vernacular{ya[kuliishɪ]
                    mú{\downstep}yáyi/musáatsa/mutéeshi/muundu}  &   
                     \gloss{‘named’}  &  \\

                     \vernacular{ya[lakhuuli]
                    mú{\downstep}yáyi/musáatsa/mutéeshi/muundu}  &   
                     \gloss{‘released’}  &  \\

                     \vernacular{ya[seebuulɪ]
                    mú{\downstep}yáyi/musáatsa/mutéeshi/muundu}  &   
                     \gloss{‘said bye
                    to’}  &  \\

                     \vernacular{ya[kalushiitsɪ]
                    mú{\downstep}yáyi/musáatsa/mutéeshi/muundu}  &   
                     \gloss{‘returned’}  &  \\

                     \vernacular{ya[reebireebi]
                    mú{\downstep}yáyi/musáatsa/mutéeshi/muundu}  &   
                     \gloss{‘asked
                    (iter)’}  &  \\

                     \vernacular{ya[kalukhanyiinyɪ]
                    mú{\downstep}yáyi/musáatsa/mutéeshi/muundu}  &   
                     \gloss{
                    ‘turned...over’}  &  \\
\end{tabular}
%\caption{\nocaption}
     
\begin{tabular}{lll}  
  \multicolumn{2}{l}{
                     \vernacular{(459) /H/
                    C-Initial +OP Phrase-Medial} \gloss{‘s/he...the boy
                    \ob mú{\downstep}yáyi\cb  /} } &  \\
\multicolumn{2}{l}{
                     \gloss{someone \ob muundu\cb 
                    for him/her’} } &  \\

                     \vernacular{yamu[réeleelɛ]
                    mú{\downstep}yáyi/muundu}  &   
                     \gloss{‘buried’}  &  \\

                     \vernacular{yamu[bécheelɛ]
                    mú{\downstep}yáyi/muundu}  &   
                     \gloss{‘shaved’}  &  \\

                     \vernacular{yamu[léereelɛ]
                    mú{\downstep}yáyi/muundu}  &   
                     \gloss{‘brought’}  &  \\

                     \vernacular{yamu[khálachiilɪ]
                    mú{\downstep}yáyi/muundu}  &   
                     \gloss{‘cut’}  &  \\

                     \vernacular{yamu[sítaachiilɪ]
                    mú{\downstep}yáyi/muundu}  &   
                     \gloss{‘accused’}  &  \\

                     \vernacular{yamu[bóolitsiilɪ]
                    mú{\downstep}yáyi/muundu}  &   
                     \gloss{‘seduced’}  &  \\

                     \vernacular{
                    yamu[khóng’oondeelɛ]
                    mú{\downstep}yáyi/muundu}  &   
                     \gloss{‘knocked’}  &  \\

                     \vernacular{yamu[bóhololeelɛ]
                    mú{\downstep}yáyi/muundu}  &   
                     \gloss{‘untied’}  &  \\

                     \vernacular{
                    yamu[bóyong’aniilɪ]
                    mú{\downstep}yáyi/muundu}  &   
                     \gloss{‘went
                    around’}  &  \\
\end{tabular}
%\caption{\nocaption}
     
\begin{tabular}{lll}  
  \multicolumn{2}{l}{
                     \vernacular{(460) /Ø/
                    C-Initial +OP Phrase-Medial} \gloss{‘s/he...the boy
                    \ob mú{\downstep}yáyi\cb  /} } &  \\
\multicolumn{2}{l}{
                     \gloss{someone \ob muundu\cb 
                    for him/her’} } &  \\

                     \vernacular{yamu[tsiiliilɪ]
                    mú{\downstep}yáyi/muundu}  &   
                     \gloss{‘went for’}  &  \\

                     \vernacular{yamu[lesheelɛ]
                    mú{\downstep}yáyi/muundu}  &   
                     \gloss{‘left’}  &  \\

                     \vernacular{yamu[loondeelɛ]
                    mú{\downstep}yáyi/muundu}  &   
                     \gloss{‘followed’}  &  \\

                     \vernacular{yamu[kulishiilɪ]
                    mú{\downstep}yáyi/muundu}  &   
                     \gloss{‘named’}  &  \\

                     \vernacular{yamu[lakhuuliilɪ]
                    mú{\downstep}yáyi/muundu}  &   
                     \gloss{‘released’}  &  \\

                     \vernacular{yamu[seebuliilɪ]
                    mú{\downstep}yáyi/muundu}  &   
                     \gloss{‘said bye
                    to’}  &  \\

                     \vernacular{
                    yamu[kalushitsiilɪ] mú{\downstep}yáyi/muundu}  &   
                     \gloss{‘returned’}  &  \\

                     \vernacular{yamu[reebireebi]
                    mú{\downstep}yáyi/muundu}  &   
                     \gloss{‘asked
                    (iter)’}  &  \\

                     \vernacular{
                    yamu[kalukhanyinyiilɪ]
                    mú{\downstep}yáyi/muundu}  &   
                     \gloss{
                    ‘turned...over’}  &  \\
\end{tabular}
%\caption{\nocaption}
     
\begin{tabular}{lll}  
  \multicolumn{2}{l}{
                     \vernacular{(461) /H/
                    C-Initial +OP + OP
                    } \gloss{‘s/he...the boy
                    \ob mú{\downstep}yáyi\cb  /} } &  \\
\multicolumn{2}{l}{
                     \gloss{someone \ob muundu\cb 
                    for him/her for me’} } &  \\

                     \vernacular{yamuú[ndeeleelɛ]
                    mú{\downstep}yáyi/muundu}  &   
                     \gloss{‘buried’}  &  \\

                     \vernacular{yamuú[mbecheelɛ]
                    mú{\downstep}yáyi/muundu}  &   
                     \gloss{‘shaved’}  &  \\

                     \vernacular{yamuú[ndeereelɛ]
                    mú{\downstep}yáyi/muundu}  &   
                     \gloss{‘brought’}  &  \\

                     \vernacular{
                    yamuú[khalachiilɪ] mú{\downstep}yáyi/muundu}  &   
                     \gloss{‘cut’}  &  \\

                     \vernacular{
                    yamuú[sitaachiilɪ] mú{\downstep}yáyi/muundu}  &   
                     \gloss{‘accused’}  &  \\

                     \vernacular{
                    yamuú[mboolitsiilɪ]
                    mú{\downstep}yáyi/muundu}  &   
                     \gloss{‘seduced’}  &  \\

                     \vernacular{
                    yamuú[mbohololeelɛ]
                    mú{\downstep}yáyi/muundu}  &   
                     \gloss{‘untied’}  &  \\
\end{tabular}
%\caption{\nocaption}
     
\begin{tabular}{lll}  
  \multicolumn{2}{l}{
                     \vernacular{(462) /Ø/
                    C-Initial +OP + OP
                    } \gloss{‘s/he...the boy
                    \ob mú{\downstep}yáyi\cb  /} } &  \\
\multicolumn{2}{l}{
                     \gloss{someone \ob muundu\cb 
                    for him/her for me’} } &  \\

                     \vernacular{yamuú[nziiliilɪ]
                    mú{\downstep}yáyi/muundu}  &   
                     \gloss{‘went for’}  &  \\

                     \vernacular{yamuú[ndesheelɛ]
                    mú{\downstep}yáyi/muundu}  &   
                     \gloss{‘left’}  &  \\

                     \vernacular{yamuú[noondeelɛ]
                    mú{\downstep}yáyi/muundu}  &   
                     \gloss{‘followed’}  &  \\

                     \vernacular{
                    yamuú[ngulishiilɪ] mú{\downstep}yáyi/muundu}  &   
                     \gloss{‘named’}  &  \\

                     \vernacular{
                    yamuú[ndakhuuliilɪ]
                    mú{\downstep}yáyi/muundu}  &   
                     \gloss{‘released’}  &  \\

                     \vernacular{yamuú[seebuliilɪ]
                    mú{\downstep}yáyi/muundu}  &   
                     \gloss{‘said bye
                    to’}  &  \\

                     \vernacular{
                    yamuú[siinjilitsiilɪ]
                    mú{\downstep}yáyi/muundu}  &   
                     \gloss{
                    ‘made...stand’}  &  \\
\end{tabular}
%\caption{\nocaption}
    

\subsection{Hesternal Perfective Negative: Pattern
              7}\label{sec:sHestPerfNeg}


\begin{tabular}{llllll}  
  \multicolumn{5}{l}{
                     \vernacular{(463) /H/
                    C-Initial} \gloss{‘s/he did
                    not...’} } &  \\
\multicolumn{5}{l}{ } &  \\

                     \vernacular{ya[réélé]
                    {\downstep}tá}  &   
                     \gloss{‘bury’}  &     &   
                     \vernacular{ya[ng’wéélé]
                    {\downstep}tá}  &   
                     \gloss{‘drink’}  &  \\

                     \vernacular{ya[khwéélé]
                    {\downstep}tá}  &   
                     \gloss{‘eat’}  &     &   
                     \vernacular{ya[líílí]
                    {\downstep}tá}  &   
                     \gloss{‘pay dowry’} \footnote{\label{fn:nHeDidNotCry} The first recording of this paradigm
                      includes a production of the segmentally
                      identical form meaning to ‘s/he did not cry’
                      ya[liíli] tá, from the /Ø/ disyllabic verb,
                      khu[liila] ‘to cry’. 


}%
 &  \\

                     \vernacular{ya[lúmí]
                    {\downstep}tá}  &   
                     \gloss{‘bite’}  &     &   
                     \vernacular{ya[béchí]
                    {\downstep}tá}  &   
                     \gloss{‘shave’}  &  \\

                     \vernacular{ya[tééshí]
                    {\downstep}tá}  &   
                     \gloss{‘cook’}  &     &   
                     \vernacular{ya[léérí]
                    {\downstep}tá}  &   
                     \gloss{‘bring’}  &  \\

                     \vernacular{ya[khálááchɛ́]
                    {\downstep}tá}  &   
                     \gloss{‘cut’}  &     &   
                     \vernacular{ya[káláánjí]
                    {\downstep}tá}  &   
                     \gloss{‘fry’}  &  \\

                     \vernacular{ya[sítááchí]
                    {\downstep}tá}  &   
                     \gloss{‘accuse’}  &     &   
                     \vernacular{ya[bóólíítsɪ́]
                    {\downstep}tá}  &   
                     \gloss{‘seduce’}  &  \\

                     \vernacular{
                    ya[sáándíítsɪ́] {\downstep}tá}  &   
                     \gloss{‘thank’}  &     &   
                     \vernacular{
                    ya[khóng’óóndí] {\downstep}tá}  &   
                     \gloss{‘knock’}  &  \\

                     \vernacular{ya[bóhólóólɛ́]
                    {\downstep}tá}  &   
                     \gloss{‘untie’}  &     &   
                     \vernacular{
                    ya[bóyóng’áánɛ́] {\downstep}tá}  &   
                     \gloss{‘go around’}  &  \\

                     \vernacular{
                    ya[ng’óng’óólíítsɪ́] {\downstep}tá}  &   
                     \gloss{‘tease’}  &     &   
                     \vernacular{
                    ya[líng(ák)ányíínyɪ́] {\downstep}tá}  &   
                     \gloss{‘crumple’}  &  \\
\end{tabular}
%\caption{\nocaption}
     
\begin{tabular}{llllll}  
  \multicolumn{5}{l}{
                     \vernacular{(464) /Ø/
                    C-Initial} \gloss{‘s/he did
                    not...’} \footnote{\label{fn:nHestNegNoHonFV} While the negative element tá is
                      transcribed as realizing a downstepped H, few
                      of Burula’s productions appear to produced in
                      this way. Instead, tá appears often to be
                      incorporated into the latter H span of the
                      verb, e.g., ya[lákhú{\downstep}úlí] tá. 


}%
} &  \\
\multicolumn{5}{l}{ } &  \\

                     \vernacular{ya[tsíí{\downstep}lí]
                    {\downstep}tá}  &   
                     \gloss{‘go’}  &     &   
                     \vernacular{ya[kwíí{\downstep}lí]
                    {\downstep}tá}  &   
                     \gloss{‘fall’}  &  \\

                     \vernacular{ya[léshí]
                    {\downstep}tá}  &   
                     \gloss{‘leave’}  &     &   
                     \vernacular{ya[réé{\downstep}bí]
                    {\downstep}tá}  &   
                     \gloss{‘ask’}  &  \\

                     \vernacular{ya[lóó{\downstep}ndí]
                    {\downstep}tá}  &   
                     \gloss{‘follow’}  &     &   
                     \vernacular{ya[kúmíí{\downstep}lɪ́]
                    {\downstep}tá}  &   
                     \gloss{‘hold’}  &  \\

                     \vernacular{ya[kúlíí{\downstep}shɪ́]
                    {\downstep}tá}  &   
                     \gloss{‘name’}  &     &   
                     \vernacular{ya[hómó{\downstep}ólí]
                    {\downstep}tá}  &   
                     \gloss{‘massage’}  &  \\

                     \vernacular{ya[lákhú{\downstep}úlí]
                    {\downstep}tá}  &   
                     \gloss{‘release’}  &     &   
                     \vernacular{ya[séébú{\downstep}úlɪ́]
                    {\downstep}tá}  &   
                     \gloss{‘say bye’}  &  \\

                     \vernacular{
                    ya[hóómbé{\downstep}líítsɪ́] {\downstep}tá}  &   
                     \gloss{‘comfort’}  &     &   
                     \vernacular{
                    ya[kálúshí{\downstep}ítsɪ́] {\downstep}tá}  &   
                     \gloss{‘return’}  &  \\

                     \vernacular{
                    ya[síínjí{\downstep}líítsɪ́] {\downstep}tá}  &   
                     \gloss{‘make stand’}  &     &   
                     \vernacular{
                    ya[réébí{\downstep}réébí] {\downstep}tá}  &   
                     \gloss{‘ask (iter)’}  &  \\

                     \vernacular{
                    ya[kálúkhá{\downstep}nyíínyɪ́] {\downstep}tá}  &   
                     \gloss{‘turn over’}  &     &   
                     \vernacular{
                    ya[sébúlú{\downstep}khányíínyɪ́] {\downstep}tá}  &   
                     \gloss{‘scatter’}  &  \\
\end{tabular}
%\caption{\nocaption}
     
\begin{tabular}{llllll}  
  \multicolumn{5}{l}{
                     \vernacular{(465) /H/
                    C-Initial + OP} \gloss{‘s/he did
                    not...him/her’} } &  \\
\multicolumn{5}{l}{ } &  \\

                     \vernacular{yamu[ré{\downstep}élé]
                    {\downstep}tá}  &   
                     \gloss{‘bury’}  &     &   
                     \vernacular{yamu[bé{\downstep}chí]
                    {\downstep}tá}  &   
                     \gloss{‘shave’}  &  \\

                     \vernacular{yamu[lé{\downstep}érí]
                    {\downstep}tá}  &   
                     \gloss{‘bring’}  &     &   
                     \vernacular{
                    yamu[khá{\downstep}lááchɛ́] {\downstep}tá}  &   
                     \gloss{‘cut’}  &  \\

                     \vernacular{
                    yamu[sí{\downstep}tááchí] {\downstep}tá}  &   
                     \gloss{‘accuse’}  &     &   
                     \vernacular{
                    yamu[bó{\downstep}ólíítsɪ́] {\downstep}tá}  &   
                     \gloss{‘seduce’}  &  \\

                     \vernacular{
                    yamu[khó{\downstep}ng’óóndí] {\downstep}tá}  &   
                     \gloss{‘knock’}  &     &   
                     \vernacular{
                    yamu[bó{\downstep}hólóólɛ́] {\downstep}tá}  &   
                     \gloss{‘untie’}  &  \\

                     \vernacular{
                    yamu[bó{\downstep}yóng’áánɛ́] {\downstep}tá}  &   
                     \gloss{‘go around’}  &     &   
                     \vernacular{
                    yamu[ng’ó{\downstep}ng’óólíítsɪ́] {\downstep}tá}  &   
                     \gloss{‘tease’}  &  \\

                     \vernacular{
                    yamu[lí{\downstep}ngákányíínyɪ́] {\downstep}tá}  &   
                     \gloss{‘bend’}  &     &     &     &  \\
\end{tabular}
%\caption{\nocaption}
     
\begin{tabular}{llllll}  
  \multicolumn{5}{l}{
                     \vernacular{(466) /Ø/
                    C-Initial + OP} \gloss{‘s/he did
                    not...him/her \ob mu-\cb  / them
                    } } &  \\
\multicolumn{5}{l}{ } &  \\

                     \vernacular{yamu[tsíí{\downstep}lí]
                    {\downstep}tá}  &   
                     \gloss{‘go for’}  &  \\

                     \vernacular{yamu[léshí]
                    {\downstep}tá}  &   
                     \gloss{‘leave’}  &  \\

                     \vernacular{yamu[lóó{\downstep}ndí]
                    {\downstep}tá}  &   
                     \gloss{‘follow’}  &  \\

                     \vernacular{
                    yamu[kúlíí{\downstep}shɪ́] {\downstep}tá}  &   
                     \gloss{‘name’}  &  \\

                     \vernacular{
                    yamu[lákhú{\downstep}úlí] {\downstep}tá}  &   
                     \gloss{‘release’}  &  \\

                     \vernacular{
                    yamu[séé{\downstep}búúlɪ́] {\downstep}tá}  &   
                     \gloss{‘say bye to’}  &  \\

                     \vernacular{
                    yamu[hóómbé{\downstep}líítsɪ́] {\downstep}tá}  &   
                     \gloss{‘comfort’}  &  \\

                     \vernacular{
                    yamu[kálú{\downstep}shíítsɪ́] {\downstep}tá}  &   
                     \gloss{‘return’}  &  \\

                     \vernacular{
                    yamu[síínjí{\downstep}líítsɪ́] {\downstep}tá}  &   
                     \gloss{
                    ‘make...stand’}  &  \\

                     \vernacular{
                    yamu[réébí{\downstep}réébí] {\downstep}tá}  &   
                     \gloss{‘ask (iter)’}  &  \\

                     \vernacular{
                    yamu[kálúkhá{\downstep}nyíínyɪ́] {\downstep}tá}  &   
                     \gloss{
                    ‘turn...over’}  &  \\

                     \vernacular{
                    yabi[sébúlú{\downstep}khányíínyɪ́] {\downstep}tá}  &   
                     \gloss{‘scatter’}  &  \\
\end{tabular}
%\caption{\nocaption}
     
\begin{tabular}{llllll}  
  \multicolumn{5}{l}{
                     \vernacular{(467) /H/
                    C-Initial + OP + OP
                    } \gloss{‘s/he did
                    not...him/her for me’} } &  \\
\multicolumn{5}{l}{ } &  \\

                     \vernacular{
                    yamuu[ndé{\downstep}éléélɛ́] {\downstep}tá}  &   
                     \gloss{‘bury’}  &     &   
                     \vernacular{
                    yamuu[mbé{\downstep}chéélɛ́] {\downstep}tá}  &   
                     \gloss{‘shave’}  &  \\

                     \vernacular{
                    yamuu[ndé{\downstep}éréélɛ́] {\downstep}tá}  &   
                     \gloss{‘bring’}  &     &   
                     \vernacular{
                    yamuu[khá{\downstep}láchíílɪ́] {\downstep}tá}  &   
                     \gloss{‘cut’}  &  \\

                     \vernacular{
                    yamuu[sí{\downstep}tááchíílɪ́] {\downstep}tá}  &   
                     \gloss{‘accuse’}  &     &   
                     \vernacular{
                    yamuu[mbó{\downstep}ólítsíílɪ́] {\downstep}tá}  &   
                     \gloss{‘seduce’}  &  \\

                     \vernacular{
                    yamuu[mbó{\downstep}hólóléélɛ́] {\downstep}tá}  &   
                     \gloss{‘untie’}  &     &     &     &  \\
\end{tabular}
%\caption{\nocaption}
     
\begin{tabular}{llllll}  
  \multicolumn{5}{l}{
                     \vernacular{(468) /Ø/
                    C-Initial + OP + OP
                    } \gloss{‘s/he did
                    not...him/her for me’} } &  \\
\multicolumn{5}{l}{ } &  \\

                     \vernacular{
                    yamuú[{\downstep}nzíílí{\downstep}ílɪ́] {\downstep}tá}  &   
                     \gloss{‘go for’}  &     &   
                     \vernacular{
                    yamuú[{\downstep}ndéshé{\downstep}élɛ́] {\downstep}tá}  &   
                     \gloss{‘go for’}  &  \\

                     \vernacular{
                    yamuú[{\downstep}nóóndé{\downstep}élɛ́] {\downstep}tá}  &   
                     \gloss{‘leave’}  &     &   
                     \vernacular{
                    yamuú[{\downstep}ngúlíshí{\downstep}ílɪ́] {\downstep}tá}  &   
                     \gloss{‘follow’}  &  \\

                     \vernacular{
                    yamuú[{\downstep}ndákhú{\downstep}úlíílɪ́] {\downstep}tá}  &   
                     \gloss{‘name’}  &     &   
                     \vernacular{
                    yamuú[{\downstep}séébú{\downstep}líílɪ́] {\downstep}tá}  &   
                     \gloss{‘release’}  &  \\

                     \vernacular{
                    yamuú[{\downstep}mbóómbé{\downstep}lítsíílɪ́]
                    {\downstep}tá}  &   
                     \gloss{‘say bye to’}  &     &   
                     \vernacular{
                    yamuú[{\downstep}síínjí{\downstep}lítsíílɪ́] {\downstep}tá}  &   
                     \gloss{‘comfort’}  &  \\
\end{tabular}
%\caption{\nocaption}
     
\begin{tabular}{lll}  
  \multicolumn{2}{l}{
                     \vernacular{(469) /H/
                    C-Initial Phrase-Medial} \gloss{‘s/he did
                    not...the boy \ob mú{\downstep}yáyi\cb  /} } &  \\
\multicolumn{2}{l}{
                     \gloss{someone
                    \ob muundu\cb ’} } &  \\

                     \vernacular{ya[reele]
                    mú{\downstep}yáyi/muundu tá}  &   
                     \gloss{‘bury’}  &  \\

                     \vernacular{ya[bechi]
                    mú{\downstep}yáyi/muundu tá}  &   
                     \gloss{‘shave’}  &  \\

                     \vernacular{ya[leeri]
                    mú{\downstep}yáyi/muundu tá}  &   
                     \gloss{‘bring’}  &  \\

                     \vernacular{ya[khalaachɛ]
                    mú{\downstep}yáyi/muundu tá}  &   
                     \gloss{‘cut’}  &  \\

                     \vernacular{ya[sitaachi]
                    mú{\downstep}yáyi/muundu tá}  &   
                     \gloss{‘accuse’}  &  \\

                     \vernacular{ya[booliitsɪ]
                    mú{\downstep}yáyi/muundu tá}  &   
                     \gloss{‘seduce’}  &  \\

                     \vernacular{ya[khong’oondi]
                    mú{\downstep}yáyi/muundu tá}  &   
                     \gloss{‘knock’}  &  \\

                     \vernacular{ya[boholoolɛ]
                    mú{\downstep}yáyi/muundu tá}  &   
                     \gloss{‘untie’}  &  \\

                     \vernacular{ya[boyong’aanɛ]
                    mú{\downstep}yáyi/muundu tá}  &   
                     \gloss{‘go around’}  &  \\
\end{tabular}
%\caption{\nocaption}
     
\begin{tabular}{lll}  
  \multicolumn{2}{l}{
                     \vernacular{(470) /Ø/
                    C-Initial Phrase-Medial} \gloss{‘s/he did
                    not...the boy \ob mú{\downstep}yáyi\cb } } &  \\
\multicolumn{2}{l}{
                     \gloss{someone
                    \ob muundu\cb ’} } &  \\

                     \vernacular{ya[tsiili]
                    mú{\downstep}yáyi/muundu tá}  &   
                     \gloss{‘go for’}  &  \\

                     \vernacular{ya[leshi]
                    mú{\downstep}yáyi/muundu tá}  &   
                     \gloss{‘leave’}  &  \\

                     \vernacular{ya[loondi]
                    mú{\downstep}yáyi/muundu tá}  &   
                     \gloss{‘follow’}  &  \\

                     \vernacular{ya[kuliishɪ]
                    mú{\downstep}yáyi/muundu tá}  &   
                     \gloss{‘name’}  &  \\

                     \vernacular{ya[lakhuuli]
                    mú{\downstep}yáyi/muundu tá}  &   
                     \gloss{‘release’}  &  \\

                     \vernacular{ya[seebuulɪ]
                    mú{\downstep}yáyi/muundu tá}  &   
                     \gloss{‘say bye to’}  &  \\

                     \vernacular{ya[kalushiitsɪ]
                    mú{\downstep}yáyi/muundu tá}  &   
                     \gloss{‘return’}  &  \\

                     \vernacular{ya[reebireebi]
                    mú{\downstep}yáyi/muundu tá}  &   
                     \gloss{‘ask (iter)’}  &  \\

                     \vernacular{ya[kalukhanyiinyɪ]
                    mú{\downstep}yáyi/muundu tá}  &   
                     \gloss{
                    ‘turn...over’}  &  \\
\end{tabular}
%\caption{\nocaption}
     
\begin{tabular}{lll}  
  \multicolumn{2}{l}{
                     \vernacular{(471) /H/
                    C-Initial +OP Phrase-Medial} \gloss{‘s/he did
                    not...the boy \ob mú{\downstep}yáyi\cb  /} } &  \\
\multicolumn{2}{l}{
                     \gloss{someone \ob muundu\cb 
                    for him/her’} } &  \\

                     \vernacular{yamu[réeleelɛ]
                    mú{\downstep}yáyi/muundu tá}  &   
                     \gloss{‘bury’}  &  \\

                     \vernacular{yamu[bécheelɛ]
                    mú{\downstep}yáyi/muundu tá}  &   
                     \gloss{‘shave’}  &  \\

                     \vernacular{yamu[léereelɛ]
                    mú{\downstep}yáyi/muundu tá}  &   
                     \gloss{‘bring’}  &  \\

                     \vernacular{yamu[khálachiilɪ]
                    mú{\downstep}yáyi/muundu tá}  &   
                     \gloss{‘cut’}  &  \\

                     \vernacular{yamu[sítaachiilɪ]
                    mú{\downstep}yáyi/muundu tá}  &   
                     \gloss{‘accuse’}  &  \\

                     \vernacular{yamu[bóolitsiilɪ]
                    mú{\downstep}yáyi/muundu tá}  &   
                     \gloss{‘seduce’}  &  \\

                     \vernacular{
                    yamu[khóng’oondeelɛ] mú{\downstep}yáyi/muundu
                    tá}  &   
                     \gloss{‘knock’}  &  \\

                     \vernacular{yamu[bóhololeelɛ]
                    mú{\downstep}yáyi/muundu tá}  &   
                     \gloss{‘untie’}  &  \\

                     \vernacular{
                    yamu[bóyong’aniilɪ] mú{\downstep}yáyi/muundu
                    tá}  &   
                     \gloss{‘go around’}  &  \\
\end{tabular}
%\caption{\nocaption}
     
\begin{tabular}{lll}  
  \multicolumn{2}{l}{
                     \vernacular{(472) /Ø/
                    C-Initial +OP Phrase-Medial} \gloss{‘s/he did
                    not...the boy \ob mú{\downstep}yáyi\cb  /} } &  \\
\multicolumn{2}{l}{
                     \gloss{the man
                    \ob musáatsa\cb  / someone \ob muundu\cb  for
                    him/her’} } &  \\

                     \vernacular{yamu[tsiiliilɪ]
                    mú{\downstep}yáyi/musáatsa/muundu tá}  &   
                     \gloss{‘go for’}  &  \\

                     \vernacular{yamu[lesheelɛ]
                    mú{\downstep}yáyi/musáatsa/muundu tá}  &   
                     \gloss{‘leave’}  &  \\

                     \vernacular{yamu[loondeelɛ]
                    mú{\downstep}yáyi/musáatsa/muundu tá}  &   
                     \gloss{‘follow’}  &  \\

                     \vernacular{yamu[kulishiilɪ]
                    mú{\downstep}yáyi/musáatsa/muundu tá}  &   
                     \gloss{‘name’}  &  \\

                     \vernacular{yamu[lakhuuliilɪ]
                    mú{\downstep}yáyi/musáatsa/muundu tá}  &   
                     \gloss{‘release’}  &  \\

                     \vernacular{yamu[seebuliilɪ]
                    mú{\downstep}yáyi/musáatsa/muundu tá}  &   
                     \gloss{‘say bye to’}  &  \\

                     \vernacular{
                    yamu[kalushitsiilɪ] mú{\downstep}yáyi/musáatsa/muundu
                    tá}  &   
                     \gloss{‘return’}  &  \\

                     \vernacular{yamu[reebireebi]
                    mú{\downstep}yáyi/musáatsa/muundu tá}  &   
                     \gloss{‘ask (iter)’}  &  \\

                     \vernacular{
                    yamu[kalukhanyinyiilɪ]
                    mú{\downstep}yáyi/musáatsa/muundu tá}  &   
                     \gloss{
                    ‘turn...over’}  &  \\
\end{tabular}
%\caption{\nocaption}
     
\begin{tabular}{lll}  
  \multicolumn{2}{l}{
                     \vernacular{(473) /H/
                    C-Initial +OP + OP
                    } \gloss{‘s/he did
                    not...the boy \ob mú{\downstep}yáyi\cb  /} } &  \\
\multicolumn{2}{l}{
                     \gloss{someone \ob muundu\cb 
                    for him/her for me’} } &  \\

                     \vernacular{yamuu[ndeeleelɛ]
                    mú{\downstep}yáyi/muundu tá}  &   
                     \gloss{‘bury’}  &  \\

                     \vernacular{yamuu[mbecheelɛ]
                    mú{\downstep}yáyi/muundu tá}  &   
                     \gloss{‘shave’}  &  \\

                     \vernacular{yamuu[ndeereelɛ]
                    mú{\downstep}yáyi/muundu tá}  &   
                     \gloss{‘bring’}  &  \\

                     \vernacular{yamuu[khalachiilɪ]
                    mú{\downstep}yáyi/muundu tá}  &   
                     \gloss{‘cut’}  &  \\

                     \vernacular{yamuu[sitaachiilɪ]
                    mú{\downstep}yáyi/muundu tá}  &   
                     \gloss{‘accuse’}  &  \\

                     \vernacular{
                    yamuu[mboolitsiilɪ] mú{\downstep}yáyi/muundu
                    tá}  &   
                     \gloss{‘seduce’}  &  \\

                     \vernacular{
                    yamuu[mbohololeelɛ] mú{\downstep}yáyi/muundu
                    tá}  &   
                     \gloss{‘untie’}  &  \\
\end{tabular}
%\caption{\nocaption}
     
\begin{tabular}{lll}  
  \multicolumn{2}{l}{
                     \vernacular{(474) /Ø/
                    C-Initial +OP + OP
                    } \gloss{‘s/he did
                    not...the boy \ob mú{\downstep}yáyi\cb  /} } &  \\
\multicolumn{2}{l}{
                     \gloss{the man
                    \ob musáatsa\cb  / someone \ob muundu\cb  for him/her for
                    me’} } &  \\

                     \vernacular{yamuú[nziiliilɪ]
                    mú{\downstep}yáyi/musáatsa/muundu tá}  &   
                     \gloss{‘go for’}  &  \\

                     \vernacular{yamuú[ndesheelɛ]
                    mú{\downstep}yáyi/musáatsa/muundu tá}  &   
                     \gloss{‘leave’}  &  \\

                     \vernacular{yamuú[noondeelɛ]
                    mú{\downstep}yáyi/musáatsa/muundu tá}  &   
                     \gloss{‘follow’}  &  \\

                     \vernacular{
                    yamuú[ngulishiilɪ] mú{\downstep}yáyi/musáatsa/muundu
                    tá}  &   
                     \gloss{‘name’}  &  \\

                     \vernacular{
                    yamuú[ndakhuuliilɪ] mú{\downstep}yáyi/musáatsa/muundu
                    tá}  &   
                     \gloss{‘release’}  &  \\

                     \vernacular{yamuú[seebuliilɪ]
                    mú{\downstep}yáyi/musáatsa/muundu tá}  &   
                     \gloss{‘say bye to’}  &  \\

                     \vernacular{
                    yamuú[siinjilitsiilɪ]
                    mú{\downstep}yáyi/musáatsa/muundu tá}  &   
                     \gloss{
                    ‘make...stand’}  &  \\
\end{tabular}
%\caption{\nocaption}
    

\subsection{Perfect (3
              }\label{sec:sPerf3rdSg}


\begin{tabular}{llllll}  
  \multicolumn{5}{l}{
                     \vernacular{(475) /H/
                    C-Initial} \gloss{‘s/he
                    has...’} } &  \\
\multicolumn{5}{l}{ } &  \\

                     \vernacular{
                    aa[réele]}  &   
                     \gloss{‘buried’}  &     &   
                     \vernacular{
                    aa[ng’wéele]}  &   
                     \gloss{‘drunk’}  &  \\

                     \vernacular{
                    aa[khwéele]}  &   
                     \gloss{‘eaten’}  &     &   
                     \vernacular{
                    aa[líili]}  &   
                     \gloss{‘paid dowry’}  &  \\

                     \vernacular{
                    aa[lúmi]}  &   
                     \gloss{‘bitten’}  &     &   
                     \vernacular{
                    aa[béchi]}  &   
                     \gloss{‘shaved’}  &  \\

                     \vernacular{
                    aa[téeshi]}  &   
                     \gloss{‘cooked’}  &     &   
                     \vernacular{
                    aa[léeri]}  &   
                     \gloss{‘brought’}  &  \\

                     \vernacular{
                    aa[khálaachɛ]}  &   
                     \gloss{‘cut’}  &     &   
                     \vernacular{
                    aa[kálaanji]}  &   
                     \gloss{‘fried’}  &  \\

                     \vernacular{
                    aa[sítaachi]}  &   
                     \gloss{‘accused’}  &     &   
                     \vernacular{
                    aa[bóoliitsɪ]}  &   
                     \gloss{‘seduced’}  &  \\

                     \vernacular{
                    aa[sáandiitsɪ]}  &   
                     \gloss{‘thanked’}  &     &   
                     \vernacular{
                    aa[khóng’oondi]}  &   
                     \gloss{‘knocked’}  &  \\

                     \vernacular{
                    aa[bóholoolɛ]}  &   
                     \gloss{‘untied’}  &     &   
                     \vernacular{
                    aa[bóyong’aanɛ]}  &   
                     \gloss{‘gone
                    around’}  &  \\

                     \vernacular{
                    aa[ng’óng’ooliitsɪ]}  &   
                     \gloss{‘teased’}  &     &   
                     \vernacular{
                    aa[líng(ak)anyiinyɪ]}  &   
                     \gloss{‘crumpled’}  &  \\
\end{tabular}
%\caption{\nocaption}
     
\begin{tabular}{llllll}  
  \multicolumn{5}{l}{
                     \vernacular{(476) /H/
                    V-Initial} \gloss{‘s/he
                    has...’} } &  \\
\multicolumn{5}{l}{ } &  \\

                     \vernacular{
                    y[ííri]}  &   
                     \gloss{‘killed’}  &     &   
                     \vernacular{
                    y[iíkoombi]}  &   
                     \gloss{‘admired’}  &  \\

                     \vernacular{
                    y[iísiachi]}  &   
                     \gloss{‘smacked’}  &     &   
                     \vernacular{
                    y[iíkoboolɛ]}  &   
                     \gloss{‘belched’}  &  \\

                     \vernacular{
                    y[oónonyiinyɪ]}  &   
                     \gloss{‘spoiled’}  &     &   
                     \vernacular{
                    y[aábukhanyiinyɪ]}  &   
                     \gloss{‘separated’}  &  \\
\end{tabular}
%\caption{\nocaption}
     
\begin{tabular}{llllll}  
  \multicolumn{5}{l}{
                     \vernacular{(477) /Ø/
                    C-Initial} \gloss{‘s/he
                    has...’} } &  \\
\multicolumn{5}{l}{ } &  \\

                     \vernacular{
                    aa[tsiili]}  &   
                     \gloss{‘gone’}  &     &   
                     \vernacular{
                    aa[kwiili]}  &   
                     \gloss{‘fallen’}  &  \\

                     \vernacular{
                    aa[leshi]}  &   
                     \gloss{‘left’}  &     &   
                     \vernacular{
                    aa[reebi]}  &   
                     \gloss{‘asked’}  &  \\

                     \vernacular{
                    aa[loondi]}  &   
                     \gloss{‘followed’}  &     &   
                     \vernacular{
                    aa[kumiilɪ]}  &   
                     \gloss{‘held’}  &  \\

                     \vernacular{
                    aa[kuliishɪ]}  &   
                     \gloss{‘named’}  &     &   
                     \vernacular{
                    aa[homooli]}  &   
                     \gloss{‘massaged’}  &  \\

                     \vernacular{
                    aa[lakhuuli]}  &   
                     \gloss{‘released’}  &     &   
                     \vernacular{
                    aa[seebuulɪ]}  &   
                     \gloss{‘said bye’}  &  \\

                     \vernacular{
                    aa[hoombeliitsɪ]}  &   
                     \gloss{‘comforted’}  &     &   
                     \vernacular{
                    aa[kalushiitsɪ]}  &   
                     \gloss{‘returned’}  &  \\

                     \vernacular{
                    aa[siinjiliitsɪ]}  &   
                     \gloss{‘made stand’}  &     &   
                     \vernacular{
                    aa[reebireebi]}  &   
                     \gloss{‘asked
                    (iter)’}  &  \\

                     \vernacular{
                    aa[kalukhanyiinyɪ]}  &   
                     \gloss{‘turned
                    over’}  &     &   
                     \vernacular{
                    aa[sebulukhanyiinyɪ]}  &   
                     \gloss{‘scattered’}  &  \\
\end{tabular}
%\caption{\nocaption}
     
\begin{tabular}{llllll}  
  \multicolumn{5}{l}{
                     \vernacular{(478) /Ø/
                    V-Initial} \gloss{‘s/he
                    has...’} } &  \\
\multicolumn{5}{l}{ } &  \\

                     \vernacular{
                    y[eényi]}  &   
                     \gloss{‘wanted’}  &     &   
                     \vernacular{
                    y[eéyeelɛ]}  &   
                     \gloss{‘wiped for’}  &  \\

                     \vernacular{
                    y[iíluuli]}  &   
                     \gloss{‘winnowed’}  &     &   
                     \vernacular{
                    y[aámbakhaanɛ]}  &   
                     \gloss{‘refused’}  &  \\

                     \vernacular{
                    y[eéleeliitsɪ]}  &   
                     \gloss{‘hung up’}  &     &     &     &  \\
\end{tabular}
%\caption{\nocaption}
     
\begin{tabular}{llllll}  
  \multicolumn{5}{l}{
                     \vernacular{(479) /H/
                    C-Initial + OP} \gloss{‘s/he
                    has...him/her’} } &  \\
\multicolumn{5}{l}{ } &  \\

                     \vernacular{
                    aamú[reele]}  &   
                     \gloss{‘buried’}  &     &   
                     \vernacular{
                    aamú[bechi]}  &   
                     \gloss{‘shaved’}  &  \\

                     \vernacular{
                    aamú[leeri]}  &   
                     \gloss{‘brought’}  &     &   
                     \vernacular{
                    aamú[khalaachɛ]}  &   
                     \gloss{‘cut’}  &  \\

                     \vernacular{
                    aamú[sitaachi]}  &   
                     \gloss{‘accused’}  &     &   
                     \vernacular{
                    aamú[booliitsɪ]}  &   
                     \gloss{‘seduced’}  &  \\

                     \vernacular{
                    aamú[khong’oondi]}  &   
                     \gloss{‘knocked’}  &     &   
                     \vernacular{
                    aamú[boholoolɛ]}  &   
                     \gloss{‘untied’}  &  \\

                     \vernacular{
                    aamú[boyong’aanɛ]}  &   
                     \gloss{‘gone
                    around’}  &     &   
                     \vernacular{
                    aamú[ng’ong’ooliitsɪ]}  &   
                     \gloss{‘teased’}  &  \\

                     \vernacular{
                    aamú[lingakanyiinyɪ]}  &   
                     \gloss{‘bent’}  &     &     &     &  \\
\end{tabular}
%\caption{\nocaption}
     
\begin{tabular}{llllll}  
  \multicolumn{5}{l}{
                     \vernacular{(480) /H/
                    V-Initial + OP} \gloss{‘s/he
                    has...him/her’} } &  \\
\multicolumn{5}{l}{ } &  \\

                     \vernacular{
                    aamw[íiri]}  &   
                     \gloss{‘killed’}  &     &   
                     \vernacular{
                    aamw[íikoombi]}  &   
                     \gloss{‘admired’}  &  \\

                     \vernacular{
                    aamw[íisiachi]}  &   
                     \gloss{‘smacked’}  &     &   
                     \vernacular{
                    aamw[óononyiinyɪ]}  &   
                     \gloss{‘spoiled’}  &  \\

                     \vernacular{
                    aamw[áabukhanyiinyɪ]}  &   
                     \gloss{‘separated’}  &  \\
\end{tabular}
%\caption{\nocaption}
     
\begin{tabular}{llllll}  
  \multicolumn{5}{l}{
                     \vernacular{(481) /Ø/
                    C-Initial + OP} \gloss{‘s/he
                    has...him/her \ob mu-\cb  / them
                    } } &  \\
\multicolumn{5}{l}{ } &  \\

                     \vernacular{
                    aamú[tsiili]}  &   
                     \gloss{‘gone for’}  &  \\

                     \vernacular{
                    aamú[leshi]}  &   
                     \gloss{‘left’}  &  \\

                     \vernacular{
                    aamú[loondi]}  &   
                     \gloss{‘followed’}  &  \\

                     \vernacular{
                    aamú[kuliishɪ]}  &   
                     \gloss{‘named’}  &  \\

                     \vernacular{
                    aamú[lakhuuli]}  &   
                     \gloss{‘released’}  &  \\

                     \vernacular{
                    aamú[seebuulɪ]}  &   
                     \gloss{‘said bye
                    to’}  &  \\

                     \vernacular{
                    aamú[hoombeliitsɪ]}  &   
                     \gloss{‘comforted’}  &  \\

                     \vernacular{
                    aamú[kalushiitsɪ]}  &   
                     \gloss{‘returned’}  &  \\

                     \vernacular{
                    aamú[siinjiliitsɪ]}  &   
                     \gloss{
                    ‘made...stand’}  &  \\

                     \vernacular{
                    aamú[reebireebi]}  &   
                     \gloss{‘asked
                    (iter)’}  &  \\

                     \vernacular{
                    aamú[kalukhanyiinyɪ]}  &   
                     \gloss{
                    ‘turned...over’}  &  \\

                     \vernacular{
                    aabí[sebulukhanyiinyɪ]}  &   
                     \gloss{‘scattered’}  &  \\
\end{tabular}
%\caption{\nocaption}
     
\begin{tabular}{llllll}  
  \multicolumn{5}{l}{
                     \vernacular{(482) /Ø/
                    V-Initial + OP} \gloss{‘s/he
                    has...him/her \ob mw-\cb  / it
                    } } &  \\
\multicolumn{5}{l}{ } &  \\

                     \vernacular{
                    aamw[éenyi]}  &   
                     \gloss{‘wanted’}  &     &   
                     \vernacular{
                    aamw[éeyeelɛ]}  &   
                     \gloss{‘wiped for’}  &  \\

                     \vernacular{
                    aabw[íiluuli]}  &   
                     \gloss{‘winnowed’}  &     &   
                     \vernacular{
                    aamw[áambakhaanɛ]}  &   
                     \gloss{‘refused’}  &  \\

                     \vernacular{
                    aamw[éeleeliitsɪ]}  &   
                     \gloss{
                    ‘carried...hanging’}  &  \\
\end{tabular}
%\caption{\nocaption}
     
\begin{tabular}{llllll}  
  \multicolumn{5}{l}{
                     \vernacular{(483) /H/
                    C-Initial + OP
                    } \gloss{‘s/he
                    has...me’} } &  \\
\multicolumn{5}{l}{ } &  \\

                     \vernacular{
                    aá[riili]}  &   
                     \gloss{‘feared’}  &     &   
                     \vernacular{
                    aá[mbechi]}  &   
                     \gloss{‘shaved’}  &  \\

                     \vernacular{
                    aá[ndeeri]}  &   
                     \gloss{‘brought’}  &     &   
                     \vernacular{
                    aá[khalaachɛ]}  &   
                     \gloss{‘cut’}  &  \\

                     \vernacular{
                    aá[sitaachi]}  &   
                     \gloss{‘accused’}  &     &   
                     \vernacular{
                    aá[mbooliitsɪ]}  &   
                     \gloss{‘seduced’}  &  \\

                     \vernacular{
                    aá[khong’oondi]}  &   
                     \gloss{‘knocked’}  &     &   
                     \vernacular{
                    aá[mboholoolɛ]}  &   
                     \gloss{‘untied’}  &  \\

                     \vernacular{
                    aá[mboyong’aanɛ]}  &   
                     \gloss{‘gone
                    around’}  &     &   
                     \vernacular{
                    aá[ng’ong’ooliitsɪ]}  &   
                     \gloss{‘teased’}  &  \\

                     \vernacular{
                    aá[ningakanyiinyɪ]}  &   
                     \gloss{‘bent’}  &  \\
\end{tabular}
%\caption{\nocaption}
     
\begin{tabular}{llllll}  
  \multicolumn{5}{l}{
                     \vernacular{(484) /H/
                    V-Initial + OP
                    } \gloss{‘s/he
                    has...me’} } &  \\
\multicolumn{5}{l}{ } &  \\

                     \vernacular{
                    aá[nziri]}  &   
                     \gloss{‘killed’}  &     &   
                     \vernacular{
                    aá[nzikoombi]}  &   
                     \gloss{‘admired’}  &  \\

                     \vernacular{
                    aá[nzisiachi]}  &   
                     \gloss{‘smacked’}  &     &   
                     \vernacular{
                    aá[nzononyiinyɪ]}  &   
                     \gloss{‘spoiled’}  &  \\

                     \vernacular{
                    aá[nzabukhanyiinyɪ]}  &   
                     \gloss{‘separated’}  &  \\
\end{tabular}
%\caption{\nocaption}
     
\begin{tabular}{llllll}  
  \multicolumn{5}{l}{
                     \vernacular{(485) /Ø/
                    C-Initial + OP
                    } \gloss{‘s/he
                    has...me’} } &  \\
\multicolumn{5}{l}{ } &  \\

                     \vernacular{
                    aá[siele]}  &   
                     \gloss{‘ground’}  &     &   
                     \vernacular{
                    aá[ndeshi]}  &   
                     \gloss{‘left’}  &  \\

                     \vernacular{
                    aá[noondi]}  &   
                     \gloss{‘followed’}  &     &   
                     \vernacular{
                    aá[nguliishɪ]}  &   
                     \gloss{‘named’}  &  \\

                     \vernacular{
                    aá[ndakhuuli]}  &   
                     \gloss{‘released’}  &     &   
                     \vernacular{
                    aá[seebuulɪ]}  &   
                     \gloss{‘said bye
                    to’}  &  \\

                     \vernacular{
                    aá[mboombeliitsɪ]}  &   
                     \gloss{‘comforted’}  &     &   
                     \vernacular{
                    aá[siinjiliitsɪ]}  &   
                     \gloss{
                    ‘made..stand’}  &  \\

                     \vernacular{
                    aá[ndeebindeebi]}  &   
                     \gloss{‘asked
                    (iter)’}  &     &   
                     \vernacular{
                    aá[ngalukhanyiinyɪ]}  &   
                     \gloss{
                    ‘turned...over’}  &  \\
\end{tabular}
%\caption{\nocaption}
     
\begin{tabular}{llllll}  
  \multicolumn{5}{l}{
                     \vernacular{(486) /Ø/
                    V-Initial + OP
                    } \gloss{‘s/he
                    has...me’} } &  \\
\multicolumn{5}{l}{ } &  \\

                     \vernacular{
                    aá[nzenyi]}  &   
                     \gloss{‘wanted’}  &     &   
                     \vernacular{
                    aá[nzeyeelɛ]}  &   
                     \gloss{‘wiped for’}  &  \\

                     \vernacular{
                    aá[nyambakhaanɛ]}  &   
                     \gloss{‘refused’}  &     &   
                     \vernacular{
                    aá[nzeleeliitsɪ]}  &   
                     \gloss{
                    ‘carried...hanging’}  &  \\
\end{tabular}
%\caption{\nocaption}
     
\begin{tabular}{llllll}  
  \multicolumn{5}{l}{
                     \vernacular{(487) /H/
                    C-Initial + OP
                    } \gloss{‘s/he
                    has...him/herself’} } &  \\
\multicolumn{5}{l}{ } &  \\

                     \vernacular{
                    yií[reele]}  &   
                     \gloss{‘buried’}  &     &   
                     \vernacular{
                    yií[bechi]}  &   
                     \gloss{‘shaved’}  &  \\

                     \vernacular{
                    yií[suunji]}  &   
                     \gloss{‘hung’}  &     &   
                     \vernacular{
                    yií[khalaachɛ]}  &   
                     \gloss{‘cut’}  &  \\

                     \vernacular{
                    yií[sitaachi]}  &   
                     \gloss{‘accused’}  &     &   
                     \vernacular{
                    yií[saandiitsɪ]}  &   
                     \gloss{‘thanked’}  &  \\

                     \vernacular{
                    yií[khong’oondi]}  &   
                     \gloss{‘knocked’}  &     &   
                     \vernacular{
                    yií[boholoolɛ]}  &   
                     \gloss{‘untied’}  &  \\
\end{tabular}
%\caption{\nocaption}
     
\begin{tabular}{llllll}  
  \multicolumn{5}{l}{
                     \vernacular{(488) /H/
                    V-Initial + OP
                    } \gloss{‘s/he
                    has...him/herself’} } &  \\
\multicolumn{5}{l}{ } &  \\

                     \vernacular{
                    yií[yiri]}  &   
                     \gloss{‘killed’}  &     &   
                     \vernacular{
                    yií[yikoombi]}  &   
                     \gloss{‘admired’}  &  \\

                     \vernacular{
                    yií[yisiachi]}  &   
                     \gloss{‘smacked’}  &     &   
                     \vernacular{
                    yií[yononyiinyɪ]}  &   
                     \gloss{‘spoiled’}  &  \\

                     \vernacular{
                    yií[yabukhanyiinyɪ]}  &   
                     \gloss{‘separated’}  &  \\
\end{tabular}
%\caption{\nocaption}
     
\begin{tabular}{llllll}  
  \multicolumn{5}{l}{
                     \vernacular{(489) /Ø/
                    C-Initial + OP
                    } \gloss{‘s/he
                    has...him/herself’} } &  \\
\multicolumn{5}{l}{ } &  \\

                     \vernacular{
                    yií[siele]}  &   
                     \gloss{‘ground’}  &     &   
                     \vernacular{
                    yií[leshi]}  &   
                     \gloss{‘left’}  &  \\

                     \vernacular{
                    yií[siinji]}  &   
                     \gloss{‘bathed’}  &     &   
                     \vernacular{
                    yií[kuliishɪ]}  &   
                     \gloss{‘named’}  &  \\

                     \vernacular{
                    yií[naabuulɪ]}  &   
                     \gloss{‘undressed’}  &     &   
                     \vernacular{
                    yií[lakhuuli]}  &   
                     \gloss{‘released’}  &  \\

                     \vernacular{
                    yií[hoombeliitsɪ]}  &   
                     \gloss{‘comforted’}  &     &   
                     \vernacular{
                    yií[siinjiliitsɪ]}  &   
                     \gloss{
                    ‘made...stand’}  &  \\

                     \vernacular{
                    yií[reebireebi]}  &   
                     \gloss{‘asked
                    (iter)’}  &     &   
                     \vernacular{
                    yií[kalukhanyiinyɪ]}  &   
                     \gloss{
                    ‘turned...over’}  &  \\
\end{tabular}
%\caption{\nocaption}
     
\begin{tabular}{llllll}  
  \multicolumn{5}{l}{
                     \vernacular{(490) /Ø/
                    V-Initial + OP
                    } \gloss{‘s/he
                    has...him/herself’} } &  \\
\multicolumn{5}{l}{ } &  \\

                     \vernacular{
                    yií[yali]}  &   
                     \gloss{‘exposed’}  &     &   
                     \vernacular{
                    yií[yeyeelɛ]}  &   
                     \gloss{‘wiped for’}  &  \\

                     \vernacular{
                    yií[yambakhaanɛ]}  &   
                     \gloss{‘refused’}  &     &   
                     \vernacular{
                    yií[yeleeliitsɪ]}  &   
                     \gloss{‘hung...up’}  &  \\
\end{tabular}
%\caption{\nocaption}
     
\begin{tabular}{llllll}  
  \multicolumn{5}{l}{
                     \vernacular{(491) /H/
                    C-Initial + OP + OP
                    } \gloss{‘s/he
                    has...him/her for me’} } &  \\
\multicolumn{5}{l}{ } &  \\

                     \vernacular{
                    aamúu[ndeeleelɛ]}  &   
                     \gloss{‘buried’}  &     &   
                     \vernacular{
                    aamúu[mbecheelɛ]}  &   
                     \gloss{‘shaved’}  &  \\

                     \vernacular{
                    aamúu[ndeereelɛ]}  &   
                     \gloss{‘brought’}  &     &   
                     \vernacular{
                    aamúu[khalachiilɪ]}  &   
                     \gloss{‘cut’}  &  \\

                     \vernacular{
                    aamúu[sitaachiilɪ]}  &   
                     \gloss{‘accused’}  &     &   
                     \vernacular{
                    aamúu[mboolitsiilɪ]}  &   
                     \gloss{‘seduced’}  &  \\

                     \vernacular{
                    aamúu[mbohololeelɛ]}  &   
                     \gloss{‘untied’}  &     &     &     &  \\
\end{tabular}
%\caption{\nocaption}
     
\begin{tabular}{llllll}  
  \multicolumn{5}{l}{
                     \vernacular{(492) /H/
                    V-Initial + OP + OP
                    } \gloss{‘s/he
                    has...him/her for me’} } &  \\
\multicolumn{5}{l}{ } &  \\

                     \vernacular{
                    aamúu[nziiriilɪ]}  &   
                     \gloss{‘killed’}  &     &   
                     \vernacular{
                    aamúu[nzechitsiilɪ]}  &   
                     \gloss{‘admired’}  &  \\

                     \vernacular{
                    aamúu[nzisiachiilɪ]}  &   
                     \gloss{‘smacked’}  &     &   
                     \vernacular{
                    aamúu[nzononyinyiilɪ]}  &   
                     \gloss{‘spoiled’}  &  \\

                     \vernacular{
                    aamúu[nzabukhanyinyiilɪ]}  &   
                     \gloss{‘separated’}  &     &     &     &  \\
\end{tabular}
%\caption{\nocaption}
     
\begin{tabular}{llllll}  
  \multicolumn{5}{l}{
                     \vernacular{(493) /Ø/
                    C-Initial + OP + OP
                    } \gloss{‘s/he
                    has...him/her for me’} } &  \\
\multicolumn{5}{l}{ } &  \\

                     \vernacular{
                    aamúu[nziiliilɪ]}  &   
                     \gloss{‘gone for’}  &     &   
                     \vernacular{
                    aamúu[ndesheelɛ]}  &   
                     \gloss{‘gone for’}  &  \\

                     \vernacular{
                    aamúu[noondeelɛ]}  &   
                     \gloss{‘left’}  &     &   
                     \vernacular{
                    aamúu[ngulishiilɪ]}  &   
                     \gloss{‘followed’}  &  \\

                     \vernacular{
                    aamúu[ndakhuuliilɪ]}  &   
                     \gloss{‘named’}  &     &   
                     \vernacular{
                    aamúu[seebuliilɪ]}  &   
                     \gloss{‘released’}  &  \\

                     \vernacular{
                    aamúu[mboombelitsiilɪ]}  &   
                     \gloss{‘said bye
                    to’}  &     &   
                     \vernacular{
                    aamúu[siinjilitsiilɪ]}  &   
                     \gloss{‘comforted’}  &  \\
\end{tabular}
%\caption{\nocaption}
     
\begin{tabular}{llllll}  
  \multicolumn{5}{l}{
                     \vernacular{(494) /Ø/
                    V-Initial + OP + OP
                    } \gloss{‘s/he
                    has...him/her \ob mu-\cb  / it
                    } } &  \\
\multicolumn{5}{l}{ } &  \\

                     \vernacular{
                    aamúu[nzeyeelɛ]}  &   
                     \gloss{‘wiped’}  &     &   
                     \vernacular{
                    aakúu[nzashitsiilɪ]}  &   
                     \gloss{‘lit’}  &  \\

                     \vernacular{
                    aabúu[nziluuliilɪ]}  &   
                     \gloss{‘winnowed’}  &     &   
                     \vernacular{
                    aakúu[nzéléelitsiilɪ]}  &   
                     \gloss{‘hung’}  &  \\
\end{tabular}
%\caption{\nocaption}
     
\begin{tabular}{lll}  
  \multicolumn{2}{l}{
                     \vernacular{(495) /H/
                    C-Initial Phrase-Medial} \gloss{‘s/he has...the
                    boy \ob mú{\downstep}yáyi\cb  /} } &  \\
\multicolumn{2}{l}{
                     \gloss{someone
                    \ob muundu\cb ’} } &  \\

                     \vernacular{aa[réele]
                    mú{\downstep}yáyi/muundu}  &   
                     \gloss{‘buried’}  &  \\

                     \vernacular{aa[béchi]
                    mú{\downstep}yáyi/muundu}  &   
                     \gloss{‘shaved’}  &  \\

                     \vernacular{aa[léeri]
                    mú{\downstep}yáyi/muundu}  &   
                     \gloss{‘brought’}  &  \\

                     \vernacular{aa[khálaachɛ]
                    mú{\downstep}yáyi/muundu}  &   
                     \gloss{‘cut’}  &  \\

                     \vernacular{aa[sítaachi]
                    mú{\downstep}yáyi/muundu}  &   
                     \gloss{‘accused’}  &  \\

                     \vernacular{aa[bóoliitsɪ]
                    mú{\downstep}yáyi/muundu}  &   
                     \gloss{‘seduced’}  &  \\

                     \vernacular{aa[khóng’oondi]
                    mú{\downstep}yáyi/muundu}  &   
                     \gloss{‘knocked’}  &  \\

                     \vernacular{aa[bóholoolɛ]
                    mú{\downstep}yáyi/muundu}  &   
                     \gloss{‘untied’}  &  \\

                     \vernacular{aa[bóyong’aanɛ]
                    mú{\downstep}yáyi/muundu}  &   
                     \gloss{‘gone
                    around’}  &  \\
\end{tabular}
%\caption{\nocaption}
     
\begin{tabular}{lll}  
  \multicolumn{2}{l}{
                     \vernacular{(496) /Ø/
                    C-Initial Phrase-Medial} \gloss{‘s/he has...the
                    boy \ob mú{\downstep}yáyi\cb  /} } &  \\
\multicolumn{2}{l}{
                     \gloss{the man
                    \ob musáatsa\cb  / someone \ob muundu\cb ’} } &  \\

                     \vernacular{aa[tsiili]
                    mú{\downstep}yáyi/musáatsa/muundu}  &   
                     \gloss{‘gone for’}  &  \\

                     \vernacular{aa[leshi]
                    mú{\downstep}yáyi/musáatsa/muundu}  &   
                     \gloss{‘left’}  &  \\

                     \vernacular{aa[loondi]
                    mú{\downstep}yáyi/musáatsa/muundu}  &   
                     \gloss{‘followed’}  &  \\

                     \vernacular{aa[kuliishɪ]
                    mú{\downstep}yáyi/musáatsa/muundu}  &   
                     \gloss{‘named’}  &  \\

                     \vernacular{aa[lakhuuli]
                    mú{\downstep}yáyi/musáatsa/muundu}  &   
                     \gloss{‘released’}  &  \\

                     \vernacular{aa[seebuulɪ]
                    mú{\downstep}yáyi/musáatsa/muundu}  &   
                     \gloss{‘said bye
                    to’}  &  \\

                     \vernacular{aa[kalushiitsɪ]
                    mú{\downstep}yáyi/musáatsa/muundu}  &   
                     \gloss{‘returned’}  &  \\

                     \vernacular{aa[reebireebi]
                    mú{\downstep}yáyi/musáatsa/muundu}  &   
                     \gloss{‘asked
                    (iter)’}  &  \\
\end{tabular}
%\caption{\nocaption}
     
\begin{tabular}{lll}  
  \multicolumn{2}{l}{
                     \vernacular{(497) /H/
                    C-Initial +OP Phrase-Medial} \gloss{‘s/he has...the
                    boy \ob mú{\downstep}yáyi\cb  /} } &  \\
\multicolumn{2}{l}{
                     \gloss{someone \ob muundu\cb 
                    for him/her’} } &  \\

                     \vernacular{aamú[reeleelɛ]
                    mú{\downstep}yáyi/muundu}  &   
                     \gloss{‘buried’}  &  \\

                     \vernacular{aamú[becheelɛ]
                    mú{\downstep}yáyi/muundu}  &   
                     \gloss{‘shaved’}  &  \\

                     \vernacular{aamú[leereelɛ]
                    mú{\downstep}yáyi/muundu}  &   
                     \gloss{‘brought’}  &  \\

                     \vernacular{aamú[khalachiilɪ]
                    mú{\downstep}yáyi/muundu}  &   
                     \gloss{‘cut’}  &  \\

                     \vernacular{aamú[sitaachiilɪ]
                    mú{\downstep}yáyi/muundu}  &   
                     \gloss{‘accused’}  &  \\

                     \vernacular{aamú[boolitsiilɪ]
                    mú{\downstep}yáyi/muundu}  &   
                     \gloss{‘seduced’}  &  \\

                     \vernacular{
                    aamú[khong’oondeelɛ]
                    mú{\downstep}yáyi/muundu}  &   
                     \gloss{‘knocked’}  &  \\

                     \vernacular{aamú[bohololeelɛ]
                    mú{\downstep}yáyi/muundu}  &   
                     \gloss{‘untied’}  &  \\

                     \vernacular{
                    aamú[boyong’aniilɪ]
                    mú{\downstep}yáyi/muundu}  &   
                     \gloss{‘gone
                    around’}  &  \\
\end{tabular}
%\caption{\nocaption}
     
\begin{tabular}{lll}  
  \multicolumn{2}{l}{
                     \vernacular{(498) /Ø/
                    C-Initial +OP Phrase-Medial} \gloss{‘s/he has...the
                    boy \ob mú{\downstep}yáyi\cb  /} } &  \\
\multicolumn{2}{l}{
                     \gloss{the man
                    \ob musáatsa\cb  / someone \ob muundu\cb  for
                    him/her’} } &  \\

                     \vernacular{aamú[tsiiliilɪ]
                    mú{\downstep}yáyi/musáatsa/muundu}  &   
                     \gloss{‘gone for’}  &  \\

                     \vernacular{aamú[lesheelɛ]
                    mú{\downstep}yáyi/musáatsa/muundu}  &   
                     \gloss{‘left’}  &  \\

                     \vernacular{aamú[loondeelɛ]
                    mú{\downstep}yáyi/musáatsa/muundu}  &   
                     \gloss{‘followed’}  &  \\

                     \vernacular{aamú[kulishiilɪ]
                    mú{\downstep}yáyi/musáatsa/muundu}  &   
                     \gloss{‘named’}  &  \\

                     \vernacular{aamú[lakhuuliilɪ]
                    mú{\downstep}yáyi/musáatsa/muundu}  &   
                     \gloss{‘released’}  &  \\

                     \vernacular{aamú[seebuliilɪ]
                    mú{\downstep}yáyi/musáatsa/muundu}  &   
                     \gloss{‘said bye
                    to’}  &  \\

                     \vernacular{
                    aamú[kalushitsiilɪ]
                    mú{\downstep}yáyi/musáatsa/muundu}  &   
                     \gloss{‘returned’}  &  \\

                     \vernacular{
                    aamú[reebɛreebeelɛ]
                    mú{\downstep}yáyi/musáatsa/muundu}  &   
                     \gloss{‘asked
                    (iter)’}  &  \\
\end{tabular}
%\caption{\nocaption}
     
\begin{tabular}{lll}  
  \multicolumn{2}{l}{
                     \vernacular{(499) /H/
                    C-Initial +OP + OP
                    } \gloss{‘s/he has...the
                    boy \ob mú{\downstep}yáyi\cb  /} } &  \\
\multicolumn{2}{l}{
                     \gloss{someone \ob muundu\cb 
                    for him/her for me’} } &  \\

                     \vernacular{aamúu[ndeeleelɛ]
                    mú{\downstep}yáyi/muundu}  &   
                     \gloss{‘buried’}  &  \\

                     \vernacular{aamúu[mbecheelɛ]
                    mú{\downstep}yáyi/muundu}  &   
                     \gloss{‘shaved’}  &  \\

                     \vernacular{aamúu[ndeereelɛ]
                    mú{\downstep}yáyi/muundu}  &   
                     \gloss{‘brought’}  &  \\

                     \vernacular{
                    aamúu[khalachiilɪ] mú{\downstep}yáyi/muundu}  &   
                     \gloss{‘cut’}  &  \\

                     \vernacular{
                    aamúu[sitaachiilɪ] mú{\downstep}yáyi/muundu}  &   
                     \gloss{‘accused’}  &  \\

                     \vernacular{
                    aamúu[mboolitsiilɪ]
                    mú{\downstep}yáyi/muundu}  &   
                     \gloss{‘seduced’}  &  \\

                     \vernacular{
                    aamúu[mbohololeelɛ]
                    mú{\downstep}yáyi/muundu}  &   
                     \gloss{‘untied’}  &  \\
\end{tabular}
%\caption{\nocaption}
     
\begin{tabular}{lll}  
  \multicolumn{2}{l}{
                     \vernacular{(500) /Ø/
                    C-Initial +OP + OP
                    } \gloss{‘s/he has...the
                    boy \ob mú{\downstep}yáyi\cb  /} } &  \\
\multicolumn{2}{l}{
                     \gloss{someone \ob muundu\cb 
                    for him/her for me’} } &  \\

                     \vernacular{aamúu[nziiliilɪ]
                    mú{\downstep}yáyi/muundu}  &   
                     \gloss{‘gone for’}  &  \\

                     \vernacular{aamúu[ndesheelɛ]
                    mú{\downstep}yáyi/muundu}  &   
                     \gloss{‘left’}  &  \\

                     \vernacular{aamúu[noondeelɛ]
                    mú{\downstep}yáyi/muundu}  &   
                     \gloss{‘followed’}  &  \\

                     \vernacular{
                    aamúu[ngulishiilɪ] mú{\downstep}yáyi/muundu}  &   
                     \gloss{‘named’}  &  \\

                     \vernacular{
                    aamúu[ndakhuuliilɪ]
                    mú{\downstep}yáyi/muundu}  &   
                     \gloss{‘released’}  &  \\

                     \vernacular{aamúu[seebuliilɪ]
                    mú{\downstep}yáyi/muundu}  &   
                     \gloss{‘said bye
                    to’}  &  \\

                     \vernacular{
                    aamúu[siinjilitsiilɪ]
                    mú{\downstep}yáyi/muundu}  &   
                     \gloss{
                    ‘made...stand’}  &  \\
\end{tabular}
%\caption{\nocaption}
    

\subsection{Perfect Negative (3
              }\label{sec:sPerf3rdSgNeg}


\begin{tabular}{llllll}  
  \multicolumn{5}{l}{
                     \vernacular{(501) /H/
                    C-Initial} \gloss{‘s/he has
                    not...’} } &  \\
\multicolumn{5}{l}{ } &  \\

                     \vernacular{aa[ré{\downstep}élé]
                    tá}  &   
                     \gloss{‘buried’}  &     &   
                     \vernacular{aa[ng’wé{\downstep}élé]
                    tá}  &   
                     \gloss{‘drunk’}  &  \\

                     \vernacular{aa[khwé{\downstep}élé]
                    tá}  &   
                     \gloss{‘eaten’}  &     &   
                     \vernacular{aa[lí{\downstep}ílí]
                    tá}  &   
                     \gloss{‘paid dowry’}  &  \\

                     \vernacular{aa[lú{\downstep}mí]
                    tá}  &   
                     \gloss{‘bitten’}  &     &   
                     \vernacular{aa[bé{\downstep}chí]
                    tá}  &   
                     \gloss{‘shaved’}  &  \\

                     \vernacular{aa[té{\downstep}éshí]
                    tá}  &   
                     \gloss{‘cooked’}  &     &   
                     \vernacular{aa[lé{\downstep}érí]
                    tá}  &   
                     \gloss{‘brought’}  &  \\

                     \vernacular{aa[khá{\downstep}lááchɛ́]
                    tá}  &   
                     \gloss{‘cut’}  &     &   
                     \vernacular{aa[ká{\downstep}láánjí]
                    tá}  &   
                     \gloss{‘fried’}  &  \\

                     \vernacular{aa[sí{\downstep}tááchí]
                    tá}  &   
                     \gloss{‘accused’}  &     &   
                     \vernacular{
                    aa[bó{\downstep}ólíítsɪ́] tá}  &   
                     \gloss{‘seduced’}  &  \\

                     \vernacular{
                    aa[sá{\downstep}ándíítsɪ́] tá}  &   
                     \gloss{‘thanked’}  &     &   
                     \vernacular{
                    aa[khó{\downstep}ng’óóndí] tá}  &   
                     \gloss{‘knocked’}  &  \\

                     \vernacular{
                    aa[bó{\downstep}hólóólɛ́] tá}  &   
                     \gloss{‘untied’}  &     &   
                     \vernacular{
                    aa[bó{\downstep}yóng’áánɛ́] tá}  &   
                     \gloss{‘gone
                    around’}  &  \\

                     \vernacular{
                    aa[ng’ó{\downstep}ng’óólíítsɪ́] tá}  &   
                     \gloss{‘teased’}  &     &   
                     \vernacular{
                    aa[lí{\downstep}ng(ák)ányíínyɪ́] tá}  &   
                     \gloss{‘crumpled’}  &  \\
\end{tabular}
%\caption{\nocaption}
     
\begin{tabular}{llllll}  
  \multicolumn{5}{l}{
                     \vernacular{(502) /H/
                    V-Initial} \gloss{‘s/he has
                    not...’} } &  \\
\multicolumn{5}{l}{ } &  \\

                     \vernacular{y[íí{\downstep}rí]
                    tá}  &   
                     \gloss{‘killed’}  &     &   
                     \vernacular{y[ií{\downstep}kóómbí]
                    tá}  &   
                     \gloss{‘admired’}  &  \\

                     \vernacular{y[ií{\downstep}síáchí]
                    tá}  &   
                     \gloss{‘smacked’}  &     &   
                     \vernacular{y[ií{\downstep}kóbóólɛ́]
                    tá}  &   
                     \gloss{‘belched’}  &  \\

                     \vernacular{
                    y[oó{\downstep}nónyíínyɪ́] tá}  &   
                     \gloss{‘spoiled’}  &     &   
                     \vernacular{
                    y[aá{\downstep}búkhányíínyɪ́] tá}  &   
                     \gloss{‘separated’}  &  \\
\end{tabular}
%\caption{\nocaption}
     
\begin{tabular}{llllll}  
  \multicolumn{5}{l}{
                     \vernacular{(503) /Ø/
                    C-Initial} \gloss{‘s/he has
                    not...’} } &  \\
\multicolumn{5}{l}{ } &  \\

                     \vernacular{aa[tsíílí]
                    tá}  &   
                     \gloss{‘gone’}  &     &   
                     \vernacular{aa[kwíílí]
                    tá}  &   
                     \gloss{‘fallen’}  &  \\

                     \vernacular{aa[léshí]
                    tá}  &   
                     \gloss{‘left’}  &     &   
                     \vernacular{aa[réébí]
                    tá}  &   
                     \gloss{‘asked’}  &  \\

                     \vernacular{aa[lóóndí]
                    tá}  &   
                     \gloss{‘followed’}  &     &   
                     \vernacular{aa[kúmíílɪ́]
                    tá}  &   
                     \gloss{‘held’}  &  \\

                     \vernacular{aa[kúlííshɪ́]
                    tá}  &   
                     \gloss{‘named’}  &     &   
                     \vernacular{aa[hómóólí]
                    tá}  &   
                     \gloss{‘massaged’}  &  \\

                     \vernacular{aa[lákhúúlí]
                    tá}  &   
                     \gloss{‘released’}  &     &   
                     \vernacular{aa[séébúúlɪ́]
                    tá}  &   
                     \gloss{‘said bye’}  &  \\

                     \vernacular{
                    aa[hóómbélíítsɪ́] tá}  &   
                     \gloss{‘comforted’}  &     &   
                     \vernacular{
                    aa[kálúshíítsɪ́] tá}  &   
                     \gloss{‘returned’}  &  \\

                     \vernacular{
                    aa[síínjílíítsɪ́] tá}  &   
                     \gloss{‘made stand’}  &     &   
                     \vernacular{
                    aa[réébíréébí] tá}  &   
                     \gloss{‘asked
                    (iter)’}  &  \\

                     \vernacular{
                    aa[kálúkhányíínyɪ́] tá}  &   
                     \gloss{‘turn over’}  &     &   
                     \vernacular{
                    aa[sébúlúkhányíínyɪ́] tá}  &   
                     \gloss{‘scattered’}  &  \\
\end{tabular}
%\caption{\nocaption}
     
\begin{tabular}{llllll}  
  \multicolumn{5}{l}{
                     \vernacular{(504) /Ø/
                    V-Initial} \gloss{‘s/he has
                    not...’} } &  \\
\multicolumn{5}{l}{ } &  \\

                     \vernacular{y[eé{\downstep}nyí]
                    tá}  &   
                     \gloss{‘wanted’}  &     &   
                     \vernacular{y[eé{\downstep}yéélɛ́]
                    tá}  &   
                     \gloss{‘wiped for’}  &  \\

                     \vernacular{y[ií{\downstep}lúúlí]
                    tá}  &   
                     \gloss{‘winnowed’}  &     &   
                     \vernacular{
                    y[aá{\downstep}mbákháánɛ́] tá}  &   
                     \gloss{‘refused’}  &  \\

                     \vernacular{
                    y[eé{\downstep}léélíítsɪ́] tá}  &   
                     \gloss{‘hung up’}  &     &     &     &  \\
\end{tabular}
%\caption{\nocaption}
     
\begin{tabular}{llllll}  
  \multicolumn{5}{l}{
                     \vernacular{(505) /H/
                    C-Initial + OP} \gloss{‘s/he has
                    not...him/her’} } &  \\
\multicolumn{5}{l}{ } &  \\

                     \vernacular{aamú[{\downstep}réélé]
                    tá}  &   
                     \gloss{‘buried’}  &     &   
                     \vernacular{aamú[{\downstep}béchí]
                    tá}  &   
                     \gloss{‘shaved’}  &  \\

                     \vernacular{aamú[{\downstep}léérí]
                    tá}  &   
                     \gloss{‘brought’}  &     &   
                     \vernacular{
                    aamú[{\downstep}khálááchɛ́] tá}  &   
                     \gloss{‘cut’}  &  \\

                     \vernacular{
                    aamú[{\downstep}sítááchí] tá}  &   
                     \gloss{‘accused’}  &     &   
                     \vernacular{
                    aamú[{\downstep}bóólíítsɪ́] tá}  &   
                     \gloss{‘seduced’}  &  \\

                     \vernacular{
                    aamú[{\downstep}khóng’óóndí] tá}  &   
                     \gloss{‘knocked’}  &     &   
                     \vernacular{
                    aamú[{\downstep}bóhólóólɛ́] tá}  &   
                     \gloss{‘untied’}  &  \\

                     \vernacular{
                    aamú[{\downstep}bóyóng’áánɛ́] tá}  &   
                     \gloss{‘gone
                    around’}  &     &   
                     \vernacular{
                    aamú[{\downstep}ng’óng’óólíítsɪ́] tá}  &   
                     \gloss{‘teased’}  &  \\

                     \vernacular{
                    aamú[{\downstep}língákányíínyɪ́] tá}  &   
                     \gloss{‘bent’}  &     &     &     &  \\
\end{tabular}
%\caption{\nocaption}
     
\begin{tabular}{llllll}  
  \multicolumn{5}{l}{
                     \vernacular{(506) /Ø/
                    C-Initial + OP} \gloss{‘s/he has
                    not...him/her \ob mu-\cb  / them
                    } } &  \\
\multicolumn{5}{l}{ } &  \\

                     \vernacular{aamú[{\downstep}tsíílí]
                    tá}  &   
                     \gloss{‘gone for’}  &  \\

                     \vernacular{aamú[{\downstep}léshí]
                    tá}  &   
                     \gloss{‘left’}  &  \\

                     \vernacular{aamú[{\downstep}lóóndí]
                    tá}  &   
                     \gloss{‘followed’}  &  \\

                     \vernacular{
                    aamú[{\downstep}kúlííshɪ́] tá}  &   
                     \gloss{‘named’}  &  \\

                     \vernacular{
                    aamú[{\downstep}lákhúúlí] tá}  &   
                     \gloss{‘released’}  &  \\

                     \vernacular{
                    aamú[{\downstep}séébúúlɪ́] tá}  &   
                     \gloss{‘said bye
                    to’}  &  \\

                     \vernacular{
                    aamú[{\downstep}hóómbélíítsɪ́] tá}  &   
                     \gloss{‘comforted’}  &  \\

                     \vernacular{
                    aamú[{\downstep}kálúshíítsɪ́] tá}  &   
                     \gloss{‘returned’}  &  \\

                     \vernacular{
                    aamú[{\downstep}síínjílíítsɪ́] tá}  &   
                     \gloss{
                    ‘made...stand’}  &  \\

                     \vernacular{
                    aamú[{\downstep}réébíréébí] tá}  &   
                     \gloss{‘asked
                    (iter)’}  &  \\

                     \vernacular{
                    aamú[{\downstep}kálúkhányíínyɪ́] tá}  &   
                     \gloss{
                    ‘turned...over’}  &  \\

                     \vernacular{
                    aabí[{\downstep}sébúlúkhányíínyɪ́] tá}  &   
                     \gloss{‘scattered’}  &  \\
\end{tabular}
%\caption{\nocaption}
     
\begin{tabular}{llllll}  
  \multicolumn{5}{l}{
                     \vernacular{(507) /H/
                    C-Initial + OP + OP
                    } \gloss{‘s/he has
                    not...him/her for me’} } &  \\
\multicolumn{5}{l}{ } &  \\

                     \vernacular{
                    aamú{\downstep}ú[ndééléélɛ́] tá}  &   
                     \gloss{‘buried’}  &     &   
                     \vernacular{
                    aamú{\downstep}ú[mbéchéélɛ́] tá}  &   
                     \gloss{‘shaved’}  &  \\

                     \vernacular{
                    aamú{\downstep}ú[ndééréélɛ́] tá}  &   
                     \gloss{‘brought’}  &     &   
                     \vernacular{
                    aamú{\downstep}ú[kháláchíílɪ́] tá}  &   
                     \gloss{‘cut’}  &  \\

                     \vernacular{
                    aamú{\downstep}ú[sítááchíílɪ́] tá}  &   
                     \gloss{‘accused’}  &     &   
                     \vernacular{
                    aamú{\downstep}ú[mbóólítsíílɪ́] tá}  &   
                     \gloss{‘seduced’}  &  \\

                     \vernacular{
                    aamú{\downstep}ú[mbóhólóléélɛ́] tá}  &   
                     \gloss{‘untied’}  &     &     &     &  \\
\end{tabular}
%\caption{\nocaption}
     
\begin{tabular}{llllll}  
  \multicolumn{5}{l}{
                     \vernacular{(508) /Ø/
                    C-Initial + OP + OP
                    } \gloss{‘s/he has
                    not...him/her for me’} } &  \\
\multicolumn{5}{l}{ } &  \\

                     \vernacular{
                    aamú{\downstep}ú[nzíílíílɪ́] tá}  &   
                     \gloss{‘gone for’}  &     &   
                     \vernacular{
                    aamú{\downstep}ú[ndéshéélɛ́] tá}  &   
                     \gloss{‘gone for’}  &  \\

                     \vernacular{
                    aamú{\downstep}ú[nóóndéélɛ́] tá}  &   
                     \gloss{‘left’}  &     &   
                     \vernacular{
                    aamú{\downstep}ú[ngúlíshíílɪ́] tá}  &   
                     \gloss{‘followed’}  &  \\

                     \vernacular{
                    aamú{\downstep}ú[ndákhúúlíílɪ́] tá}  &   
                     \gloss{‘named’}  &     &   
                     \vernacular{
                    aamú{\downstep}ú[séébúlíílɪ́] tá}  &   
                     \gloss{‘released’}  &  \\

                     \vernacular{
                    aamú{\downstep}ú[mbóómbélítsíílɪ́] tá}  &   
                     \gloss{‘said bye
                    to’}  &     &   
                     \vernacular{
                    aamú{\downstep}ú[síínjílítsíílɪ́] tá}  &   
                     \gloss{‘comforted’}  &  \\
\end{tabular}
%\caption{\nocaption}
     
\begin{tabular}{lll}  
  \multicolumn{2}{l}{
                     \vernacular{(509) /H/
                    C-Initial Phrase-Medial} \gloss{‘s/he has
                    not...the boy \ob mú{\downstep}yáyi\cb  /} } &  \\
\multicolumn{2}{l}{
                     \gloss{someone
                    \ob muundu\cb ’} } &  \\

                     \vernacular{aa[réele]
                    mú{\downstep}yáyi/muundu tá}  &   
                     \gloss{‘bury’}  &  \\

                     \vernacular{aa[béchi]
                    mú{\downstep}yáyi/muundu tá}  &   
                     \gloss{‘shave’}  &  \\

                     \vernacular{aa[léeri]
                    mú{\downstep}yáyi/muundu tá}  &   
                     \gloss{‘bring’}  &  \\

                     \vernacular{aa[khálaachɛ]
                    mú{\downstep}yáyi/muundu tá}  &   
                     \gloss{‘cut’}  &  \\

                     \vernacular{aa[sítaachi]
                    mú{\downstep}yáyi/muundu tá}  &   
                     \gloss{‘accuse’}  &  \\

                     \vernacular{aa[bóoliitsɪ]
                    mú{\downstep}yáyi/muundu tá}  &   
                     \gloss{‘seduce’}  &  \\

                     \vernacular{aa[khóng’oondi]
                    mú{\downstep}yáyi/muundu tá}  &   
                     \gloss{‘knock’}  &  \\

                     \vernacular{aa[bóholoolɛ]
                    mú{\downstep}yáyi/muundu tá}  &   
                     \gloss{‘untie’}  &  \\

                     \vernacular{aa[bóyong’aanɛ]
                    mú{\downstep}yáyi/muundu tá}  &   
                     \gloss{‘go around’}  &  \\
\end{tabular}
%\caption{\nocaption}
     
\begin{tabular}{lll}  
  \multicolumn{2}{l}{
                     \vernacular{(510) /Ø/
                    C-Initial Phrase-Medial} \gloss{‘s/he has
                    not...the boy \ob mú{\downstep}yáyi\cb  /} } &  \\
\multicolumn{2}{l}{
                     \gloss{someone
                    \ob muundu\cb ’} } &  \\

                     \vernacular{aa[tsiili]
                    mú{\downstep}yáyi/muundu tá}  &   
                     \gloss{‘go for’}  &  \\

                     \vernacular{aa[leshi]
                    mú{\downstep}yáyi/muundu tá}  &   
                     \gloss{‘leave’}  &  \\

                     \vernacular{aa[loondi]
                    mú{\downstep}yáyi/muundu tá}  &   
                     \gloss{‘follow’}  &  \\

                     \vernacular{aa[kuliishɪ]
                    mú{\downstep}yáyi/muundu tá}  &   
                     \gloss{‘name’}  &  \\

                     \vernacular{aa[lakhuuli]
                    mú{\downstep}yáyi/muundu tá}  &   
                     \gloss{‘release’}  &  \\

                     \vernacular{aa[seebuulɪ]
                    mú{\downstep}yáyi/muundu tá}  &   
                     \gloss{‘say bye to’}  &  \\

                     \vernacular{aa[kalushiitsɪ]
                    mú{\downstep}yáyi/muundu tá}  &   
                     \gloss{‘return’}  &  \\

                     \vernacular{aa[reebireebi]
                    mú{\downstep}yáyi/muundu tá}  &   
                     \gloss{‘ask (iter)’}  &  \\
\end{tabular}
%\caption{\nocaption}
     
\begin{tabular}{lll}  
  \multicolumn{2}{l}{
                     \vernacular{(511) /H/
                    C-Initial +OP Phrase-Medial} \gloss{‘s/he has
                    not...the boy \ob mú{\downstep}yáyi\cb  /} } &  \\
\multicolumn{2}{l}{
                     \gloss{someone \ob muundu\cb 
                    for him/her’} } &  \\

                     \vernacular{aamú[reeleelɛ]
                    mú{\downstep}yáyi/muundu tá}  &   
                     \gloss{‘bury’}  &  \\

                     \vernacular{aamú[becheelɛ]
                    mú{\downstep}yáyi/muundu tá}  &   
                     \gloss{‘shave’}  &  \\

                     \vernacular{aamú[leereelɛ]
                    mú{\downstep}yáyi/muundu tá}  &   
                     \gloss{‘bring’}  &  \\

                     \vernacular{aamú[khalachiilɪ]
                    mú{\downstep}yáyi/muundu tá}  &   
                     \gloss{‘cut’}  &  \\

                     \vernacular{aamú[sitaachiilɪ]
                    mú{\downstep}yáyi/muundu tá}  &   
                     \gloss{‘accuse’}  &  \\

                     \vernacular{aamú[boolitsiilɪ]
                    mú{\downstep}yáyi/muundu tá}  &   
                     \gloss{‘seduce’}  &  \\

                     \vernacular{
                    aamú[khong’oondeelɛ] mú{\downstep}yáyi/muundu
                    tá}  &   
                     \gloss{‘knock’}  &  \\

                     \vernacular{aamú[bohololeelɛ]
                    mú{\downstep}yáyi/muundu tá}  &   
                     \gloss{‘untie’}  &  \\

                     \vernacular{
                    aamú[boyong’aniilɪ] mú{\downstep}yáyi/muundu
                    tá}  &   
                     \gloss{‘go around’}  &  \\
\end{tabular}
%\caption{\nocaption}
     
\begin{tabular}{lll}  
  \multicolumn{2}{l}{
                     \vernacular{(512) /Ø/
                    C-Initial +OP Phrase-Medial} \gloss{‘s/he has
                    not...the boy \ob mú{\downstep}yáyi\cb  /} } &  \\
\multicolumn{2}{l}{
                     \gloss{someone \ob muundu\cb 
                    for him/her’} } &  \\

                     \vernacular{aamú[tsiiliilɪ]
                    mú{\downstep}yáyi/muundu tá}  &   
                     \gloss{‘go for’}  &  \\

                     \vernacular{aamú[lesheelɛ]
                    mú{\downstep}yáyi/muundu tá}  &   
                     \gloss{‘leave’}  &  \\

                     \vernacular{aamú[loondeelɛ]
                    mú{\downstep}yáyi/muundu tá}  &   
                     \gloss{‘follow’}  &  \\

                     \vernacular{aamú[kulishiilɪ]
                    mú{\downstep}yáyi/muundu tá}  &   
                     \gloss{‘name’}  &  \\

                     \vernacular{aamú[lakhuuliilɪ]
                    mú{\downstep}yáyi/muundu tá}  &   
                     \gloss{‘release’}  &  \\

                     \vernacular{aamú[seebuliilɪ]
                    mú{\downstep}yáyi/muundu tá}  &   
                     \gloss{‘say bye to’}  &  \\

                     \vernacular{
                    aamú[kalushitsiilɪ] mú{\downstep}yáyi/muundu
                    tá}  &   
                     \gloss{‘return’}  &  \\

                     \vernacular{
                    aamú[reebɛreebeelɛ] mú{\downstep}yáyi/muundu
                    tá}  &   
                     \gloss{‘ask (iter)’}  &  \\
\end{tabular}
%\caption{\nocaption}
     
\begin{tabular}{lll}  
  \multicolumn{2}{l}{
                     \vernacular{(513) /H/
                    C-Initial +OP + OP
                    } \gloss{‘s/he has
                    not...the boy \ob mú{\downstep}yáyi\cb  /} } &  \\
\multicolumn{2}{l}{
                     \gloss{someone \ob muundu\cb 
                    for him/her for me’} } &  \\

                     \vernacular{aamúu[ndeeleelɛ]
                    mú{\downstep}yáyi/muundu tá}  &   
                     \gloss{‘bury’}  &  \\

                     \vernacular{aamúu[mbecheelɛ]
                    mú{\downstep}yáyi/muundu tá}  &   
                     \gloss{‘shave’}  &  \\

                     \vernacular{aamúu[ndeereelɛ]
                    mú{\downstep}yáyi/muundu tá}  &   
                     \gloss{‘bring’}  &  \\

                     \vernacular{
                    aamúu[khalachiilɪ] mú{\downstep}yáyi/muundu
                    tá}  &   
                     \gloss{‘cut’}  &  \\

                     \vernacular{
                    aamúu[sitaachiilɪ] mú{\downstep}yáyi/muundu
                    tá}  &   
                     \gloss{‘accuse’}  &  \\

                     \vernacular{
                    aamúu[mboolitsiilɪ] mú{\downstep}yáyi/muundu
                    tá}  &   
                     \gloss{‘seduce’}  &  \\

                     \vernacular{
                    aamúu[mbohololeelɛ] mú{\downstep}yáyi/muundu
                    tá}  &   
                     \gloss{‘untie’}  &  \\
\end{tabular}
%\caption{\nocaption}
     
\begin{tabular}{lll}  
  \multicolumn{2}{l}{
                     \vernacular{(514) /Ø/
                    C-Initial +OP + OP
                    } \gloss{‘s/he has
                    not...the boy \ob mú{\downstep}yáyi\cb  /} } &  \\
\multicolumn{2}{l}{
                     \gloss{someone \ob muundu\cb 
                    for him/her for me’} } &  \\

                     \vernacular{aamúu[nziiliilɪ]
                    mú{\downstep}yáyi/muundu tá}  &   
                     \gloss{‘go for’}  &  \\

                     \vernacular{aamúu[ndesheelɛ]
                    mú{\downstep}yáyi/muundu tá}  &   
                     \gloss{‘leave’}  &  \\

                     \vernacular{aamúu[noondeelɛ]
                    mú{\downstep}yáyi/muundu tá}  &   
                     \gloss{‘follow’}  &  \\

                     \vernacular{
                    aamúu[ngulishiilɪ] mú{\downstep}yáyi/muundu
                    tá}  &   
                     \gloss{‘name’}  &  \\

                     \vernacular{
                    aamúu[ndakhuuliilɪ] mú{\downstep}yáyi/muundu
                    tá}  &   
                     \gloss{‘release’}  &  \\

                     \vernacular{aamúu[seebuliilɪ]
                    mú{\downstep}yáyi/muundu tá}  &   
                     \gloss{‘say bye to’}  &  \\

                     \vernacular{
                    aamúu[siinjilitsiilɪ] mú{\downstep}yáyi/muundu
                    tá}  &   
                     \gloss{
                    ‘make...stand’}  &  \\
\end{tabular}
%\caption{\nocaption}
    

\subsection{Perfect (2
              }\label{sec:sPerf2ndSg}


\begin{tabular}{llllll}  
  \multicolumn{5}{l}{
                     \vernacular{(515) /H/
                    C-Initial} \gloss{‘you
                    have...’} } &  \\
\multicolumn{5}{l}{ } &  \\

                     \vernacular{
                    uu[reele]}  &   
                     \gloss{‘buried’}  &     &   
                     \vernacular{
                    uu[ng’weele]}  &   
                     \gloss{‘drunk’}  &  \\

                     \vernacular{
                    uu[khweele]}  &   
                     \gloss{‘eaten’}  &     &   
                     \vernacular{
                    uu[liili]}  &   
                     \gloss{‘paid dowry’}  &  \\

                     \vernacular{
                    uu[lumi]}  &   
                     \gloss{‘bitten’}  &     &   
                     \vernacular{
                    uu[bechi]}  &   
                     \gloss{‘shaved’}  &  \\

                     \vernacular{
                    uu[teeshi]}  &   
                     \gloss{‘cooked’}  &     &   
                     \vernacular{
                    uu[leeri]}  &   
                     \gloss{‘brought’}  &  \\

                     \vernacular{
                    uu[khalaachɛ]}  &   
                     \gloss{‘cut’}  &     &   
                     \vernacular{
                    uu[kalaanji]}  &   
                     \gloss{‘fried’}  &  \\

                     \vernacular{
                    uu[sitaachi]}  &   
                     \gloss{‘accused’}  &     &   
                     \vernacular{
                    uu[booliitsɪ]}  &   
                     \gloss{‘seduced’}  &  \\

                     \vernacular{
                    uu[saandiitsɪ]}  &   
                     \gloss{‘thanked’}  &     &   
                     \vernacular{
                    uu[khong’oondi]}  &   
                     \gloss{‘knocked’}  &  \\

                     \vernacular{
                    uu[boholoolɛ]}  &   
                     \gloss{‘untied’}  &     &   
                     \vernacular{
                    uu[boyong’aanɛ]}  &   
                     \gloss{‘gone
                    around’}  &  \\

                     \vernacular{
                    uu[ng’ong’ooliitsɪ]}  &   
                     \gloss{‘teased’}  &     &   
                     \vernacular{
                    uu[ling(ak)anyiinyɪ]}  &   
                     \gloss{‘crumpled’}  &  \\
\end{tabular}
%\caption{\nocaption}
     
\begin{tabular}{llllll}  
  \multicolumn{5}{l}{
                     \vernacular{(516) /H/
                    V-Initial} \gloss{‘you
                    have...’} } &  \\
\multicolumn{5}{l}{ } &  \\

                     \vernacular{w[iiri]}  &   
                     \gloss{‘killed’}  &     &   
                     \vernacular{
                    w[iikoombi]}  &   
                     \gloss{‘admired’}  &  \\

                     \vernacular{
                    w[iisiachi]}  &   
                     \gloss{‘smacked’}  &     &   
                     \vernacular{
                    w[iikoboolɛ]}  &   
                     \gloss{‘belched’}  &  \\

                     \vernacular{
                    w[oononyiinyɪ]}  &   
                     \gloss{‘spoiled’}  &     &   
                     \vernacular{
                    w[aabukhanyiinyɪ]}  &   
                     \gloss{‘separated’}  &  \\
\end{tabular}
%\caption{\nocaption}
     
\begin{tabular}{llllll}  
  \multicolumn{5}{l}{
                     \vernacular{(517) /Ø/
                    C-Initial} \gloss{‘you
                    have...’} } &  \\
\multicolumn{5}{l}{ } &  \\

                     \vernacular{
                    uu[tsiili]}  &   
                     \gloss{‘gone’}  &     &   
                     \vernacular{
                    uu[kwiili]}  &   
                     \gloss{‘fallen’}  &  \\

                     \vernacular{
                    uu[leshi]}  &   
                     \gloss{‘left’}  &     &   
                     \vernacular{
                    uu[reebi]}  &   
                     \gloss{‘asked’}  &  \\

                     \vernacular{
                    uu[loondi]}  &   
                     \gloss{‘followed’}  &     &   
                     \vernacular{
                    uu[kumiilɪ]}  &   
                     \gloss{‘held’}  &  \\

                     \vernacular{
                    uu[kuliishɪ]}  &   
                     \gloss{‘named’}  &     &   
                     \vernacular{
                    uu[homooli]}  &   
                     \gloss{‘massaged’}  &  \\

                     \vernacular{
                    uu[lakhuuli]}  &   
                     \gloss{‘released’}  &     &   
                     \vernacular{
                    uu[seebuulɪ]}  &   
                     \gloss{‘said bye’}  &  \\

                     \vernacular{
                    uu[hoombeliitsɪ]}  &   
                     \gloss{‘comforted’}  &     &   
                     \vernacular{
                    uu[kalushiitsɪ]}  &   
                     \gloss{‘returned’}  &  \\

                     \vernacular{
                    uu[siinjiliitsɪ]}  &   
                     \gloss{‘made stand’}  &     &   
                     \vernacular{
                    uu[reebireebi]}  &   
                     \gloss{‘asked
                    (iter)’}  &  \\

                     \vernacular{
                    uu[kalukhanyiinyɪ]}  &   
                     \gloss{‘turned
                    over’}  &     &   
                     \vernacular{
                    uu[sebulukhanyiinyɪ]}  &   
                     \gloss{‘scattered’}  &  \\
\end{tabular}
%\caption{\nocaption}
     
\begin{tabular}{llllll}  
  \multicolumn{5}{l}{
                     \vernacular{(518) /Ø/
                    V-Initial} \gloss{‘you
                    have...’} } &  \\
\multicolumn{5}{l}{ } &  \\

                     \vernacular{
                    w[eenyi]}  &   
                     \gloss{‘wanted’}  &     &   
                     \vernacular{
                    w[eeyeelɛ]}  &   
                     \gloss{‘wiped for’}  &  \\

                     \vernacular{
                    w[iiluuli]}  &   
                     \gloss{‘winnowed’}  &     &   
                     \vernacular{
                    w[aambakhaanɛ]}  &   
                     \gloss{‘refused’}  &  \\

                     \vernacular{
                    w[eeleeliitsɪ]}  &   
                     \gloss{‘hung up’}  &     &     &     &  \\
\end{tabular}
%\caption{\nocaption}
     
\begin{tabular}{llllll}  
  \multicolumn{5}{l}{
                     \vernacular{(519) /H/
                    C-Initial + OP} \gloss{‘you
                    have...him/her’} } &  \\
\multicolumn{5}{l}{ } &  \\

                     \vernacular{
                    uumu[reele]}  &   
                     \gloss{‘buried’}  &     &   
                     \vernacular{
                    uumu[bechi]}  &   
                     \gloss{‘shaved’}  &  \\

                     \vernacular{
                    uumu[leeri]}  &   
                     \gloss{‘brought’}  &     &   
                     \vernacular{
                    uumu[khalaachɛ]}  &   
                     \gloss{‘cut’}  &  \\

                     \vernacular{
                    uumu[sitaachi]}  &   
                     \gloss{‘accused’}  &     &   
                     \vernacular{
                    uumu[booliitsɪ]}  &   
                     \gloss{‘seduced’}  &  \\

                     \vernacular{
                    uumu[khong’oondi]}  &   
                     \gloss{‘knocked’}  &     &   
                     \vernacular{
                    uumu[boholoolɛ]}  &   
                     \gloss{‘untied’}  &  \\

                     \vernacular{
                    uumu[boyong’aanɛ]}  &   
                     \gloss{‘gone
                    around’}  &     &   
                     \vernacular{
                    uumu[ng’ong’ooliitsɪ]}  &   
                     \gloss{‘teased’}  &  \\

                     \vernacular{
                    uumu[lingakanyiinyɪ]}  &   
                     \gloss{‘bent’}  &     &     &     &  \\
\end{tabular}
%\caption{\nocaption}
     
\begin{tabular}{llllll}  
  \multicolumn{5}{l}{
                     \vernacular{(520) /H/
                    V-Initial + OP} \gloss{‘you
                    have...him/her’} } &  \\
\multicolumn{5}{l}{ } &  \\

                     \vernacular{
                    uumw[iiri]}  &   
                     \gloss{‘killed’}  &     &   
                     \vernacular{
                    uumw[iikoombi]}  &   
                     \gloss{‘admired’}  &  \\

                     \vernacular{
                    uumw[iisiachi]}  &   
                     \gloss{‘smacked’}  &     &   
                     \vernacular{
                    uumw[oononyiinyɪ]}  &   
                     \gloss{‘spoiled’}  &  \\

                     \vernacular{
                    uumw[aabukhanyiinyɪ]}  &   
                     \gloss{‘separated’}  &  \\
\end{tabular}
%\caption{\nocaption}
     
\begin{tabular}{llllll}  
  \multicolumn{5}{l}{
                     \vernacular{(521) /Ø/
                    C-Initial + OP} \gloss{‘you
                    have...him/her \ob mu-\cb  / them
                    } } &  \\
\multicolumn{5}{l}{ } &  \\

                     \vernacular{
                    uumu[tsiili]}  &   
                     \gloss{‘gone for’}  &  \\

                     \vernacular{
                    uumu[leshi]}  &   
                     \gloss{‘left’}  &  \\

                     \vernacular{
                    uumu[loondi]}  &   
                     \gloss{‘followed’}  &  \\

                     \vernacular{
                    uumu[kuliishɪ]}  &   
                     \gloss{‘named’}  &  \\

                     \vernacular{
                    uumu[lakhuuli]}  &   
                     \gloss{‘released’}  &  \\

                     \vernacular{
                    uumu[seebuulɪ]}  &   
                     \gloss{‘said bye
                    to’}  &  \\

                     \vernacular{
                    uumu[hoombeliitsɪ]}  &   
                     \gloss{‘comforted’}  &  \\

                     \vernacular{
                    uumu[kalushiitsɪ]}  &   
                     \gloss{‘returned’}  &  \\

                     \vernacular{
                    uumu[siinjiliitsɪ]}  &   
                     \gloss{
                    ‘made...stand’}  &  \\

                     \vernacular{
                    uumu[reebireebi]}  &   
                     \gloss{‘asked
                    (iter)’}  &  \\

                     \vernacular{
                    uumu[kalukhanyiinyɪ]}  &   
                     \gloss{
                    ‘turned...over’}  &  \\

                     \vernacular{
                    uubi[sebulukhanyiinyɪ]}  &   
                     \gloss{‘scattered’}  &  \\
\end{tabular}
%\caption{\nocaption}
     
\begin{tabular}{llllll}  
  \multicolumn{5}{l}{
                     \vernacular{(522) /Ø/
                    V-Initial + OP} \gloss{‘you
                    have...him/her \ob mw-\cb  / it
                    } } &  \\
\multicolumn{5}{l}{ } &  \\

                     \vernacular{
                    uumw[eenyi]}  &   
                     \gloss{‘wanted’}  &     &   
                     \vernacular{
                    uumw[eeyeelɛ]}  &   
                     \gloss{‘wiped for’}  &  \\

                     \vernacular{
                    uubw[iiluuli]}  &   
                     \gloss{‘winnowed’}  &     &   
                     \vernacular{
                    uumw[aambakhaanɛ]}  &   
                     \gloss{‘refused’}  &  \\

                     \vernacular{
                    uumw[eeleeliitsɪ]}  &   
                     \gloss{
                    ‘carried...hanging’}  &  \\
\end{tabular}
%\caption{\nocaption}
     
\begin{tabular}{llllll}  
  \multicolumn{5}{l}{
                     \vernacular{(523) /H/
                    C-Initial + OP
                    } \gloss{‘you
                    have...me’} } &  \\
\multicolumn{5}{l}{ } &  \\

                     \vernacular{
                    uu[riili]}  &   
                     \gloss{‘feared’}  &     &   
                     \vernacular{
                    uu[mbechi]}  &   
                     \gloss{‘shaved’}  &  \\

                     \vernacular{
                    uu[ndeeri]}  &   
                     \gloss{‘brought’}  &     &   
                     \vernacular{
                    uu[khalaachɛ]}  &   
                     \gloss{‘cut’}  &  \\

                     \vernacular{
                    uu[sitaachi]}  &   
                     \gloss{‘accused’}  &     &   
                     \vernacular{
                    uu[mbooliitsɪ]}  &   
                     \gloss{‘seduced’}  &  \\

                     \vernacular{
                    uu[khong’oondi]}  &   
                     \gloss{‘knocked’}  &     &   
                     \vernacular{
                    uu[mboholoolɛ]}  &   
                     \gloss{‘untied’}  &  \\

                     \vernacular{
                    uu[mboyong’aanɛ]}  &   
                     \gloss{‘gone
                    around’}  &     &   
                     \vernacular{
                    uu[ng’ong’ooliitsɪ]}  &   
                     \gloss{‘teased’}  &  \\

                     \vernacular{
                    uu[ningakanyiinyɪ]}  &   
                     \gloss{‘bent’}  &  \\
\end{tabular}
%\caption{\nocaption}
     
\begin{tabular}{llllll}  
  \multicolumn{5}{l}{
                     \vernacular{(524) /H/
                    V-Initial + OP
                    } \gloss{‘you
                    have...me’} } &  \\
\multicolumn{5}{l}{ } &  \\

                     \vernacular{
                    uu[nziri]}  &   
                     \gloss{‘killed’}  &     &   
                     \vernacular{
                    uu[nzikoombi]}  &   
                     \gloss{‘admired’}  &  \\

                     \vernacular{
                    uu[nzisiachi]}  &   
                     \gloss{‘smacked’}  &     &   
                     \vernacular{
                    uu[nzononyiinyɪ]}  &   
                     \gloss{‘spoiled’}  &  \\

                     \vernacular{
                    uu[nzabukhanyiinyɪ]}  &   
                     \gloss{‘separated’}  &  \\
\end{tabular}
%\caption{\nocaption}
     
\begin{tabular}{llllll}  
  \multicolumn{5}{l}{
                     \vernacular{(525) /Ø/
                    C-Initial + OP
                    } \gloss{‘you
                    have...me’} } &  \\
\multicolumn{5}{l}{ } &  \\

                     \vernacular{
                    uu[siele]}  &   
                     \gloss{‘ground’}  &     &   
                     \vernacular{
                    uu[ndeshi]}  &   
                     \gloss{‘left’}  &  \\

                     \vernacular{
                    uu[noondi]}  &   
                     \gloss{‘followed’}  &     &   
                     \vernacular{
                    uu[nguliishɪ]}  &   
                     \gloss{‘named’}  &  \\

                     \vernacular{
                    uu[ndakhuuli]}  &   
                     \gloss{‘released’}  &     &   
                     \vernacular{
                    uu[seebuulɪ]}  &   
                     \gloss{‘said bye
                    to’}  &  \\

                     \vernacular{
                    uu[mboombeliitsɪ]}  &   
                     \gloss{‘comforted’}  &     &   
                     \vernacular{
                    uu[siinjiliitsɪ]}  &   
                     \gloss{
                    ‘made..stand’}  &  \\

                     \vernacular{
                    uu[ndeebindeebi]}  &   
                     \gloss{‘asked
                    (iter)’}  &     &   
                     \vernacular{
                    uu[ngalukhanyiinyɪ]}  &   
                     \gloss{
                    ‘turned...over’}  &  \\
\end{tabular}
%\caption{\nocaption}
     
\begin{tabular}{llllll}  
  \multicolumn{5}{l}{
                     \vernacular{(526) /Ø/
                    V-Initial + OP
                    } \gloss{‘you
                    have...me’} } &  \\
\multicolumn{5}{l}{ } &  \\

                     \vernacular{
                    uu[nzenyi]}  &   
                     \gloss{‘wanted’}  &     &   
                     \vernacular{
                    uu[nzeyeelɛ]}  &   
                     \gloss{‘wiped for’}  &  \\

                     \vernacular{
                    uu[nyambakhaanɛ]}  &   
                     \gloss{‘refused’}  &     &   
                     \vernacular{
                    uu[nzeleeliitsɪ]}  &   
                     \gloss{
                    ‘carried...hanging’}  &  \\
\end{tabular}
%\caption{\nocaption}
     
\begin{tabular}{llllll}  
  \multicolumn{5}{l}{
                     \vernacular{(527) /H/
                    C-Initial + OP
                    } \gloss{‘you
                    have...him/herself’} } &  \\
\multicolumn{5}{l}{ } &  \\

                     \vernacular{
                    wii[reele]}  &   
                     \gloss{‘buried’}  &     &   
                     \vernacular{
                    wii[bechi]}  &   
                     \gloss{‘shaved’}  &  \\

                     \vernacular{
                    wii[suunji]}  &   
                     \gloss{‘hung’}  &     &   
                     \vernacular{
                    wii[khalaachɛ]}  &   
                     \gloss{‘cut’}  &  \\

                     \vernacular{
                    wii[sitaachi]}  &   
                     \gloss{‘accused’}  &     &   
                     \vernacular{
                    wii[saandiitsɪ]}  &   
                     \gloss{‘thanked’}  &  \\

                     \vernacular{
                    wii[khong’oondi]}  &   
                     \gloss{‘knocked’}  &     &   
                     \vernacular{
                    wii[boholoolɛ]}  &   
                     \gloss{‘untied’}  &  \\
\end{tabular}
%\caption{\nocaption}
     
\begin{tabular}{llllll}  
  \multicolumn{5}{l}{
                     \vernacular{(528) /H/
                    V-Initial + OP
                    } \gloss{‘you
                    have...him/herself’} } &  \\
\multicolumn{5}{l}{ } &  \\

                     \vernacular{
                    wii[yiri]}  &   
                     \gloss{‘killed’}  &     &   
                     \vernacular{
                    wii[yikoombi]}  &   
                     \gloss{‘admired’}  &  \\

                     \vernacular{
                    wii[yisiachi]}  &   
                     \gloss{‘smacked’}  &     &   
                     \vernacular{
                    wii[yononyiinyɪ]}  &   
                     \gloss{‘spoiled’}  &  \\

                     \vernacular{
                    wii[yabukhanyiinyɪ]}  &   
                     \gloss{‘separated’}  &  \\
\end{tabular}
%\caption{\nocaption}
     
\begin{tabular}{llllll}  
  \multicolumn{5}{l}{
                     \vernacular{(529) /Ø/
                    C-Initial + OP
                    } \gloss{‘you
                    have...him/herself’} } &  \\
\multicolumn{5}{l}{ } &  \\

                     \vernacular{
                    wii[siele]}  &   
                     \gloss{‘ground’}  &     &   
                     \vernacular{
                    wii[leshi]}  &   
                     \gloss{‘left’}  &  \\

                     \vernacular{
                    wii[siinji]}  &   
                     \gloss{‘bathed’}  &     &   
                     \vernacular{
                    wii[kuliishɪ]}  &   
                     \gloss{‘named’}  &  \\

                     \vernacular{
                    wii[naabuulɪ]}  &   
                     \gloss{‘undressed’}  &     &   
                     \vernacular{
                    wii[lakhuuli]}  &   
                     \gloss{‘released’}  &  \\

                     \vernacular{
                    wii[hoombeliitsɪ]}  &   
                     \gloss{‘comforted’}  &     &   
                     \vernacular{
                    wii[siinjiliitsɪ]}  &   
                     \gloss{
                    ‘made...stand’}  &  \\

                     \vernacular{
                    wii[reebireebi]}  &   
                     \gloss{‘asked
                    (iter)’}  &     &   
                     \vernacular{
                    wii[kalukhanyiinyɪ]}  &   
                     \gloss{
                    ‘turned...over’}  &  \\
\end{tabular}
%\caption{\nocaption}
     
\begin{tabular}{llllll}  
  \multicolumn{5}{l}{
                     \vernacular{(530) /Ø/
                    V-Initial + OP
                    } \gloss{‘you
                    have...him/herself’} } &  \\
\multicolumn{5}{l}{ } &  \\

                     \vernacular{
                    wii[yali]}  &   
                     \gloss{‘exposed’}  &     &   
                     \vernacular{
                    wii[yeyeelɛ]}  &   
                     \gloss{‘wiped for’}  &  \\

                     \vernacular{
                    wii[yambakhaanɛ]}  &   
                     \gloss{‘refused’}  &     &   
                     \vernacular{
                    wii[yeleeliitsɪ]}  &   
                     \gloss{‘hung...up’}  &  \\
\end{tabular}
%\caption{\nocaption}
     
\begin{tabular}{llllll}  
  \multicolumn{5}{l}{
                     \vernacular{(531) /H/
                    C-Initial + OP + OP
                    } \gloss{‘you
                    have...him/her for me’} } &  \\
\multicolumn{5}{l}{ } &  \\

                     \vernacular{
                    uumuu[ndeeleelɛ]}  &   
                     \gloss{‘buried’}  &     &   
                     \vernacular{
                    uumuu[mbecheelɛ]}  &   
                     \gloss{‘shaved’}  &  \\

                     \vernacular{
                    uumuu[ndeereelɛ]}  &   
                     \gloss{‘brought’}  &     &   
                     \vernacular{
                    uumuu[khalachiilɪ]}  &   
                     \gloss{‘cut’}  &  \\

                     \vernacular{
                    uumuu[sitaachiilɪ]}  &   
                     \gloss{‘accused’}  &     &   
                     \vernacular{
                    uumuu[mboolitsiilɪ]}  &   
                     \gloss{‘seduced’}  &  \\

                     \vernacular{
                    uumuu[mbohololeelɛ]}  &   
                     \gloss{‘untied’}  &     &     &     &  \\
\end{tabular}
%\caption{\nocaption}
     
\begin{tabular}{llllll}  
  \multicolumn{5}{l}{
                     \vernacular{(532) /H/
                    V-Initial + OP + OP
                    } \gloss{‘you
                    have...him/her for me’} } &  \\
\multicolumn{5}{l}{ } &  \\

                     \vernacular{
                    uumuu[nziiriilɪ]}  &   
                     \gloss{‘killed’}  &     &   
                     \vernacular{
                    uumuu[nzechitsiilɪ]}  &   
                     \gloss{‘admired’}  &  \\

                     \vernacular{
                    uumuu[nzisiachiilɪ]}  &   
                     \gloss{‘smacked’}  &     &   
                     \vernacular{
                    uumuu[nzononyinyiilɪ]}  &   
                     \gloss{‘spoiled’}  &  \\

                     \vernacular{
                    uumuu[nzabukhanyinyiilɪ]}  &   
                     \gloss{‘separated’}  &     &     &     &  \\
\end{tabular}
%\caption{\nocaption}
     
\begin{tabular}{llllll}  
  \multicolumn{5}{l}{
                     \vernacular{(533) /Ø/
                    C-Initial + OP + OP
                    } \gloss{‘you
                    have...him/her for me’} } &  \\
\multicolumn{5}{l}{ } &  \\

                     \vernacular{
                    uumuu[nziiliilɪ]}  &   
                     \gloss{‘gone for’}  &     &   
                     \vernacular{
                    uumuu[ndesheelɛ]}  &   
                     \gloss{‘gone for’}  &  \\

                     \vernacular{
                    uumuu[noondeelɛ]}  &   
                     \gloss{‘left’}  &     &   
                     \vernacular{
                    uumuu[ngulishiilɪ]}  &   
                     \gloss{‘followed’}  &  \\

                     \vernacular{
                    uumuu[ndakhuuliilɪ]}  &   
                     \gloss{‘named’}  &     &   
                     \vernacular{
                    uumuu[seebuliilɪ]}  &   
                     \gloss{‘released’}  &  \\

                     \vernacular{
                    uumuu[mboombelitsiilɪ]}  &   
                     \gloss{‘said bye
                    to’}  &     &   
                     \vernacular{
                    uumuu[siinjilitsiilɪ]}  &   
                     \gloss{‘comforted’}  &  \\
\end{tabular}
%\caption{\nocaption}
     
\begin{tabular}{llllll}  
  \multicolumn{5}{l}{
                     \vernacular{(534) /Ø/
                    V-Initial + OP + OP
                    } \gloss{‘you
                    have...him/her \ob mu-\cb  / it
                    } } &  \\
\multicolumn{5}{l}{ } &  \\

                     \vernacular{
                    uumuu[nzeyeelɛ]}  &   
                     \gloss{‘wiped’}  &     &   
                     \vernacular{
                    uukuu[nzashitsiilɪ]}  &   
                     \gloss{‘lit’}  &  \\

                     \vernacular{
                    uubuu[nziluuliilɪ]}  &   
                     \gloss{‘winnowed’}  &     &   
                     \vernacular{
                    uukuu[nzéléelitsiilɪ]}  &   
                     \gloss{‘hung’}  &  \\
\end{tabular}
%\caption{\nocaption}
     
\begin{tabular}{lll}  
  \multicolumn{2}{l}{
                     \vernacular{(535) /H/
                    C-Initial Phrase-Medial} \gloss{‘you have...the
                    boy \ob mú{\downstep}yáyi\cb  /} } &  \\
\multicolumn{2}{l}{
                     \gloss{someone
                    \ob muundu\cb ’} } &  \\

                     \vernacular{uu[reele]
                    mú{\downstep}yáyi/muundu}  &   
                     \gloss{‘buried’}  &  \\

                     \vernacular{uu[bechi]
                    mú{\downstep}yáyi/muundu}  &   
                     \gloss{‘shaved’}  &  \\

                     \vernacular{uu[leeri]
                    mú{\downstep}yáyi/muundu}  &   
                     \gloss{‘brought’}  &  \\

                     \vernacular{uu[khalaachɛ]
                    mú{\downstep}yáyi/muundu}  &   
                     \gloss{‘cut’}  &  \\

                     \vernacular{uu[sitaachi]
                    mú{\downstep}yáyi/muundu}  &   
                     \gloss{‘accused’}  &  \\

                     \vernacular{uu[booliitsɪ]
                    mú{\downstep}yáyi/muundu}  &   
                     \gloss{‘seduced’}  &  \\

                     \vernacular{uu[khong’oondi]
                    mú{\downstep}yáyi/muundu}  &   
                     \gloss{‘knocked’}  &  \\

                     \vernacular{uu[boholoolɛ]
                    mú{\downstep}yáyi/muundu}  &   
                     \gloss{‘untied’}  &  \\

                     \vernacular{uu[boyong’aanɛ]
                    mú{\downstep}yáyi/muundu}  &   
                     \gloss{‘gone
                    around’}  &  \\
\end{tabular}
%\caption{\nocaption}
     
\begin{tabular}{lll}  
  \multicolumn{2}{l}{
                     \vernacular{(536) /Ø/
                    C-Initial Phrase-Medial} \gloss{‘you have...the
                    boy \ob mú{\downstep}yáyi\cb  /} } &  \\
\multicolumn{2}{l}{
                     \gloss{the man
                    \ob musáatsa\cb  / someone \ob muundu\cb ’} } &  \\

                     \vernacular{uu[tsiili]
                    mú{\downstep}yáyi/musáatsa/muundu}  &   
                     \gloss{‘gone for’}  &  \\

                     \vernacular{uu[leshi]
                    mú{\downstep}yáyi/musáatsa/muundu}  &   
                     \gloss{‘left’}  &  \\

                     \vernacular{uu[loondi]
                    mú{\downstep}yáyi/musáatsa/muundu}  &   
                     \gloss{‘followed’}  &  \\

                     \vernacular{uu[kuliishɪ]
                    mú{\downstep}yáyi/musáatsa/muundu}  &   
                     \gloss{‘named’}  &  \\

                     \vernacular{uu[lakhuuli]
                    mú{\downstep}yáyi/musáatsa/muundu}  &   
                     \gloss{‘released’}  &  \\

                     \vernacular{uu[seebuulɪ]
                    mú{\downstep}yáyi/musáatsa/muundu}  &   
                     \gloss{‘said bye
                    to’}  &  \\

                     \vernacular{uu[kalushiitsɪ]
                    mú{\downstep}yáyi/musáatsa/muundu}  &   
                     \gloss{‘returned’}  &  \\

                     \vernacular{uu[reebireebi]
                    mú{\downstep}yáyi/musáatsa/muundu}  &   
                     \gloss{‘asked
                    (iter)’}  &  \\
\end{tabular}
%\caption{\nocaption}
     
\begin{tabular}{lll}  
  \multicolumn{2}{l}{
                     \vernacular{(537) /H/
                    C-Initial +OP Phrase-Medial} \gloss{‘you have...the
                    boy \ob mú{\downstep}yáyi\cb  /} } &  \\
\multicolumn{2}{l}{
                     \gloss{someone \ob muundu\cb 
                    for him/her’} } &  \\

                     \vernacular{uumu[reeleelɛ]
                    mú{\downstep}yáyi/muundu}  &   
                     \gloss{‘buried’}  &  \\

                     \vernacular{uumu[becheelɛ]
                    mú{\downstep}yáyi/muundu}  &   
                     \gloss{‘shaved’}  &  \\

                     \vernacular{uumu[leereelɛ]
                    mú{\downstep}yáyi/muundu}  &   
                     \gloss{‘brought’}  &  \\

                     \vernacular{uumu[khalachiilɪ]
                    mú{\downstep}yáyi/muundu}  &   
                     \gloss{‘cut’}  &  \\

                     \vernacular{uumu[sitaachiilɪ]
                    mú{\downstep}yáyi/muundu}  &   
                     \gloss{‘accused’}  &  \\

                     \vernacular{uumu[boolitsiilɪ]
                    mú{\downstep}yáyi/muundu}  &   
                     \gloss{‘seduced’}  &  \\

                     \vernacular{
                    uumu[khong’oondeelɛ]
                    mú{\downstep}yáyi/muundu}  &   
                     \gloss{‘knocked’}  &  \\

                     \vernacular{uumu[bohololeelɛ]
                    mú{\downstep}yáyi/muundu}  &   
                     \gloss{‘untied’}  &  \\

                     \vernacular{
                    uumu[boyong’aniilɪ] mú{\downstep}yáyi/muundu}  &   
                     \gloss{‘gone
                    around’}  &  \\
\end{tabular}
%\caption{\nocaption}
     
\begin{tabular}{lll}  
  \multicolumn{2}{l}{
                     \vernacular{(538) /Ø/
                    C-Initial +OP Phrase-Medial} \gloss{‘you have...the
                    boy \ob mú{\downstep}yáyi\cb  /} } &  \\
\multicolumn{2}{l}{
                     \gloss{the man
                    \ob musáatsa\cb  / someone \ob muundu\cb  for
                    him/her’} } &  \\

                     \vernacular{uumu[tsiiliilɪ]
                    mú{\downstep}yáyi/musáatsa/muundu}  &   
                     \gloss{‘gone for’}  &  \\

                     \vernacular{uumu[lesheelɛ]
                    mú{\downstep}yáyi/musáatsa/muundu}  &   
                     \gloss{‘left’}  &  \\

                     \vernacular{uumu[loondeelɛ]
                    mú{\downstep}yáyi/musáatsa/muundu}  &   
                     \gloss{‘followed’}  &  \\

                     \vernacular{uumu[kulishiilɪ]
                    mú{\downstep}yáyi/musáatsa/muundu}  &   
                     \gloss{‘named’}  &  \\

                     \vernacular{uumu[lakhuuliilɪ]
                    mú{\downstep}yáyi/musáatsa/muundu}  &   
                     \gloss{‘released’}  &  \\

                     \vernacular{uumu[seebuliilɪ]
                    mú{\downstep}yáyi/musáatsa/muundu}  &   
                     \gloss{‘said bye
                    to’}  &  \\

                     \vernacular{
                    uumu[kalushitsiilɪ]
                    mú{\downstep}yáyi/musáatsa/muundu}  &   
                     \gloss{‘returned’}  &  \\

                     \vernacular{
                    uumu[reebɛreebeelɛ]
                    mú{\downstep}yáyi/musáatsa/muundu}  &   
                     \gloss{‘asked
                    (iter)’}  &  \\
\end{tabular}
%\caption{\nocaption}
     
\begin{tabular}{lll}  
  \multicolumn{2}{l}{
                     \vernacular{(539) /H/
                    C-Initial +OP + OP
                    } \gloss{‘you have...the
                    boy \ob mú{\downstep}yáyi\cb  /} } &  \\
\multicolumn{2}{l}{
                     \gloss{someone \ob muundu\cb 
                    for him/her for me’} } &  \\

                     \vernacular{uumuu[ndeeleelɛ]
                    mú{\downstep}yáyi/muundu}  &   
                     \gloss{‘buried’}  &  \\

                     \vernacular{uumuu[mbecheelɛ]
                    mú{\downstep}yáyi/muundu}  &   
                     \gloss{‘shaved’}  &  \\

                     \vernacular{uumuu[ndeereelɛ]
                    mú{\downstep}yáyi/muundu}  &   
                     \gloss{‘brought’}  &  \\

                     \vernacular{uumuu[khalachiilɪ]
                    mú{\downstep}yáyi/muundu}  &   
                     \gloss{‘cut’}  &  \\

                     \vernacular{uumuu[sitaachiilɪ]
                    mú{\downstep}yáyi/muundu}  &   
                     \gloss{‘accused’}  &  \\

                     \vernacular{
                    uumuu[mboolitsiilɪ] mú{\downstep}yáyi/muundu}  &   
                     \gloss{‘seduced’}  &  \\

                     \vernacular{
                    uumuu[mbohololeelɛ] mú{\downstep}yáyi/muundu}  &   
                     \gloss{‘untied’}  &  \\
\end{tabular}
%\caption{\nocaption}
     
\begin{tabular}{lll}  
  \multicolumn{2}{l}{
                     \vernacular{(540) /Ø/
                    C-Initial +OP + OP
                    } \gloss{‘you have...the
                    boy \ob mú{\downstep}yáyi\cb  /} } &  \\
\multicolumn{2}{l}{
                     \gloss{someone \ob muundu\cb 
                    for him/her for me’} } &  \\

                     \vernacular{uumuu[nziiliilɪ]
                    mú{\downstep}yáyi/muundu}  &   
                     \gloss{‘gone for’}  &  \\

                     \vernacular{uumuu[ndesheelɛ]
                    mú{\downstep}yáyi/muundu}  &   
                     \gloss{‘left’}  &  \\

                     \vernacular{uumuu[noondeelɛ]
                    mú{\downstep}yáyi/muundu}  &   
                     \gloss{‘followed’}  &  \\

                     \vernacular{uumuu[ngulishiilɪ]
                    mú{\downstep}yáyi/muundu}  &   
                     \gloss{‘named’}  &  \\

                     \vernacular{
                    uumuu[ndakhuuliilɪ] mú{\downstep}yáyi/muundu}  &   
                     \gloss{‘released’}  &  \\

                     \vernacular{uumuu[seebuliilɪ]
                    mú{\downstep}yáyi/muundu}  &   
                     \gloss{‘said bye
                    to’}  &  \\

                     \vernacular{
                    uumuu[siinjilitsiilɪ]
                    mú{\downstep}yáyi/muundu}  &   
                     \gloss{
                    ‘made...stand’}  &  \\
\end{tabular}
%\caption{\nocaption}
    

\subsection{Perfect Negative (2
              }\label{sec:sPerf2ndSgNeg}


\begin{tabular}{llllll}  
  \multicolumn{5}{l}{
                     \vernacular{(541) /H/
                    C-Initial} \gloss{‘you have
                    not...’} } &  \\
\multicolumn{5}{l}{ } &  \\

                     \vernacular{uu[réélé]
                    tá}  &   
                     \gloss{‘buried’}  &     &   
                     \vernacular{uu[ng’wéélé]
                    tá}  &   
                     \gloss{‘drunk’}  &  \\

                     \vernacular{uu[khwéélé]
                    tá}  &   
                     \gloss{‘eaten’}  &     &   
                     \vernacular{uu[líílí]
                    tá}  &   
                     \gloss{‘paid dowry’}  &  \\

                     \vernacular{uu[lúmí]
                    tá}  &   
                     \gloss{‘bitten’}  &     &   
                     \vernacular{uu[béchí]
                    tá}  &   
                     \gloss{‘shaved’}  &  \\

                     \vernacular{uu[tééshí]
                    tá}  &   
                     \gloss{‘cooked’}  &     &   
                     \vernacular{uu[léérí]
                    tá}  &   
                     \gloss{‘brought’}  &  \\

                     \vernacular{uu[khálááchɛ́]
                    tá}  &   
                     \gloss{‘cut’}  &     &   
                     \vernacular{uu[káláánjí]
                    tá}  &   
                     \gloss{‘fried’}  &  \\

                     \vernacular{uu[sítááchí]
                    tá}  &   
                     \gloss{‘accused’}  &     &   
                     \vernacular{uu[bóólíítsɪ́]
                    tá}  &   
                     \gloss{‘seduced’}  &  \\

                     \vernacular{
                    uu[sáándíítsɪ́] tá}  &   
                     \gloss{‘thanked’}  &     &   
                     \vernacular{
                    uu[khóng’óóndí] tá}  &   
                     \gloss{‘knocked’}  &  \\

                     \vernacular{uu[bóhólóólɛ́]
                    tá}  &   
                     \gloss{‘untied’}  &     &   
                     \vernacular{
                    uu[bóyóng’áánɛ́] tá}  &   
                     \gloss{‘gone
                    around’}  &  \\

                     \vernacular{
                    uu[ng’óng’óólíítsɪ́] tá}  &   
                     \gloss{‘teased’}  &     &   
                     \vernacular{
                    uu[líng(ák)ányíínyɪ́] tá}  &   
                     \gloss{‘crumpled’}  &  \\
\end{tabular}
%\caption{\nocaption}
     
\begin{tabular}{llllll}  
  \multicolumn{5}{l}{
                     \vernacular{(542) /Ø/
                    C-Initial} \gloss{‘you have
                    not...’} } &  \\
\multicolumn{5}{l}{ } &  \\

                     \vernacular{uu[tsíílí]
                    tá}  &   
                     \gloss{‘gone’}  &     &   
                     \vernacular{uu[kwíílí]
                    tá}  &   
                     \gloss{‘fallen’}  &  \\

                     \vernacular{uu[léshí]
                    tá}  &   
                     \gloss{‘left’}  &     &   
                     \vernacular{uu[réébí]
                    tá}  &   
                     \gloss{‘asked’}  &  \\

                     \vernacular{uu[lóóndí]
                    tá}  &   
                     \gloss{‘followed’}  &     &   
                     \vernacular{uu[kúmíílɪ́]
                    tá}  &   
                     \gloss{‘held’}  &  \\

                     \vernacular{uu[kúlííshɪ́]
                    tá}  &   
                     \gloss{‘named’}  &     &   
                     \vernacular{uu[hómóólí]
                    tá}  &   
                     \gloss{‘massaged’}  &  \\

                     \vernacular{uu[lákhúúlí]
                    tá}  &   
                     \gloss{‘released’}  &     &   
                     \vernacular{uu[séébúúlɪ́]
                    tá}  &   
                     \gloss{‘said bye’}  &  \\

                     \vernacular{
                    uu[hóómbélíítsɪ́] tá}  &   
                     \gloss{‘comforted’}  &     &   
                     \vernacular{
                    uu[kálúshíítsɪ́] tá}  &   
                     \gloss{‘returned’}  &  \\

                     \vernacular{
                    uu[síínjílíítsɪ́] tá}  &   
                     \gloss{‘made stand’}  &     &   
                     \vernacular{
                    uu[réébíréébí] tá}  &   
                     \gloss{‘asked
                    (iter)’}  &  \\

                     \vernacular{
                    uu[kálúkhányíínyɪ́] tá}  &   
                     \gloss{‘turn over’}  &     &   
                     \vernacular{
                    uu[sébúlúkhányíínyɪ́] tá}  &   
                     \gloss{‘scattered’}  &  \\
\end{tabular}
%\caption{\nocaption}
     
\begin{tabular}{llllll}  
  \multicolumn{5}{l}{
                     \vernacular{(543) /H/
                    C-Initial + OP} \gloss{‘you have
                    not...him/her’} } &  \\
\multicolumn{5}{l}{ } &  \\

                     \vernacular{uumu[réélé]
                    tá}  &   
                     \gloss{‘buried’}  &     &   
                     \vernacular{uumu[béchí]
                    tá}  &   
                     \gloss{‘shaved’}  &  \\

                     \vernacular{uumu[léérí]
                    tá}  &   
                     \gloss{‘brought’}  &     &   
                     \vernacular{
                    uumu[khálááchɛ́] tá}  &   
                     \gloss{‘cut’}  &  \\

                     \vernacular{uumu[sítááchí]
                    tá}  &   
                     \gloss{‘accused’}  &     &   
                     \vernacular{
                    uumu[bóólíítsɪ́] tá}  &   
                     \gloss{‘seduced’}  &  \\

                     \vernacular{
                    uumu[khóng’óóndí] tá}  &   
                     \gloss{‘knocked’}  &     &   
                     \vernacular{
                    uumu[bóhólóólɛ́] tá}  &   
                     \gloss{‘untied’}  &  \\

                     \vernacular{
                    uumu[bóyóng’áánɛ́] tá}  &   
                     \gloss{‘gone
                    around’}  &     &   
                     \vernacular{
                    uumu[ng’óng’óólíítsɪ́] tá}  &   
                     \gloss{‘teased’}  &  \\

                     \vernacular{
                    uumu[língákányíínyɪ́] tá}  &   
                     \gloss{‘bent’}  &     &     &     &  \\
\end{tabular}
%\caption{\nocaption}
     
\begin{tabular}{llllll}  
  \multicolumn{5}{l}{
                     \vernacular{(544) /Ø/
                    C-Initial + OP} \gloss{‘you have
                    not...him/her \ob mu-\cb  / them
                    } } &  \\
\multicolumn{5}{l}{ } &  \\

                     \vernacular{uumu[tsíílí]
                    tá}  &   
                     \gloss{‘gone for’}  &  \\

                     \vernacular{uumu[léshí]
                    tá}  &   
                     \gloss{‘left’}  &  \\

                     \vernacular{uumu[lóóndí]
                    tá}  &   
                     \gloss{‘followed’}  &  \\

                     \vernacular{uumu[kúlííshɪ́]
                    tá}  &   
                     \gloss{‘named’}  &  \\

                     \vernacular{uumu[lákhúúlí]
                    tá}  &   
                     \gloss{‘released’}  &  \\

                     \vernacular{
                    uumu[séébúúlɪ́] tá}  &   
                     \gloss{‘said bye
                    to’}  &  \\

                     \vernacular{
                    uumu[hóómbélíítsɪ́] tá}  &   
                     \gloss{‘comforted’}  &  \\

                     \vernacular{
                    uumu[kálúshíítsɪ́] tá}  &   
                     \gloss{‘returned’}  &  \\

                     \vernacular{
                    uumu[síínjílíítsɪ́] tá}  &   
                     \gloss{
                    ‘made...stand’}  &  \\

                     \vernacular{
                    uumu[réébíréébí] tá}  &   
                     \gloss{‘asked
                    (iter)’}  &  \\

                     \vernacular{
                    uumu[kálúkhányíínyɪ́] tá}  &   
                     \gloss{
                    ‘turned...over’}  &  \\

                     \vernacular{
                    uubi[sébúlúkhányíínyɪ́] tá}  &   
                     \gloss{‘scattered’}  &  \\
\end{tabular}
%\caption{\nocaption}
     
\begin{tabular}{llllll}  
  \multicolumn{5}{l}{
                     \vernacular{(545) /H/
                    C-Initial + OP + OP
                    } \gloss{‘you have
                    not...him/her for me’} } &  \\
\multicolumn{5}{l}{ } &  \\

                     \vernacular{
                    uumuu[ndééléélɛ́] tá}  &   
                     \gloss{‘buried’}  &     &   
                     \vernacular{
                    uumuu[mbéchéélɛ́] tá}  &   
                     \gloss{‘shaved’}  &  \\

                     \vernacular{
                    uumuu[ndééréélɛ́] tá}  &   
                     \gloss{‘brought’}  &     &   
                     \vernacular{
                    uumuu[kháláchíílɪ́] tá}  &   
                     \gloss{‘cut’}  &  \\

                     \vernacular{
                    uumuu[sítááchíílɪ́] tá}  &   
                     \gloss{‘accused’}  &     &   
                     \vernacular{
                    uumuu[mbóólítsíílɪ́] tá}  &   
                     \gloss{‘seduced’}  &  \\

                     \vernacular{
                    uumuu[mbóhólóléélɛ́] tá}  &   
                     \gloss{‘untied’}  &     &     &     &  \\
\end{tabular}
%\caption{\nocaption}
     
\begin{tabular}{llllll}  
  \multicolumn{5}{l}{
                     \vernacular{(546) /Ø/
                    C-Initial + OP + OP
                    } \gloss{‘you have
                    not...him/her for me’} } &  \\
\multicolumn{5}{l}{ } &  \\

                     \vernacular{
                    uumuu[nzíílíílɪ́] tá}  &   
                     \gloss{‘gone for’}  &     &   
                     \vernacular{
                    uumuu[ndéshéélɛ́] tá}  &   
                     \gloss{‘gone for’}  &  \\

                     \vernacular{
                    uumuu[nóóndéélɛ́] tá}  &   
                     \gloss{‘left’}  &     &   
                     \vernacular{
                    uumuu[ngúlíshíílɪ́] tá}  &   
                     \gloss{‘followed’}  &  \\

                     \vernacular{
                    uumuu[ndákhúúlíílɪ́] tá}  &   
                     \gloss{‘named’}  &     &   
                     \vernacular{
                    uumuu[séébúlíílɪ́] tá}  &   
                     \gloss{‘released’}  &  \\

                     \vernacular{
                    uumuu[mbóómbélítsíílɪ́] tá}  &   
                     \gloss{‘said bye
                    to’}  &     &   
                     \vernacular{
                    uumuu[síínjílítsíílɪ́] tá}  &   
                     \gloss{‘comforted’}  &  \\
\end{tabular}
%\caption{\nocaption}
     
\begin{tabular}{lll}  
  \multicolumn{2}{l}{
                     \vernacular{(547) /H/
                    C-Initial Phrase-Medial} \gloss{‘you have
                    not...the boy \ob mú{\downstep}yáyi\cb  /} } &  \\
\multicolumn{2}{l}{
                     \gloss{someone
                    \ob muundu\cb ’} } &  \\

                     \vernacular{uu[reele]
                    mú{\downstep}yáyi/muundu tá}  &   
                     \gloss{‘bury’}  &  \\

                     \vernacular{uu[bechi]
                    mú{\downstep}yáyi/muundu tá}  &   
                     \gloss{‘shave’}  &  \\

                     \vernacular{uu[leeri]
                    mú{\downstep}yáyi/muundu tá}  &   
                     \gloss{‘bring’}  &  \\

                     \vernacular{uu[khalaachɛ]
                    mú{\downstep}yáyi/muundu tá}  &   
                     \gloss{‘cut’}  &  \\

                     \vernacular{uu[sitaachi]
                    mú{\downstep}yáyi/muundu tá}  &   
                     \gloss{‘accuse’}  &  \\

                     \vernacular{uu[booliitsɪ]
                    mú{\downstep}yáyi/muundu tá}  &   
                     \gloss{‘seduce’}  &  \\

                     \vernacular{uu[khong’oondi]
                    mú{\downstep}yáyi/muundu tá}  &   
                     \gloss{‘knock’}  &  \\

                     \vernacular{uu[boholoolɛ]
                    mú{\downstep}yáyi/muundu tá}  &   
                     \gloss{‘untie’}  &  \\

                     \vernacular{uu[boyong’aanɛ]
                    mú{\downstep}yáyi/muundu tá}  &   
                     \gloss{‘go around’}  &  \\
\end{tabular}
%\caption{\nocaption}
     
\begin{tabular}{lll}  
  \multicolumn{2}{l}{
                     \vernacular{(548) /Ø/
                    C-Initial Phrase-Medial} \gloss{‘you have
                    not...the boy \ob mú{\downstep}yáyi\cb  /} } &  \\
\multicolumn{2}{l}{
                     \gloss{someone
                    \ob muundu\cb ’} } &  \\

                     \vernacular{uu[tsiili]
                    mú{\downstep}yáyi/muundu tá}  &   
                     \gloss{‘go for’}  &  \\

                     \vernacular{uu[leshi]
                    mú{\downstep}yáyi/muundu tá}  &   
                     \gloss{‘leave’}  &  \\

                     \vernacular{uu[loondi]
                    mú{\downstep}yáyi/muundu tá}  &   
                     \gloss{‘follow’}  &  \\

                     \vernacular{uu[kuliishɪ]
                    mú{\downstep}yáyi/muundu tá}  &   
                     \gloss{‘name’}  &  \\

                     \vernacular{uu[lakhuuli]
                    mú{\downstep}yáyi/muundu tá}  &   
                     \gloss{‘release’}  &  \\

                     \vernacular{uu[seebuulɪ]
                    mú{\downstep}yáyi/muundu tá}  &   
                     \gloss{‘say bye to’}  &  \\

                     \vernacular{uu[kalushiitsɪ]
                    mú{\downstep}yáyi/muundu tá}  &   
                     \gloss{‘return’}  &  \\

                     \vernacular{uu[reebireebi]
                    mú{\downstep}yáyi/muundu tá}  &   
                     \gloss{‘ask (iter)’}  &  \\
\end{tabular}
%\caption{\nocaption}
     
\begin{tabular}{lll}  
  \multicolumn{2}{l}{
                     \vernacular{(549) /H/
                    C-Initial +OP Phrase-Medial} \gloss{‘you have
                    not...the boy \ob mú{\downstep}yáyi\cb  /} } &  \\
\multicolumn{2}{l}{
                     \gloss{someone \ob muundu\cb 
                    for him/her’} } &  \\

                     \vernacular{uumu[reeleelɛ]
                    mú{\downstep}yáyi/muundu tá}  &   
                     \gloss{‘bury’}  &  \\

                     \vernacular{uumu[becheelɛ]
                    mú{\downstep}yáyi/muundu tá}  &   
                     \gloss{‘shave’}  &  \\

                     \vernacular{uumu[leereelɛ]
                    mú{\downstep}yáyi/muundu tá}  &   
                     \gloss{‘bring’}  &  \\

                     \vernacular{uumu[khalachiilɪ]
                    mú{\downstep}yáyi/muundu tá}  &   
                     \gloss{‘cut’}  &  \\

                     \vernacular{uumu[sitaachiilɪ]
                    mú{\downstep}yáyi/muundu tá}  &   
                     \gloss{‘accuse’}  &  \\

                     \vernacular{uumu[boolitsiilɪ]
                    mú{\downstep}yáyi/muundu tá}  &   
                     \gloss{‘seduce’}  &  \\

                     \vernacular{
                    uumu[khong’oondeelɛ] mú{\downstep}yáyi/muundu
                    tá}  &   
                     \gloss{‘knock’}  &  \\

                     \vernacular{uumu[bohololeelɛ]
                    mú{\downstep}yáyi/muundu tá}  &   
                     \gloss{‘untie’}  &  \\

                     \vernacular{
                    uumu[boyong’aniilɪ] mú{\downstep}yáyi/muundu
                    tá}  &   
                     \gloss{‘go around’}  &  \\
\end{tabular}
%\caption{\nocaption}
     
\begin{tabular}{lll}  
  \multicolumn{2}{l}{
                     \vernacular{(550) /Ø/
                    C-Initial +OP Phrase-Medial} \gloss{‘you have
                    not...the boy \ob mú{\downstep}yáyi\cb  /} } &  \\
\multicolumn{2}{l}{
                     \gloss{someone \ob muundu\cb 
                    for him/her’} } &  \\

                     \vernacular{uumu[tsiiliilɪ]
                    mú{\downstep}yáyi/muundu tá}  &   
                     \gloss{‘go for’}  &  \\

                     \vernacular{uumu[lesheelɛ]
                    mú{\downstep}yáyi/muundu tá}  &   
                     \gloss{‘leave’}  &  \\

                     \vernacular{uumu[loondeelɛ]
                    mú{\downstep}yáyi/muundu tá}  &   
                     \gloss{‘follow’}  &  \\

                     \vernacular{uumu[kulishiilɪ]
                    mú{\downstep}yáyi/muundu tá}  &   
                     \gloss{‘name’}  &  \\

                     \vernacular{uumu[lakhuuliilɪ]
                    mú{\downstep}yáyi/muundu tá}  &   
                     \gloss{‘release’}  &  \\

                     \vernacular{uumu[seebuliilɪ]
                    mú{\downstep}yáyi/muundu tá}  &   
                     \gloss{‘say bye to’}  &  \\

                     \vernacular{
                    uumu[kalushitsiilɪ] mú{\downstep}yáyi/muundu
                    tá}  &   
                     \gloss{‘return’}  &  \\

                     \vernacular{
                    uumu[reebɛreebeelɛ] mú{\downstep}yáyi/muundu
                    tá}  &   
                     \gloss{‘ask (iter)’}  &  \\
\end{tabular}
%\caption{\nocaption}
     
\begin{tabular}{lll}  
  \multicolumn{2}{l}{
                     \vernacular{(551) /H/
                    C-Initial +OP + OP
                    } \gloss{‘you have
                    not...the boy \ob mú{\downstep}yáyi\cb  /} } &  \\
\multicolumn{2}{l}{
                     \gloss{someone \ob muundu\cb 
                    for him/her for me’} } &  \\

                     \vernacular{uumuu[ndeeleelɛ]
                    mú{\downstep}yáyi/muundu tá}  &   
                     \gloss{‘bury’}  &  \\

                     \vernacular{uumuu[mbecheelɛ]
                    mú{\downstep}yáyi/muundu tá}  &   
                     \gloss{‘shave’}  &  \\

                     \vernacular{uumuu[ndeereelɛ]
                    mú{\downstep}yáyi/muundu tá}  &   
                     \gloss{‘bring’}  &  \\

                     \vernacular{uumuu[khalachiilɪ]
                    mú{\downstep}yáyi/muundu tá}  &   
                     \gloss{‘cut’}  &  \\

                     \vernacular{uumuu[sitaachiilɪ]
                    mú{\downstep}yáyi/muundu tá}  &   
                     \gloss{‘accuse’}  &  \\

                     \vernacular{
                    uumuu[mboolitsiilɪ] mú{\downstep}yáyi/muundu
                    tá}  &   
                     \gloss{‘seduce’}  &  \\

                     \vernacular{
                    uumuu[mbohololeelɛ] mú{\downstep}yáyi/muundu
                    tá}  &   
                     \gloss{‘untie’}  &  \\
\end{tabular}
%\caption{\nocaption}
     
\begin{tabular}{lll}  
  \multicolumn{2}{l}{
                     \vernacular{(552) /Ø/
                    C-Initial +OP + OP
                    } \gloss{‘you have
                    not...the boy \ob mú{\downstep}yáyi\cb  /} } &  \\
\multicolumn{2}{l}{
                     \gloss{someone \ob muundu\cb 
                    for him/her for me’} } &  \\

                     \vernacular{uumuu[nziiliilɪ]
                    mú{\downstep}yáyi/muundu tá}  &   
                     \gloss{‘go for’}  &  \\

                     \vernacular{uumuu[ndesheelɛ]
                    mú{\downstep}yáyi/muundu tá}  &   
                     \gloss{‘leave’}  &  \\

                     \vernacular{uumuu[noondeelɛ]
                    mú{\downstep}yáyi/muundu tá}  &   
                     \gloss{‘follow’}  &  \\

                     \vernacular{uumuu[ngulishiilɪ]
                    mú{\downstep}yáyi/muundu tá}  &   
                     \gloss{‘name’}  &  \\

                     \vernacular{
                    uumuu[ndakhuuliilɪ] mú{\downstep}yáyi/muundu
                    tá}  &   
                     \gloss{‘release’}  &  \\

                     \vernacular{uumuu[seebuliilɪ]
                    mú{\downstep}yáyi/muundu tá}  &   
                     \gloss{‘say bye to’}  &  \\

                     \vernacular{
                    uumuu[siinjilitsiilɪ] mú{\downstep}yáyi/muundu
                    tá}  &   
                     \gloss{
                    ‘make...stand’}  &  \\
\end{tabular}
%\caption{\nocaption}
    

\subsection{Hodiernal Perfective: Pattern 2a}\label{sec:sHodPerf}


\begin{tabular}{llllll}  
  \multicolumn{5}{l}{
                     \vernacular{(553) /H/
                    C-Initial} \gloss{
                    ‘s/he...’} } &  \\
\multicolumn{5}{l}{ } &  \\

                     \vernacular{
                    a[reele]}  &   
                     \gloss{‘buried’}  &     &   
                     \vernacular{
                    a[ng’weele]}  &   
                     \gloss{‘drank’}  &  \\

                     \vernacular{
                    a[khweele]}  &   
                     \gloss{‘ate’}  &     &   
                     \vernacular{
                    a[liili]}  &   
                     \gloss{‘paid dowry’}  &  \\

                     \vernacular{a[lumi]}  &   
                     \gloss{‘bit’}  &     &   
                     \vernacular{
                    a[bechi]}  &   
                     \gloss{‘shaved’}  &  \\

                     \vernacular{
                    a[teeshi]}  &   
                     \gloss{‘cooked’}  &     &   
                     \vernacular{
                    a[leeri]}  &   
                     \gloss{‘brought’}  &  \\

                     \vernacular{
                    a[khalaachɛ]}  &   
                     \gloss{‘cut’}  &     &   
                     \vernacular{
                    a[kalaanji]}  &   
                     \gloss{‘fried’}  &  \\

                     \vernacular{
                    a[sitaachi]}  &   
                     \gloss{‘accused’}  &     &   
                     \vernacular{
                    a[booliitsɪ]}  &   
                     \gloss{‘seduced’}  &  \\

                     \vernacular{
                    a[saandiitsɪ]}  &   
                     \gloss{‘thanked’}  &     &   
                     \vernacular{
                    a[khong’oondi]}  &   
                     \gloss{‘knocked’}  &  \\

                     \vernacular{
                    a[boholoolɛ]}  &   
                     \gloss{‘untied’}  &     &   
                     \vernacular{
                    a[boyong’aanɛ]}  &   
                     \gloss{‘went
                    around’}  &  \\

                     \vernacular{
                    a[ng’ong’ooliitsɪ]}  &   
                     \gloss{‘teased’}  &     &   
                     \vernacular{
                    a[ling(ak)anyiinyɪ]}  &   
                     \gloss{‘crumpled’}  &  \\
\end{tabular}
%\caption{\nocaption}
     
\begin{tabular}{llllll}  
  \multicolumn{5}{l}{
                     \vernacular{(554) /H/
                    V-Initial} \gloss{
                    ‘s/he...’} } &  \\
\multicolumn{5}{l}{ } &  \\

                     \vernacular{y[iiri]}  &   
                     \gloss{‘killed’}  &     &   
                     \vernacular{
                    y[iikoómbi]}  &   
                     \gloss{‘admired’}  &  \\

                     \vernacular{
                    y[iisiáchi]}  &   
                     \gloss{‘smacked’}  &     &   
                     \vernacular{
                    y[iikobóolɛ]}  &   
                     \gloss{‘belched’}  &  \\

                     \vernacular{
                    y[oononyiinyɪ]}  &   
                     \gloss{‘spoiled’}  &     &   
                     \vernacular{
                    y[aabukhanyiinyɪ]}  &   
                     \gloss{‘separated’}  &  \\
\end{tabular}
%\caption{\nocaption}
     
\begin{tabular}{llllll}  
  \multicolumn{5}{l}{
                     \vernacular{(555) /Ø/
                    C-Initial} \gloss{
                    ‘s/he...’} } &  \\
\multicolumn{5}{l}{ } &  \\

                     \vernacular{
                    a[tsiíli]}  &   
                     \gloss{‘went’}  &     &   
                     \vernacular{
                    a[kwiíli]}  &   
                     \gloss{‘fell’}  &  \\

                     \vernacular{
                    a[leshí]}  &   
                     \gloss{‘left’}  &     &   
                     \vernacular{
                    a[reébi]}  &   
                     \gloss{‘asked’}  &  \\

                     \vernacular{
                    a[loóndi]}  &   
                     \gloss{‘followed’}  &     &   
                     \vernacular{
                    a[kumíilɪ]}  &   
                     \gloss{‘held’}  &  \\

                     \vernacular{
                    a[kulíishɪ]}  &   
                     \gloss{‘named’}  &     &   
                     \vernacular{
                    a[homóoli]}  &   
                     \gloss{‘massaged’}  &  \\

                     \vernacular{
                    a[lakhúuli]}  &   
                     \gloss{‘released’}  &     &   
                     \vernacular{
                    a[seébúulɪ]}  &   
                     \gloss{‘said bye’}  &  \\

                     \vernacular{
                    a[hoómbéliitsɪ]}  &   
                     \gloss{‘comforted’}  &     &   
                     \vernacular{
                    a[kalúshíitsɪ]}  &   
                     \gloss{‘returned’}  &  \\

                     \vernacular{
                    a[siínjíliitsɪ]}  &   
                     \gloss{‘made stand’}  &     &   
                     \vernacular{
                    a[reébíreebi]}  &   
                     \gloss{‘asked
                    (iter)’}  &  \\

                     \vernacular{
                    a[kalúkhányiinyɪ]}  &   
                     \gloss{‘turned
                    over’}  &     &   
                     \vernacular{
                    a[sebúlúkhanyiinyɪ]}  &   
                     \gloss{‘scattered’}  &  \\
\end{tabular}
%\caption{\nocaption}
     
\begin{tabular}{llllll}  
  \multicolumn{5}{l}{
                     \vernacular{(556) /Ø/
                    V-Initial} \gloss{
                    ‘s/he...’} } &  \\
\multicolumn{5}{l}{ } &  \\

                     \vernacular{
                    y[eenyí]}  &   
                     \gloss{‘wanted’}  &     &   
                     \vernacular{
                    y[eeyéelɛ]}  &   
                     \gloss{‘wiped for’}  &  \\

                     \vernacular{
                    y[iilúuli]}  &   
                     \gloss{‘winnowed’}  &     &   
                     \vernacular{
                    y[aambákhaanɛ]}  &   
                     \gloss{‘refused’}  &  \\

                     \vernacular{
                    y[eeléeliitsɪ]}  &   
                     \gloss{‘hung up’}  &     &     &     &  \\
\end{tabular}
%\caption{\nocaption}
     
\begin{tabular}{llllll}  
  \multicolumn{5}{l}{
                     \vernacular{(557) /H/
                    C-Initial + OP} \gloss{
                    ‘s/he...him/her’} } &  \\
\multicolumn{5}{l}{ } &  \\

                     \vernacular{
                    amu[réele]}  &   
                     \gloss{‘buried’}  &     &   
                     \vernacular{
                    amu[béchi]}  &   
                     \gloss{‘shaved’}  &  \\

                     \vernacular{
                    amu[léeri]}  &   
                     \gloss{‘brought’}  &     &   
                     \vernacular{
                    amu[khálaachɛ]}  &   
                     \gloss{‘cut’}  &  \\

                     \vernacular{
                    amu[sítaachi]}  &   
                     \gloss{‘accused’}  &     &   
                     \vernacular{
                    amu[bóoliitsɪ]}  &   
                     \gloss{‘seduced’}  &  \\

                     \vernacular{
                    amu[khóng’oondi]}  &   
                     \gloss{‘knocked’}  &     &   
                     \vernacular{
                    amu[bóholoolɛ]}  &   
                     \gloss{‘untied’}  &  \\

                     \vernacular{
                    amu[bóyong’aanɛ]}  &   
                     \gloss{‘went
                    around’}  &     &   
                     \vernacular{
                    amu[ng’óng’ooliitsɪ]}  &   
                     \gloss{‘teased’}  &  \\

                     \vernacular{
                    amu[língakanyiinyɪ]}  &   
                     \gloss{‘bent’}  &     &     &     &  \\
\end{tabular}
%\caption{\nocaption}
     
\begin{tabular}{llllll}  
  \multicolumn{5}{l}{
                     \vernacular{(558) /H/
                    V-Initial + OP} \gloss{
                    ‘s/he...him/her’} } &  \\
\multicolumn{5}{l}{ } &  \\

                     \vernacular{
                    amw[iíri]}  &   
                     \gloss{‘killed’}  &     &   
                     \vernacular{
                    amw[ií{\downstep}kóómbi]}  &   
                     \gloss{‘admired’}  &  \\

                     \vernacular{
                    amw[ií{\downstep}síáchi]}  &   
                     \gloss{‘smacked’}  &     &   
                     \vernacular{
                    amw[oónonyiinyɪ]}  &   
                     \gloss{‘spoiled’}  &  \\

                     \vernacular{
                    amw[aábukhanyiinyɪ]}  &   
                     \gloss{‘separated’}  &  \\
\end{tabular}
%\caption{\nocaption}
     
\begin{tabular}{llllll}  
  \multicolumn{5}{l}{
                     \vernacular{(559) /Ø/
                    C-Initial + OP} \gloss{‘s/he...him/her
                    \ob mu-\cb  / them
                    } } &  \\
\multicolumn{5}{l}{ } &  \\

                     \vernacular{
                    amu[tsiíli]}  &   
                     \gloss{‘went for’}  &  \\

                     \vernacular{
                    amu[leshí]}  &   
                     \gloss{‘left’}  &  \\

                     \vernacular{
                    amu[loóndi]}  &   
                     \gloss{‘followed’}  &  \\

                     \vernacular{
                    amu[kulíishɪ]}  &   
                     \gloss{‘named’}  &  \\

                     \vernacular{
                    amu[lakhúuli]}  &   
                     \gloss{‘released’}  &  \\

                     \vernacular{
                    amu[seébúulɪ]}  &   
                     \gloss{‘said bye
                    to’}  &  \\

                     \vernacular{
                    amu[hoómbéliitsɪ]}  &   
                     \gloss{‘comforted’}  &  \\

                     \vernacular{
                    amu[kalúshíitsɪ]}  &   
                     \gloss{‘returned’}  &  \\

                     \vernacular{
                    amu[siínjíliitsɪ]}  &   
                     \gloss{
                    ‘made...stand’}  &  \\

                     \vernacular{
                    amu[reébíreebi]}  &   
                     \gloss{‘asked
                    (iter)’}  &  \\

                     \vernacular{
                    amu[kalúkhányiinyɪ]}  &   
                     \gloss{
                    ‘turned...over’}  &  \\

                     \vernacular{
                    abi[sebúlúkhanyiinyɪ]}  &   
                     \gloss{‘scattered’}  &  \\
\end{tabular}
%\caption{\nocaption}
     
\begin{tabular}{llllll}  
  \multicolumn{5}{l}{
                     \vernacular{(560) /Ø/
                    V-Initial + OP} \gloss{‘s/he...him/her
                    \ob mw-\cb  / it
                    } } &  \\
\multicolumn{5}{l}{ } &  \\

                     \vernacular{
                    amw[eenyí]}  &   
                     \gloss{‘wanted’}  &     &   
                     \vernacular{
                    amw[eeyéelɛ]}  &   
                     \gloss{‘wiped for’}  &  \\

                     \vernacular{
                    abw[iilúuli]}  &   
                     \gloss{‘winnowed’}  &     &   
                     \vernacular{
                    amw[aambákhaanɛ]}  &   
                     \gloss{‘refused’}  &  \\

                     \vernacular{
                    amw[eeléeliitsɪ]}  &   
                     \gloss{
                    ‘carried...hanging’}  &  \\
\end{tabular}
%\caption{\nocaption}
     
\begin{tabular}{llllll}  
  \multicolumn{5}{l}{
                     \vernacular{(561) /H/
                    C-Initial + OP
                    } \gloss{
                    ‘s/he...me’} } &  \\
\multicolumn{5}{l}{ } &  \\

                     \vernacular{
                    aa[ríili]}  &   
                     \gloss{‘feared’}  &     &   
                     \vernacular{
                    aa[mbéchi]}  &   
                     \gloss{‘shaved’}  &  \\

                     \vernacular{
                    aa[ndéeri]}  &   
                     \gloss{‘brought’}  &     &   
                     \vernacular{
                    aa[khálaachɛ]}  &   
                     \gloss{‘cut’}  &  \\

                     \vernacular{
                    aa[sítaachi]}  &   
                     \gloss{‘accused’}  &     &   
                     \vernacular{
                    aa[mbóoliitsɪ]}  &   
                     \gloss{‘seduced’}  &  \\

                     \vernacular{
                    aa[khóng’oondi]}  &   
                     \gloss{‘knocked’}  &     &   
                     \vernacular{
                    aa[mbóholoolɛ]}  &   
                     \gloss{‘untied’}  &  \\

                     \vernacular{
                    aa[mbóyong’aanɛ]}  &   
                     \gloss{‘went
                    around’}  &     &   
                     \vernacular{
                    aa[ng’óng’ooliitsɪ]}  &   
                     \gloss{‘teased’}  &  \\

                     \vernacular{
                    aa[níngakanyiinyɪ]}  &   
                     \gloss{‘bent’}  &  \\
\end{tabular}
%\caption{\nocaption}
     
\begin{tabular}{llllll}  
  \multicolumn{5}{l}{
                     \vernacular{(562) /H/
                    V-Initial + OP
                    } \gloss{
                    ‘s/he...me’} } &  \\
\multicolumn{5}{l}{ } &  \\

                     \vernacular{
                    aa[nzíri]}  &   
                     \gloss{‘killed’}  &     &   
                     \vernacular{
                    aa[nzí{\downstep}kóómbi]}  &   
                     \gloss{‘admired’}  &  \\

                     \vernacular{
                    aa[nzí{\downstep}síáchi]}  &   
                     \gloss{‘smacked’}  &     &   
                     \vernacular{
                    aa[nzónonyiinyɪ]}  &   
                     \gloss{‘spoiled’}  &  \\

                     \vernacular{
                    aa[nzábukhanyiinyɪ]}  &   
                     \gloss{‘separated’}  &  \\
\end{tabular}
%\caption{\nocaption}
     
\begin{tabular}{llllll}  
  \multicolumn{5}{l}{
                     \vernacular{(563) /Ø/
                    C-Initial + OP
                    } \gloss{
                    ‘s/he...me’} } &  \\
\multicolumn{5}{l}{ } &  \\

                     \vernacular{
                    aa[siéle]}  &   
                     \gloss{‘ground’}  &     &   
                     \vernacular{
                    aa[ndeshí]}  &   
                     \gloss{‘left’}  &  \\

                     \vernacular{
                    aa[noóndi]}  &   
                     \gloss{‘followed’}  &     &   
                     \vernacular{
                    aa[ngulíishɪ]}  &   
                     \gloss{‘named’}  &  \\

                     \vernacular{
                    aa[ndakhúuli]}  &   
                     \gloss{‘released’}  &     &   
                     \vernacular{
                    aa[seébúulɪ]}  &   
                     \gloss{‘said bye
                    to’}  &  \\

                     \vernacular{
                    aa[mboómbéliitsɪ]}  &   
                     \gloss{‘comforted’}  &     &   
                     \vernacular{
                    aa[siínjíliitsɪ]}  &   
                     \gloss{
                    ‘made..stand’}  &  \\

                     \vernacular{
                    aa[ndeébíndeebi]}  &   
                     \gloss{‘asked
                    (iter)’}  &     &   
                     \vernacular{
                    aa[ngalúkhányiinyɪ]}  &   
                     \gloss{
                    ‘turned...over’}  &  \\
\end{tabular}
%\caption{\nocaption}
     
\begin{tabular}{llllll}  
  \multicolumn{5}{l}{
                     \vernacular{(564) /Ø/
                    V-Initial + OP
                    } \gloss{
                    ‘s/he...me’} } &  \\
\multicolumn{5}{l}{ } &  \\

                     \vernacular{
                    aa[nzenyí]}  &   
                     \gloss{‘wanted’}  &     &   
                     \vernacular{
                    aa[nzeyéelɛ]}  &   
                     \gloss{‘wiped for’}  &  \\

                     \vernacular{
                    aa[nyambákhaanɛ]}  &   
                     \gloss{‘refused’}  &     &   
                     \vernacular{
                    aa[nzeléeliitsɪ]}  &   
                     \gloss{
                    ‘carried...hanging’}  &  \\
\end{tabular}
%\caption{\nocaption}
     
\begin{tabular}{llllll}  
  \multicolumn{5}{l}{
                     \vernacular{(565) /H/
                    C-Initial + OP
                    } \gloss{
                    ‘s/he...him/herself’} } &  \\
\multicolumn{5}{l}{ } &  \\

                     \vernacular{
                    yii[réele]}  &   
                     \gloss{‘buried’}  &     &   
                     \vernacular{
                    yii[béchi]}  &   
                     \gloss{‘shaved’}  &  \\

                     \vernacular{
                    yii[súunji]}  &   
                     \gloss{‘hung’}  &     &   
                     \vernacular{
                    yii[khálaachɛ]}  &   
                     \gloss{‘cut’}  &  \\

                     \vernacular{
                    yii[sítaachi]}  &   
                     \gloss{‘accused’}  &     &   
                     \vernacular{
                    yii[sáandiitsɪ]}  &   
                     \gloss{‘thanked’}  &  \\

                     \vernacular{
                    yii[khóng’oondi]}  &   
                     \gloss{‘knocked’}  &     &   
                     \vernacular{
                    yii[bóholoolɛ]}  &   
                     \gloss{‘untied’}  &  \\
\end{tabular}
%\caption{\nocaption}
     
\begin{tabular}{llllll}  
  \multicolumn{5}{l}{
                     \vernacular{(566) /H/
                    V-Initial + OP
                    } \gloss{
                    ‘s/he...him/herself’} } &  \\
\multicolumn{5}{l}{ } &  \\

                     \vernacular{
                    yii[yíri]}  &   
                     \gloss{‘killed’}  &     &   
                     \vernacular{
                    yii[yí{\downstep}kóómbi]}  &   
                     \gloss{‘admired’}  &  \\

                     \vernacular{
                    yii[yí{\downstep}síáchi]}  &   
                     \gloss{‘smacked’}  &     &   
                     \vernacular{
                    yii[yónonyiinyɪ]}  &   
                     \gloss{‘spoiled’}  &  \\

                     \vernacular{
                    yii[yábukhanyiinyɪ]}  &   
                     \gloss{‘separated’}  &  \\
\end{tabular}
%\caption{\nocaption}
     
\begin{tabular}{llllll}  
  \multicolumn{5}{l}{
                     \vernacular{(567) /Ø/
                    C-Initial + OP
                    } \gloss{
                    ‘s/he...him/herself’} } &  \\
\multicolumn{5}{l}{ } &  \\

                     \vernacular{
                    yii[siéle]}  &   
                     \gloss{‘ground’}  &     &   
                     \vernacular{
                    yii[leshí]}  &   
                     \gloss{‘left’}  &  \\

                     \vernacular{
                    yii[siínji]}  &   
                     \gloss{‘bathed’}  &     &   
                     \vernacular{
                    yii[kulíishɪ]}  &   
                     \gloss{‘named’}  &  \\

                     \vernacular{
                    yii[naábuulɪ]}  &   
                     \gloss{‘undressed’}  &     &   
                     \vernacular{
                    yii[lakhúuli]}  &   
                     \gloss{‘released’}  &  \\

                     \vernacular{
                    yii[hoómbéliitsɪ]}  &   
                     \gloss{‘comforted’}  &     &   
                     \vernacular{
                    yii[siínjíliitsɪ]}  &   
                     \gloss{
                    ‘made...stand’}  &  \\

                     \vernacular{
                    yii[reébíreebi]}  &   
                     \gloss{‘asked
                    (iter)’}  &     &   
                     \vernacular{
                    yii[kalúkhányiinyɪ]}  &   
                     \gloss{
                    ‘turned...over’}  &  \\
\end{tabular}
%\caption{\nocaption}
     
\begin{tabular}{llllll}  
  \multicolumn{5}{l}{
                     \vernacular{(568) /Ø/
                    V-Initial + OP
                    } \gloss{
                    ‘s/he...him/herself’} } &  \\
\multicolumn{5}{l}{ } &  \\

                     \vernacular{
                    yii[yalí]}  &   
                     \gloss{‘exposed’}  &     &   
                     \vernacular{
                    yii[yeyéelɛ]}  &   
                     \gloss{‘wiped for’}  &  \\

                     \vernacular{
                    yii[yambákhaanɛ]}  &   
                     \gloss{‘refused’}  &     &   
                     \vernacular{
                    yii[yeléeliitsɪ]}  &   
                     \gloss{‘hung...up’}  &  \\
\end{tabular}
%\caption{\nocaption}
     
\begin{tabular}{llllll}  
  \multicolumn{5}{l}{
                     \vernacular{(569) /H/
                    C-Initial + OP + OP
                    } \gloss{‘s/he...him/her
                    for me’} } &  \\
\multicolumn{5}{l}{ } &  \\

                     \vernacular{
                    amuú[ndeeleelɛ]}  &   
                     \gloss{‘buried’}  &     &   
                     \vernacular{
                    amuú[mbecheelɛ]}  &   
                     \gloss{‘shaved’}  &  \\

                     \vernacular{
                    amuú[ndeereelɛ]}  &   
                     \gloss{‘brought’}  &     &   
                     \vernacular{
                    amuú[khalachiilɪ]}  &   
                     \gloss{‘cut’}  &  \\

                     \vernacular{
                    amuú[sitaachiilɪ]}  &   
                     \gloss{‘accused’}  &     &   
                     \vernacular{
                    amuú[mboolitsiilɪ]}  &   
                     \gloss{‘seduced’}  &  \\

                     \vernacular{
                    amuú[mbohololeelɛ]}  &   
                     \gloss{‘untied’}  &     &     &     &  \\
\end{tabular}
%\caption{\nocaption}
     
\begin{tabular}{llllll}  
  \multicolumn{5}{l}{
                     \vernacular{(570) /H/
                    V-Initial + OP + OP
                    } \gloss{‘s/he...him/her
                    for me’} } &  \\
\multicolumn{5}{l}{ } &  \\

                     \vernacular{
                    amuú[nziiriilɪ]}  &   
                     \gloss{‘killed’}  &     &   
                     \vernacular{
                    amuú[nzechitsiilɪ]}  &   
                     \gloss{‘admired’}  &  \\

                     \vernacular{
                    amuú[{\downstep}nzísíáchiilɪ]}  &   
                     \gloss{‘smacked’}  &     &   
                     \vernacular{
                    amuú[nzononyinyiilɪ]}  &   
                     \gloss{‘spoiled’}  &  \\

                     \vernacular{
                    amuú[nzabukhanyinyiilɪ]}  &   
                     \gloss{‘separated’}  &     &     &     &  \\
\end{tabular}
%\caption{\nocaption}
     
\begin{tabular}{llllll}  
  \multicolumn{5}{l}{
                     \vernacular{(571) /Ø/
                    C-Initial + OP + OP
                    } \gloss{‘s/he...him/her
                    for me’} } &  \\
\multicolumn{5}{l}{ } &  \\

                     \vernacular{
                    amuú[{\downstep}nzííliilɪ]}  &   
                     \gloss{‘went for’}  &     &   
                     \vernacular{
                    amuú[{\downstep}ndéshéelɛ]}  &   
                     \gloss{‘went for’}  &  \\

                     \vernacular{
                    amuú[{\downstep}nóóndeelɛ]}  &   
                     \gloss{‘left’}  &     &   
                     \vernacular{
                    amuú[{\downstep}ngúlíshiilɪ]}  &   
                     \gloss{‘followed’}  &  \\

                     \vernacular{
                    amuú[{\downstep}ndákhúuliilɪ]}  &   
                     \gloss{‘named’}  &     &   
                     \vernacular{
                    amuú[{\downstep}séébúliilɪ]}  &   
                     \gloss{‘released’}  &  \\

                     \vernacular{
                    amuú[{\downstep}mbóómbélitsiilɪ]}  &   
                     \gloss{‘said bye
                    to’}  &     &   
                     \vernacular{
                    amuú[{\downstep}síínjílitsiilɪ]}  &   
                     \gloss{‘comforted’}  &  \\
\end{tabular}
%\caption{\nocaption}
     
\begin{tabular}{llllll}  
  \multicolumn{5}{l}{
                     \vernacular{(572) /Ø/
                    V-Initial + OP + OP
                    } \gloss{‘s/he...him/her
                    \ob mu-\cb  / it
                    } } &  \\
\multicolumn{5}{l}{ } &  \\

                     \vernacular{
                    amuú[{\downstep}nzéyéelɛ]}  &   
                     \gloss{‘wiped’}  &     &   
                     \vernacular{
                    akuú[{\downstep}nzáshítsiilɪ]}  &   
                     \gloss{‘lit’}  &  \\

                     \vernacular{
                    abuú[{\downstep}nzílúuliilɪ]}  &   
                     \gloss{‘winnowed’}  &     &   
                     \vernacular{
                    akuú[{\downstep}nzéléelitsiilɪ]}  &   
                     \gloss{‘hung’}  &  \\
\end{tabular}
%\caption{\nocaption}
     
\begin{tabular}{lll}  
  \multicolumn{2}{l}{
                     \vernacular{(573) /H/
                    C-Initial Phrase-Medial} \gloss{‘s/he...the boy
                    \ob mú{\downstep}yáyi\cb  /} } &  \\
\multicolumn{2}{l}{
                     \gloss{someone
                    \ob muundu\cb ’} } &  \\

                     \vernacular{a[reele]
                    mú{\downstep}yáyi/muundu}  &   
                     \gloss{‘buried’}  &  \\

                     \vernacular{a[bechi]
                    mú{\downstep}yáyi/muundu}  &   
                     \gloss{‘shaved’}  &  \\

                     \vernacular{a[leeri]
                    mú{\downstep}yáyi/muundu}  &   
                     \gloss{‘brought’}  &  \\

                     \vernacular{a[khalaachɛ]
                    mú{\downstep}yáyi/muundu}  &   
                     \gloss{‘cut’}  &  \\

                     \vernacular{a[sitaachi]
                    mú{\downstep}yáyi/muundu}  &   
                     \gloss{‘accused’}  &  \\

                     \vernacular{a[booliitsɪ]
                    mú{\downstep}yáyi/muundu}  &   
                     \gloss{‘seduced’}  &  \\

                     \vernacular{a[khong’oondi]
                    mú{\downstep}yáyi/muundu}  &   
                     \gloss{‘knocked’}  &  \\

                     \vernacular{a[boholoolɛ]
                    mú{\downstep}yáyi/muundu}  &   
                     \gloss{‘untied’}  &  \\

                     \vernacular{a[boyong’aanɛ]
                    mú{\downstep}yáyi/muundu}  &   
                     \gloss{‘went
                    around’}  &  \\
\end{tabular}
%\caption{\nocaption}
     
\begin{tabular}{lll}  
  \multicolumn{2}{l}{
                     \vernacular{(574) /Ø/
                    C-Initial Phrase-Medial} \gloss{‘s/he...the boy
                    \ob mú{\downstep}yáyi\cb  /} } &  \\
\multicolumn{2}{l}{
                     \gloss{someone
                    \ob muundu\cb ’} } &  \\

                     \vernacular{a[tsiíli]
                    mú{\downstep}yáyi/muundu}  &   
                     \gloss{‘went for’}  &  \\

                     \vernacular{a[leshí]
                    {\downstep}mú{\downstep}yáyi/muundu}  &   
                     \gloss{‘left’}  &  \\

                     \vernacular{a[loóndi]
                    mú{\downstep}yáyi/muundu}  &   
                     \gloss{‘followed’}  &  \\

                     \vernacular{a[kulíishɪ]
                    mú{\downstep}yáyi/muundu}  &   
                     \gloss{‘named’}  &  \\

                     \vernacular{a[lakhúuli]
                    mú{\downstep}yáyi/muundu}  &   
                     \gloss{‘released’}  &  \\

                     \vernacular{a[seébúulɪ]
                    mú{\downstep}yáyi/muundu}  &   
                     \gloss{‘said bye
                    to’}  &  \\

                     \vernacular{a[kalúshíitsɪ]
                    mú{\downstep}yáyi/muundu}  &   
                     \gloss{‘returned’}  &  \\

                     \vernacular{a[reébíreebi]
                    mú{\downstep}yáyi/muundu}  &   
                     \gloss{‘asked
                    (iter)’}  &  \\
\end{tabular}
%\caption{\nocaption}
     
\begin{tabular}{lll}  
  \multicolumn{2}{l}{
                     \vernacular{(575) /H/
                    C-Initial +OP Phrase-Medial} \gloss{‘s/he...the boy
                    \ob mú{\downstep}yáyi\cb  /} } &  \\
\multicolumn{2}{l}{
                     \gloss{someone \ob muundu\cb 
                    for him/her’} } &  \\

                     \vernacular{amu[réeleelɛ]
                    mú{\downstep}yáyi/muundu}  &   
                     \gloss{‘buried’}  &  \\

                     \vernacular{amu[bécheelɛ]
                    mú{\downstep}yáyi/muundu}  &   
                     \gloss{‘shaved’}  &  \\

                     \vernacular{amu[léereelɛ]
                    mú{\downstep}yáyi/muundu}  &   
                     \gloss{‘brought’}  &  \\

                     \vernacular{amu[khálachiilɪ]
                    mú{\downstep}yáyi/muundu}  &   
                     \gloss{‘cut’}  &  \\

                     \vernacular{amu[sítaachiilɪ]
                    mú{\downstep}yáyi/muundu}  &   
                     \gloss{‘accused’}  &  \\

                     \vernacular{amu[bóolitsiilɪ]
                    mú{\downstep}yáyi/muundu}  &   
                     \gloss{‘seduced’}  &  \\

                     \vernacular{
                    amu[khóng’oondeelɛ]
                    mú{\downstep}yáyi/muundu}  &   
                     \gloss{‘knocked’}  &  \\

                     \vernacular{amu[bóhololeelɛ]
                    mú{\downstep}yáyi/muundu}  &   
                     \gloss{‘untied’}  &  \\

                     \vernacular{
                    amu[bóyong’aniilɪ] mú{\downstep}yáyi/muundu}  &   
                     \gloss{‘went
                    around’}  &  \\
\end{tabular}
%\caption{\nocaption}
     
\begin{tabular}{lll}  
  \multicolumn{2}{l}{
                     \vernacular{(576) /Ø/
                    C-Initial +OP Phrase-Medial} \gloss{‘s/he...the boy
                    \ob mú{\downstep}yáyi\cb  /} } &  \\
\multicolumn{2}{l}{
                     \gloss{someone \ob muundu\cb 
                    for him/her’} } &  \\

                     \vernacular{amu[tsiílíilɪ]
                    mú{\downstep}yáyi/muundu}  &   
                     \gloss{‘went for’}  &  \\

                     \vernacular{amu[leshéelɛ]
                    mú{\downstep}yáyi/muundu}  &   
                     \gloss{‘left’}  &  \\

                     \vernacular{amu[loóndéelɛ]
                    mú{\downstep}yáyi/muundu}  &   
                     \gloss{‘followed’}  &  \\

                     \vernacular{amu[kulíshíilɪ]
                    mú{\downstep}yáyi/muundu}  &   
                     \gloss{‘named’}  &  \\

                     \vernacular{amu[lakhúuliilɪ]
                    mú{\downstep}yáyi/muundu}  &   
                     \gloss{‘released’}  &  \\

                     \vernacular{amu[seébúliilɪ]
                    mú{\downstep}yáyi/muundu}  &   
                     \gloss{‘said bye
                    to’}  &  \\

                     \vernacular{
                    amu[kalúshítsiilɪ]
                    mú{\downstep}yáyi/muundu}  &   
                     \gloss{‘returned’}  &  \\

                     \vernacular{
                    amu[reébɛ́reebeelɛ]
                    mú{\downstep}yáyi/muundu}  &   
                     \gloss{‘asked
                    (iter)’}  &  \\
\end{tabular}
%\caption{\nocaption}
     
\begin{tabular}{lll}  
  \multicolumn{2}{l}{
                     \vernacular{(577) /H/
                    C-Initial +OP + OP
                    } \gloss{‘s/he...the boy
                    \ob mú{\downstep}yáyi\cb  /} } &  \\
\multicolumn{2}{l}{
                     \gloss{someone \ob muundu\cb 
                    for him/her for me’} } &  \\

                     \vernacular{amuú[ndeeleelɛ]
                    mú{\downstep}yáyi/muundu}  &   
                     \gloss{‘buried’}  &  \\

                     \vernacular{amuú[mbecheelɛ]
                    mú{\downstep}yáyi/muundu}  &   
                     \gloss{‘shaved’}  &  \\

                     \vernacular{amuú[ndeereelɛ]
                    mú{\downstep}yáyi/muundu}  &   
                     \gloss{‘brought’}  &  \\

                     \vernacular{amuú[khalachiilɪ]
                    mú{\downstep}yáyi/muundu}  &   
                     \gloss{‘cut’}  &  \\

                     \vernacular{amuú[sitaachiilɪ]
                    mú{\downstep}yáyi/muundu}  &   
                     \gloss{‘accused’}  &  \\

                     \vernacular{
                    amuú[mboolitsiilɪ] mú{\downstep}yáyi/muundu}  &   
                     \gloss{‘seduced’}  &  \\

                     \vernacular{
                    amuú[mbohololeelɛ] mú{\downstep}yáyi/muundu}  &   
                     \gloss{‘untied’}  &  \\
\end{tabular}
%\caption{\nocaption}
     
\begin{tabular}{lll}  
  \multicolumn{2}{l}{
                     \vernacular{(578) /Ø/
                    C-Initial +OP + OP
                    } \gloss{‘s/he...the boy
                    \ob mú{\downstep}yáyi\cb  /} } &  \\
\multicolumn{2}{l}{
                     \gloss{someone \ob muundu\cb 
                    for him/her for me’} } &  \\

                     \vernacular{
                    amuú[{\downstep}nzíílíilɪ]
                    mú{\downstep}yáyi/muundu}  &   
                     \gloss{‘went for’}  &  \\

                     \vernacular{
                    amuú[{\downstep}ndéshéelɛ] mú{\downstep}yáyi/muundu}  &   
                     \gloss{‘left’}  &  \\

                     \vernacular{
                    amuú[{\downstep}nóóndéelɛ]
                    mú{\downstep}yáyi/muundu}  &   
                     \gloss{‘followed’}  &  \\

                     \vernacular{
                    amuú[{\downstep}ngúlíshíilɪ]
                    mú{\downstep}yáyi/muundu}  &   
                     \gloss{‘named’}  &  \\

                     \vernacular{
                    amuú[{\downstep}ndákhúuliilɪ]
                    mú{\downstep}yáyi/muundu}  &   
                     \gloss{‘released’}  &  \\

                     \vernacular{
                    amuú[{\downstep}séébúliilɪ]
                    mú{\downstep}yáyi/muundu}  &   
                     \gloss{‘said bye
                    to’}  &  \\

                     \vernacular{
                    amuú[{\downstep}síínjílitsiilɪ]
                    mú{\downstep}yáyi/muundu}  &   
                     \gloss{
                    ‘made...stand’}  &  \\
\end{tabular}
%\caption{\nocaption}
    

\subsection{Hodiernal Perfective Negative: Pattern
              2a}\label{sec:sHodPerfNeg}


\begin{tabular}{llllll}  
  \multicolumn{5}{l}{
                     \vernacular{(579) /H/
                    C-Initial} \gloss{‘s/he did
                    not...’} } &  \\
\multicolumn{5}{l}{ } &  \\

                     \vernacular{a[reele]
                    tá}  &   
                     \gloss{‘bury’}  &     &   
                     \vernacular{a[ng’weele]
                    tá}  &   
                     \gloss{‘drink’}  &  \\

                     \vernacular{a[khweele]
                    tá}  &   
                     \gloss{‘eat’}  &     &   
                     \vernacular{a[liili]
                    tá}  &   
                     \gloss{‘pay dowry’}  &  \\

                     \vernacular{a[lumi]
                    tá}  &   
                     \gloss{‘bite’}  &     &   
                     \vernacular{a[bechi]
                    tá}  &   
                     \gloss{‘shave’}  &  \\

                     \vernacular{a[teeshi]
                    tá}  &   
                     \gloss{‘cook’}  &     &   
                     \vernacular{a[leeri]
                    tá}  &   
                     \gloss{‘bring’}  &  \\

                     \vernacular{a[khalaachɛ]
                    tá}  &   
                     \gloss{‘cut’}  &     &   
                     \vernacular{a[kalaanji]
                    tá}  &   
                     \gloss{‘fry’}  &  \\

                     \vernacular{a[sitaachi]
                    tá}  &   
                     \gloss{‘accuse’}  &     &   
                     \vernacular{a[booliitsɪ]
                    tá}  &   
                     \gloss{‘seduce’}  &  \\

                     \vernacular{a[saandiitsɪ]
                    tá}  &   
                     \gloss{‘thank’}  &     &   
                     \vernacular{a[khong’oondi]
                    tá}  &   
                     \gloss{‘knock’}  &  \\

                     \vernacular{a[boholoolɛ]
                    tá}  &   
                     \gloss{‘untie’}  &     &   
                     \vernacular{a[boyong’aanɛ]
                    tá}  &   
                     \gloss{‘go around’}  &  \\

                     \vernacular{a[ng’ong’ooliitsɪ]
                    tá}  &   
                     \gloss{‘tease’}  &     &   
                     \vernacular{
                    a[ling(ak)anyiinyɪ] tá}  &   
                     \gloss{‘crumple’}  &  \\
\end{tabular}
%\caption{\nocaption}
     
\begin{tabular}{llllll}  
  \multicolumn{5}{l}{
                     \vernacular{(580) /H/
                    V-Initial} \gloss{‘s/he did
                    not...’} } &  \\
\multicolumn{5}{l}{ } &  \\

                     \vernacular{y[iiri]
                    tá}  &   
                     \gloss{‘kill’}  &     &   
                     \vernacular{y[iikoó{\downstep}mbí]
                    tá}  &   
                     \gloss{‘admire’}  &  \\

                     \vernacular{y[iisiá{\downstep}chí]
                    tá}  &   
                     \gloss{‘smack’}  &     &   
                     \vernacular{y[iikobó{\downstep}ólɛ́]
                    tá}  &   
                     \gloss{‘belch’}  &  \\

                     \vernacular{
                    y[oó{\downstep}nónyíínyɪ́] tá}  &   
                     \gloss{‘spoil’)}  &     &   
                     \vernacular{
                    y[aá{\downstep}búkhányíínyɪ́] tá}  &   
                     \gloss{‘separate’}  &  \\
\end{tabular}
%\caption{\nocaption}
     
\begin{tabular}{llllll}  
  \multicolumn{5}{l}{
                     \vernacular{(581) /Ø/
                    C-Initial} \gloss{‘s/he did
                    not...’} } &  \\
\multicolumn{5}{l}{ } &  \\

                     \vernacular{a[tsiíli]
                    tá}  &   
                     \gloss{‘go’}  &     &   
                     \vernacular{a[kwiíli]
                    tá}  &   
                     \gloss{‘fall’}  &  \\

                     \vernacular{a[leshí]
                    tá}  &   
                     \gloss{‘leave’}  &     &   
                     \vernacular{a[reébi]
                    tá}  &   
                     \gloss{‘ask’}  &  \\

                     \vernacular{a[loóndi]
                    tá}  &   
                     \gloss{‘follow’}  &     &   
                     \vernacular{a[kumíilɪ]
                    tá}  &   
                     \gloss{‘hold’}  &  \\

                     \vernacular{a[kulíishɪ]
                    tá}  &   
                     \gloss{‘name’}  &     &   
                     \vernacular{a[homóoli]
                    tá}  &   
                     \gloss{‘massage’}  &  \\

                     \vernacular{a[lakhúuli]
                    tá}  &   
                     \gloss{‘release’}  &     &   
                     \vernacular{a[seébúulɪ]
                    tá}  &   
                     \gloss{‘say bye’}  &  \\

                     \vernacular{a[hoómbéliitsɪ]
                    tá}  &   
                     \gloss{‘comfort’}  &     &   
                     \vernacular{a[kalúshíitsɪ]
                    tá}  &   
                     \gloss{‘return’}  &  \\

                     \vernacular{a[siínjíliitsɪ]
                    tá}  &   
                     \gloss{‘make stand’}  &     &   
                     \vernacular{a[reébíreebi]
                    tá}  &   
                     \gloss{‘ask (iter)’}  &  \\

                     \vernacular{
                    a[kalúkhányiinyɪ] tá}  &   
                     \gloss{‘turn over’}  &     &   
                     \vernacular{
                    a[sebúlúkhanyiinyɪ] tá}  &   
                     \gloss{‘scatter’}  &  \\
\end{tabular}
%\caption{\nocaption}
     
\begin{tabular}{llllll}  
  \multicolumn{5}{l}{
                     \vernacular{(582) /Ø/
                    V-Initial} \gloss{‘s/he did
                    not...’} } &  \\
\multicolumn{5}{l}{ } &  \\

                     \vernacular{y[eenyí]
                    {\downstep}tá}  &   
                     \gloss{‘want’}  &     &   
                     \vernacular{y[eeyéelɛ]
                    tá}  &   
                     \gloss{‘wipe for’}  &  \\

                     \vernacular{y[iilúuli]
                    tá}  &   
                     \gloss{‘winnow’}  &     &   
                     \vernacular{y[aambákhaanɛ]
                    tá}  &   
                     \gloss{‘refuse’}  &  \\

                     \vernacular{y[eeléeliitsɪ]
                    tá}  &   
                     \gloss{‘hang up’}  &     &     &     &  \\
\end{tabular}
%\caption{\nocaption}
     
\begin{tabular}{llllll}  
  \multicolumn{5}{l}{
                     \vernacular{(583) /H/
                    C-Initial + OP} \gloss{‘s/he did
                    not...him/her’} } &  \\
\multicolumn{5}{l}{ } &  \\

                     \vernacular{amu[réele]
                    tá}  &   
                     \gloss{‘bury’}  &     &   
                     \vernacular{amu[béchi]
                    tá}  &   
                     \gloss{‘shave’}  &  \\

                     \vernacular{amu[léeri]
                    tá}  &   
                     \gloss{‘bring’}  &     &   
                     \vernacular{amu[khálaachɛ]
                    tá}  &   
                     \gloss{‘cut’}  &  \\

                     \vernacular{amu[sítaachi]
                    tá}  &   
                     \gloss{‘accuse’}  &     &   
                     \vernacular{amu[bóoliitsɪ]
                    tá}  &   
                     \gloss{‘seduce’}  &  \\

                     \vernacular{amu[khóng’oondi]
                    tá}  &   
                     \gloss{‘knock’}  &     &   
                     \vernacular{amu[bóholoolɛ]
                    tá}  &   
                     \gloss{‘untie’}  &  \\

                     \vernacular{amu[bóyong’aanɛ]
                    tá}  &   
                     \gloss{‘go around’}  &     &   
                     \vernacular{
                    amu[ng’óng’ooliitsɪ] tá}  &   
                     \gloss{‘tease’}  &  \\

                     \vernacular{
                    amu[língakanyiinyɪ] tá}  &   
                     \gloss{‘bend’}  &     &     &     &  \\
\end{tabular}
%\caption{\nocaption}
     
\begin{tabular}{llllll}  
  \multicolumn{5}{l}{
                     \vernacular{(584) /Ø/
                    C-Initial + OP} \gloss{‘s/he did
                    not...him/her \ob mu-\cb  / them
                    } } &  \\
\multicolumn{5}{l}{ } &  \\

                     \vernacular{amu[tsiíli]
                    tá}  &   
                     \gloss{‘go for’}  &  \\

                     \vernacular{amu[leshí]
                    {\downstep}tá}  &   
                     \gloss{‘leave’}  &  \\

                     \vernacular{amu[loóndi]
                    tá}  &   
                     \gloss{‘follow’}  &  \\

                     \vernacular{amu[kulíishɪ]
                    tá}  &   
                     \gloss{‘name’}  &  \\

                     \vernacular{amu[lakhúuli]
                    tá}  &   
                     \gloss{‘release’}  &  \\

                     \vernacular{amu[seébúulɪ]
                    tá}  &   
                     \gloss{‘say bye to’}  &  \\

                     \vernacular{
                    amu[hoómbéliitsɪ] tá}  &   
                     \gloss{‘comfort’}  &  \\

                     \vernacular{amu[kalúshíitsɪ]
                    tá}  &   
                     \gloss{‘return’}  &  \\

                     \vernacular{
                    amu[siínjíliitsɪ] tá}  &   
                     \gloss{
                    ‘make...stand’}  &  \\

                     \vernacular{amu[reébíreebi]
                    tá}  &   
                     \gloss{‘ask (iter)’}  &  \\

                     \vernacular{
                    amu[kalúkhányiinyɪ] tá}  &   
                     \gloss{
                    ‘turn...over’}  &  \\

                     \vernacular{
                    abi[sebúlúkhanyiinyɪ] tá}  &   
                     \gloss{‘scatter’}  &  \\
\end{tabular}
%\caption{\nocaption}
     
\begin{tabular}{llllll}  
  \multicolumn{5}{l}{
                     \vernacular{(585) /H/
                    C-Initial + OP + OP
                    } \gloss{‘s/he did
                    not...him/her for me’} } &  \\
\multicolumn{5}{l}{ } &  \\

                     \vernacular{amuú[ndeeleelɛ]
                    tá}  &   
                     \gloss{‘bury’}  &     &   
                     \vernacular{amuú[mbecheelɛ]
                    tá}  &   
                     \gloss{‘shave’}  &  \\

                     \vernacular{amuú[ndeereelɛ]
                    tá}  &   
                     \gloss{‘bring’}  &     &   
                     \vernacular{amuú[khalachiilɪ]
                    tá}  &   
                     \gloss{‘cut’}  &  \\

                     \vernacular{amuú[sitaachiilɪ]
                    tá}  &   
                     \gloss{‘accuse’}  &     &   
                     \vernacular{
                    amuú[mboolitsiilɪ] tá}  &   
                     \gloss{‘seduce’}  &  \\

                     \vernacular{
                    amuú[mbohololeelɛ] tá}  &   
                     \gloss{‘untie’}  &     &     &     &  \\
\end{tabular}
%\caption{\nocaption}
     
\begin{tabular}{llllll}  
  \multicolumn{5}{l}{
                     \vernacular{(586) /Ø/
                    C-Initial + OP + OP
                    } \gloss{‘s/he did
                    not...him/her for me’} } &  \\
\multicolumn{5}{l}{ } &  \\

                     \vernacular{
                    amuú[{\downstep}nzíílíilɪ] tá}  &   
                     \gloss{‘go for’}  &     &   
                     \vernacular{
                    amuú[{\downstep}ndéshéelɛ] tá}  &   
                     \gloss{‘go for’}  &  \\

                     \vernacular{
                    amuú[{\downstep}nóóndéelɛ] tá}  &   
                     \gloss{‘leave’}  &     &   
                     \vernacular{
                    amuú[{\downstep}ngúlíshíilɪ] tá}  &   
                     \gloss{‘follow’}  &  \\

                     \vernacular{
                    amuú[{\downstep}ndákhúuliilɪ] tá}  &   
                     \gloss{‘name’}  &     &   
                     \vernacular{
                    amuú[{\downstep}séébúliilɪ] tá}  &   
                     \gloss{‘release’}  &  \\

                     \vernacular{
                    amuú[{\downstep}mbóómbélitsiilɪ] tá}  &   
                     \gloss{‘say bye to’}  &     &   
                     \vernacular{
                    amuú[{\downstep}síínjílitsiilɪ] tá}  &   
                     \gloss{‘comfort’}  &  \\
\end{tabular}
%\caption{\nocaption}
     
\begin{tabular}{lll}  
  \multicolumn{2}{l}{
                     \vernacular{(587) /H/
                    C-Initial Phrase-Medial} \gloss{‘s/he did
                    not...the boy \ob mú{\downstep}yáyi\cb  /} } &  \\
\multicolumn{2}{l}{
                     \gloss{someone \ob muundu\cb ’
                    } } &  \\

                     \vernacular{a[reele]
                    mú{\downstep}yáyi/muundu tá}  &   
                     \gloss{‘bury’}  &  \\

                     \vernacular{a[bechi]
                    mú{\downstep}yáyi/muundu tá}  &   
                     \gloss{‘shave’}  &  \\

                     \vernacular{a[leeri]
                    mú{\downstep}yáyi/muundu tá}  &   
                     \gloss{‘bring’}  &  \\

                     \vernacular{a[khalaachɛ]
                    mú{\downstep}yáyi/muundu tá}  &   
                     \gloss{‘cut’}  &  \\

                     \vernacular{a[sitaachi]
                    mú{\downstep}yáyi/muundu tá}  &   
                     \gloss{‘accuse’}  &  \\

                     \vernacular{a[booliitsɪ]
                    mú{\downstep}yáyi/muundu tá}  &   
                     \gloss{‘seduce’}  &  \\

                     \vernacular{a[khong’oondi]
                    mú{\downstep}yáyi/muundu tá}  &   
                     \gloss{‘knock’}  &  \\

                     \vernacular{a[boholoolɛ]
                    mú{\downstep}yáyi/muundu tá}  &   
                     \gloss{‘untie’}  &  \\

                     \vernacular{a[boyong’aanɛ]
                    mú{\downstep}yáyi/muundu tá}  &   
                     \gloss{‘go around’}  &  \\
\end{tabular}
%\caption{\nocaption}
     
\begin{tabular}{lll}  
  \multicolumn{2}{l}{
                     \vernacular{(588) /Ø/
                    C-Initial Phrase-Medial} \gloss{‘s/he did
                    not...the boy \ob mú{\downstep}yáyi\cb  /} } &  \\
\multicolumn{2}{l}{
                     \gloss{someone
                    \ob muundu\cb ’} } &  \\

                     \vernacular{a[tsiíli]
                    mú{\downstep}yáyi/muundu tá}  &   
                     \gloss{‘go for’}  &  \\

                     \vernacular{a[leshí]
                    {\downstep}mú{\downstep}yáyi/muundu tá}  &   
                     \gloss{‘leave’}  &  \\

                     \vernacular{a[loóndi]
                    mú{\downstep}yáyi/muundu tá}  &   
                     \gloss{‘follow’}  &  \\

                     \vernacular{a[kulíishɪ]
                    mú{\downstep}yáyi/muundu tá}  &   
                     \gloss{‘name’}  &  \\

                     \vernacular{a[lakhúuli]
                    mú{\downstep}yáyi/muundu tá}  &   
                     \gloss{‘release’}  &  \\

                     \vernacular{a[seébúulɪ]
                    mú{\downstep}yáyi/muundu tá}  &   
                     \gloss{‘say bye to’}  &  \\

                     \vernacular{a[kalúshíitsɪ]
                    mú{\downstep}yáyi/muundu tá}  &   
                     \gloss{‘return’}  &  \\

                     \vernacular{a[reébíreebi]
                    mú{\downstep}yáyi/muundu tá}  &   
                     \gloss{‘ask (iter)’}  &  \\
\end{tabular}
%\caption{\nocaption}
     
\begin{tabular}{lll}  
  \multicolumn{2}{l}{
                     \vernacular{(589) /H/
                    C-Initial +OP Phrase-Medial} \gloss{‘s/he did
                    not...the boy \ob mú{\downstep}yáyi\cb  /} } &  \\
\multicolumn{2}{l}{
                     \gloss{someone \ob muundu\cb 
                    for him/her’} } &  \\

                     \vernacular{amu[réeleelɛ]
                    mú{\downstep}yáyi/muundu tá}  &   
                     \gloss{‘bury’}  &  \\

                     \vernacular{amu[bécheelɛ]
                    mú{\downstep}yáyi/muundu tá}  &   
                     \gloss{‘shave’}  &  \\

                     \vernacular{amu[léereelɛ]
                    mú{\downstep}yáyi/muundu tá}  &   
                     \gloss{‘bring’}  &  \\

                     \vernacular{amu[khálachiilɪ]
                    mú{\downstep}yáyi/muundu tá}  &   
                     \gloss{‘cut’}  &  \\

                     \vernacular{amu[sítaachiilɪ]
                    mú{\downstep}yáyi/muundu tá}  &   
                     \gloss{‘accuse’}  &  \\

                     \vernacular{amu[bóolitsiilɪ]
                    mú{\downstep}yáyi/muundu tá}  &   
                     \gloss{‘seduce’}  &  \\

                     \vernacular{
                    amu[khóng’oondeelɛ] mú{\downstep}yáyi/muundu
                    tá}  &   
                     \gloss{‘knock’}  &  \\

                     \vernacular{amu[bóhololeelɛ]
                    mú{\downstep}yáyi/muundu tá}  &   
                     \gloss{‘untie’}  &  \\

                     \vernacular{
                    amu[bóyong’aniilɪ] mú{\downstep}yáyi/muundu
                    tá}  &   
                     \gloss{‘go around’}  &  \\
\end{tabular}
%\caption{\nocaption}
     
\begin{tabular}{lll}  
  \multicolumn{2}{l}{
                     \vernacular{(590) /Ø/
                    C-Initial +OP Phrase-Medial} \gloss{‘s/he did
                    not...the boy \ob mú{\downstep}yáyi\cb  /} } &  \\
\multicolumn{2}{l}{
                     \gloss{someone \ob muundu\cb 
                    for him/her’} } &  \\

                     \vernacular{amu[tsiílíilɪ]
                    {\downstep}mú{\downstep}yáyi/muundu tá}  &   
                     \gloss{‘go for’}  &  \\

                     \vernacular{amu[leshéelɛ]
                    {\downstep}mú{\downstep}yáyi/muundu tá}  &   
                     \gloss{‘leave’}  &  \\

                     \vernacular{amu[loóndéelɛ]
                    mú{\downstep}yáyi/muundu tá}  &   
                     \gloss{‘follow’}  &  \\

                     \vernacular{amu[kulíshíilɪ]
                    mú{\downstep}yáyi/muundu tá}  &   
                     \gloss{‘name’}  &  \\

                     \vernacular{amu[lakhúuliilɪ]
                    mú{\downstep}yáyi/muundu tá}  &   
                     \gloss{‘release’}  &  \\

                     \vernacular{amu[seébúliilɪ]
                    mú{\downstep}yáyi/muundu tá}  &   
                     \gloss{‘say bye to’}  &  \\

                     \vernacular{
                    amu[kalúshítsiilɪ] mú{\downstep}yáyi/muundu
                    tá}  &   
                     \gloss{‘return’}  &  \\

                     \vernacular{
                    amu[reébɛ́reebeelɛ] mú{\downstep}yáyi/muundu
                    tá}  &   
                     \gloss{‘ask (iter)’}  &  \\
\end{tabular}
%\caption{\nocaption}
     
\begin{tabular}{lll}  
  \multicolumn{2}{l}{
                     \vernacular{(591) /H/
                    C-Initial +OP + OP
                    } \gloss{‘s/he did
                    not...the boy \ob mú{\downstep}yáyi\cb  /} } &  \\
\multicolumn{2}{l}{
                     \gloss{someone \ob muundu\cb 
                    for him/her for me’} } &  \\

                     \vernacular{amuú[ndeeleelɛ]
                    mú{\downstep}yáyi/muundu tá}  &   
                     \gloss{‘bury’}  &  \\

                     \vernacular{amuú[mbecheelɛ]
                    mú{\downstep}yáyi/muundu tá}  &   
                     \gloss{‘shave’}  &  \\

                     \vernacular{amuú[ndeereelɛ]
                    mú{\downstep}yáyi/muundu tá}  &   
                     \gloss{‘bring’}  &  \\

                     \vernacular{amuú[khalachiilɪ]
                    mú{\downstep}yáyi/muundu tá}  &   
                     \gloss{‘cut’}  &  \\

                     \vernacular{amuú[sitaachiilɪ]
                    mú{\downstep}yáyi/muundu tá}  &   
                     \gloss{‘accuse’}  &  \\

                     \vernacular{
                    amuú[mboolitsiilɪ] mú{\downstep}yáyi/muundu
                    tá}  &   
                     \gloss{‘seduce’}  &  \\

                     \vernacular{
                    amuú[mbohololeelɛ] mú{\downstep}yáyi/muundu
                    tá}  &   
                     \gloss{‘untie’}  &  \\
\end{tabular}
%\caption{\nocaption}
     
\begin{tabular}{lll}  
  \multicolumn{2}{l}{
                     \vernacular{(592) /Ø/
                    C-Initial +OP + OP
                    } \gloss{‘s/he did
                    not...the boy \ob mú{\downstep}yáyi\cb  /} } &  \\
\multicolumn{2}{l}{
                     \gloss{someone \ob muundu\cb 
                    for him/her for me’} } &  \\

                     \vernacular{
                    amuú[{\downstep}nzíílíilɪ] mú{\downstep}yáyi/muundu
                    tá}  &   
                     \gloss{‘go for’}  &  \\

                     \vernacular{
                    amuú[{\downstep}ndéshéelɛ] mú{\downstep}yáyi/muundu
                    tá}  &   
                     \gloss{‘leave’}  &  \\

                     \vernacular{
                    amuú[{\downstep}nóóndéelɛ] mú{\downstep}yáyi/muundu
                    tá}  &   
                     \gloss{‘follow’}  &  \\

                     \vernacular{
                    amuú[{\downstep}ngúlíshíilɪ] mú{\downstep}yáyi/muundu
                    tá}  &   
                     \gloss{‘name’}  &  \\

                     \vernacular{
                    amuú[{\downstep}ndákhúuliilɪ] mú{\downstep}yáyi/muundu
                    tá}  &   
                     \gloss{‘release’}  &  \\

                     \vernacular{
                    amuú[{\downstep}séébúliilɪ] mú{\downstep}yáyi/muundu
                    tá}  &   
                     \gloss{‘say bye to’}  &  \\

                     \vernacular{
                    amuú[{\downstep}síínjílitsiilɪ] mú{\downstep}yáyi/muundu
                    tá}  &   
                     \gloss{
                    ‘make...stand’}  &  \\
\end{tabular}
%\caption{\nocaption}
    

\subsection{Conditional: Pattern 5c}\label{sec:sCond}


\begin{tabular}{llllll}  
  \multicolumn{5}{l}{
                     \vernacular{(593) /H/
                    C-Initial} \gloss{‘if
                    s/he...’} } &  \\
\multicolumn{5}{l}{ } &  \\

                     \vernacular{
                    naá[ra]}  &   
                     \gloss{‘buries’}  &     &   
                     \vernacular{
                    naá[ng’wa]}  &   
                     \gloss{‘drinks’}  &  \\

                     \vernacular{
                    naá[khwa]}  &   
                     \gloss{‘eats’}  &     &   
                     \vernacular{
                    naá[lia]}  &   
                     \gloss{‘pays dowry’}  &  \\

                     \vernacular{
                    naá[luma]}  &   
                     \gloss{‘bites’}  &     &   
                     \vernacular{
                    naá[beka]}  &   
                     \gloss{‘shaves’}  &  \\

                     \vernacular{
                    naá[{\downstep}téékhá]}  &   
                     \gloss{‘cooks’}  &     &   
                     \vernacular{
                    naá[{\downstep}léérá]}  &   
                     \gloss{‘brings’}  &  \\

                     \vernacular{
                    naá[{\downstep}kháláká]}  &   
                     \gloss{‘cuts’}  &     &   
                     \vernacular{
                    naá[{\downstep}káláángá]}  &   
                     \gloss{‘fries’}  &  \\

                     \vernacular{
                    naá[{\downstep}sítááká]}  &   
                     \gloss{‘accuses’}  &     &   
                     \vernacular{
                    naá[{\downstep}bóólítsá]}  &   
                     \gloss{‘seduces’}  &  \\

                     \vernacular{
                    naá[{\downstep}sáándítsá]}  &   
                     \gloss{‘thanks’}  &     &   
                     \vernacular{
                    naá[{\downstep}khóng’óóndá]}  &   
                     \gloss{‘knocks’}  &  \\

                     \vernacular{
                    naá[{\downstep}bóhólólá]}  &   
                     \gloss{‘unties’}  &     &   
                     \vernacular{
                    naá[{\downstep}bóyóng’áná]}  &   
                     \gloss{‘goes
                    around’}  &  \\

                     \vernacular{
                    naá[{\downstep}ng’óng’óólítsá]}  &   
                     \gloss{‘teases’}  &     &   
                     \vernacular{
                    naá[{\downstep}língákányínyá]}  &   
                     \gloss{‘crumples’}  &  \\
  &     &     &     &     &  \\
\multicolumn{2}{l}{
                     \vernacular{
                    naá[{\downstep}khóng’óóndá], khuliikula} } &     &   \multicolumn{2}{l}{
                     \gloss{‘if s/he knocks, we
                    will open (the door)’} } &  \\
\multicolumn{2}{l}{
                     \vernacular{khuliikula
                    naá[{\downstep}khóng’óóndá]} } &     &   \multicolumn{2}{l}{
                     \gloss{‘we will open (the
                    door) if s/he knocks’} } &  \\
\end{tabular}
%\caption{\nocaption}
     
\begin{tabular}{llllll}  
  \multicolumn{5}{l}{
                     \vernacular{(594) /H/
                    V-Initial} \gloss{‘if
                    s/he...’} } &  \\
\multicolumn{5}{l}{ } &  \\

                     \vernacular{
                    niy[í{\downstep}irá]}  &   
                     \gloss{‘kills’}  &     &   
                     \vernacular{
                    niy[í{\downstep}íkóómba]}  &   
                     \gloss{‘admires’}  &  \\

                     \vernacular{
                    niy[í{\downstep}isiáka]}  &   
                     \gloss{‘smacks’}  &     &   
                     \vernacular{
                    niy[í{\downstep}íkóbóla]}  &   
                     \gloss{‘belches’}  &  \\

                     \vernacular{
                    niy[ó{\downstep}ónónyínyá]}  &   
                     \gloss{‘spoils’}  &     &   
                     \vernacular{
                    niy[á{\downstep}ábúkhányínyá]}  &   
                     \gloss{‘separates’}  &  \\
\end{tabular}
%\caption{\nocaption}
     
\begin{tabular}{llllll}  
  \multicolumn{5}{l}{
                     \vernacular{(595) /Ø/
                    C-Initial} \gloss{‘if
                    s/he...’} } &  \\
\multicolumn{5}{l}{ } &  \\

                     \vernacular{
                    naá[{\downstep}tsía]}  &   
                     \gloss{‘goes’}  &     &   
                     \vernacular{
                    naá[{\downstep}kwá]}  &   
                     \gloss{‘falls’}  &  \\

                     \vernacular{
                    naá[{\downstep}lékhá]}  &   
                     \gloss{‘leaves’}  &     &   
                     \vernacular{
                    naá[{\downstep}rééba]}  &   
                     \gloss{‘asks’}  &  \\

                     \vernacular{
                    naá[{\downstep}lóónda]}  &   
                     \gloss{‘follows’}  &     &   
                     \vernacular{
                    naá[{\downstep}kúmíla]}  &   
                     \gloss{‘holds’}  &  \\

                     \vernacular{
                    naá[{\downstep}kúlíkha]}  &   
                     \gloss{‘names’}  &     &   
                     \vernacular{
                    naá[{\downstep}hómóola]}  &   
                     \gloss{‘massages’}  &  \\

                     \vernacular{
                    naá[{\downstep}lákhúula]}  &   
                     \gloss{‘releases’}  &     &   
                     \vernacular{
                    naá[{\downstep}séébula]}  &   
                     \gloss{‘says bye’}  &  \\

                     \vernacular{
                    naá[{\downstep}hóómbélitsa]}  &   
                     \gloss{‘comforts’}  &     &   
                     \vernacular{
                    naá[{\downstep}kálúshitsa]}  &   
                     \gloss{‘returns’}  &  \\

                     \vernacular{
                    naá[{\downstep}síínjílitsa]}  &   
                     \gloss{‘makes
                    stand’}  &     &   
                     \vernacular{
                    naá[{\downstep}réébáreeba]}  &   
                     \gloss{‘asks
                    (iter)’}  &  \\

                     \vernacular{
                    naá[{\downstep}kálúkhányinya]}  &   
                     \gloss{‘turns over’}  &     &   
                     \vernacular{
                    naá[{\downstep}sébúlúkhanyinya]}  &   
                     \gloss{‘scatters’}  &  \\
\end{tabular}
%\caption{\nocaption}
     
\begin{tabular}{llllll}  
  \multicolumn{5}{l}{
                     \vernacular{(596) /Ø/
                    V-Initial} \gloss{‘if
                    s/he...’} } &  \\
\multicolumn{5}{l}{ } &  \\

                     \vernacular{
                    niy[é{\downstep}ényá]}  &   
                     \gloss{‘wants’}  &     &   
                     \vernacular{
                    niy[é{\downstep}éyéla]}  &   
                     \gloss{‘wipes for’}  &  \\

                     \vernacular{
                    niy[í{\downstep}ílúúla]}  &   
                     \gloss{‘winnows’}  &     &   
                     \vernacular{
                    niy[á{\downstep}ámbákhana]}  &   
                     \gloss{‘refuses’}  &  \\

                     \vernacular{
                    niy[é{\downstep}éléélitsa]}  &   
                     \gloss{‘hangs up’}  &     &   
                     \vernacular{
                    niy[í{\downstep}íkóómba]}  &   
                     \gloss{‘admires’}  &  \\
\end{tabular}
%\caption{\nocaption}
     
\begin{tabular}{llllll}  
  \multicolumn{5}{l}{
                     \vernacular{(597) /H/
                    C-Initial + OP} \gloss{‘if
                    s/he...him/her’} } &  \\
\multicolumn{5}{l}{ } &  \\

                     \vernacular{
                    naá{\downstep}mú[rá]}  &   
                     \gloss{‘buries’}  &     &   
                     \vernacular{
                    naá{\downstep}mú[bé{\downstep}ká]}  &   
                     \gloss{‘shaves’}  &  \\

                     \vernacular{
                    naá{\downstep}mú[lé{\downstep}érá]}  &   
                     \gloss{‘brings’}  &     &   
                     \vernacular{
                    naá{\downstep}mú[khá{\downstep}láká]}  &   
                     \gloss{‘cuts’}  &  \\

                     \vernacular{
                    naá{\downstep}mú[sí{\downstep}tááká]}  &   
                     \gloss{‘accuses’}  &     &   
                     \vernacular{
                    naá{\downstep}mú[bó{\downstep}ólítsá]}  &   
                     \gloss{‘seduces’}  &  \\

                     \vernacular{
                    naá{\downstep}mú[khó{\downstep}ng’óóndá]}  &   
                     \gloss{‘knocks’}  &     &   
                     \vernacular{
                    naá{\downstep}mú[bó{\downstep}hólólá]}  &   
                     \gloss{‘unties’}  &  \\

                     \vernacular{
                    naá{\downstep}mú[bó{\downstep}yóng’áná]}  &   
                     \gloss{‘goes
                    around’}  &     &   
                     \vernacular{
                    naá{\downstep}mú[ng’ó{\downstep}ng’óólítsá]}  &   
                     \gloss{‘teases’}  &  \\

                     \vernacular{
                    naá{\downstep}mú[lí{\downstep}ngákányínyá]}  &   
                     \gloss{‘bends’}  &     &     &     &  \\
\end{tabular}
%\caption{\nocaption}
     
\begin{tabular}{llllll}  
  \multicolumn{5}{l}{
                     \vernacular{(598) /H/
                    V-Initial + OP} \gloss{‘if
                    s/he...him/her’} } &  \\
\multicolumn{5}{l}{ } &  \\

                     \vernacular{
                    naá{\downstep}mw[íí{\downstep}rá]}  &   
                     \gloss{‘kills’}  &     &   
                     \vernacular{
                    naá{\downstep}mw[íí{\downstep}kóómbá]}  &   
                     \gloss{‘admires’}  &  \\

                     \vernacular{
                    naá{\downstep}mw[íí{\downstep}síáká]}  &   
                     \gloss{‘smacks’}  &     &   
                     \vernacular{
                    naá{\downstep}mw[óó{\downstep}nónyínyá]}  &   
                     \gloss{‘spoils’}  &  \\

                     \vernacular{
                    naá{\downstep}mw[áá{\downstep}búkhányínyá]}  &   
                     \gloss{‘separates’}  &  \\
\end{tabular}
%\caption{\nocaption}
     
\begin{tabular}{llllll}  
  \multicolumn{5}{l}{
                     \vernacular{(599) /Ø/
                    C-Initial + OP} \gloss{‘if
                    s/he...him/her \ob mu-\cb ’} } &  \\
\multicolumn{5}{l}{ } &  \\

                     \vernacular{
                    naá{\downstep}mú[tsía]}  &   
                     \gloss{‘goes for’}  &  \\

                     \vernacular{
                    naá{\downstep}mú[lékhá]}  &   
                     \gloss{‘leaves’}  &  \\

                     \vernacular{
                    naá{\downstep}mú[lóónda]}  &   
                     \gloss{‘follows’}  &  \\

                     \vernacular{
                    naá{\downstep}mú[kúlíkha]}  &   
                     \gloss{‘names’}  &  \\

                     \vernacular{
                    naá{\downstep}mú[lákhúula]}  &   
                     \gloss{‘releases’}  &  \\

                     \vernacular{
                    naá{\downstep}mú[séébula]}  &   
                     \gloss{‘says bye
                    to’}  &  \\

                     \vernacular{
                    naá{\downstep}mú[hóómbélitsa]}  &   
                     \gloss{‘comforts’}  &  \\

                     \vernacular{
                    naá{\downstep}mú[kálúshitsa]}  &   
                     \gloss{‘returns’}  &  \\

                     \vernacular{
                    naá{\downstep}mú[síínjílitsa]}  &   
                     \gloss{
                    ‘makes...stand’}  &  \\

                     \vernacular{
                    naá{\downstep}mú[réébáreeba]}  &   
                     \gloss{‘asks
                    (iter)’}  &  \\

                     \vernacular{
                    naá{\downstep}mú[kálúkhányinya]}  &   
                     \gloss{
                    ‘turns...over’}  &  \\

                     \vernacular{
                    naá{\downstep}mú[sébúlúkhanyinya]}  &   
                     \gloss{‘scatters’}  &  \\
\end{tabular}
%\caption{\nocaption}
     
\begin{tabular}{llllll}  
  \multicolumn{5}{l}{
                     \vernacular{(600) /Ø/
                    V-Initial + OP} \gloss{‘if
                    s/he...him/her \ob mw-\cb  / it
                    } } &  \\
\multicolumn{5}{l}{ } &  \\

                     \vernacular{
                    naá{\downstep}mw[éényá]}  &   
                     \gloss{‘wants’}  &     &   
                     \vernacular{
                    naá{\downstep}mw[ééyéla]}  &   
                     \gloss{‘wipes for’}  &  \\

                     \vernacular{
                    naá{\downstep}bw[íílúula]}  &   
                     \gloss{‘winnows’}  &     &   
                     \vernacular{
                    naá{\downstep}mw[áámbákhana]}  &   
                     \gloss{‘refuses’}  &  \\

                     \vernacular{
                    naá{\downstep}mw[ééléelitsa]}  &   
                     \gloss{
                    ‘carries...hanging’}  &  \\
\end{tabular}
%\caption{\nocaption}
     
\begin{tabular}{llllll}  
  \multicolumn{5}{l}{
                     \vernacular{(601) /H/
                    C-Initial + OP
                    } \gloss{‘if
                    s/he...me’} } &  \\
\multicolumn{5}{l}{ } &  \\

                     \vernacular{
                    naá[{\downstep}rí{\downstep}á]}  &   
                     \gloss{‘fears’}  &     &   
                     \vernacular{
                    naá[{\downstep}mbé{\downstep}ká]}  &   
                     \gloss{‘shaves’}  &  \\

                     \vernacular{
                    naá[{\downstep}ndé{\downstep}érá]}  &   
                     \gloss{‘brings’}  &     &   
                     \vernacular{
                    naá[{\downstep}khá{\downstep}láká]}  &   
                     \gloss{‘cuts’}  &  \\

                     \vernacular{
                    naá[{\downstep}sí{\downstep}tááká]}  &   
                     \gloss{‘accuses’}  &     &   
                     \vernacular{
                    naá[{\downstep}mbó{\downstep}ólítsá]}  &   
                     \gloss{‘seduces’}  &  \\

                     \vernacular{
                    naá[{\downstep}khó{\downstep}ng’óóndá]}  &   
                     \gloss{‘knocks’}  &     &   
                     \vernacular{
                    naá[{\downstep}mbó{\downstep}hólólá]}  &   
                     \gloss{‘unties’}  &  \\

                     \vernacular{
                    naá[{\downstep}mbó{\downstep}yóng’áná]}  &   
                     \gloss{‘goes
                    around’}  &     &   
                     \vernacular{
                    naá[{\downstep}ng’ó{\downstep}ng’óólítsá]}  &   
                     \gloss{‘teases’}  &  \\

                     \vernacular{
                    naá[{\downstep}ní{\downstep}ngákányínyá]}  &   
                     \gloss{‘bends’}  &  \\
\end{tabular}
%\caption{\nocaption}
     
\begin{tabular}{llllll}  
  \multicolumn{5}{l}{
                     \vernacular{(602) /H/
                    V-Initial + OP
                    } \gloss{‘if
                    s/he...me’} } &  \\
\multicolumn{5}{l}{ } &  \\

                     \vernacular{
                    naá[{\downstep}nzí{\downstep}rá]}  &   
                     \gloss{‘kills’}  &     &   
                     \vernacular{
                    naá[{\downstep}nzí{\downstep}kóómbá]}  &   
                     \gloss{‘admires’}  &  \\

                     \vernacular{
                    naá[{\downstep}nzí{\downstep}síáká]}  &   
                     \gloss{‘smacks’}  &     &   
                     \vernacular{
                    naá[{\downstep}nzó{\downstep}nónyínyá]}  &   
                     \gloss{‘spoils’}  &  \\

                     \vernacular{
                    naá[{\downstep}nzá{\downstep}búkhányínyá]}  &   
                     \gloss{‘separates’}  &  \\
\end{tabular}
%\caption{\nocaption}
     
\begin{tabular}{llllll}  
  \multicolumn{5}{l}{
                     \vernacular{(603) /Ø/
                    C-Initial + OP
                    } \gloss{‘if
                    s/he...me’} } &  \\
\multicolumn{5}{l}{ } &  \\

                     \vernacular{
                    naá[{\downstep}sía]}  &   
                     \gloss{‘grinds’}  &     &   
                     \vernacular{
                    naá[{\downstep}ndékhá]}  &   
                     \gloss{‘leaves’}  &  \\

                     \vernacular{
                    naá[{\downstep}nóónda]}  &   
                     \gloss{‘follows’}  &     &   
                     \vernacular{
                    naá[{\downstep}ngúlíkha]}  &   
                     \gloss{‘names’}  &  \\

                     \vernacular{
                    naá[{\downstep}ndákhúula]}  &   
                     \gloss{‘releases’}  &     &   
                     \vernacular{
                    naá[{\downstep}séébula]}  &   
                     \gloss{‘says bye
                    to’}  &  \\

                     \vernacular{
                    naá[{\downstep}mbóómbélitsa]}  &   
                     \gloss{‘comforts’}  &     &   
                     \vernacular{
                    naá[{\downstep}síínjílitsa]}  &   
                     \gloss{
                    ‘makes..stand’}  &  \\

                     \vernacular{
                    naá[{\downstep}ndéébándeeba]}  &   
                     \gloss{‘asks
                    (iter)’}  &     &   
                     \vernacular{
                    naá[{\downstep}ngálúkhányinya]}  &   
                     \gloss{
                    ‘turns...over’}  &  \\
\end{tabular}
%\caption{\nocaption}
     
\begin{tabular}{llllll}  
  \multicolumn{5}{l}{
                     \vernacular{(604) /Ø/
                    V-Initial + OP
                    } \gloss{‘if
                    s/he...me’} } &  \\
\multicolumn{5}{l}{ } &  \\

                     \vernacular{
                    naá[{\downstep}nzényá]}  &   
                     \gloss{‘wants’}  &     &   
                     \vernacular{
                    naá[{\downstep}nzéyéla]}  &   
                     \gloss{‘wipes for’}  &  \\

                     \vernacular{
                    naá[{\downstep}nyámbákhana]}  &   
                     \gloss{‘refuses’}  &     &   
                     \vernacular{
                    naá[{\downstep}nzéléelitsa]}  &   
                     \gloss{
                    ‘carries...hanging’}  &  \\
\end{tabular}
%\caption{\nocaption}
     
\begin{tabular}{llllll}  
  \multicolumn{5}{l}{
                     \vernacular{(605) /H/
                    C-Initial + OP
                    } \gloss{‘if
                    s/he...him/herself’} } &  \\
\multicolumn{5}{l}{ } &  \\

                     \vernacular{
                    niyí{\downstep}í[rá]}  &   
                     \gloss{‘buries’}  &     &   
                     \vernacular{
                    niyí{\downstep}í[bé{\downstep}ká]}  &   
                     \gloss{‘shaves’}  &  \\

                     \vernacular{
                    niyí{\downstep}í[sú{\downstep}úngá]}  &   
                     \gloss{‘hangs’}  &     &   
                     \vernacular{
                    niyí{\downstep}í[khá{\downstep}láká]}  &   
                     \gloss{‘cuts’}  &  \\

                     \vernacular{
                    niyí{\downstep}í[sí{\downstep}tááká]}  &   
                     \gloss{‘accuses’}  &     &   
                     \vernacular{
                    niyí{\downstep}í[sá{\downstep}ándítsá]}  &   
                     \gloss{‘thanks’}  &  \\

                     \vernacular{
                    niyí{\downstep}í[khó{\downstep}ng’óóndá]}  &   
                     \gloss{‘knocks’}  &     &   
                     \vernacular{
                    niyí{\downstep}í[bó{\downstep}hólólá]}  &   
                     \gloss{‘unties’}  &  \\
\end{tabular}
%\caption{\nocaption}
     
\begin{tabular}{llllll}  
  \multicolumn{5}{l}{
                     \vernacular{(606) /H/
                    V-Initial + OP
                    } \gloss{‘if
                    s/he...him/herself’} } &  \\
\multicolumn{5}{l}{ } &  \\

                     \vernacular{
                    niyí{\downstep}í[yí{\downstep}rá]}  &   
                     \gloss{‘kills’}  &     &   
                     \vernacular{
                    niyí{\downstep}í[yí{\downstep}kóómbá]}  &   
                     \gloss{‘admires’}  &  \\

                     \vernacular{
                    niyí{\downstep}í[yí{\downstep}síáká]}  &   
                     \gloss{‘smacks’}  &     &   
                     \vernacular{
                    niyí{\downstep}í[yó{\downstep}nónyínyá]}  &   
                     \gloss{‘spoils’}  &  \\

                     \vernacular{
                    niyí{\downstep}í[yá{\downstep}búkhányínyá]}  &   
                     \gloss{‘separates’}  &  \\
\end{tabular}
%\caption{\nocaption}
     
\begin{tabular}{llllll}  
  \multicolumn{5}{l}{
                     \vernacular{(607) /Ø/
                    C-Initial + OP
                    } \gloss{‘if
                    s/he...him/herself’} } &  \\
\multicolumn{5}{l}{ } &  \\

                     \vernacular{
                    niyí{\downstep}í[sía]}  &   
                     \gloss{‘grinds’}  &     &   
                     \vernacular{
                    niyí{\downstep}í[lékhá]}  &   
                     \gloss{‘leaves’}  &  \\

                     \vernacular{
                    niyí{\downstep}í[síínga]}  &   
                     \gloss{‘bathes’}  &     &   
                     \vernacular{
                    niyí{\downstep}í[kúlíkha]}  &   
                     \gloss{‘names’}  &  \\

                     \vernacular{
                    niyí{\downstep}í[náábula]}  &   
                     \gloss{‘undresses’}  &     &   
                     \vernacular{
                    niyí{\downstep}í[lákhúula]}  &   
                     \gloss{‘releases’}  &  \\

                     \vernacular{
                    niyí{\downstep}í[hóómbélitsa]}  &   
                     \gloss{‘comforts’}  &     &   
                     \vernacular{
                    niyí{\downstep}í[síínjílitsa]}  &   
                     \gloss{
                    ‘makes...stand’}  &  \\

                     \vernacular{
                    niyí{\downstep}í[réébáreeba]}  &   
                     \gloss{‘asks
                    (iter)’}  &     &   
                     \vernacular{
                    niyí{\downstep}í[kálúkhányinya]}  &   
                     \gloss{
                    ‘turns...over’}  &  \\
\end{tabular}
%\caption{\nocaption}
     
\begin{tabular}{llllll}  
  \multicolumn{5}{l}{
                     \vernacular{(608) /Ø/
                    V-Initial + OP
                    } \gloss{‘if
                    s/he...him/herself’} } &  \\
\multicolumn{5}{l}{ } &  \\

                     \vernacular{
                    niyí{\downstep}í[yálá]}  &   
                     \gloss{‘exposes’}  &     &   
                     \vernacular{
                    niyí{\downstep}í[yéyéla]}  &   
                     \gloss{‘wipes for’}  &  \\

                     \vernacular{
                    niyí{\downstep}í[yámbákhana]}  &   
                     \gloss{‘refuses’}  &     &   
                     \vernacular{
                    niyí{\downstep}í[yéléelitsa]}  &   
                     \gloss{‘hangs...up’}  &  \\
\end{tabular}
%\caption{\nocaption}
     
\begin{tabular}{llllll}  
  \multicolumn{5}{l}{
                     \vernacular{(609) /H/
                    C-Initial + OP + OP
                    } \gloss{‘if
                    s/he...him/her for me’} } &  \\
\multicolumn{5}{l}{ } &  \\

                     \vernacular{
                    naá{\downstep}múú[{\downstep}ndé{\downstep}élá]}  &   
                     \gloss{‘buries’}  &     &   
                     \vernacular{
                    naá{\downstep}múú[{\downstep}mbé{\downstep}chélá]}  &   
                     \gloss{‘shaves’}  &  \\

                     \vernacular{
                    naá{\downstep}múú[{\downstep}ndé{\downstep}érélá]}  &   
                     \gloss{‘brings’}  &     &   
                     \vernacular{
                    naá{\downstep}múú[{\downstep}khá{\downstep}láchílá]}  &   
                     \gloss{‘cuts’}  &  \\

                     \vernacular{
                    naá{\downstep}múú[{\downstep}sí{\downstep}tááchílá]}  &   
                     \gloss{‘accuses’}  &     &   
                     \vernacular{
                    naá{\downstep}múú[{\downstep}mbó{\downstep}ólítsílá]}  &   
                     \gloss{‘seduces’}  &  \\

                     \vernacular{
                    naá{\downstep}múú[{\downstep}mbó{\downstep}hólólélá]}  &   
                     \gloss{‘unties’}  &     &     &     &  \\
\end{tabular}
%\caption{\nocaption}
     
\begin{tabular}{llllll}  
  \multicolumn{5}{l}{
                     \vernacular{(610) /H/
                    V-Initial + OP + OP
                    } \gloss{‘if
                    s/he...him/her for me’} } &  \\
\multicolumn{5}{l}{ } &  \\

                     \vernacular{
                    naá{\downstep}múú[{\downstep}nzí{\downstep}rílá]}  &   
                     \gloss{‘kills’}  &  \\

                     \vernacular{
                    naá{\downstep}múú[{\downstep}nzé{\downstep}chítsílá]}  &   
                     \gloss{‘admires’}  &  \\

                     \vernacular{
                    naá{\downstep}múú[{\downstep}nzí{\downstep}síáchílá]}  &   
                     \gloss{‘smacks’}  &  \\

                     \vernacular{
                    naá{\downstep}múú[{\downstep}nzó{\downstep}nónyínyílá]}  &   
                     \gloss{‘spoils’}  &  \\

                     \vernacular{
                    naá{\downstep}múú[{\downstep}nzá{\downstep}búkhányínyílá]}  &   
                     \gloss{‘separates’}  &  \\
\end{tabular}
%\caption{\nocaption}
     
\begin{tabular}{llllll}  
  \multicolumn{5}{l}{
                     \vernacular{(611) /Ø/
                    C-Initial + OP + OP
                    } \gloss{‘if
                    s/he...him/her for me’} } &  \\
\multicolumn{5}{l}{ } &  \\

                     \vernacular{
                    naá{\downstep}múú[{\downstep}nzííla]}  &   
                     \gloss{‘goes for’}  &     &   
                     \vernacular{
                    naá{\downstep}múú[{\downstep}ndéshéla]}  &   
                     \gloss{‘leaves’}  &  \\

                     \vernacular{
                    naá{\downstep}múú[{\downstep}nóóndela]}  &   
                     \gloss{‘follows’}  &     &   
                     \vernacular{
                    naá{\downstep}múú[{\downstep}ngúlíshila]}  &   
                     \gloss{‘names’}  &  \\

                     \vernacular{
                    naá{\downstep}múú[{\downstep}ndákhúulila]}  &   
                     \gloss{‘releases’}  &     &   
                     \vernacular{
                    naá{\downstep}múú[{\downstep}séébúlila]}  &   
                     \gloss{‘says bye
                    to’}  &  \\

                     \vernacular{
                    naá{\downstep}múú[{\downstep}mbóómbélitsila]}  &   
                     \gloss{‘comforts’}  &     &   
                     \vernacular{
                    naá{\downstep}múú[{\downstep}síínjílitsila]}  &   
                     \gloss{
                    ‘makes...stand’}  &  \\
\end{tabular}
%\caption{\nocaption}
     
\begin{tabular}{llllll}  
  \multicolumn{5}{l}{
                     \vernacular{(612) /Ø/
                    V-Initial + OP + OP
                    } \gloss{‘if
                    s/he...him/her \ob mu-\cb  / it
                    } } &  \\
\multicolumn{5}{l}{ } &  \\

                     \vernacular{
                    naá{\downstep}múú[{\downstep}nzéyéla]}  &   
                     \gloss{‘wipes’}  &     &   
                     \vernacular{
                    naá{\downstep}kúú[{\downstep}nzáshítsila]}  &   
                     \gloss{‘lights’}  &  \\

                     \vernacular{
                    naá{\downstep}búú[{\downstep}nzílúulila]}  &   
                     \gloss{‘winnows’}  &     &   
                     \vernacular{
                    naá{\downstep}kúú[{\downstep}nzéléelitsila]}  &   
                     \gloss{‘hangs’}  &  \\
\end{tabular}
%\caption{\nocaption}
     
\begin{tabular}{lll}  
  \multicolumn{2}{l}{
                     \vernacular{(613) /H/
                    C-Initial Phrase-Medial} \gloss{‘if s/he...the
                    boy \ob mú{\downstep}yáyi\cb  /} } &  \\
\multicolumn{2}{l}{
                     \gloss{someone
                    \ob muundu\cb ’} } &  \\

                     \vernacular{naá[ra]
                    mú{\downstep}yáyi/muundu}  &   
                     \gloss{‘buries’}  &  \\

                     \vernacular{naá[beka]
                    mú{\downstep}yáyi/muundu}  &   
                     \gloss{‘shaves’}  &  \\

                     \vernacular{naá[{\downstep}léérá]
                    {\downstep}mú{\downstep}yáyi/muundu}  &   
                     \gloss{‘brings’}  &  \\

                     \vernacular{naá[{\downstep}kháláká]
                    {\downstep}mú{\downstep}yáyi/muundu}  &   
                     \gloss{‘cuts’}  &  \\

                     \vernacular{naá[{\downstep}sítááká]
                    {\downstep}mú{\downstep}yáyi/muundu}  &   
                     \gloss{‘accuses’}  &  \\

                     \vernacular{
                    naá[{\downstep}bóólítsá]
                    {\downstep}mú{\downstep}yáyi/muundu}  &   
                     \gloss{‘seduces’}  &  \\

                     \vernacular{
                    naá[{\downstep}khóng’óóndá]
                    {\downstep}mú{\downstep}yáyi/muundu}  &   
                     \gloss{‘knocks’}  &  \\

                     \vernacular{
                    naá[{\downstep}bóhólólá]
                    {\downstep}mú{\downstep}yáyi/muundu}  &   
                     \gloss{‘unties’}  &  \\

                     \vernacular{
                    naá[{\downstep}bóyóng’áná]
                    {\downstep}mú{\downstep}yáyi/muundu}  &   
                     \gloss{‘goes
                    around’}  &  \\
\end{tabular}
%\caption{\nocaption}
     
\begin{tabular}{lll}  
  \multicolumn{2}{l}{
                     \vernacular{(614) /Ø/
                    C-Initial Phrase-Medial} \gloss{‘if s/he...the
                    boy \ob mú{\downstep}yáyi\cb  /} } &  \\
\multicolumn{2}{l}{
                     \gloss{someone
                    \ob muundu\cb ’} } &  \\

                     \vernacular{naá[{\downstep}tsíá]
                    {\downstep}mú{\downstep}yáyi/muundu}  &   
                     \gloss{‘goes for’}  &  \\

                     \vernacular{naá[{\downstep}lékhá]
                    {\downstep}mú{\downstep}yáyi/muundu}  &   
                     \gloss{‘leaves’}  &  \\

                     \vernacular{naá[{\downstep}lóónda]
                    mú{\downstep}yáyi/muundu}  &   
                     \gloss{‘follows’}  &  \\

                     \vernacular{naá[{\downstep}kúlíkha]
                    mú{\downstep}yáyi/muundu}  &   
                     \gloss{‘names’}  &  \\

                     \vernacular{naá[{\downstep}lákhúula]
                    mú{\downstep}yáyi/muundu}  &   
                     \gloss{‘releases’}  &  \\

                     \vernacular{naá[{\downstep}séébúla]
                    mú{\downstep}yáyi/muundu}  &   
                     \gloss{‘says bye
                    to’}  &  \\

                     \vernacular{
                    naá[{\downstep}kálúshítsa]
                    mú{\downstep}yáyi/muundu}  &   
                     \gloss{‘returns’}  &  \\

                     \vernacular{
                    naá[{\downstep}réébáreeba]
                    mú{\downstep}yáyi/muundu}  &   
                     \gloss{‘asks
                    (iter)’}  &  \\

                     \vernacular{
                    naá[{\downstep}kálúkhányinya]
                    mú{\downstep}yáyi/muundu}  &   
                     \gloss{
                    ‘turns...over’}  &  \\
\end{tabular}
%\caption{\nocaption}
     
\begin{tabular}{lll}  
  \multicolumn{2}{l}{
                     \vernacular{(615) /H/
                    C-Initial +OP Phrase-Medial} \gloss{‘if s/he...the
                    boy \ob mú{\downstep}yáyi\cb  /} } &  \\
\multicolumn{2}{l}{
                     \gloss{someone \ob muundu\cb 
                    for him/her’} } &  \\

                     \vernacular{
                    naá{\downstep}mú[ré{\downstep}élá]
                    {\downstep}mú{\downstep}yáyi/muundu}  &   
                     \gloss{‘buries’}  &  \\

                     \vernacular{
                    naá{\downstep}mú[bé{\downstep}chélá]
                    {\downstep}mú{\downstep}yáyi/muundu}  &   
                     \gloss{‘shaves’}  &  \\

                     \vernacular{
                    naá{\downstep}mú[lé{\downstep}érélá]
                    {\downstep}mú{\downstep}yáyi/muundu}  &   
                     \gloss{‘brings’}  &  \\

                     \vernacular{
                    naá{\downstep}mú[khá{\downstep}láchílá]
                    {\downstep}mú{\downstep}yáyi/muundu}  &   
                     \gloss{‘cuts’}  &  \\

                     \vernacular{
                    naá{\downstep}mú[sí{\downstep}tááchílá]
                    {\downstep}mú{\downstep}yáyi/muundu}  &   
                     \gloss{‘accuses’}  &  \\

                     \vernacular{
                    naá{\downstep}mú[bó{\downstep}ólítsílá]
                    {\downstep}mú{\downstep}yáyi/muundu}  &   
                     \gloss{‘seduces’}  &  \\

                     \vernacular{
                    naá{\downstep}mú[khó{\downstep}ng’óóndélá]
                    {\downstep}mú{\downstep}yáyi/muundu}  &   
                     \gloss{‘knocks’}  &  \\

                     \vernacular{
                    naá{\downstep}mú[bó{\downstep}hólólélá]
                    {\downstep}mú{\downstep}yáyi/muundu}  &   
                     \gloss{‘unties’}  &  \\

                     \vernacular{
                    naá{\downstep}mú[bó{\downstep}yóng’ánílá]
                    {\downstep}mú{\downstep}yáyi/muundu}  &   
                     \gloss{‘goes
                    around’}  &  \\
\end{tabular}
%\caption{\nocaption}
     
\begin{tabular}{lll}  
  \multicolumn{2}{l}{
                     \vernacular{(616) /Ø/
                    C-Initial +OP Phrase-Medial} \gloss{‘if s/he...the
                    boy \ob mú{\downstep}yáyi\cb  /} } &  \\
\multicolumn{2}{l}{
                     \gloss{someone \ob muundu\cb 
                    for him/her’} } &  \\

                     \vernacular{naá{\downstep}mú[tsííla]
                    mú{\downstep}yáyi/muundu}  &   
                     \gloss{‘goes for’}  &  \\

                     \vernacular{
                    naá{\downstep}mú[léshéla] mú{\downstep}yáyi/muundu}  &   
                     \gloss{‘leaves’}  &  \\

                     \vernacular{
                    naá{\downstep}mú[lóóndéla]
                    mú{\downstep}yáyi/muundu}  &   
                     \gloss{‘follows’}  &  \\

                     \vernacular{
                    naá{\downstep}mú[kúlíshíla]
                    mú{\downstep}yáyi/muundu}  &   
                     \gloss{‘names’}  &  \\

                     \vernacular{
                    naá{\downstep}mú[lákhúulila]
                    mú{\downstep}yáyi/muundu}  &   
                     \gloss{‘releases’}  &  \\

                     \vernacular{
                    naá{\downstep}mú[séébúlila]
                    mú{\downstep}yáyi/muundu}  &   
                     \gloss{‘says bye
                    to’}  &  \\

                     \vernacular{
                    naá{\downstep}mú[kálúshítsila]
                    mú{\downstep}yáyi/muundu}  &   
                     \gloss{‘returns’}  &  \\

                     \vernacular{
                    naá{\downstep}mú[réébáreebela]
                    mú{\downstep}yáyi/muundu}  &   
                     \gloss{‘asks
                    (iter)’}  &  \\
\end{tabular}
%\caption{\nocaption}
     
\begin{tabular}{lll}  
  \multicolumn{2}{l}{
                     \vernacular{(617) /H/
                    C-Initial +OP + OP
                    } \gloss{‘if s/he...the
                    boy \ob mú{\downstep}yáyi\cb  /} } &  \\
\multicolumn{2}{l}{
                     \gloss{someone \ob muundu\cb 
                    for him/her for me’} } &  \\

                     \vernacular{
                    naá{\downstep}múú[{\downstep}ndéélá]
                    {\downstep}mú{\downstep}yáyi/muundu}  &   
                     \gloss{‘buries’}  &  \\

                     \vernacular{
                    naá{\downstep}múú[{\downstep}mbéchélá]
                    {\downstep}mú{\downstep}yáyi/muundu}  &   
                     \gloss{‘shaves’}  &  \\

                     \vernacular{
                    naá{\downstep}múú[{\downstep}ndéérélá]
                    {\downstep}mú{\downstep}yáyi/muundu}  &   
                     \gloss{‘brings’}  &  \\

                     \vernacular{
                    naá{\downstep}múú[{\downstep}kháláchílá]
                    {\downstep}mú{\downstep}yáyi/muundu}  &   
                     \gloss{‘cuts’}  &  \\

                     \vernacular{
                    naá{\downstep}múú[{\downstep}sítááchílá]
                    {\downstep}mú{\downstep}yáyi/muundu}  &   
                     \gloss{‘accuses’}  &  \\

                     \vernacular{
                    naá{\downstep}múú[{\downstep}mbóólítsílá]
                    {\downstep}mú{\downstep}yáyi/muundu}  &   
                     \gloss{‘seduces’}  &  \\

                     \vernacular{
                    naá{\downstep}múú[{\downstep}mbóhólólélá]
                    {\downstep}mú{\downstep}yáyi/muundu}  &   
                     \gloss{‘unties’}  &  \\
\end{tabular}
%\caption{\nocaption}
     
\begin{tabular}{lll}  
  \multicolumn{2}{l}{
                     \vernacular{(618) /Ø/
                    C-Initial +OP + OP
                    } \gloss{‘if s/he...the
                    boy \ob mú{\downstep}yáyi\cb  /} } &  \\
\multicolumn{2}{l}{
                     \gloss{someone \ob muundu\cb 
                    for him/her for me’} } &  \\

                     \vernacular{
                    naá{\downstep}múú[{\downstep}nzííla]
                    mú{\downstep}yáyi/muundu}  &   
                     \gloss{‘goes for’}  &  \\

                     \vernacular{
                    naá{\downstep}múú[{\downstep}ndéshéla]
                    mú{\downstep}yáyi/muundu}  &   
                     \gloss{‘leaves’}  &  \\

                     \vernacular{
                    naá{\downstep}múú[{\downstep}nóóndéla]
                    mú{\downstep}yáyi/muundu}  &   
                     \gloss{‘follows’}  &  \\

                     \vernacular{
                    naá{\downstep}múú[{\downstep}ngúlíshíla]
                    mú{\downstep}yáyi/muundu}  &   
                     \gloss{‘names’}  &  \\

                     \vernacular{
                    naá{\downstep}múú[{\downstep}ndákhúulila]
                    mú{\downstep}yáyi/muundu}  &   
                     \gloss{‘releases’}  &  \\

                     \vernacular{
                    naá{\downstep}múú[{\downstep}séébúlila]
                    mú{\downstep}yáyi/muundu}  &   
                     \gloss{‘says bye
                    to’}  &  \\

                     \vernacular{
                    naá{\downstep}múú[{\downstep}síínjílitsila]
                    mú{\downstep}yáyi/muundu}  &   
                     \gloss{
                    ‘makes...stand’}  &  \\
\end{tabular}
%\caption{\nocaption}
    

\subsection{Conditional Negative: Pattern 2b}\label{sec:sCondNeg}


\begin{tabular}{llllll}  
  \multicolumn{5}{l}{
                     \vernacular{(619) /H/
                    C-Initial} \gloss{‘if s/he does
                    not...’} } &  \\
\multicolumn{5}{l}{ } &  \\

                     \vernacular{naákha[ra]
                    tá}  &   
                     \gloss{‘bury’}  &     &   
                     \vernacular{naákha[ng’wa]
                    tá}  &   
                     \gloss{‘drink’}  &  \\

                     \vernacular{naákha[khwa]
                    tá}  &   
                     \gloss{‘eat’}  &     &   
                     \vernacular{naákha[lia]
                    tá}  &   
                     \gloss{‘pay dowry’}  &  \\

                     \vernacular{naákha[luma]
                    tá}  &   
                     \gloss{‘bite’}  &     &   
                     \vernacular{naákha[beka]
                    tá}  &   
                     \gloss{‘shave’}  &  \\

                     \vernacular{naákha[teekha]
                    tá}  &   
                     \gloss{‘cook’}  &     &   
                     \vernacular{naákha[leera]
                    tá}  &   
                     \gloss{‘bring’}  &  \\

                     \vernacular{naákha[khalaka]
                    tá}  &   
                     \gloss{‘cut’}  &     &   
                     \vernacular{naákha[kalaanga]
                    tá}  &   
                     \gloss{‘fry’}  &  \\

                     \vernacular{naákha[sitaaka]
                    tá}  &   
                     \gloss{‘accuse’}  &     &   
                     \vernacular{naákha[boolitsa]
                    tá}  &   
                     \gloss{‘seduce’}  &  \\

                     \vernacular{naákha[saanditsa]
                    tá}  &   
                     \gloss{‘thank’}  &     &   
                     \vernacular{
                    naákha[khong’oonda] tá}  &   
                     \gloss{‘knock’}  &  \\

                     \vernacular{naákha[boholola]
                    tá}  &   
                     \gloss{‘untie’}  &     &   
                     \vernacular{
                    naákha[boyong’ana] tá}  &   
                     \gloss{‘go around’}  &  \\

                     \vernacular{
                    naákha[ng’ong’oolitsa] tá}  &   
                     \gloss{‘tease’}  &     &   
                     \vernacular{
                    naákha[lingakanyinya] tá}  &   
                     \gloss{‘crumple’}  &  \\
\end{tabular}
%\caption{\nocaption}
     
\begin{tabular}{llllll}  
  \multicolumn{5}{l}{
                     \vernacular{(620) /Ø/
                    C-Initial} \gloss{‘if s/he does
                    not...’} } &  \\
\multicolumn{5}{l}{ } &  \\

                     \vernacular{naákha[tsíá]
                    {\downstep}tá}  &   
                     \gloss{‘go’}  &     &   
                     \vernacular{naákha[kwá]
                    {\downstep}tá}  &   
                     \gloss{‘fall’}  &  \\

                     \vernacular{naákha[lekhá]
                    {\downstep}tá}  &   
                     \gloss{‘leave’}  &     &   
                     \vernacular{naákha[reéba]
                    tá}  &   
                     \gloss{‘ask’}  &  \\

                     \vernacular{naákha[loónda]
                    tá}  &   
                     \gloss{‘follow’}  &     &   
                     \vernacular{naákha[kumíla]
                    tá}  &   
                     \gloss{‘hold’}  &  \\

                     \vernacular{naákha[kulíkha]
                    tá}  &   
                     \gloss{‘name’}  &     &   
                     \vernacular{naákha[homóola]
                    tá}  &   
                     \gloss{‘massage’}  &  \\

                     \vernacular{naákha[lakhúula]
                    tá}  &   
                     \gloss{‘release’}  &     &   
                     \vernacular{naákha[seébúla]
                    tá}  &   
                     \gloss{‘say bye’}  &  \\

                     \vernacular{
                    naákha[hoómbélitsa] tá}  &   
                     \gloss{‘comfort’}  &     &   
                     \vernacular{
                    naákha[kalúshitsa] tá}  &   
                     \gloss{‘return’}  &  \\

                     \vernacular{
                    naákha[siínjílitsa] tá}  &   
                     \gloss{‘make stand’}  &     &   
                     \vernacular{
                    naákha[reébáreeba] tá}  &   
                     \gloss{‘ask (iter)’}  &  \\

                     \vernacular{
                    naákha[kalúkhányinya] tá}  &   
                     \gloss{‘turn over’}  &     &   
                     \vernacular{
                    naákha[sebúlúkhanyinya] tá}  &   
                     \gloss{‘scatter’}  &  \\
\end{tabular}
%\caption{\nocaption}
     
\begin{tabular}{llllll}  
  \multicolumn{5}{l}{
                     \vernacular{(621) /H/
                    C-Initial + OP} \gloss{‘if s/he does
                    not...him/her’} } &  \\
\multicolumn{5}{l}{ } &  \\

                     \vernacular{naákhamu[rá]
                    {\downstep}tá}  &   
                     \gloss{‘bury’}  &     &   
                     \vernacular{naákhamu[béka]
                    tá}  &   
                     \gloss{‘shave’}  &  \\

                     \vernacular{naákhamu[léera]
                    tá}  &   
                     \gloss{‘bring’}  &     &   
                     \vernacular{
                    naákhamu[khálaka] tá}  &   
                     \gloss{‘cut’}  &  \\

                     \vernacular{
                    naákhamu[sítaaka] tá}  &   
                     \gloss{‘accuse’}  &     &   
                     \vernacular{
                    naákhamu[bóolitsa] tá}  &   
                     \gloss{‘seduce’}  &  \\

                     \vernacular{
                    naákhamu[khóng’oonda] tá}  &   
                     \gloss{‘knock’}  &     &   
                     \vernacular{
                    naákhamu[bóholola] tá}  &   
                     \gloss{‘untie’}  &  \\

                     \vernacular{
                    naákhamu[bóyong’ana] tá}  &   
                     \gloss{‘go around’}  &     &   
                     \vernacular{
                    naákhamu[ng’óng’oolitsa] tá}  &   
                     \gloss{‘tease’}  &  \\

                     \vernacular{
                    naákhamu[língakanyinya] tá}  &   
                     \gloss{‘bend’}  &     &     &     &  \\
\end{tabular}
%\caption{\nocaption}
     
\begin{tabular}{llllll}  
  \multicolumn{5}{l}{
                     \vernacular{(622) /Ø/
                    C-Initial + OP} \gloss{‘if s/he does
                    not...him/her \ob mu-\cb  / them
                    } } &  \\
\multicolumn{5}{l}{ } &  \\

                     \vernacular{naákhamu[tsíá]
                    {\downstep}tá}  &   
                     \gloss{‘go for’}  &  \\

                     \vernacular{naákhamu[lekhá]
                    {\downstep}tá}  &   
                     \gloss{‘leave’}  &  \\

                     \vernacular{naákhamu[loónda]
                    tá}  &   
                     \gloss{‘follow’}  &  \\

                     \vernacular{
                    naákhamu[kulíkha] tá}  &   
                     \gloss{‘name’}  &  \\

                     \vernacular{
                    naákhamu[lakhúula] tá}  &   
                     \gloss{‘release’}  &  \\

                     \vernacular{
                    naákhamu[seébúla] tá}  &   
                     \gloss{‘say bye to’}  &  \\

                     \vernacular{
                    naákhamu[hoómbélitsa] tá}  &   
                     \gloss{‘comfort’}  &  \\

                     \vernacular{
                    naákhamu[kalúshitsa] tá}  &   
                     \gloss{‘return’}  &  \\

                     \vernacular{
                    naákhamu[siínjílitsa] tá}  &   
                     \gloss{
                    ‘make...stand’}  &  \\

                     \vernacular{
                    naákhamu[reébáreeba] tá}  &   
                     \gloss{‘ask (iter)’}  &  \\

                     \vernacular{
                    naákhamu[kalúkhányinya] tá}  &   
                     \gloss{
                    ‘turn...over’}  &  \\

                     \vernacular{
                    naákhabi[sebúlúkhanyinya] tá}  &   
                     \gloss{‘scatter’}  &  \\
\end{tabular}
%\caption{\nocaption}
     
\begin{tabular}{llllll}  
  \multicolumn{5}{l}{
                     \vernacular{(623) /H/
                    C-Initial + OP + OP
                    } \gloss{‘if s/he does
                    not...him/her for me’} } &  \\
\multicolumn{5}{l}{ } &  \\

                     \vernacular{
                    naákhamuú[ndeelá] {\downstep}tá}  &   
                     \gloss{‘bury’}  &     &   
                     \vernacular{
                    naákhamuú[mbechelá] {\downstep}tá}  &   
                     \gloss{‘shave’}  &  \\

                     \vernacular{
                    naákhamuú[ndeerelá] {\downstep}tá}  &   
                     \gloss{‘bring’}  &     &   
                     \vernacular{
                    naákhamuú[khalachilá] {\downstep}tá}  &   
                     \gloss{‘cut’}  &  \\

                     \vernacular{
                    naákhamuú[sitaachilá] {\downstep}tá}  &   
                     \gloss{‘accuse’}  &     &   
                     \vernacular{
                    naákhamuú[mboolitsilá] {\downstep}tá}  &   
                     \gloss{‘seduce’}  &  \\

                     \vernacular{
                    naákhamuú[mbohololelá] {\downstep}tá}  &   
                     \gloss{‘untie’}  &     &     &     &  \\
\end{tabular}
%\caption{\nocaption}
     
\begin{tabular}{llllll}  
  \multicolumn{5}{l}{
                     \vernacular{(624) /Ø/
                    C-Initial + OP + OP
                    } \gloss{‘if s/he does
                    not...him/her for me’} } &  \\
\multicolumn{5}{l}{ } &  \\

                     \vernacular{
                    naákhamuú[{\downstep}nzííla] tá}  &   
                     \gloss{‘go for’}  &  \\

                     \vernacular{
                    naákhamuú[{\downstep}ndéshéla] tá}  &   
                     \gloss{‘leave’}  &  \\

                     \vernacular{
                    naákhamuú[{\downstep}nóóndéla] tá}  &   
                     \gloss{‘follow’}  &  \\

                     \vernacular{
                    naákhamuú[{\downstep}ngúlíshíla] tá}  &   
                     \gloss{‘name’}  &  \\

                     \vernacular{
                    naákhamuú[{\downstep}ndákhúulila] tá}  &   
                     \gloss{‘release’}  &  \\

                     \vernacular{
                    naákhamuú[{\downstep}séébúlila] tá}  &   
                     \gloss{‘say bye to’}  &  \\

                     \vernacular{
                    naákhamuú[{\downstep}mbóómbélitsila] tá}  &   
                     \gloss{‘comfort’}  &  \\

                     \vernacular{
                    naákhamuú[{\downstep}síínjílitsila] tá}  &   
                     \gloss{
                    ‘make...stand’}  &  \\
\end{tabular}
%\caption{\nocaption}
     
\begin{tabular}{lll}  
  \multicolumn{2}{l}{
                     \vernacular{(625) /H/
                    C-Initial Phrase-Medial} \gloss{‘if s/he does
                    not...the boy \ob mú{\downstep}yáyi\cb  /} } &  \\
\multicolumn{2}{l}{
                     \gloss{someone
                    \ob muundu\cb ’} } &  \\

                     \vernacular{naákha[ra]
                    mú{\downstep}yáyi/muundu tá}  &   
                     \gloss{‘bury’}  &  \\

                     \vernacular{naákha[beka]
                    mú{\downstep}yáyi/muundu tá}  &   
                     \gloss{‘shave’}  &  \\

                     \vernacular{naákha[leera]
                    mú{\downstep}yáyi/muundu tá}  &   
                     \gloss{‘bring’}  &  \\

                     \vernacular{naákha[khalaka]
                    mú{\downstep}yáyi/muundu tá}  &   
                     \gloss{‘cut’}  &  \\

                     \vernacular{naákha[sitaaka]
                    mú{\downstep}yáyi/muundu tá}  &   
                     \gloss{‘accuse’}  &  \\

                     \vernacular{naákha[boolitsa]
                    mú{\downstep}yáyi/muundu tá}  &   
                     \gloss{‘seduce’}  &  \\

                     \vernacular{
                    naákha[khong’oonda] mú{\downstep}yáyi/muundu
                    tá}  &   
                     \gloss{‘knock’}  &  \\

                     \vernacular{naákha[boholola]
                    mú{\downstep}yáyi/muundu tá}  &   
                     \gloss{‘untie’}  &  \\

                     \vernacular{
                    naákha[boyong’ana] mú{\downstep}yáyi/muundu
                    tá}  &   
                     \gloss{‘go around’}  &  \\
\end{tabular}
%\caption{\nocaption}
     
\begin{tabular}{lll}  
  \multicolumn{2}{l}{
                     \vernacular{(626) /Ø/
                    C-Initial Phrase-Medial} \gloss{‘if s/he does
                    not...the boy \ob mú{\downstep}yáyi\cb  /} } &  \\
\multicolumn{2}{l}{
                     \gloss{someone
                    \ob muundu\cb ’} } &  \\

                     \vernacular{naákha[tsíá]
                    {\downstep}mú{\downstep}yáyi/muundu tá}  &   
                     \gloss{‘go for’}  &  \\

                     \vernacular{naákha[lekhá]
                    {\downstep}mú{\downstep}yáyi/muundu tá}  &   
                     \gloss{‘leave’}  &  \\

                     \vernacular{naákha[loónda]
                    mú{\downstep}yáyi/muundu tá}  &   
                     \gloss{‘follow’}  &  \\

                     \vernacular{naákha[kulíkha]
                    mú{\downstep}yáyi/muundu tá}  &   
                     \gloss{‘name’}  &  \\

                     \vernacular{naákha[lakhúula]
                    mú{\downstep}yáyi/muundu tá}  &   
                     \gloss{‘release’}  &  \\

                     \vernacular{naákha[seébúla]
                    mú{\downstep}yáyi/muundu tá}  &   
                     \gloss{‘say bye to’}  &  \\

                     \vernacular{
                    naákha[kalúshítsa] mú{\downstep}yáyi/muundu
                    tá}  &   
                     \gloss{‘return’}  &  \\

                     \vernacular{
                    naákha[reébáreeba] mú{\downstep}yáyi/muundu
                    tá}  &   
                     \gloss{‘ask (iter)’}  &  \\

                     \vernacular{
                    naákha[kalúkhányinya] mú{\downstep}yáyi/muundu
                    tá}  &   
                     \gloss{
                    ‘turn...over’}  &  \\
\end{tabular}
%\caption{\nocaption}
     
\begin{tabular}{lll}  
  \multicolumn{2}{l}{
                     \vernacular{(627) /H/
                    C-Initial +OP Phrase-Medial} \gloss{‘if s/he does
                    not...the boy \ob mú{\downstep}yáyi\cb  /} } &  \\
\multicolumn{2}{l}{
                     \gloss{someone \ob muundu\cb 
                    for him/her’} } &  \\

                     \vernacular{naákhamu[reela]
                    mú{\downstep}yáyi/muundu tá}  &   
                     \gloss{‘bury’}  &  \\

                     \vernacular{
                    naákhamu[béchela] mú{\downstep}yáyi/muundu
                    tá}  &   
                     \gloss{‘shave’}  &  \\

                     \vernacular{
                    naákhamu[léerela] mú{\downstep}yáyi/muundu
                    tá}  &   
                     \gloss{‘bring’}  &  \\

                     \vernacular{
                    naákhamu[khálachila] mú{\downstep}yáyi/muundu
                    tá}  &   
                     \gloss{‘cut’}  &  \\

                     \vernacular{
                    naákhamu[sítaachila] mú{\downstep}yáyi/muundu
                    tá}  &   
                     \gloss{‘accuse’}  &  \\

                     \vernacular{
                    naákhamu[bóolitsila] mú{\downstep}yáyi/muundu
                    tá}  &   
                     \gloss{‘seduce’}  &  \\

                     \vernacular{
                    naákhamu[khóng’oondela] mú{\downstep}yáyi/muundu
                    tá}  &   
                     \gloss{‘knock’}  &  \\

                     \vernacular{
                    naákhamu[bóhololela] mú{\downstep}yáyi/muundu
                    tá}  &   
                     \gloss{‘untie’}  &  \\

                     \vernacular{
                    naákhamu[bóyong’anila] mú{\downstep}yáyi/muundu
                    tá}  &   
                     \gloss{‘go around’}  &  \\
\end{tabular}
%\caption{\nocaption}
     
\begin{tabular}{lll}  
  \multicolumn{2}{l}{
                     \vernacular{(628) /Ø/
                    C-Initial +OP Phrase-Medial} \gloss{‘if s/he does
                    not...the boy \ob mú{\downstep}yáyi\cb  /} } &  \\
\multicolumn{2}{l}{
                     \gloss{someone \ob muundu\cb 
                    for him/her’} } &  \\

                     \vernacular{
                    naákhamu[tsííla] mú{\downstep}yáyi/muundu
                    tá}  &   
                     \gloss{‘go for’}  &  \\

                     \vernacular{
                    naákhamu[leshéla] mú{\downstep}yáyi/muundu
                    tá}  &   
                     \gloss{‘leave’}  &  \\

                     \vernacular{
                    naákhamu[loóndela] mú{\downstep}yáyi/muundu
                    tá}  &   
                     \gloss{‘follow’}  &  \\

                     \vernacular{
                    naákhamu[kulíshila] mú{\downstep}yáyi/muundu
                    tá}  &   
                     \gloss{‘name’}  &  \\

                     \vernacular{
                    naákhamu[lakhúulila] mú{\downstep}yáyi/muundu
                    tá}  &   
                     \gloss{‘release’}  &  \\

                     \vernacular{
                    naákhamu[seébúlila] mú{\downstep}yáyi/muundu
                    tá}  &   
                     \gloss{‘say bye to’}  &  \\

                     \vernacular{
                    naákhamu[kalúshítsila] mú{\downstep}yáyi/muundu
                    tá}  &   
                     \gloss{‘return’}  &  \\

                     \vernacular{
                    naákhamu[reébáreebela] mú{\downstep}yáyi/muundu
                    tá}  &   
                     \gloss{‘ask (iter)’}  &  \\
\end{tabular}
%\caption{\nocaption}
     
\begin{tabular}{lll}  
  \multicolumn{2}{l}{
                     \vernacular{(629) /H/
                    C-Initial +OP + OP
                    } \gloss{‘if s/he does
                    not...the boy \ob mú{\downstep}yáyi\cb  /} } &  \\
\multicolumn{2}{l}{
                     \gloss{someone \ob muundu\cb 
                    for him/her for me’} } &  \\

                     \vernacular{
                    naákhamuú[ndeelá] {\downstep}mú{\downstep}yáyi/muundu
                    tá}  &   
                     \gloss{‘bury’}  &  \\

                     \vernacular{
                    naákhamuú[mbechelá] {\downstep}mú{\downstep}yáyi/muundu
                    tá}  &   
                     \gloss{‘shave’}  &  \\

                     \vernacular{
                    naákhamuú[ndeerelá] {\downstep}mú{\downstep}yáyi/muundu
                    tá}  &   
                     \gloss{‘bring’}  &  \\

                     \vernacular{
                    naákhamuú[khalachilá] {\downstep}mú{\downstep}yáyi/muundu
                    tá}  &   
                     \gloss{‘cut’}  &  \\

                     \vernacular{
                    naákhamuú[sitaachilá] {\downstep}mú{\downstep}yáyi/muundu
                    tá}  &   
                     \gloss{‘accuse’}  &  \\

                     \vernacular{
                    naákhamuú[mboolitsilá] {\downstep}mú{\downstep}yáyi/muundu
                    tá}  &   
                     \gloss{‘seduce’}  &  \\

                     \vernacular{
                    naákhamuú[mbohololelá] {\downstep}mú{\downstep}yáyi/muundu
                    tá}  &   
                     \gloss{‘untie’}  &  \\
\end{tabular}
%\caption{\nocaption}
     
\begin{tabular}{lll}  
  \multicolumn{2}{l}{
                     \vernacular{(630) /Ø/
                    C-Initial +OP + OP
                    } \gloss{‘if s/he does
                    not...the boy \ob mú{\downstep}yáyi\cb  /} } &  \\
\multicolumn{2}{l}{
                     \gloss{someone \ob muundu\cb 
                    for him/her for me’} } &  \\

                     \vernacular{
                    naákhamuú[{\downstep}nzííla] mú{\downstep}yáyi/muundu
                    tá}  &   
                     \gloss{‘go for’}  &  \\

                     \vernacular{
                    naákhamuú[{\downstep}ndéshéla] mú{\downstep}yáyi/muundu
                    tá}  &   
                     \gloss{‘leave’}  &  \\

                     \vernacular{
                    naákhamuú[{\downstep}nóóndéla] mú{\downstep}yáyi/muundu
                    tá}  &   
                     \gloss{‘follow’}  &  \\

                     \vernacular{
                    naákhamuú[{\downstep}ngúlíshíla] mú{\downstep}yáyi/muundu
                    tá}  &   
                     \gloss{‘name’}  &  \\

                     \vernacular{
                    naákhamuú[{\downstep}ndákhúulila] mú{\downstep}yáyi/muundu
                    tá}  &   
                     \gloss{‘release’}  &  \\

                     \vernacular{
                    naákhamuú[{\downstep}séébúlila] mú{\downstep}yáyi/muundu
                    tá}  &   
                     \gloss{‘say bye to’}  &  \\

                     \vernacular{
                    naákhamuú[{\downstep}síínjílitsila] mú{\downstep}yáyi/muundu
                    tá}  &   
                     \gloss{
                    ‘make...stand’}  &  \\
\end{tabular}
%\caption{\nocaption}
    

\subsection{Persistive: Pattern 5a
              }\label{sec:sPers}


\begin{tabular}{llllll}  
  \multicolumn{5}{l}{
                     \vernacular{(631) /H/
                    C-Initial} \gloss{‘s/he is
                    still...’} } &  \\
\multicolumn{5}{l}{ } &  \\

                     \vernacular{
                    ashi[reetsáángá]}  &   
                     \gloss{‘burying’}  &  \\

                     \vernacular{
                    ashi[ng’weetsáángá]}  &   
                     \gloss{‘drinking’}  &  \\

                     \vernacular{
                    ashi[khweetsáángá]}  &   
                     \gloss{‘paying
                    dowry’}  &  \\

                     \vernacular{
                    ashi[liitsáángá]}  &   
                     \gloss{‘eating’}  &  \\

                     \vernacular{
                    ashi[lumaángá]}  &   
                     \gloss{‘biting’}  &  \\

                     \vernacular{
                    ashi[bekaángá]}  &   
                     \gloss{‘shaving’}  &  \\

                     \vernacular{
                    ashi[teekháángá]}  &   
                     \gloss{‘cooking’}  &  \\

                     \vernacular{
                    ashi[leeráángá]}  &   
                     \gloss{‘bringing’}  &  \\

                     \vernacular{
                    ashi[khalakáánga]}  &   
                     \gloss{‘cutting’}  &  \\

                     \vernacular{
                    ashi[kalaangáánga]}  &   
                     \gloss{‘frying’}  &  \\

                     \vernacular{
                    ashi[sitaakáánga]}  &   
                     \gloss{‘accusing’}  &  \\

                     \vernacular{
                    ashi[boolitsáánga]}  &   
                     \gloss{‘seducing’}  &  \\

                     \vernacular{
                    ashi[saanditsáánga]}  &   
                     \gloss{‘thanking’}  &  \\

                     \vernacular{
                    ashi[khong’oondáánga]}  &   
                     \gloss{‘knocking’}  &  \\

                     \vernacular{
                    ashi[boholóláanga]}  &   
                     \gloss{‘untying’}  &  \\

                     \vernacular{
                    ashi[boyong’ánáanga]}  &   
                     \gloss{‘going
                    around’}  &  \\

                     \vernacular{
                    ashi[ng’ong’oolítsáanga]}  &   
                     \gloss{‘teasing’}  &  \\

                     \vernacular{
                    ashi[linga(ka)nyínyáanga]}  &   
                     \gloss{‘crumpling’}  &  \\
\end{tabular}
%\caption{\nocaption}
     
\begin{tabular}{llllll}  
  \multicolumn{5}{l}{
                     \vernacular{(632) /H/
                    V-Initial} \gloss{‘s/he is
                    still...’} } &  \\
\multicolumn{5}{l}{ } &  \\

                     \vernacular{
                    ash[iiraángá]}  &   
                     \gloss{‘killing’}  &     &   
                     \vernacular{
                    ash[iikóómbáanga]}  &   
                     \gloss{‘admiring’}  &  \\

                     \vernacular{
                    ash[iisíákáanga]}  &   
                     \gloss{‘smacking’}  &     &   
                     \vernacular{
                    ash[iikobóláanga]}  &   
                     \gloss{‘belching’}  &  \\

                     \vernacular{
                    ash[oononyínyáanga]}  &   
                     \gloss{‘spoiling’}  &     &   
                     \vernacular{
                    ash[aabukhányínyaanga]}  &   
                     \gloss{‘separating’}  &  \\
\end{tabular}
%\caption{\nocaption}
     
\begin{tabular}{llllll}  
  \multicolumn{5}{l}{
                     \vernacular{(633) /Ø/
                    C-Initial} \gloss{‘s/he is
                    still...’} } &  \\
\multicolumn{5}{l}{ } &  \\

                     \vernacular{
                    ashi[tsiítsaanga]}  &   
                     \gloss{‘going’}  &  \\

                     \vernacular{
                    ashi[kwiítsaanga]}  &   
                     \gloss{‘falling’}  &  \\

                     \vernacular{
                    ashi[lekháanga]}  &   
                     \gloss{‘leaving’}  &  \\

                     \vernacular{
                    ashi[reébáanga]}  &   
                     \gloss{‘asking’}  &  \\

                     \vernacular{
                    ashi[loóndáanga]}  &   
                     \gloss{‘following’}  &  \\

                     \vernacular{
                    ashi[kumíláanga]}  &   
                     \gloss{‘holding’}  &  \\

                     \vernacular{
                    ashi[kulíkháanga]}  &   
                     \gloss{‘naming’}  &  \\

                     \vernacular{
                    ashi[homóolaanga]}  &   
                     \gloss{‘massaging’}  &  \\

                     \vernacular{
                    ashi[lakhúulaanga]}  &   
                     \gloss{‘releasing’}  &  \\

                     \vernacular{
                    ashi[seébúlaanga]}  &   
                     \gloss{‘saying bye’}  &  \\

                     \vernacular{
                    ashi[hoómbélitsaanga]}  &   
                     \gloss{‘comforting’}  &  \\

                     \vernacular{
                    ashi[kalúshítsaanga]}  &   
                     \gloss{‘returning’}  &  \\

                     \vernacular{
                    ashi[siínjílitsaanga]}  &   
                     \gloss{‘making
                    stand’}  &  \\

                     \vernacular{
                    ashi[reébáreebaanga]}  &   
                     \gloss{‘asking
                    (iter)’}  &  \\

                     \vernacular{
                    ashi[kalúkhányinyaanga]}  &   
                     \gloss{‘turning
                    over’}  &  \\

                     \vernacular{
                    ashi[sebúlúkhanyinyaanga]}  &   
                     \gloss{‘scattering’}  &  \\
\end{tabular}
%\caption{\nocaption}
     
\begin{tabular}{llllll}  
  \multicolumn{5}{l}{
                     \vernacular{(634) /Ø/
                    V-Initial} \gloss{‘s/he is
                    still...’} } &  \\
\multicolumn{5}{l}{ } &  \\

                     \vernacular{
                    ash[eenyáanga]}  &   
                     \gloss{‘wanting’}  &     &   
                     \vernacular{
                    ash[eeyéláanga]}  &   
                     \gloss{‘wiping for’}  &  \\

                     \vernacular{
                    ash[iilúulaanga]}  &   
                     \gloss{‘winnowing’}  &     &   
                     \vernacular{
                    ash[aambákhánaanga]}  &   
                     \gloss{‘refusing’}  &  \\

                     \vernacular{
                    ash[eeléelitsaanga]}  &   
                     \gloss{‘hanging up’}  &  \\
\end{tabular}
%\caption{\nocaption}
     
\begin{tabular}{llllll}  
  \multicolumn{5}{l}{
                     \vernacular{(635) /H/
                    C-Initial + OP} \gloss{‘s/he is
                    still...him/her’} } &  \\
\multicolumn{5}{l}{ } &  \\

                     \vernacular{
                    ashimu[ré{\downstep}étsáángá]}  &   
                     \gloss{‘burying’}  &  \\

                     \vernacular{
                    ashimu[bé{\downstep}káángá]}  &   
                     \gloss{‘shaving’}  &  \\

                     \vernacular{
                    ashimu[lé{\downstep}éráángá]}  &   
                     \gloss{‘bringing’}  &  \\

                     \vernacular{
                    ashimu[khá{\downstep}lákáánga]}  &   
                     \gloss{‘cutting’}  &  \\

                     \vernacular{
                    ashimu[sí{\downstep}táákáánga]}  &   
                     \gloss{‘accusing’}  &  \\

                     \vernacular{
                    ashimu[bó{\downstep}ólítsáánga]}  &   
                     \gloss{‘seducing’}  &  \\

                     \vernacular{
                    ashimu[khó{\downstep}ng’óóndáánga]}  &   
                     \gloss{‘knocking’}  &  \\

                     \vernacular{
                    ashimu[bó{\downstep}hólóláanga]}  &   
                     \gloss{‘untying’}  &  \\

                     \vernacular{
                    ashimu[bó{\downstep}yóng’ánáanga]}  &   
                     \gloss{‘going
                    around’}  &  \\

                     \vernacular{
                    ashimu[ng’ó{\downstep}ng’óólítsáanga]}  &   
                     \gloss{‘teasing’}  &  \\

                     \vernacular{
                    ashimu[lí{\downstep}ngá(ka)nyínyáanga]}  &   
                     \gloss{‘bending’}  &  \\
\end{tabular}
%\caption{\nocaption}
     
\begin{tabular}{llllll}  
  \multicolumn{5}{l}{
                     \vernacular{(636) /H/
                    V-Initial + OP} \gloss{‘s/he is
                    still...him/her’} } &  \\
\multicolumn{5}{l}{ } &  \\

                     \vernacular{
                    ashimw[ií{\downstep}ráángá]}  &   
                     \gloss{‘killing’}  &  \\

                     \vernacular{
                    ashimw[ií{\downstep}kóómbáánga]}  &   
                     \gloss{‘admiring’}  &  \\

                     \vernacular{
                    ashimw[ií{\downstep}síákáánga]}  &   
                     \gloss{‘smacking’}  &  \\

                     \vernacular{
                    ashimw[oó{\downstep}nónyínyáanga]}  &   
                     \gloss{‘spoiling’}  &  \\

                     \vernacular{
                    ashimw[aá{\downstep}búkhányínyaanga]}  &   
                     \gloss{‘separating’}  &  \\
\end{tabular}
%\caption{\nocaption}
     
\begin{tabular}{llllll}  
  \multicolumn{5}{l}{
                     \vernacular{(637) /Ø/
                    C-Initial + OP} \gloss{‘s/he is
                    still...him/her \ob mu-\cb  / them
                    } } &  \\
\multicolumn{5}{l}{ } &  \\

                     \vernacular{
                    ashimu[tsiítsaanga]}  &   
                     \gloss{‘going for’}  &  \\

                     \vernacular{
                    ashimu[lekháanga]}  &   
                     \gloss{‘leaving’}  &  \\

                     \vernacular{
                    ashimu[loóndáanga]}  &   
                     \gloss{‘following’}  &  \\

                     \vernacular{
                    ashimu[kulíkháanga]}  &   
                     \gloss{‘naming’}  &  \\

                     \vernacular{
                    ashimu[lakhúulaanga]}  &   
                     \gloss{‘releasing’}  &  \\

                     \vernacular{
                    ashimu[seébúlaanga]}  &   
                     \gloss{‘saying bye
                    to’}  &  \\

                     \vernacular{
                    ashimu[hoómbélitsaanga]}  &   
                     \gloss{‘comforting’}  &  \\

                     \vernacular{
                    ashimu[kalúshítsaanga]}  &   
                     \gloss{‘returning’}  &  \\

                     \vernacular{
                    ashimu[siínjílitsaanga]}  &   
                     \gloss{
                    ‘making...stand’}  &  \\

                     \vernacular{
                    ashimu[reébáreebaanga]}  &   
                     \gloss{‘asking
                    (iter)’}  &  \\

                     \vernacular{
                    ashimu[kalúkhányinyaanga]}  &   
                     \gloss{
                    ‘turning...over’}  &  \\

                     \vernacular{
                    abi[sebúlúkhanyinyaanga]}  &   
                     \gloss{‘scattering’}  &  \\
\end{tabular}
%\caption{\nocaption}
     
\begin{tabular}{llllll}  
  \multicolumn{5}{l}{
                     \vernacular{(638) /Ø/
                    V-Initial + OP} \gloss{‘s/he is
                    still...him/her \ob mw-\cb  / it
                    } } &  \\
\multicolumn{5}{l}{ } &  \\

                     \vernacular{
                    ashimw[eenyáanga]}  &   
                     \gloss{‘wanting’}  &  \\

                     \vernacular{
                    ashimw[eeyéláanga]}  &   
                     \gloss{‘lighting’}  &  \\

                     \vernacular{
                    ashibw[iilúulaanga]}  &   
                     \gloss{‘winnowing’}  &  \\

                     \vernacular{
                    ashimw[aambákhánaanga]}  &   
                     \gloss{‘refusing’}  &  \\

                     \vernacular{
                    ashimw[eeléelitsaanga]}  &   
                     \gloss{
                    ‘hanging...up’}  &  \\
\end{tabular}
%\caption{\nocaption}
     
\begin{tabular}{llllll}  
  \multicolumn{5}{l}{
                     \vernacular{(639) /H/
                    C-Initial + OP
                    } \gloss{‘s/he is
                    still...me’} } &  \\
\multicolumn{5}{l}{ } &  \\

                     \vernacular{
                    ashii[rí{\downstep}ítsáángá]}  &   
                     \gloss{‘fearing’}  &  \\

                     \vernacular{
                    ashii[mbé{\downstep}káángá]}  &   
                     \gloss{‘shaving’}  &  \\

                     \vernacular{
                    ashii[ndé{\downstep}éráángá]}  &   
                     \gloss{‘bringing’}  &  \\

                     \vernacular{
                    ashii[khá{\downstep}lákáánga]}  &   
                     \gloss{‘cutting’}  &  \\

                     \vernacular{
                    ashii[sí{\downstep}táákáánga]}  &   
                     \gloss{‘accusing’}  &  \\

                     \vernacular{
                    ashii[mbó{\downstep}ólítsáánga]}  &   
                     \gloss{‘seducing’}  &  \\

                     \vernacular{
                    ashii[khó{\downstep}ng’óóndáánga]}  &   
                     \gloss{‘knocking’}  &  \\

                     \vernacular{
                    ashii[mbó{\downstep}hólóláanga]}  &   
                     \gloss{‘untying’}  &  \\

                     \vernacular{
                    ashii[mbó{\downstep}yóng’ánáanga]}  &   
                     \gloss{‘going
                    around’}  &  \\

                     \vernacular{
                    ashii[ng’ó{\downstep}ng’óólítsáanga]}  &   
                     \gloss{‘teasing’}  &  \\

                     \vernacular{
                    ashii[ní{\downstep}ngá(ká)nyínyáanga]}  &   
                     \gloss{‘bending’}  &  \\
\end{tabular}
%\caption{\nocaption}
     
\begin{tabular}{llllll}  
  \multicolumn{5}{l}{
                     \vernacular{(640) /H/
                    V-Initial + OP
                    } \gloss{‘s/he is
                    still...me’} } &  \\
\multicolumn{5}{l}{ } &  \\

                     \vernacular{
                    ashii[nzí{\downstep}ráángá]}  &   
                     \gloss{‘killing’}  &  \\

                     \vernacular{
                    ashii[nzí{\downstep}kóómbaanga]}  &   
                     \gloss{‘admiring’}  &  \\

                     \vernacular{
                    ashii[nzí{\downstep}síákaanga]}  &   
                     \gloss{‘smacking’}  &  \\

                     \vernacular{
                    ashii[nzó{\downstep}nónyínyáanga]}  &   
                     \gloss{‘spoiling’}  &  \\

                     \vernacular{
                    ashii[nzá{\downstep}búkhányínyaanga]}  &   
                     \gloss{‘separating’}  &  \\
\end{tabular}
%\caption{\nocaption}
     
\begin{tabular}{llllll}  
  \multicolumn{5}{l}{
                     \vernacular{(641) /Ø/
                    C-Initial + OP
                    } \gloss{‘s/he is
                    still...me’} } &  \\
\multicolumn{5}{l}{ } &  \\

                     \vernacular{
                    ashii[siétsáanga]}  &   
                     \gloss{‘grinding’}  &  \\

                     \vernacular{
                    ashii[ndekháanga]}  &   
                     \gloss{‘leaving’}  &  \\

                     \vernacular{
                    ashii[noóndáanga]}  &   
                     \gloss{‘following’}  &  \\

                     \vernacular{
                    ashii[ngulíkháanga]}  &   
                     \gloss{‘naming’}  &  \\

                     \vernacular{
                    ashii[ndakhúulaanga]}  &   
                     \gloss{‘releasing’}  &  \\

                     \vernacular{
                    ashii[seébúlaanga]}  &   
                     \gloss{‘saying bye
                    to’}  &  \\

                     \vernacular{
                    ashii[mboómbélitsaanga]}  &   
                     \gloss{‘comforting’}  &  \\

                     \vernacular{
                    ashii[siínjílitsaanga]}  &   
                     \gloss{
                    ‘making...stand’}  &  \\

                     \vernacular{
                    ashii[ndeébándeebaanga]}  &   
                     \gloss{‘asking
                    (iter)’}  &  \\

                     \vernacular{
                    ashii[ngalúkhányinyaanga]}  &   
                     \gloss{
                    ‘turning...over’}  &  \\
\end{tabular}
%\caption{\nocaption}
     
\begin{tabular}{llllll}  
  \multicolumn{5}{l}{
                     \vernacular{(642) /Ø/
                    V-Initial + OP
                    } \gloss{‘s/he is
                    still...me’} } &  \\
\multicolumn{5}{l}{ } &  \\

                     \vernacular{
                    ashii[nzenyáanga]}  &   
                     \gloss{‘wanting’}  &  \\

                     \vernacular{
                    ashii[nzeyéláanga]}  &   
                     \gloss{‘wiping for’}  &  \\

                     \vernacular{
                    ashii[nyambákhánaanga]}  &   
                     \gloss{‘refusing’}  &  \\

                     \vernacular{
                    ashii[nzeléelitsaanga]}  &   
                     \gloss{
                    ‘carrying...hanging’}  &  \\
\end{tabular}
%\caption{\nocaption}
     
\begin{tabular}{llllll}  
  \multicolumn{5}{l}{
                     \vernacular{(643) /H/
                    C-Initial + OP
                    } \gloss{‘s/he is
                    still...him/herself’} } &  \\
\multicolumn{5}{l}{ } &  \\

                     \vernacular{
                    ashii[ré{\downstep}étsáángá]}  &   
                     \gloss{‘burying’}  &     &   
                     \vernacular{
                    ashii[bé{\downstep}káángá]}  &   
                     \gloss{‘shaving’}  &  \\

                     \vernacular{
                    ashii[sú{\downstep}úngáángá]}  &   
                     \gloss{‘hanging’}  &     &   
                     \vernacular{
                    ashii[khá{\downstep}lákáánga]}  &   
                     \gloss{‘cutting’}  &  \\

                     \vernacular{
                    ashii[sí{\downstep}táákáánga]}  &   
                     \gloss{‘accusing’}  &     &   
                     \vernacular{
                    ashii[sá{\downstep}ándítsáánga]}  &   
                     \gloss{‘thanking’}  &  \\

                     \vernacular{
                    ashii[khó{\downstep}ng’óóndáánga]}  &   
                     \gloss{‘knocking’}  &     &   
                     \vernacular{
                    ashii[bó{\downstep}hólóláanga]}  &   
                     \gloss{‘untying’}  &  \\
\end{tabular}
%\caption{\nocaption}
     
\begin{tabular}{llllll}  
  \multicolumn{5}{l}{
                     \vernacular{(644) /H/
                    V-Initial + OP
                    } \gloss{‘s/he is
                    still...him/herself’} } &  \\
\multicolumn{5}{l}{ } &  \\

                     \vernacular{
                    ashii[yí{\downstep}ráángá]}  &   
                     \gloss{‘killing’}  &     &   
                     \vernacular{
                    ashii[yí{\downstep}kóómbáánga]}  &   
                     \gloss{‘admiring’}  &  \\

                     \vernacular{
                    ashii[yí{\downstep}síákáánga]}  &   
                     \gloss{‘smacking’}  &     &   
                     \vernacular{
                    ashii[yó{\downstep}nónyínyáanga]}  &   
                     \gloss{‘spoiling’}  &  \\

                     \vernacular{
                    ashii[yá{\downstep}búkhányínyaanga]}  &   
                     \gloss{‘separating’}  &     &     &     &  \\
\end{tabular}
%\caption{\nocaption}
     
\begin{tabular}{llllll}  
  \multicolumn{5}{l}{
                     \vernacular{(645) /Ø/
                    C-Initial + OP
                    } \gloss{‘s/he is
                    still...him/herself’} } &  \\
\multicolumn{5}{l}{ } &  \\

                     \vernacular{
                    ashii[siétsaanga]}  &   
                     \gloss{‘grinding’}  &  \\

                     \vernacular{
                    ashii[lekháanga]}  &   
                     \gloss{‘leaving’}  &  \\

                     \vernacular{
                    ashii[siíngáanga]}  &   
                     \gloss{‘bathing’}  &  \\

                     \vernacular{
                    ashii[kulíkháanga]}  &   
                     \gloss{‘naming’}  &  \\

                     \vernacular{
                    ashii[naábúlaanga]}  &   
                     \gloss{‘undressing’}  &  \\

                     \vernacular{
                    ashii[lakhúulaanga]}  &   
                     \gloss{‘releasing’}  &  \\

                     \vernacular{
                    ashii[hoómbélitsaanga]}  &   
                     \gloss{‘comforting’}  &  \\

                     \vernacular{
                    ashii[siínjílitsaanga]}  &   
                     \gloss{
                    ‘making...stand’}  &  \\

                     \vernacular{
                    ashii[reébáreebaanga]}  &   
                     \gloss{‘asking
                    (iter)’}  &  \\

                     \vernacular{
                    ashii[kalúkhányinyaanga]}  &   
                     \gloss{
                    ‘turning...over’}  &  \\
\end{tabular}
%\caption{\nocaption}
     
\begin{tabular}{llllll}  
  \multicolumn{5}{l}{
                     \vernacular{(646) /Ø/
                    V-Initial + OP
                    } \gloss{‘s/he is
                    still...him/herself’} } &  \\
\multicolumn{5}{l}{ } &  \\

                     \vernacular{
                    ashii[yaláanga]}  &   
                     \gloss{‘exposing’}  &     &   
                     \vernacular{
                    ashii[yeyéláanga]}  &   
                     \gloss{‘wiping for’}  &  \\

                     \vernacular{
                    ashii[yambákhánaanga]}  &   
                     \gloss{‘despising’}  &     &   
                     \vernacular{
                    ashii[yeléelitsaanga]}  &   
                     \gloss{‘hanging’}  &  \\
\end{tabular}
%\caption{\nocaption}
     
\begin{tabular}{llllll}  
  \multicolumn{5}{l}{
                     \vernacular{(647) /H/
                    C-Initial + OP + OP
                    } \gloss{‘s/he is
                    still...him/her for me’} } &  \\
\multicolumn{5}{l}{ } &  \\

                     \vernacular{
                    ashimuú[{\downstep}ndééláángá]}  &   
                     \gloss{‘burying’}  &  \\

                     \vernacular{
                    ashimuú[{\downstep}mbéchéláánga]}  &   
                     \gloss{‘shaving’}  &  \\

                     \vernacular{
                    ashimuú[{\downstep}ndééréláánga]}  &   
                     \gloss{‘bringing’}  &  \\

                     \vernacular{
                    ashimuú[{\downstep}kháláchíláanga]}  &   
                     \gloss{‘cutting’}  &  \\

                     \vernacular{
                    ashimuú[{\downstep}sítááchíláanga]}  &   
                     \gloss{‘accusing’}  &  \\

                     \vernacular{
                    ashimuú[{\downstep}mbóólítsíláanga]}  &   
                     \gloss{‘seducing’}  &  \\

                     \vernacular{
                    ashimuú[{\downstep}mbóhólólélaanga]}  &   
                     \gloss{‘untying’}  &  \\
\end{tabular}
%\caption{\nocaption}
     
\begin{tabular}{llllll}  
  \multicolumn{5}{l}{
                     \vernacular{(648) /H/
                    V-Initial + OP + OP
                    } \gloss{‘s/he is
                    still...him/her for me’} } &  \\
\multicolumn{5}{l}{ } &  \\

                     \vernacular{
                    ashimuú[{\downstep}nzíráángá]}  &   
                     \gloss{‘killing’}  &  \\

                     \vernacular{
                    ashimuú[{\downstep}nzéchítsíláanga]}  &   
                     \gloss{‘admiring’}  &  \\

                     \vernacular{
                    ashimuú[{\downstep}nzísíáchíláánga]}  &   
                     \gloss{‘smacking’}  &  \\

                     \vernacular{
                    ashimuú[{\downstep}nzónónyínyílaanga]}  &   
                     \gloss{‘spoiling’}  &  \\

                     \vernacular{
                    ashimuú[{\downstep}nzábúkhányínyilaanga]}  &   
                     \gloss{‘separating’}  &  \\
\end{tabular}
%\caption{\nocaption}
     
\begin{tabular}{llllll}  
  \multicolumn{5}{l}{
                     \vernacular{(649) /Ø/
                    C-Initial + OP + OP
                    } \gloss{‘s/he is
                    still...him/her for me’} } &  \\
\multicolumn{5}{l}{ } &  \\

                     \vernacular{
                    ashimuú[{\downstep}nzííláanga]}  &   
                     \gloss{‘going for’}  &  \\

                     \vernacular{
                    ashimuú[{\downstep}ndéshéláanga]}  &   
                     \gloss{‘leaving’}  &  \\

                     \vernacular{
                    ashimuú[{\downstep}nóóndélaanga]}  &   
                     \gloss{‘following’}  &  \\

                     \vernacular{
                    ashimuú[{\downstep}ngúlíshílaanga]}  &   
                     \gloss{‘naming’}  &  \\

                     \vernacular{
                    ashimuú[{\downstep}ndákhúulilaanga]}  &   
                     \gloss{‘releasing’}  &  \\

                     \vernacular{
                    ashimuú[{\downstep}séébúlilaanga]}  &   
                     \gloss{‘saying bye
                    to’}  &  \\

                     \vernacular{
                    ashimuú[{\downstep}mbóómbélitsilaanga]}  &   
                     \gloss{‘comforting’}  &  \\

                     \vernacular{
                    ashimuú[{\downstep}síínjílitsilaanga]}  &   
                     \gloss{
                    ‘making...stand’}  &  \\
\end{tabular}
%\caption{\nocaption}
     
\begin{tabular}{llllll}  
  \multicolumn{5}{l}{
                     \vernacular{(650) /Ø/
                    V-Initial + OP + OP
                    } \gloss{‘s/he is
                    still...him/her \ob mu-\cb  / it
                    } } &  \\
\multicolumn{5}{l}{ } &  \\

                     \vernacular{
                    ashimuú[{\downstep}nzéyéláanga]}  &   
                     \gloss{‘wiping for’}  &     &   
                     \vernacular{
                    ashikuú[nzáshítsílaanga]}  &   
                     \gloss{‘lighting’}  &  \\

                     \vernacular{
                    ashibuú[{\downstep}nzílúulilaanga]}  &   
                     \gloss{‘winnowing’}  &     &   
                     \vernacular{
                    ashikuú[nzéléelitsilaanga]}  &   
                     \gloss{‘hanging’}  &  \\
\end{tabular}
%\caption{\nocaption}
     
\begin{tabular}{lll}  
  \multicolumn{2}{l}{
                     \vernacular{(651) /H/
                    C-Initial Phrase-Medial} \gloss{‘s/he is
                    still...the boy \ob mú{\downstep}yáyi\cb  /} } &  \\
\multicolumn{2}{l}{
                     \gloss{someone
                    \ob muundu\cb ’} } &  \\

                     \vernacular{ashi[reetsaanga]
                    mú{\downstep}yáyi/muundu}  &   
                     \gloss{‘burying’}  &  \\

                     \vernacular{ashi[bekaanga]
                    mú{\downstep}yáyi/muundu}  &   
                     \gloss{‘shaving’}  &  \\

                     \vernacular{ashi[leeraanga]
                    mú{\downstep}yáyi/muundu}  &   
                     \gloss{‘bringing’}  &  \\

                     \vernacular{ashi[khalakaanga]
                    mú{\downstep}yáyi/muundu}  &   
                     \gloss{‘cutting’}  &  \\

                     \vernacular{ashi[sitaakaanga]
                    mú{\downstep}yáyi/muundu}  &   
                     \gloss{‘accusing’}  &  \\

                     \vernacular{ashi[boolitsaanga]
                    mú{\downstep}yáyi/muundu}  &   
                     \gloss{‘seducing’}  &  \\

                     \vernacular{
                    ashi[khong’oondaanga]
                    mú{\downstep}yáyi/muundu}  &   
                     \gloss{‘sucking’}  &  \\

                     \vernacular{ashi[bohololaanga]
                    mú{\downstep}yáyi/muundu}  &   
                     \gloss{‘untying’}  &  \\

                     \vernacular{
                    ashi[boyong’anaanga]
                    mú{\downstep}yáyi/muundu}  &   
                     \gloss{‘going
                    around’}  &  \\
\end{tabular}
%\caption{\nocaption}
     
\begin{tabular}{lll}  
  \multicolumn{2}{l}{
                     \vernacular{(652) /Ø/
                    C-Initial Phrase-Medial} \gloss{‘s/he is
                    still...the boy \ob mú{\downstep}yáyi\cb  /} } &  \\
\multicolumn{2}{l}{
                     \gloss{someone
                    \ob muundu\cb ’} } &  \\

                     \vernacular{ashi[tsiitsaanga]
                    mú{\downstep}yáyi/muundu}  &   
                     \gloss{‘going for’}  &  \\

                     \vernacular{ashi[lekhaanga]
                    mú{\downstep}yáyi/muundu}  &   
                     \gloss{‘leaving’}  &  \\

                     \vernacular{ashi[loondaanga]
                    mú{\downstep}yáyi/muundu}  &   
                     \gloss{‘following’}  &  \\

                     \vernacular{ashi[kulikhaanga]
                    mú{\downstep}yáyi/muundu}  &   
                     \gloss{‘naming’}  &  \\

                     \vernacular{ashi[lakhuulaanga]
                    mú{\downstep}yáyi/muundu}  &   
                     \gloss{‘releasing’}  &  \\

                     \vernacular{ashi[seebulaanga]
                    mú{\downstep}yáyi/muundu}  &   
                     \gloss{‘saying bye’}  &  \\

                     \vernacular{
                    ashi[kalushitsaanga]
                    mú{\downstep}yáyi/muundu}  &   
                     \gloss{‘returning’}  &  \\

                     \vernacular{
                    ashi[reebareebaanga]
                    mú{\downstep}yáyi/muundu}  &   
                     \gloss{‘asking
                    (iter)’}  &  \\
\end{tabular}
%\caption{\nocaption}
     
\begin{tabular}{lll}  
  \multicolumn{2}{l}{
                     \vernacular{(653) /H/
                    C-Initial +OP Phrase-Medial} \gloss{‘s/he is
                    still...the boy \ob mú{\downstep}yáyi\cb  /} } &  \\
\multicolumn{2}{l}{
                     \gloss{someone \ob muundu\cb 
                    for him/her’} } &  \\

                     \vernacular{ashimu[réelaanga]
                    mú{\downstep}yáyi/muundu}  &   
                     \gloss{‘burying’}  &  \\

                     \vernacular{
                    ashimu[béchelaanga]
                    mú{\downstep}yáyi/muundu}  &   
                     \gloss{‘shaving’}  &  \\

                     \vernacular{
                    ashimu[léerelaanga]
                    mú{\downstep}yáyi/muundu}  &   
                     \gloss{‘bringing’}  &  \\

                     \vernacular{
                    ashimu[khálachilaanga]
                    mú{\downstep}yáyi/muundu}  &   
                     \gloss{‘cutting’}  &  \\

                     \vernacular{
                    ashimu[sítaachilaanga]
                    mú{\downstep}yáyi/muundu}  &   
                     \gloss{‘accusing’}  &  \\

                     \vernacular{
                    ashimu[bóolitsilaanga]
                    mú{\downstep}yáyi/muundu}  &   
                     \gloss{‘seducing’}  &  \\

                     \vernacular{
                    ashimu[khóng’oondelaanga]
                    mú{\downstep}yáyi/muundu}  &   
                     \gloss{‘sucking’}  &  \\

                     \vernacular{
                    ashimu[bóhololelaanga]
                    mú{\downstep}yáyi/muundu}  &   
                     \gloss{‘untying’}  &  \\

                     \vernacular{
                    ashimu[bóyong’anilaanga]
                    mú{\downstep}yáyi/muundu}  &   
                     \gloss{‘going
                    around’}  &  \\
\end{tabular}
%\caption{\nocaption}
     
\begin{tabular}{lll}  
  \multicolumn{2}{l}{
                     \vernacular{(654) /Ø/
                    C-Initial +OP Phrase-Medial} \gloss{‘s/he is
                    still...the boy \ob mú{\downstep}yáyi\cb  /} } &  \\
\multicolumn{2}{l}{
                     \gloss{someone \ob muundu\cb 
                    for him/her’} } &  \\

                     \vernacular{ashimu[tsiilaanga]
                    mú{\downstep}yáyi/muundu}  &   
                     \gloss{‘going for’}  &  \\

                     \vernacular{
                    ashimu[leshelaanga] mú{\downstep}yáyi/muundu}  &   
                     \gloss{‘leaving’}  &  \\

                     \vernacular{
                    ashimu[loondelaanga]
                    mú{\downstep}yáyi/muundu}  &   
                     \gloss{‘following’}  &  \\

                     \vernacular{
                    ashimu[kulishilaanga]
                    mú{\downstep}yáyi/muundu}  &   
                     \gloss{‘naming’}  &  \\

                     \vernacular{
                    ashimu[lakhuulilaanga]
                    mú{\downstep}yáyi/muundu}  &   
                     \gloss{‘releasing’}  &  \\

                     \vernacular{
                    ashimu[seebulilaanga]
                    mú{\downstep}yáyi/muundu}  &   
                     \gloss{‘saying bye
                    to’}  &  \\

                     \vernacular{
                    ashimu[kalushitsilaanga]
                    mú{\downstep}yáyi/muundu}  &   
                     \gloss{‘returning’}  &  \\

                     \vernacular{
                    ashimu[reebareebelaanga]
                    mú{\downstep}yáyi/muundu}  &   
                     \gloss{‘asking
                    (iter)’}  &  \\
\end{tabular}
%\caption{\nocaption}
     
\begin{tabular}{lll}  
  \multicolumn{2}{l}{
                     \vernacular{(655) /H/
                    C-Initial +OP + OP
                    } \gloss{‘s/he is
                    still...the boy \ob mú{\downstep}yáyi\cb  /} } &  \\
\multicolumn{2}{l}{
                     \gloss{someone \ob muundu\cb 
                    for him/her for me’} } &  \\

                     \vernacular{
                    ashimuú[ndeelaanga]
                    mú{\downstep}yáyi/muundu}  &   
                     \gloss{‘buring’}  &  \\

                     \vernacular{
                    ashimuú[mbechelaanga]
                    mú{\downstep}yáyi/muundu}  &   
                     \gloss{‘shaving’}  &  \\

                     \vernacular{
                    ashimuú[ndeerelaanga]
                    mú{\downstep}yáyi/muundu}  &   
                     \gloss{‘bringing’}  &  \\

                     \vernacular{
                    ashimuú[khalachilaanga]
                    mú{\downstep}yáyi/muundu}  &   
                     \gloss{‘cutting’}  &  \\

                     \vernacular{
                    ashimuú[sitaachilaanga]
                    mú{\downstep}yáyi/muundu}  &   
                     \gloss{‘accusing’}  &  \\

                     \vernacular{
                    ashimuú[mboolitsilaanga]
                    mú{\downstep}yáyi/muundu}  &   
                     \gloss{‘seducing’}  &  \\

                     \vernacular{
                    ashimuú[mbohololelaanga]
                    mú{\downstep}yáyi/muundu}  &   
                     \gloss{‘untying’}  &  \\
\end{tabular}
%\caption{\nocaption}
     
\begin{tabular}{lll}  
  \multicolumn{2}{l}{
                     \vernacular{(656) /Ø/
                    C-Initial +OP + OP
                    } \gloss{‘s/he is
                    still...the boy \ob mú{\downstep}yáyi\cb  /} } &  \\
\multicolumn{2}{l}{
                     \gloss{someone \ob muundu\cb 
                    for him/her for me’} } &  \\

                     \vernacular{
                    ashimuú[nziilaanga]
                    mú{\downstep}yáyi/muundu}  &   
                     \gloss{‘going for’}  &  \\

                     \vernacular{
                    ashimuú[ndeshelaanga]
                    mú{\downstep}yáyi/muundu}  &   
                     \gloss{‘leaving’}  &  \\

                     \vernacular{
                    ashimuú[noondelaanga]
                    mú{\downstep}yáyi/muundu}  &   
                     \gloss{‘following’}  &  \\

                     \vernacular{
                    ashimuú[ngulishilaanga]
                    mú{\downstep}yáyi/muundu}  &   
                     \gloss{‘naming’}  &  \\

                     \vernacular{
                    ashimuú[ndakhuulilaanga]
                    mú{\downstep}yáyi/muundu}  &   
                     \gloss{‘releasing’}  &  \\

                     \vernacular{
                    ashimuú[seebulilaanga]
                    mú{\downstep}yáyi/muundu}  &   
                     \gloss{‘saying bye
                    to’}  &  \\
\end{tabular}
%\caption{\nocaption}
    

\subsection{Persistive Negative: Pattern 5a
              }\label{sec:sPersNeg}


\begin{tabular}{lll}  
  \multicolumn{2}{l}{
                     \vernacular{(657) /H/
                    C-Initial} \gloss{‘s/he is not
                    still...’} } &  \\
\multicolumn{2}{l}{ } &  \\

                     \vernacular{
                    ashi[reetsáángá] {\downstep}tá}  &   
                     \gloss{‘burying’}  &  \\

                     \vernacular{
                    ashi[ng’weetsáángá] {\downstep}tá}  &   
                     \gloss{‘drinking’}  &  \\

                     \vernacular{
                    ashi[khweetsáángá] {\downstep}tá}  &   
                     \gloss{‘paying
                    dowry’}  &  \\

                     \vernacular{
                    ashi[liitsáángá] {\downstep}tá}  &   
                     \gloss{‘eating’}  &  \\

                     \vernacular{ashi[lumaángá]
                    {\downstep}tá}  &   
                     \gloss{‘biting’}  &  \\

                     \vernacular{ashi[bekaángá]
                    {\downstep}tá}  &   
                     \gloss{‘shaving’}  &  \\

                     \vernacular{
                    ashi[teekháángá] {\downstep}tá}  &   
                     \gloss{‘cooking’}  &  \\

                     \vernacular{ashi[leeráángá]
                    {\downstep}tá}  &   
                     \gloss{‘bringing’}  &  \\

                     \vernacular{
                    ashi[khalakáá{\downstep}ngá] tá}  &   
                     \gloss{‘cutting’}  &  \\

                     \vernacular{
                    ashi[kalaangáá{\downstep}ngá] tá}  &   
                     \gloss{‘frying’}  &  \\

                     \vernacular{
                    ashi[sitaakáá{\downstep}ngá] tá}  &   
                     \gloss{‘accusing’}  &  \\

                     \vernacular{
                    ashi[boolitsáá{\downstep}ngá] tá}  &   
                     \gloss{‘seducing’}  &  \\

                     \vernacular{
                    ashi[saanditsáá{\downstep}ngá] tá}  &   
                     \gloss{‘thanking’}  &  \\

                     \vernacular{
                    ashi[khong’oondáá{\downstep}ngá] tá}  &   
                     \gloss{‘knocking’}  &  \\

                     \vernacular{
                    ashi[boholólá{\downstep}ángá] tá}  &   
                     \gloss{‘untying’}  &  \\

                     \vernacular{
                    ashi[boyong’áná{\downstep}ángá] tá}  &   
                     \gloss{‘going
                    around’}  &  \\

                     \vernacular{
                    ashi[ng’ong’oolítsá{\downstep}ángá] tá}  &   
                     \gloss{‘teasing’}  &  \\

                     \vernacular{
                    ashi[linga(ka)nyínyá{\downstep}ángá] tá}  &   
                     \gloss{‘crumpling’}  &  \\
\end{tabular}
%\caption{\nocaption}
     
\begin{tabular}{llllll}  
  \multicolumn{5}{l}{
                     \vernacular{(658) /Ø/
                    C-Initial} \gloss{‘s/he is not
                    still...’} } &  \\
\multicolumn{5}{l}{ } &  \\

                     \vernacular{
                    ashi[tsií{\downstep}tsáángá] tá}  &   
                     \gloss{‘going’}  &  \\

                     \vernacular{
                    ashi[kwií{\downstep}tsáángá] tá}  &   
                     \gloss{‘falling’}  &  \\

                     \vernacular{
                    ashi[lekhá{\downstep}ángá] tá}  &   
                     \gloss{‘leaving’}  &  \\

                     \vernacular{
                    ashi[reébá{\downstep}ángá] tá}  &   
                     \gloss{‘asking’}  &  \\

                     \vernacular{
                    ashi[loóndá{\downstep}ángá] tá}  &   
                     \gloss{‘following’}  &  \\

                     \vernacular{
                    ashi[kumílá{\downstep}ángá] tá}  &   
                     \gloss{‘holding’}  &  \\

                     \vernacular{
                    ashi[kulíkhá{\downstep}ángá] tá}  &   
                     \gloss{‘naming’}  &  \\

                     \vernacular{
                    ashi[homó{\downstep}óláángá] tá}  &   
                     \gloss{‘massaging’}  &  \\

                     \vernacular{
                    ashi[lakhú{\downstep}úláángá] tá}  &   
                     \gloss{‘releasing’}  &  \\

                     \vernacular{
                    ashi[seébú{\downstep}láángá] tá}  &   
                     \gloss{‘saying bye’}  &  \\

                     \vernacular{
                    ashi[hoómbé{\downstep}lítsáángá] tá}  &   
                     \gloss{‘comforting’}  &  \\

                     \vernacular{
                    ashi[kalúshí{\downstep}tsáángá] tá}  &   
                     \gloss{‘returning’}  &  \\

                     \vernacular{
                    ashi[siínjí{\downstep}lítsáángá] tá}  &   
                     \gloss{‘making
                    stand’}  &  \\

                     \vernacular{
                    ashi[reébá{\downstep}réébáángá] tá}  &   
                     \gloss{‘asking
                    (iter)’}  &  \\

                     \vernacular{
                    ashi[kalúkhá{\downstep}nyínyáángá] tá}  &   
                     \gloss{‘turning
                    over’}  &  \\

                     \vernacular{
                    ashi[sebúlú{\downstep}khányínyáángá]
                    tá}  &   
                     \gloss{‘scattering’}  &  \\
\end{tabular}
%\caption{\nocaption}
     
\begin{tabular}{llllll}  
  \multicolumn{5}{l}{
                     \vernacular{(659) /H/
                    C-Initial + OP} \gloss{‘s/he is not
                    still...him/her’} } &  \\
\multicolumn{5}{l}{ } &  \\

                     \vernacular{
                    ashimu[ré{\downstep}étsáángá] {\downstep}tá}  &   
                     \gloss{‘burying’}  &  \\

                     \vernacular{
                    ashimu[bé{\downstep}káángá] {\downstep}tá}  &   
                     \gloss{‘shaving’}  &  \\

                     \vernacular{
                    ashimu[lé{\downstep}éráángá] {\downstep}tá}  &   
                     \gloss{‘bringing’}  &  \\

                     \vernacular{
                    ashimu[khá{\downstep}lákáá{\downstep}ngá] tá}  &   
                     \gloss{‘cutting’}  &  \\

                     \vernacular{
                    ashimu[sí{\downstep}táákáá{\downstep}ngá] tá}  &   
                     \gloss{‘accusing’}  &  \\

                     \vernacular{
                    ashimu[bó{\downstep}ólítsáá{\downstep}ngá] tá}  &   
                     \gloss{‘seducing’}  &  \\

                     \vernacular{
                    ashimu[khó{\downstep}ng’óóndáá{\downstep}ngá] tá}  &   
                     \gloss{‘knocking’}  &  \\

                     \vernacular{
                    ashimu[bó{\downstep}hólólá{\downstep}ángá] tá}  &   
                     \gloss{‘untying’}  &  \\

                     \vernacular{
                    ashimu[bó{\downstep}yóng’áná{\downstep}ángá] tá}  &   
                     \gloss{‘going
                    around’}  &  \\

                     \vernacular{
                    ashimu[ng’ó{\downstep}ng’óólítsá{\downstep}ángá]
                    tá}  &   
                     \gloss{‘teasing’}  &  \\

                     \vernacular{
                    ashimu[lí{\downstep}ngá(ka)nyínyá{\downstep}ángá]
                    tá}  &   
                     \gloss{‘bending’}  &  \\
\end{tabular}
%\caption{\nocaption}
     
\begin{tabular}{llllll}  
  \multicolumn{5}{l}{
                     \vernacular{(660) /Ø/
                    C-Initial + OP} \gloss{‘s/he is not
                    still...him/her \ob mu-\cb  / them
                    } } &  \\
\multicolumn{5}{l}{ } &  \\

                     \vernacular{
                    ashimu[tsií{\downstep}tsáángá] tá}  &   
                     \gloss{‘going for’}  &  \\

                     \vernacular{
                    ashimu[lekhá{\downstep}ángá] tá}  &   
                     \gloss{‘leaving’}  &  \\

                     \vernacular{
                    ashimu[loóndá{\downstep}ángá] tá}  &   
                     \gloss{‘following’}  &  \\

                     \vernacular{
                    ashimu[kulíkhá{\downstep}ángá] tá}  &   
                     \gloss{‘naming’}  &  \\

                     \vernacular{
                    ashimu[lakhú{\downstep}úláángá] tá}  &   
                     \gloss{‘releasing’}  &  \\

                     \vernacular{
                    ashimu[seébú{\downstep}láángá] tá}  &   
                     \gloss{‘saying bye
                    to’}  &  \\

                     \vernacular{
                    ashimu[hoómbé{\downstep}lítsáángá] tá}  &   
                     \gloss{‘comforting’}  &  \\

                     \vernacular{
                    ashimu[kalúshí{\downstep}tsáángá] tá}  &   
                     \gloss{‘returning’}  &  \\

                     \vernacular{
                    ashimu[siínjí{\downstep}lítsáángá] tá}  &   
                     \gloss{
                    ‘making...stand’}  &  \\

                     \vernacular{
                    ashimu[reébá{\downstep}réébáángá] tá}  &   
                     \gloss{‘asking
                    (iter)’}  &  \\

                     \vernacular{
                    ashimu[kalúkhá{\downstep}nyínyáángá] tá}  &   
                     \gloss{
                    ‘turning...over’}  &  \\

                     \vernacular{
                    ashibi[sebúlú{\downstep}khányínyáángá]
                    tá}  &   
                     \gloss{‘scattering’}  &  \\
\end{tabular}
%\caption{\nocaption}
     
\begin{tabular}{llllll}  
  \multicolumn{5}{l}{
                     \vernacular{(661) /H/
                    C-Initial + OP + OP
                    } \gloss{‘s/he is not
                    still...him/her for me’} } &  \\
\multicolumn{5}{l}{ } &  \\

                     \vernacular{
                    ashimuú[{\downstep}ndééláángá] {\downstep}tá}  &   
                     \gloss{‘burying’}  &  \\

                     \vernacular{
                    ashimuú[{\downstep}mbéchéláá{\downstep}ngá] tá}  &   
                     \gloss{‘shaving’}  &  \\

                     \vernacular{
                    ashimuú[{\downstep}ndééréláá{\downstep}ngá] tá}  &   
                     \gloss{‘bringing’}  &  \\

                     \vernacular{
                    ashimuú[{\downstep}kháláchílá{\downstep}ángá] tá}  &   
                     \gloss{‘cutting’}  &  \\

                     \vernacular{
                    ashimuú[{\downstep}sítááchílá{\downstep}ángá]
                    tá}  &   
                     \gloss{‘accusing’}  &  \\

                     \vernacular{
                    ashimuú[{\downstep}mbóólítsílá{\downstep}ángá]
                    tá}  &   
                     \gloss{‘seducing’}  &  \\

                     \vernacular{
                    ashimuú[{\downstep}mbóhólólé{\downstep}láángá]
                    tá}  &   
                     \gloss{‘untying’}  &  \\
\end{tabular}
%\caption{\nocaption}
     
\begin{tabular}{llllll}  
  \multicolumn{5}{l}{
                     \vernacular{(662) /Ø/
                    C-Initial + OP + OP
                    } \gloss{‘s/he is not
                    still...him/her for me’} } &  \\
\multicolumn{5}{l}{ } &  \\

                     \vernacular{
                    ashimuú[{\downstep}nzíílá{\downstep}ángá] tá}  &   
                     \gloss{‘going for’}  &  \\

                     \vernacular{
                    ashimuú[{\downstep}ndéshélá{\downstep}ángá] tá}  &   
                     \gloss{‘leaving’}  &  \\

                     \vernacular{
                    ashimuú[{\downstep}nóóndé{\downstep}láángá] tá}  &   
                     \gloss{‘following’}  &  \\

                     \vernacular{
                    ashimuú[{\downstep}ngúlíshí{\downstep}láángá] tá}  &   
                     \gloss{‘naming’}  &  \\

                     \vernacular{
                    ashimuú[{\downstep}ndákhú{\downstep}úlíláángá]
                    tá}  &   
                     \gloss{‘releasing’}  &  \\

                     \vernacular{
                    ashimuú[{\downstep}séébú{\downstep}líláángá] tá}  &   
                     \gloss{‘saying bye
                    to’}  &  \\

                     \vernacular{
                    ashimuú[{\downstep}mbóómbé{\downstep}lítsíláángá]
                    tá}  &   
                     \gloss{‘comforting’}  &  \\

                     \vernacular{
                    ashimuú[{\downstep}síínjí{\downstep}lítsíláángá]
                    tá}  &   
                     \gloss{
                    ‘making...stand’}  &  \\
\end{tabular}
%\caption{\nocaption}
     
\begin{tabular}{lll}  
  \multicolumn{2}{l}{
                     \vernacular{(663) /H/
                    C-Initial Phrase-Medial} \gloss{‘s/he is not
                    still...the boy \ob mú{\downstep}yáyi\cb  /} } &  \\
\multicolumn{2}{l}{
                     \gloss{someone
                    \ob muundu\cb ’} } &  \\

                     \vernacular{ashi[reetsaanga]
                    mú{\downstep}yáyi/muundu tá}  &   
                     \gloss{‘burying’}  &  \\

                     \vernacular{ashi[bekaanga]
                    mú{\downstep}yáyi/muundu tá}  &   
                     \gloss{‘shaving’}  &  \\

                     \vernacular{ashi[leeraanga]
                    mú{\downstep}yáyi/muundu tá}  &   
                     \gloss{‘bringing’}  &  \\

                     \vernacular{ashi[khalakaanga]
                    mú{\downstep}yáyi/muundu tá}  &   
                     \gloss{‘cutting’}  &  \\

                     \vernacular{ashi[sitaakaanga]
                    mú{\downstep}yáyi/muundu tá}  &   
                     \gloss{‘accusing’}  &  \\

                     \vernacular{ashi[boolitsaanga]
                    mú{\downstep}yáyi/muundu tá}  &   
                     \gloss{‘seducing’}  &  \\

                     \vernacular{
                    ashi[khong’oondaanga] mú{\downstep}yáyi/muundu
                    tá}  &   
                     \gloss{‘knocking’}  &  \\

                     \vernacular{ashi[bohololaanga]
                    mú{\downstep}yáyi/muundu tá}  &   
                     \gloss{‘untying’}  &  \\

                     \vernacular{
                    ashi[boyong’anaanga] mú{\downstep}yáyi/muundu
                    tá}  &   
                     \gloss{‘going
                    around’}  &  \\
\end{tabular}
%\caption{\nocaption}
     
\begin{tabular}{lll}  
  \multicolumn{2}{l}{
                     \vernacular{(664) /Ø/
                    C-Initial Phrase-Medial} \gloss{‘s/he is not
                    still...the boy \ob mú{\downstep}yáyi\cb  /} } &  \\
\multicolumn{2}{l}{
                     \gloss{someone
                    \ob muundu\cb ’} } &  \\

                     \vernacular{ashi[tsiitsaanga]
                    mú{\downstep}yáyi/muundu tá}  &   
                     \gloss{‘going for’}  &  \\

                     \vernacular{ashi[lekhaanga]
                    mú{\downstep}yáyi/muundu tá}  &   
                     \gloss{‘leaving’}  &  \\

                     \vernacular{ashi[loondaanga]
                    mú{\downstep}yáyi/muundu tá}  &   
                     \gloss{‘following’}  &  \\

                     \vernacular{ashi[kulikhaanga]
                    mú{\downstep}yáyi/muundu tá}  &   
                     \gloss{‘naming’}  &  \\

                     \vernacular{ashi[lakhuulaanga]
                    mú{\downstep}yáyi/muundu tá}  &   
                     \gloss{‘releasing’}  &  \\

                     \vernacular{ashi[seebulaanga]
                    mú{\downstep}yáyi/muundu tá}  &   
                     \gloss{‘saying bye
                    to’}  &  \\

                     \vernacular{
                    ashi[kalushitsaanga] mú{\downstep}yáyi/muundu
                    tá}  &   
                     \gloss{‘returning’}  &  \\

                     \vernacular{
                    ashi[reebareebaanga] mú{\downstep}yáyi/muundu
                    tá}  &   
                     \gloss{‘asking
                    (iter)’}  &  \\
\end{tabular}
%\caption{\nocaption}
     
\begin{tabular}{lll}  
  \multicolumn{2}{l}{
                     \vernacular{(665) /H/
                    C-Initial +OP Phrase-Medial} \gloss{‘s/he is not
                    still...the boy \ob mú{\downstep}yáyi\cb  /} } &  \\
\multicolumn{2}{l}{
                     \gloss{someone \ob muundu\cb 
                    for him/her’} } &  \\

                     \vernacular{ashimu[réelaanga]
                    mú{\downstep}yáyi/muundu tá}  &   
                     \gloss{‘burying’}  &  \\

                     \vernacular{
                    ashimu[béchelaanga] mú{\downstep}yáyi/muundu
                    tá}  &   
                     \gloss{‘shaving’}  &  \\

                     \vernacular{
                    ashimu[léerelaanga] mú{\downstep}yáyi/muundu
                    tá}  &   
                     \gloss{‘bringing’}  &  \\

                     \vernacular{
                    ashimu[khálachilaanga] mú{\downstep}yáyi/muundu
                    tá}  &   
                     \gloss{‘cutting’}  &  \\

                     \vernacular{
                    ashimu[sítaachilaanga] mú{\downstep}yáyi/muundu
                    tá}  &   
                     \gloss{‘accusing’}  &  \\

                     \vernacular{
                    ashimu[bóolitsilaanga] mú{\downstep}yáyi/muundu
                    tá}  &   
                     \gloss{‘seducing’}  &  \\

                     \vernacular{
                    ashimu[khóng’oondelaanga] mú{\downstep}yáyi/muundu
                    tá}  &   
                     \gloss{‘knocking’}  &  \\

                     \vernacular{
                    ashimu[bóhololelaanga] mú{\downstep}yáyi/muundu
                    tá}  &   
                     \gloss{‘untying’}  &  \\

                     \vernacular{
                    ashimu[bóyong’anilaanga] mú{\downstep}yáyi/muundu
                    tá}  &   
                     \gloss{‘going
                    around’}  &  \\
\end{tabular}
%\caption{\nocaption}
     
\begin{tabular}{lll}  
  \multicolumn{2}{l}{
                     \vernacular{(666) /Ø/
                    C-Initial +OP Phrase-Medial} \gloss{‘s/he is not
                    still...the boy \ob mú{\downstep}yáyi\cb  /} } &  \\
\multicolumn{2}{l}{
                     \gloss{someone \ob muundu\cb 
                    for him/her’} } &  \\

                     \vernacular{
                    ashimu[tsiitsaanga] mú{\downstep}yáyi/muundu
                    tá}  &   
                     \gloss{‘going for’}  &  \\

                     \vernacular{ashimu[lekhaanga]
                    mú{\downstep}yáyi/muundu tá}  &   
                     \gloss{‘leaving’}  &  \\

                     \vernacular{ashimu[loondaanga]
                    mú{\downstep}yáyi/muundu tá}  &   
                     \gloss{‘following’}  &  \\

                     \vernacular{
                    ashimu[kulikhaanga] mú{\downstep}yáyi/muundu
                    tá}  &   
                     \gloss{‘naming’}  &  \\

                     \vernacular{
                    ashimu[lakhuulaanga] mú{\downstep}yáyi/muundu
                    tá}  &   
                     \gloss{‘releasing’}  &  \\

                     \vernacular{
                    ashimu[seebulaanga] mú{\downstep}yáyi/muundu
                    tá}  &   
                     \gloss{‘saying bye
                    to’}  &  \\

                     \vernacular{
                    ashimu[kalushitsaanga] mú{\downstep}yáyi/muundu
                    tá}  &   
                     \gloss{‘returning’}  &  \\

                     \vernacular{
                    ashimu[reebareebaanga] mú{\downstep}yáyi/muundu
                    tá}  &   
                     \gloss{‘asking
                    (iter)’}  &  \\
\end{tabular}
%\caption{\nocaption}
     
\begin{tabular}{lll}  
  \multicolumn{2}{l}{
                     \vernacular{(667) /H/
                    C-Initial +OP + OP
                    } \gloss{‘s/he is not
                    still...the boy \ob mú{\downstep}yáyi\cb  /} } &  \\
\multicolumn{2}{l}{
                     \gloss{someone \ob muundu\cb 
                    for him/her for me’} } &  \\

                     \vernacular{
                    ashimuú[ndeelaanga] mú{\downstep}yáyi/muundu
                    tá}  &   
                     \gloss{‘buring’}  &  \\

                     \vernacular{
                    ashimuú[mbechelaanga] mú{\downstep}yáyi/muundu
                    tá}  &   
                     \gloss{‘shaving’}  &  \\

                     \vernacular{
                    ashimuú[ndeerelaanga] mú{\downstep}yáyi/muundu
                    tá}  &   
                     \gloss{‘bringing’}  &  \\

                     \vernacular{
                    ashimuú[khalachilaanga] mú{\downstep}yáyi/muundu
                    tá}  &   
                     \gloss{‘cutting’}  &  \\

                     \vernacular{
                    ashimuú[sitaachilaanga] mú{\downstep}yáyi/muundu
                    tá}  &   
                     \gloss{‘accusing’}  &  \\

                     \vernacular{
                    ashimuú[mboolitsilaanga] mú{\downstep}yáyi/muundu
                    tá}  &   
                     \gloss{‘seducing’}  &  \\

                     \vernacular{
                    ashimuú[mbohololelaanga] mú{\downstep}yáyi/muundu
                    tá}  &   
                     \gloss{‘untying’}  &  \\
\end{tabular}
%\caption{\nocaption}
     
\begin{tabular}{lll}  
  \multicolumn{2}{l}{
                     \vernacular{(668) /Ø/
                    C-Initial +OP + OP
                    } \gloss{‘s/he is not
                    still...the boy \ob mú{\downstep}yáyi\cb  /} } &  \\
\multicolumn{2}{l}{
                     \gloss{someone \ob muundu\cb 
                    for him/her for me’} } &  \\

                     \vernacular{
                    ashimuú[nziilaanga] mú{\downstep}yáyi/muundu
                    tá}  &   
                     \gloss{‘going for’}  &  \\

                     \vernacular{
                    ashimuú[ndeshelaanga] mú{\downstep}yáyi/muundu
                    tá}  &   
                     \gloss{‘leaving’}  &  \\

                     \vernacular{
                    ashimuú[noondelaanga] mú{\downstep}yáyi/muundu
                    tá}  &   
                     \gloss{‘following’}  &  \\

                     \vernacular{
                    ashimuú[ngulishilaanga] mú{\downstep}yáyi/muundu
                    tá}  &   
                     \gloss{‘naming’}  &  \\

                     \vernacular{
                    ashimuú[ndakhuulilaanga] mú{\downstep}yáyi/muundu
                    tá}  &   
                     \gloss{‘releasing’}  &  \\

                     \vernacular{
                    ashimuú[seebulilaanga] mú{\downstep}yáyi/muundu
                    tá}  &   
                     \gloss{‘saying bye
                    to’}  &  \\
\end{tabular}
%\caption{\nocaption}
    

\subsection{Habitual: Pattern 8}\label{sec:sHabit}


\begin{tabular}{llllll}  
  \multicolumn{5}{l}{
                     \vernacular{(669) /H/
                    C-Initial} \gloss{‘s/he is
                    ever/always...’} } &  \\
\multicolumn{5}{l}{ } &  \\

                     \vernacular{
                    yaá[{\downstep}rá]}  &   
                     \gloss{‘burying’}  &     &   
                     \vernacular{
                    yaá[{\downstep}ng’wá]}  &   
                     \gloss{‘drinking’}  &  \\

                     \vernacular{
                    yaá[{\downstep}líá]}  &   
                     \gloss{‘eating’}  &     &   
                     \vernacular{
                    yaá[{\downstep}lúmá]}  &   
                     \gloss{‘biting’}  &  \\

                     \vernacular{
                    yaá[{\downstep}béká]}  &   
                     \gloss{‘shaving’}  &     &   
                     \vernacular{
                    yaá[{\downstep}téékhá]}  &   
                     \gloss{‘cooking’}  &  \\

                     \vernacular{
                    yaá[{\downstep}léérá]}  &   
                     \gloss{‘bringing’}  &     &   
                     \vernacular{
                    yaá[{\downstep}kháláká]}  &   
                     \gloss{‘cutting’}  &  \\

                     \vernacular{
                    yaá[{\downstep}káláángá]}  &   
                     \gloss{‘frying’}  &     &   
                     \vernacular{
                    yaá[{\downstep}sítááká]}  &   
                     \gloss{‘accusing’}  &  \\

                     \vernacular{
                    yaá[{\downstep}bóólítsá]}  &   
                     \gloss{‘seducing’}  &     &   
                     \vernacular{
                    yaá[{\downstep}sáándítsá]}  &   
                     \gloss{‘thanking’}  &  \\

                     \vernacular{
                    yaá[{\downstep}tsúúnzúúná]}  &   
                     \gloss{‘sucking’}  &     &   
                     \vernacular{
                    yaá[{\downstep}bóhólólá]}  &   
                     \gloss{‘untying’}  &  \\

                     \vernacular{
                    yaá[{\downstep}bóyóng’áná]}  &   
                     \gloss{‘going
                    around’}  &     &   
                     \vernacular{
                    yaá[{\downstep}ng’óng’óólítsá]}  &   
                     \gloss{‘teasing’}  &  \\

                     \vernacular{
                    yaá[{\downstep}língákányínyá]}  &   
                     \gloss{‘crumpling’}  &  \\
\end{tabular}
%\caption{\nocaption}
     
\begin{tabular}{llllll}  
  \multicolumn{5}{l}{
                     \vernacular{(670) /H/
                    V-Initial} \gloss{‘s/he is
                    ever/always...’} } &  \\
\multicolumn{5}{l}{ } &  \\

                     \vernacular{
                    yaá[{\downstep}yírá]}  &   
                     \gloss{‘killing’}  &     &   
                     \vernacular{
                    yaá[{\downstep}yíkóómbá]}  &   
                     \gloss{‘admiring’}  &  \\

                     \vernacular{
                    yaá[{\downstep}yísíáká]}  &   
                     \gloss{‘smacking’}  &     &   
                     \vernacular{
                    yaá[{\downstep}yíkóbólá]}  &   
                     \gloss{‘belching’}  &  \\

                     \vernacular{
                    yaá[{\downstep}yónónyínyá]}  &   
                     \gloss{‘spoiling’}  &     &   
                     \vernacular{
                    yaá[{\downstep}yábúkhányínyá]}  &   
                     \gloss{‘separating’}  &  \\
\end{tabular}
%\caption{\nocaption}
     
\begin{tabular}{llllll}  
  \multicolumn{5}{l}{
                     \vernacular{(671) /Ø/
                    C-Initial} \gloss{‘s/he is
                    ever/always...’} } &  \\
\multicolumn{5}{l}{ } &  \\

                     \vernacular{
                    yaá[{\downstep}tsíá]}  &   
                     \gloss{‘going’}  &     &   
                     \vernacular{
                    yaá[{\downstep}kwá]}  &   
                     \gloss{‘falling’}  &  \\

                     \vernacular{
                    yaá[{\downstep}lékhá]}  &   
                     \gloss{‘leaving’}  &     &   
                     \vernacular{
                    yaá[{\downstep}réébá]}  &   
                     \gloss{‘asking’}  &  \\

                     \vernacular{
                    yaá[{\downstep}lóóndá]}  &   
                     \gloss{‘following’}  &     &   
                     \vernacular{
                    yaá[{\downstep}kúmílá]}  &   
                     \gloss{‘holding’}  &  \\

                     \vernacular{
                    yaá[{\downstep}kúlíkhá]}  &   
                     \gloss{‘naming’}  &     &   
                     \vernacular{
                    yaá[{\downstep}hómóólá]}  &   
                     \gloss{‘massaging’}  &  \\

                     \vernacular{
                    yaá[{\downstep}lákhúúlá]}  &   
                     \gloss{‘releasing’}  &     &   
                     \vernacular{
                    yaá[{\downstep}séébúlá]}  &   
                     \gloss{‘saying bye’}  &  \\

                     \vernacular{
                    yaá[{\downstep}hóómbélítsá]}  &   
                     \gloss{‘comforting’}  &     &   
                     \vernacular{
                    yaá[{\downstep}kálúshítsá]}  &   
                     \gloss{‘returning’}  &  \\

                     \vernacular{
                    yaá[{\downstep}síínjílítsá]}  &   
                     \gloss{‘making
                    stand’}  &     &   
                     \vernacular{
                    yaá[{\downstep}réébáréébá]}  &   
                     \gloss{‘asking
                    (iter)’}  &  \\

                     \vernacular{
                    yaá[{\downstep}kálúkhányínyá]}  &   
                     \gloss{‘turning
                    over’}  &     &   
                     \vernacular{
                    yaá[{\downstep}sébúlúkhányínyá]}  &   
                     \gloss{‘scattering’}  &  \\
\end{tabular}
%\caption{\nocaption}
     
\begin{tabular}{llllll}  
  \multicolumn{5}{l}{
                     \vernacular{(672) /Ø/
                    V-Initial} \gloss{‘s/he is
                    ever/always...’} } &  \\
\multicolumn{5}{l}{ } &  \\

                     \vernacular{
                    yaá[{\downstep}yényá]}  &   
                     \gloss{‘wanting’}  &     &   
                     \vernacular{
                    yaá[{\downstep}yéyélá]}  &   
                     \gloss{‘wiping for’}  &  \\

                     \vernacular{
                    yaá[{\downstep}yílúúlá]}  &   
                     \gloss{‘winnowing’}  &     &   
                     \vernacular{
                    yaá[{\downstep}yámbákháná]}  &   
                     \gloss{‘refusing’}  &  \\

                     \vernacular{
                    yaá[{\downstep}yéléélítsá]}  &   
                     \gloss{‘hanging up’}  &     &   
                     \vernacular{
                    yaá[{\downstep}yíkóómbá]}  &   
                     \gloss{‘admiring’}  &  \\
\end{tabular}
%\caption{\nocaption}
     
\begin{tabular}{llllll}  
  \multicolumn{5}{l}{
                     \vernacular{(673) /H/
                    C-Initial + OP} \gloss{‘s/he is
                    ever/always...him/her’} } &  \\
\multicolumn{5}{l}{ } &  \\

                     \vernacular{
                    yaá{\downstep}mú[rá]}  &   
                     \gloss{‘burying’}  &     &   
                     \vernacular{
                    yaá{\downstep}mú[khúa]}  &   
                     \gloss{‘paying
                    dowry’}  &  \\

                     \vernacular{
                    yaá{\downstep}mú[bé{\downstep}ká]}  &   
                     \gloss{‘shaving’}  &     &   
                     \vernacular{
                    yaá{\downstep}mú[lé{\downstep}érá]}  &   
                     \gloss{‘bringing’}  &  \\

                     \vernacular{
                    yaá{\downstep}mú[khá{\downstep}láká]}  &   
                     \gloss{‘cutting’}  &     &   
                     \vernacular{
                    yaá{\downstep}mú[sí{\downstep}tááká]}  &   
                     \gloss{‘accusing’}  &  \\

                     \vernacular{
                    yaá{\downstep}mú[bó{\downstep}ólítsá]}  &   
                     \gloss{‘seducing’}  &     &   
                     \vernacular{
                    yaá{\downstep}mú[khó{\downstep}ng’óóndá]}  &   
                     \gloss{‘knocking’}  &  \\

                     \vernacular{
                    yaá{\downstep}mú[bó{\downstep}hólólá]}  &   
                     \gloss{‘untying’}  &     &   
                     \vernacular{
                    yaá{\downstep}mú[bó{\downstep}yóng’áná]}  &   
                     \gloss{‘going
                    around’}  &  \\

                     \vernacular{
                    yaá{\downstep}mú[ng’ó{\downstep}ng’óólítsá]}  &   
                     \gloss{‘teasing’}  &     &   
                     \vernacular{
                    yaá{\downstep}mú[lí{\downstep}ngákányínyá]}  &   
                     \gloss{‘bending’}  &  \\
\end{tabular}
%\caption{\nocaption}
     
\begin{tabular}{llllll}  
  \multicolumn{5}{l}{
                     \vernacular{(674) /H/
                    V-Initial + OP} \gloss{‘s/he is
                    ever/always...him/her’} } &  \\
\multicolumn{5}{l}{ } &  \\

                     \vernacular{
                    yaá{\downstep}mw[íí{\downstep}rá]}  &   
                     \gloss{‘killing’}  &     &   
                     \vernacular{
                    yaá{\downstep}mw[íí{\downstep}kóómbá]}  &   
                     \gloss{‘admiring’}  &  \\

                     \vernacular{
                    yaá{\downstep}mw[íí{\downstep}síáká]}  &   
                     \gloss{‘smacking’}  &     &   
                     \vernacular{
                    yaá{\downstep}mw[óó{\downstep}nónyínyá]}  &   
                     \gloss{‘spoiling’}  &  \\

                     \vernacular{
                    yaá{\downstep}mw[áá{\downstep}búkhányínyá]}  &   
                     \gloss{‘separating’}  &  \\
\end{tabular}
%\caption{\nocaption}
     
\begin{tabular}{llllll}  
  \multicolumn{5}{l}{
                     \vernacular{(675) /Ø/
                    C-Initial + OP} \gloss{‘s/he is
                    ever/always...him/her \ob mu-\cb ’} } &  \\
\multicolumn{5}{l}{ } &  \\

                     \vernacular{
                    yaá{\downstep}mú[tsía]}  &   
                     \gloss{‘going for’}  &  \\

                     \vernacular{
                    yaá{\downstep}mú[lékhá]}  &   
                     \gloss{‘leaving’}  &  \\

                     \vernacular{
                    yaá{\downstep}mú[lóóndá]}  &   
                     \gloss{‘following’}  &  \\

                     \vernacular{
                    yaá{\downstep}mú[kúlíkhá]}  &   
                     \gloss{‘naming’}  &  \\

                     \vernacular{
                    yaá{\downstep}mú[lákhúúlá]}  &   
                     \gloss{‘releasing’}  &  \\

                     \vernacular{
                    yaá{\downstep}mú[séébúlá]}  &   
                     \gloss{‘saying bye
                    to’}  &  \\

                     \vernacular{
                    yaá{\downstep}mú[hóómbélítsá]}  &   
                     \gloss{‘comforting’}  &  \\

                     \vernacular{
                    yaá{\downstep}mú[kálúshítsá]}  &   
                     \gloss{‘returning’}  &  \\

                     \vernacular{
                    yaá{\downstep}mú[síínjílítsá]}  &   
                     \gloss{
                    ‘making...stand’}  &  \\

                     \vernacular{
                    yaá{\downstep}mú[réébáréébá]}  &   
                     \gloss{‘asking
                    (iter)’}  &  \\

                     \vernacular{
                    yaá{\downstep}mú[kálúkhányínyá]}  &   
                     \gloss{
                    ‘turning...over’}  &  \\

                     \vernacular{
                    yaá{\downstep}mú[sébúlúkhányínyá]}  &   
                     \gloss{‘scattering’}  &  \\
\end{tabular}
%\caption{\nocaption}
     
\begin{tabular}{llllll}  
  \multicolumn{5}{l}{
                     \vernacular{(676) /Ø/
                    V-Initial + OP} \gloss{‘s/he is
                    ever/always...him/her \ob mw-\cb  / it
                    } } &  \\
\multicolumn{5}{l}{ } &  \\

                     \vernacular{
                    yaá{\downstep}mw[éényá]}  &   
                     \gloss{‘wanting’}  &     &   
                     \vernacular{
                    yaá{\downstep}mw[ééyélá]}  &   
                     \gloss{‘wiping for’}  &  \\

                     \vernacular{
                    yaá{\downstep}bw[íílúúlá]}  &   
                     \gloss{‘winnowing’}  &     &   
                     \vernacular{
                    yaá{\downstep}mw[áámbákháná]}  &   
                     \gloss{‘refusing’}  &  \\

                     \vernacular{
                    yaá{\downstep}mw[ééléélítsá]}  &   
                     \gloss{
                    ‘carrying...hanging’}  &  \\
\end{tabular}
%\caption{\nocaption}
     
\begin{tabular}{llllll}  
  \multicolumn{5}{l}{
                     \vernacular{(677) /H/
                    C-Initial + OP
                    } \gloss{‘s/he is
                    ever/always...me’} } &  \\
\multicolumn{5}{l}{ } &  \\

                     \vernacular{
                    yaá{\downstep}á[rí{\downstep}á]}  &   
                     \gloss{‘fearing’}  &     &   
                     \vernacular{
                    yaá{\downstep}á[khwá]}  &   
                     \gloss{‘paying
                    dowry’}  &  \\

                     \vernacular{
                    yaá{\downstep}á[mbé{\downstep}ká]}  &   
                     \gloss{‘shaving’}  &     &   
                     \vernacular{
                    yaá{\downstep}á[ndé{\downstep}érá]}  &   
                     \gloss{‘bringing’}  &  \\

                     \vernacular{
                    yaá{\downstep}á[khá{\downstep}láká]}  &   
                     \gloss{‘cutting’}  &     &   
                     \vernacular{
                    yaá{\downstep}á[sí{\downstep}tááká]}  &   
                     \gloss{‘accusing’}  &  \\

                     \vernacular{
                    yaá{\downstep}á[mbó{\downstep}ólítsá]}  &   
                     \gloss{‘seducing’}  &     &   
                     \vernacular{
                    yaá{\downstep}á[khó{\downstep}ng’óóndá]}  &   
                     \gloss{‘knocking’}  &  \\

                     \vernacular{
                    yaá{\downstep}á[mbó{\downstep}hólólá]}  &   
                     \gloss{‘untying’}  &     &   
                     \vernacular{
                    yaá{\downstep}á[mbó{\downstep}yóng’áná]}  &   
                     \gloss{‘going
                    around’}  &  \\

                     \vernacular{
                    yaá{\downstep}á[ng’ó{\downstep}ng’óólítsá]}  &   
                     \gloss{‘teasing’}  &     &   
                     \vernacular{
                    yaá{\downstep}á[ní{\downstep}ngákányínyá]}  &   
                     \gloss{‘bending’}  &  \\
\end{tabular}
%\caption{\nocaption}
     
\begin{tabular}{llllll}  
  \multicolumn{5}{l}{
                     \vernacular{(678) /H/
                    V-Initial + OP
                    } \gloss{‘s/he is
                    ever/always...me’} } &  \\
\multicolumn{5}{l}{ } &  \\

                     \vernacular{
                    yaá{\downstep}á[nzí{\downstep}rá]}  &   
                     \gloss{‘killing’}  &     &   
                     \vernacular{
                    yaá{\downstep}á[nzí{\downstep}kóómbá]}  &   
                     \gloss{‘admiring’}  &  \\

                     \vernacular{
                    yaá{\downstep}á[nzí{\downstep}síáká]}  &   
                     \gloss{‘smacking’}  &     &   
                     \vernacular{
                    yaá{\downstep}á[nzó{\downstep}nónyínyá]}  &   
                     \gloss{‘spoiling’}  &  \\

                     \vernacular{
                    yaá{\downstep}á[nzá{\downstep}búkhányínyá]}  &   
                     \gloss{‘separating’}  &  \\
\end{tabular}
%\caption{\nocaption}
     
\begin{tabular}{llllll}  
  \multicolumn{5}{l}{
                     \vernacular{(679) /Ø/
                    C-Initial + OP
                    } \gloss{‘s/he is
                    ever/always...me’} } &  \\
\multicolumn{5}{l}{ } &  \\

                     \vernacular{
                    yaá{\downstep}á[síá]}  &   
                     \gloss{‘grinding’}  &     &   
                     \vernacular{
                    yaá{\downstep}á[ndékhá]}  &   
                     \gloss{‘leaving’}  &  \\

                     \vernacular{
                    yaá{\downstep}á[nóóndá]}  &   
                     \gloss{‘following’}  &     &   
                     \vernacular{
                    yaá{\downstep}á[ngúlíkhá]}  &   
                     \gloss{‘naming’}  &  \\

                     \vernacular{
                    yaá{\downstep}á[ndákhúúlá]}  &   
                     \gloss{‘releasing’}  &     &   
                     \vernacular{
                    yaá{\downstep}á[séébúlá]}  &   
                     \gloss{‘saying bye
                    to’}  &  \\

                     \vernacular{
                    yaá{\downstep}á[mbóómbélítsá]}  &   
                     \gloss{‘comforting’}  &     &   
                     \vernacular{
                    yaá{\downstep}á[síínjílítsá]}  &   
                     \gloss{
                    ‘making..stand’}  &  \\

                     \vernacular{
                    yaá{\downstep}á[ndéébándéébá]}  &   
                     \gloss{‘asking
                    (iter)’}  &     &   
                     \vernacular{
                    yaá{\downstep}á[ngálúkhányínyá]}  &   
                     \gloss{
                    ‘turning...over’}  &  \\
\end{tabular}
%\caption{\nocaption}
     
\begin{tabular}{llllll}  
  \multicolumn{5}{l}{
                     \vernacular{(680) /Ø/
                    V-Initial + OP
                    } \gloss{‘s/he is
                    ever/always...me’} } &  \\
\multicolumn{5}{l}{ } &  \\

                     \vernacular{
                    yaá{\downstep}á[nzényá]}  &   
                     \gloss{‘wanting’}  &     &   
                     \vernacular{
                    yaá{\downstep}á[nzéyélá]}  &   
                     \gloss{‘wiping for’}  &  \\

                     \vernacular{
                    yaá{\downstep}á[nyámbákháná]}  &   
                     \gloss{‘refusing’}  &     &   
                     \vernacular{
                    yaá{\downstep}á[nzéléélítsá]}  &   
                     \gloss{
                    ‘carrying...hanging’}  &  \\
\end{tabular}
%\caption{\nocaption}
     
\begin{tabular}{llllll}  
  \multicolumn{5}{l}{
                     \vernacular{(681) /H/
                    C-Initial + OP
                    } \gloss{‘s/he is
                    ever/always...him/herself’} } &  \\
\multicolumn{5}{l}{ } &  \\

                     \vernacular{
                    yaá{\downstep}yí[rá]}  &   
                     \gloss{‘killing’}  &     &   
                     \vernacular{
                    yaá{\downstep}yí[khwá]}  &   
                     \gloss{‘paying
                    dowry’}  &  \\

                     \vernacular{
                    yaá{\downstep}yí[bé{\downstep}ká]}  &   
                     \gloss{‘shaving’}  &     &   
                     \vernacular{
                    yaá{\downstep}yí[sú{\downstep}úngá]}  &   
                     \gloss{‘hanging’}  &  \\

                     \vernacular{
                    yaá{\downstep}yí[khá{\downstep}láká]}  &   
                     \gloss{‘cutting’}  &     &   
                     \vernacular{
                    yaá{\downstep}yí[sí{\downstep}tááká]}  &   
                     \gloss{‘accusing’}  &  \\

                     \vernacular{
                    yaá{\downstep}yí[sá{\downstep}ándítsá]}  &   
                     \gloss{‘thanking’}  &     &   
                     \vernacular{
                    yaá{\downstep}yí[khó{\downstep}ng’óóndá]}  &   
                     \gloss{‘knocking’}  &  \\

                     \vernacular{
                    yaá{\downstep}yí[bó{\downstep}hólólá]}  &   
                     \gloss{‘untying’}  &  \\
\end{tabular}
%\caption{\nocaption}
     
\begin{tabular}{llllll}  
  \multicolumn{5}{l}{
                     \vernacular{(682) /H/
                    V-Initial + OP
                    } \gloss{‘s/he is
                    ever/always...him/herself’} } &  \\
\multicolumn{5}{l}{ } &  \\

                     \vernacular{
                    yaá{\downstep}yí[yí{\downstep}rá]}  &   
                     \gloss{‘killing’}  &     &   
                     \vernacular{
                    yaá{\downstep}yí[yí{\downstep}kóómbá]}  &   
                     \gloss{‘admiring’}  &  \\

                     \vernacular{
                    yaá{\downstep}yí[yí{\downstep}síáká]}  &   
                     \gloss{‘smacking’}  &     &   
                     \vernacular{
                    yaá{\downstep}yí[yó{\downstep}nónyínyá]}  &   
                     \gloss{‘spoiling’}  &  \\

                     \vernacular{
                    yaá{\downstep}yí[yá{\downstep}búkhányínyá]}  &   
                     \gloss{‘separating’}  &  \\
\end{tabular}
%\caption{\nocaption}
     
\begin{tabular}{llllll}  
  \multicolumn{5}{l}{
                     \vernacular{(683) /Ø/
                    C-Initial + OP
                    } \gloss{‘s/he is
                    ever/always...him/herself’} } &  \\
\multicolumn{5}{l}{ } &  \\

                     \vernacular{
                    yaá{\downstep}yí[sía]}  &   
                     \gloss{‘grinding’}  &     &   
                     \vernacular{
                    yaá{\downstep}yí[lékhá]}  &   
                     \gloss{‘leaving’}  &  \\

                     \vernacular{
                    yaá{\downstep}yí[sííngá]}  &   
                     \gloss{‘bathing’}  &     &   
                     \vernacular{
                    yaá{\downstep}yí[kúlíkhá]}  &   
                     \gloss{‘naming’}  &  \\

                     \vernacular{
                    yaá{\downstep}yí[náábúlá]}  &   
                     \gloss{‘undressing’}  &     &   
                     \vernacular{
                    yaá{\downstep}yí[lákhúúlá]}  &   
                     \gloss{‘releasing’}  &  \\

                     \vernacular{
                    yaá{\downstep}yí[hóómbélítsá]}  &   
                     \gloss{‘comforting’}  &     &   
                     \vernacular{
                    yaá{\downstep}yí[síínjílítsá]}  &   
                     \gloss{
                    ‘making...stand’}  &  \\

                     \vernacular{
                    yaá{\downstep}yí[réébáréébá]}  &   
                     \gloss{‘asking
                    (iter)’}  &     &   
                     \vernacular{
                    yaá{\downstep}yí[kálúkhányínyá]}  &   
                     \gloss{
                    ‘turning...over’}  &  \\
\end{tabular}
%\caption{\nocaption}
     
\begin{tabular}{llllll}  
  \multicolumn{5}{l}{
                     \vernacular{(684) /Ø/
                    V-Initial + OP
                    } \gloss{‘s/he is
                    ever/always...him/herself’} } &  \\
\multicolumn{5}{l}{ } &  \\

                     \vernacular{
                    yaá{\downstep}yí[yálá]}  &   
                     \gloss{‘exposing’}  &     &   
                     \vernacular{
                    yaá{\downstep}yí[yéyélá]}  &   
                     \gloss{‘wiping for’}  &  \\

                     \vernacular{
                    yaá{\downstep}yí[yámbákháná]}  &   
                     \gloss{‘refusing’}  &     &   
                     \vernacular{
                    yaá{\downstep}yí[yéléélítsá]}  &   
                     \gloss{
                    ‘hanging...up’}  &  \\
\end{tabular}
%\caption{\nocaption}
     
\begin{tabular}{llllll}  
  \multicolumn{5}{l}{
                     \vernacular{(685) /H/
                    C-Initial + OP + OP
                    } \gloss{‘s/he is
                    ever/always...him/her for me’} } &  \\
\multicolumn{5}{l}{ } &  \\

                     \vernacular{
                    yaá{\downstep}múú[{\downstep}ndéélá]}  &   
                     \gloss{‘burying’}  &     &   
                     \vernacular{
                    yaá{\downstep}múú[{\downstep}mbéchélá]}  &   
                     \gloss{‘shaving’}  &  \\

                     \vernacular{
                    yaá{\downstep}múú[{\downstep}ndéérélá]}  &   
                     \gloss{‘bringing’}  &     &   
                     \vernacular{
                    yaá{\downstep}múú[{\downstep}kháláchílá]}  &   
                     \gloss{‘cutting’}  &  \\

                     \vernacular{
                    yaá{\downstep}múú[{\downstep}sítááchílá]}  &   
                     \gloss{‘accusing’}  &     &   
                     \vernacular{
                    yaá{\downstep}múú[{\downstep}mbóólítsílá]}  &   
                     \gloss{‘seducing’}  &  \\

                     \vernacular{
                    yaá{\downstep}múú[{\downstep}mbóhólólélá]}  &   
                     \gloss{‘untying’}  &     &     &     &  \\
\end{tabular}
%\caption{\nocaption}
     
\begin{tabular}{llllll}  
  \multicolumn{5}{l}{
                     \vernacular{(686) /H/
                    V-Initial + OP + OP
                    } \gloss{‘s/he is
                    ever/always...him/her for me’} } &  \\
\multicolumn{5}{l}{ } &  \\

                     \vernacular{
                    yaá{\downstep}múú[{\downstep}nzírílá]}  &   
                     \gloss{‘killing’}  &  \\

                     \vernacular{
                    yaá{\downstep}múú[{\downstep}nzéchítsílá]}  &   
                     \gloss{‘admiring’}  &  \\

                     \vernacular{
                    yaá{\downstep}múú[{\downstep}nzísíáchílá]}  &   
                     \gloss{‘smacking’}  &  \\

                     \vernacular{
                    yaá{\downstep}múú[{\downstep}nzónónyínyílá]}  &   
                     \gloss{‘spoiling’}  &  \\

                     \vernacular{
                    yaá{\downstep}múú[{\downstep}nzábúkhányínyílá]}  &   
                     \gloss{‘separating’}  &  \\
\end{tabular}
%\caption{\nocaption}
     
\begin{tabular}{llllll}  
  \multicolumn{5}{l}{
                     \vernacular{(687) /Ø/
                    C-Initial + OP + OP
                    } \gloss{‘s/he is
                    ever/always...him/her for me’} } &  \\
\multicolumn{5}{l}{ } &  \\

                     \vernacular{
                    yaá{\downstep}múú[{\downstep}nzíílá]}  &   
                     \gloss{‘going for’}  &  \\

                     \vernacular{
                    yaá{\downstep}múú[{\downstep}ndéshélá]}  &   
                     \gloss{‘leaving’}  &  \\

                     \vernacular{
                    yaá{\downstep}múú[{\downstep}nóóndélá]}  &   
                     \gloss{‘following’}  &  \\

                     \vernacular{
                    yaá{\downstep}múú[{\downstep}ngúlíshílá]}  &   
                     \gloss{‘naming’}  &  \\

                     \vernacular{
                    yaá{\downstep}múú[{\downstep}ndákhúúlílá]}  &   
                     \gloss{‘releasing’}  &  \\

                     \vernacular{
                    yaá{\downstep}múú[{\downstep}séébúlílá]}  &   
                     \gloss{‘saying bye
                    to’}  &  \\

                     \vernacular{
                    yaá{\downstep}múú[{\downstep}mbóómbélítsílá]}  &   
                     \gloss{‘comforting’}  &  \\

                     \vernacular{
                    yaá{\downstep}múú[{\downstep}síínjílítsílá]}  &   
                     \gloss{
                    ‘making...stand’}  &  \\
\end{tabular}
%\caption{\nocaption}
     
\begin{tabular}{llllll}  
  \multicolumn{5}{l}{
                     \vernacular{(688) /Ø/
                    V-Initial + OP + OP
                    } \gloss{‘s/he is
                    ever/always...it
                    } } &  \\
\multicolumn{5}{l}{ } &  \\

                     \vernacular{
                    yaá{\downstep}búú[{\downstep}nzálílá]}  &   
                     \gloss{‘displaying’}  &     &   
                     \vernacular{
                    yaá{\downstep}kúú[{\downstep}nzáshítsílá]}  &   
                     \gloss{‘lighting’}  &  \\

                     \vernacular{
                    yaá{\downstep}búú[{\downstep}nzílúúlílá]}  &   
                     \gloss{‘winnowing’}  &     &   
                     \vernacular{
                    yaá{\downstep}kúú[{\downstep}nzéléélítsílá]}  &   
                     \gloss{‘hanging’}  &  \\
\end{tabular}
%\caption{\nocaption}
     
\begin{tabular}{lll}  
  \multicolumn{2}{l}{
                     \vernacular{(689) /H/
                    C-Initial Phrase-Medial} \gloss{‘s/he is
                    ever/always...the boy \ob mú{\downstep}yáyi\cb  /} } &  \\
\multicolumn{2}{l}{
                     \gloss{someone
                    \ob muundu\cb ’} } &  \\

                     \vernacular{yaá[ra]
                    mú{\downstep}yáyi/muundu}  &   
                     \gloss{‘burying’}  &  \\

                     \vernacular{yaá[beka]
                    mú{\downstep}yáyi/muundu}  &   
                     \gloss{‘shaving’}  &  \\

                     \vernacular{yaá[leera]
                    mú{\downstep}’yáyi/muundu}  &   
                     \gloss{‘bringing’}  &  \\

                     \vernacular{yaá[khalaka]
                    mú{\downstep}’yáyi/muundu}  &   
                     \gloss{‘cutting’}  &  \\

                     \vernacular{yaá[sitaaka]
                    mú{\downstep}’yáyi/muundu}  &   
                     \gloss{‘accusing’}  &  \\

                     \vernacular{yaá[boolitsa]
                    mú{\downstep}’yáyi/muundu}  &   
                     \gloss{‘seducing’}  &  \\

                     \vernacular{yaá[khong’oonda]
                    mú{\downstep}’yáyi/muundu}  &   
                     \gloss{‘knocking’}  &  \\

                     \vernacular{yaá[boholola]
                    mú{\downstep}’yáyi/muundu}  &   
                     \gloss{‘untying’}  &  \\

                     \vernacular{yaá[boyong’ana]
                    mú{\downstep}’yáyi/muundu}  &   
                     \gloss{‘going
                    around’}  &  \\

                     \vernacular{
                    yaá[ng’ong’oolitsa]
                    mú{\downstep}’yáyi/muundu}  &   
                     \gloss{‘teasing’}  &  \\
\end{tabular}
%\caption{\nocaption}
     
\begin{tabular}{lll}  
  \multicolumn{2}{l}{
                     \vernacular{(690) /Ø/
                    C-Initial Phrase-Medial} \gloss{‘s/he is
                    ever/always...the boy \ob mú{\downstep}yáyi\cb  /} } &  \\
\multicolumn{2}{l}{
                     \gloss{someone
                    \ob muundu\cb ’} } &  \\

                     \vernacular{yaá[tsia]
                    mú{\downstep}’yáyi/muundu}  &   
                     \gloss{‘going for’}  &  \\

                     \vernacular{yaá[lekha]
                    mú{\downstep}’yáyi/muundu}  &   
                     \gloss{‘leaving’}  &  \\

                     \vernacular{yaá[loonda]
                    mú{\downstep}yáyi/muundu}  &   
                     \gloss{‘following’}  &  \\

                     \vernacular{yaá[kulikha]
                    mú{\downstep}yáyi/muundu}  &   
                     \gloss{‘naming’}  &  \\

                     \vernacular{yaá[lakhuula]
                    mú{\downstep}yáyi/muundu}  &   
                     \gloss{‘releasing’}  &  \\

                     \vernacular{yaá[seebula]
                    mú{\downstep}yáyi/muundu}  &   
                     \gloss{‘saying bye
                    to’}  &  \\

                     \vernacular{yaá[kalushitsa]
                    mú{\downstep}yáyi/muundu}  &   
                     \gloss{‘returning’}  &  \\

                     \vernacular{yaá[siinjilitsa]
                    mú{\downstep}yáyi/muundu}  &   
                     \gloss{
                    ‘making...stand’}  &  \\

                     \vernacular{yaá[reebareeba]
                    mú{\downstep}yáyi/muundu}  &   
                     \gloss{‘asking
                    (iter)’}  &  \\

                     \vernacular{
                    yaá[kalukhanyinya] mú{\downstep}yáyi/muundu}  &   
                     \gloss{
                    ‘turning...over’}  &  \\
\end{tabular}
%\caption{\nocaption}
     
\begin{tabular}{lll}  
  \multicolumn{2}{l}{
                     \vernacular{(691) /H/
                    C-Initial +OP Phrase-Medial} \gloss{‘s/he is
                    ever/always...the boy \ob mú{\downstep}yáyi\cb  /} } &  \\
\multicolumn{2}{l}{
                     \gloss{someone \ob muundu\cb 
                    for him/her’} } &  \\

                     \vernacular{yaá{\downstep}mú[réela]
                    mú{\downstep}’yáyi/muundu}  &   
                     \gloss{‘burying’}  &  \\

                     \vernacular{yaá{\downstep}mú[béchela]
                    mú{\downstep}’yáyi/muundu}  &   
                     \gloss{‘shaving’}  &  \\

                     \vernacular{yaá{\downstep}mú[léerela]
                    mú{\downstep}’yáyi/muundu}  &   
                     \gloss{‘bringing’}  &  \\

                     \vernacular{
                    yaá{\downstep}mú[khálachila]
                    mú{\downstep}’yáyi/muundu}  &   
                     \gloss{‘cutting’}  &  \\

                     \vernacular{
                    yaá{\downstep}mú[sítaachila]
                    mú{\downstep}’yáyi/muundu}  &   
                     \gloss{‘accusing’}  &  \\

                     \vernacular{
                    yaá{\downstep}mú[bóolitsila]
                    mú{\downstep}’yáyi/muundu}  &   
                     \gloss{‘seducing’}  &  \\

                     \vernacular{
                    yaá{\downstep}mú[khóng’oondela]
                    mú{\downstep}’yáyi/muundu}  &   
                     \gloss{‘knocking’}  &  \\

                     \vernacular{
                    yaá{\downstep}mú[bóhololela]
                    mú{\downstep}’yáyi/muundu}  &   
                     \gloss{‘untying’}  &  \\

                     \vernacular{
                    yaá{\downstep}mú[bóyong’anila]
                    mú{\downstep}’yáyi/muundu}  &   
                     \gloss{‘going
                    around’}  &  \\

                     \vernacular{
                    yaá{\downstep}mú[ng’óng’oolitsila]
                    mú{\downstep}’yáyi/muundu}  &   
                     \gloss{‘teasing’}  &  \\
\end{tabular}
%\caption{\nocaption}
     
\begin{tabular}{lll}  
  \multicolumn{2}{l}{
                     \vernacular{(692) /Ø/
                    C-Initial +OP Phrase-Medial} \gloss{‘s/he is
                    ever/always...the boy \ob mú{\downstep}yáyi\cb  /} } &  \\
\multicolumn{2}{l}{
                     \gloss{someone \ob muundu\cb 
                    for him/her’} } &  \\

                     \vernacular{yaámu[tsiila]
                    mú{\downstep}yáyi/muundu}  &   
                     \gloss{‘going for’}  &  \\

                     \vernacular{yaámu[leshela]
                    mú{\downstep}yáyi/muundu}  &   
                     \gloss{‘leaving’}  &  \\

                     \vernacular{yaámu[loondela]
                    mú{\downstep}yáyi/muundu}  &   
                     \gloss{‘following’}  &  \\

                     \vernacular{yaámu[kulishila]
                    mú{\downstep}yáyi/muundu}  &   
                     \gloss{‘naming’}  &  \\

                     \vernacular{yaámu[lakhuulila]
                    mú{\downstep}yáyi/muundu}  &   
                     \gloss{‘releasing’}  &  \\

                     \vernacular{yaámu[seebulila]
                    mú{\downstep}yáyi/muundu}  &   
                     \gloss{‘saying bye
                    to’}  &  \\

                     \vernacular{
                    yaámu[reebareebela]
                    mú{\downstep}yáyi/muundu}  &   
                     \gloss{‘asking
                    (iter)’}  &  \\
\end{tabular}
%\caption{\nocaption}
     
\begin{tabular}{lll}  
  \multicolumn{2}{l}{
                     \vernacular{(693) /H/
                    C-Initial +OP + OP
                    } \gloss{‘s/he is
                    ever/always...the boy \ob mú{\downstep}yáyi\cb  /} } &  \\
\multicolumn{2}{l}{
                     \gloss{someone \ob muundu\cb 
                    for him/her for me’} } &  \\

                     \vernacular{yaá{\downstep}múú[ndeela]
                    mú{\downstep}’yáyi/muundu}  &   
                     \gloss{‘burying’}  &  \\

                     \vernacular{
                    yaá{\downstep}múú[mbechela]
                    mú{\downstep}’yáyi/muundu}  &   
                     \gloss{‘shaving’}  &  \\

                     \vernacular{
                    yaá{\downstep}múú[ndeerela]
                    mú{\downstep}’yáyi/muundu}  &   
                     \gloss{‘bringing’}  &  \\

                     \vernacular{
                    yaá{\downstep}múú[khalachila]
                    mú{\downstep}’yáyi/muundu}  &   
                     \gloss{‘cutting’}  &  \\

                     \vernacular{
                    yaá{\downstep}múú[sitaachila]
                    mú{\downstep}’yáyi/muundu}  &   
                     \gloss{‘accusing’}  &  \\

                     \vernacular{
                    yaá{\downstep}múú[mboolitsila]
                    mú{\downstep}’yáyi/muundu}  &   
                     \gloss{‘seducing’}  &  \\

                     \vernacular{
                    yaá{\downstep}múú[mbohololela]
                    mú{\downstep}’yáyi/muundu}  &   
                     \gloss{‘untying’}  &  \\
\end{tabular}
%\caption{\nocaption}
     
\begin{tabular}{lll}  
  \multicolumn{2}{l}{
                     \vernacular{(694) /Ø/
                    C-Initial +OP + OP
                    } \gloss{‘s/he is
                    ever/always...the boy \ob mú{\downstep}yáyi\cb  /} } &  \\
\multicolumn{2}{l}{
                     \gloss{someone \ob muundu\cb 
                    for him/her for me’} } &  \\

                     \vernacular{yaá{\downstep}múú[nziila]
                    mú{\downstep}yáyi/muundu}  &   
                     \gloss{‘going for’}  &  \\

                     \vernacular{
                    yaá{\downstep}múú[ndeshela]
                    mú{\downstep}yáyi/muundu}  &   
                     \gloss{‘leaving’}  &  \\

                     \vernacular{
                    yaá{\downstep}múú[noondela]
                    mú{\downstep}yáyi/muundu}  &   
                     \gloss{‘following’}  &  \\

                     \vernacular{
                    yaá{\downstep}múú[ngulishila]
                    mú{\downstep}yáyi/muundu}  &   
                     \gloss{‘naming’}  &  \\

                     \vernacular{
                    yaá{\downstep}múú[ndakhuulila]
                    mú{\downstep}yáyi/muundu}  &   
                     \gloss{‘releasing’}  &  \\

                     \vernacular{
                    yaá{\downstep}múú[seebulila]
                    mú{\downstep}yáyi/muundu}  &   
                     \gloss{‘saying bye
                    to’}  &  \\

                     \vernacular{
                    yaá{\downstep}múú[mboombelitsila]
                    mú{\downstep}yáyi/muundu}  &   
                     \gloss{‘comforting’}  &  \\

                     \vernacular{
                    yaá{\downstep}múú[siinjilitsila]
                    mú{\downstep}yáyi/muundu}  &   
                     \gloss{
                    ‘making...stand’}  &  \\
\end{tabular}
%\caption{\nocaption}
    

\subsection{Habitual Negative: Pattern 8}\label{sec:sHabitNeg}


\begin{tabular}{llllll}  
  \multicolumn{5}{l}{
                     \vernacular{(695) /H/
                    C-Initial} \gloss{‘s/he is not
                    ever/always...’} } &  \\
\multicolumn{5}{l}{ } &  \\

                     \vernacular{yaá[{\downstep}rá]
                    tá}  &   
                     \gloss{‘burying’}  &     &   
                     \vernacular{yaá[{\downstep}ng’wá]
                    tá}  &   
                     \gloss{‘drinking’}  &  \\

                     \vernacular{yaá[{\downstep}líá]
                    tá}  &   
                     \gloss{‘eating’}  &     &   
                     \vernacular{yaá[{\downstep}lúmá]
                    tá}  &   
                     \gloss{‘biting’}  &  \\

                     \vernacular{yaá[{\downstep}béká]
                    tá}  &   
                     \gloss{‘shaving’}  &     &   
                     \vernacular{yaá[{\downstep}téékhá]
                    tá}  &   
                     \gloss{‘cooking’}  &  \\

                     \vernacular{yaá[{\downstep}léérá]
                    tá}  &   
                     \gloss{‘bringing’}  &     &   
                     \vernacular{yaá[{\downstep}kháláká]
                    tá}  &   
                     \gloss{‘cutting’}  &  \\

                     \vernacular{
                    yaá[{\downstep}káláángá] tá}  &   
                     \gloss{‘frying’}  &     &   
                     \vernacular{yaá[{\downstep}sítááká]
                    tá}  &   
                     \gloss{‘accusing’}  &  \\

                     \vernacular{
                    yaá[{\downstep}bóólítsá] tá}  &   
                     \gloss{‘seducing’}  &     &   
                     \vernacular{
                    yaá[{\downstep}sáándítsá] tá}  &   
                     \gloss{‘thanking’}  &  \\

                     \vernacular{
                    yaá[{\downstep}tsúúnzúúná] tá}  &   
                     \gloss{‘sucking’}  &     &   
                     \vernacular{
                    yaá[{\downstep}bóhólólá] tá}  &   
                     \gloss{‘untying’}  &  \\

                     \vernacular{
                    yaá[{\downstep}bóyóng’áná] tá}  &   
                     \gloss{‘going
                    around’}  &     &   
                     \vernacular{
                    yaá[{\downstep}ng’óng’óólítsá] tá}  &   
                     \gloss{‘teasing’}  &  \\

                     \vernacular{
                    yaá[{\downstep}língákányínyá] tá}  &   
                     \gloss{‘crumpling’}  &  \\
\end{tabular}
%\caption{\nocaption}
     
\begin{tabular}{llllll}  
  \multicolumn{5}{l}{
                     \vernacular{(696) /H/
                    V-Initial} \gloss{‘s/he is not
                    ever/always...’} } &  \\
\multicolumn{5}{l}{ } &  \\

                     \vernacular{yaá[{\downstep}yírá]
                    tá}  &   
                     \gloss{‘killing’}  &     &   
                     \vernacular{
                    yaá[{\downstep}yíkóómbá] tá}  &   
                     \gloss{‘admiring’}  &  \\

                     \vernacular{yaá[{\downstep}yísíáká]
                    tá}  &   
                     \gloss{‘smacking’}  &     &   
                     \vernacular{
                    yaá[{\downstep}yíkóbólá] tá}  &   
                     \gloss{‘belching’}  &  \\

                     \vernacular{
                    yaá[{\downstep}yónónyínyá] tá}  &   
                     \gloss{‘spoiling’}  &     &   
                     \vernacular{
                    yaá[{\downstep}yábúkhányínyá] tá}  &   
                     \gloss{‘separating’}  &  \\
\end{tabular}
%\caption{\nocaption}
     
\begin{tabular}{llllll}  
  \multicolumn{5}{l}{
                     \vernacular{(697) /Ø/
                    C-Initial} \gloss{‘s/he is not
                    ever/always...’} } &  \\
\multicolumn{5}{l}{ } &  \\

                     \vernacular{yaá[{\downstep}tsíá]
                    tá}  &   
                     \gloss{‘going’}  &  \\

                     \vernacular{yaá[{\downstep}kwá]
                    tá}  &   
                     \gloss{‘falling’}  &  \\

                     \vernacular{yaá[{\downstep}lékhá]
                    tá}  &   
                     \gloss{‘leaving’}  &  \\

                     \vernacular{yaá[{\downstep}réébá]
                    tá}  &   
                     \gloss{‘asking’}  &  \\

                     \vernacular{yaá[{\downstep}lóóndá]
                    tá}  &   
                     \gloss{‘following’}  &  \\

                     \vernacular{yaá[{\downstep}kúmílá]
                    tá}  &   
                     \gloss{‘holding’}  &  \\

                     \vernacular{yaá[{\downstep}kúlíkhá]
                    tá}  &   
                     \gloss{‘naming’}  &  \\

                     \vernacular{yaá[{\downstep}hómóólá]
                    tá}  &   
                     \gloss{‘massaging’}  &  \\

                     \vernacular{
                    yaá[{\downstep}lákhúúlá] tá}  &   
                     \gloss{‘releasing’}  &  \\

                     \vernacular{yaá[{\downstep}séébúlá]
                    tá}  &   
                     \gloss{‘saying bye’}  &  \\

                     \vernacular{
                    yaá[{\downstep}hóómbélítsá] tá}  &   
                     \gloss{‘comforting’}  &  \\

                     \vernacular{
                    yaá[{\downstep}kálúshítsá] tá}  &   
                     \gloss{‘returning’}  &  \\

                     \vernacular{
                    yaá[{\downstep}síínjílítsá] tá}  &   
                     \gloss{‘making
                    stand’}  &  \\

                     \vernacular{
                    yaá[{\downstep}réébáréébá] tá}  &   
                     \gloss{‘asking
                    (iter)’}  &  \\

                     \vernacular{
                    yaá[{\downstep}kálúkhányínyá] tá}  &   
                     \gloss{‘turning
                    over’}  &  \\

                     \vernacular{
                    yaá[{\downstep}sébúlúkhányínyá] tá}  &   
                     \gloss{‘scattering’}  &  \\
\end{tabular}
%\caption{\nocaption}
     
\begin{tabular}{llllll}  
  \multicolumn{5}{l}{
                     \vernacular{(698) /Ø/
                    V-Initial} \gloss{‘s/he is not
                    ever/always...’} } &  \\
\multicolumn{5}{l}{ } &  \\

                     \vernacular{yaá[{\downstep}yényá]
                    tá}  &   
                     \gloss{‘wanting’}  &     &   
                     \vernacular{yaá[{\downstep}yéyélá]
                    tá}  &   
                     \gloss{‘wiping for’}  &  \\

                     \vernacular{yaá[{\downstep}yílúúlá]
                    tá}  &   
                     \gloss{‘winnowing’}  &     &   
                     \vernacular{
                    yaá[{\downstep}yámbákháná] tá}  &   
                     \gloss{‘refusing’}  &  \\

                     \vernacular{
                    yaá[{\downstep}yéléélítsá] tá}  &   
                     \gloss{‘hanging up’}  &     &   
                     \vernacular{
                    yaá[{\downstep}yíkóómbá] tá}  &   
                     \gloss{‘admiring’}  &  \\
\end{tabular}
%\caption{\nocaption}
     
\begin{tabular}{llllll}  
  \multicolumn{5}{l}{
                     \vernacular{(699) /H/
                    C-Initial + OP} \gloss{‘s/he is not
                    ever/always...him/her’} } &  \\
\multicolumn{5}{l}{ } &  \\

                     \vernacular{yaá{\downstep}mú[rá]
                    {\downstep}tá}  &   
                     \gloss{‘burying’}  &  \\

                     \vernacular{yaá{\downstep}mú[khúá]
                    {\downstep}tá}  &   
                     \gloss{‘paying
                    dowry’}  &  \\

                     \vernacular{yaá{\downstep}mú[bé{\downstep}ká]
                    tá}  &   
                     \gloss{‘shaving’}  &  \\

                     \vernacular{
                    yaá{\downstep}mú[lé{\downstep}érá] tá}  &   
                     \gloss{‘bringing’}  &  \\

                     \vernacular{
                    yaá{\downstep}mú[khá{\downstep}láká] tá}  &   
                     \gloss{‘cutting’}  &  \\

                     \vernacular{
                    yaá{\downstep}mú[sí{\downstep}tááká] tá}  &   
                     \gloss{‘accusing’}  &  \\

                     \vernacular{
                    yaá{\downstep}mú[bó{\downstep}ólítsá] tá}  &   
                     \gloss{‘seducing’}  &  \\

                     \vernacular{
                    yaá{\downstep}mú[khó{\downstep}ng’óóndá] tá}  &   
                     \gloss{‘knocking’}  &  \\

                     \vernacular{
                    yaá{\downstep}mú[bó{\downstep}hólólá] tá}  &   
                     \gloss{‘untying’}  &  \\

                     \vernacular{
                    yaá{\downstep}mú[bó{\downstep}yóng’áná] tá}  &   
                     \gloss{‘going
                    around’}  &  \\

                     \vernacular{
                    yaá{\downstep}mú[ng’ó{\downstep}ng’óólítsá] tá}  &   
                     \gloss{‘teasing’}  &  \\

                     \vernacular{
                    yaá{\downstep}mú[lí{\downstep}ngákányínyá] tá}  &   
                     \gloss{‘bending’}  &  \\
\end{tabular}
%\caption{\nocaption}
     
\begin{tabular}{llllll}  
  \multicolumn{5}{l}{
                     \vernacular{(700) /Ø/
                    C-Initial + OP} \gloss{‘s/he is not
                    ever/always...him/her \ob mu-\cb ’} } &  \\
\multicolumn{5}{l}{ } &  \\

                     \vernacular{yaá{\downstep}mú[tsíá]
                    tá}  &   
                     \gloss{‘going for’}  &  \\

                     \vernacular{yaá{\downstep}mú[lékhá]
                    tá}  &   
                     \gloss{‘leaving’}  &  \\

                     \vernacular{
                    yaá{\downstep}mú[lóóndá] tá}  &   
                     \gloss{‘following’}  &  \\

                     \vernacular{
                    yaá{\downstep}mú[kúlíkhá] tá}  &   
                     \gloss{‘naming’}  &  \\

                     \vernacular{
                    yaá{\downstep}mú[lákhúúlá] tá}  &   
                     \gloss{‘releasing’}  &  \\

                     \vernacular{
                    yaá{\downstep}mú[séébúlá] tá}  &   
                     \gloss{‘saying bye
                    to’}  &  \\

                     \vernacular{
                    yaá{\downstep}mú[hóómbélítsá] tá}  &   
                     \gloss{‘comforting’}  &  \\

                     \vernacular{
                    yaá{\downstep}mú[kálúshítsá] tá}  &   
                     \gloss{‘returning’}  &  \\

                     \vernacular{
                    yaá{\downstep}mú[síínjílítsá] tá}  &   
                     \gloss{
                    ‘making...stand’}  &  \\

                     \vernacular{
                    yaá{\downstep}mú[réébáréébá] tá}  &   
                     \gloss{‘asking
                    (iter)’}  &  \\

                     \vernacular{
                    yaá{\downstep}mú[kálúkhányínyá] tá}  &   
                     \gloss{
                    ‘turning...over’}  &  \\

                     \vernacular{
                    yaá{\downstep}mú[sébúlúkhányínyá] tá}  &   
                     \gloss{‘scattering’}  &  \\
\end{tabular}
%\caption{\nocaption}
     
\begin{tabular}{llllll}  
  \multicolumn{5}{l}{
                     \vernacular{(701) /H/
                    C-Initial + OP
                    } \gloss{‘s/he is not
                    ever/always...me’} } &  \\
\multicolumn{5}{l}{ } &  \\

                     \vernacular{yaá{\downstep}á[ríá]
                    {\downstep}tá}  &   
                     \gloss{‘fearing’}  &     &   
                     \vernacular{yaá{\downstep}á[khwá]
                    {\downstep}tá}  &   
                     \gloss{‘paying
                    dowry’}  &  \\

                     \vernacular{yaá{\downstep}á[mbé{\downstep}ká]
                    tá}  &   
                     \gloss{‘shaving’}  &     &   
                     \vernacular{
                    yaá{\downstep}á[ndé{\downstep}érá] tá}  &   
                     \gloss{‘bringing’}  &  \\

                     \vernacular{
                    yaá{\downstep}á[khá{\downstep}láká] tá}  &   
                     \gloss{‘cutting’}  &     &   
                     \vernacular{
                    yaá{\downstep}á[sí{\downstep}tááká] tá}  &   
                     \gloss{‘accusing’}  &  \\

                     \vernacular{
                    yaá{\downstep}á[mbó{\downstep}ólítsá] tá}  &   
                     \gloss{‘seducing’}  &     &   
                     \vernacular{
                    yaá{\downstep}á[khó{\downstep}ng’óóndá] tá}  &   
                     \gloss{‘knocking’}  &  \\

                     \vernacular{
                    yaá{\downstep}á[mbó{\downstep}hólólá] tá}  &   
                     \gloss{‘untying’}  &     &   
                     \vernacular{
                    yaá{\downstep}á[mbó{\downstep}yóng’áná] tá}  &   
                     \gloss{‘going
                    around’}  &  \\

                     \vernacular{
                    yaá{\downstep}á[ng’ó{\downstep}ng’óólítsá] tá}  &   
                     \gloss{‘teasing’}  &     &   
                     \vernacular{
                    yaá{\downstep}á[ní{\downstep}ngákányínyá] tá}  &   
                     \gloss{‘bending’}  &  \\
\end{tabular}
%\caption{\nocaption}
     
\begin{tabular}{llllll}  
  \multicolumn{5}{l}{
                     \vernacular{(702) /Ø/
                    C-Initial + OP
                    } \gloss{‘s/he is not
                    ever/always...me’} } &  \\
\multicolumn{5}{l}{ } &  \\

                     \vernacular{yaá{\downstep}á[síá]
                    tá}  &   
                     \gloss{‘grinding’}  &  \\

                     \vernacular{yaá{\downstep}á[ndékhá]
                    tá}  &   
                     \gloss{‘leaving’}  &  \\

                     \vernacular{yaá{\downstep}á[nóóndá]
                    tá}  &   
                     \gloss{‘following’}  &  \\

                     \vernacular{
                    yaá{\downstep}á[ngúlíkhá] tá}  &   
                     \gloss{‘naming’}  &  \\

                     \vernacular{
                    yaá{\downstep}á[ndákhúúlá] tá}  &   
                     \gloss{‘releasing’}  &  \\

                     \vernacular{
                    yaá{\downstep}á[séébúlá] tá}  &   
                     \gloss{‘saying bye
                    to’}  &  \\

                     \vernacular{
                    yaá{\downstep}á[mbóómbélítsá] tá}  &   
                     \gloss{‘comforting’}  &  \\

                     \vernacular{
                    yaá{\downstep}á[síínjílítsá] tá}  &   
                     \gloss{
                    ‘making..stand’}  &  \\

                     \vernacular{
                    yaá{\downstep}á[réébáréébá] tá}  &   
                     \gloss{‘asking
                    (iter)’}  &  \\

                     \vernacular{
                    yaá{\downstep}á[ngálúkhányínyá] tá}  &   
                     \gloss{
                    ‘turning...over’}  &  \\
\end{tabular}
%\caption{\nocaption}
     
\begin{tabular}{llllll}  
  \multicolumn{5}{l}{
                     \vernacular{(703) /H/
                    C-Initial + OP + OP
                    } \gloss{‘s/he is not
                    ever/always...him/her for me’} } &  \\
\multicolumn{5}{l}{ } &  \\

                     \vernacular{
                    yaá{\downstep}múú[{\downstep}ndéélá] tá}  &   
                     \gloss{‘burying’}  &     &   
                     \vernacular{
                    yaá{\downstep}múú[{\downstep}mbéchélá] tá}  &   
                     \gloss{‘shaving’}  &  \\

                     \vernacular{
                    yaá{\downstep}múú[{\downstep}ndéérélá] tá}  &   
                     \gloss{‘bringing’}  &     &   
                     \vernacular{
                    yaá{\downstep}múú[{\downstep}kháláchílá] tá}  &   
                     \gloss{‘cutting’}  &  \\

                     \vernacular{
                    yaá{\downstep}múú[{\downstep}sítááchílá] tá}  &   
                     \gloss{‘accusing’}  &     &   
                     \vernacular{
                    yaá{\downstep}múú[{\downstep}mbóólítsílá] tá}  &   
                     \gloss{‘seducing’}  &  \\

                     \vernacular{
                    yaá{\downstep}múú[{\downstep}mbóhólólélá] tá}  &   
                     \gloss{‘untying’}  &     &     &     &  \\
\end{tabular}
%\caption{\nocaption}
     
\begin{tabular}{llllll}  
  \multicolumn{5}{l}{
                     \vernacular{(704) /Ø/
                    C-Initial + OP + OP
                    } \gloss{‘s/he is not
                    ever/always...him/her for me’} } &  \\
\multicolumn{5}{l}{ } &  \\

                     \vernacular{
                    yaá{\downstep}múú[{\downstep}nzíílá] tá}  &   
                     \gloss{‘going for’}  &  \\

                     \vernacular{
                    yaá{\downstep}múú[{\downstep}ndéshélá] tá}  &   
                     \gloss{‘leaving’}  &  \\

                     \vernacular{
                    yaá{\downstep}múú[{\downstep}nóóndélá] tá}  &   
                     \gloss{‘following’}  &  \\

                     \vernacular{
                    yaá{\downstep}múú[{\downstep}ngúlíshílá] tá}  &   
                     \gloss{‘naming’}  &  \\

                     \vernacular{
                    yaá{\downstep}múú[{\downstep}ndákhúúlílá] tá}  &   
                     \gloss{‘releasing’}  &  \\

                     \vernacular{
                    yaá{\downstep}múú[{\downstep}séébúlílá] tá}  &   
                     \gloss{‘saying bye
                    to’}  &  \\

                     \vernacular{
                    yaá{\downstep}múú[{\downstep}mbóómbélítsílá]
                    tá}  &   
                     \gloss{‘comforting’}  &  \\

                     \vernacular{
                    yaá{\downstep}múú[{\downstep}síínjílítsílá] tá}  &   
                     \gloss{
                    ‘making...stand’}  &  \\
\end{tabular}
%\caption{\nocaption}
     
\begin{tabular}{lll}  
  \multicolumn{2}{l}{
                     \vernacular{(705) /H/
                    C-Initial Phrase-Medial} \gloss{‘s/he is not
                    ever/always...the boy \ob mú{\downstep}yáyi\cb  /} } &  \\
\multicolumn{2}{l}{
                     \gloss{someone
                    \ob muundu\cb ’} } &  \\

                     \vernacular{yaá[ra]
                    mú{\downstep}yáyi/muundu tá}  &   
                     \gloss{‘burying’}  &  \\

                     \vernacular{yaá[beka]
                    mú{\downstep}yáyi/muundu tá}  &   
                     \gloss{‘shaving’}  &  \\

                     \vernacular{yaá[leera]
                    mú{\downstep}’yáyi/muundu tá}  &   
                     \gloss{‘bringing’}  &  \\

                     \vernacular{yaá[khalaka]
                    mú{\downstep}’yáyi/muundu tá}  &   
                     \gloss{‘cutting’}  &  \\

                     \vernacular{yaá[sitaaka]
                    mú{\downstep}’yáyi/muundu tá}  &   
                     \gloss{‘accusing’}  &  \\

                     \vernacular{yaá[boolitsa]
                    mú{\downstep}’yáyi/muundu tá}  &   
                     \gloss{‘seducing’}  &  \\

                     \vernacular{yaá[khong’oonda]
                    mú{\downstep}’yáyi/muundu tá}  &   
                     \gloss{‘knocking’}  &  \\

                     \vernacular{yaá[boholola]
                    mú{\downstep}’yáyi/muundu tá}  &   
                     \gloss{‘untying’}  &  \\

                     \vernacular{yaá[boyong’ana]
                    mú{\downstep}’yáyi/muundu tá}  &   
                     \gloss{‘going
                    around’}  &  \\

                     \vernacular{
                    yaá[ng’ong’oolitsa] mú{\downstep}’yáyi/muundu
                    tá}  &   
                     \gloss{‘teasing’}  &  \\
\end{tabular}
%\caption{\nocaption}
     
\begin{tabular}{lll}  
  \multicolumn{2}{l}{
                     \vernacular{(706) /Ø/
                    C-Initial Phrase-Medial} \gloss{‘s/he is not
                    ever/always...the boy \ob mú{\downstep}yáyi\cb  /} } &  \\
\multicolumn{2}{l}{
                     \gloss{someone \ob muundu
                    tá\cb ’} } &  \\

                     \vernacular{yaá[tsia]
                    mú{\downstep}’yáyi/muundu tá}  &   
                     \gloss{‘going for’}  &  \\

                     \vernacular{yaá[lekha]
                    mú{\downstep}’yáyi/muundu tá}  &   
                     \gloss{‘leaving’}  &  \\

                     \vernacular{yaá[loonda]
                    mú{\downstep}yáyi/muundu tá}  &   
                     \gloss{‘following’}  &  \\

                     \vernacular{yaá[kulikha]
                    mú{\downstep}yáyi/muundu tá}  &   
                     \gloss{‘naming’}  &  \\

                     \vernacular{yaá[lakhuula]
                    mú{\downstep}yáyi/muundu tá}  &   
                     \gloss{‘releasing’}  &  \\

                     \vernacular{yaá[seebula]
                    mú{\downstep}yáyi/muundu tá}  &   
                     \gloss{‘saying bye
                    to’}  &  \\

                     \vernacular{yaá[kalushitsa]
                    mú{\downstep}yáyi/muundu tá}  &   
                     \gloss{‘returning’}  &  \\

                     \vernacular{yaá[siinjilitsa]
                    mú{\downstep}yáyi/muundu tá}  &   
                     \gloss{
                    ‘making...stand’}  &  \\

                     \vernacular{yaá[reebareeba]
                    mú{\downstep}yáyi/muundu tá}  &   
                     \gloss{‘asking
                    (iter)’}  &  \\

                     \vernacular{
                    yaá[kalukhanyinya] mú{\downstep}yáyi/muundu
                    tá}  &   
                     \gloss{
                    ‘turning...over’}  &  \\
\end{tabular}
%\caption{\nocaption}
     
\begin{tabular}{lll}  
  \multicolumn{2}{l}{
                     \vernacular{(707) /H/
                    C-Initial +OP Phrase-Medial} \gloss{‘s/he is not
                    ever/always...the boy \ob mú{\downstep}yáyi\cb  /} } &  \\
\multicolumn{2}{l}{
                     \gloss{someone \ob muundu
                    tá\cb  for him/her’} } &  \\

                     \vernacular{yaá{\downstep}mú[réela]
                    mú{\downstep}’yáyi/muundu tá}  &   
                     \gloss{‘burying’}  &  \\

                     \vernacular{yaá{\downstep}mú[béchela]
                    mú{\downstep}’yáyi/muundu tá}  &   
                     \gloss{‘shaving’}  &  \\

                     \vernacular{yaá{\downstep}mú[léerela]
                    mú{\downstep}’yáyi/muundu tá}  &   
                     \gloss{‘bringing’}  &  \\

                     \vernacular{
                    yaá{\downstep}mú[khálachila] mú{\downstep}’yáyi/muundu
                    tá}  &   
                     \gloss{‘cutting’}  &  \\

                     \vernacular{
                    yaá{\downstep}mú[sítaachila] mú{\downstep}’yáyi/muundu
                    tá}  &   
                     \gloss{‘accusing’}  &  \\

                     \vernacular{
                    yaá{\downstep}mú[bóolitsila] mú{\downstep}’yáyi/muundu
                    tá}  &   
                     \gloss{‘seducing’}  &  \\

                     \vernacular{
                    yaá{\downstep}mú[khóng’oondela] mú{\downstep}’yáyi/muundu
                    tá}  &   
                     \gloss{‘knocking’}  &  \\

                     \vernacular{
                    yaá{\downstep}mú[bóhololela] mú{\downstep}’yáyi/muundu
                    tá}  &   
                     \gloss{‘untying’}  &  \\

                     \vernacular{
                    yaá{\downstep}mú[bóyong’anila] mú{\downstep}’yáyi/muundu
                    tá}  &   
                     \gloss{‘going
                    around’}  &  \\

                     \vernacular{
                    yaá{\downstep}mú[ng’óng’oolitsila] mú{\downstep}’yáyi/muundu
                    tá}  &   
                     \gloss{‘teasing’}  &  \\
\end{tabular}
%\caption{\nocaption}
     
\begin{tabular}{lll}  
  \multicolumn{2}{l}{
                     \vernacular{(708) /Ø/
                    C-Initial +OP Phrase-Medial} \gloss{‘s/he is not
                    ever/always...the boy \ob mú{\downstep}yáyi\cb  /} } &  \\
\multicolumn{2}{l}{
                     \gloss{someone \ob muundu
                    tá\cb  for him/her’} } &  \\

                     \vernacular{yaámu[tsiila]
                    mú{\downstep}yáyi/muundu tá}  &   
                     \gloss{‘going for’}  &  \\

                     \vernacular{yaámu[leshela]
                    mú{\downstep}yáyi/muundu tá}  &   
                     \gloss{‘leaving’}  &  \\

                     \vernacular{yaámu[loondela]
                    mú{\downstep}yáyi/muundu tá}  &   
                     \gloss{‘following’}  &  \\

                     \vernacular{yaámu[kulishila]
                    mú{\downstep}yáyi/muundu tá}  &   
                     \gloss{‘naming’}  &  \\

                     \vernacular{yaámu[lakhuulila]
                    mú{\downstep}yáyi/muundu tá}  &   
                     \gloss{‘releasing’}  &  \\

                     \vernacular{yaámu[seebulila]
                    mú{\downstep}yáyi/muundu tá}  &   
                     \gloss{‘saying bye
                    to’}  &  \\

                     \vernacular{
                    yaámu[reebareebela] mú{\downstep}yáyi/muundu
                    tá}  &   
                     \gloss{‘asking
                    (iter)’}  &  \\
\end{tabular}
%\caption{\nocaption}
     
\begin{tabular}{lll}  
  \multicolumn{2}{l}{
                     \vernacular{(709) /H/
                    C-Initial +OP + OP
                    } \gloss{‘s/he is not
                    ever/always...} } &  \\
\multicolumn{2}{l}{
                     \gloss{the boy
                    \ob mú{\downstep}yáyi\cb  / someone \ob muundu tá\cb  for him/her
                    for me’} } &  \\

                     \vernacular{yaá{\downstep}múú[ndeela]
                    mú{\downstep}’yáyi/muundu tá}  &   
                     \gloss{‘burying’}  &  \\

                     \vernacular{
                    yaá{\downstep}múú[mbechela] mú{\downstep}’yáyi/muundu
                    tá}  &   
                     \gloss{‘shaving’}  &  \\

                     \vernacular{
                    yaá{\downstep}múú[ndeerela] mú{\downstep}’yáyi/muundu
                    tá}  &   
                     \gloss{‘bringing’}  &  \\

                     \vernacular{
                    yaá{\downstep}múú[khalachila] mú{\downstep}’yáyi/muundu
                    tá}  &   
                     \gloss{‘cutting’}  &  \\

                     \vernacular{
                    yaá{\downstep}múú[sitaachila] mú{\downstep}’yáyi/muundu
                    tá}  &   
                     \gloss{‘accusing’}  &  \\

                     \vernacular{
                    yaá{\downstep}múú[mboolitsila] mú{\downstep}’yáyi/muundu
                    tá}  &   
                     \gloss{‘seducing’}  &  \\

                     \vernacular{
                    yaá{\downstep}múú[mbohololela] mú{\downstep}’yáyi/muundu
                    tá}  &   
                     \gloss{‘untying’}  &  \\
\end{tabular}
%\caption{\nocaption}
     
\begin{tabular}{lll}  
  \multicolumn{2}{l}{
                     \vernacular{(710) /Ø/
                    C-Initial +OP + OP
                    } \gloss{‘s/he is not
                    ever/always...} } &  \\
\multicolumn{2}{l}{
                     \gloss{the boy
                    \ob mú{\downstep}yáyi\cb  / someone \ob muundu tá\cb  for him/her
                    for me’} } &  \\

                     \vernacular{yaá{\downstep}múú[nziila]
                    mú{\downstep}yáyi/muundu tá}  &   
                     \gloss{‘going for’}  &  \\

                     \vernacular{
                    yaá{\downstep}múú[ndeshela] mú{\downstep}yáyi/muundu
                    tá}  &   
                     \gloss{‘leaving’}  &  \\

                     \vernacular{
                    yaá{\downstep}múú[noondela] mú{\downstep}yáyi/muundu
                    tá}  &   
                     \gloss{‘following’}  &  \\

                     \vernacular{
                    yaá{\downstep}múú[ngulishila] mú{\downstep}yáyi/muundu
                    tá}  &   
                     \gloss{‘naming’}  &  \\

                     \vernacular{
                    yaá{\downstep}múú[ndakhuulila] mú{\downstep}yáyi/muundu
                    tá}  &   
                     \gloss{‘releasing’}  &  \\

                     \vernacular{
                    yaá{\downstep}múú[seebulila] mú{\downstep}yáyi/muundu
                    tá}  &   
                     \gloss{‘saying bye
                    to’}  &  \\

                     \vernacular{
                    yaá{\downstep}múú[mboombelitsila] mú{\downstep}yáyi/muundu
                    tá}  &   
                     \gloss{‘comforting’}  &  \\

                     \vernacular{
                    yaá{\downstep}múú[siinjilitsila] mú{\downstep}yáyi/muundu
                    tá}  &   
                     \gloss{
                    ‘making...stand’}  &  \\
\end{tabular}
%\caption{\nocaption}
    

\section{Enclitics}\label{sec:sEnclitics}

Prompts in this section of the questionnaire were
            generated to investigate how the presence of the
            H-toned enclitic \vernacular{khú} \gloss{‘a bit’}and the
            toneless enclitic \vernacular{tsa} \gloss{‘just’}interact
            with verb tone.

 This section includes eight paradigms for each of
            the 34 constructions considered in the study. These
            include sets of consonant-initial /H/ and /Ø/ verbs
            with and without a CV- object prefix, with \vernacular{khú}or
            with \vernacular{tsa}.
            Though the parallel paradigms with \vernacular{khú}and \vernacular{tsa were presented
            and recorded separately, they are collapsed below to
            conserve space.}

 The data in this appendix could not be included in
            the thesis for lack of time. The transcriptions below
            reflect the tonal properties of each construction as
            described above, and no attempt has yet been made to
            adapt the transcriptions according to how enclitics
            influence verb tone. Whether the H of the enclitic is
            downstepped relative to verb-final Hs is also not
            reflected in the transcriptions below. 


\subsection{Near Future}\label{sec:sEncNearFut}


\begin{tabular}{lll}  
  \multicolumn{2}{l}{
                   \vernacular{(711/715) /H/
                  C-Initial} \gloss{‘s/he will...a bit
                  \ob khú\cb ’}/} &  \\
\multicolumn{2}{l}{
                     \gloss{‘s/he will just
                    \ob tsa\cb ...’} } &  \\

                     \vernacular{ala[khwá]
                    khú/tsa}  &   
                     \gloss{‘pay dowry’}  &  \\

                     \vernacular{ala[lúma]
                    khú/tsa}  &   
                     \gloss{‘bite’}  &  \\

                     \vernacular{ala[téekha]
                    khú/tsa}  &   
                     \gloss{‘cook’}  &  \\

                     \vernacular{ala[khálaka]
                    khú/tsa}  &   
                     \gloss{‘cut’}  &  \\

                     \vernacular{ala[kálaanga]
                    khú/tsa}  &   
                     \gloss{‘fry’}  &  \\

                     \vernacular{ala[bóolitsa]
                    khú/tsa}  &   
                     \gloss{‘seduce’}  &  \\

                     \vernacular{ala[tsúunzuuna]
                    khú/tsa}  &   
                     \gloss{‘suck’}  &  \\

                     \vernacular{ala[bóyong’ana]
                    khú/tsa}  &   
                     \gloss{‘go around’}  &  \\
\end{tabular}
%\caption{\nocaption}
     
\begin{tabular}{lll}  
  \multicolumn{2}{l}{
                   \vernacular{(712/716) /Ø/
                  C-Initial} \gloss{‘s/he will...a bit
                  \ob khú\cb ’}/} &  \\
\multicolumn{2}{l}{
                     \gloss{‘s/he will just
                    \ob tsa\cb ...’} } &  \\

                     \vernacular{ala[kwa]
                    khú/tsa}  &   
                     \gloss{‘fall’}  &  \\

                     \vernacular{ala[lekha]
                    khú/tsa}  &   
                     \gloss{‘leave’}  &  \\

                     \vernacular{ala[reeba]
                    khú/tsa}  &   
                     \gloss{‘ask’}  &  \\

                     \vernacular{ala[kulikha]
                    khú/tsa}  &   
                     \gloss{‘name’}  &  \\

                     \vernacular{ala[lakhuula]
                    khú/tsa}  &   
                     \gloss{‘release’}  &  \\

                     \vernacular{ala[seebula]
                    khú/tsa}  &   
                     \gloss{‘say bye’}  &  \\

                     \vernacular{ala[kalushitsa]
                    khú/tsa}  &   
                     \gloss{‘return’}  &  \\

                     \vernacular{ala[hoombelitsa]
                    khú/tsa}  &   
                     \gloss{‘comfort’}  &  \\
\end{tabular}
%\caption{\nocaption}
     
\begin{tabular}{lll}  
  \multicolumn{2}{l}{
                   \vernacular{(713/717) /H/
                  C-Initial} \gloss{‘s/he
                  will...him/her a bit \ob khú\cb ’}/} &  \\
\multicolumn{2}{l}{
                     \gloss{‘s/he will just
                    \ob tsa\cb ...him/her’} } &  \\

                     \vernacular{alamú[khwa]
                    khú/tsa}  &   
                     \gloss{‘pay (her)
                    dowry’}  &  \\

                     \vernacular{alamú[beka]
                    khú/tsa}  &   
                     \gloss{‘shave’}  &  \\

                     \vernacular{alamú[leera]
                    khú/tsa}  &   
                     \gloss{‘bring’}  &  \\

                     \vernacular{alamú[khalaka]
                    khú/tsa}  &   
                     \gloss{‘cut’}  &  \\

                     \vernacular{alamú[sitaaka]
                    khú/tsa}  &   
                     \gloss{‘accuse’}  &  \\

                     \vernacular{alamú[boolitsa]
                    khú/tsa}  &   
                     \gloss{‘seduce’}  &  \\

                     \vernacular{alamú[tsuunzuuna]
                    khú/tsa}  &   
                     \gloss{‘suck’}  &  \\

                     \vernacular{alamú[boyong’ana]
                    khú/tsa}  &   
                     \gloss{‘go around’}  &  \\
\end{tabular}
%\caption{\nocaption}
     
\begin{tabular}{lll}  
  \multicolumn{2}{l}{
                   \vernacular{(714/718) /Ø/
                  C-Initial} \gloss{‘s/he
                  will...him/her a bit \ob khú\cb ’}/} &  \\
\multicolumn{2}{l}{
                     \gloss{‘s/he will just
                    \ob tsa\cb ...him/her’} } &  \\

                     \vernacular{alamú[tsia]
                    khú/tsa}  &   
                     \gloss{‘go for’}  &  \\

                     \vernacular{alamú[lekha]
                    khú/tsa}  &   
                     \gloss{‘leave’}  &  \\

                     \vernacular{alamú[loonda]
                    khú/tsa}  &   
                     \gloss{‘ask’}  &  \\

                     \vernacular{alamú[kulikha]
                    khú/tsa}  &   
                     \gloss{‘name’}  &  \\

                     \vernacular{alamú[lakhuula]
                    khú/tsa}  &   
                     \gloss{‘release’}  &  \\

                     \vernacular{alamú[seebula]
                    khú/tsa}  &   
                     \gloss{‘say bye’}  &  \\

                     \vernacular{alamú[kalushitsa]
                    khú/tsa}  &   
                     \gloss{‘return’}  &  \\

                     \vernacular{
                    alamú[hoombelitsa] khú/tsa}  &   
                     \gloss{‘comfort’}  &  \\
\end{tabular}
%\caption{\nocaption}
    

\subsection{Near Future Negative}\label{sec:sEncNearFutNeg}


\begin{tabular}{lll}  
  \multicolumn{2}{l}{
                   \vernacular{(719/723) /H/
                  C-Initial} \gloss{‘s/he will not...a
                  bit \ob khú\cb ’}/} &  \\
\multicolumn{2}{l}{
                     \gloss{‘s/he will not
                    just \ob tsa\cb ...’} } &  \\

                     \vernacular{ala[khwá]
                    khú/tsa tá}  &   
                     \gloss{‘pay dowry’}  &  \\

                     \vernacular{ala[lúma]
                    khú/tsa tá}  &   
                     \gloss{‘bite’}  &  \\

                     \vernacular{ala[téekha]
                    khú/tsa tá}  &   
                     \gloss{‘cook’}  &  \\

                     \vernacular{ala[khálaka]
                    khú/tsa tá}  &   
                     \gloss{‘cut’}  &  \\

                     \vernacular{ala[kálaanga]
                    khú/tsa tá}  &   
                     \gloss{‘fry’}  &  \\

                     \vernacular{ala[bóolitsa]
                    khú/tsa tá}  &   
                     \gloss{‘seduce’}  &  \\

                     \vernacular{ala[tsúunzuuna]
                    khú/tsa tá}  &   
                     \gloss{‘suck’}  &  \\

                     \vernacular{ala[bóyong’ana]
                    khú/tsa tá}  &   
                     \gloss{‘go around’}  &  \\
\end{tabular}
%\caption{\nocaption}
     
\begin{tabular}{lll}  
  \multicolumn{2}{l}{
                   \vernacular{(720/724) /Ø/
                  C-Initial} \gloss{‘s/he will not...a
                  bit \ob khú\cb ’}/} &  \\
\multicolumn{2}{l}{
                     \gloss{‘s/he will not
                    just \ob tsa\cb ...’} } &  \\

                     \vernacular{ala[kwa] khú/tsa
                    tá}  &   
                     \gloss{‘fall’}  &  \\

                     \vernacular{ala[lekha]
                    khú/tsa tá}  &   
                     \gloss{‘leave’}  &  \\

                     \vernacular{ala[reeba]
                    khú/tsa tá}  &   
                     \gloss{‘ask’}  &  \\

                     \vernacular{ala[kulikha]
                    khú/tsa tá}  &   
                     \gloss{‘name’}  &  \\

                     \vernacular{ala[lakhuula]
                    khú/tsa tá}  &   
                     \gloss{‘release’}  &  \\

                     \vernacular{ala[seebula]
                    khú/tsa tá}  &   
                     \gloss{‘say bye’}  &  \\

                     \vernacular{ala[kalushitsa]
                    khú/tsa tá}  &   
                     \gloss{‘return’}  &  \\

                     \vernacular{ala[hoombelitsa]
                    khú/tsa tá}  &   
                     \gloss{‘comfort’}  &  \\
\end{tabular}
%\caption{\nocaption}
     
\begin{tabular}{lll}  
  \multicolumn{2}{l}{
                   \vernacular{(721/725) /H/
                  C-Initial} \gloss{‘s/he will
                  not...him/her a bit \ob khú\cb ’}/} &  \\
\multicolumn{2}{l}{
                     \gloss{‘s/he will not
                    just \ob tsa\cb ...him/her’} } &  \\

                     \vernacular{alamú[khwa]
                    khú/tsa tá}  &   
                     \gloss{‘pay (her)
                    dowry’}  &  \\

                     \vernacular{alamú[beka]
                    khú/tsa tá}  &   
                     \gloss{‘shave’}  &  \\

                     \vernacular{alamú[leera]
                    khú/tsa tá}  &   
                     \gloss{‘bring’}  &  \\

                     \vernacular{alamú[khalaka]
                    khú/tsa tá}  &   
                     \gloss{‘cut’}  &  \\

                     \vernacular{alamú[sitaaka]
                    khú/tsa tá}  &   
                     \gloss{‘accuse’}  &  \\

                     \vernacular{alamú[boolitsa]
                    khú/tsa tá}  &   
                     \gloss{‘seduce’}  &  \\

                     \vernacular{alamú[tsuunzuuna]
                    khú/tsa tá}  &   
                     \gloss{‘suck’}  &  \\

                     \vernacular{alamú[boyong’ana]
                    khú/tsa tá}  &   
                     \gloss{‘go around’}  &  \\
\end{tabular}
%\caption{\nocaption}
     
\begin{tabular}{lll}  
  \multicolumn{2}{l}{
                   \vernacular{(722/726) /Ø/
                  C-Initial} \gloss{‘s/he will
                  not...him/her a bit \ob khú\cb ’}/} &  \\
\multicolumn{2}{l}{
                     \gloss{‘s/he will not
                    just \ob tsa\cb ...him/her’} } &  \\

                     \vernacular{alamú[tsia]
                    khú/tsa tá}  &   
                     \gloss{‘go for’}  &  \\

                     \vernacular{alamú[lekha]
                    khú/tsa tá}  &   
                     \gloss{‘leave’}  &  \\

                     \vernacular{alamú[loonda]
                    khú/tsa tá}  &   
                     \gloss{‘ask’}  &  \\

                     \vernacular{alamú[kulikha]
                    khú/tsa tá}  &   
                     \gloss{‘name’}  &  \\

                     \vernacular{alamú[lakhuula]
                    khú/tsa tá}  &   
                     \gloss{‘release’}  &  \\

                     \vernacular{alamú[seebula]
                    khú/tsa tá}  &   
                     \gloss{‘say bye’}  &  \\

                     \vernacular{alamú[kalushitsa]
                    khú/tsa tá}  &   
                     \gloss{‘return’}  &  \\

                     \vernacular{
                    alamú[hoombelitsa] khú/tsa tá}  &   
                     \gloss{‘comfort’}  &  \\
\end{tabular}
%\caption{\nocaption}
    

\subsection{Remote Past}\label{sec:sEncRemPast}


\begin{tabular}{lll}  
  \multicolumn{2}{l}{
                   \vernacular{(727/731) /H/
                  C-Initial} \gloss{‘s/he...a bit
                  \ob khú\cb ’}/} &  \\
\multicolumn{2}{l}{
                     \gloss{‘s/he just
                    \ob tsa\cb ...’} } &  \\

                     \vernacular{yaa[khwá]
                    khú/tsa}  &   
                     \gloss{‘paid dowry’}  &  \\

                     \vernacular{yaa[lúma]
                    khú/tsa}  &   
                     \gloss{‘bit’}  &  \\

                     \vernacular{yaa[téekha]
                    khú/tsa}  &   
                     \gloss{‘cooked’}  &  \\

                     \vernacular{yaa[khálaka]
                    khú/tsa}  &   
                     \gloss{‘cut’}  &  \\

                     \vernacular{yaa[kálaanga]
                    khú/tsa}  &   
                     \gloss{‘fried’}  &  \\

                     \vernacular{yaa[bóolitsa]
                    khú/tsa}  &   
                     \gloss{‘seduced’}  &  \\

                     \vernacular{yaa[tsúunzuuna]
                    khú/tsa}  &   
                     \gloss{‘sucked’}  &  \\

                     \vernacular{yaa[bóyong’ana]
                    khú/tsa}  &   
                     \gloss{‘went
                    around’}  &  \\
\end{tabular}
%\caption{\nocaption}
     
\begin{tabular}{lll}  
  \multicolumn{2}{l}{
                   \vernacular{(728/732) /Ø/
                  C-Initial} \gloss{‘s/he...a bit
                  \ob khú\cb ’}/} &  \\
\multicolumn{2}{l}{
                     \gloss{‘s/he just
                    \ob tsa\cb ...’} } &  \\

                     \vernacular{yaa[kwá]
                    khú/tsa}  &   
                     \gloss{‘fell’}  &  \\

                     \vernacular{yaa[lékha]
                    khú/tsa}  &   
                     \gloss{‘left’}  &  \\

                     \vernacular{yaa[réeba]
                    khú/tsa}  &   
                     \gloss{‘asked’}  &  \\

                     \vernacular{yaa[kúlikha]
                    khú/tsa}  &   
                     \gloss{‘named’}  &  \\

                     \vernacular{yaa[lákhuula]
                    khú/tsa}  &   
                     \gloss{‘released’}  &  \\

                     \vernacular{yaa[séebula]
                    khú/tsa}  &   
                     \gloss{‘said bye’}  &  \\

                     \vernacular{yaa[kálushitsa]
                    khú/tsa}  &   
                     \gloss{‘returned’}  &  \\

                     \vernacular{yaa[hóombelitsa]
                    khú/tsa}  &   
                     \gloss{‘comforted’}  &  \\
\end{tabular}
%\caption{\nocaption}
     
\begin{tabular}{lll}  
  \multicolumn{2}{l}{
                   \vernacular{(729/733) /H/
                  C-Initial} \gloss{‘s/he...him/her a
                  bit \ob khú\cb ’}/} &  \\
\multicolumn{2}{l}{
                     \gloss{‘s/he just
                    \ob tsa\cb ...him/her’} } &  \\

                     \vernacular{yaamú[khwa]
                    khú/tsa}  &   
                     \gloss{‘paid (her)
                    dowry’}  &  \\

                     \vernacular{yaamú[beka]
                    khú/tsa}  &   
                     \gloss{‘shaved’}  &  \\

                     \vernacular{yaamú[leera]
                    khú/tsa}  &   
                     \gloss{‘brought’}  &  \\

                     \vernacular{yaamú[khalaka]
                    khú/tsa}  &   
                     \gloss{‘cut’}  &  \\

                     \vernacular{yaamú[sitaaka]
                    khú/tsa}  &   
                     \gloss{‘accused’}  &  \\

                     \vernacular{yaamú[boolitsa]
                    khú/tsa}  &   
                     \gloss{‘seduced’}  &  \\

                     \vernacular{yaamú[tsuunzuuna]
                    khú/tsa}  &   
                     \gloss{‘sucked’}  &  \\

                     \vernacular{yaamú[boyong’ana]
                    khú/tsa}  &   
                     \gloss{‘went
                    around’}  &  \\
\end{tabular}
%\caption{\nocaption}
     
\begin{tabular}{lll}  
  \multicolumn{2}{l}{
                   \vernacular{(730/734) /Ø/
                  C-Initial} \gloss{‘s/he...him/her a
                  bit \ob khú\cb ’}/} &  \\
\multicolumn{2}{l}{
                     \gloss{‘s/he just
                    \ob tsa\cb ...him/her’} } &  \\

                     \vernacular{yaamú[{\downstep}tsíá]
                    khú/tsa}  &   
                     \gloss{‘went for’}  &  \\

                     \vernacular{yaamú[{\downstep}lékhá]
                    khú/tsa}  &   
                     \gloss{‘left’}  &  \\

                     \vernacular{yaamú[{\downstep}lóónda]
                    khú/tsa}  &   
                     \gloss{‘asked’}  &  \\

                     \vernacular{yaamú[{\downstep}kúlíkha]
                    khú/tsa}  &   
                     \gloss{‘named’}  &  \\

                     \vernacular{
                    yaamú[{\downstep}lákhúula] khú/tsa}  &   
                     \gloss{‘released’}  &  \\

                     \vernacular{yaamú[{\downstep}séébula]
                    khú/tsa}  &   
                     \gloss{‘said bye’}  &  \\

                     \vernacular{
                    yaamú[{\downstep}kálúshitsa] khú/tsa}  &   
                     \gloss{‘returned’}  &  \\

                     \vernacular{
                    yaamú[{\downstep}hóómbélitsa] khú/tsa}  &   
                     \gloss{‘comforted’}  &  \\
\end{tabular}
%\caption{\nocaption}
    

\subsection{Remote Past Negative}\label{sec:sEncRemPastNeg}


\begin{tabular}{lll}  
  \multicolumn{2}{l}{
                   \vernacular{(735/739) /H/
                  C-Initial} \gloss{‘s/he did not...a
                  bit \ob khú\cb ’}/} &  \\
\multicolumn{2}{l}{
                     \gloss{‘s/he did not
                    just \ob tsa\cb ...’} } &  \\

                     \vernacular{yaa[khwá]
                    khú/tsa tá}  &   
                     \gloss{‘pay dowry’}  &  \\

                     \vernacular{yaa[lúma]
                    khú/tsa tá}  &   
                     \gloss{‘bite’}  &  \\

                     \vernacular{yaa[téekha]
                    khú/tsa tá}  &   
                     \gloss{‘cook’}  &  \\

                     \vernacular{yaa[khálaka]
                    khú/tsa tá}  &   
                     \gloss{‘cut’}  &  \\

                     \vernacular{yaa[kálaanga]
                    khú/tsa tá}  &   
                     \gloss{‘fry’}  &  \\

                     \vernacular{yaa[bóolitsa]
                    khú/tsa tá}  &   
                     \gloss{‘seduce’}  &  \\

                     \vernacular{yaa[tsúunzuuna]
                    khú/tsa tá}  &   
                     \gloss{‘suck’}  &  \\

                     \vernacular{yaa[bóyong’ana]
                    khú/tsa tá}  &   
                     \gloss{‘go around’}  &  \\
\end{tabular}
%\caption{\nocaption}
     
\begin{tabular}{lll}  
  \multicolumn{2}{l}{
                   \vernacular{(736/740) /Ø/
                  C-Initial} \gloss{‘s/he did not...a
                  bit \ob khú\cb ’}/} &  \\
\multicolumn{2}{l}{
                     \gloss{‘s/he did not
                    just \ob tsa\cb ...’} } &  \\

                     \vernacular{yaa[kwá] khú/tsa
                    tá}  &   
                     \gloss{‘fall’}  &  \\

                     \vernacular{yaa[lékha]
                    khú/tsa tá}  &   
                     \gloss{‘leave’}  &  \\

                     \vernacular{yaa[réeba]
                    khú/tsa tá}  &   
                     \gloss{‘ask’}  &  \\

                     \vernacular{yaa[kúlikha]
                    khú/tsa tá}  &   
                     \gloss{‘name’}  &  \\

                     \vernacular{yaa[lákhuula]
                    khú/tsa tá}  &   
                     \gloss{‘release’}  &  \\

                     \vernacular{yaa[séebula]
                    khú/tsa tá}  &   
                     \gloss{‘say bye’}  &  \\

                     \vernacular{yaa[kálushitsa]
                    khú/tsa tá}  &   
                     \gloss{‘return’}  &  \\

                     \vernacular{yaa[hóombelitsa]
                    khú/tsa tá}  &   
                     \gloss{‘comfort’}  &  \\
\end{tabular}
%\caption{\nocaption}
     
\begin{tabular}{lll}  
  \multicolumn{2}{l}{
                   \vernacular{(737/741) /H/
                  C-Initial} \gloss{‘s/he did
                  not...him/her a bit \ob khú\cb ’}/} &  \\
\multicolumn{2}{l}{
                     \gloss{‘s/he did not
                    just \ob tsa\cb ...him/her’} } &  \\

                     \vernacular{yaamú[ra]
                    khú/tsa tá}  &   
                     \gloss{‘pay (her)
                    dowry’}  &  \\

                     \vernacular{yaamú[beka]
                    khú/tsa tá}  &   
                     \gloss{‘shave’}  &  \\

                     \vernacular{yaamú[leera]
                    khú/tsa tá}  &   
                     \gloss{‘bring’}  &  \\

                     \vernacular{yaamú[khalaka]
                    khú/tsa tá}  &   
                     \gloss{‘cut’}  &  \\

                     \vernacular{yaamú[sitaaka]
                    khú/tsa tá}  &   
                     \gloss{‘accuse’}  &  \\

                     \vernacular{yaamú[boolitsa]
                    khú/tsa tá}  &   
                     \gloss{‘seduce’}  &  \\

                     \vernacular{yaamú[tsuunzuuna]
                    khú/tsa tá}  &   
                     \gloss{‘suck’}  &  \\

                     \vernacular{yaamú[boyong’ana]
                    khú/tsa tá}  &   
                     \gloss{‘go around’}  &  \\
\end{tabular}
%\caption{\nocaption}
     
\begin{tabular}{lll}  
  \multicolumn{2}{l}{
                   \vernacular{(738/742) /Ø/
                  C-Initial} \gloss{‘s/he did
                  not...him/her a bit \ob khú\cb ’}/} &  \\
\multicolumn{2}{l}{
                     \gloss{‘s/he did not
                    just \ob tsa\cb ...him/her’} } &  \\

                     \vernacular{yaamú[{\downstep}tsíá]
                    khú/tsa tá}  &   
                     \gloss{‘go for’}  &  \\

                     \vernacular{yaamú[{\downstep}lékhá]
                    khú/tsa tá}  &   
                     \gloss{‘leave’}  &  \\

                     \vernacular{yaamú[{\downstep}lóónda]
                    khú/tsa tá}  &   
                     \gloss{‘ask’}  &  \\

                     \vernacular{yaamú[{\downstep}kúlíkha]
                    khú/tsa tá}  &   
                     \gloss{‘name’}  &  \\

                     \vernacular{
                    yaamú[{\downstep}lákhúula] khú/tsa tá}  &   
                     \gloss{‘release’}  &  \\

                     \vernacular{yaamú[{\downstep}séébula]
                    khú/tsa tá}  &   
                     \gloss{‘say bye’}  &  \\

                     \vernacular{
                    yaamú[{\downstep}kálúshitsa] khú/tsa tá}  &   
                     \gloss{‘return’}  &  \\

                     \vernacular{
                    yaamú[{\downstep}hóómbélitsa] khú/tsa tá}  &   
                     \gloss{‘comfort’}  &  \\
\end{tabular}
%\caption{\nocaption}
    

\subsection{Immediate Past}\label{sec:sEncImmPast}


\begin{tabular}{lll}  
  \multicolumn{2}{l}{
                   \vernacular{(743/747) /H/
                  C-Initial} \gloss{‘s/he just...a bit
                  \ob khú\cb ’}/} &  \\
\multicolumn{2}{l}{
                     \gloss{‘s/he just
                    \ob tsa\cb ...’} } &  \\

                     \vernacular{yá{\downstep}khá[khwá]
                    khú/tsa}  &   
                     \gloss{‘paid dowry’}  &  \\

                     \vernacular{yá{\downstep}khá[lúma]
                    khú/tsa}  &   
                     \gloss{‘bit’}  &  \\

                     \vernacular{yá{\downstep}khá[téekha]
                    khú/tsa}  &   
                     \gloss{‘cooked’}  &  \\

                     \vernacular{yá{\downstep}khá[khálaka]
                    khú/tsa}  &   
                     \gloss{‘cut’}  &  \\

                     \vernacular{
                    yá{\downstep}khá[kálaanga] khú/tsa}  &   
                     \gloss{‘fried’}  &  \\

                     \vernacular{
                    yá{\downstep}khá[bóolitsa] khú/tsa}  &   
                     \gloss{‘seduced’}  &  \\

                     \vernacular{
                    yá{\downstep}khá[tsúunzuuna] khú/tsa}  &   
                     \gloss{‘sucked’}  &  \\

                     \vernacular{
                    yá{\downstep}khá[bóyong’ana] khú/tsa}  &   
                     \gloss{‘went
                    around’}  &  \\
\end{tabular}
%\caption{\nocaption}
     
\begin{tabular}{lll}  
  \multicolumn{2}{l}{
                   \vernacular{(744/748) /Ø/
                  C-Initial} \gloss{‘s/he just...a bit
                  \ob khú\cb ’}/} &  \\
\multicolumn{2}{l}{
                     \gloss{‘s/he just
                    \ob tsa\cb ...’} } &  \\

                     \vernacular{yákha[kwa]
                    khú/tsa}  &   
                     \gloss{‘fell’}  &  \\

                     \vernacular{yákha[lekha]
                    khú/tsa}  &   
                     \gloss{‘left’}  &  \\

                     \vernacular{yákha[reeba]
                    khú/tsa}  &   
                     \gloss{‘asked’}  &  \\

                     \vernacular{yákha[kulikha]
                    khú/tsa}  &   
                     \gloss{‘named’}  &  \\

                     \vernacular{yákha[lakhuula]
                    khú/tsa}  &   
                     \gloss{‘released’}  &  \\

                     \vernacular{yákha[seebula]
                    khú/tsa}  &   
                     \gloss{‘said bye’}  &  \\

                     \vernacular{yákha[kalushitsa]
                    khú/tsa}  &   
                     \gloss{‘returned’}  &  \\

                     \vernacular{
                    yákha[hoombelitsa] khú/tsa}  &   
                     \gloss{‘comforted’}  &  \\
\end{tabular}
%\caption{\nocaption}
     
\begin{tabular}{lll}  
  \multicolumn{2}{l}{
                   \vernacular{(745/749) /H/
                  C-Initial} \gloss{‘s/he
                  just...him/her a bit \ob khú\cb ’}/} &  \\
\multicolumn{2}{l}{
                     \gloss{‘s/he just
                    \ob tsa\cb ...him/her’} } &  \\

                     \vernacular{yá{\downstep}khámú[khwa]
                    khú/tsa}  &   
                     \gloss{‘paid (her)
                    dowry’}  &  \\

                     \vernacular{yá{\downstep}khámú[beka]
                    khú/tsa}  &   
                     \gloss{‘shaved’}  &  \\

                     \vernacular{yá{\downstep}khámú[leera]
                    khú/tsa}  &   
                     \gloss{‘brought’}  &  \\

                     \vernacular{
                    yá{\downstep}khámú[khalaka] khú/tsa}  &   
                     \gloss{‘cut’}  &  \\

                     \vernacular{
                    yá{\downstep}khámú[sitaaka] khú/tsa}  &   
                     \gloss{‘accused’}  &  \\

                     \vernacular{
                    yá{\downstep}khámú[boolitsa] khú/tsa}  &   
                     \gloss{‘seduced’}  &  \\

                     \vernacular{
                    yá{\downstep}khámú[tsuunzuuna] khú/tsa}  &   
                     \gloss{‘sucked’}  &  \\

                     \vernacular{
                    yá{\downstep}khámú[boyong’ana] khú/tsa}  &   
                     \gloss{‘went
                    around’}  &  \\
\end{tabular}
%\caption{\nocaption}
     
\begin{tabular}{lll}  
  \multicolumn{2}{l}{
                   \vernacular{(746/750) /Ø/
                  C-Initial} \gloss{‘s/he
                  just...him/her a bit \ob khú\cb ’}/} &  \\
\multicolumn{2}{l}{
                     \gloss{‘s/he just
                    \ob tsa\cb ...him/her’} } &  \\

                     \vernacular{yá{\downstep}khámú[tsia]
                    khú/tsa}  &   
                     \gloss{‘went for’}  &  \\

                     \vernacular{yá{\downstep}khámú[lekha]
                    khú/tsa}  &   
                     \gloss{‘left’}  &  \\

                     \vernacular{
                    yá{\downstep}khámú[loonda] khú/tsa}  &   
                     \gloss{‘asked’}  &  \\

                     \vernacular{
                    yá{\downstep}khámú[kulikha] khú/tsa}  &   
                     \gloss{‘named’}  &  \\

                     \vernacular{
                    yá{\downstep}khámú[lakhuula] khú/tsa}  &   
                     \gloss{‘released’}  &  \\

                     \vernacular{
                    yá{\downstep}khámú[seebula] khú/tsa}  &   
                     \gloss{‘said bye’}  &  \\

                     \vernacular{
                    yá{\downstep}khámú[kalushitsa] khú/tsa}  &   
                     \gloss{‘returned’}  &  \\

                     \vernacular{
                    yá{\downstep}khámú[hoombelitsa] khú/tsa}  &   
                     \gloss{‘comforted’}  &  \\
\end{tabular}
%\caption{\nocaption}
    

\subsection{Immediate Past Negative}\label{sec:sEncImmPastNeg}


\begin{tabular}{lll}  
  \multicolumn{2}{l}{
                   \vernacular{(751/755) /H/
                  C-Initial} \gloss{‘s/he did not
                  just...a bit \ob khú\cb ’}/} &  \\
\multicolumn{2}{l}{
                     \gloss{‘s/he did not
                    just \ob tsa\cb ...’} } &  \\

                     \vernacular{yá{\downstep}khá[khwá]
                    khú/tsa tá}  &   
                     \gloss{‘pay dowry’}  &  \\

                     \vernacular{yá{\downstep}khá[lúma]
                    khú/tsa tá}  &   
                     \gloss{‘bite’}  &  \\

                     \vernacular{yá{\downstep}khá[téekha]
                    khú/tsa tá}  &   
                     \gloss{‘cook’}  &  \\

                     \vernacular{yá{\downstep}khá[khálaka]
                    khú/tsa tá}  &   
                     \gloss{‘cut’}  &  \\

                     \vernacular{
                    yá{\downstep}khá[kálaanga] khú/tsa tá}  &   
                     \gloss{‘fry’}  &  \\

                     \vernacular{
                    yá{\downstep}khá[bóolitsa] khú/tsa tá}  &   
                     \gloss{‘seduce’}  &  \\

                     \vernacular{
                    yá{\downstep}khá[tsúunzuuna] khú/tsa tá}  &   
                     \gloss{‘suck’}  &  \\

                     \vernacular{
                    yá{\downstep}khá[bóyong’ana] khú/tsa tá}  &   
                     \gloss{‘go around’}  &  \\
\end{tabular}
%\caption{\nocaption}
     
\begin{tabular}{lll}  
  \multicolumn{2}{l}{
                   \vernacular{(752/756) /Ø/
                  C-Initial} \gloss{‘s/he did not
                  just...a bit \ob khú\cb ’}/} &  \\
\multicolumn{2}{l}{
                     \gloss{‘s/he did not
                    just \ob tsa\cb ...’} } &  \\

                     \vernacular{yákha[kwa]
                    khú/tsa tá}  &   
                     \gloss{‘fall’}  &  \\

                     \vernacular{yákha[lekha]
                    khú/tsa tá}  &   
                     \gloss{‘leave’}  &  \\

                     \vernacular{yákha[reeba]
                    khú/tsa tá}  &   
                     \gloss{‘ask’}  &  \\

                     \vernacular{yákha[kulikha]
                    khú/tsa tá}  &   
                     \gloss{‘name’}  &  \\

                     \vernacular{yákha[lakhuula]
                    khú/tsa tá}  &   
                     \gloss{‘release’}  &  \\

                     \vernacular{yákha[seebula]
                    khú/tsa tá}  &   
                     \gloss{‘say bye’}  &  \\

                     \vernacular{yákha[kalushitsa]
                    khú/tsa tá}  &   
                     \gloss{‘return’}  &  \\

                     \vernacular{
                    yákha[hoombelitsa] khú/tsa tá}  &   
                     \gloss{‘comfort’}  &  \\
\end{tabular}
%\caption{\nocaption}
     
\begin{tabular}{lll}  
  \multicolumn{2}{l}{
                   \vernacular{(753/757) /H/
                  C-Initial} \gloss{‘s/he did not
                  just...him/her a bit \ob khú\cb ’}/} &  \\
\multicolumn{2}{l}{
                     \gloss{‘s/he did not
                    just \ob tsa\cb ...him/her’} } &  \\

                     \vernacular{yá{\downstep}khámú[ra]
                    khú/tsa tá}  &   
                     \gloss{‘bury’}  &  \\

                     \vernacular{yá{\downstep}khámú[beka]
                    khú/tsa tá}  &   
                     \gloss{‘shave’}  &  \\

                     \vernacular{yá{\downstep}khámú[leera]
                    khú/tsa tá}  &   
                     \gloss{‘bring’}  &  \\

                     \vernacular{
                    yá{\downstep}khámú[khalaka] khú/tsa tá}  &   
                     \gloss{‘cut’}  &  \\

                     \vernacular{
                    yá{\downstep}khámú[sitaaka] khú/tsa tá}  &   
                     \gloss{‘accuse’}  &  \\

                     \vernacular{
                    yá{\downstep}khámú[boolitsa] khú/tsa tá}  &   
                     \gloss{‘seduce’}  &  \\

                     \vernacular{
                    yá{\downstep}khámú[tsuunzuuna] khú/tsa tá}  &   
                     \gloss{‘suck’}  &  \\

                     \vernacular{
                    yá{\downstep}khámú[boyong’ana] khú/tsa tá}  &   
                     \gloss{‘go around’}  &  \\
\end{tabular}
%\caption{\nocaption}
     
\begin{tabular}{lll}  
  \multicolumn{2}{l}{
                   \vernacular{(754/758) /Ø/
                  C-Initial} \gloss{‘s/he did not
                  just...him/her a bit \ob khú\cb ’}/} &  \\
\multicolumn{2}{l}{
                     \gloss{‘s/he did not
                    just \ob tsa\cb ...him/her’} } &  \\

                     \vernacular{yá{\downstep}khámú[tsia]
                    khú/tsa tá}  &   
                     \gloss{‘go for’}  &  \\

                     \vernacular{yá{\downstep}khámú[lekha]
                    khú/tsa tá}  &   
                     \gloss{‘leave’}  &  \\

                     \vernacular{
                    yá{\downstep}khámú[loonda] khú/tsa tá}  &   
                     \gloss{‘ask’}  &  \\

                     \vernacular{
                    yá{\downstep}khámú[kulikha] khú/tsa tá}  &   
                     \gloss{‘name’}  &  \\

                     \vernacular{
                    yá{\downstep}khámú[lakhuula] khú/tsa tá}  &   
                     \gloss{‘release’}  &  \\

                     \vernacular{
                    yá{\downstep}khámú[seebula] khú/tsa tá}  &   
                     \gloss{‘say bye’}  &  \\

                     \vernacular{
                    yá{\downstep}khámú[kalushitsa] khú/tsa tá}  &   
                     \gloss{‘return’}  &  \\

                     \vernacular{
                    yá{\downstep}khámú[hoombelitsa] khú/tsa
                    tá}  &   
                     \gloss{‘comfort’}  &  \\
\end{tabular}
%\caption{\nocaption}
    

\subsection{Remote Future
              }\label{sec:sEncRemFut}


\begin{tabular}{lll}  
  \multicolumn{2}{l}{
                   \vernacular{(759/763) /H/
                  C-Initial} \gloss{‘s/he just...a bit
                  \ob khú\cb ’}/} &  \\
\multicolumn{2}{l}{
                     \gloss{‘s/he just
                    \ob tsa\cb ...’} } &  \\

                     \vernacular{yá{\downstep}khá[khwí]
                    khú/tsa}  &   
                     \gloss{‘pay dowry’}  &  \\

                     \vernacular{yá{\downstep}khá[lúmɪ]
                    khú/tsa}  &   
                     \gloss{‘bite’}  &  \\

                     \vernacular{yá{\downstep}khá[téeshɛ]
                    khú/tsa}  &   
                     \gloss{‘cook’}  &  \\

                     \vernacular{
                    yá{\downstep}khá[khálachɛ] khú/tsa}  &   
                     \gloss{‘cut’}  &  \\

                     \vernacular{
                    yá{\downstep}khá[kálaanjɛ] khú/tsa}  &   
                     \gloss{‘fry’}  &  \\

                     \vernacular{
                    yá{\downstep}khá[bóolitsɪ] khú/tsa}  &   
                     \gloss{‘seduce’}  &  \\

                     \vernacular{
                    yá{\downstep}khá[tsúunzuunɪ] khú/tsa}  &   
                     \gloss{‘suck’}  &  \\

                     \vernacular{
                    yá{\downstep}khá[bóyong’anɛ] khú/tsa}  &   
                     \gloss{‘go around’}  &  \\
\end{tabular}
%\caption{\nocaption}
     
\begin{tabular}{lll}  
  \multicolumn{2}{l}{
                   \vernacular{(760/764) /Ø/
                  C-Initial} \gloss{‘s/he just...a bit
                  \ob khú\cb ’}/} &  \\
\multicolumn{2}{l}{
                     \gloss{‘s/he just
                    \ob tsa\cb ...’} } &  \\

                     \vernacular{yákha[kwi]
                    khú/tsa}  &   
                     \gloss{‘fall’}  &  \\

                     \vernacular{yákha[leshɛ]
                    khú/tsa}  &   
                     \gloss{‘leave’}  &  \\

                     \vernacular{yákha[reebɛ]
                    khú/tsa}  &   
                     \gloss{‘ask’}  &  \\

                     \vernacular{yákha[kulishɪ]
                    khú/tsa}  &   
                     \gloss{‘name’}  &  \\

                     \vernacular{yákha[lakhuulɪ]
                    khú/tsa}  &   
                     \gloss{‘release’}  &  \\

                     \vernacular{yákha[seebulɪ]
                    khú/tsa}  &   
                     \gloss{‘say bye’}  &  \\

                     \vernacular{yákha[kalushitsɪ]
                    khú/tsa}  &   
                     \gloss{‘return’}  &  \\

                     \vernacular{
                    yákha[hoombelitsɪ] khú/tsa}  &   
                     \gloss{‘comfort’}  &  \\
\end{tabular}
%\caption{\nocaption}
     
\begin{tabular}{lll}  
  \multicolumn{2}{l}{
                   \vernacular{(761/765) /H/
                  C-Initial} \gloss{‘s/he
                  just...him/her a bit \ob khú\cb ’}/} &  \\
\multicolumn{2}{l}{
                     \gloss{‘s/he just
                    \ob tsa\cb ...him/her’} } &  \\

                     \vernacular{yá{\downstep}khámú[rɛ]
                    khú/tsa}  &   
                     \gloss{‘bury’}  &  \\

                     \vernacular{yá{\downstep}khámú[bechɛ]
                    khú/tsa}  &   
                     \gloss{‘shave’}  &  \\

                     \vernacular{yá{\downstep}khámú[leerɛ]
                    khú/tsa}  &   
                     \gloss{‘bring’}  &  \\

                     \vernacular{
                    yá{\downstep}khámú[khalachɛ] khú/tsa}  &   
                     \gloss{‘cut’}  &  \\

                     \vernacular{
                    yá{\downstep}khámú[sitaachɛ] khú/tsa}  &   
                     \gloss{‘accuse’}  &  \\

                     \vernacular{
                    yá{\downstep}khámú[boolitsɪ] khú/tsa}  &   
                     \gloss{‘seduce’}  &  \\

                     \vernacular{
                    yá{\downstep}khámú[tsuunzuunɪ] khú/tsa}  &   
                     \gloss{‘suck’}  &  \\

                     \vernacular{
                    yá{\downstep}khámú[boyong’anɛ] khú/tsa}  &   
                     \gloss{‘go around’}  &  \\
\end{tabular}
%\caption{\nocaption}
     
\begin{tabular}{lll}  
  \multicolumn{2}{l}{
                   \vernacular{(762/766) /Ø/
                  C-Initial} \gloss{‘s/he
                  just...him/her a bit \ob khú\cb ’}/} &  \\
\multicolumn{2}{l}{
                     \gloss{‘s/he just
                    \ob tsa\cb ...him/her’} } &  \\

                     \vernacular{yá{\downstep}khámú[tsi]
                    khú/tsa}  &   
                     \gloss{‘go for’}  &  \\

                     \vernacular{yá{\downstep}khámú[leshɛ]
                    khú/tsa}  &   
                     \gloss{‘leave’}  &  \\

                     \vernacular{
                    yá{\downstep}khámú[loondɛ] khú/tsa}  &   
                     \gloss{‘ask’}  &  \\

                     \vernacular{
                    yá{\downstep}khámú[kulishɪ] khú/tsa}  &   
                     \gloss{‘name’}  &  \\

                     \vernacular{
                    yá{\downstep}khámú[lakhuulɪ] khú/tsa}  &   
                     \gloss{‘release’}  &  \\

                     \vernacular{
                    yá{\downstep}khámú[seebulɪ] khú/tsa}  &   
                     \gloss{‘say bye’}  &  \\

                     \vernacular{
                    yá{\downstep}khámú[kalushitsɪ] khú/tsa}  &   
                     \gloss{‘return’}  &  \\

                     \vernacular{
                    yá{\downstep}khámú[hoombelitsɪ] khú/tsa}  &   
                     \gloss{‘comfort’}  &  \\
\end{tabular}
%\caption{\nocaption}
    

\subsection{Present}\label{sec:sEncPres}


\begin{tabular}{lll}  
  \multicolumn{2}{l}{
                   \vernacular{(767/771) /H/
                  C-Initial} \gloss{‘s/he is...a bit
                  \ob khú\cb ’}/} &  \\
\multicolumn{2}{l}{
                     \gloss{‘s/he is just
                    \ob tsa\cb ...’} } &  \\

                     \vernacular{a[khweetsáángá]
                    khú/tsa}  &   
                     \gloss{‘paying
                    dowry’}  &  \\

                     \vernacular{a[lumaángá]
                    khú/tsa}  &   
                     \gloss{‘biting’}  &  \\

                     \vernacular{a[teekháángá]
                    khú/tsa}  &   
                     \gloss{‘cooking’}  &  \\

                     \vernacular{a[khalakáánga]
                    khú/tsa}  &   
                     \gloss{‘cutting’}  &  \\

                     \vernacular{a[kalaangáánga]
                    khú/tsa}  &   
                     \gloss{‘frying’}  &  \\

                     \vernacular{a[boolitsáánga]
                    khú/tsa}  &   
                     \gloss{‘seducing’}  &  \\

                     \vernacular{
                    a[tsuunzuunáánga] khú/tsa}  &   
                     \gloss{‘sucking’}  &  \\

                     \vernacular{
                    a[boyong’ánáanga] khú/tsa}  &   
                     \gloss{‘going
                    around’}  &  \\
\end{tabular}
%\caption{\nocaption}
     
\begin{tabular}{lll}  
  \multicolumn{2}{l}{
                   \vernacular{(768/772) /Ø/
                  C-Initial} \gloss{‘s/he is...a bit
                  \ob khú\cb ’}/} &  \\
\multicolumn{2}{l}{
                     \gloss{‘s/he is just
                    \ob tsa\cb ...’} } &  \\

                     \vernacular{a[kwiítsáanga]
                    khú/tsa}  &   
                     \gloss{‘falling’}  &  \\

                     \vernacular{a[lekháanga]
                    khú/tsa}  &   
                     \gloss{‘leaving’}  &  \\

                     \vernacular{a[reébáanga]
                    khú/tsa}  &   
                     \gloss{‘asking’}  &  \\

                     \vernacular{a[kulíkháanga]
                    khú/tsa}  &   
                     \gloss{‘naming’}  &  \\

                     \vernacular{a[lakhúulaanga]
                    khú/tsa}  &   
                     \gloss{‘releasing’}  &  \\

                     \vernacular{a[seébúlaanga]
                    khú/tsa}  &   
                     \gloss{‘saying bye’}  &  \\

                     \vernacular{
                    a[kalúshítsaanga] khú/tsa}  &   
                     \gloss{‘returning’}  &  \\

                     \vernacular{
                    a[hoómbélitsaanga] khú/tsa}  &   
                     \gloss{‘comforting’}  &  \\
\end{tabular}
%\caption{\nocaption}
     
\begin{tabular}{lll}  
  \multicolumn{2}{l}{
                   \vernacular{(769/773) /H/
                  C-Initial} \gloss{‘s/he is...him/her
                  a bit \ob khú\cb ’}/} &  \\
\multicolumn{2}{l}{
                     \gloss{‘s/he is just
                    \ob tsa\cb ...him/her’} } &  \\

                     \vernacular{
                    amu[ré{\downstep}étsáángá] khú/tsa}  &   
                     \gloss{‘burying’}  &  \\

                     \vernacular{amu[bé{\downstep}káángá]
                    khú/tsa}  &   
                     \gloss{‘shaving’}  &  \\

                     \vernacular{
                    amu[lé{\downstep}éráángá] khú/tsa}  &   
                     \gloss{‘bringing’}  &  \\

                     \vernacular{
                    amu[khá{\downstep}lákáánga] khú/tsa}  &   
                     \gloss{‘cutting’}  &  \\

                     \vernacular{
                    amu[sí{\downstep}táákáánga] khú/tsa}  &   
                     \gloss{‘accusing’}  &  \\

                     \vernacular{
                    amu[bó{\downstep}ólítsáánga] khú/tsa}  &   
                     \gloss{‘seducing’}  &  \\

                     \vernacular{
                    amu[tsú{\downstep}únzúúnáánga] khú/tsa}  &   
                     \gloss{‘sucking’}  &  \\

                     \vernacular{
                    amu[bó{\downstep}yóng’ánaanga] khú/tsa}  &   
                     \gloss{‘going
                    around’}  &  \\
\end{tabular}
%\caption{\nocaption}
     
\begin{tabular}{lll}  
  \multicolumn{2}{l}{
                   \vernacular{(770/774) /Ø/
                  C-Initial} \gloss{‘s/he is...him/her
                  a bit \ob khú\cb ’}/} &  \\
\multicolumn{2}{l}{
                     \gloss{‘s/he is just
                    \ob tsa\cb ...him/her’} } &  \\

                     \vernacular{amu[tsiítsáanga]
                    khú/tsa}  &   
                     \gloss{‘going for’}  &  \\

                     \vernacular{amu[lekháanga]
                    khú/tsa}  &   
                     \gloss{‘leaving’}  &  \\

                     \vernacular{amu[loóndáanga]
                    khú/tsa}  &   
                     \gloss{‘asking’}  &  \\

                     \vernacular{amu[kulíkháanga]
                    khú/tsa}  &   
                     \gloss{‘naming’}  &  \\

                     \vernacular{amu[lakhúulaanga]
                    khú/tsa}  &   
                     \gloss{‘releasing’}  &  \\

                     \vernacular{amu[seébúlaanga]
                    khú/tsa}  &   
                     \gloss{‘saying bye’}  &  \\

                     \vernacular{
                    amu[kalúshítsaanga] khú/tsa}  &   
                     \gloss{‘returning’}  &  \\

                     \vernacular{
                    amu[hoómbélitsaanga] khú/tsa}  &   
                     \gloss{‘comforting’}  &  \\
\end{tabular}
%\caption{\nocaption}
    

\subsection{Indefinite Future}\label{sec:sEncIndefFut}


\begin{tabular}{lll}  
  \multicolumn{2}{l}{
                   \vernacular{(775/779) /H/
                  C-Initial} \gloss{‘s/he will...a bit
                  \ob khú\cb ’}/} &  \\
\multicolumn{2}{l}{
                     \gloss{‘s/he will just
                    \ob tsa\cb ...’
                    } } &  \\

                     \vernacular{ali[khwa]
                    khú/tsa}  &   
                     \gloss{‘pay dowry’}  &  \\

                     \vernacular{ali[luma]
                    khú/tsa}  &   
                     \gloss{‘bite’}  &  \\

                     \vernacular{ali[teekhá]
                    khú/tsa}  &   
                     \gloss{‘cook’}  &  \\

                     \vernacular{ali[khalaká]
                    khú/tsa}  &   
                     \gloss{‘cut’}  &  \\

                     \vernacular{ali[kalaangá]
                    khú/tsa}  &   
                     \gloss{‘fry’}  &  \\

                     \vernacular{ali[boolitsá]
                    khú/tsa}  &   
                     \gloss{‘seduce’}  &  \\

                     \vernacular{ali[tsuunzuuná]
                    khú/tsa}  &   
                     \gloss{‘suck’}  &  \\

                     \vernacular{ali[boyong’aná]
                    khú/tsa}  &   
                     \gloss{‘go around’}  &  \\
\end{tabular}
%\caption{\nocaption}
     
\begin{tabular}{lll}  
  \multicolumn{2}{l}{
                   \vernacular{(776/780) /Ø/
                  C-Initial} \gloss{‘s/he will...a bit
                  \ob khú\cb ’}/} &  \\
\multicolumn{2}{l}{
                     \gloss{‘s/he will just
                    \ob tsa\cb ...’} } &  \\

                     \vernacular{ali[kwá]
                    khú/tsa}  &   
                     \gloss{‘fall’}  &  \\

                     \vernacular{ali[lekhá]
                    khú/tsa}  &   
                     \gloss{‘leave’}  &  \\

                     \vernacular{ali[reéba]
                    khú/tsa}  &   
                     \gloss{‘ask’}  &  \\

                     \vernacular{ali[kulíkha]
                    khú/tsa}  &   
                     \gloss{‘name’}  &  \\

                     \vernacular{ali[lakhúula]
                    khú/tsa}  &   
                     \gloss{‘release’}  &  \\

                     \vernacular{ali[seébúla]
                    khú/tsa}  &   
                     \gloss{‘say bye’}  &  \\

                     \vernacular{ali[kalúshítsa]
                    khú/tsa}  &   
                     \gloss{‘return’}  &  \\

                     \vernacular{ali[hoómbélitsa]
                    khú/tsa}  &   
                     \gloss{‘comfort’}  &  \\
\end{tabular}
%\caption{\nocaption}
     
\begin{tabular}{lll}  
  \multicolumn{2}{l}{
                   \vernacular{(777/781) /H/
                  C-Initial} \gloss{‘s/he
                  will...him/her a bit \ob khú\cb ’}/} &  \\
\multicolumn{2}{l}{
                     \gloss{‘s/he will just
                    \ob tsa\cb ...him/her’} } &  \\

                     \vernacular{alimu[rá]
                    khú/tsa}  &   
                     \gloss{‘bury’}  &  \\

                     \vernacular{alimu[béka]
                    khú/tsa}  &   
                     \gloss{‘shave’}  &  \\

                     \vernacular{alimu[léera]
                    khú/tsa}  &   
                     \gloss{‘bring’}  &  \\

                     \vernacular{alimu[khálaka]
                    khú/tsa}  &   
                     \gloss{‘cut’}  &  \\

                     \vernacular{alimu[sítaaka]
                    khú/tsa}  &   
                     \gloss{‘accuse’}  &  \\

                     \vernacular{alimu[bóolitsa]
                    khú/tsa}  &   
                     \gloss{‘seduce’}  &  \\

                     \vernacular{alimu[tsúunzuuna]
                    khú/tsa}  &   
                     \gloss{‘suck’}  &  \\

                     \vernacular{alimu[bóyong’ana]
                    khú/tsa}  &   
                     \gloss{‘go around’}  &  \\
\end{tabular}
%\caption{\nocaption}
     
\begin{tabular}{lll}  
  \multicolumn{2}{l}{
                   \vernacular{(778/782) /Ø/
                  C-Initial} \gloss{‘s/he
                  will...him/her a bit \ob khú\cb ’}/} &  \\
\multicolumn{2}{l}{
                     \gloss{‘s/he will just
                    \ob tsa\cb ...him/her’
                    } } &  \\

                     \vernacular{alimu[tsíá]
                    khú/tsa}  &   
                     \gloss{‘go for’}  &  \\

                     \vernacular{alimu[lekhá]
                    khú/tsa}  &   
                     \gloss{‘leave’}  &  \\

                     \vernacular{alimu[loónda]
                    khú/tsa}  &   
                     \gloss{‘ask’}  &  \\

                     \vernacular{alimu[kulíkha]
                    khú/tsa}  &   
                     \gloss{‘name’}  &  \\

                     \vernacular{alimu[lakhúula]
                    khú/tsa}  &   
                     \gloss{‘release’}  &  \\

                     \vernacular{alimu[seébúla]
                    khú/tsa}  &   
                     \gloss{‘say bye’}  &  \\

                     \vernacular{
                    alimu[kalúshítsa] khú/tsa}  &   
                     \gloss{‘return’}  &  \\

                     \vernacular{
                    alimu[hoómbélitsa] khú/tsa}  &   
                     \gloss{‘comfort’}  &  \\
\end{tabular}
%\caption{\nocaption}
    

\subsection{Imperative
              }\label{sec:sEncImpSg}


\begin{tabular}{lll}  
  \multicolumn{2}{l}{
                   \vernacular{(783/787) /H/
                  C-Initial} \gloss{‘...a bit
                  \ob khú\cb !’}/} &  \\
\multicolumn{2}{l}{
                     \gloss{‘ just
                    \ob tsa\cb ...!’} } &  \\

                     \vernacular{[khwa]
                    khú/tsa}  &   
                     \gloss{‘pay dowry’}  &  \\

                     \vernacular{[luma]
                    khú/tsa}  &   
                     \gloss{‘bite’}  &  \\

                     \vernacular{[teekhá] khú/tsa
                    }  &   
                     \gloss{‘cook’}  &  \\

                     \vernacular{[khalaká]
                    khú/tsa}  &   
                     \gloss{‘cut’}  &  \\

                     \vernacular{[kalaangá]
                    khú/tsa}  &   
                     \gloss{‘fry’}  &  \\

                     \vernacular{[boolitsá]
                    khú/tsa}  &   
                     \gloss{‘seduce’}  &  \\

                     \vernacular{[tsuunzuuná]
                    khú/tsa}  &   
                     \gloss{‘suck’}  &  \\

                     \vernacular{[boyong’aná]
                    khú/tsa}  &   
                     \gloss{‘go around’}  &  \\
\end{tabular}
%\caption{\nocaption}
     
\begin{tabular}{lll}  
  \multicolumn{2}{l}{
                   \vernacular{(784/788) /Ø/
                  C-Initial} \gloss{‘...a bit
                  \ob khú\cb !’}/} &  \\
\multicolumn{2}{l}{
                     \gloss{‘ just
                    \ob tsa\cb ...!’} } &  \\

                     \vernacular{[kwá]
                    khú/tsa}  &   
                     \gloss{‘fall’}  &  \\

                     \vernacular{[lé{\downstep}khá]
                    khú/tsa}  &   
                     \gloss{‘leave’}  &  \\

                     \vernacular{[réé{\downstep}bá]
                    khú/tsa}  &   
                     \gloss{‘ask’}  &  \\

                     \vernacular{[kúlí{\downstep}khá]
                    khú/tsa}  &   
                     \gloss{‘name’}  &  \\

                     \vernacular{[lákhúú{\downstep}lá]
                    khú/tsa}  &   
                     \gloss{‘release’}  &  \\

                     \vernacular{[séé{\downstep}búlá]
                    khú/tsa}  &   
                     \gloss{‘say bye’}  &  \\

                     \vernacular{[kálú{\downstep}shítsá]
                    khú/tsa}  &   
                     \gloss{‘return’}  &  \\

                     \vernacular{
                    [hóómbé{\downstep}lítsá] khú/tsa}  &   
                     \gloss{‘comfort’}  &  \\
\end{tabular}
%\caption{\nocaption}
     
\begin{tabular}{lll}  
  \multicolumn{2}{l}{
                   \vernacular{(785/789) /H/
                  C-Initial} \gloss{‘...him/her a bit
                  \ob khú\cb !’}/} &  \\
\multicolumn{2}{l}{
                     \gloss{‘ just
                    \ob tsa\cb ...him/her!’} } &  \\

                     \vernacular{mu[rɛ́]
                    khú/tsa}  &   
                     \gloss{‘pay (her)
                    dowry’}  &  \\

                     \vernacular{mu[bé{\downstep}chɛ́]
                    khú/tsa}  &   
                     \gloss{‘shave’}  &  \\

                     \vernacular{mu[léé{\downstep}rɛ́]
                    khú/tsa}  &   
                     \gloss{‘bring’}  &  \\

                     \vernacular{mu[khá{\downstep}láchɛ́]
                    khú/tsa}  &   
                     \gloss{‘cut’}  &  \\

                     \vernacular{mu[sí{\downstep}tááchɛ́]
                    khú/tsa}  &   
                     \gloss{‘accuse’}  &  \\

                     \vernacular{mu[bóó{\downstep}lítsɪ́]
                    khú/tsa}  &   
                     \gloss{‘seduce’}  &  \\

                     \vernacular{
                    mu[tsúú{\downstep}nzúúnɪ́] khú/tsa}  &   
                     \gloss{‘suck’}  &  \\

                     \vernacular{
                    mu[bó{\downstep}yóng’ánɛ́] khú/tsa}  &   
                     \gloss{‘go around’}  &  \\
\end{tabular}
%\caption{\nocaption}
     
\begin{tabular}{lll}  
  \multicolumn{2}{l}{
                   \vernacular{(786/790) /Ø/
                  C-Initial} \gloss{‘...him/her a bit
                  \ob khú\cb !’}/} &  \\
\multicolumn{2}{l}{
                     \gloss{‘ just
                    \ob tsa\cb ...him/her!’} } &  \\

                     \vernacular{mu[tsí]
                    khú/tsa}  &   
                     \gloss{‘go for’}  &  \\

                     \vernacular{mu[léshɛ́]
                    khú/tsa}  &   
                     \gloss{‘leave’}  &  \\

                     \vernacular{mu[lóó{\downstep}ndɛ́]
                    khú/tsa}  &   
                     \gloss{‘ask’}  &  \\

                     \vernacular{mu[kúlí{\downstep}shɪ́]
                    khú/tsa}  &   
                     \gloss{‘name’}  &  \\

                     \vernacular{mu[lákhú{\downstep}úlɪ́]
                    khú/tsa}  &   
                     \gloss{‘release’}  &  \\

                     \vernacular{mu[séébú{\downstep}lɪ́]
                    khú/tsa}  &   
                     \gloss{‘say bye’}  &  \\

                     \vernacular{
                    mu[kálú{\downstep}shítsɪ́] khú/tsa}  &   
                     \gloss{‘return’}  &  \\

                     \vernacular{
                    mu[hóómbé{\downstep}lítsɪ́] khú/tsa}  &   
                     \gloss{‘comfort’}  &  \\
\end{tabular}
%\caption{\nocaption}
    

\subsection{Imperative
              }\label{sec:sEncImpSgNeg}


\begin{tabular}{lll}  
  \multicolumn{2}{l}{
                   \vernacular{(791/795) /H/
                  C-Initial} \gloss{‘do not...a bit
                  \ob khú\cb !’}/} &  \\
\multicolumn{2}{l}{
                     \gloss{‘do not just
                    \ob tsa\cb ...!’} } &  \\

                     \vernacular{ukha[khwa]
                    khú/tsa tá}  &   
                     \gloss{‘pay dowry’}  &  \\

                     \vernacular{ukha[luma]
                    khú/tsa tá}  &   
                     \gloss{‘bite’}  &  \\

                     \vernacular{ukha[teekha]
                    khú/tsa tá}  &   
                     \gloss{‘cook’}  &  \\

                     \vernacular{ukha[khalaka]
                    khú/tsa tá}  &   
                     \gloss{‘cut’}  &  \\

                     \vernacular{ukha[kalaanga]
                    khú/tsa tá}  &   
                     \gloss{‘fry’}  &  \\

                     \vernacular{ukha[boolitsa]
                    khú/tsa tá}  &   
                     \gloss{‘seduce’}  &  \\

                     \vernacular{ukha[tsuunzuuna]
                    khú/tsa tá}  &   
                     \gloss{‘suck’}  &  \\

                     \vernacular{ukha[boyong’ana]
                    khú/tsa tá}  &   
                     \gloss{‘go around’}  &  \\
\end{tabular}
%\caption{\nocaption}
     
\begin{tabular}{lll}  
  \multicolumn{2}{l}{
                   \vernacular{(792/796) /Ø/
                  C-Initial} \gloss{‘do not...a bit
                  \ob khú\cb !’}/} &  \\
\multicolumn{2}{l}{
                     \gloss{‘do not just
                    \ob tsa\cb ...!’} } &  \\

                     \vernacular{ukha[kwá]
                    khú/tsa tá}  &   
                     \gloss{‘fall’}  &  \\

                     \vernacular{ukha[lekhá]
                    khú/tsa tá}  &   
                     \gloss{‘leave’}  &  \\

                     \vernacular{ukha[reéba]
                    khú/tsa tá}  &   
                     \gloss{‘ask’}  &  \\

                     \vernacular{ukha[kulíkha]
                    khú/tsa tá}  &   
                     \gloss{‘name’}  &  \\

                     \vernacular{ukha[lakhúula]
                    khú/tsa tá}  &   
                     \gloss{‘release’}  &  \\

                     \vernacular{ukha[seébula]
                    khú/tsa tá}  &   
                     \gloss{‘say bye’}  &  \\

                     \vernacular{ukha[kalúshitsa]
                    khú/tsa tá}  &   
                     \gloss{‘return’}  &  \\

                     \vernacular{
                    ukha[hoómbélitsa] khú/tsa tá}  &   
                     \gloss{‘comfort’}  &  \\
\end{tabular}
%\caption{\nocaption}
     
\begin{tabular}{lll}  
  \multicolumn{2}{l}{
                   \vernacular{(793/797) /H/
                  C-Initial} \gloss{‘do not...him/her a
                  bit \ob khú\cb !’}/} &  \\
\multicolumn{2}{l}{
                     \gloss{‘do not just
                    \ob tsa\cb ...him/her!’} } &  \\

                     \vernacular{ukhamu[rá]
                    khú/tsa tá}  &   
                     \gloss{‘bury’}  &  \\

                     \vernacular{ukhamu[béka]
                    khú/tsa tá}  &   
                     \gloss{‘shave’}  &  \\

                     \vernacular{ukhamu[léera]
                    khú/tsa tá}  &   
                     \gloss{‘bring’}  &  \\

                     \vernacular{ukhamu[khálaka]
                    khú/tsa tá}  &   
                     \gloss{‘cut’}  &  \\

                     \vernacular{ukhamu[sítaaka]
                    khú/tsa tá}  &   
                     \gloss{‘accuse’}  &  \\

                     \vernacular{ukhamu[bóolitsa]
                    khú/tsa tá}  &   
                     \gloss{‘seduce’}  &  \\

                     \vernacular{
                    ukhamu[tsúunzuuna] khú/tsa tá}  &   
                     \gloss{‘suck’}  &  \\

                     \vernacular{
                    ukhamu[bóyong’ana] khú/tsa tá}  &   
                     \gloss{‘go around’}  &  \\
\end{tabular}
%\caption{\nocaption}
     
\begin{tabular}{lll}  
  \multicolumn{2}{l}{
                   \vernacular{(794/798) /Ø/
                  C-Initial} \gloss{‘do not...him/her a
                  bit \ob khú\cb !’}/} &  \\
\multicolumn{2}{l}{
                     \gloss{‘do not just
                    \ob tsa\cb ...him/her!’} } &  \\

                     \vernacular{ukhamu[tsíá]
                    khú/tsa tá}  &   
                     \gloss{‘go for’}  &  \\

                     \vernacular{ukhamu[lekhá]
                    khú/tsa tá}  &   
                     \gloss{‘leave’}  &  \\

                     \vernacular{ukhamu[loónda]
                    khú/tsa tá}  &   
                     \gloss{‘ask’}  &  \\

                     \vernacular{ukhamu[kulíkha]
                    khú/tsa tá}  &   
                     \gloss{‘name’}  &  \\

                     \vernacular{ukhamu[lakhúula]
                    khú/tsa tá}  &   
                     \gloss{‘release’}  &  \\

                     \vernacular{ukhamu[seébula]
                    khú/tsa tá}  &   
                     \gloss{‘say bye’}  &  \\

                     \vernacular{
                    ukhamu[kalúshitsa] khú/tsa tá}  &   
                     \gloss{‘return’}  &  \\

                     \vernacular{
                    ukhamu[hoómbélitsa] khú/tsa tá}  &   
                     \gloss{‘comfort’}  &  \\
\end{tabular}
%\caption{\nocaption}
    

\subsection{Subjunctive}\label{sec:sEncSubj}


\begin{tabular}{lll}  
  \multicolumn{2}{l}{
                   \vernacular{(799/803) /H/
                  C-Initial} \gloss{‘let him/her...a
                  bit \ob khú\cb !’}/} &  \\
\multicolumn{2}{l}{
                     \gloss{‘let him/her just
                    \ob tsa\cb ...!’} } &  \\

                     \vernacular{a[khwí]
                    khú/tsa}  &   
                     \gloss{‘pay dowry’}  &  \\

                     \vernacular{a[lumɪ́]
                    khú/tsa}  &   
                     \gloss{‘bite’}  &  \\

                     \vernacular{a[teeshɛ́]
                    khú/tsa}  &   
                     \gloss{‘cook’}  &  \\

                     \vernacular{a[khalachɛ́]
                    khú/tsa}  &   
                     \gloss{‘cut’}  &  \\

                     \vernacular{a[kalaanjɛ́]
                    khú/tsa}  &   
                     \gloss{‘fry’}  &  \\

                     \vernacular{a[boolitsɪ́]
                    khú/tsa}  &   
                     \gloss{‘seduce’}  &  \\

                     \vernacular{a[tsuunzuúnɛ]
                    khú/tsa}  &   
                     \gloss{‘suck’}  &  \\

                     \vernacular{a[boyong’ánɛ]
                    khú/tsa}  &   
                     \gloss{‘go around’}  &  \\
\end{tabular}
%\caption{\nocaption}
     
\begin{tabular}{lll}  
  \multicolumn{2}{l}{
                   \vernacular{(800/804) /Ø/
                  C-Initial} \gloss{‘let him/her...a
                  bit \ob khú\cb !’}/} &  \\
\multicolumn{2}{l}{
                     \gloss{‘let him/her just
                    \ob tsa\cb ...!’} } &  \\

                     \vernacular{a[kwí]
                    khú/tsa}  &   
                     \gloss{‘fall’}  &  \\

                     \vernacular{a[leshɛ́]
                    khú/tsa}  &   
                     \gloss{‘leave’}  &  \\

                     \vernacular{a[reebɛ́]
                    khú/tsa}  &   
                     \gloss{‘ask’}  &  \\

                     \vernacular{a[kulishɪ́]
                    khú/tsa}  &   
                     \gloss{‘name’}  &  \\

                     \vernacular{a[lakhuúlɪ]
                    khú/tsa}  &   
                     \gloss{‘release’}  &  \\

                     \vernacular{a[seebulɪ́]
                    khú/tsa}  &   
                     \gloss{‘say bye’}  &  \\

                     \vernacular{a[kalushítsɪ]
                    khú/tsa}  &   
                     \gloss{‘return’}  &  \\

                     \vernacular{a[hoombelítsɪ]
                    khú/tsa}  &   
                     \gloss{‘comfort’}  &  \\
\end{tabular}
%\caption{\nocaption}
     
\begin{tabular}{lll}  
  \multicolumn{2}{l}{
                   \vernacular{(801/805) /H/
                  C-Initial} \gloss{‘let
                  him/her...him/her a bit \ob khú\cb !’}/} &  \\
\multicolumn{2}{l}{
                     \gloss{‘let him/her just
                    \ob tsa\cb ...him/her!’} } &  \\

                     \vernacular{amu[rɛ́]
                    khú/tsa}  &   
                     \gloss{‘pay (her)
                    dowry’}  &  \\

                     \vernacular{amu[béchɛ]
                    khú/tsa}  &   
                     \gloss{‘shave’}  &  \\

                     \vernacular{amu[léerɛ]
                    khú/tsa}  &   
                     \gloss{‘bring’}  &  \\

                     \vernacular{amu[khálachɛ]
                    khú/tsa}  &   
                     \gloss{‘cut’}  &  \\

                     \vernacular{amu[sítaachɛ]
                    khú/tsa}  &   
                     \gloss{‘accuse’}  &  \\

                     \vernacular{amu[bóolitsɪ]
                    khú/tsa}  &   
                     \gloss{‘seduce’}  &  \\

                     \vernacular{amu[tsúunzuunɪ]
                    khú/tsa}  &   
                     \gloss{‘suck’}  &  \\

                     \vernacular{amu[bóyong’anɛ]
                    khú/tsa}  &   
                     \gloss{‘go around’}  &  \\
\end{tabular}
%\caption{\nocaption}
     
\begin{tabular}{lll}  
  \multicolumn{2}{l}{
                   \vernacular{(802/806) /Ø/
                  C-Initial} \gloss{‘let
                  him/her...him/her a bit \ob khú\cb !’}/} &  \\
\multicolumn{2}{l}{
                     \gloss{‘let him/her just
                    \ob tsa\cb ...him/her!’} } &  \\

                     \vernacular{amu[tsí]
                    khú/tsa}  &   
                     \gloss{‘go for’}  &  \\

                     \vernacular{amu[leshɛ́]
                    khú/tsa}  &   
                     \gloss{‘leave’}  &  \\

                     \vernacular{amu[loóndɛ]
                    khú/tsa}  &   
                     \gloss{‘ask’}  &  \\

                     \vernacular{amu[kulíshɪ]
                    khú/tsa}  &   
                     \gloss{‘name’}  &  \\

                     \vernacular{amu[lakhúulɪ]
                    khú/tsa}  &   
                     \gloss{‘release’}  &  \\

                     \vernacular{amu[seébulɪ]
                    khú/tsa}  &   
                     \gloss{‘say bye’}  &  \\

                     \vernacular{amu[kalúshitsɪ]
                    khú/tsa}  &   
                     \gloss{‘return’}  &  \\

                     \vernacular{amu[hoómbélitsɪ]
                    khú/tsa}  &   
                     \gloss{‘comfort’}  &  \\
\end{tabular}
%\caption{\nocaption}
    

\subsection{Hesternal Perfective}\label{sec:sEncHestPerf}


\begin{tabular}{lll}  
  \multicolumn{2}{l}{
                   \vernacular{(807/811) /H/
                  C-Initial} \gloss{‘s/he...a bit
                  \ob khú\cb !’}/} &  \\
\multicolumn{2}{l}{
                     \gloss{‘s/he just
                    \ob tsa\cb ...!’} } &  \\

                     \vernacular{ya[khwéélé]
                    khú/tsa}  &   
                     \gloss{‘paid dowry’}  &  \\

                     \vernacular{ya[lúmí]
                    khú/tsa}  &   
                     \gloss{‘bit’}  &  \\

                     \vernacular{ya[tééshí]
                    khú/tsa}  &   
                     \gloss{‘cooked’}  &  \\

                     \vernacular{ya[khálááchɛ́]
                    khú/tsa}  &   
                     \gloss{‘cut’}  &  \\

                     \vernacular{ya[káláánjí]
                    khú/tsa}  &   
                     \gloss{‘fried’}  &  \\

                     \vernacular{ya[bóólíítsɪ́]
                    khú/tsa}  &   
                     \gloss{‘seduced’}  &  \\

                     \vernacular{
                    ya[tsúúnzúúní] khú/tsa}  &   
                     \gloss{‘sucked’}  &  \\

                     \vernacular{
                    ya[bóyóng’áánɛ́] khú/tsa}  &   
                     \gloss{‘went
                    around’}  &  \\
\end{tabular}
%\caption{\nocaption}
     
\begin{tabular}{lll}  
  \multicolumn{2}{l}{
                   \vernacular{(808/812) /Ø/
                  C-Initial} \gloss{‘s/he...a bit
                  \ob khú\cb !’}/} &  \\
\multicolumn{2}{l}{
                     \gloss{‘s/he just
                    \ob tsa\cb ...!’} } &  \\

                     \vernacular{ya[kwíí{\downstep}lí]
                    khú/tsa}  &   
                     \gloss{‘fell’}  &  \\

                     \vernacular{ya[léshí]
                    khú/tsa}  &   
                     \gloss{‘left’}  &  \\

                     \vernacular{ya[réé{\downstep}bí]
                    khú/tsa}  &   
                     \gloss{‘asked’}  &  \\

                     \vernacular{ya[kúlíí{\downstep}shɪ́]
                    khú/tsa}  &   
                     \gloss{‘named’}  &  \\

                     \vernacular{ya[lákhú{\downstep}úlí]
                    khú/tsa}  &   
                     \gloss{‘released’}  &  \\

                     \vernacular{ya[séé{\downstep}búúlɪ́]
                    khú/tsa}  &   
                     \gloss{‘said bye’}  &  \\

                     \vernacular{
                    ya[kálú{\downstep}shíítsɪ́] khú/tsa}  &   
                     \gloss{‘returned’}  &  \\

                     \vernacular{
                    ya[hóómbé{\downstep}líítsɪ́] khú/tsa}  &   
                     \gloss{‘comforted’}  &  \\
\end{tabular}
%\caption{\nocaption}
     
\begin{tabular}{lll}  
  \multicolumn{2}{l}{
                   \vernacular{(809/813) /H/
                  C-Initial} \gloss{‘s/he...him/her a
                  bit \ob khú\cb !’}/} &  \\
\multicolumn{2}{l}{
                     \gloss{‘s/he just
                    \ob tsa\cb ...him/her!’} } &  \\

                     \vernacular{yamu[ré{\downstep}élé]
                    khú/tsa}  &   
                     \gloss{‘paid (her)
                    dowry’}  &  \\

                     \vernacular{yamu[bé{\downstep}chí]
                    khú/tsa}  &   
                     \gloss{‘shaved’}  &  \\

                     \vernacular{yamu[lé{\downstep}érí]
                    khú/tsa}  &   
                     \gloss{‘brought’}  &  \\

                     \vernacular{
                    yamu[khá{\downstep}lááchɛ́] khú/tsa}  &   
                     \gloss{‘cut’}  &  \\

                     \vernacular{
                    yamu[sí{\downstep}tááchí] khú/tsa}  &   
                     \gloss{‘accused’}  &  \\

                     \vernacular{
                    yamu[bó{\downstep}ólíítsɪ́] khú/tsa}  &   
                     \gloss{‘seduced’}  &  \\

                     \vernacular{
                    yamu[tsú{\downstep}únzúúní] khú/tsa}  &   
                     \gloss{‘sucked’}  &  \\

                     \vernacular{
                    yamu[bó{\downstep}yóng’áánɛ́] khú/tsa}  &   
                     \gloss{‘went
                    around’}  &  \\
\end{tabular}
%\caption{\nocaption}
     
\begin{tabular}{lll}  
  \multicolumn{2}{l}{
                   \vernacular{(810/814) /Ø/
                  C-Initial} \gloss{‘s/he...him/her a
                  bit \ob khú\cb !’}/} &  \\
\multicolumn{2}{l}{
                     \gloss{‘s/he just
                    \ob tsa\cb ...him/her!’} } &  \\

                     \vernacular{yamu[tsíí{\downstep}lí]
                    khú/tsa}  &   
                     \gloss{‘went for’}  &  \\

                     \vernacular{yamu[léshí]
                    khú/tsa}  &   
                     \gloss{‘left’}  &  \\

                     \vernacular{yamu[lóó{\downstep}ndí]
                    khú/tsa}  &   
                     \gloss{‘asked’}  &  \\

                     \vernacular{
                    yamu[kúlíí{\downstep}shɪ́] khú/tsa}  &   
                     \gloss{‘named’}  &  \\

                     \vernacular{
                    yamu[lákhú{\downstep}úlí] khú/tsa}  &   
                     \gloss{‘released’}  &  \\

                     \vernacular{
                    yamu[séé{\downstep}búúlɪ́] khú/tsa}  &   
                     \gloss{‘said bye’}  &  \\

                     \vernacular{
                    yamu[kálú{\downstep}shíítsɪ́] khú/tsa}  &   
                     \gloss{‘returned’}  &  \\

                     \vernacular{
                    yamu[hóómbé{\downstep}líítsɪ́] khú/tsa}  &   
                     \gloss{‘comforted’}  &  \\
\end{tabular}
%\caption{\nocaption}
    

\subsection{Perfect (2
              }\label{sec:sEncPerf2ndSg}


\begin{tabular}{lll}  
  \multicolumn{2}{l}{
                   \vernacular{(815/819) /H/
                  C-Initial} \gloss{‘you have...a bit
                  \ob khú\cb !’}/} &  \\
\multicolumn{2}{l}{
                     \gloss{‘you have just
                    \ob tsa\cb ...!’} } &  \\

                     \vernacular{u[khweele]
                    khú/tsa}  &   
                     \gloss{‘paid dowry’}  &  \\

                     \vernacular{u[lumi]
                    khú/tsa}  &   
                     \gloss{‘bitten’}  &  \\

                     \vernacular{u[teeshi]
                    khú/tsa}  &   
                     \gloss{‘cooked’}  &  \\

                     \vernacular{u[khalaachɛ]
                    khú/tsa}  &   
                     \gloss{‘cut’}  &  \\

                     \vernacular{u[kalaanji]
                    khú/tsa}  &   
                     \gloss{‘fried’}  &  \\

                     \vernacular{u[booliitsɪ]
                    khú/tsa}  &   
                     \gloss{‘seduced’}  &  \\

                     \vernacular{u[tsuunzuuni]
                    khú/tsa}  &   
                     \gloss{‘sucked’}  &  \\

                     \vernacular{u[boyong’aanɛ]
                    khú/tsa}  &   
                     \gloss{‘gone
                    around’}  &  \\
\end{tabular}
%\caption{\nocaption}
     
\begin{tabular}{lll}  
  \multicolumn{2}{l}{
                   \vernacular{(816/820) /Ø/
                  C-Initial} \gloss{‘you have...a bit
                  \ob khú\cb !’}/} &  \\
\multicolumn{2}{l}{
                     \gloss{‘you have just
                    \ob tsa\cb ...!’} } &  \\

                     \vernacular{u[kwiili]
                    khú/tsa}  &   
                     \gloss{‘fallen’}  &  \\

                     \vernacular{u[leshi]
                    khú/tsa}  &   
                     \gloss{‘left’}  &  \\

                     \vernacular{u[reebi]
                    khú/tsa}  &   
                     \gloss{‘asked’}  &  \\

                     \vernacular{u[kuliishɪ]
                    khú/tsa}  &   
                     \gloss{‘named’}  &  \\

                     \vernacular{u[lakhuuli]
                    khú/tsa}  &   
                     \gloss{‘released’}  &  \\

                     \vernacular{u[seebuulɪ]
                    khú/tsa}  &   
                     \gloss{‘said bye’}  &  \\

                     \vernacular{u[kalushiitsɪ]
                    khú/tsa}  &   
                     \gloss{‘returned’}  &  \\

                     \vernacular{u[hoombeliitsɪ]
                    khú/tsa}  &   
                     \gloss{‘comforted’}  &  \\
\end{tabular}
%\caption{\nocaption}
     
\begin{tabular}{lll}  
  \multicolumn{2}{l}{
                   \vernacular{(817/821) /H/
                  C-Initial} \gloss{‘you have...him/her
                  a bit \ob khú\cb !’}/} &  \\
\multicolumn{2}{l}{
                     \gloss{‘you have just
                    \ob tsa\cb ...him/her!’} } &  \\

                     \vernacular{umu[reele]
                    khú/tsa}  &   
                     \gloss{‘paid (her)
                    dowry’}  &  \\

                     \vernacular{umu[bechi]
                    khú/tsa}  &   
                     \gloss{‘shaved’}  &  \\

                     \vernacular{umu[leeri]
                    khú/tsa}  &   
                     \gloss{‘brought’}  &  \\

                     \vernacular{umu[khalaachɛ]
                    khú/tsa}  &   
                     \gloss{‘cut’}  &  \\

                     \vernacular{umu[sitaachi]
                    khú/tsa}  &   
                     \gloss{‘accused’}  &  \\

                     \vernacular{umu[booliitsɪ]
                    khú/tsa}  &   
                     \gloss{‘seduced’}  &  \\

                     \vernacular{umu[tsuunzuuni]
                    khú/tsa}  &   
                     \gloss{‘sucked’}  &  \\

                     \vernacular{umu[boyong’aanɛ]
                    khú/tsa}  &   
                     \gloss{‘gone
                    around’}  &  \\
\end{tabular}
%\caption{\nocaption}
     
\begin{tabular}{lll}  
  \multicolumn{2}{l}{
                   \vernacular{(818/822) /Ø/
                  C-Initial} \gloss{‘you have...him/her
                  a bit \ob khú\cb !’}/} &  \\
\multicolumn{2}{l}{
                     \gloss{‘you have just
                    \ob tsa\cb ...him/her!’} } &  \\

                     \vernacular{umu[tsiili]
                    khú/tsa}  &   
                     \gloss{‘gone for’}  &  \\

                     \vernacular{umu[leshi]
                    khú/tsa}  &   
                     \gloss{‘left’}  &  \\

                     \vernacular{umu[loondi]
                    khú/tsa}  &   
                     \gloss{‘asked’}  &  \\

                     \vernacular{umu[kuliishɪ]
                    khú/tsa}  &   
                     \gloss{‘named’}  &  \\

                     \vernacular{umu[lakhuuli]
                    khú/tsa}  &   
                     \gloss{‘released’}  &  \\

                     \vernacular{umu[seebuulɪ]
                    khú/tsa}  &   
                     \gloss{‘said bye’}  &  \\

                     \vernacular{umu[kalushiitsɪ]
                    khú/tsa}  &   
                     \gloss{‘returned’}  &  \\

                     \vernacular{umu[hoombeliitsɪ]
                    khú/tsa}  &   
                     \gloss{‘comforted’}  &  \\
\end{tabular}
%\caption{\nocaption}
    

\subsection{Habitual}\label{sec:sEncHabit}


\begin{tabular}{lll}  
  \multicolumn{2}{l}{
                   \vernacular{(823/827) /H/
                  C-Initial} \gloss{‘s/he is
                  always/ever...a bit \ob khú\cb ’}/} &  \\
\multicolumn{2}{l}{
                     \gloss{‘s/he is
                    always/ever just \ob tsa\cb ...’} } &  \\

                     \vernacular{yaá[{\downstep}khwá]
                    khú/tsa}  &   
                     \gloss{‘paying
                    dowry’}  &  \\

                     \vernacular{yaá[{\downstep}lúmá]
                    khú/tsa}  &   
                     \gloss{‘biting’}  &  \\

                     \vernacular{yaá[{\downstep}téékhá]
                    khú/tsa}  &   
                     \gloss{‘cooking’}  &  \\

                     \vernacular{yaá[{\downstep}kháláká]
                    khú/tsa}  &   
                     \gloss{‘cutting’}  &  \\

                     \vernacular{
                    yaá[{\downstep}káláángá] khú/tsa}  &   
                     \gloss{‘frying’}  &  \\

                     \vernacular{
                    yaá[{\downstep}bóólítsá] khú/tsa}  &   
                     \gloss{‘seducing’}  &  \\

                     \vernacular{
                    yaá[{\downstep}tsúúnzúúná] khú/tsa}  &   
                     \gloss{‘sucking’}  &  \\

                     \vernacular{
                    yaá[{\downstep}bóyóng’áná] khú/tsa}  &   
                     \gloss{‘going
                    around’}  &  \\
\end{tabular}
%\caption{\nocaption}
     
\begin{tabular}{lll}  
  \multicolumn{2}{l}{
                   \vernacular{(824/828) /Ø/
                  C-Initial} \gloss{‘s/he is
                  always/ever...a bit \ob khú\cb ’}/} &  \\
\multicolumn{2}{l}{
                     \gloss{‘s/he is
                    always/ever just \ob tsa\cb ...’} } &  \\

                     \vernacular{yaá[{\downstep}kwá]
                    khú/tsa}  &   
                     \gloss{‘falling’}  &  \\

                     \vernacular{yaá[{\downstep}lékhá]
                    khú/tsa}  &   
                     \gloss{‘leaving’}  &  \\

                     \vernacular{yaá[{\downstep}réébá]
                    khú/tsa}  &   
                     \gloss{‘asking’}  &  \\

                     \vernacular{yaá[{\downstep}kúlíkhá]
                    khú/tsa}  &   
                     \gloss{‘naming’}  &  \\

                     \vernacular{
                    yaá[{\downstep}lákhúúlá] khú/tsa}  &   
                     \gloss{‘releasing’}  &  \\

                     \vernacular{yaá[{\downstep}séébúlá]
                    khú/tsa}  &   
                     \gloss{‘saying bye’}  &  \\

                     \vernacular{
                    yaá[{\downstep}kálúshítsá] khú/tsa}  &   
                     \gloss{‘returning’}  &  \\

                     \vernacular{
                    yaá[{\downstep}hóómbélítsá] khú/tsa}  &   
                     \gloss{‘comforting’}  &  \\
\end{tabular}
%\caption{\nocaption}
     
\begin{tabular}{lll}  
  \multicolumn{2}{l}{
                   \vernacular{(825/829) /H/
                  C-Initial} \gloss{‘s/he is
                  always/ever...him/her a bit
                  \ob khú\cb ’}/} &  \\
\multicolumn{2}{l}{
                     \gloss{‘s/he is
                    always/ever just \ob tsa\cb ...him/her’} } &  \\

                     \vernacular{yaá{\downstep}mú[khwá]
                    khú/tsa}  &   
                     \gloss{‘paying (her)
                    dowry’}  &  \\

                     \vernacular{yaá{\downstep}mú[bé{\downstep}ká]
                    khú/tsa}  &   
                     \gloss{‘shaving’}  &  \\

                     \vernacular{
                    yaá{\downstep}mú[lé{\downstep}érá] khú/tsa}  &   
                     \gloss{‘bringing’}  &  \\

                     \vernacular{
                    yaá{\downstep}mú[khá{\downstep}láká] khú/tsa}  &   
                     \gloss{‘cutting’}  &  \\

                     \vernacular{
                    yaá{\downstep}mú[sí{\downstep}tááká] khú/tsa}  &   
                     \gloss{‘accusing’}  &  \\

                     \vernacular{
                    yaá{\downstep}mú[bó{\downstep}ólítsá] khú/tsa}  &   
                     \gloss{‘seducing’}  &  \\

                     \vernacular{
                    yaá{\downstep}mú[tsú{\downstep}únzúúná] khú/tsa}  &   
                     \gloss{‘sucking’}  &  \\

                     \vernacular{
                    yaá{\downstep}mú[bó{\downstep}yóng’áná] khú/tsa}  &   
                     \gloss{‘going
                    around’}  &  \\
\end{tabular}
%\caption{\nocaption}
     
\begin{tabular}{lll}  
  \multicolumn{2}{l}{
                   \vernacular{(826/830) /Ø/
                  C-Initial} \gloss{‘s/he is
                  always/ever...him/her a bit
                  \ob khú\cb ’}/} &  \\
\multicolumn{2}{l}{
                     \gloss{‘s/he is
                    always/ever just \ob tsa\cb ...him/her’} } &  \\

                     \vernacular{yaá{\downstep}mú[tsíá]
                    khú/tsa}  &   
                     \gloss{‘going for’}  &  \\

                     \vernacular{yaá{\downstep}mú[lékhá]
                    khú/tsa}  &   
                     \gloss{‘leaving’}  &  \\

                     \vernacular{
                    yaá{\downstep}mú[lóóndá] khú/tsa}  &   
                     \gloss{‘asking’}  &  \\

                     \vernacular{
                    yaá{\downstep}mú[kúlíkhá] khú/tsa}  &   
                     \gloss{‘naming’}  &  \\

                     \vernacular{
                    yaá{\downstep}mú[lákhúúlá] khú/tsa}  &   
                     \gloss{‘releasing’}  &  \\

                     \vernacular{
                    yaá{\downstep}mú[séébúlá] khú/tsa}  &   
                     \gloss{‘saying bye’}  &  \\

                     \vernacular{
                    yaá{\downstep}mú[kálúshítsá] khú/tsa}  &   
                     \gloss{‘returning’}  &  \\

                     \vernacular{
                    yaá{\downstep}mú[hóómbélítsá] khú/tsa}  &   
                     \gloss{‘comforting’}  &  \\
\end{tabular}
%\caption{\nocaption}
    

\section{Passives, Questions, and Subject
            Relatives}\label{sec:sPassQsRel}

This portion of Appendix A constitutes the pilot
            questionnaire I prepared to survey the properties of
            the passive suffix, yes/no and WH questions, and
            subject relatives. For each construction surveyed, a
            small paradigm consisting of 4-6 verb forms was
            generated to probe each of these four topics. 

 These data were not discussed in the thesis. The
            transcriptions below are impressionistic and require
            further verification and analysis. 


\subsection{Near Future}\label{sec:sPQRNearFut}


\begin{tabular}{lll}  
  \multicolumn{2}{l}{
                     \vernacular{(831)
                    Passives} \gloss{‘s/he will
                    be...’} } &  \\
\multicolumn{2}{l}{ } &  \\

                     \vernacular{
                    ala[khálakwa]}  &   
                     \gloss{‘cut’}  &  \\

                     \vernacular{
                    ala[tsúunzuunwa]}  &   
                     \gloss{‘sucked’}  &  \\

                     \vernacular{
                    ala[lakhuulwa]}  &   
                     \gloss{‘released’}  &  \\

                     \vernacular{
                    ala[kalushilwa]}  &   
                     \gloss{‘defended’}  &  \\
\end{tabular}
%\caption{\nocaption}
     
\begin{tabular}{lll}  
  \multicolumn{2}{l}{
                     \vernacular{(832) Yes/No
                    Questions} \gloss{‘will
                    s/he...?’} } &  \\
\multicolumn{2}{l}{ } &  \\

                     \vernacular{
                    ala[khálaka]?}  &   
                     \gloss{‘cut’}  &  \\

                     \vernacular{
                    ala[tsúunzuuna]?}  &   
                     \gloss{‘suck’}  &  \\

                     \vernacular{
                    ala[lakhuula]?}  &   
                     \gloss{‘release’}  &  \\

                     \vernacular{
                    ala[kalushila]?}  &   
                     \gloss{‘re-do’}  &  \\
\end{tabular}
%\caption{\nocaption}
     
\begin{tabular}{lll}  
  \multicolumn{2}{l}{
                     \vernacular{(833)
                    WH-Questions} \gloss{‘who will
                    s/he...?’} } &  \\
\multicolumn{2}{l}{ } &  \\

                     \vernacular{ala[khá{\downstep}láká]
                    bí?}  &   
                     \gloss{‘cut’}  &  \\

                     \vernacular{
                    ala[tsú{\downstep}únzúúná] bí?}  &   
                     \gloss{‘suck’}  &  \\

                     \vernacular{ala[lakhúúla]
                    bi?}  &   
                     \gloss{‘release’}  &  \\

                     \vernacular{ala[kalúshítsa]
                    bi?}  &   
                     \gloss{‘return’}  &  \\
\end{tabular}
%\caption{\nocaption}
     
\begin{tabular}{lll}  
  \multicolumn{2}{l}{
                     \vernacular{(834) Subject
                    Relatives} \gloss{‘the person
                    \ob muundu\cb  / the man \ob musáatsa\cb  who
                    will...’} } &  \\
\multicolumn{2}{l}{ } &  \\

                     \vernacular{muúndu
                    u[khalaká]}  &   
                     \gloss{‘cut’}  &  \\

                     \vernacular{muúndu
                    u[lakhúula]}  &   
                     \gloss{‘release’}  &  \\

                     \vernacular{musáatsa
                    u[khalaká]}  &   
                     \gloss{‘cut’}  &  \\

                     \vernacular{musáatsa
                    u[lakhúula]}  &   
                     \gloss{‘release’}  &  \\
\end{tabular}
%\caption{\nocaption}
    

\subsection{Near Future Negative}\label{sec:sPQRNearFutNeg}


\begin{tabular}{lll}  
  \multicolumn{2}{l}{
                     \vernacular{(835)
                    Passives} \gloss{‘s/he will not
                    be...’} } &  \\
\multicolumn{2}{l}{ } &  \\

                     \vernacular{ala[khá{\downstep}lákwá]
                    tá}  &   
                     \gloss{‘cut’}  &  \\

                     \vernacular{
                    ala[tsú{\downstep}únzúúnwá] tá}  &   
                     \gloss{‘sucked’}  &  \\

                     \vernacular{ala[lákhúúlwá]
                    tá}  &   
                     \gloss{‘released’}  &  \\

                     \vernacular{
                    ala[kálúshítswá] tá}  &   
                     \gloss{‘returned’}  &  \\
\end{tabular}
%\caption{\nocaption}
     
\begin{tabular}{lll}  
  \multicolumn{2}{l}{
                     \vernacular{(836) Yes/No
                    Questions} \gloss{‘will s/he
                    not...?’} } &  \\
\multicolumn{2}{l}{ } &  \\

                     \vernacular{ala[khá{\downstep}láká]
                    tá?}  &   
                     \gloss{‘cut’}  &  \\

                     \vernacular{
                    ala[tsú{\downstep}únzúúná] tá?}  &   
                     \gloss{‘suck’}  &  \\

                     \vernacular{ala[lákhúúlá]
                    tá?}  &   
                     \gloss{‘release’}  &  \\

                     \vernacular{
                    ala[kálúshítsá] tá?}  &   
                     \gloss{‘return’}  &  \\
\end{tabular}
%\caption{\nocaption}
     
\begin{tabular}{lll}  
  \multicolumn{2}{l}{
                     \vernacular{(837)
                    WH-Questions} \gloss{‘who will s/he
                    not...?’} } &  \\
\multicolumn{2}{l}{ } &  \\

                     \vernacular{ala[khá{\downstep}láká]
                    bína tá?}  &   
                     \gloss{‘cut’}  &  \\

                     \vernacular{
                    ala[tsú{\downstep}únzúúná] bína tá?}  &   
                     \gloss{‘suck’}  &  \\

                     \vernacular{ala[lákhúúlá]
                    bína tá?}  &   
                     \gloss{‘release’}  &  \\

                     \vernacular{
                    ala[kálúshítsá] bína tá?}  &   
                     \gloss{‘return’}  &  \\
\end{tabular}
%\caption{\nocaption}
     
\begin{tabular}{lll}  
  \multicolumn{2}{l}{
                     \vernacular{(838) Subject
                    Relatives} \gloss{‘the person
                    \ob muundu\cb  / the man \ob musáatsa\cb  who will
                    not...’} } &  \\
\multicolumn{2}{l}{ } &  \\

                     \vernacular{muúndu
                    ukhá[khalaka] tá}  &   
                     \gloss{‘cut’}  &  \\

                     \vernacular{muúndu
                    ukhá[tsuunzuuna] tá}  &   
                     \gloss{‘suck’}  &  \\

                     \vernacular{muúndu
                    ukhá[{\downstep}lákhúula] tá}  &   
                     \gloss{‘release’}  &  \\

                     \vernacular{musáatsa
                    ukhá[khalaka] tá}  &   
                     \gloss{‘cut’}  &  \\

                     \vernacular{musáatsa
                    ukhá[tsuunzuuna] tá}  &   
                     \gloss{‘suck’}  &  \\

                     \vernacular{musáatsa
                    ukhá[{\downstep}lákhúula] tá}  &   
                     \gloss{‘release’}  &  \\
\end{tabular}
%\caption{\nocaption}
    

\subsection{Remote Past}\label{sec:sPQRRemPast}


\begin{tabular}{lll}  
  \multicolumn{2}{l}{
                     \vernacular{(839)
                    Passives} \gloss{‘s/he
                    was...’} } &  \\
\multicolumn{2}{l}{ } &  \\

                     \vernacular{
                    yaa[khá{\downstep}lákwá]}  &   
                     \gloss{‘cut’}  &  \\

                     \vernacular{
                    yaa[tsú{\downstep}únzúúnwá]}  &   
                     \gloss{‘sucked’}  &  \\

                     \vernacular{
                    yaa[lá{\downstep}khúúlwá]}  &   
                     \gloss{‘released’}  &  \\

                     \vernacular{
                    yaa[ká{\downstep}lúshítswá]}  &   
                     \gloss{‘returned’}  &  \\
\end{tabular}
%\caption{\nocaption}
     
\begin{tabular}{lll}  
  \multicolumn{2}{l}{
                     \vernacular{(840) Yes/No
                    Questions} \gloss{‘did
                    s/he...?’} } &  \\
\multicolumn{2}{l}{ } &  \\

                     \vernacular{
                    yaa[khálaka]?}  &   
                     \gloss{‘cut’}  &  \\

                     \vernacular{
                    yaa[tsúunzuuna]?}  &   
                     \gloss{‘suck’}  &  \\

                     \vernacular{
                    yaa[lákhuula]?}  &   
                     \gloss{‘release’}  &  \\

                     \vernacular{
                    yaa[kálushitsa]?}  &   
                     \gloss{‘return’}  &  \\
\end{tabular}
%\caption{\nocaption}
     
\begin{tabular}{lll}  
  \multicolumn{2}{l}{
                     \vernacular{(841)
                    WH-Questions} \gloss{‘who did
                    s/he...?’} } &  \\
\multicolumn{2}{l}{ } &  \\

                     \vernacular{yaa[khálaka]
                    bi?}  &   
                     \gloss{‘cut’}  &  \\

                     \vernacular{yaa[tsúunzuuna]
                    bi?}  &   
                     \gloss{‘suck’}  &  \\

                     \vernacular{yaa[lákhuula]
                    bi?}  &   
                     \gloss{‘release’}  &  \\

                     \vernacular{yaa[kálushitsa]
                    bi?}  &   
                     \gloss{‘return’}  &  \\
\end{tabular}
%\caption{\nocaption}
     
\begin{tabular}{lll}  
  \multicolumn{2}{l}{
                     \vernacular{(842) Subject
                    Relatives} \gloss{‘the person
                    \ob muundu\cb  / the man \ob musáatsa\cb 
                    who...’} } &  \\
\multicolumn{2}{l}{ } &  \\

                     \vernacular{muúndu
                    waa[khálaka]}  &   
                     \gloss{‘cut’}  &  \\

                     \vernacular{muúndu
                    waa[lákhuula]}  &   
                     \gloss{‘released’}  &  \\

                     \vernacular{musáatsa
                    waa[khálaka]}  &   
                     \gloss{‘cut’}  &  \\

                     \vernacular{musáatsa
                    waa[lákhuula]}  &   
                     \gloss{‘released’}  &  \\
\end{tabular}
%\caption{\nocaption}
    

\subsection{Remote Past Negative}\label{sec:sPQRRemPastNeg}


\begin{tabular}{lll}  
  \multicolumn{2}{l}{
                     \vernacular{(843)
                    Passives} \gloss{‘s/he was
                    not...’} } &  \\
\multicolumn{2}{l}{ } &  \\

                     \vernacular{yaa[khá{\downstep}lákwá]
                    {\downstep}tá}  &   
                     \gloss{‘cut’}  &  \\

                     \vernacular{
                    yaa[tsú{\downstep}únzúúnwá] {\downstep}tá}  &   
                     \gloss{‘sucked’}  &  \\

                     \vernacular{
                    yaa[lá{\downstep}khúúlwá] {\downstep}tá}  &   
                     \gloss{‘released’}  &  \\

                     \vernacular{
                    yaa[ká{\downstep}lúshítswá] {\downstep}tá}  &   
                     \gloss{‘returned’}  &  \\
\end{tabular}
%\caption{\nocaption}
     
\begin{tabular}{lll}  
  \multicolumn{2}{l}{
                     \vernacular{(844) Yes/No
                    Questions} \gloss{‘did s/he
                    not...?’} } &  \\
\multicolumn{2}{l}{ } &  \\

                     \vernacular{yaa[khálaka]
                    tá?}  &   
                     \gloss{‘cut’}  &  \\

                     \vernacular{yaa[tsúunzuuna]
                    tá?}  &   
                     \gloss{‘suck’}  &  \\

                     \vernacular{yaa[lákhuula]
                    tá?}  &   
                     \gloss{‘release’}  &  \\

                     \vernacular{yaa[kálushitsa]
                    tá?}  &   
                     \gloss{‘return’}  &  \\
\end{tabular}
%\caption{\nocaption}
     
\begin{tabular}{lll}  
  \multicolumn{2}{l}{
                     \vernacular{(845)
                    WH-Questions} \gloss{‘who did s/he
                    not...?’} } &  \\
\multicolumn{2}{l}{ } &  \\

                     \vernacular{yaa[khálaka] bi
                    tá?}  &   
                     \gloss{‘cut’}  &  \\

                     \vernacular{yaa[tsúunzuuna]
                    bi tá?}  &   
                     \gloss{‘suck’}  &  \\

                     \vernacular{yaa[lákhuula] bi
                    tá?}  &   
                     \gloss{‘release’}  &  \\

                     \vernacular{yaa[kálushitsa]
                    bi tá?}  &   
                     \gloss{‘return’}  &  \\
\end{tabular}
%\caption{\nocaption}
     
\begin{tabular}{lll}  
  \multicolumn{2}{l}{
                     \vernacular{(846) Subject
                    Relatives} \gloss{‘the person
                    \ob muundu\cb  / the man \ob musáatsa\cb  who did
                    not...’} } &  \\
\multicolumn{2}{l}{ } &  \\

                     \vernacular{muúndu
                    waa[khálaka] tá}  &   
                     \gloss{‘cut’}  &  \\

                     \vernacular{muúndu
                    waa[lákhuula] tá}  &   
                     \gloss{‘release’}  &  \\

                     \vernacular{musáatsa
                    waa[khálaka] tá}  &   
                     \gloss{‘cut’}  &  \\

                     \vernacular{musáatsa
                    waa[lákhuula] tá}  &   
                     \gloss{‘release’}  &  \\
\end{tabular}
%\caption{\nocaption}
    

\subsection{Immediate Past}\label{sec:sPQRImmPast}


\begin{tabular}{lll}  
  \multicolumn{2}{l}{
                     \vernacular{(847)
                    Passives} \gloss{‘s/he was
                    just...’} } &  \\
\multicolumn{2}{l}{ } &  \\

                     \vernacular{
                    yá{\downstep}khá[khálakwa]}  &   
                     \gloss{‘cut’}  &  \\

                     \vernacular{
                    yá{\downstep}khá[tsúunzuunwa]}  &   
                     \gloss{‘sucked’}  &  \\

                     \vernacular{
                    yákha[lakhuulwa]}  &   
                     \gloss{‘released’}  &  \\

                     \vernacular{
                    yákha[kalushitswa]}  &   
                     \gloss{‘returned’}  &  \\
\end{tabular}
%\caption{\nocaption}
     
\begin{tabular}{lll}  
  \multicolumn{2}{l}{
                     \vernacular{(848) Yes/No
                    Questions} \gloss{‘did s/he
                    just...?’} } &  \\
\multicolumn{2}{l}{ } &  \\

                     \vernacular{
                    yá{\downstep}khá[khálaka]?}  &   
                     \gloss{‘cut’}  &  \\

                     \vernacular{
                    yá{\downstep}khá[tsúunzuuna]?}  &   
                     \gloss{‘suck’}  &  \\

                     \vernacular{
                    yákha[lakhuula]?}  &   
                     \gloss{‘release’}  &  \\

                     \vernacular{
                    yákha[kalushitsa]?}  &   
                     \gloss{‘return’}  &  \\
\end{tabular}
%\caption{\nocaption}
     
\begin{tabular}{lll}  
  \multicolumn{2}{l}{
                     \vernacular{(849)
                    WH-Questions} \gloss{‘who did s/he
                    just...?’} } &  \\
\multicolumn{2}{l}{ } &  \\

                     \vernacular{
                    yá{\downstep}khá[khá{\downstep}láká] bi?}  &   
                     \gloss{‘cut’}  &  \\

                     \vernacular{
                    yá{\downstep}khá[tsú{\downstep}úzúúná] bi?}  &   
                     \gloss{‘suck’}  &  \\

                     \vernacular{
                    yá{\downstep}khá[lákhúula] bi?}  &   
                     \gloss{‘release’}  &  \\

                     \vernacular{
                    yá{\downstep}khá[kálúshítsa] bi?}  &   
                     \gloss{‘return’}  &  \\
\end{tabular}
%\caption{\nocaption}
     
\begin{tabular}{lll}  
  \multicolumn{2}{l}{
                     \vernacular{(850) Subject
                    Relatives} \gloss{‘the person
                    \ob muundu\cb  / the man \ob musáatsa\cb  who
                    just...’} } &  \\
\multicolumn{2}{l}{ } &  \\

                     \vernacular{muúndu
                    wá{\downstep}khá[khá{\downstep}láká]}  &   
                     \gloss{‘cut’}  &  \\

                     \vernacular{muúndu
                    wá{\downstep}khá[tsú{\downstep}únzúúná]}  &   
                     \gloss{‘sucked’}  &  \\

                     \vernacular{muúndu
                    wá{\downstep}khá[lákhúula]}  &   
                     \gloss{‘released’}  &  \\

                     \vernacular{musáatsa
                    wá{\downstep}khá[khá{\downstep}láká]}  &   
                     \gloss{‘cut’}  &  \\

                     \vernacular{musáatsa
                    wá{\downstep}khá[tsú{\downstep}únzúúná]}  &   
                     \gloss{‘sucked’}  &  \\

                     \vernacular{musáatsa
                    wá{\downstep}khá[lákhúula]}  &   
                     \gloss{‘released’}  &  \\
\end{tabular}
%\caption{\nocaption}
    

\subsection{Immediate Past Negative}\label{sec:sPQRImmPastNeg}


\begin{tabular}{lll}  
  \multicolumn{2}{l}{
                     \vernacular{(851)
                    Passives} \gloss{‘s/he was not
                    just...’} } &  \\
\multicolumn{2}{l}{ } &  \\

                     \vernacular{
                    yá{\downstep}khá[khá{\downstep}lákwá] tá}  &   
                     \gloss{‘cut’}  &  \\

                     \vernacular{
                    yá{\downstep}khá[tsú{\downstep}únzúúnwá] tá}  &   
                     \gloss{‘sucked’}  &  \\

                     \vernacular{
                    yá{\downstep}khá[lákhúúlwá] tá}  &   
                     \gloss{‘released’}  &  \\

                     \vernacular{
                    yá{\downstep}khá[kálúshítswá] tá}  &   
                     \gloss{‘returned’}  &  \\
\end{tabular}
%\caption{\nocaption}
     
\begin{tabular}{lll}  
  \multicolumn{2}{l}{
                     \vernacular{(852) Yes/No
                    Questions} \gloss{‘did s/he not
                    just...?’} } &  \\
\multicolumn{2}{l}{ } &  \\

                     \vernacular{
                    yá{\downstep}khá[khá{\downstep}laka] tá?}  &   
                     \gloss{‘cut’}  &  \\

                     \vernacular{
                    yá{\downstep}khá[tsú{\downstep}únzúúná] tá?}  &   
                     \gloss{‘suck’}  &  \\

                     \vernacular{
                    yá{\downstep}khá[lákhúúlá] tá?}  &   
                     \gloss{‘release’}  &  \\

                     \vernacular{
                    yá{\downstep}khá[kálúshítsá] tá?}  &   
                     \gloss{‘return’}  &  \\
\end{tabular}
%\caption{\nocaption}
     
\begin{tabular}{lll}  
  \multicolumn{2}{l}{
                     \vernacular{(853)
                    WH-Questions} \gloss{‘who did s/he not
                    just...?’} } &  \\
\multicolumn{2}{l}{ } &  \\

                     \vernacular{
                    yá{\downstep}khá[khá{\downstep}láká] bi tá?}  &   
                     \gloss{‘cut’}  &  \\

                     \vernacular{
                    yá{\downstep}khá[tsú{\downstep}úzúúná] bi tá?}  &   
                     \gloss{‘suck’}  &  \\

                     \vernacular{
                    yá{\downstep}khá[lákhúula] bi tá?}  &   
                     \gloss{‘release’}  &  \\

                     \vernacular{
                    yá{\downstep}khá[kálúshítsa] bi tá?}  &   
                     \gloss{‘return’}  &  \\
\end{tabular}
%\caption{\nocaption}
     
\begin{tabular}{lll}  
  \multicolumn{2}{l}{
                     \vernacular{(854) Subject
                    Relatives} \gloss{‘the person
                    \ob muundu\cb  / the man \ob musáatsa\cb  who did not
                    just...’} } &  \\
\multicolumn{2}{l}{ } &  \\

                     \vernacular{muúndu
                    wá{\downstep}khá[khá{\downstep}láká] {\downstep}tá}  &   
                     \gloss{‘cut’}  &  \\

                     \vernacular{muúndu
                    wá{\downstep}khá[tsú{\downstep}únzúúná] {\downstep}tá}  &   
                     \gloss{‘suck’}  &  \\

                     \vernacular{muúndu
                    wá{\downstep}khá[lákhúula] tá}  &   
                     \gloss{‘release’}  &  \\

                     \vernacular{musáatsa
                    wá{\downstep}khá[khá{\downstep}láká] {\downstep}tá}  &   
                     \gloss{‘cut’}  &  \\

                     \vernacular{musáatsa
                    wá{\downstep}khá[tsú{\downstep}únzúúná] {\downstep}tá}  &   
                     \gloss{‘suck’}  &  \\

                     \vernacular{musáatsa
                    wá{\downstep}khá[lákhúula] tá}  &   
                     \gloss{‘release’}  &  \\
\end{tabular}
%\caption{\nocaption}
    

\subsection{Remote Future}\label{sec:sPQRRemFut}


\begin{tabular}{lll}  
  \multicolumn{2}{l}{
                     \vernacular{(855)
                    Passives} \gloss{‘s/he might
                    be...’} } &  \\
\multicolumn{2}{l}{ } &  \\

                     \vernacular{
                    yakha[khálakwi]}  &   
                     \gloss{‘cut’}  &  \\

                     \vernacular{
                    yakha[tsúunzuunwi]}  &   
                     \gloss{‘sucked’}  &  \\

                     \vernacular{
                    yakha[lakhúulwi]}  &   
                     \gloss{‘released’}  &  \\

                     \vernacular{
                    yakha[kalúshitswi]}  &   
                     \gloss{‘returned’}  &  \\
\end{tabular}
%\caption{\nocaption}
     
\begin{tabular}{lll}  
  \multicolumn{2}{l}{
                     \vernacular{(856) Yes/No
                    Questions} \gloss{‘might
                    s/he...?’} } &  \\
\multicolumn{2}{l}{ } &  \\

                     \vernacular{
                    yakha[khálachɛ]?}  &   
                     \gloss{‘cut’}  &  \\

                     \vernacular{
                    yakha[tsúunzuunɪ]?}  &   
                     \gloss{‘suck’}  &  \\

                     \vernacular{
                    yakha[lakhúulɪ]?}  &   
                     \gloss{‘release’}  &  \\

                     \vernacular{
                    yakha[kalúshitsɪ]?}  &   
                     \gloss{‘return’}  &  \\
\end{tabular}
%\caption{\nocaption}
     
\begin{tabular}{lll}  
  \multicolumn{2}{l}{
                     \vernacular{(857)
                    WH-Questions} \gloss{‘who might
                    s/he...?’} } &  \\
\multicolumn{2}{l}{ } &  \\

                     \vernacular{
                    yakha[khá{\downstep}láchɛ́] bí?}  &   
                     \gloss{‘cut’}  &  \\

                     \vernacular{
                    yakha[tsú{\downstep}úzúúnɪ́] bí?}  &   
                     \gloss{‘suck’}  &  \\

                     \vernacular{yakha[lakhúúlɪ]
                    bi?}  &   
                     \gloss{‘release’}  &  \\

                     \vernacular{
                    yakha[kalúshítsɪ] bi?}  &   
                     \gloss{‘return’}  &  \\
\end{tabular}
%\caption{\nocaption}
     
\begin{tabular}{lll}  
  \multicolumn{2}{l}{
                     \vernacular{(858) Subject
                    Relatives} \gloss{‘the person
                    \ob muundu\cb  / the man \ob musáatsa\cb  who
                    might...’} } &  \\
\multicolumn{2}{l}{ } &  \\

                     \vernacular{muúndu
                    wá{\downstep}khá[khá{\downstep}láchɛ́]}  &   
                     \gloss{‘cut’}  &  \\

                     \vernacular{muúndu
                    wá{\downstep}khá[tsú{\downstep}únzúúnɪ́]}  &   
                     \gloss{‘suck’}  &  \\

                     \vernacular{muúndu
                    wá{\downstep}khá[lákhúulɪ]}  &   
                     \gloss{‘release’}  &  \\

                     \vernacular{musáatsa
                    wá{\downstep}khá[khá{\downstep}láchɛ́]}  &   
                     \gloss{‘cut’}  &  \\

                     \vernacular{musáatsa
                    wá{\downstep}khá[tsú{\downstep}únzúúnɪ́]}  &   
                     \gloss{‘suck’}  &  \\

                     \vernacular{musáatsa
                    wá{\downstep}khá[lákhúulɪ]}  &   
                     \gloss{‘release’}  &  \\
\end{tabular}
%\caption{\nocaption}
    

\subsection{Remote Future Negative}\label{sec:sPQRRemFutNeg}


\begin{tabular}{lll}  
  \multicolumn{2}{l}{
                     \vernacular{(859)
                    Passives} \gloss{‘s/he might not
                    be...’} } &  \\
\multicolumn{2}{l}{ } &  \\

                     \vernacular{
                    yakha[khá{\downstep}lákwí] tá}  &   
                     \gloss{‘cut’}  &  \\

                     \vernacular{
                    yakha[tsú{\downstep}únzúúnwí] tá}  &   
                     \gloss{‘sucked’}  &  \\

                     \vernacular{yakha[lakhúulwi]
                    tá}  &   
                     \gloss{‘released’}  &  \\

                     \vernacular{
                    yakha[kalúshitswi] tá}  &   
                     \gloss{‘returned’}  &  \\
\end{tabular}
%\caption{\nocaption}
     
\begin{tabular}{lll}  
  \multicolumn{2}{l}{
                     \vernacular{(860) Yes/No
                    Questions} \gloss{‘might s/he
                    not...?’} } &  \\
\multicolumn{2}{l}{ } &  \\

                     \vernacular{
                    yakha[khá{\downstep}láchɛ́] tá?}  &   
                     \gloss{‘cut’}  &  \\

                     \vernacular{
                    yakha[tsú{\downstep}únzúúnɪ́] tá?}  &   
                     \gloss{‘suck’}  &  \\

                     \vernacular{yakha[lakhúulɪ]
                    tá?}  &   
                     \gloss{‘release’}  &  \\

                     \vernacular{yakha[kalúshitsɪ]
                    tá?}  &   
                     \gloss{‘return’}  &  \\
\end{tabular}
%\caption{\nocaption}
     
\begin{tabular}{lll}  
  \multicolumn{2}{l}{
                     \vernacular{(861)
                    WH-Questions} \gloss{‘who might s/he
                    not...?’} } &  \\
\multicolumn{2}{l}{ } &  \\

                     \vernacular{
                    yakha[khá{\downstep}láchɛ́] bí {\downstep}tá?}  &   
                     \gloss{‘cut’}  &  \\

                     \vernacular{
                    yakha[tsú{\downstep}úzúúnɪ́] bí {\downstep}tá?}  &   
                     \gloss{‘suck’}  &  \\

                     \vernacular{yakha[lakhúúlɪ]
                    bit á?}  &   
                     \gloss{‘release’}  &  \\

                     \vernacular{
                    yakha[kalúshítsɪ] bi tá?}  &   
                     \gloss{‘return’}  &  \\
\end{tabular}
%\caption{\nocaption}
     
\begin{tabular}{lll}  
  \multicolumn{2}{l}{
                     \vernacular{(862) Subject
                    Relatives} \gloss{‘the person
                    \ob muundu\cb  / the man \ob musáatsa\cb  who might
                    not...’} } &  \\
\multicolumn{2}{l}{ } &  \\

                     \vernacular{muúndu
                    wá{\downstep}khá[kháláchɛ́] {\downstep}tá}  &   
                     \gloss{‘cut’}  &  \\

                     \vernacular{muúndu
                    wá{\downstep}khá[tsúúnzúúnɪ́] {\downstep}tá}  &   
                     \gloss{‘suck’}  &  \\

                     \vernacular{muúndu
                    wá{\downstep}khá[lákhúulɪ] tá}  &   
                     \gloss{‘release’}  &  \\

                     \vernacular{musáatsa
                    wá{\downstep}khá[kháláchɛ́] {\downstep}tá}  &   
                     \gloss{‘cut’}  &  \\

                     \vernacular{musáatsa
                    wá{\downstep}khá[tsúúnzúúnɪ́] {\downstep}tá}  &   
                     \gloss{‘suck’}  &  \\

                     \vernacular{musáatsa
                    wá{\downstep}khá[lákhúulɪ] tá}  &   
                     \gloss{‘release’}  &  \\
\end{tabular}
%\caption{\nocaption}
    

\subsection{Present}\label{sec:sPQRPres}


\begin{tabular}{lll}  
  \multicolumn{2}{l}{
                     \vernacular{(863)
                    Passives} \gloss{‘s/he is
                    being...’} } &  \\
\multicolumn{2}{l}{ } &  \\

                     \vernacular{
                    a[khalakwáá{\downstep}ngúa]}  &   
                     \gloss{‘cutting’}  &  \\

                     \vernacular{
                    a[tsuunzuunwáá{\downstep}ngúa]}  &   
                     \gloss{‘sucking’}  &  \\

                     \vernacular{
                    a[lakhú{\downstep}úlwáángúa]}  &   
                     \gloss{‘releasing’}  &  \\

                     \vernacular{
                    a[kalúshítswaangua]}  &   
                     \gloss{‘returning’}  &  \\
\end{tabular}
%\caption{\nocaption}
     
\begin{tabular}{lll}  
  \multicolumn{2}{l}{
                     \vernacular{(864) Yes/No
                    Questions} \gloss{‘is
                    s/he...?’} } &  \\
\multicolumn{2}{l}{ } &  \\

                     \vernacular{
                    a[khalakáánga]?}  &   
                     \gloss{‘cutting’}  &  \\

                     \vernacular{
                    a[tsuunzuunáánga]?}  &   
                     \gloss{‘sucking’}  &  \\

                     \vernacular{
                    a[lakhúulaanga]?}  &   
                     \gloss{‘releasing’}  &  \\

                     \vernacular{
                    a[kalúshítsaanga]?}  &   
                     \gloss{‘returning’}  &  \\
\end{tabular}
%\caption{\nocaption}
     
\begin{tabular}{lll}  
  \multicolumn{2}{l}{
                     \vernacular{(865)
                    WH-Questions} \gloss{‘who is
                    s/he...?’} } &  \\
\multicolumn{2}{l}{ } &  \\

                     \vernacular{a[khalakáá{\downstep}ngá]
                    bi?}  &   
                     \gloss{‘cutting’}  &  \\

                     \vernacular{
                    a[tsuunzuunáá{\downstep}ngá] bi?}  &   
                     \gloss{‘sucking’}  &  \\

                     \vernacular{
                    a[lakhú{\downstep}úláángá] bi?}  &   
                     \gloss{‘releasing’}  &  \\

                     \vernacular{
                    a[kalúshí{\downstep}tsáángá] bi?}  &   
                     \gloss{‘returning’}  &  \\
\end{tabular}
%\caption{\nocaption}
     
\begin{tabular}{lll}  
  \multicolumn{2}{l}{
                     \vernacular{(866) Subject
                    Relatives} \gloss{‘the person
                    \ob muundu\cb  / the man \ob musáatsa\cb  who
                    is...’} } &  \\
\multicolumn{2}{l}{ } &  \\

                     \vernacular{muúndu
                    u[khá{\downstep}lákáánga]}  &   
                     \gloss{‘cutting’}  &  \\

                     \vernacular{muúndu
                    u[lákhuulaanga]}  &   
                     \gloss{‘releasing’}  &  \\

                     \vernacular{musáatsa
                    u[khá{\downstep}lákáánga]}  &   
                     \gloss{‘cutting’}  &  \\

                     \vernacular{musáatsa
                    u[lákhuulaanga]}  &   
                     \gloss{‘releasing’}  &  \\
\end{tabular}
%\caption{\nocaption}
    

\subsection{Present Negativ/e}\label{sec:sPQRPresNeg}


\begin{tabular}{lll}  
  \multicolumn{2}{l}{
                     \vernacular{(867)
                    Passives} \gloss{‘s/he is not
                    being...’} } &  \\
\multicolumn{2}{l}{ } &  \\

                     \vernacular{
                    a[khalakwáá{\downstep}ngwá] tá}  &   
                     \gloss{‘cutting’}  &  \\

                     \vernacular{
                    a[tsuunzuunwáá{\downstep}ngwá] tá}  &   
                     \gloss{‘sucking’}  &  \\

                     \vernacular{
                    a[lakhú{\downstep}úlwáángwá] tá}  &   
                     \gloss{‘releasing’}  &  \\

                     \vernacular{
                    a[kalúshí{\downstep}tswáángwá] tá}  &   
                     \gloss{‘returning’}  &  \\
\end{tabular}
%\caption{\nocaption}
     
\begin{tabular}{lll}  
  \multicolumn{2}{l}{
                     \vernacular{(868) Yes/No
                    Questions} \gloss{‘is s/he
                    not...?’} } &  \\
\multicolumn{2}{l}{ } &  \\

                     \vernacular{a[khalakáá{\downstep}ngá]
                    tá?}  &   
                     \gloss{‘cutting’}  &  \\

                     \vernacular{
                    a[tsuunzuunáá{\downstep}ngá] tá?}  &   
                     \gloss{‘sucking’}  &  \\

                     \vernacular{
                    a[lakhú{\downstep}úláángá] tá?}  &   
                     \gloss{‘releasing’}  &  \\

                     \vernacular{
                    a[kalúshí{\downstep}tsáángá] tá?}  &   
                     \gloss{‘returning’}  &  \\
\end{tabular}
%\caption{\nocaption}
     
\begin{tabular}{lll}  
  \multicolumn{2}{l}{
                     \vernacular{(869)
                    WH-Questions} \gloss{‘who is s/he
                    not...?’} } &  \\
\multicolumn{2}{l}{ } &  \\

                     \vernacular{
                    a[khalakáá({\downstep}ngá)] {\downstep}bí tá?}  &   
                     \gloss{‘cutting’}  &  \\

                     \vernacular{
                    a[tsuunzuunáá({\downstep}ngá)] {\downstep}bí tá?}  &   
                     \gloss{‘sucking’}  &  \\

                     \vernacular{
                    a[lakhúúlá{\downstep}á(ngá)] bí tá?}  &   
                     \gloss{‘releasing’}  &  \\

                     \vernacular{
                    a[kalúshí{\downstep}tsáá(ngá)] bí tá?}  &   
                     \gloss{‘returning’}  &  \\
\end{tabular}
%\caption{\nocaption}
     
\begin{tabular}{lll}  
  \multicolumn{2}{l}{
                     \vernacular{(870) Subject
                    Relatives} \gloss{‘the person
                    \ob muundu\cb  / the man \ob musáatsa\cb  who is
                    not...’} } &  \\
\multicolumn{2}{l}{ } &  \\

                     \vernacular{muúndu
                    ukhá[{\downstep}khálákáánga]}  &   
                     \gloss{‘cutting’}  &  \\

                     \vernacular{muúndu
                    ukhá[tsuunzuunaanga]}  &   
                     \gloss{‘sucking’}  &  \\

                     \vernacular{muúndu
                    ukhá[{\downstep}lákhúulaanga]}  &   
                     \gloss{‘releasing’}  &  \\

                     \vernacular{musáatsa
                    ukhá[{\downstep}khálákáánga]}  &   
                     \gloss{‘cutting’}  &  \\

                     \vernacular{musáatsa
                    ukhá[tsuunzuunaanga]}  &   
                     \gloss{‘sucking’}  &  \\

                     \vernacular{musáatsa
                    ukhá[{\downstep}lákhúulaanga]}  &   
                     \gloss{‘releasing’}  &  \\
\end{tabular}
%\caption{\nocaption}
    

\subsection{Indefinite Future}\label{sec:sPQRIndefFut}


\begin{tabular}{lll}  
  \multicolumn{2}{l}{
                     \vernacular{(871)
                    Passives} \gloss{‘s/he will
                    be...’} } &  \\
\multicolumn{2}{l}{ } &  \\

                     \vernacular{
                    ali[khalakwá]}  &   
                     \gloss{‘cut’}  &  \\

                     \vernacular{
                    ali[tsuunzuunwá]}  &   
                     \gloss{‘sucked’}  &  \\

                     \vernacular{
                    ali[lakhú{\downstep}úlwá]}  &   
                     \gloss{‘released’}  &  \\

                     \vernacular{
                    ali[kalú{\downstep}shítswá]}  &   
                     \gloss{‘returned’}  &  \\
\end{tabular}
%\caption{\nocaption}
     
\begin{tabular}{lll}  
  \multicolumn{2}{l}{
                     \vernacular{(872) Yes/No
                    Questions} \gloss{‘will s/he...?’
                    } } &  \\
\multicolumn{2}{l}{ } &  \\

                     \vernacular{
                    ali[khalaká]?}  &   
                     \gloss{‘cut’}  &  \\

                     \vernacular{
                    ali[tsuunzuuná]?}  &   
                     \gloss{‘suck’}  &  \\

                     \vernacular{
                    ali[lakhúula]?}  &   
                     \gloss{‘release’}  &  \\

                     \vernacular{
                    ali[kalúshitsa]?}  &   
                     \gloss{‘return’}  &  \\
\end{tabular}
%\caption{\nocaption}
     
\begin{tabular}{lll}  
  \multicolumn{2}{l}{
                     \vernacular{(873)
                    WH-Questions} \gloss{‘who will
                    s/he...?’} } &  \\
\multicolumn{2}{l}{ } &  \\

                     \vernacular{ali[khalaká]
                    bí?}  &   
                     \gloss{‘cut’}  &  \\

                     \vernacular{ali[tsuunzuuná]
                    bí?}  &   
                     \gloss{‘suck’}  &  \\

                     \vernacular{ali[lákhúúlá]
                    bí?}  &   
                     \gloss{‘release’}  &  \\

                     \vernacular{ali[kalushí{\downstep}tsá]
                    bí?}  &   
                     \gloss{‘return’}  &  \\
\end{tabular}
%\caption{\nocaption}
     
\begin{tabular}{lll}  
  \multicolumn{2}{l}{
                     \vernacular{(874) Subject
                    Relatives} \gloss{‘the person
                    \ob muundu\cb  / the man \ob musáatsa\cb  who
                    will...’} } &  \\
\multicolumn{2}{l}{ } &  \\

                     \vernacular{muúndu
                    ulí[{\downstep}kháláká]}  &   
                     \gloss{‘cut’}  &  \\

                     \vernacular{muúndu
                    ulí[{\downstep}lákhúula]}  &   
                     \gloss{‘release’}  &  \\

                     \vernacular{musáatsa
                    ulí[{\downstep}kháláká]}  &   
                     \gloss{‘cut’}  &  \\

                     \vernacular{musáatsa
                    ulí[{\downstep}lákhúula]}  &   
                     \gloss{‘release’}  &  \\
\end{tabular}
%\caption{\nocaption}
    

\subsection{Indefinite Future Negative}\label{sec:sPQRIndefFutNeg}


\begin{tabular}{lll}  
  \multicolumn{2}{l}{
                     \vernacular{(875)
                    Passives} \gloss{‘s/he will not
                    be...’} } &  \\
\multicolumn{2}{l}{ } &  \\

                     \vernacular{ali[khalakwá]
                    {\downstep}tá}  &   
                     \gloss{‘cut’}  &  \\

                     \vernacular{ali[tsuunzuunwá]
                    {\downstep}tá}  &   
                     \gloss{‘sucked’}  &  \\

                     \vernacular{ali[lakhú{\downstep}úlwá]
                    {\downstep}tá}  &   
                     \gloss{‘released’}  &  \\

                     \vernacular{
                    ali[kalúshí{\downstep}tswá] {\downstep}tá}  &   
                     \gloss{‘returned’}  &  \\
\end{tabular}
%\caption{\nocaption}
     
\begin{tabular}{lll}  
  \multicolumn{2}{l}{
                     \vernacular{(876) Yes/No
                    Questions} \gloss{‘will s/he
                    not...?’} } &  \\
\multicolumn{2}{l}{ } &  \\

                     \vernacular{ali[khalaká]
                    {\downstep}tá?}  &   
                     \gloss{‘cut’}  &  \\

                     \vernacular{ali[tsuunzuuná]
                    {\downstep}tá?}  &   
                     \gloss{‘suck’}  &  \\

                     \vernacular{ali[lakhúula]
                    tá?}  &   
                     \gloss{‘release’}  &  \\

                     \vernacular{ali[kalúshítsa]
                    tá?}  &   
                     \gloss{‘return’}  &  \\
\end{tabular}
%\caption{\nocaption}
     
\begin{tabular}{lll}  
  \multicolumn{2}{l}{
                     \vernacular{(877)
                    WH-Questions} \gloss{‘who will s/he
                    not...?’} } &  \\
\multicolumn{2}{l}{ } &  \\

                     \vernacular{ali[khalaká] bi
                    tá?}  &   
                     \gloss{‘cut’}  &  \\

                     \vernacular{ali[tsuunzuuná]
                    bi tá?}  &   
                     \gloss{‘suck’}  &  \\

                     \vernacular{ali[lakhúúla] bi
                    tá?}  &   
                     \gloss{‘release’}  &  \\

                     \vernacular{ali[kalushítsa]
                    bi tá?}  &   
                     \gloss{‘return’}  &  \\
\end{tabular}
%\caption{\nocaption}
     
\begin{tabular}{lll}  
  \multicolumn{2}{l}{
                     \vernacular{(878) Subject
                    Relatives} \gloss{‘the person
                    \ob muundu\cb  / the man \ob musáatsa\cb  who will
                    not...’} } &  \\
\multicolumn{2}{l}{ } &  \\

                     \vernacular{muúndu
                    ulí[{\downstep}kháláká] {\downstep}tá}  &   
                     \gloss{‘cut’}  &  \\

                     \vernacular{muúndu
                    ulí[{\downstep}lákhúula] tá}  &   
                     \gloss{‘release’}  &  \\

                     \vernacular{musáatsa
                    ulí[{\downstep}kháláká] {\downstep}tá}  &   
                     \gloss{‘cut’}  &  \\

                     \vernacular{musáatsa
                    ulí[{\downstep}lákhúula] tá}  &   
                     \gloss{‘release’}  &  \\
\end{tabular}
%\caption{\nocaption}
    

\subsection{Imperative
              }\label{sec:sPQRImpSg}


\begin{tabular}{lll}  
  \multicolumn{2}{l}{
                     \vernacular{(879)
                    Passives} \gloss{
                    ‘be...!’} } &  \\
\multicolumn{2}{l}{ } &  \\

                     \vernacular{
                    ali[khalakúa]}  &   
                     \gloss{‘cut’}  &  \\

                     \vernacular{
                    ali[tsúúnzúúnúa]}  &   
                     \gloss{‘sucked’}  &  \\

                     \vernacular{
                    ali[lákhúú{\downstep}lúa]}  &   
                     \gloss{‘released’}  &  \\

                     \vernacular{
                    ali[kálúshí{\downstep}tsúa]}  &   
                     \gloss{‘returned’}  &  \\
\end{tabular}
%\caption{\nocaption}
     
\begin{tabular}{lll}  
  \multicolumn{2}{l}{
                     \vernacular{(880) Yes/No
                    Questions} \gloss{
                    ‘...!?’} } &  \\
\multicolumn{2}{l}{ } &  \\

                     \vernacular{
                    ali[khalaká]?}  &   
                     \gloss{‘cut’}  &  \\

                     \vernacular{
                    ali[tsuunzuuná]?}  &   
                     \gloss{‘suck’}  &  \\

                     \vernacular{
                    ali[lakhúula]?}  &   
                     \gloss{‘release’}  &  \\

                     \vernacular{
                    ali[kálúshí{\downstep}tsá]?}  &   
                     \gloss{‘return’}  &  \\
\end{tabular}
%\caption{\nocaption}
     
\begin{tabular}{lll}  
  \multicolumn{2}{l}{
                     \vernacular{(881)
                    WH-Questions} \gloss{
                    ‘...!?’} } &  \\
\multicolumn{2}{l}{ } &  \\

                     \vernacular{ali[khalaka]
                    bí?}  &   
                     \gloss{‘cut’}  &  \\

                     \vernacular{ali[tsuunzuuna]
                    bí?}  &   
                     \gloss{‘suck’}  &  \\

                     \vernacular{ali[lakhuula]
                    bi?}  &   
                     \gloss{‘release’}  &  \\

                     \vernacular{ali[kalushitsa]
                    bi?}  &   
                     \gloss{‘return’}  &  \\
\end{tabular}
%\caption{\nocaption}
    

\subsection{Imperative
              }\label{sec:sPQRImpSgNeg}


\begin{tabular}{lll}  
  \multicolumn{2}{l}{
                     \vernacular{(882)
                    Passives} \gloss{‘do not
                    be...!’} } &  \\
\multicolumn{2}{l}{ } &  \\

                     \vernacular{ukha[khalakwa]
                    tá}  &   
                     \gloss{‘cut’}  &  \\

                     \vernacular{ukha[tsuunzuunwa]
                    tá}  &   
                     \gloss{‘sucked’}  &  \\

                     \vernacular{
                    ukha[lakhú{\downstep}úlwá] {\downstep}tá}  &   
                     \gloss{‘released’}  &  \\

                     \vernacular{
                    ukha[kalú{\downstep}shítswá] {\downstep}tá}  &   
                     \gloss{‘returned’}  &  \\
\end{tabular}
%\caption{\nocaption}
     
\begin{tabular}{lll}  
  \multicolumn{2}{l}{
                     \vernacular{(883) Yes/No
                    Questions} \gloss{‘do
                    not...!?’} } &  \\
\multicolumn{2}{l}{ } &  \\

                     \vernacular{ukha[khalaka]
                    tá?}  &   
                     \gloss{‘cut’}  &  \\

                     \vernacular{ukha[tsuunzuuna]
                    tá?}  &   
                     \gloss{‘suck’}  &  \\

                     \vernacular{ukha[lakhúula]
                    tá?}  &   
                     \gloss{‘release’}  &  \\

                     \vernacular{ukha[kalúshítsa]
                    tá?}  &   
                     \gloss{‘return’}  &  \\
\end{tabular}
%\caption{\nocaption}
     
\begin{tabular}{lll}  
  \multicolumn{2}{l}{
                     \vernacular{(884)
                    WH-Questions} \gloss{‘do not...who!?’
                    } } &  \\
\multicolumn{2}{l}{ } &  \\

                     \vernacular{ukha[khalaka] bi
                    tá?}  &   
                     \gloss{‘cut’}  &  \\

                     \vernacular{ukha[tsuunzuuna]
                    bi tá?}  &   
                     \gloss{‘suck’}  &  \\

                     \vernacular{ukha[lakhúula] bi
                    tá?}  &   
                     \gloss{‘release’}  &  \\

                     \vernacular{ukha[kalúshítsa]
                    bi tá?}  &   
                     \gloss{‘return’}  &  \\
\end{tabular}
%\caption{\nocaption}
    

\subsection{Crastinal Future}\label{sec:sPQRCrastFut}


\begin{tabular}{lll}  
  \multicolumn{2}{l}{
                     \vernacular{(885)
                    Passives} \gloss{‘s/he will
                    be...’} } &  \\
\multicolumn{2}{l}{ } &  \\

                     \vernacular{
                    naa[khalakwí]}  &   
                     \gloss{‘cut’}  &  \\

                     \vernacular{
                    naa[tsuunzuú{\downstep}nwí]}  &   
                     \gloss{‘sucked’}  &  \\

                     \vernacular{
                    naa[lakhuú{\downstep}lwí]}  &   
                     \gloss{‘released’}  &  \\

                     \vernacular{
                    naa[kalushí{\downstep}tswí]}  &   
                     \gloss{‘returned’}  &  \\
\end{tabular}
%\caption{\nocaption}
     
\begin{tabular}{lll}  
  \multicolumn{2}{l}{
                     \vernacular{(886) Yes/No
                    Questions} \gloss{‘will
                    s/he...?’} } &  \\
\multicolumn{2}{l}{ } &  \\

                     \vernacular{
                    naa[khalachɛ́]?}  &   
                     \gloss{‘cut’}  &  \\

                     \vernacular{
                    naa[tsuunzuúnɪ]?}  &   
                     \gloss{‘suck’}  &  \\

                     \vernacular{
                    naa[lakhuúlɪ]?}  &   
                     \gloss{‘release’}  &  \\

                     \vernacular{
                    naa[kalushítsɪ]?}  &   
                     \gloss{‘return’}  &  \\
\end{tabular}
%\caption{\nocaption}
     
\begin{tabular}{lll}  
  \multicolumn{2}{l}{
                     \vernacular{(887)
                    WH-Questions} \gloss{‘who will
                    s/he...?’} } &  \\
\multicolumn{2}{l}{ } &  \\

                     \vernacular{naa[khalachɛ́]
                    bi?}  &   
                     \gloss{‘cut’}  &  \\

                     \vernacular{naa[tsuunzuúnɪ]
                    bi?}  &   
                     \gloss{‘suck’}  &  \\

                     \vernacular{naa[lakhuúlɪ]
                    bi?}  &   
                     \gloss{‘release’}  &  \\

                     \vernacular{naa[kalushítsɪ]
                    bi?}  &   
                     \gloss{‘return’}  &  \\
\end{tabular}
%\caption{\nocaption}
     
\begin{tabular}{lll}  
  \multicolumn{2}{l}{
                     \vernacular{(888) Subject
                    Relatives} \gloss{‘the person
                    \ob muundu\cb  / the man \ob musáatsa\cb  who
                    will...’} } &  \\
\multicolumn{2}{l}{ } &  \\

                     \vernacular{muúndu
                    wanaa[khalachɛ́]}  &   
                     \gloss{‘cut’}  &  \\

                     \vernacular{muúndu
                    wanaa[lakhúúlɪ]}  &   
                     \gloss{‘release’}  &  \\

                     \vernacular{musáatsa
                    wanaa[khalachɛ́]}  &   
                     \gloss{‘cut’}  &  \\

                     \vernacular{musáatsa
                    wanaa[lakhúúlɪ]}  &   
                     \gloss{‘release’}  &  \\
\end{tabular}
%\caption{\nocaption}
    

\subsection{Crastinal Future Negative}\label{sec:sPQRCrastFutNeg}


\begin{tabular}{lll}  
  \multicolumn{2}{l}{
                     \vernacular{(889)
                    Passives} \gloss{‘s/he will not
                    be...’} } &  \\
\multicolumn{2}{l}{ } &  \\

                     \vernacular{naa[khalakwí]
                    {\downstep}tá}  &   
                     \gloss{‘cut’}  &  \\

                     \vernacular{
                    naa[tsuunzuú{\downstep}nwí] {\downstep}tá}  &   
                     \gloss{‘sucked’}  &  \\

                     \vernacular{naa[lakhuú{\downstep}lwí]
                    {\downstep}tá}  &   
                     \gloss{‘released’}  &  \\

                     \vernacular{
                    naa[kalushí{\downstep}tswí] {\downstep}tá}  &   
                     \gloss{‘returned’}  &  \\

                     \vernacular{
                    naa[kalushí{\downstep}tsílwí] {\downstep}tá}  &   
                     \gloss{‘returned
                    for’}  &  \\
\end{tabular}
%\caption{\nocaption}
     
\begin{tabular}{lll}  
  \multicolumn{2}{l}{
                     \vernacular{(890) Yes/No
                    Questions} \gloss{‘will s/he
                    not...?’} } &  \\
\multicolumn{2}{l}{ } &  \\

                     \vernacular{naa[khalachɛ́]
                    {\downstep}tá?}  &   
                     \gloss{‘cut’}  &  \\

                     \vernacular{naa[tsuunzuúnɪ]
                    tá?}  &   
                     \gloss{‘suck’}  &  \\

                     \vernacular{naa[lakhuúlɪ]
                    tá?}  &   
                     \gloss{‘release’}  &  \\

                     \vernacular{naa[kalushítsɪ]
                    tá?}  &   
                     \gloss{‘return’}  &  \\
\end{tabular}
%\caption{\nocaption}
     
\begin{tabular}{lll}  
  \multicolumn{2}{l}{
                     \vernacular{(891)
                    WH-Questions} \gloss{‘who will s/he
                    not...?’} } &  \\
\multicolumn{2}{l}{ } &  \\

                     \vernacular{naa[khalachɛ́] bi
                    tá?}  &   
                     \gloss{‘cut’}  &  \\

                     \vernacular{naa[tsuunzuúnɪ]
                    bi tá?}  &   
                     \gloss{‘suck’}  &  \\

                     \vernacular{naa[lakhuúlɪ] bi
                    tá?}  &   
                     \gloss{‘release’}  &  \\

                     \vernacular{naa[kalushítsɪ]
                    bi tá?}  &   
                     \gloss{‘return’}  &  \\
\end{tabular}
%\caption{\nocaption}
     
\begin{tabular}{lll}  
  \multicolumn{2}{l}{
                     \vernacular{(892) Subject
                    Relatives} \gloss{‘the person
                    \ob muundu\cb  / the man \ob musáatsa\cb  who will
                    not...’} } &  \\
\multicolumn{2}{l}{ } &  \\

                     \vernacular{muúndu
                    wanaa[khalachɛ́] {\downstep}tá}  &   
                     \gloss{‘cut’}  &  \\

                     \vernacular{muúndu
                    wanaa[lakhuúlɪ] tá}  &   
                     \gloss{‘release’}  &  \\

                     \vernacular{musáatsa
                    wanaa[khalachɛ́] {\downstep}tá}  &   
                     \gloss{‘cut’}  &  \\

                     \vernacular{musáatsa
                    wanaa[lakhuúlɪ] tá}  &   
                     \gloss{‘release’}  &  \\
\end{tabular}
%\caption{\nocaption}
    

\subsection{Imperative
              }\label{sec:sPQRImpPl}


\begin{tabular}{lll}  
  \multicolumn{2}{l}{
                     \vernacular{(893)
                    Passives} \gloss{
                    ‘be...!’} } &  \\
\multicolumn{2}{l}{ } &  \\

                     \vernacular{
                    [khalakwí]}  &   
                     \gloss{‘cut’}  &  \\

                     \vernacular{
                    [tsuunzuunwí]}  &   
                     \gloss{‘sucked’}  &  \\

                     \vernacular{
                    [lákhúú{\downstep}lwí]}  &   
                     \gloss{‘released’}  &  \\

                     \vernacular{
                    [kálúshí{\downstep}tswí]}  &   
                     \gloss{‘returned’}  &  \\
\end{tabular}
%\caption{\nocaption}
     
\begin{tabular}{lll}  
  \multicolumn{2}{l}{
                     \vernacular{(894) Yes/No
                    Questions} \gloss{
                    ‘...!?’} } &  \\
\multicolumn{2}{l}{ } &  \\

                     \vernacular{
                    [khalachí]?}  &   
                     \gloss{‘cut’}  &  \\

                     \vernacular{
                    [tsuunzuuní]?}  &   
                     \gloss{‘suck’}  &  \\

                     \vernacular{
                    [lákhúú{\downstep}lí]?}  &   
                     \gloss{‘release’}  &  \\

                     \vernacular{
                    [kálúshí{\downstep}tsí]?}  &   
                     \gloss{‘return’}  &  \\
\end{tabular}
%\caption{\nocaption}
     
\begin{tabular}{lll}  
  \multicolumn{2}{l}{
                     \vernacular{(895)
                    WH-Questions} \gloss{
                    ‘...!?’} } &  \\
\multicolumn{2}{l}{ } &  \\

                     \vernacular{[khalachí]
                    bí?}  &   
                     \gloss{‘cut’}  &  \\

                     \vernacular{[tsuunzuuní]
                    bí?}  &   
                     \gloss{‘suck’}  &  \\

                     \vernacular{[lákhúú{\downstep}lí]
                    bi?}  &   
                     \gloss{‘release’}  &  \\

                     \vernacular{[kálúshí{\downstep}tsí]
                    bi?}  &   
                     \gloss{‘return’}  &  \\
\end{tabular}
%\caption{\nocaption}
    

\subsection{Imperative
              }\label{sec:sPQRImpPlNeg}


\begin{tabular}{lll}  
  \multicolumn{2}{l}{
                     \vernacular{(896)
                    Passives} \gloss{‘do not
                    be...!’} } &  \\
\multicolumn{2}{l}{ } &  \\

                     \vernacular{mukha[khalakwi]
                    tá}  &   
                     \gloss{‘cut’}  &  \\

                     \vernacular{mukha[tsuunzuunwi]
                    tá}  &   
                     \gloss{‘sucked’}  &  \\

                     \vernacular{
                    mukha[lakhú{\downstep}úlwí] {\downstep}tá}  &   
                     \gloss{‘released’}  &  \\

                     \vernacular{
                    mukha[kalúshí{\downstep}tswí] {\downstep}tá}  &   
                     \gloss{‘returned’}  &  \\

                     \vernacular{
                    mukha[kalúshí{\downstep}tsílwí] {\downstep}tá}  &   
                     \gloss{‘returned
                    for’}  &  \\
\end{tabular}
%\caption{\nocaption}
     
\begin{tabular}{lll}  
  \multicolumn{2}{l}{
                     \vernacular{(897) Yes/No
                    Questions} \gloss{‘do
                    not...!?’} } &  \\
\multicolumn{2}{l}{ } &  \\

                     \vernacular{mukha[khalachi]
                    tá?}  &   
                     \gloss{‘cut’}  &  \\

                     \vernacular{mukha[tsuunzuuni]
                    tá?}  &   
                     \gloss{‘suck’}  &  \\

                     \vernacular{mukha[lakhúuli]
                    tá?}  &   
                     \gloss{‘release’}  &  \\

                     \vernacular{
                    mukha[kalúshítsi] tá?}  &   
                     \gloss{‘return’}  &  \\
\end{tabular}
%\caption{\nocaption}
     
\begin{tabular}{lll}  
  \multicolumn{2}{l}{
                     \vernacular{(898)
                    WH-Questions} \gloss{‘do
                    not...who!?’} } &  \\
\multicolumn{2}{l}{ } &  \\

                     \vernacular{mukha[khalachi] bi
                    tá?}  &   
                     \gloss{‘cut’}  &  \\

                     \vernacular{mukha[tsuunzuuni]
                    bi tá?}  &   
                     \gloss{‘suck’}  &  \\

                     \vernacular{mukha[lakhúuli]
                    bi tá?}  &   
                     \gloss{‘release’}  &  \\

                     \vernacular{
                    mukha[kalúshítsi] bi tá?}  &   
                     \gloss{‘return’}  &  \\
\end{tabular}
%\caption{\nocaption}
    

\subsection{Subjunctive}\label{sec:sPQRSubj}


\begin{tabular}{lll}  
  \multicolumn{2}{l}{
                     \vernacular{(899)
                    Passives} \gloss{‘let him/her
                    be...’} } &  \\
\multicolumn{2}{l}{ } &  \\

                     \vernacular{
                    a[khalakwí]}  &   
                     \gloss{‘cut’}  &  \\

                     \vernacular{
                    a[tsuunzuú{\downstep}nwí]}  &   
                     \gloss{‘sucked’}  &  \\

                     \vernacular{
                    a[lakhuú{\downstep}lwí]}  &   
                     \gloss{‘released’}  &  \\

                     \vernacular{
                    a[kalushí{\downstep}tswí]}  &   
                     \gloss{‘returned’}  &  \\

                     \vernacular{
                    a[kalushí{\downstep}tsilwí]}  &   
                     \gloss{‘returned
                    for’}  &  \\
\end{tabular}
%\caption{\nocaption}
     
\begin{tabular}{lll}  
  \multicolumn{2}{l}{
                     \vernacular{(900) Yes/No
                    Questions} \gloss{‘let
                    him/her...?’} } &  \\
\multicolumn{2}{l}{ } &  \\

                     \vernacular{
                    a[khalachɛ́]?}  &   
                     \gloss{‘cut’}  &  \\

                     \vernacular{
                    a[tsuunzuúnɪ]?}  &   
                     \gloss{‘suck’}  &  \\

                     \vernacular{
                    a[lakhuúlɪ]?}  &   
                     \gloss{‘release’}  &  \\

                     \vernacular{
                    a[kalushítsɪ]?}  &   
                     \gloss{‘return’}  &  \\
\end{tabular}
%\caption{\nocaption}
     
\begin{tabular}{lll}  
  \multicolumn{2}{l}{
                     \vernacular{(901)
                    WH-Questions} \gloss{‘let
                    him/her...who?’} } &  \\
\multicolumn{2}{l}{ } &  \\

                     \vernacular{a[khalachɛ́]
                    bi?}  &   
                     \gloss{‘cut’}  &  \\

                     \vernacular{a[tsuunzuúnɪ]
                    bi?}  &   
                     \gloss{‘suck’}  &  \\

                     \vernacular{a[lakhuúlɪ]
                    bi?}  &   
                     \gloss{‘release’}  &  \\

                     \vernacular{a[kalushítsɪ]
                    bi?}  &   
                     \gloss{‘return’}  &  \\
\end{tabular}
%\caption{\nocaption}
    

\subsection{Subjunctive Negative}\label{sec:sPQRSubjNeg}


\begin{tabular}{lll}  
  \multicolumn{2}{l}{
                     \vernacular{(902)
                    Passives} \gloss{‘let him/her not
                    be...’} } &  \\
\multicolumn{2}{l}{ } &  \\

                     \vernacular{a[khalakwí]
                    {\downstep}tá}  &   
                     \gloss{‘cut’}  &  \\

                     \vernacular{a[tsuunzuú{\downstep}nwí]
                    {\downstep}tá}  &   
                     \gloss{‘sucked’}  &  \\

                     \vernacular{a[lakhuú{\downstep}lwí]
                    {\downstep}tá}  &   
                     \gloss{‘released’}  &  \\

                     \vernacular{a[kalushí{\downstep}tswí]
                    {\downstep}tá}  &   
                     \gloss{‘returned’}  &  \\

                     \vernacular{
                    a[kalushí{\downstep}tsílwí] {\downstep}tá}  &   
                     \gloss{‘returned
                    for’}  &  \\
\end{tabular}
%\caption{\nocaption}
     
\begin{tabular}{lll}  
  \multicolumn{2}{l}{
                     \vernacular{(903) Yes/No
                    Questions} \gloss{‘let him/her
                    not...?’} } &  \\
\multicolumn{2}{l}{ } &  \\

                     \vernacular{a[khalachɛ́]
                    {\downstep}tá?}  &   
                     \gloss{‘cut’}  &  \\

                     \vernacular{a[tsuunzuúnɪ]
                    tá?}  &   
                     \gloss{‘suck’}  &  \\

                     \vernacular{a[lakhuúlɪ]
                    tá?}  &   
                     \gloss{‘release’}  &  \\

                     \vernacular{a[kalushítsɪ]
                    tá?}  &   
                     \gloss{‘return’}  &  \\
\end{tabular}
%\caption{\nocaption}
     
\begin{tabular}{lll}  
  \multicolumn{2}{l}{
                     \vernacular{(904)
                    WH-Questions} \gloss{‘let him/her
                    not...who?’} } &  \\
\multicolumn{2}{l}{ } &  \\

                     \vernacular{a[khalachɛ́] bi
                    tá?}  &   
                     \gloss{‘cut’}  &  \\

                     \vernacular{a[tsuunzuúnɪ] bi
                    tá?}  &   
                     \gloss{‘suck’}  &  \\

                     \vernacular{a[lakhuúlɪ] bi
                    tá?}  &   
                     \gloss{‘release’}  &  \\

                     \vernacular{a[kalushítsɪ] bi
                    tá?}  &   
                     \gloss{‘return’}  &  \\
\end{tabular}
%\caption{\nocaption}
     
\begin{tabular}{lll}  
  \multicolumn{2}{l}{
                     \vernacular{(905) Subject
                    Relatives} \gloss{‘the person
                    \ob muundu\cb  / the man \ob musáatsa\cb  who should
                    not...’} } &  \\
\multicolumn{2}{l}{ } &  \\

                     \vernacular{muúndu
                    ukhá[khalaka] tá}  &   
                     \gloss{‘cut’}  &  \\

                     \vernacular{muúndu
                    ukhá[{\downstep}lákhúula] tá}  &   
                     \gloss{‘release’}  &  \\

                     \vernacular{musáatsa
                    ukhá[khalaka] {\downstep}tá}  &   
                     \gloss{‘cut’}  &  \\

                     \vernacular{musáatsa
                    ukhá[{\downstep}lákhúula] tá}  &   
                     \gloss{‘release’}  &  \\
\end{tabular}
%\caption{\nocaption}
    

\subsection{Hesternal Perfective}\label{sec:sPQRHestPerf}


\begin{tabular}{lll}  
  \multicolumn{2}{l}{
                     \vernacular{(906)
                    Passives} \gloss{‘s/he
                    was...’} } &  \\
\multicolumn{2}{l}{ } &  \\

                     \vernacular{
                    ya[khálááchwí]}  &   
                     \gloss{‘cut’}  &  \\

                     \vernacular{
                    ya[tsúúnzúúnwí]}  &   
                     \gloss{‘sucked’}  &  \\

                     \vernacular{
                    ya[lákhúú{\downstep}lwí]}  &   
                     \gloss{‘released’}  &  \\

                     \vernacular{
                    ya[kálú{\downstep}shíítswí]}  &   
                     \gloss{‘returned’}  &  \\
\end{tabular}
%\caption{\nocaption}
     
\begin{tabular}{lll}  
  \multicolumn{2}{l}{
                     \vernacular{(907) Yes/No
                    Questions} \gloss{
                    ‘s/he...?’} } &  \\
\multicolumn{2}{l}{ } &  \\

                     \vernacular{
                    ya[khálááchɛ́]?}  &   
                     \gloss{‘cut’}  &  \\

                     \vernacular{
                    ya[tsúúnzúúní]?}  &   
                     \gloss{‘sucked’}  &  \\

                     \vernacular{
                    ya[lákhú{\downstep}úlí]?}  &   
                     \gloss{‘released’}  &  \\

                     \vernacular{
                    ya[kálú{\downstep}shíítsí]?}  &   
                     \gloss{‘returned’}  &  \\
\end{tabular}
%\caption{\nocaption}
     
\begin{tabular}{lll}  
  \multicolumn{2}{l}{
                     \vernacular{(908)
                    WH-Questions} \gloss{
                    ‘s/he...who?’} } &  \\
\multicolumn{2}{l}{ } &  \\

                     \vernacular{ya[khalaachɛ]
                    bí?}  &   
                     \gloss{‘cut’}  &  \\

                     \vernacular{ya[tsuunzuuni]
                    bí?}  &   
                     \gloss{‘sucked’}  &  \\

                     \vernacular{ya[lakhuuli]
                    bí?}  &   
                     \gloss{‘released’}  &  \\

                     \vernacular{ya[kalushiitsɪ]
                    bí?}  &   
                     \gloss{‘returned’}  &  \\
\end{tabular}
%\caption{\nocaption}
     
\begin{tabular}{lll}  
  \multicolumn{2}{l}{
                     \vernacular{(909) Subject
                    Relatives} \gloss{‘the person
                    \ob muundu\cb  / the man \ob musáatsa\cb 
                    who...’} } &  \\
\multicolumn{2}{l}{ } &  \\

                     \vernacular{muúndu
                    wá[{\downstep}kháláachɛ́]}  &   
                     \gloss{‘cut’}  &  \\

                     \vernacular{muúndu
                    wá[{\downstep}lákhúuli]}  &   
                     \gloss{‘released’}  &  \\

                     \vernacular{musáatsa
                    wá[{\downstep}kháláachɛ́]}  &   
                     \gloss{‘cut’}  &  \\

                     \vernacular{musáatsa
                    wá[{\downstep}lákhúuli]}  &   
                     \gloss{‘released’}  &  \\
\end{tabular}
%\caption{\nocaption}
    

\subsection{Hesternal Perfective Negative}\label{sec:sPQRHestPerfNeg}


\begin{tabular}{lll}  
  \multicolumn{2}{l}{
                     \vernacular{(910)
                    Passives} \gloss{‘s/he was
                    not...’} } &  \\
\multicolumn{2}{l}{ } &  \\

                     \vernacular{ya[khálááchwí]
                    {\downstep}tá}  &   
                     \gloss{‘cut’}  &  \\

                     \vernacular{
                    ya[tsúúnzúúnwí] {\downstep}tá}  &   
                     \gloss{‘sucked’}  &  \\

                     \vernacular{ya[lákhúú{\downstep}lwí]
                    {\downstep}tá}  &   
                     \gloss{‘released’}  &  \\

                     \vernacular{
                    ya[kálú{\downstep}shíítswí] {\downstep}tá}  &   
                     \gloss{‘returned’}  &  \\
\end{tabular}
%\caption{\nocaption}
     
\begin{tabular}{lll}  
  \multicolumn{2}{l}{
                     \vernacular{(911) Yes/No
                    Questions} \gloss{‘s/he did
                    not...?’} } &  \\
\multicolumn{2}{l}{ } &  \\

                     \vernacular{ya[khálááchɛ́]
                    {\downstep}tá?}  &   
                     \gloss{‘cut’}  &  \\

                     \vernacular{
                    ya[tsúúnzúúní] {\downstep}tá?}  &   
                     \gloss{‘suck’}  &  \\

                     \vernacular{ya[lákhú{\downstep}úlí]
                    {\downstep}tá?}  &   
                     \gloss{‘release’}  &  \\

                     \vernacular{
                    ya[kálú{\downstep}shíítsí] {\downstep}tá?}  &   
                     \gloss{‘return’}  &  \\
\end{tabular}
%\caption{\nocaption}
     
\begin{tabular}{lll}  
  \multicolumn{2}{l}{
                     \vernacular{(912)
                    WH-Questions} \gloss{‘s/he did
                    not...who?’} } &  \\
\multicolumn{2}{l}{ } &  \\

                     \vernacular{ya[khalaachɛ́] bi
                    tá?}  &   
                     \gloss{‘cut’}  &  \\

                     \vernacular{ya[tsuunzuuní] bi
                    tá?}  &   
                     \gloss{‘suck’}  &  \\

                     \vernacular{ya[lakhúúli] bi
                    tá?}  &   
                     \gloss{‘release’}  &  \\

                     \vernacular{ya[kalushíítsɪ]
                    bi tá?}  &   
                     \gloss{‘return’}  &  \\
\end{tabular}
%\caption{\nocaption}
     
\begin{tabular}{lll}  
  \multicolumn{2}{l}{
                     \vernacular{(913) Subject
                    Relatives} \gloss{‘the person
                    \ob muundu\cb  / the man \ob musáatsa\cb  who did
                    not...’} } &  \\
\multicolumn{2}{l}{ } &  \\

                     \vernacular{muúndu
                    wá{\downstep}khá[kháláachɛ́]}  &   
                     \gloss{‘cut’}  &  \\

                     \vernacular{muúndu
                    wá{\downstep}khá[lákhú{\downstep}úlí]}  &   
                     \gloss{‘release’}  &  \\

                     \vernacular{musáatsa
                    wá{\downstep}khá[khálááchɛ́]}  &   
                     \gloss{‘cut’}  &  \\

                     \vernacular{musáatsa
                    wá{\downstep}khá[lákhú{\downstep}úlí]}  &   
                     \gloss{‘release’}  &  \\
\end{tabular}
%\caption{\nocaption}
    

\subsection{Perfect (3
              }\label{sec:sPQRPerf3rdSg}


\begin{tabular}{lll}  
  \multicolumn{2}{l}{
                     \vernacular{(914)
                    Passives} \gloss{‘s/he has
                    been...’} } &  \\
\multicolumn{2}{l}{ } &  \\

                     \vernacular{
                    uu[khálaachwi]}  &   
                     \gloss{‘cut’}  &  \\

                     \vernacular{
                    uu[tsúunzuunwi]}  &   
                     \gloss{‘sucked’}  &  \\

                     \vernacular{
                    uu[lakhuulwi]}  &   
                     \gloss{‘released’}  &  \\

                     \vernacular{
                    uu[kalushiitswi]}  &   
                     \gloss{‘returned’}  &  \\
\end{tabular}
%\caption{\nocaption}
     
\begin{tabular}{lll}  
  \multicolumn{2}{l}{
                     \vernacular{(915) Yes/No
                    Questions} \gloss{‘has
                    s/he...?’} } &  \\
\multicolumn{2}{l}{ } &  \\

                     \vernacular{
                    uu[khálaachɛ]?}  &   
                     \gloss{‘cut’}  &  \\

                     \vernacular{
                    uu[tsúunzuuni]?}  &   
                     \gloss{‘sucked’}  &  \\

                     \vernacular{
                    uu[lakhuuli]?}  &   
                     \gloss{‘released’}  &  \\

                     \vernacular{
                    uu[kalushiitsi]?}  &   
                     \gloss{‘returned’}  &  \\
\end{tabular}
%\caption{\nocaption}
     
\begin{tabular}{lll}  
  \multicolumn{2}{l}{
                     \vernacular{(916)
                    WH-Questions} \gloss{‘who has
                    s/he...?’} } &  \\
\multicolumn{2}{l}{ } &  \\

                     \vernacular{aa[khá{\downstep}lááchɛ́]
                    bí?}  &   
                     \gloss{‘cut’}  &  \\

                     \vernacular{
                    aa[tsú{\downstep}únzúúní] bí?}  &   
                     \gloss{‘sucked’}  &  \\

                     \vernacular{aa[lakhúuli]
                    bi?}  &   
                     \gloss{‘released’}  &  \\

                     \vernacular{aa[kalúshíitsɪ]
                    bi?}  &   
                     \gloss{‘returned’}  &  \\
\end{tabular}
%\caption{\nocaption}
     
\begin{tabular}{lll}  
  \multicolumn{2}{l}{
                     \vernacular{(917) Subject
                    Relatives} \gloss{‘the person
                    \ob muundu\cb  / the man \ob musáatsa\cb  who
                    has...’} } &  \\
\multicolumn{2}{l}{ } &  \\

                     \vernacular{muúndu
                    uu[khá{\downstep}lááchɛ́]}  &   
                     \gloss{‘cut’}  &  \\

                     \vernacular{muúndu
                    uu[lá{\downstep}khúúlí]}  &   
                     \gloss{‘released’}  &  \\

                     \vernacular{musáatsa
                    uu[khá{\downstep}lááchɛ́]}  &   
                     \gloss{‘cut’}  &  \\

                     \vernacular{musáatsa
                    uu[lá{\downstep}khúúlí]}  &   
                     \gloss{‘released’}  &  \\
\end{tabular}
%\caption{\nocaption}
    

\subsection{Perfect (3
              }\label{sec:sPQRPerf3rdSgNeg}


\begin{tabular}{lll}  
  \multicolumn{2}{l}{
                     \vernacular{(918)
                    Passives} \gloss{‘s/he has not
                    been...’} } &  \\
\multicolumn{2}{l}{ } &  \\

                     \vernacular{
                    aa[khá{\downstep}lááchwí] tá}  &   
                     \gloss{‘cut’}  &  \\

                     \vernacular{
                    aa[tsú{\downstep}únzúúnwí] tá}  &   
                     \gloss{‘sucked’}  &  \\

                     \vernacular{aa[lákhúúlwí]
                    tá}  &   
                     \gloss{‘released’}  &  \\

                     \vernacular{
                    aa[kálúshíítswí] tá}  &   
                     \gloss{‘returned’}  &  \\
\end{tabular}
%\caption{\nocaption}
     
\begin{tabular}{lll}  
  \multicolumn{2}{l}{
                     \vernacular{(919) Yes/No
                    Questions} \gloss{‘has s/he
                    not...?’} } &  \\
\multicolumn{2}{l}{ } &  \\

                     \vernacular{aa[khá{\downstep}lááchɛ́]
                    tá?}  &   
                     \gloss{‘cut’}  &  \\

                     \vernacular{
                    aa[tsú{\downstep}únzúúní] tá?}  &   
                     \gloss{‘sucked’}  &  \\

                     \vernacular{aa[lákhúúlí]
                    tá?}  &   
                     \gloss{‘released’}  &  \\

                     \vernacular{
                    aa[kálúshíítsí] tá?}  &   
                     \gloss{‘returned’}  &  \\
\end{tabular}
%\caption{\nocaption}
     
\begin{tabular}{lll}  
  \multicolumn{2}{l}{
                     \vernacular{(920)
                    WH-Questions} \gloss{‘who has s/he
                    not...?’} } &  \\
\multicolumn{2}{l}{ } &  \\

                     \vernacular{aa[khá{\downstep}lááchɛ́]
                    bí {\downstep}tá?}  &   
                     \gloss{‘cut’}  &  \\

                     \vernacular{
                    aa[tsú{\downstep}únzúúní] bí {\downstep}tá?}  &   
                     \gloss{‘sucked’}  &  \\

                     \vernacular{aa[lakhúuli] bi
                    tá?}  &   
                     \gloss{‘released’}  &  \\

                     \vernacular{aa[kalúshíitsɪ]
                    bi tá?}  &   
                     \gloss{‘returned’}  &  \\
\end{tabular}
%\caption{\nocaption}
     
\begin{tabular}{lll}  
  \multicolumn{2}{l}{
                     \vernacular{(921) Subject
                    Relatives} \gloss{‘the person
                    \ob muundu\cb  / the man \ob musáatsa\cb  who has
                    not...’} } &  \\
\multicolumn{2}{l}{ } &  \\

                     \vernacular{muúndu
                    ukhá[{\downstep}khálááchɛ́]}  &   
                     \gloss{‘cut’}  &  \\

                     \vernacular{muúndu
                    ukhá[{\downstep}tsúúnzúúní]}  &   
                     \gloss{‘sucked’}  &  \\

                     \vernacular{muúndu
                    ukhá[{\downstep}lákhúuli]}  &   
                     \gloss{‘released’}  &  \\

                     \vernacular{musáatsa
                    ukhá[{\downstep}khálááchɛ́]}  &   
                     \gloss{‘cut’}  &  \\

                     \vernacular{musáatsa
                    ukhá[{\downstep}tsúúnzúúní]}  &   
                     \gloss{‘sucked’}  &  \\

                     \vernacular{musáatsa
                    ukhá[{\downstep}lákhúuli]}  &   
                     \gloss{‘released’}  &  \\
\end{tabular}
%\caption{\nocaption}
    

\subsection{Perfect (2
              }\label{sec:sPQRPerf2ndSg}


\begin{tabular}{lll}  
  \multicolumn{2}{l}{
                     \vernacular{(922)
                    Passives} \gloss{‘you have
                    been...’} } &  \\
\multicolumn{2}{l}{ } &  \\

                     \vernacular{
                    uu[khalaachwi]}  &   
                     \gloss{‘cut’}  &  \\

                     \vernacular{
                    uu[tsuunzuunwi]}  &   
                     \gloss{‘sucked’}  &  \\

                     \vernacular{
                    uu[lakhuulwi]}  &   
                     \gloss{‘released’}  &  \\

                     \vernacular{
                    uu[kalushiitswi]}  &   
                     \gloss{‘returned’}  &  \\
\end{tabular}
%\caption{\nocaption}
     
\begin{tabular}{lll}  
  \multicolumn{2}{l}{
                     \vernacular{(923) Yes/No
                    Questions} \gloss{‘have
                    you...?’} } &  \\
\multicolumn{2}{l}{ } &  \\

                     \vernacular{
                    uu[khalaachɛ]?}  &   
                     \gloss{‘cut’}  &  \\

                     \vernacular{
                    uu[tsuunzuuni]?}  &   
                     \gloss{‘sucked’}  &  \\

                     \vernacular{
                    uu[lakhuuli]?}  &   
                     \gloss{‘released’}  &  \\

                     \vernacular{
                    uu[kalushiitsi]?}  &   
                     \gloss{‘returned’}  &  \\
\end{tabular}
%\caption{\nocaption}
     
\begin{tabular}{lll}  
  \multicolumn{2}{l}{
                     \vernacular{(924)
                    WH-Questions} \gloss{‘who have
                    you...?’} } &  \\
\multicolumn{2}{l}{ } &  \\

                     \vernacular{uu[khá{\downstep}lááchɛ́]
                    bí?}  &   
                     \gloss{‘cut’}  &  \\

                     \vernacular{
                    uu[tsú{\downstep}únzúúní] bí?}  &   
                     \gloss{‘sucked’}  &  \\

                     \vernacular{uu[lakhúuli]
                    bi?}  &   
                     \gloss{‘released’}  &  \\

                     \vernacular{uu[kalúshíitsɪ]
                    bi?}  &   
                     \gloss{‘returned’}  &  \\
\end{tabular}
%\caption{\nocaption}
    

\subsection{Perfect (2
              }\label{sec:sPQRPerf2ndSgNeg}


\begin{tabular}{lll}  
  \multicolumn{2}{l}{
                     \vernacular{(925)
                    Passives} \gloss{‘you have not
                    been...’} } &  \\
\multicolumn{2}{l}{ } &  \\

                     \vernacular{uu[khalaachwi]
                    tá}  &   
                     \gloss{‘cut’}  &  \\

                     \vernacular{uu[tsuunzuunwi]
                    tá}  &   
                     \gloss{‘sucked’}  &  \\

                     \vernacular{uu[lakhuulwi]
                    tá}  &   
                     \gloss{‘released’}  &  \\

                     \vernacular{uu[kalushiitswi]
                    tá}  &   
                     \gloss{‘returned’}  &  \\
\end{tabular}
%\caption{\nocaption}
     
\begin{tabular}{lll}  
  \multicolumn{2}{l}{
                     \vernacular{(926) Yes/No
                    Questions} \gloss{‘have you
                    not...?’} } &  \\
\multicolumn{2}{l}{ } &  \\

                     \vernacular{uu[khalaachɛ]
                    tá?}  &   
                     \gloss{‘cut’}  &  \\

                     \vernacular{uu[tsuunzuuni]
                    tá?}  &   
                     \gloss{‘sucked’}  &  \\

                     \vernacular{uu[lakhúuli]
                    tá?}  &   
                     \gloss{‘released’}  &  \\

                     \vernacular{uu[kalúshíitsi]
                    tá?}  &   
                     \gloss{‘returned’}  &  \\
\end{tabular}
%\caption{\nocaption}
     
\begin{tabular}{lll}  
  \multicolumn{2}{l}{
                     \vernacular{(927)
                    WH-Questions} \gloss{‘who have you
                    not...?’} } &  \\
\multicolumn{2}{l}{ } &  \\

                     \vernacular{uu[khalaachɛ] bí
                    {\downstep}tá?}  &   
                     \gloss{‘cut’}  &  \\

                     \vernacular{uu[tsuunzuuni] bí
                    {\downstep}tá?}  &   
                     \gloss{‘sucked’}  &  \\

                     \vernacular{uu[lakhúuli] bi
                    tá?}  &   
                     \gloss{‘released’}  &  \\

                     \vernacular{uu[kalúshíitsɪ]
                    bi tá?}  &   
                     \gloss{‘returned’}  &  \\
\end{tabular}
%\caption{\nocaption}
    

\subsection{Hodiernal Perfective}\label{sec:sPQRHodPerf}


\begin{tabular}{lll}  
  \multicolumn{2}{l}{
                     \vernacular{(928)
                    Passives} \gloss{‘s/he
                    was...’} } &  \\
\multicolumn{2}{l}{ } &  \\

                     \vernacular{
                    a[khalaachwí]}  &   
                     \gloss{‘cut’}  &  \\

                     \vernacular{
                    a[tsuunzuunwí]}  &   
                     \gloss{‘sucked’}  &  \\

                     \vernacular{
                    a[lakhú{\downstep}úlwí]}  &   
                     \gloss{‘released’}  &  \\

                     \vernacular{
                    a[kalúshí{\downstep}ítswí]}  &   
                     \gloss{‘returned’}  &  \\
\end{tabular}
%\caption{\nocaption}
     
\begin{tabular}{lll}  
  \multicolumn{2}{l}{
                     \vernacular{(929) Yes/No
                    Questions} \gloss{‘did
                    s/he...?’} } &  \\
\multicolumn{2}{l}{ } &  \\

                     \vernacular{
                    a[khalaachɛ]?}  &   
                     \gloss{‘cut’}  &  \\

                     \vernacular{
                    a[tsuunzuuni]?}  &   
                     \gloss{‘suck’}  &  \\

                     \vernacular{
                    a[lakhúuli]?}  &   
                     \gloss{‘release’}  &  \\

                     \vernacular{
                    a[kalúshíítsi]?}  &   
                     \gloss{‘return’}  &  \\
\end{tabular}
%\caption{\nocaption}
     
\begin{tabular}{lll}  
  \multicolumn{2}{l}{
                     \vernacular{(930)
                    WH-Questions} \gloss{‘who did
                    s/he...?’} } &  \\
\multicolumn{2}{l}{ } &  \\

                     \vernacular{a[khalaachɛ]
                    bí?}  &   
                     \gloss{‘cut’}  &  \\

                     \vernacular{a[tsuunzuuni]
                    bí?}  &   
                     \gloss{‘suck’}  &  \\

                     \vernacular{a[lakhúuli]
                    bí?}  &   
                     \gloss{‘release’}  &  \\

                     \vernacular{a[kalúshíítsɪ]
                    bí?}  &   
                     \gloss{‘return’}  &  \\
\end{tabular}
%\caption{\nocaption}
     
\begin{tabular}{lll}  
  \multicolumn{2}{l}{
                     \vernacular{(931) Subject
                    Relatives} \gloss{‘the person
                    \ob muundu\cb  / the man \ob musáatsa\cb 
                    who...’} } &  \\
\multicolumn{2}{l}{ } &  \\

                     \vernacular{muúndu
                    u[khá{\downstep}lááchɛ́]}  &   
                     \gloss{‘cut’}  &  \\

                     \vernacular{muúndu
                    u[lá{\downstep}khúúlí]}  &   
                     \gloss{‘released’}  &  \\

                     \vernacular{musáatsa
                    u[khá{\downstep}lááchɛ́]}  &   
                     \gloss{‘cut’}  &  \\

                     \vernacular{musáatsa
                    u[lá{\downstep}khúúlí]}  &   
                     \gloss{‘released’}  &  \\
\end{tabular}
%\caption{\nocaption}
    

\subsection{Hodiernal Perfective Negative}\label{sec:sPQRHodPerfNeg}


\begin{tabular}{lll}  
  \multicolumn{2}{l}{
                     \vernacular{(932)
                    Passives} \gloss{‘s/he was
                    not...’} } &  \\
\multicolumn{2}{l}{ } &  \\

                     \vernacular{a[khalaachwí]
                    {\downstep}tá}  &   
                     \gloss{‘cut’}  &  \\

                     \vernacular{a[tsuunzuunwí]
                    {\downstep}tá}  &   
                     \gloss{‘sucked’}  &  \\

                     \vernacular{a[lakhú{\downstep}úlwí]
                    {\downstep}tá}  &   
                     \gloss{‘released’}  &  \\

                     \vernacular{
                    a[kalúshí{\downstep}ítswí] {\downstep}tá}  &   
                     \gloss{‘returned’}  &  \\
\end{tabular}
%\caption{\nocaption}
     
\begin{tabular}{lll}  
  \multicolumn{2}{l}{
                     \vernacular{(933) Yes/No
                    Questions} \gloss{‘did s/he
                    not...?’} } &  \\
\multicolumn{2}{l}{ } &  \\

                     \vernacular{a[khalaachɛ]
                    tá?}  &   
                     \gloss{‘cut’}  &  \\

                     \vernacular{a[tsuunzuuni]
                    tá?}  &   
                     \gloss{‘suck’}  &  \\

                     \vernacular{a[lakhúuli]
                    tá?}  &   
                     \gloss{‘release’}  &  \\

                     \vernacular{a[kalúshíítsi]
                    tá?}  &   
                     \gloss{‘return’}  &  \\
\end{tabular}
%\caption{\nocaption}
     
\begin{tabular}{lll}  
  \multicolumn{2}{l}{
                     \vernacular{(934)
                    WH-Questions} \gloss{‘who did s/he
                    not...?’} } &  \\
\multicolumn{2}{l}{ } &  \\

                     \vernacular{a[khalaachɛ] bí
                    {\downstep}tá?}  &   
                     \gloss{‘cut’}  &  \\

                     \vernacular{a[tsuunzuuni] bí
                    {\downstep}tá?}  &   
                     \gloss{‘suck’}  &  \\

                     \vernacular{a[lakhúuli] bí
                    {\downstep}tá?}  &   
                     \gloss{‘release’}  &  \\

                     \vernacular{a[kalúshíítsɪ]
                    bí {\downstep}tá?}  &   
                     \gloss{‘return’}  &  \\
\end{tabular}
%\caption{\nocaption}
     
\begin{tabular}{lll}  
  \multicolumn{2}{l}{
                     \vernacular{(935) Subject
                    Relatives} \gloss{‘the person
                    \ob muundu\cb  / the man \ob musáatsa\cb  who did
                    not...’} } &  \\
\multicolumn{2}{l}{ } &  \\

                     \vernacular{muúndu
                    ukhá[{\downstep}khálááchɛ́]}  &   
                     \gloss{‘cut’}  &  \\

                     \vernacular{muúndu
                    ukhá[{\downstep}lákhúuli]}  &   
                     \gloss{‘release’}  &  \\

                     \vernacular{musáatsa
                    ukhá[{\downstep}khálááchɛ́]}  &   
                     \gloss{‘cut’}  &  \\

                     \vernacular{musáatsa
                    ukhá[{\downstep}lákhúuli]}  &   
                     \gloss{‘release’}  &  \\
\end{tabular}
%\caption{\nocaption}
    

\subsection{Conditional}\label{sec:sPQRCond}


\begin{tabular}{lll}  
  \multicolumn{2}{l}{
                     \vernacular{(936)
                    Passives} \gloss{‘if s/he
                    is...’} } &  \\
\multicolumn{2}{l}{ } &  \\

                     \vernacular{
                    naá[{\downstep}khálákwá]}  &   
                     \gloss{‘cut’}  &  \\

                     \vernacular{
                    naá[{\downstep}tsúúnzúúnwá]}  &   
                     \gloss{‘sucked’}  &  \\

                     \vernacular{
                    naá[{\downstep}lákhú{\downstep}úlwá]}  &   
                     \gloss{‘released’}  &  \\

                     \vernacular{
                    naá[{\downstep}kálú{\downstep}shítswá]}  &   
                     \gloss{‘returned’}  &  \\
\end{tabular}
%\caption{\nocaption}
     
\begin{tabular}{lll}  
  \multicolumn{2}{l}{
                     \vernacular{(937) Yes/No
                    Questions} \gloss{‘if
                    s/he...?’} } &  \\
\multicolumn{2}{l}{ } &  \\

                     \vernacular{
                    naá[{\downstep}kháláká]?}  &   
                     \gloss{‘cuts’}  &  \\

                     \vernacular{
                    naá[{\downstep}tsúúnzúúná]?}  &   
                     \gloss{‘sucks’}  &  \\

                     \vernacular{
                    naá[{\downstep}lákhúula]?}  &   
                     \gloss{‘releases’}  &  \\

                     \vernacular{
                    naá[{\downstep}kálúshitsa]?}  &   
                     \gloss{‘returns’}  &  \\
\end{tabular}
%\caption{\nocaption}
     
\begin{tabular}{lll}  
  \multicolumn{2}{l}{
                     \vernacular{(938)
                    WH-Questions} \gloss{‘if
                    s/he...who?’} } &  \\
\multicolumn{2}{l}{ } &  \\

                     \vernacular{naá[{\downstep}kháláká]
                    bi?}  &   
                     \gloss{‘cuts’}  &  \\

                     \vernacular{
                    naá[{\downstep}tsúúnzúúná] bi?}  &   
                     \gloss{‘sucks’}  &  \\

                     \vernacular{naá[{\downstep}lákhúula]
                    bi?}  &   
                     \gloss{‘released’}  &  \\

                     \vernacular{
                    naá[{\downstep}kálúshítsa] bi?}  &   
                     \gloss{‘returned’}  &  \\
\end{tabular}
%\caption{\nocaption}
    

\subsection{Conditional Negative}\label{sec:sPQRCondNeg}


\begin{tabular}{lll}  
  \multicolumn{2}{l}{
                     \vernacular{(939)
                    Passives} \gloss{‘if s/he is
                    not...!’} } &  \\
\multicolumn{2}{l}{ } &  \\

                     \vernacular{
                    naákha[khalakwa]}  &   
                     \gloss{‘cut’}  &  \\

                     \vernacular{
                    naákha[tsuunzuunwa]}  &   
                     \gloss{‘sucked’}  &  \\

                     \vernacular{
                    naákha[lakhú{\downstep}úlwá]}  &   
                     \gloss{‘released’}  &  \\

                     \vernacular{
                    naákha[kalú{\downstep}shítswá]}  &   
                     \gloss{‘returned’}  &  \\
\end{tabular}
%\caption{\nocaption}
     
\begin{tabular}{lll}  
  \multicolumn{2}{l}{
                     \vernacular{(940) Yes/No
                    Questions} \gloss{‘if s/he does
                    not...!?’} } &  \\
\multicolumn{2}{l}{ } &  \\

                     \vernacular{
                    naákha[khalaka]?}  &   
                     \gloss{‘cut’}  &  \\

                     \vernacular{
                    naákha[tsuunzuuna]?}  &   
                     \gloss{‘suck’}  &  \\

                     \vernacular{
                    naákha[lakhúula]?}  &   
                     \gloss{‘release’}  &  \\

                     \vernacular{
                    naákha[kalúshitsa]?}  &   
                     \gloss{‘return’}  &  \\
\end{tabular}
%\caption{\nocaption}
     
\begin{tabular}{lll}  
  \multicolumn{2}{l}{
                     \vernacular{(941)
                    WH-Questions} \gloss{‘if s/he does
                    not...who!?’} } &  \\
\multicolumn{2}{l}{ } &  \\

                     \vernacular{naákha[khalaka]
                    bi?}  &   
                     \gloss{‘cut’}  &  \\

                     \vernacular{
                    naákha[tsuunzuuna] bi?}  &   
                     \gloss{‘suck’}  &  \\

                     \vernacular{naákha[lakhúula]
                    bi?}  &   
                     \gloss{‘release’}  &  \\

                     \vernacular{
                    naákha[kalúshítsa] bi?}  &   
                     \gloss{‘return’}  &  \\
\end{tabular}
%\caption{\nocaption}
    

\subsection{Persistive}\label{sec:sPQRPers}


\begin{tabular}{lll}  
  \multicolumn{2}{l}{
                     \vernacular{(942)
                    Passives} \gloss{‘s/he is still
                    being...’} } &  \\
\multicolumn{2}{l}{ } &  \\

                     \vernacular{
                    ashi[khalakwáángua]}  &   
                     \gloss{‘cutting’}  &  \\

                     \vernacular{
                    ashi[tsuunzuunwáángua]}  &   
                     \gloss{‘sucking’}  &  \\

                     \vernacular{
                    ashi[lakhú{\downstep}úlwáángúa]}  &   
                     \gloss{‘releasing’}  &  \\

                     \vernacular{
                    ashi[kalúshí{\downstep}tswáángúa]}  &   
                     \gloss{‘returning’}  &  \\
\end{tabular}
%\caption{\nocaption}
     
\begin{tabular}{lll}  
  \multicolumn{2}{l}{
                     \vernacular{(943) Yes/No
                    Questions} \gloss{‘is s/he
                    still...?’} } &  \\
\multicolumn{2}{l}{ } &  \\

                     \vernacular{
                    ashi[khalakáánga]?}  &   
                     \gloss{‘cutting’}  &  \\

                     \vernacular{
                    ashi[tsuunzuunáánga]?}  &   
                     \gloss{‘sucking’}  &  \\

                     \vernacular{
                    ashi[lakhúulaanga]?}  &   
                     \gloss{‘releasing’}  &  \\

                     \vernacular{
                    ashi[kalúshítsaanga]?}  &   
                     \gloss{‘returning’}  &  \\
\end{tabular}
%\caption{\nocaption}
     
\begin{tabular}{lll}  
  \multicolumn{2}{l}{
                     \vernacular{(944)
                    WH-Questions} \gloss{‘who is s/he
                    still...?’} } &  \\
\multicolumn{2}{l}{ } &  \\

                     \vernacular{
                    ashi[khalakáá{\downstep}ngá] bi?}  &   
                     \gloss{‘cutting’}  &  \\

                     \vernacular{
                    ashi[tsuunzuunáá{\downstep}ngá] bi?}  &   
                     \gloss{‘sucking’}  &  \\

                     \vernacular{
                    ashi[lakhú{\downstep}úláángá] bi?}  &   
                     \gloss{‘releasing’}  &  \\

                     \vernacular{
                    ashi[kalúshí{\downstep}tsáángá] bi?}  &   
                     \gloss{‘returning’}  &  \\
\end{tabular}
%\caption{\nocaption}
     
\begin{tabular}{lll}  
  \multicolumn{2}{l}{
                     \vernacular{(945) Subject
                    Relatives} \gloss{‘the person
                    \ob muundu\cb  / the man \ob musáatsa\cb  who is
                    still...’} } &  \\
\multicolumn{2}{l}{ } &  \\

                     \vernacular{muúndu
                    ushí[{\downstep}khálákáánga]}  &   
                     \gloss{‘cutting’}  &  \\

                     \vernacular{muúndu
                    ushí[{\downstep}lákhúulaanga]}  &   
                     \gloss{‘releasing’}  &  \\

                     \vernacular{musáatsa
                    ushí[{\downstep}khálákáánga]}  &   
                     \gloss{‘cutting’}  &  \\

                     \vernacular{musáatsa
                    ushí[{\downstep}lákhúulaanga]}  &   
                     \gloss{‘releasing’}  &  \\
\end{tabular}
%\caption{\nocaption}
    

\subsection{Persistive Negative}\label{sec:sPQRPersNeg}


\begin{tabular}{lll}  
  \multicolumn{2}{l}{
                     \vernacular{(946)
                    Passives} \gloss{‘s/he is not
                    still being...’} } &  \\
\multicolumn{2}{l}{ } &  \\

                     \vernacular{
                    ashi[khalakwáá{\downstep}ngúá] {\downstep}tá}  &   
                     \gloss{‘cutting’}  &  \\

                     \vernacular{
                    ashi[tsuunzuunwáá{\downstep}ngúá] {\downstep}tá}  &   
                     \gloss{‘sucking’}  &  \\

                     \vernacular{
                    ashi[lakhúulwaangua] tá}  &   
                     \gloss{‘releasing’}  &  \\

                     \vernacular{
                    ashi[kalúshítswaangua] tá}  &   
                     \gloss{‘returning’}  &  \\
\end{tabular}
%\caption{\nocaption}
     
\begin{tabular}{lll}  
  \multicolumn{2}{l}{
                     \vernacular{(947) Yes/No
                    Questions} \gloss{‘is s/he not
                    still...?’} } &  \\
\multicolumn{2}{l}{ } &  \\

                     \vernacular{
                    ashi[khalakáá(nga)] tá?}  &   
                     \gloss{‘cutting’}  &  \\

                     \vernacular{
                    ashi[tsuunzuunáá(nga)] tá?}  &   
                     \gloss{‘sucking’}  &  \\

                     \vernacular{
                    ashi[lakhúulaa(nga)] tá?}  &   
                     \gloss{‘releasing’}  &  \\

                     \vernacular{
                    ashi[kalúshítsaa(nga)] tá?}  &   
                     \gloss{‘returning’}  &  \\
\end{tabular}
%\caption{\nocaption}
     
\begin{tabular}{lll}  
  \multicolumn{2}{l}{
                     \vernacular{(948)
                    WH-Questions} \gloss{‘who is s/he not
                    still...?’} } &  \\
\multicolumn{2}{l}{ } &  \\

                     \vernacular{ashi[khalakáá]
                    bi tá?}  &   
                     \gloss{‘cutting’}  &  \\

                     \vernacular{
                    ashi[tsuunzuunáá] bi tá?}  &   
                     \gloss{‘sucking’}  &  \\

                     \vernacular{ashi[lakhuúlaa]
                    bi tá?}  &   
                     \gloss{‘releasing’}  &  \\

                     \vernacular{
                    ashi[kalúshítsaa] bi tá?}  &   
                     \gloss{‘returning’}  &  \\
\end{tabular}
%\caption{\nocaption}
     
\begin{tabular}{lll}  
  \multicolumn{2}{l}{
                     \vernacular{(949) Subject
                    Relatives} \gloss{‘the person
                    \ob muundu\cb  / the man \ob musáatsa\cb  who is not
                    still...’} } &  \\
\multicolumn{2}{l}{ } &  \\

                     \vernacular{muúndu
                    ushí[{\downstep}khálákáá] {\downstep}tá}  &   
                     \gloss{‘cutting’}  &  \\

                     \vernacular{muúndu
                    ushí[{\downstep}lákhúulaa] tá}  &   
                     \gloss{‘releasing’}  &  \\

                     \vernacular{musáatsa
                    ushí[{\downstep}khálákáá] {\downstep}tá}  &   
                     \gloss{‘cutting’}  &  \\

                     \vernacular{musáatsa
                    ushí[{\downstep}lákhúulaa] tá}  &   
                     \gloss{‘releasing’}  &  \\
\end{tabular}
%\caption{\nocaption}
    

\subsection{Habitual}\label{sec:sPQRHabit}


\begin{tabular}{lll}  
  \multicolumn{2}{l}{
                     \vernacular{(950)
                    Passives} \gloss{‘s/he is
                    ever/always being...’} } &  \\
\multicolumn{2}{l}{ } &  \\

                     \vernacular{
                    yaa[khá{\downstep}lákwá]}  &   
                     \gloss{‘cut’}  &  \\

                     \vernacular{
                    yaá[{\downstep}tsúúnzúúnwá]}  &   
                     \gloss{‘sucked’}  &  \\

                     \vernacular{
                    yaá[{\downstep}lákhúúlwá]}  &   
                     \gloss{‘released’}  &  \\

                     \vernacular{
                    yaá[{\downstep}kálúshítswá]}  &   
                     \gloss{‘returned’}  &  \\
\end{tabular}
%\caption{\nocaption}
     
\begin{tabular}{lll}  
  \multicolumn{2}{l}{
                     \vernacular{(951) Yes/No
                    Questions} \gloss{‘is s/he
                    ever/always...?’} } &  \\
\multicolumn{2}{l}{ } &  \\

                     \vernacular{
                    yaá[{\downstep}kháláká]?}  &   
                     \gloss{‘cutting’}  &  \\

                     \vernacular{
                    yaá[{\downstep}tsúúnzúúná]?}  &   
                     \gloss{‘sucking’}  &  \\

                     \vernacular{
                    yaá[{\downstep}lákhúúlá]?}  &   
                     \gloss{‘releasing’}  &  \\

                     \vernacular{
                    yaá[{\downstep}kálúshítsá]?}  &   
                     \gloss{‘returning’}  &  \\
\end{tabular}
%\caption{\nocaption}
     
\begin{tabular}{lll}  
  \multicolumn{2}{l}{
                     \vernacular{(952)
                    WH-Questions} \gloss{‘who is s/he
                    ever/always...?’} } &  \\
\multicolumn{2}{l}{ } &  \\

                     \vernacular{yaá[{\downstep}kháláká]
                    bí?}  &   
                     \gloss{‘cutting’}  &  \\

                     \vernacular{
                    yaá[{\downstep}tsúúnzúúná] bí?}  &   
                     \gloss{‘sucking’}  &  \\

                     \vernacular{
                    yaá[{\downstep}lákhúúlá] bí?}  &   
                     \gloss{‘releasing’}  &  \\

                     \vernacular{
                    yaá[{\downstep}kálúshítsá] bí?}  &   
                     \gloss{‘returning’}  &  \\
\end{tabular}
%\caption{\nocaption}
     
\begin{tabular}{lll}  
  \multicolumn{2}{l}{
                     \vernacular{(́́953)
                    Subject Relatives} \gloss{‘the person
                    \ob muundu\cb  / the man \ob musáatsa\cb  who is
                    ever/always...’} } &  \\
\multicolumn{2}{l}{ } &  \\

                     \vernacular{muúndu
                    waa[khálaka]}  &   
                     \gloss{‘cutting’}  &  \\

                     \vernacular{muúndu
                    waa[lákhuula]}  &   
                     \gloss{‘releasing’}  &  \\

                     \vernacular{musáatsa
                    waa[khálaka]}  &   
                     \gloss{‘cutting’}  &  \\

                     \vernacular{musáatsa
                    waa[lákhuula]}  &   
                     \gloss{‘releasing’}  &  \\
\end{tabular}
%\caption{\nocaption}
    

\subsection{Habitual Negative}\label{sec:sPQRHabitNeg}


\begin{tabular}{lll}  
  \multicolumn{2}{l}{
                     \vernacular{(954)
                    Passives} \gloss{‘s/he is
                    ever/always being...’} } &  \\
\multicolumn{2}{l}{ } &  \\

                     \vernacular{yaá[{\downstep}khálákwá]
                    tá}  &   
                     \gloss{‘cut’}  &  \\

                     \vernacular{
                    yaá[{\downstep}tsúúnzúúnwá] tá}  &   
                     \gloss{‘sucked’}  &  \\

                     \vernacular{
                    yaá[{\downstep}lákhúúlwá] tá}  &   
                     \gloss{‘released’}  &  \\

                     \vernacular{
                    yaá[{\downstep}kálúshítswá] tá}  &   
                     \gloss{‘returned’}  &  \\
\end{tabular}
%\caption{\nocaption}
     
\begin{tabular}{lll}  
  \multicolumn{2}{l}{
                     \vernacular{(955) Yes/No
                    Questions} \gloss{‘is s/he
                    ever/always...?’} } &  \\
\multicolumn{2}{l}{ } &  \\

                     \vernacular{yaá[{\downstep}kháláká]
                    tá?}  &   
                     \gloss{‘cutting’}  &  \\

                     \vernacular{
                    yaá[{\downstep}tsúúnzúúná] tá?}  &   
                     \gloss{‘sucking’}  &  \\

                     \vernacular{
                    yaá[{\downstep}lákhúúlá] tá?}  &   
                     \gloss{‘releasing’}  &  \\

                     \vernacular{
                    yaá[{\downstep}kálúshítsá] tá?}  &   
                     \gloss{‘returning’}  &  \\
\end{tabular}
%\caption{\nocaption}
     
\begin{tabular}{lll}  
  \multicolumn{2}{l}{
                     \vernacular{(956)
                    WH-Questions} \gloss{‘who is s/he
                    ever/always...?’} } &  \\
\multicolumn{2}{l}{ } &  \\

                     \vernacular{yaá[khalaká]
                    {\downstep}bí tá?}  &   
                     \gloss{‘cutting’}  &  \\

                     \vernacular{yaá[tsuunzuuná]
                    bi tá?}  &   
                     \gloss{‘sucking’}  &  \\

                     \vernacular{yaá[{\downstep}lákhúúla]
                    bi tá?}  &   
                     \gloss{‘releasing’}  &  \\

                     \vernacular{
                    yaá[{\downstep}kálúshítsa] bi tá?}  &   
                     \gloss{‘returning’}  &  \\
\end{tabular}
%\caption{\nocaption}
     
\begin{tabular}{lll}  
  \multicolumn{2}{l}{
                     \vernacular{(́́957)
                    Subject Relatives} \gloss{‘the person
                    \ob muundu\cb  / the man \ob musáatsa\cb  who is
                    ever/always...’} } &  \\
\multicolumn{2}{l}{ } &  \\

                     \vernacular{muúndu
                    waá[kháláká] tá}  &   
                     \gloss{‘cutting’}  &  \\

                     \vernacular{muúndu
                    waa[lá{\downstep}khúúlá] tá}  &   
                     \gloss{‘releasing’}  &  \\

                     \vernacular{musáatsa
                    waá[{\downstep}kháláká] tá}  &   
                     \gloss{‘cutting’}  &  \\

                     \vernacular{musáatsa
                    waa[lá{\downstep}khúúlá] tá}  &   
                     \gloss{‘releasing’}  &  \\
\end{tabular}
%\caption{\nocaption}
    

\chapter{Verbal Subject Tests}\label{sec:aVerbSubjTests}

The brief questionnaire below was administered
          primarily to determine the effect of subject choice on
          verb tone in constructions known to exhibit a variety of
          tonal melodies. While most of the paradigms included
          herein serve this purpose, the first four paradigms were
          added to confirm the existence and tonal properties of a
          construction that was identified only after the main
          questionnaire (featured in Appendix \appref{sec:aIdakhoVerbalToneQuestionnaire} ) had been completed. The construction in
          question is an alternative of the Near Future Negative,
          described in \sectref{sec:sPattern1a} .

 With support from Pomona College, the audio archive
          may currently be downloaded freely via \href{https://pomona.box.com/s/dprn6ub6zurfptfkwki0}{
          https://pomona.box.com/s/dprn6ub6zurfptfkwki0}.

   

  \vernacular{Near Future
            Negative} 

 
\begin{tabular}{llllll}  
  \multicolumn{5}{l}{
                 \vernacular{(1) /H/
                C-Initial} \gloss{‘s/he will
                not...’} } &  \\
\multicolumn{5}{l}{ } &  \\

                 \vernacular{sha[khwa]
                tá}  &   
                 \gloss{‘pay dowry’}  &     &   
                 \vernacular{sha[luma]
                tá}  &   
                 \gloss{‘bite’}  &  \\

                 \vernacular{sha[teekha]
                tá}  &   
                 \gloss{‘cook’}  &     &   
                 \vernacular{sha[khalaka]
                tá}  &   
                 \gloss{‘cut’}  &  \\

                 \vernacular{sha[kalaanga]
                tá}  &   
                 \gloss{‘fry’}  &     &   
                 \vernacular{sha[boolitsa]
                tá}  &   
                 \gloss{‘seduce’}  &  \\

                 \vernacular{sha[ng’ong’oolitsa]
                tá}  &   
                 \gloss{‘tease’}  &     &   
                 \vernacular{sha[khunishila]
                tá}  &   
                 \gloss{‘close in’}  &  \\

                 \vernacular{sha[kholomoshitsa]
                tá}  &   
                 \gloss{‘push down’}  &  \\
\end{tabular}
%\caption{\nocaption}
     
\begin{tabular}{llllll}  
  \multicolumn{5}{l}{
                 \vernacular{(2) /Ø/
                C-Initial} \gloss{‘s/he will
                not...’} } &  \\
\multicolumn{5}{l}{ } &  \\

                 \vernacular{sha[kwá]
                {\downstep}tá}  &   
                 \gloss{‘fall’}  &     &   
                 \vernacular{sha[lekhá]
                {\downstep}tá}  &   
                 \gloss{‘leave’}  &  \\

                 \vernacular{sha[reéba]
                tá}  &   
                 \gloss{‘ask’}  &     &   
                 \vernacular{sha[kulíkha]
                tá}  &   
                 \gloss{‘name’}  &  \\

                 \vernacular{sha[lakhúula]
                tá}  &   
                 \gloss{‘release’}  &     &   
                 \vernacular{sha[seébúla]
                tá}  &   
                 \gloss{‘say bye’}  &  \\

                 \vernacular{sha[kalúshíla]
                tá}  &   
                 \gloss{‘repeat’}  &     &   
                 \vernacular{sha[siínjílitsa]
                tá}  &   
                 \gloss{‘make stand’}  &  \\

                 \vernacular{sha[kalúkhányinya]
                tá}  &   
                 \gloss{‘turn over’}  &  \\
\end{tabular}
%\caption{\nocaption}
     
\begin{tabular}{llllll}  
  \multicolumn{5}{l}{
                 \vernacular{(3) /H/ C-Initial
                + OP} \gloss{‘s/he will
                not...him/her’} } &  \\
\multicolumn{5}{l}{ } &  \\

                 \vernacular{shamu[khwá]
                {\downstep}tá}  &   
                 \gloss{‘pay dowry’}  &     &   
                 \vernacular{shamu[lúma]
                tá}  &   
                 \gloss{‘bite’}  &  \\

                 \vernacular{shamu[léera]
                tá}  &   
                 \gloss{‘bring’}  &     &   
                 \vernacular{shamu[khálaka]
                tá}  &   
                 \gloss{‘cut’}  &  \\

                 \vernacular{shamu[khóng’oonda]
                tá}  &   
                 \gloss{‘knock’}  &     &   
                 \vernacular{shamu[bóolitsa]
                tá}  &   
                 \gloss{‘seduce’}  &  \\

                 \vernacular{shamu[tsúunzuuna]
                tá}  &   
                 \gloss{‘suck’}  &     &   
                 \vernacular{shamu[khúnishila]
                tá}  &   
                 \gloss{‘close in’}  &  \\

                 \vernacular{shamu[kholomoshitsa]
                tá}  &   
                 \gloss{‘push...down’}  &  \\
\end{tabular}
%\caption{\nocaption}
     
\begin{tabular}{llllll}  
  \multicolumn{5}{l}{
                 \vernacular{(4) /Ø/ C-Initial
                + OP} \gloss{‘s/he will
                not...him/her’} } &  \\
\multicolumn{5}{l}{ } &  \\

                 \vernacular{shamu[tsía]
                {\downstep}tá}  &   
                 \gloss{‘go for’}  &     &   
                 \vernacular{shamu[lekhá]
                {\downstep}tá}  &   
                 \gloss{‘leave’}  &  \\

                 \vernacular{shamu[loónda]
                tá}  &   
                 \gloss{‘follow’}  &     &   
                 \vernacular{shamu[kulíkha]
                tá}  &   
                 \gloss{‘name’}  &  \\

                 \vernacular{shamu[lakhúula]
                tá}  &   
                 \gloss{‘release’}  &     &   
                 \vernacular{shamu[seébúla]
                tá}  &   
                 \gloss{‘say bye to’}  &  \\

                 \vernacular{shamu[kalúshíla]
                tá}  &   
                 \gloss{‘defend’}  &     &   
                 \vernacular{shamu[siínjílitsa]
                tá}  &   
                 \gloss{‘make...stand’}  &  \\

                 \vernacular{shamu[kalúkhányinya]
                tá}  &   
                 \gloss{‘turn...over’}  &  \\
\end{tabular}
%\caption{\nocaption}
       

  \vernacular{Near Future} 

 
\begin{tabular}{llllll}  
  \multicolumn{5}{l}{
                 \vernacular{(5) /H/
                C-Initial} \gloss{‘...will
                bring’} } &  \\
\multicolumn{5}{l}{ } &  \\

                 \vernacular{na[leera]}  &   
                 \gloss{‘I’}  &     &   
                 \vernacular{
                khula[leera]}  &   
                 \gloss{‘we’}  &  \\

                 \vernacular{ula[leera]}  &   
                 \gloss{‘you’}  &     &   
                 \vernacular{mula[leera]}  &   
                 \gloss{‘you (pl)’}  &  \\

                 \vernacular{ala[léera]}  &   
                 \gloss{‘s/he’}  &     &   
                 \vernacular{
                bala[léera]}  &   
                 \gloss{‘they’}  &  \\
\end{tabular}
%\caption{\nocaption}
     
\begin{tabular}{llllll}  
  \multicolumn{5}{l}{
                 \vernacular{(6) /Ø/
                C-Initial} \gloss{‘...will
                ask’} } &  \\
\multicolumn{5}{l}{ } &  \\

                 \vernacular{na[reeba]}  &   
                 \gloss{‘I’}  &     &   
                 \vernacular{
                khula[reeba]}  &   
                 \gloss{‘we’}  &  \\

                 \vernacular{ula[reeba]}  &   
                 \gloss{‘you’}  &     &   
                 \vernacular{mula[reeba]}  &   
                 \gloss{‘you (pl)’}  &  \\

                 \vernacular{ala[reeba]}  &   
                 \gloss{‘s/he’}  &     &   
                 \vernacular{bala[reeba]}  &   
                 \gloss{‘they’}  &  \\
\end{tabular}
%\caption{\nocaption}
     
\begin{tabular}{llllll}  
  \multicolumn{5}{l}{
                 \vernacular{(7) /H/ C-Initial
                + OP} \gloss{‘...will bring
                him/her’} } &  \\
\multicolumn{5}{l}{ } &  \\

                 \vernacular{namu[leera]}  &   
                 \gloss{‘I’}  &     &   
                 \vernacular{
                khulamu[leera]}  &   
                 \gloss{‘we’}  &  \\

                 \vernacular{
                ulamu[leera]}  &   
                 \gloss{‘you’}  &     &   
                 \vernacular{
                mulamu[leera]}  &   
                 \gloss{‘you (pl)’}  &  \\

                 \vernacular{
                alamú[leera]}  &   
                 \gloss{‘s/he’}  &     &   
                 \vernacular{
                balamú[leera]}  &   
                 \gloss{‘they’}  &  \\
\end{tabular}
%\caption{\nocaption}
     
\begin{tabular}{llllll}  
  \multicolumn{5}{l}{
                 \vernacular{(8) /Ø/ C-Initial
                + OP} \gloss{‘...will ask
                him/her’} } &  \\
\multicolumn{5}{l}{ } &  \\

                 \vernacular{namu[reeba]}  &   
                 \gloss{‘I’}  &     &   
                 \vernacular{
                khulamu[reeba]}  &   
                 \gloss{‘we’}  &  \\

                 \vernacular{
                ulamu[reeba]}  &   
                 \gloss{‘you’}  &     &   
                 \vernacular{
                mulamu[reeba]}  &   
                 \gloss{‘you (pl)’}  &  \\

                 \vernacular{
                alamú[reeba]}  &   
                 \gloss{‘s/he’}  &     &   
                 \vernacular{
                balamú[reeba]}  &   
                 \gloss{‘they’}  &  \\
\end{tabular}
%\caption{\nocaption}
       

  \vernacular{Near Future
            Negative} 

 
\begin{tabular}{llllll}  
  \multicolumn{5}{l}{
                 \vernacular{(9) /H/
                C-Initial} \gloss{‘...will not
                bring’} } &  \\
\multicolumn{5}{l}{ } &  \\

                 \vernacular{na[leera]
                tá}  &   
                 \gloss{‘I’}  &     &   
                 \vernacular{khula[leera]
                tá}  &   
                 \gloss{‘we’}  &  \\

                 \vernacular{ula[leera]
                tá}  &   
                 \gloss{‘you’}  &     &   
                 \vernacular{mula[leera]
                tá}  &   
                 \gloss{‘you (pl)’}  &  \\

                 \vernacular{ala[léera]
                tá}  &   
                 \gloss{‘s/he’}  &     &   
                 \vernacular{bala[léera]
                tá}  &   
                 \gloss{‘they’}  &  \\
\end{tabular}
%\caption{\nocaption}
       

   

   

   

  \vernacular{Remote Past} 

 
\begin{tabular}{llllll}  
  \multicolumn{5}{l}{
                 \vernacular{(10) /H/
                C-Initial} \gloss{
                ‘...brought’} } &  \\
\multicolumn{5}{l}{ } &  \\

                 \vernacular{naa[léera]}  &   
                 \gloss{‘I’}  &     &   
                 \vernacular{
                khwaa[léera]}  &   
                 \gloss{‘we’}  &  \\

                 \vernacular{waa[léera]}  &   
                 \gloss{‘you’}  &     &   
                 \vernacular{
                mwaa[léera]}  &   
                 \gloss{‘you (pl)’}  &  \\

                 \vernacular{yaa[léera]}  &   
                 \gloss{‘s/he’}  &     &   
                 \vernacular{baa[léera]}  &   
                 \gloss{‘they’}  &  \\
\end{tabular}
%\caption{\nocaption}
     
\begin{tabular}{llllll}  
  \multicolumn{5}{l}{
                 \vernacular{(11) /Ø/
                C-Initial} \gloss{‘...asked’} } &  \\
\multicolumn{5}{l}{ } &  \\

                 \vernacular{naa[réeba]}  &   
                 \gloss{‘I’}  &     &   
                 \vernacular{
                khwaa[réeba]}  &   
                 \gloss{‘we’}  &  \\

                 \vernacular{waa[réeba]}  &   
                 \gloss{‘you’}  &     &   
                 \vernacular{
                mwaa[réeba]}  &   
                 \gloss{‘you (pl)’}  &  \\

                 \vernacular{yaa[réeba]}  &   
                 \gloss{‘s/he’}  &     &   
                 \vernacular{baa[réeba]}  &   
                 \gloss{‘they’}  &  \\
\end{tabular}
%\caption{\nocaption}
       

  \vernacular{Immediate Past} 

 
\begin{tabular}{llllll}  
  \multicolumn{5}{l}{
                 \vernacular{(12) /H/
                C-Initial} \gloss{‘...just
                brought’} } &  \\
\multicolumn{5}{l}{ } &  \\

                 \vernacular{
                ná{\downstep}khá[léera]}  &   
                 \gloss{‘I’}  &     &   
                 \vernacular{
                khwá{\downstep}khá[léera]}  &   
                 \gloss{‘we’}  &  \\

                 \vernacular{
                wá{\downstep}khá[léera]}  &   
                 \gloss{‘you’}  &     &   
                 \vernacular{
                mwá{\downstep}khá[léera]}  &   
                 \gloss{‘you (pl)’}  &  \\

                 \vernacular{
                yá{\downstep}khá[léera]}  &   
                 \gloss{‘s/he’}  &     &   
                 \vernacular{
                bá{\downstep}khá[léera]}  &   
                 \gloss{‘they’}  &  \\
\end{tabular}
%\caption{\nocaption}
       

  \vernacular{Immediate Past
            Negative} 

 
\begin{tabular}{llllll}  
  \multicolumn{5}{l}{
                 \vernacular{(13) /H/
                C-Initial} \gloss{‘...did not just
                bring’} } &  \\
\multicolumn{5}{l}{ } &  \\

                 \vernacular{ná{\downstep}khá[léera]
                tá}  &   
                 \gloss{‘I’}  &     &   
                 \vernacular{khwá{\downstep}khá[léera]
                tá}  &   
                 \gloss{‘we’}  &  \\

                 \vernacular{wá{\downstep}khá[léera]
                tá}  &   
                 \gloss{‘you’}  &     &   
                 \vernacular{mwá{\downstep}khá[léera]
                tá}  &   
                 \gloss{‘you (pl)’}  &  \\

                 \vernacular{yá{\downstep}khá[léera]
                tá}  &   
                 \gloss{‘s/he’}  &     &   
                 \vernacular{bá{\downstep}khá[léera]
                tá}  &   
                 \gloss{‘they’}  &  \\
\end{tabular}
%\caption{\nocaption}
       

  \vernacular{Present} 

 
\begin{tabular}{llllll}  
  \multicolumn{5}{l}{
                 \vernacular{(14) /H/
                C-Initial} \gloss{‘...am/are/is
                bringing’} } &  \\
\multicolumn{5}{l}{ } &  \\

                 \vernacular{
                [ndeeráángá]}  &   
                 \gloss{‘I’}  &     &   
                 \vernacular{
                khu[leeráángá]}  &   
                 \gloss{‘we’}  &  \\

                 \vernacular{
                u[leeráángá]}  &   
                 \gloss{‘you’}  &     &   
                 \vernacular{
                mu[leeráángá]}  &   
                 \gloss{‘you (pl)’}  &  \\

                 \vernacular{
                a[leeráángá]}  &   
                 \gloss{‘s/he’}  &     &   
                 \vernacular{
                ba[léeráángá]}  &   
                 \gloss{‘they’}  &  \\
\end{tabular}
%\caption{\nocaption}
     
\begin{tabular}{llllll}  
  \multicolumn{5}{l}{
                 \vernacular{(15) /Ø/
                C-Initial} \gloss{‘...am/are/is
                asking’} } &  \\
\multicolumn{5}{l}{ } &  \\

                 \vernacular{
                n[ndeébáanga]}  &   
                 \gloss{‘I’}  &     &   
                 \vernacular{
                khu[reébáanga]}  &   
                 \gloss{‘we’}  &  \\

                 \vernacular{
                u[reébáanga]}  &   
                 \gloss{‘you’}  &     &   
                 \vernacular{
                mu[reébáanga]}  &   
                 \gloss{‘you (pl)’}  &  \\

                 \vernacular{
                a[reébáanga]}  &   
                 \gloss{‘s/he’}  &     &   
                 \vernacular{
                ba[reébáanga]}  &   
                 \gloss{‘they’}  &  \\
\end{tabular}
%\caption{\nocaption}
     
\begin{tabular}{llllll}  
  \multicolumn{5}{l}{
                 \vernacular{(16) /Ø/ C-Initial
                + OP} \gloss{‘...am/are/is asking
                him/her’} } &  \\
\multicolumn{5}{l}{ } &  \\

                 \vernacular{
                mu[reébáanga]}  &   
                 \gloss{‘I’}  &     &   
                 \vernacular{
                khumu[reébáanga]}  &   
                 \gloss{‘we’}  &  \\

                 \vernacular{
                umu[reébáanga]}  &   
                 \gloss{‘you’}  &     &   
                 \vernacular{
                mumu[reébáanga]}  &   
                 \gloss{‘you (pl)’}  &  \\

                 \vernacular{
                amú[reébáanga]}  &   
                 \gloss{‘s/he’}  &     &   
                 \vernacular{
                bamú[reébáanga]}  &   
                 \gloss{‘they’}  &  \\
\end{tabular}
%\caption{\nocaption}
       

  \vernacular{Immediate Past
            Negative} 

 
\begin{tabular}{llllll}  
  \multicolumn{5}{l}{
                 \vernacular{(17) /Ø/
                C-Initial} \gloss{‘...did not just
                ask’} } &  \\
\multicolumn{5}{l}{ } &  \\

                 \vernacular{
                nákha[reeba]}  &   
                 \gloss{‘I’}  &     &   
                 \vernacular{
                khwákha[reeba]}  &   
                 \gloss{‘we’}  &  \\

                 \vernacular{
                wákha[reeba]}  &   
                 \gloss{‘you’}  &     &   
                 \vernacular{
                mwákha[reeba]}  &   
                 \gloss{‘you (pl)’}  &  \\

                 \vernacular{
                yákha[reeba]}  &   
                 \gloss{‘s/he’}  &     &   
                 \vernacular{
                bákha[reeba]}  &   
                 \gloss{‘they’}  &  \\
\end{tabular}
%\caption{\nocaption}
     
\begin{tabular}{llllll}  
  \multicolumn{5}{l}{
                 \vernacular{(18) /H/ C-Initial
                + OP} \gloss{‘...did not just
                bring him/her’} } &  \\
\multicolumn{5}{l}{ } &  \\

                 \vernacular{
                ná{\downstep}khámú[leera]}  &   
                 \gloss{‘I’}  &     &   
                 \vernacular{
                khwá{\downstep}khámú[leera]}  &   
                 \gloss{‘we’}  &  \\

                 \vernacular{
                wá{\downstep}khámú[leera]}  &   
                 \gloss{‘you’}  &     &   
                 \vernacular{
                mwá{\downstep}khámuá{\downstep}khá[leera]}  &   
                 \gloss{‘you (pl)’}  &  \\

                 \vernacular{
                yá{\downstep}khámú[leera]}  &   
                 \gloss{‘s/he’}  &     &   
                 \vernacular{
                bá{\downstep}khámú[leera]}  &   
                 \gloss{‘they’}  &  \\
\end{tabular}
%\caption{\nocaption}
     
\begin{tabular}{llllll}  
  \multicolumn{5}{l}{
                 \vernacular{(19) /Ø/ C-Initial
                + OP} \gloss{‘...did not just ask
                him/her’} } &  \\
\multicolumn{5}{l}{ } &  \\

                 \vernacular{
                ná{\downstep}khámú[reeba]}  &   
                 \gloss{‘I’}  &     &   
                 \vernacular{
                khwá{\downstep}khámú[reeba]}  &   
                 \gloss{‘we’}  &  \\

                 \vernacular{
                wá{\downstep}khámú[reeba]}  &   
                 \gloss{‘you’}  &     &   
                 \vernacular{
                mwá{\downstep}khámú[reeba]}  &   
                 \gloss{‘you (pl)’}  &  \\

                 \vernacular{
                yá{\downstep}khámú[reeba]}  &   
                 \gloss{‘s/he’}  &     &   
                 \vernacular{
                bá{\downstep}khámú[reeba]}  &   
                 \gloss{‘they’}  &  \\
\end{tabular}
%\caption{\nocaption}
       

  \vernacular{Indefinite
            Future} 

 
\begin{tabular}{llllll}  
  \multicolumn{5}{l}{
                 \vernacular{(20) /H/
                C-Initial} \gloss{‘...will
                bring’} } &  \\
\multicolumn{5}{l}{ } &  \\

                 \vernacular{
                nɪlɪ[leerá]}  &   
                 \gloss{‘I’}  &     &   
                 \vernacular{
                khuli[leerá]}  &   
                 \gloss{‘we’}  &  \\

                 \vernacular{uli[leerá]}  &   
                 \gloss{‘you’}  &     &   
                 \vernacular{
                muli[leerá]}  &   
                 \gloss{‘you (pl)’}  &  \\

                 \vernacular{ali[leerá]}  &   
                 \gloss{‘s/he’}  &     &   
                 \vernacular{
                bali[leerá]}  &   
                 \gloss{‘they’}  &  \\
\end{tabular}
%\caption{\nocaption}
     
\begin{tabular}{llllll}  
  \multicolumn{5}{l}{
                 \vernacular{(21) /Ø/
                C-Initial} \gloss{‘...will
                ask’} } &  \\
\multicolumn{5}{l}{ } &  \\

                 \vernacular{
                nɪlɪ[reéba]}  &   
                 \gloss{‘I’}  &     &   
                 \vernacular{
                khuli[reéba]}  &   
                 \gloss{‘we’}  &  \\

                 \vernacular{uli[reéba]}  &   
                 \gloss{‘you’}  &     &   
                 \vernacular{
                muli[reéba]}  &   
                 \gloss{‘you (pl)’}  &  \\

                 \vernacular{ali[reéba]}  &   
                 \gloss{‘s/he’}  &     &   
                 \vernacular{
                bali[reéba]}  &   
                 \gloss{‘they’}  &  \\
\end{tabular}
%\caption{\nocaption}
     
\begin{tabular}{llllll}  
  \multicolumn{5}{l}{
                 \vernacular{(22) /H/ C-Initial
                + OP} \gloss{‘...will bring
                him/her’} } &  \\
\multicolumn{5}{l}{ } &  \\

                 \vernacular{
                nilimu[leerá]}  &   
                 \gloss{‘I’}  &     &   
                 \vernacular{
                khulimu[leerá]}  &   
                 \gloss{‘we’}  &  \\

                 \vernacular{
                ulimu[leerá]}  &   
                 \gloss{‘you’}  &     &   
                 \vernacular{
                mulimu[leerá]}  &   
                 \gloss{‘you (pl)’}  &  \\

                 \vernacular{
                alimu[leerá]}  &   
                 \gloss{‘s/he’}  &     &   
                 \vernacular{
                balimu[leerá]}  &   
                 \gloss{‘they’}  &  \\
\end{tabular}
%\caption{\nocaption}
       

 
\begin{tabular}{llllll}  
  \multicolumn{5}{l}{
                 \vernacular{(23) /Ø/ C-Initial
                + OP} \gloss{‘...will ask
                him/her’} } &  \\
\multicolumn{5}{l}{ } &  \\

                 \vernacular{
                nilimu[reéba]}  &   
                 \gloss{‘I’}  &     &   
                 \vernacular{
                khulimu[reéba]}  &   
                 \gloss{‘we’}  &  \\

                 \vernacular{
                ulimu[reéba]}  &   
                 \gloss{‘you’}  &     &   
                 \vernacular{
                mulimu[reéba]}  &   
                 \gloss{‘you (pl)’}  &  \\

                 \vernacular{
                alimu[reéba]}  &   
                 \gloss{‘s/he’}  &     &   
                 \vernacular{
                balimu[reéba]}  &   
                 \gloss{‘they’}  &  \\
\end{tabular}
%\caption{\nocaption}
       

  \vernacular{Crastinal
            Future} 

 
\begin{tabular}{llllll}  
  \multicolumn{5}{l}{
                 \vernacular{(24) /H/
                C-Initial} \gloss{‘...will
                bring’} } &  \\
\multicolumn{5}{l}{ } &  \\

                 \vernacular{
                nee[ndeerɛ́]}  &   
                 \gloss{‘I’}  &     &   
                 \vernacular{
                nɪkhu[leerɛ́]}  &   
                 \gloss{‘we’}  &  \\

                 \vernacular{nuu[leerɛ́]}  &   
                 \gloss{‘you’}  &     &   
                 \vernacular{
                nimu[leerɛ́]}  &   
                 \gloss{‘you (pl)’}  &  \\

                 \vernacular{naa[leerɛ́]}  &   
                 \gloss{‘s/he’}  &     &   
                 \vernacular{
                niba[leerɛ́]}  &   
                 \gloss{‘they’}  &  \\
\end{tabular}
%\caption{\nocaption}
       

  \vernacular{Subjunctive
            Negative} 

 
\begin{tabular}{llllll}  
  \multicolumn{5}{l}{
                 \vernacular{(25) /H/
                C-Initial} \gloss{‘let...not
                bring’} } &  \\
\multicolumn{5}{l}{ } &  \\

                 \vernacular{kha[leera]}  &   
                 \gloss{‘me’}  &     &   
                 \vernacular{
                khukha[leera]}  &   
                 \gloss{‘us’}  &  \\

                 \vernacular{ukha[leera]}  &   
                 \gloss{‘you’}  &     &   
                 \vernacular{
                mukha[leera]}  &   
                 \gloss{‘you (pl)’}  &  \\

                 \vernacular{akha[leera]}  &   
                 \gloss{‘him/her’}  &     &   
                 \vernacular{
                bakha[leera]}  &   
                 \gloss{‘them’}  &  \\
\end{tabular}
%\caption{\nocaption}
     
\begin{tabular}{llllll}  
  \multicolumn{5}{l}{
                 \vernacular{(26) /Ø/
                C-Initial} \gloss{‘let...not
                ask’} } &  \\
\multicolumn{5}{l}{ } &  \\

                 \vernacular{kha[reéba]}  &   
                 \gloss{‘me’}  &     &   
                 \vernacular{
                khukha[reéba]}  &   
                 \gloss{‘us’}  &  \\

                 \vernacular{
                ukha[reéba]}  &   
                 \gloss{‘you’}  &     &   
                 \vernacular{
                mukha[reéba]}  &   
                 \gloss{‘you (pl)’}  &  \\

                 \vernacular{
                akha[reéba]}  &   
                 \gloss{‘him/her’}  &     &   
                 \vernacular{
                bakha[reéba]}  &   
                 \gloss{‘them’}  &  \\
\end{tabular}
%\caption{\nocaption}
       

  \vernacular{Habitual} 

 
\begin{tabular}{llllll}  
  \multicolumn{5}{l}{
                 \vernacular{(27) /H/
                C-Initial} \gloss{‘...am/are drunk
                \ob ng’wa\cb  / ...are/is married \ob teekha\cb ’} } &  \\
\multicolumn{5}{l}{ } &  \\

                 \vernacular{
                naá[{\downstep}ng’wá]}  &   
                 \gloss{‘I’}  &     &   
                 \vernacular{
                khwaá[{\downstep}ng’wá]}  &   
                 \gloss{‘we’}  &  \\

                 \vernacular{
                waá[{\downstep}téékhá]}  &   
                 \gloss{‘you’}  &     &   
                 \vernacular{
                mwaá[{\downstep}téékhá]}  &   
                 \gloss{‘you (pl)’}  &  \\

                 \vernacular{
                yaá[{\downstep}téékhá]}  &   
                 \gloss{‘s/he’}  &     &   
                 \vernacular{
                baá[{\downstep}téékhá]}  &   
                 \gloss{‘they’}  &  \\
\end{tabular}
%\caption{\nocaption}
     
\begin{tabular}{llllll}  
  \multicolumn{5}{l}{
                 \vernacular{(28) /Ø/
                C-Initial} \gloss{‘...am/are/is
                always/ever asking’} } &  \\
\multicolumn{5}{l}{ } &  \\

                 \vernacular{
                naá[{\downstep}réébá]}  &   
                 \gloss{‘I’}  &     &   
                 \vernacular{
                khwaá[{\downstep}réébá]}  &   
                 \gloss{‘we’}  &  \\

                 \vernacular{
                waá[{\downstep}réébá]}  &   
                 \gloss{‘you’}  &     &   
                 \vernacular{
                mwaá[{\downstep}réébá]}  &   
                 \gloss{‘you (pl)’}  &  \\

                 \vernacular{
                yaá[{\downstep}réébá]}  &   
                 \gloss{‘s/he’}  &     &   
                 \vernacular{
                baá[{\downstep}réébá]}  &   
                 \gloss{‘they’}  &  \\
\end{tabular}
%\caption{\nocaption}
     
\begin{tabular}{llllll}  
  \multicolumn{5}{l}{
                 \vernacular{(29) /H/ C-Initial
                + OP} \gloss{‘...am/are/is
                always/ever cooking for him/her’} } &  \\
\multicolumn{5}{l}{ } &  \\

                 \vernacular{
                naá{\downstep}mú[té{\downstep}éshélá]}  &   
                 \gloss{‘I’}  &     &   
                 \vernacular{
                khwaá{\downstep}mú[té{\downstep}éshélá]}  &   
                 \gloss{‘we’}  &  \\

                 \vernacular{
                waá{\downstep}mú[té{\downstep}éshélá]}  &   
                 \gloss{‘you’}  &     &   
                 \vernacular{
                mwaá{\downstep}mú[té{\downstep}éshélá]}  &   
                 \gloss{‘you (pl)’}  &  \\

                 \vernacular{
                yaá{\downstep}mú[té{\downstep}éshélá]}  &   
                 \gloss{‘s/he’}  &     &   
                 \vernacular{
                baá{\downstep}mú[té{\downstep}éshélá]}  &   
                 \gloss{‘they’}  &  \\
\end{tabular}
%\caption{\nocaption}
     
\begin{tabular}{llllll}  
  \multicolumn{5}{l}{
                 \vernacular{(30) /Ø/ C-Initial
                + OP} \gloss{‘...am/are/is
                always/ever asking him/her’} } &  \\
\multicolumn{5}{l}{ } &  \\

                 \vernacular{
                naá{\downstep}mú[réébá]}  &   
                 \gloss{‘I’}  &     &   
                 \vernacular{
                khwaá{\downstep}mú[réébá]}  &   
                 \gloss{‘we’}  &  \\

                 \vernacular{
                waá{\downstep}mú[réébá]}  &   
                 \gloss{‘you’}  &     &   
                 \vernacular{
                mwaá{\downstep}mú[réébá]}  &   
                 \gloss{‘you (pl)’}  &  \\

                 \vernacular{
                yaá{\downstep}mú[réébá]}  &   
                 \gloss{‘s/he’}  &     &   
                 \vernacular{
                baá{\downstep}mú[réébá]}  &   
                 \gloss{‘they’}  &  \\
\end{tabular}
%\caption{\nocaption}
       

  \vernacular{Perfect} 

 
\begin{tabular}{llllll}  
  \multicolumn{5}{l}{
                 \vernacular{(31) /H/
                C-Initial} \gloss{‘...have/has
                brought’} } &  \\
\multicolumn{5}{l}{ } &  \\

                 \vernacular{[ndeeri]}  &   
                 \gloss{‘I’}  &     &   
                 \vernacular{khuu[leeri]}  &   
                 \gloss{‘we’}  &  \\

                 \vernacular{uu[leeri]}  &   
                 \gloss{‘you’}  &     &   
                 \vernacular{muu[leeri]}  &   
                 \gloss{‘you (pl)’}  &  \\

                 \vernacular{uu[léeri]}  &   
                 \gloss{‘s/he’}  &     &   
                 \vernacular{baa[léeri]}  &   
                 \gloss{‘they’}  &  \\
\end{tabular}
%\caption{\nocaption}
     
\begin{tabular}{llllll}  
  \multicolumn{5}{l}{
                 \vernacular{(32) /Ø/
                C-Initial} \gloss{‘...have/has
                asked’} } &  \\
\multicolumn{5}{l}{ } &  \\

                 \vernacular{[ndeebi]}  &   
                 \gloss{‘I’}  &     &   
                 \vernacular{khuu[reebi]}  &   
                 \gloss{‘we’}  &  \\

                 \vernacular{uu[reebi]}  &   
                 \gloss{‘you’}  &     &   
                 \vernacular{muu[reebi]}  &   
                 \gloss{‘you (pl)’}  &  \\

                 \vernacular{uu[reebi]}  &   
                 \gloss{‘s/he’}  &     &   
                 \vernacular{baa[reebi]}  &   
                 \gloss{‘they’}  &  \\
\end{tabular}
%\caption{\nocaption}
     
\begin{tabular}{llllll}  
  \multicolumn{5}{l}{
                 \vernacular{(33) /H/ C-Initial
                + OP} \gloss{‘...have/has brought
                him/her’} } &  \\
\multicolumn{5}{l}{ } &  \\

                 \vernacular{mu[leeri]}  &   
                 \gloss{‘I’}  &     &   
                 \vernacular{
                khuumu[leeri]}  &   
                 \gloss{‘we’}  &  \\

                 \vernacular{uumu[leeri]}  &   
                 \gloss{‘you’}  &     &   
                 \vernacular{
                muumu[leeri]}  &   
                 \gloss{‘you (pl)’}  &  \\

                 \vernacular{
                uumú[leeri]}  &   
                 \gloss{‘s/he’}  &     &   
                 \vernacular{
                baamú[leeri]}  &   
                 \gloss{‘they’}  &  \\
\end{tabular}
%\caption{\nocaption}
     
\begin{tabular}{llllll}  
  \multicolumn{5}{l}{
                 \vernacular{(34) /Ø/ C-Initial
                + OP} \gloss{‘...have/has asked
                him/her’} } &  \\
\multicolumn{5}{l}{ } &  \\

                 \vernacular{mu[reebi]}  &   
                 \gloss{‘I’}  &     &   
                 \vernacular{
                khuumu[reebi]}  &   
                 \gloss{‘we’}  &  \\

                 \vernacular{uumu[reebi]}  &   
                 \gloss{‘you’}  &     &   
                 \vernacular{
                muumu[reebi]}  &   
                 \gloss{‘you (pl)’}  &  \\

                 \vernacular{
                uumú[reebi]}  &   
                 \gloss{‘s/he’}  &     &   
                 \vernacular{
                baamú[reebi]}  &   
                 \gloss{‘they’}  &  \\
\end{tabular}
%\caption{\nocaption}
    

>>>>>>> 8b4e50a5274c30935bb67139e066062bc0b7d3e9
